
v related: http://chaudhuri.info/papers/lpar08synstra.pdf

why map into pullbacks not just from 1:

``Such pullbacks are detected by mapping out of the 1-point category and
the ``interval category'', so the only other interesting case is the
latter, which says that the arrows in the top-left corner are also
determined by arrows in the other three corners.  That means essentially
that this factorization respects the action of 2-cells in M on arrows of
D.  I'm not sure whether or not that extra condition is necessary to
postulate; Hormann says it in the opcartesian case but not the cartesian
case, which confuses me.''

----------------------------------------------------------------------

check focalization: U_\alpha that doesn't mention c is weird?

----------------------------------------------------------------------

Just noticed that there is some study here for symmetric
multicategories: http://arxiv.org/abs/1505.00974 (Appendix A) In
particular, the definition of ``cartesian morphism at the i-th slot''
(A.2.5) looks kind of like a categorical analogue of your generalized
U-functors.

----------------------------------------------------------------------

applications:
fluet substructural
subexponentials
resource counting F_(x * x * x * x)
fractional permissions?

----------------------------------------------------------------------

This setup is especially cool because it unifies a lot of different
kinds of generalized multicategories, including most of the ones that
seem relevant to type theory, but in a setting that is noticeably less
abstract than the full-blown theory of generalized multicategories
over 2-monads.  I didn't realize before that you can get bunched
implication this way too, but now I see it.  In fact, I bet there is a
general theorem like: given any cartesian 2-multicategory, it
generates a 2-monad on Cat with the same algebras, and generalized
multicategories for that 2-monad are equivalent to multicategories
over the 2-multicategory, with representability etc. corresponding.

----------------------------------------------------------------------

Coming back to this old thread, I'm not sure why it took me this long to
see clearly that this is clearly the right way to generalize adjoint
logic to allow multiple formulas in the context.  Maybe it's that I'm
only now really internalizing the general picture I mentioned to you in
Toronto that the judgmental structure of a type theory corresponds to a
kind of generalized multicategory.  Or maybe it's because I was still
too focused on the dependently typed case.

We noticed for one-variable adjoint logic (and mentioned in a footnote)
that the judgmental structure would be more directly modeled
categorically by a structure with a basic notion of ``morphism along
\alpha'', like a Grothendieck bifibration.  In fact (although this isn't
really necessary to know) a ``category over M'' is a kind of generalized
multicategory, and it is ``representable'' (the property of generalized
multicategories that corresponds to an ordinary multicategory being a
monoidal category) just when it is an opfibration.  So clearly a natural
way to generalize to multiple-formula contexts would be to consider
(bi)fibrations of multicategories.

Opfibrations of multicategories are defined here:
http://sqig.math.ist.utl.pt/pub/HermidaC/fib-mul.pdf
In general, if M is a multicategory of modes, then multicategories over
M should be the natural semantics of this theory with sequents
(i.e. morphisms) parametrized by (i.e. living over) a morphism in M.
Being an opfibration then corresponds to having the F connectives for
all morphisms in M (we want the version of the definition remarked on in
2.2.1 in the paper above).  An opfibration over the terminal
multicategory is the same as a representable multicategory, which just
means that if there is only one mode and a unique morphism of every
arity, we just have an ordinary type theory with a tensor product and
unit (which is symmetric or cartesian according as our multicategories
are symmetric or cartesian).

(Non-op) fibrations of multicategories aren't defined in that note, but
I think the right definition has cartesian morphisms over unary
morphisms only but with a universal property relating to morphisms of
arbitrary arity; this should correspond to having U connectives.  And
the generalization to 2-multicategories ought to be reasonably
straightforward.

And as you said in the first email, this also explains from a different
point of view why we were seeing that the F's seem to have to preserve
the product, and gives a way to talk about functors that don't.  The
most obvious way to have ``two modes, each with products, and a functor
from one to the other'' is to have modes p and q and unique morphisms
that look like (p,p,...,p) -> p (giving p a product) and (q,q,...,q) ->
q) (giving q a product) and also (p,q,p,...,q) -> q (an arbitrary domain
of p's and q's with q as codomain, giving the action of F).  The
resulting judgmental structure looks like Pfenning-Davies and spatial
TT, with p=valid/crisp and q=true/cohesive, and no label on the
turnstile needed because all morphisms are uniquely determined by their
domain and codomain.  The latter fact also implies that the composites
(p,p) -> p -> q   and    (p,p) -> (q,q) -> q
are equal, which means exactly that F : p -> q preserves products; but
we could also consider a mode theory in which those two morphisms are
different (or, perhaps, related by a 2-cell).

\documentclass{amsart}
\usepackage{mathpartir,mathtools}
\def\flet#1:=#2in{\mathsf{let}\;#1 \coloneqq #2\;\mathsf{in}\;}
\def\ftype{\mathsf{type}}
\def\lax{\sim\hspace{-5pt}:\hspace{3pt}}
\def\inr{\mathsf{inr}\;}
\def\inl{\mathsf{inl}\;}
\def\case{\mathsf{case}}
\begin{document}

\section{Rules for $\flat$}
\label{sec:rules-flat}

The comonad $\flat$ is implemented like the Pfenning-Davies $\Box$, with a context of ``discrete assumptions'' $u::A$, which must all appear to the left of the ordinary context.
We write $\Delta;\Gamma$ when $\Delta$ is the valid context and $\Gamma$ the ordinary context.
A discrete variable can be used like an ordinary one.
\[ \inferrule{\ }{\Delta,u::A,\Delta';\Gamma \vdash u:A} \]
Formation and introduction rules clear the ordinary context:
\[ \inferrule{\Delta;\cdot \vdash A:\ftype}{\Delta;\Gamma \vdash \flat A:\ftype} \qquad
\inferrule{\Delta;\cdot \vdash M:A}{\Delta;\Gamma \vdash M^\flat :\flat A} \]
The elimination rule pattern-matches on $x:\flat A$ to obtain $u::A$:
\[ \inferrule{\Delta;\Gamma \vdash M:\flat A \\ \Delta,u::A;\Gamma \vdash N:C}{\Delta;\Gamma \vdash \flet u^\flat := M in N : C} \]
Note that this only makes sense if from $\Delta;\Gamma \vdash \flat A:\ftype$ we can infer $\Delta;\cdot \vdash A:\ftype$.
This is logical since the intro rule for $\flat$ clears the ordinary context.

The reduction rules are as expected:
\begin{align*}
  (\flet u^\flat := M^\flat in N) &\Longrightarrow N[M/u]\\
  M : \flat A &\Longrightarrow (\flet u^\flat := M in u^\flat)
\end{align*}
If $M:\flat A$, we write $M_\flat:A$ for ``$\flet u^\flat := M in u$''.
Thus $(M^\flat)_\flat \equiv M$.

Perhaps we also need some commuting conversions, as in Nanevski-Pfenning-Pientka, such as
\[ (\flet u := (\flet v := N_1 in N_2) in M) \Longrightarrow (\flet v := N_1 in (\flet u := N_2 in M)) \]
if $v$ is not free in $u.M$.

\section{Rules for $\sharp$}
\label{sec:rules-sharp}

The monad $\sharp$ is implemented like the Pfenning-Davies $\bigcirc$, with a ``codiscrete term judgment'' $M\lax A$.
An ordinary term is also codiscrete:
\[ \inferrule{\Delta;\Gamma \vdash M:A}{\Delta;\Gamma \vdash M\lax A} \]
Formation and introduction:
\[ \inferrule{\Delta;\Gamma \vdash A:\ftype}{\Delta;\Gamma \vdash \sharp A:\ftype}\qquad
\inferrule{\Delta;\Gamma \vdash M \lax A}{\Delta;\Gamma \vdash M^\sharp: \sharp A}
\]
Elimination:
\[ \inferrule{\Delta;\Gamma \vdash M:\sharp A \\ \Delta;\Gamma,x:A \vdash N\lax C}{\Delta;\Gamma \vdash \flet x^\sharp := M in N \lax C} \]
Reductions:
\begin{align*}
  (\flet x^\sharp := M^\sharp in N) &\Longrightarrow N[M/x]\\
  M:\sharp A &\Longrightarrow (\flet x^\sharp := M in x)^\sharp 
\end{align*}

\subsection{The universal property}
\label{sec:universal-property}

Suppose $f:A\to \sharp B$.  Then $x:A \vdash f(x) : \sharp B$ and
\[x:A \vdash (\flet y^\sharp := f(x) in y) \lax B.\]
therefore
\[u:\sharp A \vdash (\flet x^\sharp := u in (\flet y^\sharp := f(x) in y)) \lax B \]
and so
\[u:\sharp A \vdash (\flet x^\sharp := u in (\flet y^\sharp := f(x) in y))^\sharp : \sharp B \]
which we can of course abstract into a function $g:\sharp A \to \sharp B$.
Now if we substitute $x^\sharp$ for $u$, we get
\begin{align*}
  &\hspace{3ex} (\flet x^\sharp := x^\sharp in (\flet y^\sharp := f(x) in y))^\sharp\\
  &\equiv (\flet y^\sharp := f(x) in y)^\sharp\\
  &\equiv f(x).
\end{align*}
However, we seem to need something else to get uniqueness.


\section{Rules for adjointness}
\label{sec:rules-adjointness}

The rules above don't yield any interaction between $\flat$ and $\sharp$.
Here are some proposed rules implementing their intended relationship.

Firstly, we can make a codiscrete term into an ordinary one, at the cost of clearing the ordinary context:
\[ \inferrule{\Delta;\cdot \vdash M\lax A}{\Delta;\Gamma \vdash M^\natural : A} \]
The computation laws for this look slightly odd, because the ``inverse'' of this operation is the silent regarding of $x:A$ as $x\lax A$.
\[ \inferrule{\Delta;\cdot \vdash M: A}{\Delta;\Gamma \vdash M^\natural\equiv M : A}(\natural_1)\qquad
\inferrule{\Delta;\cdot \vdash M\lax A}{\Delta;\Gamma \vdash M^\natural \equiv M \lax A}(\natural_2) \]
In the first, we silently regard $M:A$ as $M\lax A$ in order to apply $(-)^\natural$ to it, and get back the same term $M:A$ that we started with.
In the second, we silently regard $M^\natural :A$ as $M^\natural \lax A$, and get back the same term $M\lax A$ that we started with.
Note that a special case of the second is $M^{\natural\sharp}\equiv M^\sharp$.

Secondly, if proving a codiscrete goal, our assumptions might as well be discrete.
\[ \inferrule{\Delta;\Gamma \vdash M:A \\ \Delta,u::A;\Gamma \vdash N \lax A}{\Delta;\Gamma \vdash (\flet u^\flat_\flat := M in N) \lax A} \]
The computation laws for this simply reduce it to the expected $N[M/u]$, but under two conditions that make this well-typed: either $M$ makes sense in  a purely discrete context (hence can be substituted for the discrete hypothesis $u::A$), or $N$ doesn't use the discreteness of this hypothesis.
\[ \inferrule{\Delta;\cdot \vdash M:A \\ \Delta,u::A;\Gamma \vdash N \lax A}{\Delta;\Gamma \vdash (\flet u^\flat_\flat := M in N) \equiv N[M/u] \lax A} ({}^\flat_{\flat1}) \]
\[ \inferrule{\Delta;\Gamma \vdash M:A \\ \Delta;\Gamma,u:A \vdash N \lax A}{\Delta;\Gamma \vdash (\flet u^\flat_\flat := M in N) \equiv N[M/u] \lax A}({}^\flat_{\flat2}) \]
We observe a couple of consequences involving the derived operation $(-)_\flat$.
\begin{alignat*}{2}
  (\flet u^\flat_\flat := M_\flat in N)
  &\equiv (\flet u^\flat_\flat := (\flet v^\flat := M in v) in N)
  &&\quad\text{(def.~of } (-)_\flat)\\
  &\equiv (\flet v^\flat := M in (\flet u^\flat_\flat := v in N))
  &&\quad\text{(commuting)}\\
  &\equiv (\flet v^\flat := M in N[v/u])
  &&\quad ({}^\flat_{\flat1})\\
  &\equiv (\flet u^\flat := M in N)
\end{alignat*}
We will refer to the above rule as $(\beta_\flat)$.
\begin{alignat*}{2}
  (\flet u^\flat_\flat := M in u^\flat)_\flat
  &\equiv (\flet v^\flat := (\flet u^\flat_\flat := M in u^\flat) in v)
  &&\quad\text{(def.~of } (-)_\flat)\\
  &\equiv (\flet u^\flat_\flat := M in (\flet v^\flat := u^\flat in v))
  &&\quad\text{(commuting)}\\
  &\equiv (\flet u^\flat_\flat := M in u)
  &&\quad (\beta^\flat)\\
  &\equiv M
  &&\quad ({}^\flat_{\flat2})
\end{alignat*}
We will refer to the above rule as $(\eta_\flat)$.


\subsection{Discretes are equivalent to codiscretes}
\label{sec:disc=codisc}

What we aim to capture semantically is that the category of $\flat$ types and the category of $\sharp$ types are equivalent, and that modulo that equivalence, $\flat$ and $\sharp$ are the operation.
We can achieve this by saying that for any type $A$ (in purely discrete context), the induced maps $\flat A \to \flat\sharp A$ and $\sharp\flat A \to \sharp A$ (which can be defined without the connecting rules) are equivalences.

The first of these is
\[ x:\flat A \vdash (\flet w^\flat := x in w^{\sharp\flat}):\flat\sharp A. \]
With the connecting rules, we can also define its inverse:
\[ y:\flat\sharp A \vdash (\flet u^\flat := y in (\flet v^\sharp := u in v)^{\natural\flat}):\flat A \]
Substituting the first for $y$ in the second, we have
\begin{alignat*}{2}
  x:\flat A\vdash &\hspace{3ex} \flet u^\flat := (\flet w^\flat := x in w^{\sharp\flat}) in (\flet v^\sharp := u in v)^{\natural\flat}\\
  &\equiv \flet w^\flat := x in (\flet u^\flat := w^{\sharp\flat} in (\flet v^\sharp := u in v)^{\natural\flat})
  &&\quad \text{(commuting)}\\
  &\equiv \flet w^\flat := x in (\flet v^\sharp := w^\sharp in v)^{\natural\flat}
  &&\quad (\beta^\flat)\\
  &\equiv \flet w^\flat := x in w^{\natural\flat}
  &&\quad (\beta^\sharp)\\
  &\equiv \flet w^\flat := x in w^{\flat}
  &&\quad (\natural_1)\\
  &\equiv x.
  &&\quad (\eta^\flat)
\end{alignat*}
And substituting the second for $x$ in the first, we have
\begin{alignat*}{2}
  Y:\flat\sharp A \vdash &\hspace{3ex} \flet w^\flat := (\flet u^\flat := y in (\flet v^\sharp := u in v)^{\natural\flat}) in w^{\sharp\flat}\\
  &\equiv \flet u^\flat := y in (\flet w^\flat := (\flet v^\sharp := u in v)^{\natural\flat} in w^{\sharp\flat})
  &&\quad \text{(commuting)}\\
  &\equiv \flet u^\flat := y in (\flet v^\sharp := u in v)^{\natural\sharp\flat}
  &&\quad (\beta^\flat)\\
  &\equiv \flet u^\flat := y in (\flet v^\sharp := u in v)^{\sharp\flat}
  &&\quad (\natural_2)\\
  &\equiv \flet u^\flat := y in u^{\flat}
  &&\quad (\eta^\sharp)\\
  &\equiv y.
  &&\quad (\eta^\flat)
\end{alignat*}
Therefore, we have a judgmental isomorphism between $\flat A$ and $\flat\sharp A$.

Similarly, we can always define
\[ x : \sharp\flat A \vdash (\flet u^\sharp := x in u_\flat)^\sharp : \sharp A. \]
With the connecting rules, we can define its inverse
\[ y : \sharp A \vdash (\flet v^\sharp := y in (\flet w^\flat_\flat := v in w^\flat))^\sharp : \sharp\flat A\]
Substituting the first for $y$ in the second, we have
\begin{alignat*}{2}
  x:\sharp\flat A \vdash &\hspace{3ex} (\flet v^\sharp := (\flet u^\sharp := x in u_\flat)^\sharp in (\flet w^\flat_\flat := v in w^\flat))^\sharp\\
  &\equiv (\flet w^\flat_\flat := (\flet u^\sharp := x in u_\flat) in w^\flat)^\sharp
  &&\quad (\beta^\sharp)\\
  &\equiv (\flet u^\sharp := x in (\flet w^\flat_\flat := u_\flat in w^\flat))^\sharp
  &&\quad \text{(commuting)}\\
  &\equiv (\flet u^\sharp := x in (\flet w^\flat := u in w^\flat))^\sharp
  &&\quad (\beta_\flat)\\
  &\equiv (\flet u^\sharp := x in u)^\sharp
  &&\quad (\beta^\flat)\\
  &\equiv x.
  &&\quad (\eta^\sharp)
\end{alignat*}
And substituting the second for $x$ in the first, we have
\begin{alignat*}{2}
  y : \sharp A \vdash&\hspace{3ex} (\flet u^\sharp := (\flet v^\sharp := y in (\flet w^\flat_\flat := v in w^\flat))^\sharp in u_\flat)^\sharp\\
  &\equiv ((\flet v^\sharp := y in (\flet w^\flat_\flat := v in w^\flat))_\flat)^\sharp
  &&\quad (\beta^\sharp)\\
  &\equiv (\flet z^\flat := (\flet v^\sharp := y in (\flet w^\flat_\flat := v in w^\flat)) in z)^\sharp
  &&\quad \text{(def.~of } (-)_\flat)\\
  &\equiv (\flet v^\sharp := y in (\flet z^\flat := (\flet w^\flat_\flat := v in w^\flat) in z))^\sharp
  &&\quad \text{(commuting)}\\
  &\equiv (\flet v^\sharp := y in (\flet w^\flat_\flat := v in w^\flat)_\flat)^\sharp
  &&\quad \text{(def.~of } (-)_\flat)\\
  &\equiv (\flet v^\sharp := y in v)^\sharp
  &&\quad (\eta_\flat)\\
  &\equiv y.
  &&\quad (\eta^\sharp)
\end{alignat*}
Thus, we also have a judgmental isomorphism between $\sharp\flat A$ and $\sharp A$.
So these rules do suffice to obtain the desired relation between $\flat$ and $\sharp$.

\subsection{The adjunction}
\label{sec:adjunction}

We now show that indeed $\flat\dashv\sharp$, or at least that we have the usual adjunct correspondence between maps $\flat A \to B$ and $A\to\sharp B$.
Suppose $A,B::\ftype$.
Given $f::\flat A \to B$, we have
\[ x:A \vdash (\flet u^\flat_\flat := x in f(u^\flat))^\sharp : \sharp B \]
And given $g::A\to \sharp B$, we have
\[ y:\flat A \vdash (\flet v^\flat := y in (\flet w^\sharp := g(v) in w)^\natural) : B \] 
Going back and forth in one direction, we have
\begin{alignat*}{2}
  y:\flat A \vdash&\hspace{3ex} \flet v^\flat := y in (\flet w^\sharp := (\flet u^\flat_\flat := v in f(u^\flat))^\sharp in w)^\natural\\
  &\equiv \flet v^\flat := y in (\flet u^\flat_\flat := v in f(u^\flat))^\natural
  &&\quad (\beta^\sharp)\\
  &\equiv \flet v^\flat := y in f(v^\flat)^\natural
  &&\quad ({}^\flat_{\flat1})\\
  &\equiv \flet v^\flat := y in f(v^\flat)
  &&\quad (\natural_1)\\
  &\equiv f(\flet v^\flat := y in v^\flat)
  &&\quad \text{(commuting)}\\
  &\equiv f(y).
  &&\quad (\eta^\flat)
\end{alignat*}
And in the other we have 
\begin{alignat*}{2}
  x:A\vdash&\hspace{3ex} (\flet u^\flat_\flat := x in (\flet v^\flat := u^\flat in (\flet w^\sharp := g(v) in w)^\natural))^\sharp\\
  &\equiv (\flet u^\flat_\flat := x in (\flet w^\sharp := g(u) in w)^\natural)^\sharp
  &&\quad (\beta^\flat)\\
  &\equiv (\flet u^\flat_\flat := x in (\flet w^\sharp := g(u) in w))^\sharp
  &&\quad (\natural_2)\\
  &\equiv (\flet w^\sharp := g(x) in w)^\sharp
  &&\quad ({}^\flat_{\flat2})\\
  &\equiv g(x).
  &&\quad (\eta^\sharp)
\end{alignat*}

\subsection{Preservation of coproducts}
\label{sec:pres-copr}

With the adjointness rules, we can also prove that $\flat$ preserves colimits (since it is a left adjoint).
\[ \inferrule{\Delta,u::A+B;\cdot \vdash C:\ftype \\ \Delta;\cdot\vdash z:A+B \\
  \Delta,a::A;\cdot \vdash c_A :C[\inl a/u] \\\Delta,b::B;\cdot \vdash c_B :C[\inr b/u] }
{\Delta;\cdot \vdash \case_{u.C}(z,x.(\flet a^\flat_\flat := x in c_A)^\natural,y.(\flet b^\flat_\flat := y in c_B)^\natural) : C[z/u]}\]
If we call the above $\case^\flat_{u.C}(z,a.c_A,b.c_B)$, then we have for instance
\[ (\lambda x:\flat(A+B). \flet u^\flat := x in \case^\flat_{\underline{\;}.\flat A+\flat B}(u,a.a^\flat,b.b^\flat)) \,:\, \flat(A+B) \to \flat A+\flat B \]
which ought to be inverse to the canonical map in the other direction (but I have not checked this yet).

\end{document}

\documentclass{drl-common/llncs}

\usepackage{multicol}
\usepackage{mathptmx}
\usepackage{color}
\usepackage[cmex10]{amsmath}
\usepackage{amssymb}
\usepackage{stmaryrd}
\usepackage{drl-common/proof}
\usepackage{drl-common/typesit}
\usepackage{drl-common/typescommon}
\usepackage[square,numbers,sort]{natbib}
%% \usepackage{arydshln}
\usepackage{graphics}
\usepackage{url}
\usepackage{relsize}
\usepackage{fancyvrb}
\usepackage{tikz}
\usetikzlibrary{decorations.pathmorphing}
\usepackage{tipa}

\usepackage{drl-common/code}
\DefineVerbatimEnvironment{code}{Verbatim}{fontsize=\small,fontfamily=tt}

%% small tightcode, with space around it
\newenvironment{stcode}
{\smallskip
\begin{small}
\begin{tightcode}}
{\end{tightcode}
\end{small}
\smallskip}

\title{Adjoint Logic with a 2-Category of Modes}

\author{Daniel R. Licata\inst{1} \and Michael Shulman\inst{2}
\thanks{
This material is based on research sponsored by The United States Air
Force Research Laboratory under agreement number FA9550-15-1-0053. The
U.S. Government is authorized to reproduce and distribute reprints for
Governmental purposes notwithstanding any copyright notation thereon.
The views and conclusions contained herein are those of the authors and
should not be interpreted as necessarily representing the official
policies or endorsements, either expressed or implied, of the United
States Air Force Research Laboratory, the U.S. Government, or Carnegie
Mellon University.
}}

\institute{Wesleyan University \and University of San Diego}

\begin{document}
\maketitle

\begin{abstract}
ABSTRACT
\end{abstract}

\newcommand{\C}{\ensuremath{\mathcal{C}}}
\newcommand{\D}{\ensuremath{\mathcal{D}}}
\newcommand{\la}{\ensuremath{\dashv}}
\newcommand{\arrow}[3]{\ensuremath{#2 \longrightarrow_{#1} #3}}
\newcommand{\sh}{\text{\textesh}}

\section{Introduction}

An adjunction $F \la U$ between categories \C and \D\/ consists of a
pair of functors $F : \C \to \D$ and $U : \D \to \C$ such that maps
\arrow{\D}{F C}{D} correspond naturally to maps \arrow{\C}{C}{U D}.  A
prototypical adjunction, which provides a mnemonic for the notation, is
where $U$ takes the underlying set of some algebraic structure such as a
group, and $F$ is the free structure on a set---the adjunction property
says that a structure-preserving map from $F C$ to $D$ corresponds to a
a map of sets from $C$ to $U D$, because the extension to the structure
is determined by the requirement of being a homomorphism.  Adjunctions
are important to the $\lambda$-calculi and proof theories of modal
logics, because the composite $FU$ is a comonad, while $UF$ is a monad.
\citet{bentonwadler96adjoint} describe an adjoint $\lambda$-calculus for
mixing linear logic and structural/cartesian logic, with functors $U$
from linear to cartesian and $F$ from cartesian to linear, where the $!
A$ modality of linear logic arises as the comonad $FU$.
\citet{reed09adjoint} describes a generalization of this idea to
situations involving more than one category: the logic is parametrized
by a preorder of \emph{modes}, where every mode $p$ determines a
category, and there is an adjunction between categories $p$ and $q$
exactly when $p \ge q$.  For example, the intuitionistic modal logics of
\citet{pfenningdavies} can be encoded as follows: the $\Box$
necessitation modality is the comonad $FU$ for an adjunction between
``truth'' and ``validity'' categories, the $\bigcirc$ modality is the
monad $UF$ of an adjunction between ``truth'' and ``lax truth''
categories, while the $\diamond$ possibility modality requires a more
complicated encoding involving four adjunctions between four categories.

In Reed's logic, modes are specified by a preorder, which allows at most
one adjunction between any two categories.  However, it is sometimes
useful to consider multiple adjunctions between the same two categories.
A motivating example is Lawvere's \emph{axiomatic
  cohesion}~\citep{lawvereXXcohesion}.  In this case, we have two
categories \C and $\mathcal{S}$, where $\mathcal{S}$ plays the role of
sets, and \C represents sets equipped with some \emph{cohesive
  structure}, which represents how elements are stuck together.  A
variety of notions of cohesion---topological, differentiable, smooth,
\ldots---can be described abstractly by a quadruple of adjoint functors

[picture]

\noindent $\Gamma$ takes the underlying set of a space, forgetting the
cohesive structure.  Its left adjoint $\Delta$ gives equips a set with
\emph{discrete cohesion}, where no points are stuck together; the
adjunction says that a map \emph{from} a discrete space is the same as a
map of sets.  Its right adjoint $\nabla$ equips a set with codiscrete
cohesion, where all points are stuck together; the adjunction says that
a map \emph{into} a codiscrete space is the same as a map of sets.  The
left adjoint of $\Delta$, $\Pi_0$ gives the set of connected
components---i.e. the sets of points that are stuck together; the
adjunction says that a map from the set of connected components the same
as a map into a discrete space.  These functors must satisfy some
additional laws, such as $\Delta$ and $\nabla$ being fully
faithful---maps between discrete or codiscrete spaces should be the same
as maps of sets.

Lawvere's notion of axiomatic cohesion is currently being explored in
the setting of homotopy type theory and univalent
foundations~\citep{voevodsky06note,uf13hott-book}.  Homotopy type theory
uses Martin-L\"of's intensional type theory as a logic of homotopy
theory, where the identity type of type theory provides an
$\infty$-groupoid structure on each type.  Theorems from homotopy theory
can be proved \emph{synthetically} in this
logic~\citep{ls13pi1s1,lb13pinsn,...}, and interpreted in a variety of
models~\citep{...}.  The identity type equips every type with an
\emph{algebraic} $\infty$-groupoid structure, providing primitive
notions of morphism between elements of types, morphisms between those
morphisms, and so on.  While we often refer to these morphisms as
``paths,'' an important but subtle distinction is that these paths have
a priori nothing to do with the topological structure on
types~\citep{scott,etc}, with the classical notion of a ``path'' as a
map from the real interval.  For example, in homotopy type theory as in
\citep{uf13hott-book}, there is no way to connect the algebraic
definition of the circle as a higher inductive type to the topological
definition of the circle as $\{ (x,y) \in \mathbb{R}^2 \mid x^2 + y^2 =
1\}$.  \emph{Cohesive homotopy type theory}~\citep{ss12,shulman15}
addresses this issue by integrating axiomatic cohesion into homotopy
type theory, using an adjoint quadruple $\sh \la \Delta \la \Gamma \la
\nabla$.  In this higher categorical generalization, $\mathcal{S}$ is
the category of $\infty$-groupoids---where the ``standard'' model of
homotopy type theory lives---and $\C$ is cohesive
$\infty$-groupoids---i.e. $\infty$-groupoids that also have some
cohesive structure on their objects, morphisms, morphisms between
morphisms, etc.  As above, $\Gamma$ forgets the $\infty$-groupoid
strucutre, $\Delta$ and $\nabla$ equip an $\infty$-groupoid with thw
discrete and codiscrete cohesion.  But $\Delta$'s left adjoint $\sh A$
(pronounced ``shape of $A$'') is, in this context, the \emph{fundamental
  $\infty$-groupoid} functor, which makes an $\infty$-groupoid from the
\emph{topological} paths, paths between paths, etc. of $A$.  Using some
axioms for cohesion, one can prove that the shape of the topological
circle is the algebraic circle, which allows synthetic homotopy theory
to be used in proofs about the topological spaces~\citep{s15}.  




%% \setlength{\bibsep}{-1pt} %% dirty trick: make this negative
{ %% \small
%% \linespread{0.70}
\bibliographystyle{abbrvnat}
\bibliography{drl-common/cs}
}


\end{document}

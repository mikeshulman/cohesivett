
\documentclass{drl-common/llncs}

\usepackage{multicol}
\usepackage{mathptmx}
\usepackage{color}
\usepackage[cmex10]{amsmath}
\usepackage{amssymb}
\usepackage{stmaryrd}
\usepackage{drl-common/proof}
\usepackage{drl-common/typesit}
\usepackage{drl-common/typescommon}
\usepackage[square,numbers,sort]{natbib}
%% \usepackage{arydshln}
\usepackage{graphics}
\usepackage{url}
\usepackage{relsize}
\usepackage{fancyvrb}
\usepackage{tikz}
\usetikzlibrary{decorations.pathmorphing}
\usepackage{tipa}

\usepackage{drl-common/code}
\DefineVerbatimEnvironment{code}{Verbatim}{fontsize=\small,fontfamily=tt}

%% small tightcode, with space around it
\newenvironment{stcode}
{\smallskip
\begin{small}
\begin{tightcode}}
{\end{tightcode}
\end{small}
\smallskip}

\title{Adjoint Logic with a 2-Category of Modes}

\author{Daniel R. Licata\inst{1} \and Michael Shulman\inst{2}
\thanks{
This material is based on research sponsored by The United States Air
Force Research Laboratory under agreement number FA9550-15-1-0053. The
U.S. Government is authorized to reproduce and distribute reprints for
Governmental purposes notwithstanding any copyright notation thereon.
The views and conclusions contained herein are those of the authors and
should not be interpreted as necessarily representing the official
policies or endorsements, either expressed or implied, of the United
States Air Force Research Laboratory, the U.S. Government, or Carnegie
Mellon University.
}}

\institute{Wesleyan University \and University of San Diego}

\begin{document}
\maketitle

\begin{abstract}
ABSTRACT
\end{abstract}

\newcommand{\C}{\ensuremath{\mathcal{C}}}
\newcommand{\D}{\ensuremath{\mathcal{D}}}
\newcommand{\M}{\ensuremath{\mathcal{M}}}
\newcommand{\la}{\ensuremath{\dashv}}
\newcommand{\arrow}[3]{\ensuremath{#2 \longrightarrow_{#1} #3}}
\newcommand{\tc}[2]{\ensuremath{#1 \Rightarrow #2}}
\newcommand{\sh}{\text{\textesh}}
\newcommand{\Adj}{\textbf{Adj}}

\section{Introduction}

An adjunction $F \la U$ between categories \C and \D\/ consists of a
pair of functors $F : \C \to \D$ and $U : \D \to \C$ such that maps
\arrow{\D}{F C}{D} correspond naturally to maps \arrow{\C}{C}{U D}.  A
prototypical adjunction, which provides a mnemonic for the notation, is
where $U$ takes the underlying set of some algebraic structure such as a
group, and $F$ is the free structure on a set---the adjunction property
says that a structure-preserving map from $F C$ to $D$ corresponds to a
a map of sets from $C$ to $U D$ (because the action on the structure is
determined by being a homomorphism).  Adjunctions are important to the
proof theories and $\lambda$-calculi of modal logics, because the
composite $FU$ is a comonad on \D, while $UF$ is a monad on $\C$.
\citet{bentonwadler96adjoint} describe an adjoint $\lambda$-calculus for
mixing linear logic and structural/cartesian logic, with functors $U$
from linear to cartesian and $F$ from cartesian to linear; the $! A$
modality of linear logic arises as the comonad $FU$, while the monad of
Moggi's metalanguage~\citep{moggi91monad} arises as $UF$.
\citet{reed09adjoint} describes a generalization of this idea to
situations involving more than one category: the logic is parametrized
by a preorder of \emph{modes}, where every mode $p$ determines a
category, and there is an adjunction $F \la U$ between categories $p$
and $q$ (with $F : q \to p$) exactly when $p \ge q$.  For example, the
intuitionistic modal logics of \citet{pfenningdavies} can be encoded as
follows: the necessitation modality $\Box$ is the comonad $FU$ for an
adjunction between ``truth'' and ``validity'' categories, the lax
modality $\bigcirc$ is the monad $UF$ of an adjunction between ``truth''
and ``lax truth'' categories, while the possibility modality $\diamond$
requires a more complicated encoding involving four adjunctions between
four categories.  While specific adjunctions such as $(- \times A) \la
(A \to -)$ arise in any logic, adjoint logics provide a formalism for
abstract/uninterpreted adjunctions.

In Reed's logic, modes are specified by a preorder, which allows at most
one adjunction between any two categories (more precisely, there can be
two isomorphic adjunctions if both $p \ge q$ and $q \ge p$).  However,
it is sometimes useful to consider multiple different adjunctions
between the same two categories.  A motivating example is Lawvere's
axiomatic cohesion~\citep{lawvereXXcohesion}, a general categorical
interface that describes \emph{cohesive spaces}, such as topological
spaces and manifolds with differentiable or smooth structures.  The
interface consists of two categories \C and $\mathcal{S}$, and a
quadruple of adjoint functors $\Gamma,\Pi_0 : \mathcal{C} \to
\mathcal{S}$ and $\Delta,\nabla : \mathcal{S} \to \mathcal{C}$ where
$\Pi_0 \la \Delta \la \Gamma \la \nabla$.  The idea is that
$\mathcal{S}$ is some category of ``sets'' that provides a notion of
``point'', and \C\/ is some category of cohesive spaces built out of
these sets, where points may be stuck together in some way (e.g. via
topology).  $\Gamma$ takes the underlying set of points a cohesive
space, forgetting the cohesive structure.  Its left adjoint $\Delta$
equips a set with \emph{discrete cohesion}, where no points are stuck
together; the adjunction says that a map \emph{from} a discrete space is
the same as a map of sets.  Its right adjoint $\nabla$ equips a set with
codiscrete cohesion, where all points are stuck together; the adjunction
says that a map \emph{into} a codiscrete space is the same as a map of
sets.  The left adjoint $\Pi_0 \la \Delta$, gives the set of connected
components---i.e. each element of $\Pi_0 C$ is an equivalence class of
points of $C$ that are stuck together.  $\Pi_0$ is important because it
translates some of the cohesive information about a space into a setting
where we no longer need to care about the cohesion.  These functors must
satisfy some additional laws, such as $\Delta$ and $\nabla$ being fully
faithful (maps between discrete or codiscrete cohesive spaces should be
the same as maps of sets).

A variation on axiomatic cohesion called \emph{cohesive homotopy type
  theory}~\citep{schreibershulman12cohesive,shulman15realcohesion} is
currently being explored in the setting of homotopy type theory and
univalent foundations~\citep{voevodsky06note,uf13hott-book}.  Homotopy
type theory uses Martin-L\"of's intensional type theory as a logic
\emph{homotopy spaces}: the identity type provides an $\infty$-groupoid
structure on each type~\citep{...}, and spaces such as the spheres can
be defined by their universal properties using higher inductive
types~\citep{...}.  Theorems from homotopy theory can be proved
\emph{synthetically} in this logic~\citep{ls13pi1s1,lb13pinsn,...}, and
these proofs can be interpreted in a variety of models~\citep{...}.
However, an important but subtle distinction is that there is no
topology (in the sense of \citep{scott,etc}) in synthetic homotopy
theory: the ``homotopical circle'' is defined as a higher inductive
type, essentially ``the free $\infty$-groupoid on a point and a loop,''
which a priori has nothing to do with the ``topological circle,'' $\{
(x,y) \in \mathbb{R}^2 \mid x^2 + y^2 = 1\}$, where $\mathbb{R}^2$ has
the usual topology.  This is both a blessing and a curse: on the one
hand, proofs are not encumbered by topological details; but on the
other, internally to homotopy type theory, we cannot use synthetic
theorems to prove facts about topological spaces.

Cohesive homotopy type theory combines the synthetic homotopy theory of
homotopy type theory with the synthetic topology of axiomatic cohesion,
using an adjoint quadruple $\sh \la \Delta \la \Gamma \la \nabla$.  In
this higher categorical generalization, $\mathcal{S}$ is an
$(\infty,1)$-category of homotopy spaces (e.g. $\infty$-groupoids), and
$\C$ is an $(\infty,1)$-category of cohesive homotopy spaces, which are
additionally equipped with a topological or other cohesive structure at
each level.  The rules of type theory are now interpreted in \C, so that
each type has an $\infty$-groupoid structure (given by the identity
type) \emph{as well as} a separate cohesive structure on its objects,
morphisms, morphisms between morphisms, etc.  For example, types have
both morphisms, given by the identity type, and topological paths, given
by maps that are continuous in the sense of the cohesion.  As in the
1-categorical case, $\Gamma$ forgets the cohesive structure, yielding
the underlying homotopy space, while $\Delta$ and $\nabla$ equip a
homotopy space with the discrete and codiscrete cohesion.  But in the
$\infty$-categorical case, $\Delta$'s left adjoint $\sh A$ (pronounced
``shape of $A$'') generalizes from the connected components to the
\emph{fundamental homotopy space} functor, which makes a homotopy space
from the topological/cohesive paths, paths between paths, etc. of $A$.
This captures the process by which homotopy spaces arise from cohesive
spaces; for example, one can prove (using some additional axioms) that
the shape of the topological circle is the homotopy
circle~\citep{shulman15realcohesion}.  This allows synthetic
homotopy theory to be used in proofs about topological spaces, and opens
up possibilities for using synthetic homotopy theory as a tool in other
areas of mathematics and theoretical physics.

This paper begins an investigation into the structural proof theory of
cohesive homotopy type theory, as a special case of generalizing Reed's
adjoint logic to allow multiple adjunctions between the same categories.
As one might expect, the first step is to generalize the mode preorder
to a mode category, so that we can have multiple different morphisms
$\alpha, \beta : p\ge q$.  This allows the logic to talk about different
but unrelated adjunctions between two categories.  However, in order to
describe an adjoint triple such as $\Delta \la \Gamma \la \nabla$, we
need to know that the same functor $\Gamma$ is both a left and right
adjoint.  To describe such a situation, we generalize to a 2-category of
modes, where each mode $p$ determines a category, each morphism $\alpha
: p \ge q$ determines adjoint functors $F_\alpha : p \to q$ and
$U_\alpha : q \to p$ where $F_\alpha \la U_\alpha$, and each 2-cell $e :
\tc \alpha  \beta : q \ge p$ determines a morphism of adjunctions between 
$F_\alpha \la U_\alpha$ and $F_\beta \la U_\beta$.  For example, an
adjoint triple is described by the 2-category with
\begin{itemize}
\item objects $c$ and $s$
\item 1-cells $\Delta : s \ge c$ and $\nabla : c \ge s$
\item 2-cells $\tc {1_c} {\nabla \Delta}$ 
and $\tc {\Delta \nabla} {1_s}$ satisfying 
some equations
\end{itemize}
The 1-cells generate $F_\Delta \la U_\Delta$ and $F_\nabla \la
U_\nabla$, while the 2-cells are sufficient to prove that $U_\Delta$ is
naturally isomorphic to $F_\nabla$, so we can define $\Delta :=
F_\Delta$, $\nabla := U_\nabla$, and $\Gamma := U_\Delta \cong F_\nabla$
and have the desired triple adjunction.  Indeed, you may recognize this
2-category as the ``walking adjunction'' with $\Delta \la \nabla$---that
is, we give a triple adjunction by saying that the morphism generating
the left adjoint is itself left adjoint to the morphism generating the
right adjoint.

The resulting logic has a good definition-to-theorem ratio, in the sense
that from some simple sequent calculus rules for $F$ and $U$, we can
prove a variety of general facts that are true for any mode 2-category
($F_\alpha$ and $U_\alpha$ are functors; $F_\alpha U_\alpha$ is a
comonad and $U_\alpha F_\alpha$ is a monad; $F_\alpha$ preserves
colimits and $U_\alpha$ preserves limits), as well as facts specific to
a particular theory (e.g. for the triple adjunction above, $\Gamma$
preserves both colimits and limits, because it is an equivalent
$U_\Delta$ and $F_\nabla$; the comonad $\flat := \Delta\Gamma$ and monad
$\sharp := \nabla\Gamma$ are themselves adjoint).  Moreover, we can use
different mode 2-categories to add aditional structure; for example,
moving from the walking adjunction to the walking reflection (taking
$\Delta \nabla = 1$) additionally gives that $\Delta$ and $\nabla$
are full and faithful, and that $\flat$ and $\sharp$ are idempotent,
which are some of the additional conditions for axiomatic cohesion.
The logic also has good proof-theoretic properties: we start from a cut-
and identity-free sequent calculus and prove that cut and identity are
admissible.  While we do not consider focusing~\citep{andreoli92focus},
we conjecture that the connectives have the same focusing behavior as in
\citep{reed09adjoint}: $F$ is postive and $U$ is negative (which,
because limits are negative and colimits are positive, and like-polarity
connectives compose together well, matches what left and right adjoints
should preserve).  

To study the basic ideas of 2-categorical adjoint logic, we make a few
simplifying restrictions for this paper. First, we consider only 
single-hypothesis, single-conclusion sequents, defering an investigation
of products and exponentials to future work.  Second, on the semantic
side, we consider only 1-categorical semantics of the derivations of the
logic, rather than the $\infty$-groupoid semantics that we are
ultimately interested in.  More precisely, for a specific 2-category
\M\/ of modes, we can interpret the logic using a pseudofunctor $S : \M
\to \Adj$, where $\Adj$ is the 2-category of categories, adjunctions,
and morphisms of adjunctions (conjugate natural transformation).  By
$S$, each mode determines a 1-category, and derivations in the logic are
interprered as morphisms in these categories.  The action of $S$ on 1-
and 2-cells is used to interpret $F$ and $U$.  Third, we consider only a
logic of simple-types, rather than a dependent type theory.

Because of the last restriction, we do not have an identity type
available for proving equalities of proof terms.  However, we need an
equational theory to make many of the statements above (e.g. ``$UF$ is a
monad'' require proving some equational laws), and the definitional
equalities arising from admissibility of cut and identity are not
sufficient.  Thus, in addition to the sequent calculus itself, we give
an equality judgement on sequent calculus derivations.  This judgement
is interpreted by actual equality of morphisms in the semantics above,
but we intend these rules to be propositional equalities in an eventual
adjoint type theory.

In Section~\ref{sec:rules}, we define the rules of the logic, prove
admissibility of identity and cut, and prove various equational
properties of these operations.  In Section~\ref{sec:examples}, we give
a variety of examples of constructions in the logic, including those
mentioned above, as well as the rules for spatial type theory used
in~\citep{shulman15realcohesion}.  In Section~\ref{sec:semantics}, we
describe the 1-categorical semantics of the logic in the settings
described above.  All of the syntactic metatheory of the logic and the
examples have been formalized in Agda~\citep{norell07thesis}.

\section{Rules}
\label{sec:rules}

\newcommand\compo[2]{\ensuremath{#1 \circ #2}}
\newcommand\compv[2]{\ensuremath{#1 \cdot #2}}
\newcommand\comph[2]{\ensuremath{#1 \star #2}}

\subsection{Sequent Calculus}

The logic is parametrized by a 2-category of modes.  We write $p,q$ for
the 0-cells (modes), $\alpha,\beta,\gamma,\delta$ for the 1-cells $p \ge
q$, and $e : \tc \alpha \beta$ for the 2-cells.  We write
\compo{\beta}{\alpha} for 1-cell composition in function composition
order (i.e. if $\beta : r \ge q$ and $\alpha : q \ge p$ then
$\compo{\beta}{\alpha} : r \ge p$), \compv{e_1}{e_2} for vertical
composition of 2-cells in diagramatic order, and \comph{e_1}{e_2} for
horizontal composition of 2-cells.  We assume the 2-category is
strict---the associativity and unit laws for 1-cells hold as equality of
1-cells, rather than as 2-cells.  For 2-cells, recall that the equations
say that \compv{}{} is associative with unit $1_\alpha$ for any
$\alpha$, that \comph{}{} is associative with unit $1_1$, and that the
interchange law $\comph{(\compv{e_1}{e_2})}{(\compv{e_3}{e_4})} =
\compv{(\comph{e_1}{e_3})}{(\comph{e_2}{e_4})}$ holds.  We treat these
equations as definitional equalities.

\renewcommand\wftp[2]{\ensuremath{#1 \,\,\, \dsd{type}_{#2}}}
\newcommand\F[2]{\ensuremath{F_{#1} \,\, #2}}
\newcommand\U[2]{\ensuremath{U_{#1} \,\, #2}}
\newcommand\coprd[2]{\ensuremath{#1 + #2}}
\newcommand\seq[3]{\ensuremath{#1 \, [ #2 ] \, \vdash \, #3}}
\renewcommand\irl[1]{\dsd{#1}}

Each object $p$ of the mode category determines a category, and objects
of that category are types; the syntactic judgement \wftp{A}{p} will
mean that $A$ is an object of the category $p$.  A morphism $\alpha : q
\ge p$ in the mode category determines an adjunction between categories
$p$ and $q$, with $\F \alpha {} : q \to p$ and $\U \alpha {} : p \to q$;
syntactically, we have $\wftp{\F \alpha A}{p}$ when $\wftp{A}{q}$ and
$\wftp{\U \alpha A}{q}$ when $\wftp{A}{p}$.  We write $P$ for atomic
propositions, each of which has a specified mode.  To add additional
structure to a category or to all categories, we can add rules for
additional connectives; for example, a rule \wftp{\coprd{A}{B}}{p} if
\wftp{A}{p} and \wftp{B}{p} (parametric in $p$) says that any mode
$p$ has a coproduct type constructor.

The sequent calculus judgement has the form \seq A \alpha C where
\wftp{A}{q} and \wftp{C}{p} and $\alpha : q \ge p$.  That is, the
judgement represents a map from an object of some category $q$ to an
object of another category $p$ along some adjunction $\F \alpha {} \la
\U \alpha {}$.  Semantically, this mixed-category map can be interpreted
equivalently as an arrow \arrow{p}{\F \alpha A}{C} or \arrow{q}{A}{\U
  \alpha C}.  In the rules, we write $A_p$ to abbreviate a premise
\wftp{A}{p}.

The rules for atomic propositions and for $U$ and $F$ are as follows:
\[
\begin{small}
\begin{array}{c}
\infer[\irl{hyp}]
      {\seq P \alpha P}
      {\tc 1 \alpha}
\\ \\
\infer[\irl{FL}]
      {\seq {\F \alpha A_p} {\beta}{C_r}}
      { \alpha : r \ge q & \beta : q \ge p & 
        \seq {A_r} {\compo{\alpha}{\beta}} {C_p}
      }
\quad
\infer[\irl{FR}]
      {\seq {C_r} \beta {\F \alpha A_q}}
      { \alpha : q \ge p & \beta : r \ge p & 
        \gamma : r \ge q & \tc{\compo{\gamma}{\alpha}}{\beta} &
        \seq {C_r} \gamma {A_q}}
\\ \\
\infer[\irl{UL}]
      {\seq {\U \alpha A_q} {\beta} {C_p}}
      {\alpha : r \ge q &
        \beta : r \ge p &
        \gamma : q \ge p &
        \tc{\compo{\alpha}{\gamma}} {\beta} &
        \seq{A_q}{\gamma}{C_p}}
\quad
\infer[\irl{UR}]
      {\seq {C_r} {\beta} {\U {\alpha} A_p}}
      {\alpha : q \ge p &
       \beta : r \ge q &
       \seq {C_r} {\compo{\beta}{\alpha}} {A_p}}
\end{array}
\end{small}
\]
We include the premises for $\alpha$ and $\beta$ because they are
helpful for reading the rules, but they really part of the
well-formedness of the conclusion sequent, so we will elide them
whenever we give derivations.   

The rules for other connectives do not change $\alpha$; for example, we
can add coproducts to every category with the following rules (where
$\alpha : q \ge p$):
\[
\begin{small}
\infer[\irl{Inl}]
      {\seq {C_q} {\alpha} {\coprd{A_p}{B_p}}}
      {\seq {C_q} {\alpha} {A_p}}
\quad
\infer[\irl{Inr}]
      {\seq {C_q} {\alpha} {\coprd{A_p}{B_p}}}
      {\seq {C_q} {\alpha} {B_p}}
\quad
\infer[\irl{Case}]
      {\seq {\coprd{A_q}{B_q}} {\alpha} {C_p}}
      {\seq {A_q} {\alpha} {C_p} & 
       \seq {B_q} {\alpha} {C_p} 
      }
\end{small}
\]
Alternatively, we could restrict these rules to where $p$ is some
specific set of modes to have coproducts only in those categories.

\emph{FIXME: need to explain what's going on, but let's see where we're going first...}

\subsection{Admissible Rules}

\paragraph{Respect for 2-cells}

The following rule provides the action of a 2-cell $e : \tc \alpha
\beta$ on a derivation $D : \seq{A}{\alpha}{B}$.  Semantically, 
$D$ is interpreted as a map \arrow{\F \alpha A}{B} (say), 
so to get a map \arrow{\F \beta A}{B} we can precompose with the
morphism of adjunctions determined by $e$.  

\newcommand\tr[2]{\ensuremath{{{#1}_{*}(#2)}}}
\newcommand\ident[1]{\ensuremath{\dsd{ident}_{#1}}}
\newcommand\cut[2]{\ensuremath{{\dsd{cut} \,\, #1 \,\, #2}}}

\[
\infer[\irl{\tr{-}{-}}]
      {\seq A {\beta} B}
      {\tc \alpha \beta &
       \seq A {\alpha} {C}}
\]

\paragraph{Identity}

The identity rule is admissible:
\[
\infer[\irl{ident}]
      {\seq {A_p} {1} {A_p}}
      {}
\]

The general strategy is ``apply the invertible rule and then the focus
rule and then the inductive hypothesis.'' For example, for $\F \alpha
A$, the following reduces the problem to identity on $A$:
\[
\infer[\irl{FL}]
      {\seq{\F {\alpha : q \ge p} A}{1}{\F \alpha A}}
      {\infer[\irl{FR}]
             {\seq{A}{\alpha}{\F \alpha A}}
             {1_q : q \ge q & 1 : \tc{\compo{1}{\alpha}}{\alpha} &
               \seq{A}{1}{A}}}
\]
$U$ is dual; for atomic propositions identity is a rule.  

\paragraph{Cut}

The following cut rule is admissible:
\[
\infer[\irl{cut}]
      {\seq {A_r} {\compo{\beta}{\alpha}} {C_p}}
      {\seq {A_r} {\beta} {B_q} &
       \seq {B_r} {\alpha} {C_p}}
\]

\emph{FIXME: how much of the code for these is necessary to understand the
  rest?}

\subsection{Equations}

\newcommand\hyp[1]{\ensuremath{\dsd{hyp}{#1}}}
\newcommand\Inl[1]{\ensuremath{\dsd{Inl}(#1)}}
\newcommand\Inr[1]{\ensuremath{\dsd{Inr}(#1)}}
\newcommand\Case[2]{\ensuremath{\dsd{Case}(#1,#2)}}
\newcommand\UL[3]{\ensuremath{\dsd{UL}^{#1}_{#2}(#3)}}
\newcommand\FR[3]{\ensuremath{\dsd{FR}^{#1}_{#2}(#3)}}
\newcommand\FL[1]{\ensuremath{\dsd{FL}(#1)}}
\newcommand\UR[1]{\ensuremath{\dsd{UR}(#1)}}

\newcommand\ap[2]{\ensuremath{#1 \approx #2}}

The above definitions of \tr{e}{D} and \ident{A} and \cut{D}{E} give a
notion of definitional equality of proofs, but to prove the desired
equations in the examples below, we will need some additional
``propositional'' equations.  We define a judgement \ap{D}{D'} on two
derivations $D,D' : \seq{A}{\alpha}{C}$ to be the least congruence
closed under the following rules.  First, we have uniqueness/$\eta$
rules, which say that any map from a positive/into a negative is equal
to a derivation that begins with an application of the left/right rule,
respectively, and then cuts the original derivation with the right/left
rule, respectively.  For $F$ and $U$ we have:

\[
\begin{array}{l}
\infer[\irl{F\eta}]
      {\ap{D}{\FL {\cut{(\FR 1 1 {\ident{A}})}{D}} }}
      {D : \seq{\F \alpha A}{\beta}{C}}
\quad
\infer[\irl{U\eta}]
      {\ap{D}{\UR {\cut{D}{(\UL 1 1 {\ident{A}})}}}}
      {D : \seq{C}{\beta}{\U \alpha A}}
\end{array}
\]

Second, we have rules arising from the 2-cell structure:

\[
FIXME
\]

The final class of rules we need are an artifact of the sequent calculus
presentation, and would not be necessary in a more traditional natural
deduction calculus; they say that left rules of negatives and right
rules of positives commute.  

\[
FIXME
\]

For other positive connectives, such as our running example of
coproducts, we will also need their $\eta$ rule, and rules commuting
negative left rules with their right rule
(e.g. \ap{\Inl{\UL{\gamma}{e}{D}}}{\UL{\gamma}{e}{\Inl{D}}} ).
For other negative connectives, we will need their $\eta$ rule, and
rules commuting their left rule with the right rules for all positives.  

\paragraph{Admissible rules}

The following equality rules are admissible.  The proofs have a lot of
cases (about 800 lines of Agda total) but are not difficult, except for
somewhat subtle staging.  The rules in each of the following groups are
proved by mutual induction, and use the preceding groups:

\begin{enumerate}
\item 
For each $D$, \tr{e}{D} is functorial in the 2-cell identity and
vertical composition in $e$: 

\[
\infer{\tr{1}{D} \equiv D}{}
\qquad
\infer{\tr{(\compv{e_1}{e_2})}{D} \equiv \tr{e_2}{\tr{e_1}{D}}}
      {}
\]

It is important for the remaining proofs that these are definitional
equalities, not just $\ap{}{}$, so that we can use them ``in context''
before we know that cut is well-defined on $\ap{}{}$.

\item 
\tr{e}{-} is well-defined on \ap{-}{-}:

\[
\infer{\ap{\tr{e}{D}}{\tr{e}{D'}}}
      {\ap{D}{D'}}
\]

and \tr{e}{-} commutes with cut:

\[
\infer{\ap{\tr{(\comph{e}{e'})}{\cut{D}{D'}}}{\cut{(\tr{e}{D})}{(\tr{e'}{D'})}}}
      {e : \tc{\alpha}{\alpha'} &
       e' : \tc{\beta}{\beta'} &
       D : \seq{A}{\alpha}{B} &
       D' : \seq{B}{\beta}{C}}
\]

\item Cut is associative:

\[
\infer{\ap{\cut{D_1}{(\cut{D_2}{D_3})}}{\cut{(\cut{D_1}{D_2})}{D_3}}}
      {}
\]

\item Cut is well-defined on $\ap{}{}$, identities are units for cut,
  and cutting with a left rule is equivalent to pushing inside:

\[
\infer{\ap{\cut{D}{\ident{}}}{D}}
      {}
\quad
\infer{\ap{\cut{\ident{}}{D}}{D}}
      {}
\quad
\infer{\ap{\cut{D}{E}}{\cut{D'}{E}}}
      {\ap{D}{D'}}
\quad
\infer{\ap{\cut{D}{E}}{\cut{D}{E'}}}
      {\ap{E}{E'}}
\]

FIXME cutUL, cutFL, 

\end{enumerate}

Because the operations \tr{e}{-} and \cut{-}{-} are well-defined, we can
quotient derivations by $\ap{}{}$.  The \wftp{A}{p} and quotiented
derivations of \seq{A}{1_p}{B} form a category, with identities given by
\ident{A} and composition given by \cut{}{}.  

\section{Examples}
\label{sec:examples}

\subsection{General}

\subsection{Adjoint Triple}

\subsection{Reflection}

%% The first presentation of cohesive homotopy type
%% theory~\citep{schreibershulman12cohesive} did not extend the judgement
%% structure of type theory, and consequently had trouble with the
%% $\Delta\Gamma$ composite comonad.  

%% Because $\Delta$ and $\nabla$ are full and faithful, it is possible to
%% avoid mention of the category $\mathcal{S}$, and instead focus on the
%% composite endofunctors $\sh := \Delta\sh$, $\flat := \Delta\Gamma$, and
%% $\sharp := \nabla\Gamma$ on \C.  For any adjoint quadruple as above,
%% these will form an adjoint triple $\sh \la \flat \la \sharp$, and $\sh$
%% and $\sharp$ will be monads, while $\flat$ will be a comonad.

%% From a proof-theoretic perspective the $\Gamma$ in a triple adjunction
%% $\Delta \la \Gamma \la \nabla$ is a bit of an odd duck, because it is
%% both a right and a left adjoint, and so should behave like both the $F$
%% and $U$ in adjoint logic.  One possibility would be to design rules for
%% a left-and-right-adjoint connective.  However, such a connective would
%% need to have an unusual focusing~\citep{andreoli92focus} behavior: In
%% any adjunction, left adjoints preserve colimits (e.g. coproducts) and
%% right adjoints preserve limits (e.g. products).  In focusing, limits are
%% negative connectives (invertible on the right, focus on the left), while
%% colimits are positive connectives (invertible on the left, focus on the
%% right), and in general like-polarity connectives compose together well.
%% Reed's adjoint logic is compatible with these two facts, because it
%% admits a focusing interpretation where the left adjoint $F$ is positive
%% and the right adjoint $U$ is negative.  However, if $\Gamma$ is both a
%% left and a right adjoint, it seems like it would need to be both
%% positive and negative.  

%% Instead of giving rules for an ambipolar connective, we will provide a
%% logic in which this comes true up to isomorphism: we can describe a quadruple
%% \[
%% F \la U \cong F' \la U'
%% \]
%% and then define $\Delta := F$, $\nabla := U'$, $\Gamma := U \cong F'$.
%% That is, the functor $\Gamma$ will be defined to be one of two naturally
%% isomorphic functors, one of which is a left adjoint/positive and the
%% other of which is a right adjoint/negative.  This idea is present in
%% Reed's encoding of labeled deduction~\citep[Section ?]{reed09adjoint},
%% where two modes $p,q$ such that $p \ge q$ and $q \ge p$ give rise to
%% adjunctions $F \la U$ and $F' \la U'$ such that $F \cong U'$ and $U
%% \cong F'$.  However, in the case of an adjoint triple, we want $U \cong
%% F'$ but not $F \cong U'$, because $\Delta$ and $\nabla$ are not the
%% same.  

\section{Semantics}
\label{sec:semantics}

\section{Conclusion}

Discuss \sh{} here?


%% \setlength{\bibsep}{-1pt} %% dirty trick: make this negative
{ %% \small
%% \linespread{0.70}
\bibliographystyle{abbrvnat}
\bibliography{drl-common/cs}
}


\end{document}

\documentclass{amsart}
\usepackage{mathpartir,latexsym,amssymb,stmaryrd,mathtools}
\usepackage[all]{xy}
\title{Something involving bifibrations of cartesian 2-multicategories and generalized substructural type theories}
\author{Daniel R. Licata \and Michael Shulman}
\thanks{
This material is based on research sponsored by The United States Air
Force Research Laboratory under agreement number FA9550-15-1-0053. The
U.S. Government is authorized to reproduce and distribute reprints for
Governmental purposes notwithstanding any copyright notation thereon.
The views and conclusions contained herein are those of the authors and
should not be interpreted as necessarily representing the official
policies or endorsements, either expressed or implied, of the United
States Air Force Research Laboratory, the U.S. Government, or Carnegie
Mellon University.
}
\newtheorem{thm}{Theorem}[section]
\theoremstyle{definition}
\newtheorem{defn}[thm]{Definition}
\def\M{\mathcal{M}}
\def\D{\mathcal{D}}
\def\toiso{\xrightarrow{\sim}}
\let\To\Rightarrow
%\newcommand\compo[2]{\ensuremath{#1 \circ #2}}
\newcommand\compv[2]{\ensuremath{#1 \cdot #2}}
\newcommand\comph[2]{\ensuremath{#1 \mathbin{\circ_2} #2}}
\newcommand\wftp[2]{\ensuremath{#1 \,\,\, \mathsf{type}_{#2}}}
\newcommand\seq[3]{\ensuremath{#1 \, [ #2 ] \, \vdash \, #3}}
\begin{document}
\maketitle

\section{Bifibrations of cartesian 2-multicategories}
\label{sec:bifib-cart-2multi}

\begin{defn}
  A \textbf{(strict) cartesian 2-multicategory} consists of
  \begin{enumerate}
  \item A set $\M_0$ of \emph{objects}.
  \item For every object $B$ and every finite list of objects $(A_1,\dots,A_n)$, a category $\M(A_1,\dots,A_n;B)$.
    The objects of this category are \emph{1-morphisms} and its morphisms are \emph{2-morphisms}; we write composition of 2-morphisms as $\compv{e_1}{e_2}$.
  \item For each object $A$, an identity arrow $1_A\in\M(A;A)$.
  \item For any object $C$ and lists of objects $(B_1,\dots,B_m)$ and $(A_{i1},\dots,A_{in_i})$ for $1\le i\le m$, a composition functor
    \begin{align*}
      \M(B_1,\dots,B_m;C) \times \prod_{i=1}^m \M(A_{i1},\dots,A_{in_i};B_i) &\longrightarrow \M(A_{11},\dots,A_{mn_m};C)\\
      (g,(f_1,\dots,f_m)) &\mapsto g\circ (f_1,\dots,f_m)
    \end{align*}
    We write the action of this functor on 2-cells as $\comph{d}{(e_1,\dots,e_m)}$.
  \item For any $f\in\M(A_1,\dots,A_n;B)$ we have natural equalities (i.e.\ natural transformations whose components are identities)
    \begin{mathpar}
      1_B \circ (f) = f\and
      f\circ (1_{A_1},\dots,1_{A_n}) = f.
    \end{mathpar}
  \item For any $h,g_i,f_{ij}$ we have natural equalities
    \begin{multline*}
      (h\circ (g_1,\dots,g_m))\circ (f_{11},\dots,f_{mn_m}) =\\
      h \circ (g_1\circ (f_{11},\dots,f_{1n_1}), \dots, g_m \circ (f_{m1},\dots,f_{mn_m}))
    \end{multline*}
  \item For any function $\sigma : \{1,\dots,m\} \to \{1,\dots,n\}$ and objects $A_1,\dots,A_n,B$, a functor
    \begin{align*}
      \M(A_{\sigma 1},\dots,A_{\sigma m}; B) &\to \M(A_1,\dots,A_n;B)\\
      f &\mapsto f\sigma^*
    \end{align*}
  \item The functors $\sigma^*$ satisfy the following natural equalities:
    \begin{gather*}
      f \sigma^* \tau^* = f(\tau\sigma)^*\\
      f (1_n)^* = f\\
      g\circ (f_1 \sigma_1^* ,\dots, f_n \sigma_n^*) = (g \circ (f_1,\dots,f_n))(\sigma_1\sqcup \cdots \sqcup \sigma_n)^*\\
      g\sigma^* \circ (f_1,\dots,f_n) = (g\circ (f_{\sigma 1},\dots, f_{\sigma m}))(\sigma \wr (k_1,\dots,k_n))^*
    \end{gather*}
    In the last equation, $k_i$ is the arity of $f_i$, and $\sigma \wr (k_1,\dots,k_n)$ denotes the composite function
    \begin{equation*}
      \{1,\dots,\textstyle\sum_{i=1}^m k_{\sigma i} \}
      \toiso \bigsqcup_{i=1}^m \{1,\dots,k_{\sigma i}\}
      \xrightarrow{\widehat{\sigma}} \bigsqcup_{j=1}^n \{1,\dots,k_{j}\}
      \toiso \{1,\dots,\textstyle\sum_{j=1}^n k_j \}
    \end{equation*}
    where $\widehat{\sigma}$ acts as the identity from the $i^{\mathrm{th}}$ summand to the $(\sigma i)^{\mathrm{th}}$ summand.
  \end{enumerate}
\end{defn}

A cartesian 2-multicategory encodes the structure of the modes.
We will generally be interested in cartesian 2-multicategories given by \emph{presentations} consisting of some objects (``modes''), some generating 1-morphisms, some equalities and some 2-cells between composites of those (possibly including $\sigma$-actions), and some equalities between composites of the generating 2-cells.
For example:
\begin{itemize}
\item Suppose we have one mode $p$, generating morphisms $\mu:(p,p)\to p$ and $\eta:()\to p$, and equations $\mu\circ (1,\eta) = 1 = \mu\circ (\eta,1)$ and $\mu\circ (1,\mu) = \mu \circ (\mu,1)$.
  This is the free cartesian 2-multicategory containing a monoid, and the type theory it generates is noncommutative intuitionistic linear logic.
\item Suppose we add the additional axiom $\mu\sigma^* = \mu$, where $\sigma$ is the nonidentity automorphism of $\{1,2\}$.
  This is the free cartesian 2-multicategory containing a commutative monoid, and the type theory it generates is (commutative) intuitionistic linear logic.
\item Suppose we also add a generating 2-cell $t:1_p \To \eta\tau^*$, where $\tau:\emptyset \to \{1\}$ is the unique function, so that $\tau^* : \M(;p) \to \M(p;p)$, and also an axiom $\comph{t}{1_\eta} = 1_\eta$.
  This is the free cartesian 2-multicategory containing a commutative ``semicartesian'' monoid (whose unit is a ``terminal object''), and the type theory it generates is intuitionistic affine logic.
\item Suppose we also add a generating 2-cell $d:1_p \to \mu\delta^*$, where $\delta:\{1,2\} \to \{1\}$, is the unique function, so that $\delta^* : \M(p,p;p) \to \M(p;p)$, and also some axioms.
  Then we get the free cartesian 2-multicategory containing a cartesian object, and the type theory it generates is ordinary intuitionistic logic.
\item Starting over, suppose the same mode $p$ has both a commutative monoid structure and an unrelated cartesian monoid structure.
  This should generate Bunched Implication.
\item With two modes $p,q$, where $p$ is a cartesian monoid and $q$ a commutative monoid, and a morphism $p\to q$ respecting the monoid structures, we should get the adjoint decomposition of the linear exponential.
\item In a similar case but where both $p$ and $q$ are cartesian, we should get the adjoint decomposition of Pfenning--Davies $\Box$.
\item With one mode $p$ having a cartesian monoid structure, and an idempotent monad on it respecting the monoid structure, we should get a simply-typed multi-variable version of spatial type theory.
\end{itemize}

Now, as a matter of fact, type theory itself excels at presenting categorical structures by generators and relations, especially those of the cartesian multicategory sort!
Ordinary simple type theory with no type operations (but with generators and axioms) yields presentations of ordinary cartesian multicategories.
Thus, we only need to enhance this with some judgments for 2-cells in order to give convenient presentations of cartesian 2-multicategories.
[I'm not sure exactly what those judgments should look like, but you were using something like them on your slides, so maybe you have a better idea.]

In any case, if we wrap up the entire type theory into one, it will have judgments for modes, mode terms, mode 2-cells, types, and type terms/sequents (and perhaps equalities of those).
The first three judgments present a cartesian 2-multicategory; the latter two or three give the following structure over it.

\begin{defn}
  A \textbf{local discrete fibration} over a cartesian 2-multicategory $\M$ consists of
  \begin{enumerate}
  \item A functor of cartesian 2-multicategories $\pi:\D\to\M$.
  \item Each induced functor of ordinary categories
    \[\D(A_1,\dots,A_n;B)\to\M(\pi A_1,\dots,\pi A_n;\pi B)\]
    is a discrete fibration.
    We write $\D_f(A_1,\dots,A_n;B)$ for the fiber of this functor over $f\in \M(\pi A_1,\dots,\pi A_n;\pi B)$; the assumption ensures that this fiber is a discrete set.
  \item For each function $\sigma : \{1,\dots,m\} \to \{1,\dots,n\}$, objects $A_1,\dots,A_n,B \in \D$, and $f\in\M(\pi A_{\sigma 1},\dots,\pi A_{\sigma m}; \pi B)$, the functor
    \[\D_f(A_{\sigma 1},\dots,A_{\sigma m}; B) \to \D_{f\sigma^*}(A_1,\dots,A_n;B)\]
    is an isomorphism.
    (By the previous point, it is actually a mere function between discrete sets.)
  \end{enumerate}
\end{defn}

Local discrete fibrations over a fixed $\M$ are the categorical structure corresponding to the \emph{judgmental structure} of the type theory generated by $\M$, as follows:
\begin{itemize}
\item Each type of the type theory is assigned an object of $\M$ as its mode, written $\wftp{A}{p}$; these correspond to objects of $\D$ living over $p$ via $\pi$.
\item Each sequent
  \[ \seq{A_1,\dots,A_n}{\alpha}{B}\]
  is labeled by a morphism $\alpha:(p_1,\dots,p_n) \to q$ of $\M$, where $\wftp{A_i}{p_i}$ and $\wftp{B}{q}$; these correspond to the elements of $\D_\alpha(A_1,\dots,A_n;B)$.
\item Composition in $\D$ corresponds to a cut rule as usual for multicategories, and functoriality of $\pi$ means that the cut rule is annotated by composition of morphisms in $\M$;
  \begin{mathpar}
    \inferrule{\seq{A_1,\dots,A_n}{\alpha}{B} \\ \seq{\Gamma_i}{\beta_i}{A_i}}{\seq{\Gamma_1,\dots\Gamma_n}{\alpha\circ(\beta_1,\dots,\beta_n)}{B}}
  \end{mathpar}
  Similarly, identities in $\D$ correspond to the variable rule $\seq{A}{1_p}{A}$, where \wftp{A}{p}.
\item The local fibration structure of $\pi$ corresponds to the pull-back rule
  \begin{mathpar}
    \inferrule{\seq{A_1,\dots,A_n}{\beta}{B} \\ e:\alpha\To\beta}{\seq{A_1,\dots,A_n}{\alpha}{B}}
  \end{mathpar}
  (As in one-variable adjoint logic, to make this admissible, one has to build it into some of the rules, including the variable rule.
  Note that the variance of this is reversed from the one-variable adjoint logic paper.
  Categorically, the reason for this is that when representing a fully \emph{covariant} 2-functor $\M\to\mathit{Cat}$ by its Grothendieck construction, the latter is nevertheless an opfibration on 1-cells but a \emph{fibration} on 2-cells.)
\item Finally, the $\sigma^*$ actions on $\D$ correspond to the structural rules of exchange, contraction, and weakening ``with resource-usage labels''.
  The assumption that each $\sigma^*$ acts isomorphically on the local fibers of $\D$ ensures that these resource labels are honest, and the ``actual'' structural rules (when permitted by the mode theory) are implemented by the action of certain 2-cells via the pull-back rule.
\end{itemize}

This last point deserves more explanation.
Suppose $p\in\M$ has a monoid structure $\mu:(p,p)\to p$, and we have $\wftp{A}{p}$ and $\wftp{B}{p}$ and a sequent $\seq{A,A}{\mu}{B}$.
Since $\M$ is a cartesian multicategory, we can act on $\mu$ by $\delta:\{1,2\} \to \{1\}$ to get $\mu\delta^*:p \to p$, which (without further assumptions on $p$) is unrelated to $1_p:p\to p$.

Similarly, we can act by $\delta$ on our sequent $\seq{A,A}{\mu}{B}$ to get $\seq{A}{\mu\delta^*}{B}$.
Since $\mu\delta^*$ is not the same as $1_p$, this is different from asserting $\seq{A}{1_p}{B}$; the mode morphism $\mu\delta^*$ is still ``recording the fact that $A$ is used twice''.
The assumption that $\delta^*$ acts isomorphically on the local fibers of $\pi$ ensures that to say $\seq{A}{\mu\delta^*}{B}$ is really exactly the same as saying $\seq{A,A}{\mu}{B}$, i.e.\ whenever we have the first we also have the second.
Type-theoretically, I expect that this would be an \emph{invertible} admissible rule
\begin{mathpar}
  \mprset{fraction={===}}
  \inferrule*{\seq{A_{\sigma 1},\dots,A_{\sigma m}}{\alpha}{B}}{\seq{A_1,\dots,A_n}{\alpha\sigma^*}{B}}
\end{mathpar}

But if it should happen that $p$ has a \emph{cartesian} monoid structure, with a 2-cell $d:1_p \to \mu\delta^*$, then we can apply the pull-back rule to $\seq{A}{\mu\delta^*}{B}$ to get $\seq{A}{1_p}{B}$.
This is the ``real'' contraction rule that is only allowed when $p$ is cartesian.

Finally, the connectives that ``internalize'' the judgmental structure are represented categorically by cartesian and opcartesian arrows.

\begin{defn}
  If $\pi:\D\to\M$ is a local discrete fibration, then a morphism $\xi\in\D(A_1,\dots,A_n;B)$ is \textbf{opcartesian} if all diagrams of the following form are pullbacks of categories:
  \[ \xymatrix{
    \D(\vec C,B,\vec D;E) \ar[r]^-{-\circ \xi} \ar[d]_\pi &
    \D(\vec C,\vec A,\vec D;E) \ar[d]^\pi \\
    \M(\pi\vec C,\pi B, \pi\vec D; \pi E) \ar[r]_-{-\circ \pi\xi} &
    \M(\pi\vec C,\pi\vec A,\pi\vec D;\pi E)
  }\]
  Dually, a morphism $\xi\in\D(\vec C,B,\vec D;E)$ is \textbf{cartesian at $B$} if all diagrams of the following form are pullbacks of categories:
  \[ \xymatrix{
    \D(\vec A;B) \ar[r]^-{\xi\circ -} \ar[d]_\pi &
    \D(\vec C,\vec A,\vec D;E) \ar[d]^\pi \\
    \M(\pi\vec A;\pi B) \ar[r]_-{\pi\xi\circ -} &
    \D(\pi\vec C,\pi\vec A,\pi\vec D;\pi E)
  }\]
  Given $\mu:(p_1,\dots,p_n) \to q$ in $\M$, we say that $\pi$ \textbf{has $\mu$-products} if for any $A_i$ with $\pi A_i = p_i$, there exists a $B$ with $\pi B = q$ and an opcartesian morphism in $\D_\mu(A_1,\dots,A_n;B)$.
  Dually, we say $\pi$ \textbf{has $\mu$-homs} if for any $i$, any $B$ with $\pi B = q$, and any $A_j$ with $\pi A_j = p_j$ for $j\neq i$, there exists an $A_i$ with $\pi A_i = p_i$ and a cartesian morphism in $\D_\mu(A_1,\dots,A_n;B)$.
\end{defn}

References: Hermida, H\"{o}rmann.

For example:
\begin{itemize}
\item The intuitionistic linear logic $A \otimes B$ is a $\mu$-product for the multiplication $\mu:(p,p)\to p$ of a (commutative) monoid mode.
  Its adjoint $A\multimap B$ is a $\mu$-hom for the same $\mu$.
  Similarly, the linear unit $\mathbf{1}$ is an $\eta$-product for the unit $\eta :()\to p$.
\item The cartesian product $A\times B$ is a $\mu$-product for the multiplication of a cartesian monoid mode, and its right adjoint $A\to B$ is similarly a $\mu$-hom.
\item If the linear logic $!$ is written as $F U$, then $F$ and $U$ are the $\alpha$-product and $\alpha$-hom for the (unary!)\ mode morphism $p\to q$ from the cartesian to the linear mode.
  The situation with Pfenning--Davies $\Box$ is similar.
\item The spatial type theory $\flat$ is an $\alpha$-product where $\alpha$ is an idempotent monad, and similarly $\sharp$ is an $\alpha$-hom for the same $\alpha$.
\end{itemize}


\section{Generalized multicategories}
\label{sec:genmulti}

Let $\M$ be a cartesian 2-multicategory, with object set $\M_0$.
Then it induces a 2-monad $T_\M$ on $\mathit{Cat}^{\M_0}$ whose algebras are a sort of ``strong pseudofunctor'' $\M\to \mathit{Cat}$, where $\mathit{Cat}$ has a cartesian 2-multicategory structure arising from its finite products, that preserve composition strictly but the $\sigma^*$ actions only up to coherent isomorphism.
The latter condition is necessary to get from a (strict) commutative monoid object in $\M$ to a non-strict symmetric monoidal category in the $T_\M$-algebras.
Similarly, we can check that a cartesian monoid in $\M$ yields a cartesian monoidal category, and so on.

A bit more explicitly, $T_\M$ is defined by starting with
\begin{equation*}
  \left(\coprod_{p_1,\dots,p_n} \left(\M(p_1,\dots,p_n;q) \times \prod_{i=1}^n X_{p_i} \right)\right)
\end{equation*}
and adding natural isomorphisms
\[ (f\sigma^*,(x_1,\dots,x_n))\cong (f,(x_{\sigma 1},\dots,x_{\sigma m})) \]
satisfying some natural coherence axioms.

We extend $T_\M$ to act on profunctors in the usual way.
I then claim that \emph{virtual $T_\M$-algebras} (the sort of ``generalized multicategory'' corresponding to $T_\M$) are equivalent to \emph{local discrete fibrations} over $\M$.
By Theorem~8.6 of Cruttwell--Shulman, a virtual $T_\M$-algebras can be described by a discrete family of \emph{sets} $D\in \mathit{Set}^{\M_0}$ together with a family of profunctors $\D_q:(T_\M D)_q^{\mathrm{op}} \times D_q \to \mathit{Set}$, with composition and unit structure.

The sets $D_p$ will be the objects of $\D$ with $\pi A = p$.
If we ignore the extra isomorphisms in $T_\M D$ for a moment, we see that the profunctor $\D$ is determined by functors
\[ \M(p_1,\dots,p_n;q)^{\mathrm{op}} \times D_{p_1}\times\cdots\times D_{p_n} \times D_q \to \mathit{Set} \]
for all $p_1,\dots, p_n$.
In other words, for each $p_1,\dots, p_n,q$ we have a family of hom-sets $\D_\alpha(A_1,\dots,A_n;B)$ indexed by objects $A_i$ and $B$ with $\pi A_i = p_i$ and $\pi B = q$ and also by a mode morphism $\alpha:(p_1,\dots,p_n)\to q$, and these sets depend contravariantly functorially on $\alpha$, just as required by the local discrete fibration condition.
The composition and identities of a generalized multicategory give these hom-sets multicategorical structure.
Finally, the extra isomorphisms in the definition of $T_\M D$ yield the bijective $\sigma^*$ actions on $\D$.

It is now straightforward to check that a virtual $T_\M$-algebra is representable, i.e.\ is a $T_\M$-algebra, exactly when its corresponding local discrete fibration is an opfibration (has $\mu$-products for all $\mu$).
Thus, the present type theory (assuming it works) provides some fairly general evidence for the correspondence between type theories and generalized multicategories.
Of course, not every 2-monad on a power of $\mathit{Cat}$ can be obtained in this way from a cartesian 2-multicategory, and not even every finitary one can (consider the 2-monad for categories with equalizers).
However, many important ones can, and this seems to be particularly true of those that extend ``canonically'' to profunctors.


\end{document}
\documentclass{amsart}
\usepackage{mathpartir,latexsym,amssymb,stmaryrd,mathtools}
\usepackage[all]{xy}
\title{Something involving cartesian cluboids and generalized substructural type theories}
\author{Daniel R. Licata \and Michael Shulman}
\thanks{
This material is based on research sponsored by The United States Air
Force Research Laboratory under agreement number FA9550-15-1-0053. The
U.S. Government is authorized to reproduce and distribute reprints for
Governmental purposes notwithstanding any copyright notation thereon.
The views and conclusions contained herein are those of the authors and
should not be interpreted as necessarily representing the official
policies or endorsements, either expressed or implied, of the United
States Air Force Research Laboratory, the U.S. Government, or Carnegie
Mellon University.
}
\newtheorem{thm}{Theorem}[section]
\theoremstyle{definition}
\newtheorem{defn}[thm]{Definition}
\def\M{\mathcal{M}}
\def\C{\mathcal{C}}
\def\D{\mathcal{D}}
\def\toiso{\xrightarrow{\sim}}
\let\To\Rightarrow
\let\types\vdash
\let\ttypes\Vdash
\newcommand{\forget}[1]{|#1|}
%\newcommand\compo[2]{\ensuremath{#1 \circ #2}}
\newcommand\compv[2]{\ensuremath{#1 \cdot #2}}
\newcommand\comph[2]{\ensuremath{#1 \mathbin{\circ_2} #2}}
\newcommand\wftp[2]{\ensuremath{#1 \,\,\, \mathsf{type}_{#2}}}
\newcommand\seq[3]{\ensuremath{#1 \, [ #2 ] \, \vdash \, #3}}
\begin{document}
\maketitle

\section{Cartesian cluboids}
\label{sec:cluboids}

Let $S$ denote the free cartesian strict monoidal category monad on $\mathbf{Cat}$.
Explicitly, the objects of $S\C$ are finite lists of objects of $\C$, and a morphism $(A_1,\dots,A_n) \to (B_1,\dots,B_m)$ consists of a function $\sigma:\{1,\dots,m\} \to \{1,\dots,n\}$ together with a tuple of functions $(f_1,\dots,f_m)$ where $f_i : A_{\sigma i}\to B_i$.
This $S$ is a cartesian monad (it preserves pullbacks, and its unit and multiplication naturality squares are pullbacks), so we can make the following definition.

\begin{defn}
  A \textbf{cartesian cluboid} is an $S$-multicategory, in the sense of Burroni--Leinster, with a discrete set of objects.
\end{defn}

Explicitly, a cartesian cluboid has a set $\M_0$ of objects and a category of morphisms $\M_1$ that fits into a span
\[ \M_0 \xleftarrow{\mathrm{cod}} \M_1 \xrightarrow{\mathrm{dom}} S\M_0. \]
Thus, each morphism (object of $\M_1$) is assigned a finite list of objects as a domain and a single object as a codomain.
Each 2-cell (morphism of $\M_1$) has a morphism $\alpha$ as a domain and another morphism $\beta$ as codomain, where the codomains of $\alpha$ and $\beta$ must be the same, but rather than their domains being the same there is a specified morphism from one to the other in $S\C$.
Explicitly, this means that if $\alpha:(A_1,\dots,A_n) \to C$ and $\beta:(B_1,\dots,B_m)\to C$, there is a specified function $\sigma:\{1,\dots,m\} \to \{1,\dots,n\}$ such that $A_{\sigma i} = B_i$ (since $\M_0$ is a discrete set).

In addition to this basic data, there are identities and composition.
The identities are assigned by a map of spans
\[ \xymatrix{ & \M_0 \ar@{=}[dl] \ar[dr]^{\eta} \ar[dd] \\ \M_0 && S\M_0 \\ & \M_1 \ar[ul] \ar[ur] } \]
where $\eta:\C \to S\C$ is the unit of the monad $S$, sending each object $A$ to the 1-element list $(A)$.
Thus, each object $A$ of $\M$ has an identity morphism $1_A : (A) \to A$.

Composition is a map $\M_1 \times_{S\M_0} S\M_1 \to \M_1$, where the domain is the pullback
\[ \xymatrix { && \M_1 \times_{S\M_0} S\M_1 \ar[dl]_p \ar[dr]^q\\
  & \M_1 \ar[dl]_{\mathrm{cod}} \ar[dr]^{\mathrm{dom}} &&
  S\M_1  \ar[dl]_{S\mathrm{cod}} \ar[dr]^{S\mathrm{dom}} \\
  \M_0 && S\M_0 && SS\M_0} \]
and the following diagrams commute:
\[ \xymatrix@C=3pc{ & \M_1 \times_{S\M_0} S\M_1 \ar[dl]_{\mathrm{cod} \circ p} \ar[r]^-{S\mathrm{dom} \circ q} \ar[dd]&
  SS\M_0 \ar[dd]^{\mathrm{mult}} \\
  \M_0 \\
  & \M_1 \ar[ul]^{\mathrm{cod}} \ar[r]_-{\mathrm{dom}} & S \M_0} \]
Explicitly, this means that morphisms compose as in a multicategory, while 2-cells compose in a somewhat more complicated way.
Associativity is straightforward to write down.

The name ``cluboid'' is a coinage based on Kelly's ``club'' that refers to the case when $\M_0=1$, based on the convention used occasionally that ``-oid'' denotes a ``horizontal categorification'' such as group $\leadsto$ groupoid or algebra $\leadsto$ algebroid.

A cartesian cluboid encodes the structure of the modes.
We will generally be interested in cartesian cluboids given by \emph{presentations} consisting of some objects (``modes''), some generating 1-morphisms, some equalities and some 2-cells between composites of those, and some equalities between composites of the generating 2-cells.
For example:
\begin{itemize}
\item Suppose we have one mode $p$, generating morphisms $\mu:(p,p)\to p$ and $\eta:()\to p$, and equations $\mu\circ (1,\eta) = 1 = \mu\circ (\eta,1)$ and $\mu\circ (1,\mu) = \mu \circ (\mu,1)$.
  This is the free cartesian cluboid containing a monoid, and the type theory it generates is noncommutative intuitionistic linear logic.
\item Suppose we add a 2-cell isomorphism $\mu \cong \mu$ living over the switch isomorphism $\sigma:(p,p) \cong (p,p)$ in $S1$.
  This is the free cartesian cluboid containing a symmetric monoid, and the type theory it generates is (commutative) intuitionistic linear logic.
\item Suppose we also add a generating 2-cell $t:1_p \To \eta$ living over the unique map $(p) \to ()$ in $S1$, and also an axiom $\comph{t}{1_\eta} = 1_\eta$.
  This is the free cartesian cluboid containing a symmetric ``semicartesian'' monoid (whose unit is a ``terminal object''), and the type theory it generates is intuitionistic affine logic.
\item Suppose we also add a generating 2-cell $d:1_p \to \mu$, living over the diagonal map $(p) \to (p,p)$ in $S1$, and also some axioms.
  Then we get the free cartesian cluboid containing a cartesian object, and the type theory it generates is ordinary intuitionistic logic.
\item Starting over, suppose the same mode $p$ has both a symmetric monoid structure and an unrelated cartesian monoid structure.
  This should generate Bunched Implication.
\item With two modes $p,q$, where $p$ is a cartesian monoid and $q$ a symmetric monoid, and a morphism $p\to q$ respecting the monoid structures, we should get the adjoint decomposition of the linear exponential.
\item In a similar case but where both $p$ and $q$ are cartesian, we should get the adjoint decomposition of Pfenning--Davies $\Box$.
\item With one mode $p$ having a cartesian monoid structure, and an idempotent monad on it respecting the monoid structure, we should get a simply-typed multi-variable version of spatial type theory.
\end{itemize}

Now, as a matter of fact, type theory itself excels at presenting categorical structures by generators and relations.


\section{Type theory of cartesian cluboids}
\label{sec:tt-cluboid}

To describe the modes and their morphisms as a cartesian cluboid, we consider a type theory with two term judgments, $\Gamma\types M:p$ and $\Gamma\ttypes M:p$.
The first is a cartesian type theory; the second is an ordered linear type theory.
There are no type operations.
The identity/variable rules are
\begin{mathpar}
  \inferrule{(x:p) \in \Gamma}{\Gamma\types x:p}\and
  \inferrule{ }{x:p \ttypes x:p}
\end{mathpar}
For a generating morphism $f:(p_1,\dots,p_n) \to q$, the generator rules are
\begin{mathpar}
  \inferrule{\Gamma\types M_1:p_1 \\\dots\\\Gamma \types M_n:p_n}{\Gamma\types f(M_1,\dots,M_n):q}\and
  \inferrule{\Gamma_1\types M_1:p_1 \\\dots\\\Gamma_n \types M_n:p_n}{\Gamma_1,\dots,\Gamma_n\ttypes f(M_1,\dots,M_n):q}
\end{mathpar}
Exchange, contraction, and weakening are admissible for $\types$, for the usual reasons.
We should also have the following admissible rules:
\begin{mathpar}
  \inferrule{\Gamma\ttypes M:p}{\Gamma\types M:p}\and
  \inferrule{\Gamma\types M:p}{\Gamma\types \theta:\Delta \\ \Delta \ttypes N:p \\ M=N[\theta]}
\end{mathpar}
where $\theta$ is a variable-for-variable substitution.
We allow generating equalities between ordered-linear terms only, although the equalities they generate apply also to nonlinear terms.
That is, we allow ourselves to assert as an axiom something like $m(x,m(y,z)) \equiv m(m(x,y),z)$, which implies that also $m(x,m(x,x)) \equiv m(m(x,x),x)$, but we don't allow ourselves to assert as an axiom something like $m(x,x) \equiv x$.

Now we have a 2-cell judgment
\[ \Gamma \ttypes e : M \To N : p \]
which implies as preconditions that $\Gamma\ttypes M:p$ and $\Gamma\types N:p$.
That is, the domain of a 2-cell is always an ordered-linear term, whereas the codomain can be nonlinear.
The rules (maybe some of them are admissible) include:
\begin{itemize}
\item Generating 2-cells, as before with ordered-linear domain.
\item ``Whiskering'' of 2-cells on both sides by ordered-linear terms:
  \begin{mathpar}
    \inferrule{\Gamma\ttypes M:p \\ \Delta,x:p,\Psi \ttypes e:N\To P:q}{\Delta,\Gamma,\Psi \ttypes e[M/x] : N[M/x] \To P[M/x] : q}\and
    \inferrule{\Gamma\ttypes e:M\To N:p \\ \Delta,x:p,\Psi\ttypes P:q}{\Delta,\Gamma,\Psi \ttypes ``P[e/x]'' : P[M/x] \To P[N/x] : q}
  \end{mathpar}
  The term $``P[e/x]''$ is in quotes because I'm not sure what the syntax for it should be.
\item ``Vertical'' composition of 2-cells:
  \begin{mathpar}
    \inferrule{\Gamma\types \theta:\Delta \\ \Gamma\ttypes e:M\To N[\theta]:p \\ \Delta\ttypes f:N\To P:p}{\Gamma\ttypes (f\cdot_\theta e) : M \To P[\theta] : p}
  \end{mathpar}
\item Identity 2-cells:
  \begin{mathpar}
    \inferrule{\Gamma\ttypes M:p} {\Gamma\ttypes 1_M : M \To M :p}
  \end{mathpar}
\end{itemize}
I think that this sort of vertical composition and whiskering ought to be enough to generate all the cluboid compositions.
Some more thought will be required to write down equations on these compositions that are equivalent to the cluboid associativity.
We also want generating equations between composites of 2-cells.


\section{Type theory over cartesian cluboids}
\label{sec:tt-club}

The type theory in a given mode theory represents semantically the following notion.

\begin{defn}
  Let $\M$ be a cartesian cluboid.
  A \textbf{local discrete fibration} over $\M$ is a cartesian cluboid $\D$ together with a map $\pi:\D\to\M$ of cartesian cluboids such that the induced functor
  \[ \D_1 \to S\D_0 \times_{S\M_0} \M_1 \]
  is a discrete fibration.
\end{defn}

Explicitly, this means that given any 2-cell $e:\alpha\To\beta$ in $\M$, and a lift of $\beta$ to a morphism $g$ in $\D$, and a lift of \emph{the domain of} $\alpha$ to a list $\vec A$ of objects of $\D$, there is a unique lift of $e$ itself to a 2-cell $f\To g$ where $f$ has domain $\vec A$.

The corresponding type theory has a judgment
\[\wftp A p\]
for types in mode $p$ (where $p$ is a mode, i.e.\ a type in the type theory of cartesian cluboids from \S\ref{sec:tt-cluboid}).
If $\wftp A p$ we write $\forget A = p$, and similarly for contexts.
Now there is also a judgment
\[ \seq{\Gamma}{\alpha}{A} \]
where $\forget\Gamma\ttypes \alpha:\forget A$ in the ordered-linear part of the mode type theory.
The crucial rule satisfied by this judgment is the following representation of the ``local discrete fibration'' condition:
\begin{mathpar}
  \inferrule{\Gamma\;\mathsf{ctx} \\ \seq{\Delta}{\beta}{A} \\ \forget\Gamma \types \theta:\forget\Delta \\ \forget\Delta \ttypes e:\alpha \To \beta[\theta]}{\seq\Gamma\alpha A}
\end{mathpar}
This can perhaps be built into enough rules to become admissible in general.
Note that its variance is reversed from the one-variable adjoint logic paper.
Categorically, the reason for this is that when representing a fully \emph{covariant} 2-functor $\M\to\mathit{Cat}$ by its Grothendieck construction, the latter is nevertheless an opfibration on 1-cells but a \emph{fibration} on 2-cells.)

The type operations correspond to further fibrational conditions on $\pi$.
I'm having trouble formulating the 2-cell conditions in a nice categorical way though.

\begin{defn}
  If $\pi:\D\to\M$ is a local discrete fibration, then a morphism $\xi:(A_1,\dots,A_n)\to B$ in $\D$ is \textbf{opcartesian} if
  \begin{enumerate}
  \item Given $\phi:(\vec C,\vec A,\vec D)\to E$ in $\D$ and $\beta : (\vec \pi C,\pi B, \vec\pi D)\to \pi E$ in $\M$ such that $\beta \circ (\vec 1,\pi \xi,\vec 1) = \pi\phi$, there is a unique $\zeta:(\vec C, B,\vec D) \to E$ in $\D$ such that $\pi\zeta = \beta$ and $\zeta\circ(\vec 1,\xi,\vec 1) = \phi$.
  \item (Some condition on 2-cells)
  \end{enumerate}
  Dually, a morphism $\xi:(\vec C,B,\vec D)\to E$ in $\D$ is \textbf{cartesian at $B$} if
  \begin{enumerate}
  \item Given $\phi:(\vec C,\vec A,\vec D)\to E$ in $\D$ and $\alpha : \vec \pi A\to \pi B$ in $\M$ such that $\pi \xi \circ (\vec 1,\alpha,\vec 1) = \pi \phi$, there exists a unique $\zeta:\vec A \to B$ in $\D$ such that $\pi \zeta = \alpha$ and $\xi\circ (\vec 1,\zeta,\vec 1) = \phi$.
  \item (Some condition on 2-cells)
  \end{enumerate}
  Given $\mu:(p_1,\dots,p_n) \to q$ in $\M$, we say that $\pi$ \textbf{has $\mu$-products} if for any $A_i$ with $\pi A_i = p_i$, there exists a $B$ with $\pi B = q$ and an opcartesian morphism in $\D_\mu(A_1,\dots,A_n;B)$.
  Dually, we say $\pi$ \textbf{has $\mu$-homs} if for any $i$, any $B$ with $\pi B = q$, and any $A_j$ with $\pi A_j = p_j$ for $j\neq i$, there exists an $A_i$ with $\pi A_i = p_i$ and a cartesian morphism in $\D_\mu(A_1,\dots,A_n;B)$.
\end{defn}

References: Hermida, H\"{o}rmann.

For example:
\begin{itemize}
\item The intuitionistic linear logic $A \otimes B$ is a $\mu$-product for the multiplication $\mu:(p,p)\to p$ of a (symmetric) monoid mode.
  Its adjoint $A\multimap B$ is a $\mu$-hom for the same $\mu$.
  Similarly, the linear unit $\mathbf{1}$ is an $\eta$-product for the unit $\eta :()\to p$.
\item The cartesian product $A\times B$ is a $\mu$-product for the multiplication of a cartesian monoid mode, and its right adjoint $A\to B$ is similarly a $\mu$-hom.
\item If the linear logic $!$ is written as $F U$, then $F$ and $U$ are the $\alpha$-product and $\alpha$-hom for the (unary!)\ mode morphism $p\to q$ from the cartesian to the linear mode.
  The situation with Pfenning--Davies $\Box$ is similar.
\item The spatial type theory $\flat$ is an $\alpha$-product where $\alpha$ is an idempotent monad, and similarly $\sharp$ is an $\alpha$-hom for the same $\alpha$.
\end{itemize}


\section{Generalized multicategories}
\label{sec:genmulti}

Let $\M$ be a cartesian cluboid, with object set $\M_0$.
Then, like any Burroni--Leinster generalized multicategory, it induces a monad $S_\M$ on $\mathbf{Cat}/{\M_0}$ defined by
\[ S_\M(X\to \M_0) = SX \times_{S\M_0} \M_1 \]
where the pullback is over $\mathrm{dom}:\M_1 \to S\M_0$, and $S_\M(X)$  is an object of $\mathbf{Cat}/{\M_0}$ via $\mathrm{cod}:\M_1 \to \M_0$.
In fact, $S_\M$ extends to the double category (or ``equipment'') $\mathbb{C}\mathsf{at}/{\M_0}$ of categories and profunctors over $\M_0$, in a straightforward way.

One can then verify, via the equivalence between presheaves and discrete fibrations, that local discrete fibrations over $\M$ are precisely generalized multicategories relative to this $S_\M$ on $\mathbb{C}\mathsf{at}/{\M_0}$ with a discrete set of objects (or, equivalently, by Cruttwell--Shulman, a normalized category of objects).
In Cruttwell--Shulman this is called a \emph{virtual $S_\M$-algebra}.

It is now straightforward to check that a virtual $S_\M$-algebra is representable, i.e.\ is a pseudo $S_\M$-algebra, exactly when its corresponding local discrete fibration is an opfibration (has $\mu$-products for all $\mu$).
Thus, the present type theory (assuming it works) provides some fairly general evidence for the correspondence between type theories and generalized multicategories.
Of course, not every monad on an equipment of this sort can be obtained in this way from a cartesian cluboid, but many important ones can.


\end{document}

\section{Introduction}

In ordinary intuitionistic logic or $\lambda$-calculus, assumptions or
variables can go unused (weakening), be used in any order (exchange), be
used more than once (contraction), and be used in any position in a
term.  \emph{Substructural} logics~\citep{?}, such as linear logic,
ordered logic, relevant logic, and affine logic, drop some of these
structural properties of weakening, exchange, and contraction, while
\emph{modal logics}~\citep{?} place restrictions on where variables may
be used---e.g. a formula $\Bx{} C$ can only be proved using assumptions
of $\Bx{} A$.  Substructural and modal logics have many applications to
programming: FIXME \citep{?}.  Substructural and modal logics also have
applications as internal languages of categories, where one uses an
appropriate logical language to do constructions ``inside'' a particular
mathematical setting, which often leads to shorter statements than
working ``externally''.  For example, to define a function when working
``externally'' in domains, one must first define the underlying
set-theoretic function, and then prove that it is continuous.  But when
using untyped $\lambda$-calculus as an internal language of domains,
there is no need to prove that a function described by a $\lambda$-term
is continuous, because all terms are shown to denote continous functions
once and for all.  Substructural logics extend this idea to various
forms of monoidal categories, while modal logics describe monads and
comonads.  Recently,
\citet{schreibershulman12cohesive,shulman15realcohesion} proposed using
modal operators in type theory to add a notion of \emph{cohesion} to
homotopy type theory~\citep{uf13hott-book,voevodsky06homotopy}.  Without
going into the precise details of the application, the general idea is
to add a triple $\sh{} \la \Flat{} \la \Sharp{}$ of type operators,
where for example $\Sharp A$ is a monad (like a modal possibility
$\diamond$ or $\bigcirc$), $\Flat A$ is a comonad (like a modal
necessity $\Box$), and there is an adjunction structure between them
(e.g. $\flat{A} \to B$ is the same as $A \to \Sharp{B}$).  This raised
the question of how to best add modal operators having these properties
to type theory.  Because other similar applications would have different
monads and comonads with different properties, we began to design a
framework for specifying such modalities, to assist in the design of new
type theories.  In previous work~\citep{ls16adjoint}, we considered the
special case of a single-assumption logic, building most directly on the
adjoint logics of \citet{bentonwadler96adjoint,reed09adjoint}.

Here we extend this previous work to the multi-assumption case.  The
resulting framework is quite general and covers many existing
intuitionistic substructural and modal connectives: cartesian, linear,
affine, relevant, ordered, and non-associative products and
implications; bunched products and implications~\citep{ohearnpymXXBI};
$n$-linear variables~\citep{jcreedXXnames}; the comonadic $\Box$ and
linear exponential $!$ and subexponentials; monadic $\diamond$ and
$\bigcirc$ modalities; and adjoint logic $F$ and
$G$~\citep{bentonwadler96adjoint,reed09adjoint}, including the
single-assumption 2-categorical version from our previous
work~\citep{ls16adjoint}.  A single structural~\citep{pfenning94cut}
proof of cut (and identity) admissibility applies to all of these
logics, as well as any new logics that can be described in the
framework.  While it is not too surprising that this is possible, given
the sense that cut proofs for these logics all follow a similar
template, it is nonetheless satisfying to codify this pattern as an
abstraction.

At a high level, the framework expresses the idea that all of the above
logics are a restriction on ordinary structural intuitionistic logic.
The first layer of the logic is a simple type theory for what we will
call \emph{modes} and \emph{context descriptors}.  The modes are just a
collection of base types, which we write as $p,q,r$, while a context
descriptor is a term built from variables and constants.  The next layer
is the main logic.  Each proposition of the logic is assigned a mode,
and the basic sequent is \seq{x_1 : A_1, \ldots, x_n : A_n}{\alpha}{C},
where if $A_i$ as mode $p_i$, and $C$ has mode $q$, then $\oftp{x_1 :
  p_1,\ldots, x_n : p_n}{\alpha}{q}$.  In a sequent
\seq{\Gamma}{\alpha}{A}, the idea is that $\Gamma$ binds some variables
for use both in $\alpha$ to be used as assumptions in the derivation,
and \emph{$\Gamma$ itself behaves like an ordinary structural/cartesian
  context}.  The substructural and modal aspects are enforced by the
\emph{term} $\alpha$, which describes how the resources from $\Gamma$
are allowed to be used.  For example, in linear logic/ordered logic/BI,
the context is usually taken to be a multiset/list/tree (respectively).
Here, we represent the multiset or list or tree using a pair of an
ordinary structural context, together with a term that describes the
multiset or list or tree structure, labeled with variables from the
ordinary context at the leaves.  We pronounce a sequent
\seq{\Gamma}{\alpha}{A} as ``$\Gamma$ proves $A$ along $\alpha$'' or
(anticipating the semantics) ``$\Gamma$ proves $A$ over $\alpha$''.

For example, suppose we have one mode $\dsd{n}$, together with a mode
morphism constant
\[
x : \dsd{n}, y:\dsd{n} \vdash x \odot y : \dsd{n}
\]
Then an example sequent
\[
\seq{x:A, y:B, z:C, w:D}{(y \odot x) \odot z}{E}
\]
should be read as saying that we must prove $E$ using the resources $y$
and $x$ and $z$ (but not $w$) according to the particular tree structure
${(y \odot x) \odot z}$.  If we say nothing else, the framework will
treat $\odot$ as describing a non-associative, linear, ordered context:
if we have a product-like type $A \odot B$ internalizing this context
operation,\footnote{We sometimes overload binary operations to refer
  both to mode constants (when applied to $x,y,z$) and to propositional
  connectives (when applied to $A,B,C$)} then we will \emph{not} be able
to prove associativity ($(A \odot B) \odot C \dashv\vdash A \odot (B
\odot C)$ or contraction ($A \vdash A \odot A$) or exchange ($A \odot B
\vdash B \odot A$) etc.

To get from this basic structure to linear or affine or relevant or
cartesian logic, we need to add some structural properties to the
context descriptor term $\alpha$.  We analyze structural properties as
\emph{equations}, or more generally \emph{inequalities}, on such terms.
For example, to specify linear logic, we will add a unit element $1 :
\dsd{n}$ together with equations making $(\odot,1)$ into a commutative
monoid:
\[
\begin{array}{c}
x \odot (y \odot z) = (x \odot y) \odot z\\
x \odot 1 = x = 1 \odot x\\
x \odot y = y \odot x
\end{array}
\]
so that the context descriptors ignore associativity and order.  To get
BI, we add an additional commutative monoid $(\times,\top)$ (with
weakening and contraction, as discussed below), so that a BI context
tree $(x:A,y:B);(z:C,w:D)$ can be represented by the ordinary context
$x:A,y:B,z:C,w:D$ with the context descriptor $(x \odot y) \times (z
\odot w)$.  Because the context descriptors are themselves ordinary
structural/cartesian terms, the same variable can occur more than once
or not at all.  A descriptor such as $x \odot x$ captures the idea that
we can use the \emph{same} variable $x$ twice, expressing the $n$-linear
types of \citet{reed}.  Thus, we can express contraction for a
particular context descriptor $\odot$ as an equation $x = x \odot x$
(one use of $x$ is the same as two, or $\odot$ is an idempotent binary
operation).  However, weakening cannot be represented as an equation
between context descriptors---an equation $x = 1$ would trivialize the
logic to ordinary intuitionistic logic.  Instead, to express weakening,
we use an \emph{inequality} $x \spr 1$ which is oriented to allow
throwing away an allowed use of $x$, but not creating an allowed use
from nothing.  Oriented structural properties are also used to describe
relationships between adjunctions as in \citep{ls16adjoint}, which
includes the preordered subexponentials in linear
logic~\citep{damos,nigman}.

In summary, to specify a particular substructural or modal logic, one
gives constants generating context descriptors $\alpha$, with equations
$\alpha = \beta$ and inequalities $\alpha \spr \beta$ expressing
structural properties.  The main sequent $\seq{\Gamma}{\alpha}{A}$
respects the specified structural properties in the sense that when
$\alpha = \beta$, we regard $\seq{\Gamma}{\alpha}{A}$ and
$\seq{\Gamma}{\beta}{A}$ as the same sequent, while when $\alpha \spr
\beta$, there will be an operation that takes a derivation of
\seq{\Gamma}{\beta}{A} to a derivation of \seq{\Gamma}{\alpha}{A}.  

In addition to respect for these object-logic structural properties
(like $x \odot y = y \odot x$ for an unordered logic), a guiding
principle of the framework is a meta-level notion of \emph{structurality
  over structurality}.  For example, we always have \emph{weakening over
  weakening}: if \seq{\Gamma}{\alpha}{A} then
\seq{\Gamma,y:B}{\alpha}{A}, where $\alpha$ itself is weakened with $y$.
This does not prevent encodings of e.g. linear logic: it is permissible
to move from a derivation of \seq{\Gamma}{x_1 \odot \ldots \odot x_n}{A}
(``use $x_1$ through $x_n$) to a derivation of \seq{\Gamma,y:B}{x_1
  \odot \ldots \odot x_n}{A} because the (weakened) context descriptor
still disallows the use of $y$.  Similarly, we always have exchange over
exchange and contraction over contraction.  The identity and and cut
principles are analogous:
\[
\infer{\seq{\Gamma,x:A}{x}{A}}{}
\qquad
\infer{\seq{\Gamma}{\subst{\beta}{\alpha}{x}}{B}}
    {\seq{\Gamma,x:A}{\beta}{B} &
     \seq{\Gamma}{\alpha}{A}}
\]
The identity principle says that we should be able to prove $A$ using
exactly an assumption $x:A$---identity over identity.  The cut principle
says that the context descriptor for the result of the cut is the
substitution of the context descriptor used to prove $A$ into the one
used to prove $B$.  For example, together with weakening-over-weakening,
this captures the usual cut principle of linear logic, which says that
cutting $\Gamma,x:A \vdash B$ and $\Delta \vdash A$ yields
$\Gamma,\Delta \vdash B$.  If $\Gamma$ binds $x_1,\ldots,x_n$ and
$\Delta$ binds $y_1,\ldots,y_n$, then we will represent the two
derivations to be cut together by sequents with
\[
\begin{array}{l}
\beta = x_1 \odot \ldots \odot x_n \odot x\\
\alpha = y_1 \odot \ldots \odot y_n
\end{array}
\]
so
\[
\beta[\alpha/x] = x_1 \odot \ldots \odot x_n \odot y_1 \odot \ldots \odot y_n
\]
deletes $x$ and replaces it with the variables from $\Delta$.  Moreover,
in more subtle situations such as BI, the substitution will insert the
resources used to prove the cut formula in the correct place in the tree.  

The framework has two main logical connectives.  The first,
\F{\alpha}{\Delta}, generalizes the \dsd{F} of adjoint
logic~\citep{bentonwadler96adjoint,reed09adjoint} and the tensor
($\otimes$) of linear logic.  The second, \U{x.\alpha}{\Delta}{A},
generalizes the $\dsd{G}/\dsd{U}$ of adjoint logic and the implication
$A \lolli B$ of linear logic.  Here $\Delta$ is a context of assumptions
$x_i:A_i$, and trivializing the context descriptors (i.e. adding an
equation $\alpha = \beta$ for all $\alpha$ and $\beta$) degenerates
$\F{\alpha}{\Delta}$ into the ordinary intuititionistic product $A_1
\times \ldots \times A_n$, while \U{x.\alpha}{\Delta}{A} becomes $A_1
\to \ldots \to A_n \to A$.  Though we do not give a full
polarized/focused proof theory in this paper, we do prove that \dsd{F}
is left-invertible and \dsd{U} is right-invertible, and conjecture that
focusing works with the polarization that one would expect based on
these degeneracies ($\F{\alpha}{\Delta^{\mathord{+}}}^{\mathord{+}}$ and
$\U{x.\alpha}{\Delta^{\mathord{+}}}{A^{\mathord{-}}}^{\mathord{-}}$).
In linear logic terms, these two connectives cover the multiplicatives
and exponentials, which are the two kinds of connectives that manipulate
the context; additives can be added separately by essentially the usual
rules.

  FIXME: comparison with display logic

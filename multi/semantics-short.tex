\newcommand\cD{\ensuremath{\mathcal{D}}}

\section{Categorical Semantics}
\label{sec:semantics}

In this section, we give a category-theoretic structure corresponding to
the above syntax.  First, we define a cartesian 2-multicategory as a
semantic analogue of the syntax in Figure~\ref{fig:2multicategory}. The
semantics uses total substitutions (for the entire context at once)
instead of single-variable substitutions, and explicit weakening and
exchange instead of named variables.

\begin{definition}
  A \textbf{(strict) cartesian 2-multicategory} consists of
  \begin{enumerate}
  \item A set $\M_0$ of \emph{objects}.
  \item For every object $B$ and every finite list of objects $(A_1,\dots,A_n)$, a category $\M(A_1,\dots,A_n;B)$.
    The objects of this category are \emph{1-morphisms} and its morphisms are \emph{2-morphisms}; we write composition of 2-morphisms as $\compv{s_1}{s_2}$.
  \item For each object $A$, an identity arrow $1_A\in\M(A;A)$.
  \item For any object $C$ and lists of objects $(B_1,\dots,B_m)$ and $(A_{i1},\dots,A_{in_i})$ for $1\le i\le m$, a composition functor
    \begin{footnotesize}
    \begin{align*}
      \M(B_1,\dots,B_m;C) \times \prod_{i=1}^m \M(A_{i1},\dots,A_{in_i};B_i) &\longrightarrow \M(A_{11},\dots,A_{mn_m};C)\\
      (g,(f_1,\dots,f_m)) &\mapsto g\circ (f_1,\dots,f_m)
    \end{align*}
    \end{footnotesize}
    %% We write the action of this functor on 2-cells as $\comph{d}{(e_1,\dots,e_m)}$.
  %% \item For any $f\in\M(A_1,\dots,A_n;B)$ we have natural equalities (i.e.\ natural transformations whose components are identities)
  %%   \[
  %%     1_B \circ (f) = f \qquad
  %%     f\circ (1_{A_1},\dots,1_{A_n}) = f.
  %%     \]
  %% \item For any $h,g_i,f_{ij}$ we have natural equalities
  %%   \begin{multline*}
  %%     (h\circ (g_1,\dots,g_m))\circ (f_{11},\dots,f_{mn_m}) =\\
  %%     h \circ (g_1\circ (f_{11},\dots,f_{1n_1}), \dots, g_m \circ (f_{m1},\dots,f_{mn_m}))
  %%   \end{multline*}
  \item For any function $\sigma : \{1,\dots,m\} \to \{1,\dots,n\}$ and
    objects $A_1,\dots,A_n,B$, a \emph{renaming} functor
    \begin{align*}
      \M(A_{\sigma 1},\dots,A_{\sigma m}; B) &\to \M(A_1,\dots,A_n;B)\\
      f &\mapsto f\sigma^*
    \end{align*}
  %% \item The functors $\sigma^*$ satisfy the following natural equalities:
  %%   \begin{gather*}
  %%     f \sigma^* \tau^* = f(\tau\sigma)^*\\
  %%     f (1_n)^* = f\\
  %%     g\circ (f_1 \sigma_1^* ,\dots, f_n \sigma_n^*) = (g \circ (f_1,\dots,f_n))(\sigma_1\sqcup \cdots \sqcup \sigma_n)^*\\
  %%     g\sigma^* \circ (f_1,\dots,f_n) = (g\circ (f_{\sigma 1},\dots, f_{\sigma m}))(\sigma \wr (k_1,\dots,k_n))^*
  %%   \end{gather*}
  %%   In the last equation, $k_i$ is the arity of $f_i$, and $\sigma \wr (k_1,\dots,k_n)$ denotes the composite function
  %%   \begin{footnotesize}
  %%   \begin{equation*}
  %%     \{1,\dots,\textstyle\sum_{i=1}^m k_{\sigma i} \}
  %%     \toiso \bigsqcup_{i=1}^m \{1,\dots,k_{\sigma i}\}
  %%     \xrightarrow{\widehat{\sigma}} \bigsqcup_{j=1}^n \{1,\dots,k_{j}\}
  %%     \toiso \{1,\dots,\textstyle\sum_{j=1}^n k_j \}
  %%   \end{equation*}
  %%   \end{footnotesize}
  %%   where $\widehat{\sigma}$ acts as the identity from the $i^{\mathrm{th}}$ summand to the $(\sigma i)^{\mathrm{th}}$ summand.
  \end{enumerate}
satisfying some equations that we elide here.  
\end{definition}

The next three definitions will be used to describe the
\seq{\Gamma}{\alpha}{A} judgement.  

\begin{definition}
  A \textbf{functor of cartesian 2-multicategories} $F:\cD\to\M$ consists
  of a function $F_0 : \cD_0 \to \M_0$ and functors $\cD(A_1,\ldots,A_n;B)
  \to \M(F_0(A_1),\ldots,F_0(A_n);F_0(B))$ such that the chosen
  identities, compositions, and renamings are preserved (strictly).
\end{definition}

\begin{definition}
  A functor of cartesian 2-multicategories $\pi:\cD\to\M$ is a \textbf{local discrete fibration} if each induced functor of ordinary categories
  $\cD(A_1,\dots,A_n;B)\to\M(\pi A_1,\dots,\pi A_n;\pi B)$
  is a discrete fibration.
\end{definition}

We write $\cD_f(A_1,\dots,A_n;B)$ for the fiber of this functor over
$f\in \M(\pi A_1,\dots,\pi A_n;\pi B)$; when $\pi$ is a local discrete
fibration, this fiber is a discrete set.

\begin{definition}
  If $\pi:\cD\to\M$ is a local discrete fibration, then a morphism $\xi\in\cD(A_1,\dots,A_n;B)$ is \textbf{opcartesian} if all diagrams of the following form are pullbacks of categories:
  \[ \xymatrix{
    \cD(\vec C,B,\vec D;E) \ar[r]^-{-\circ \xi} \ar[d]_\pi &
    \cD(\vec C,\vec A,\vec D;E) \ar[d]^\pi \\
    \M(\pi\vec C,\pi B, \pi\vec D; \pi E) \ar[r]_-{-\circ \pi\xi} &
    \M(\pi\vec C,\pi\vec A,\pi\vec D;\pi E)
  }\]
  Dually, a morphism $\xi\in\cD(\vec C,B,\vec D;E)$ is \textbf{cartesian at $B$} if all diagrams of the following form are pullbacks of categories:
  \[ \xymatrix{
    \cD(\vec A;B) \ar[r]^-{\xi\circ -} \ar[d]_\pi &
    \cD(\vec C,\vec A,\vec D;E) \ar[d]^\pi \\
    \M(\pi\vec A;\pi B) \ar[r]_-{\pi\xi\circ -} &
    \cD(\pi\vec C,\pi\vec A,\pi\vec D;\pi E)
  }\]
  Given $\mu:(p_1,\dots,p_n) \to q$ in $\M$, we say that $\pi$ \textbf{has $\mu$-products} if for any $A_i$ with $\pi A_i = p_i$, there exists a $B$ with $\pi B = q$ and an opcartesian morphism in $\cD_\mu(A_1,\dots,A_n;B)$.
  Dually, we say $\pi$ \textbf{has $\mu$-homs} if for any $i$, any $B$ with $\pi B = q$, and any $A_j$ with $\pi A_j = p_j$ for $j\neq i$, there exists an $A_i$ with $\pi A_i = p_i$ and a cartesian morphism in $\cD_\mu(A_1,\dots,A_n;B)$.

  We say that $\pi$ is a \textbf{bifibration} if it has $\mu$-products
  and $\mu$-homs for all $\mu$.
\end{definition}

We now describe how these definitions correspond to the syntax, and
sketch the beginnings of a soundness and completeness theorem.  To complete
the correspondence, we will need an equational theory $s \deq s' :
\alpha \spr \beta$ on syntactic transformations, as well as an equational
theory on derivations of \seq{\Gamma}{\alpha}{A}, that match the
semantic notions of equality of morphisms.  We have a candidate for such
an equational theory, but have not yet checked the correspondence.  

A mode theory $\Sigma$ presents a cartesian 2-multicategory, where
$\M_0$ is the set of modes, an object of $\M(p_1,\ldots,p_n;q)$ is a
term $\oftp{x_1:p_1,\ldots,x_n:p_n}{\alpha}{q}$ modulo the \deq-axioms,
and a morphism of $\M(p_1,\ldots,p_n;q)$ is a structural transformation
$s :: \wfsp{\psi}{\alpha}{\beta}{q}$ modulo some equations.  We fix a
mode theory $\Sigma$ and write $\M$ for the 2-multicategory it presents.

The overall conjecture is that the syntax is the initial bifibration
over $\M$.  The correspondence between the syntax and a bifibration $\pi
: \cD \to \M$ is as follows.  A type \wftype{A}{p} corresponds to an
element $A \in \cD_0$ with $\pi(A) = p$.  A sequent
\seq{\Gamma}{\alpha}{A} corresponds to an element $d \in \cD(\Gamma;A)$
with $\pi(d) = \alpha$.  Identity \seq{\Gamma,x:A}{x}{A} corresponds to
identity in $\cD(A;A)$ composed with a projection renaming.  Cut (part 3,
for the whole context at once) corresponds to composition in $\cD$.
Weakening, exchange, and contraction correspond to the $\sigma$-actions.
Structurality-over-structurality corresponds to functoriality and
renaming-preservation of $\pi$, which for example says that composition
in $\cD$ lives over composition in $\M$.  The absence of any notion of
morphism between derivations of \seq{\Gamma}{\alpha}{A} corresponds to
local discreteness.  The type \F{\alpha}{\Delta} is an $\alpha$-product,
while the type \U{x.\alpha}{\Delta}{A} is an $\alpha$-hom.

The first step of this correspondence is the following:

\begin{theorem}[Soundness of sequent calculus rules]
Fix a bifibration $\pi : \cD \to \M$.  Then there is a function $\llb -
\rrb$ from types \wftype{A}{p} to $\llb A \rrb \in \cD_0$ with $\pi(\llb
A \rrb) = p$ and from derivations $\seq{x:A_1, \ldots,
  x_n:A_n}{\alpha}{C}$ to morphisms $d \in \cD(\llb A_1 \rrb, \dots, \llb
A_n \rrb) \to \llb C \rrb$, such that $\pi(d) = \alpha$.
\end{theorem}

\begin{proof}
If $\pi$ is a local discrete fibration, the 2-cells of $\M$ act on the
fibers. Suppose $\alpha, \beta : \psi \to p$ and $s : \alpha \spr
\beta$. We write $s_*$ for the induced function (of sets):
$\cD_\beta(\Gamma; A) \to \cD_\alpha(\Gamma; A)$.  

The definition of an opfibration of 2-multicategories guarantees that,
given a morphism in the mode category $\alpha : \psi \to q$ and a
context $\wfctx{\Delta}{\psi}$ that lies over $\psi$, there is an
opcartesian morphism over $\alpha$ with multidomain $\Delta$. For each
$\alpha$ we choose one such lift and take the codomain of this morphism
as our interpretation of $\F{\alpha}{\Delta}$. Let us name this
opcartesian lift $\zeta_{\alpha, \Delta} : \Delta \to
\F{\alpha}{\Delta}$. $\zeta$ corresponds to the axiomatic $\FR^*$.

Similarly, the fibration structure gives us, for every morphism $\alpha
: (\psi,x) \to p$ and context $\Delta$ over $\psi$ and type $A$ over
$p$, a cartesian morphism over $\alpha$ with codomain $A$, where the
position in the domain marked by $x$ has been filled by an object over
$q$. We take this object as the interpretation of
$\U{x.\alpha}{\Delta}{A}$.  Let us name this cartesian lift
$\xi_{\alpha, \Delta,A} : \Delta,\U{x.\alpha}{\Delta}{A} \to A$; it
corresponds to the axiomatic $\UL^*$.  

We assume a given interpretation of each atomic proposition $\llb
\wftype{P}{p} \rrb$ as an object of $\cD$ that lies over $p$.

The sequent calculus rules are then interpreted as follows (we elide the
semantic brackets on objects):

\begin{itemize}
\item For the hypothesis rule
%% \[
%% \infer[\dsd{v}]{\seq{\Gamma}{\beta}{P}}
%%       {x:P \in \Gamma & s : \beta \spr x}
%% \]
we need a morphism in $\cD_\beta(\Gamma; P)$, which take to be $s_*(1_x)$.
\item For $\FL$
%% \[
%% \infer[\FL]{\seq{\Gamma,x:\F{\alpha}{\Delta},\Gamma'}{\beta}{M : C}}
%%       {\seq{\Gamma,\Gamma',\Delta}{\subst \beta {\alpha}{x}}{C}}
%% \]
the inductive hypothesis (after an exchange, which preserves the size of
the derivations) gives a morphism $\llb \D \rrb \in \cD_{\subst \beta
  {\alpha}{x}}(\Gamma, \Delta, \Gamma'; C)$ and we must produce a morphism
$\cD_{\beta}(\Gamma, \F{\alpha}{\Delta},\Gamma'; C)$. By the
opcartesian-ness of $\zeta_{\alpha, \Delta}$, the following square is a
pullback:
\[ \xymatrix{
    \cD(\Gamma,\F{\alpha}{\Delta},\Gamma';C) \ar[r]^-{-\circ \zeta_{\alpha, \Delta}} \ar[d]_S &
    \cD(\Gamma,\Delta,\Gamma';C) \ar[d]^S \\
    \M(\pi\Gamma,\pi \F{\alpha}{\Delta}, \pi\Gamma'; \pi C) \ar[r]_-{-\circ \alpha} &
    \M(\pi\Gamma,\pi\Delta,\pi\Gamma'; \pi C)
}\]
We are given an object of the bottom left ($\beta$) and the top right
($\llb \D \rrb$), with $\pi\llb \D \rrb = \beta \circ \alpha$. By the
construction of pullbacks in $Cat$, there is an object $f \in
\cD(\Gamma,\F{\alpha}{\Delta},\Gamma';C)$ so that $\pi(f) = \beta$. We take
this $f$ to be our interpretation.

\item For $\FR$
%% \[
%% \infer[\FR]{\seq{\Gamma}{\beta}{\F{\alpha}{\Delta}}}
%%       {
%%         s : \beta \spr \tsubst{\alpha}{\gamma} &
%%         \seq{\Gamma}{\gamma}{M : \Delta} 
%%       }
%% \]
where $\gamma = (\alpha_1, \dots, \alpha_n)$ and $\Delta = (C_1, \dots,
C_n)$, the premises are interpreted as a 2-cell $s : \beta \spr
{\alpha} \circ {(\alpha_1,\ldots,\alpha_n)}$ and a set of morphisms $\llb
d_i \rrb \in \cD_{\gamma_i}(\Gamma; C_i)$. We take the interpretation of
the conclusion to be $s_*(\zeta_{\alpha, \Delta} \circ (\llb d_1 \rrb,
\dots, \llb d_n \rrb))$

%% \item For $\UL$
%% \[\infer[\UL]{\seq{\Gamma}{\beta}{C}}
%%       {\begin{array}{l}
%%           x:\U{x.\alpha}{\Delta}{A} \in \Gamma \\
%%           s : \beta \spr \subst{\beta'}{\tsubst{\alpha}{\gamma}}{z} \\
%%           \seq{\Gamma}{\gamma}{M:\Delta} \\
%%           \seq{\Gamma,\tptm{z}{A}}{\beta'}{D:C}
%%        \end{array}
%%       }
%% \]
%% let us again write $\gamma = (\gamma_1, \dots, \gamma_n)$ and $\Delta = (C_1, \dots, C_n)$, so that the interpretations of the premises are $\llb M_i \rrb \in \cD_{\gamma_i}(\Gamma; C_i)$ and $\llb D \rrb \in \cD_{\beta'}(\Gamma, A; C)$. Our interpretation is then
%% $s_*(\llb D \rrb \circ_z \xi_{\alpha, \Delta,A} \circ (\llb M_1 \rrb, \dots, \llb M_n \rrb, id_x))$
%% \item For $\UR$
%% \[
%% \infer[\UR]{\seq{\Gamma}{\beta}{\U{x.\alpha}{\Delta}{A}}}
%%       {\seq{\Gamma,\Delta}{\subst{\alpha}{\beta}{x}}{M:A}}
%% \]
%% we are given a morphism $\llb M \rrb \in \cD_{\subst{\alpha}{\beta}{x}}(\Gamma, \Delta; A)$ and must produce a morphism in $\cD_{\beta}(\Gamma; \U{x.\alpha}{\Delta}{A})$. This time, by cartesian-ness of $\xi_{\alpha, \Delta,A}$, we have the pullback square
%% \[ \xymatrix{
%%     \cD(\Gamma;\U{x.\alpha}{\Delta}{A}) \ar[r]^-{\xi_{\alpha, \Delta,A}\circ -} \ar[d]_S &
%%     \cD(\Gamma,\Delta;A) \ar[d]^S \\
%%     \M(S\Gamma;S \U{x.\alpha}{\Delta}{A}) \ar[r]_-{\alpha\circ -} &
%%     \M(S\Gamma,S\Delta;S A)
%%   }\]
%% Again we have objects $\beta$ and $\llb M \rrb$ that agree in the bottom right, so an induced object in the top left which we take as our interpretation.
\end{itemize}
The cases for \UL\/ and \UR\/ are similar.  
\end{proof}

Conversely, the syntactic bifibration should be defined as
follows: an object of $\cD$ is a pair $(p,\wftype{A}{p})$, and the
hom-category has objects pairs $(\alpha,d :: \seq{\Gamma}{\alpha}{A})$
(quotiented by some equations), and morphisms $(\alpha,d) \to (\alpha',d')$
structural transformations $s :: \alpha \spr \alpha'$ such that
$\Trd{s}{d'} \deq d$ (the equational second
component of these morphisms corresponds to local discreteness).  The
functor $\pi : \cD \to \M$ is then first projection.  Identity,
composition, and renamings for $\cD$ are defined by the corresponding
syntactic lemmas, which lay over the appropriate context context
descriptor to make $\pi$ functorial.  Cartesian and opcartesian lifts
are defined to be \Fsymb{} and \Usymb{} types with $\FR^*$ and $\UL^*$.
%% What remains is to check that the equational theory
%% implies the semantic equations.

%% \begin{theorem}
%% Given a mode theory $\M$, the syntax determines a 2-multicategory $\cD$ and a bifibration $S : \cD \to \M$.
%% \end{theorem}
%% \begin{proof}



In future work, we plan to continue a preliminary investigation of
equational adequacy that is discussed in the extended version, showing
that the logical adequacy results extend to an isomorphism on
$\beta\eta$-classes of derivations.  It is generally easy to show that
object-language equations are true in the framework.  We conjecture that
the converse is true for the mode theories we have described here, which
says that the ``extra'' types and judgements available in the framework
do not add to the equations between terms in the image of encoded
sequents.  Proving this is challenging because the equational theory of
Section~\ref{sec:equational} does not itself obviously have the
subformula property---we can in principle prove equations by introducing
and then eliminating cuts.  We have sketched a proof of equational
adequacy for the simplest case (ordered logic with products only),
assuming a lemma that the equational theory from
Section~\ref{sec:equational} can be characterized as some ``permuting
conversions'' on cut-free derivations (i.e. that one can first
$\beta$-reduce and then rearrange the cut-free term)---we proved the
analogue of this lemma for the single-variable case~\citep{ls16adjoint}.

Addtionally, we plan to apply our framework to investigate more
extensions of homotopy type theory like the spatial type theory
considered here; in current work with Eric Finster, we are designing a
variant of cohesion for talking synthetically about spectra.  We also
plan to consider encodings of programming-focused type theories, such as
specialized effect calculi~\citep{gaboardi16coeffect}.  Finally, our
adequacy proofs require reasoning about the 1- and 2-cells in the mode
theory, which we have currently done entirely na\"ively; we would like
to investigate using techniques from higher-dimensional rewriting to
simplify these proofs.

\newcommand\cD{\ensuremath{\mathcal{D}}}
\newcommand\IndF[3]{\ensuremath{{#1}^\Fsymb_{{#2},{#3}}}}
\newcommand\IndU[4]{\ensuremath{{#1}^\Usymb_{{#2},{#3},{#4}}}}

\section{Categorical Semantics}
\label{sec:semantics}

In this section, we give a category-theoretic structure corresponding to
the above syntax.  First, we define a cartesian 2-multicategory as a
semantic analogue of the syntax in Figure~\ref{fig:2multicategory}. 

%The
%semantics uses total substitutions (for the entire context at once)
%instead of single-variable substitutions, and explicit weakening and
%exchange instead of named variables.

\begin{definition}
  A \textbf{(strict) cartesian 2-multicategory} consists of
  \begin{enumerate}
  \item A set $\M_0$ of \emph{objects}.
  \item For every object $B$ and every finite list of objects $(A_1,\dots,A_n)$, a category $\M(A_1,\dots,A_n;B)$.
    The objects of this category are \emph{1-morphisms} and its morphisms are \emph{2-morphisms}; we write composition of 2-morphisms as $\compv{s_1}{s_2}$.
  \item For each object $A$, an identity arrow $1_A\in\M(A;A)$.
  \item For any object $C$ and lists of objects $(B_1,\dots,B_m)$ and
    $(A_{i1},\dots,A_{in_i})$ for $1\le i\le m$, a composition functor
    $(g,(f_1,\dots,f_m)) \mapsto g\circ (f_1,\dots,f_m) : 
    \M(B_1,\dots,B_m;C) \times \prod_{i=1}^m \M(A_{i1},\dots,A_{in_i};B_i) \longrightarrow \M(A_{11},\dots,A_{mn_m};C)$.
    We write the action of this functor on 2-cells as $\comph{d}{(e_1,\dots,e_m)}$.
  \item For any function $\sigma : \{1,\dots,m\} \to \{1,\dots,n\}$ and
    objects $A_1,\dots,A_n,B$, a \emph{renaming} functor $f \mapsto
    f\sigma^* : \M(A_{\sigma 1},\dots,A_{\sigma m}; B) \to \M(A_1,\dots,A_n;B)$
  \item satisfying some equalities (see the extended version)
  \end{enumerate}
% satisfying some equations that we elide here.  
\end{definition}

The next three definitions will be used to describe the
\seq{\Gamma}{\alpha}{A} judgement.  

\begin{definition}
  A \textbf{functor of cartesian 2-multicategories} $F:\cD\to\M$
  consists of a function $F_0 : \cD_0 \to \M_0$ and functors
  $\cD(A_1,\ldots,A_n;B) \to \M(F_0(A_1),\ldots,F_0(A_n);F_0(B))$ such
  that the chosen identities, compositions, and renamings are preserved
  (strictly).  We write $\cD_\alpha(A_1,\dots,A_n;B)$ for the fiber of
  such a functor over $\alpha \in \M(\pi A_1,\dots,\pi A_n;\pi B)$.
\end{definition}

\begin{definition}
  A functor of cartesian 2-multicategories $\pi:\cD\to\M$ is a
  \textbf{local discrete fibration} if each induced functor of ordinary
  categories $\cD(A_1,\dots,A_n;B)\to\M(\pi A_1,\dots,\pi A_n;\pi B)$ is
  a discrete fibration.  When $\pi$ is a local discrete fibration, each
  fiber is a discrete set.
\end{definition}

\begin{definition}
  If $\pi:\cD\to\M$ is a local discrete fibration, then a morphism
  $\xi\in\cD(A_1,\dots,A_n;B)$ is \textbf{opcartesian} if all diagrams
  of the lefthand form are pullbacks of categories, and a morphism
  $\xi\in\cD(\vec C,B,\vec D;E)$ is \textbf{cartesian at $B$} if all
  diagrams of the right-hand form are pullbacks of categories: 
  \[ \xymatrix{
    \cD(\vec C,B,\vec D;E) \ar[r]^-{(-)\circ_B \xi} \ar[d]_\pi &
    \cD(\vec C,\vec A,\vec D;E) \ar[d]^\pi \\
    \M(\pi\vec C,\pi B, \pi\vec D; \pi E) \ar[r]_-{(-)\circ_{\pi B} \pi\xi} &
    \M(\pi\vec C,\pi\vec A,\pi\vec D;\pi E)
  }
  \qquad
  \xymatrix{
    \cD(\vec A;B) \ar[r]^-{\xi\circ_B (-)} \ar[d]_\pi &
    \cD(\vec C,\vec A,\vec D;E) \ar[d]^\pi \\
    \M(\pi\vec A;\pi B) \ar[r]_-{\pi\xi\circ_{\pi B} (-)} &
    \cD(\pi\vec C,\pi\vec A,\pi\vec D;\pi E)}
  \]
  Given $\mu:(p_1,\dots,p_n) \to q$ in $\M$, we say that $\pi$ \textbf{has $\mu$-products} if for any $A_i$ with $\pi A_i = p_i$, there exists a $B$ with $\pi B = q$ and an opcartesian morphism in $\cD_\mu(A_1,\dots,A_n;B)$.
  Dually, we say $\pi$ \textbf{has $\mu$-homs} if for any $i$, any $B$ with $\pi B = q$, and any $A_j$ with $\pi A_j = p_j$ for $j\neq i$, there exists an $A_i$ with $\pi A_i = p_i$ and a cartesian morphism in $\cD_\mu(A_1,\dots,A_n;B)$.
  We say that $\pi$ is an \textbf{opfibration} if it has $\mu$-products for all $\mu$, a \textbf{fibration} if it has $\mu$-homs for all $\mu$, and a \textbf{bifibration} if it is both an opfibration and a fibration.
\end{definition}

The proofs of the following soundness and completeness results are in
the extended version:

\begin{theorem}[Mode theory presents a mutilcategory]
\label{thm:completeness-mode-theory}
A mode theory $\Sigma$ presents a cartesian 2-multicategory $\M$, where
$\M_0$ is the set of modes, and an object of $\M(p_1,\ldots,p_n;q)$ is a
term $\oftp{x_1:p_1,\ldots,x_n:p_n}{\alpha}{q}$ and a morphism of $\M(p_1,\ldots,p_n;q)$ is a structural transformation
$s :: \wfsp{\psi}{\alpha}{\beta}{q}$, both considered modulo $\deq$.
\end{theorem}

\begin{theorem}[Completeness/Syntactic Bifibration]
For a fixed mode theory $\M$, the syntax presents a bifibration $\pi : \cD \to \M$, where:
\begin{itemize}
\item Objects of $\cD$ are pairs $(p, \wftype{A}{p})$;
\item 1-morphisms $\Gamma \to B$, i.e., objects of $\cD(\Gamma; B)$, are pairs $(\alpha, d :: \seq{\Gamma}{\alpha}{B})$ (up to \deq); 
\item 2-morphisms $(\alpha, d) \to (\alpha', d')$ are structural
  transformations $s :: \alpha \spr \alpha'$ such that $\Trd{s}{d'} \deq d$;
\item the $\mu$-products are \Fsymb-types, and the $\mu$-homs are \Usymb-types.
\end{itemize}
The functor $\pi : \cD \to \M$ is given by first projection on objects and 1-morphisms, and sends 2-morphisms to the underlying structural transformations.
\end{theorem}

\begin{theorem}[Soundness/Interpretation in any bifibration]
Fix a bifibration $\pi : \cD \to \M$.  Then there is a function $\llb -
\rrb$ from types \wftype{A}{p} to $\llb A \rrb \in \cD_0$ with $\pi(\llb
A \rrb) = p$ and from $\deq$-classes of derivations $\seq{x:A_1, \ldots,
  x_n:A_n}{\alpha}{C}$ to morphisms $d \in \cD(\llb A_1 \rrb, \dots, \llb
A_n \rrb) \to \llb C \rrb$, such that $\pi(d) = \alpha$.
\end{theorem}



\newcommand\seqa[2]{\ensuremath{#1 \vdash^{\dsd a} #2}}
\newcommand\splits{\rightrightarrows}
\newcommand\vars[1]{\ensuremath{\overline{#1}}}

\appendix[Adequacy Proofs]

\subsection{Affine Logic}

Consider the following rules for affine logic, where the context is
represented by a list of assumptions labeled with variables, and $\Gamma
\splits \Delta_1,\Delta_2$ means interleaving $\Delta_1$ and $\Delta_2$
in some order equals $\Gamma$.
\[
\begin{array}{c}
\infer{\seqa{\Gamma}{P}}{P \in \Gamma}
\qquad
\infer{\seqa{\Gamma,z:A\otimes B,\Gamma'}{C}}
      {\seqa{\Gamma,\Gamma',x:A,y:B}{C}}
\qquad
\infer{\seqa{\Gamma}{A \otimes B}}
      {\Gamma \splits \Delta_1,\Delta_2 &
        \seqa{\Delta_1}{A} &
        \seqa{\Delta_2}{B}}
\\\\
\infer{\seqa{\Gamma}{A \lolli B}}
      {\seqa{\Gamma,x:A}{B}}
\qquad
\infer{\seqa{\Gamma,f:A \lolli B,\Gamma'}{C}}
      {\Gamma,\Gamma' \splits \Delta_1,\Delta_2 &
        \seqa{\Delta_1}{A} &
        \seqa{\Delta_2,z:B}{C}
      }
\end{array}
\]
Weakening and exchange are admissible for these rules.  

Using the mode theory from Section~\ref{sec:ex:affine}, we translate the
propositions and contexts of adjoint logic as follows:
\[
\begin{array}{rcl}
P^* & = & P \\
(A\otimes B)^* & = & \F{x \otimes y}{x:A^*,y:B^*} \\
(A\lolli B)^* & = & \U{c.c \otimes x}{x:A^*}{B^*} \\
\\
\cdot^* & = & \cdot\\
(\Gamma,x:A)^* & = & \Gamma^*,x:A^*\\
\end{array}
\]
We also define a function that collects the variables from $\Gamma$ as a
context descriptor:
\[
\begin{array}{rcl}
\vars{\cdot} & = & 1\\
\vars{(\Gamma,x:A)} & = & \vars{\Gamma} \otimes x\\
\end{array}
\]

Overall, we have
\begin{theorem}[Adequacy of provability for $\seqa{}{}$] ~\\
$\seqa{\Gamma}{A}$ iff $\seq{\Gamma^*}{\vars{\Gamma}}{A^*}$
\end{theorem}

\begin{proof}

The forward direction is by induction on $\seqa{\Gamma}{A}$:
\begin{itemize}
\item For the hypothesis rule, we need to show
  \seq{\Gamma^*}{\vars{\Gamma}}{P}.  Because $x$ is in $\Gamma$, we can
  prove by induction on $\Gamma$ that $x:P$ is in $\Gamma^*$ and that
  $\vars{\Gamma} \deq \alpha \otimes x$.  Thus, the weakening
  transformation gives $\alpha \otimes x \spr 1 \otimes x \deq
  x$. Therefore we can derive
\[
\infer[\dsd{v}]
      {\seq{\Gamma^*}{\vars{\Gamma}}{P}}
      {x:P \in \Gamma & 
        {\vars{\Gamma}} \spr x}
\]

\item For $\otimes$-left, the inductive hypothesis gives
\seq{\Gamma^*,\Gamma'^*,x:A^*,y:B^*}{\vars{\Gamma} \otimes \vars{\Gamma'} \otimes x \otimes y}{C^*}
and we want 
\seq{\Gamma^*,z:\F{x\otimes y}{x:A^*,y:B^*},\Gamma'^*}{\vars{\Gamma} \otimes z \otimes \vars{\Gamma'}}{C^*}.
This is \FL on the inductive hypothesis, with using associativity and commutativity of
$\otimes$ from the mode theory to move $x \otimes y$ to the end.  

\item 
For $\otimes$-right, we 
have \seq{\Delta_1^*}{\vars{\Delta_1}}{A^*}
and \seq{\Delta_2^*}{\vars{\Delta_2}}{B^*} by the inductive hypotheses,
which we can weaken to
 \seq{\Delta_1^*,\Delta_2^*}{\vars{\Delta_1}}{A^*}
and 
 \seq{\Delta_1^*,\Delta_2^*}{\vars{\Delta_2}}{B^*}, 
and then exchange to 
 \seq{\Gamma^*}{\vars{\Delta_1}}{A^*}
and 
 \seq{\Gamma^*}{\vars{\Delta_2}}{B^*} (using a lemma that when $\Gamma \splits \Delta_1,\Delta_2$,
$\Gamma$ and $(\Delta_1,\Delta_2)$ differ only in order, and that
 reordering is preserved by the mapped application of $*$).  
To apply \FR to derive \seq{\Gamma^*}{\vars{\Gamma}}{\F{x \otimes y}{x:A,y:B}}, it thus suffices to show that 
$\vars{\Gamma} \spr \vars{\Delta_1}\otimes\vars{\Delta_2}$, which we can
prove by induction on $\Gamma \splits \Delta_1,\Delta_2$ using the
commutative monoids laws.  

\item For $\lolli$-right, we have
  \seq{\Gamma^*,x:A^*}{\vars{\Gamma}\otimes x}{B^*}
by the inductive hypothesis, which is exactly the premise of using \UR
to prove
  \seq{\Gamma^*}{\vars{\Gamma}}{\U{c.c\otimes x}{x:A^*}{B^*}}.  

\item For $\lolli$-left, we have \seq{\Delta_1^*}{\vars{\Delta_1}}{A}
  and \seq{\Delta_2^*,z:B^*}{\vars{\Delta_2}\otimes z}{C^*} by the
  inductive hypothesis, and by similar reasoning to the $\otimes R$
  case, we can weaken and exchange to
  \seq{\Gamma^*,\Gamma'^*}{\vars{\Delta_1}}{A} and
  \seq{\Gamma^*,\Gamma'^*,z:B^*}{\vars{\Delta_2}\otimes z}{C^*} and then
  finally weaken to \seq{\Gamma^*,f:(A\lolli
    B)^*,\Gamma'^*}{\vars{\Delta_1}}{A} and \seq{\Gamma^*,f:(A\lolli
    B)^*,\Gamma'^*,z:B^*}{\vars{\Delta_2}\otimes z}{C^*}.
Thus, we only need to show the transformation premise of
\[
\infer[\UL]{\seq{\Gamma^*,f:(A\lolli B)^*,\Gamma'^*}{\vars{\Gamma}\otimes f \otimes \vars{\Gamma'}}{C^*}}
      {  
        \begin{array}{l}
        {\vars{\Gamma}\otimes f \otimes \vars{\Gamma'}} \spr
        (\vars{\Delta_2} \otimes z)[f \otimes \vars{\Delta_1} /z] \\
        \seq{\Gamma^*,f:(A\lolli B)^*,\Gamma'^*}{\vars{\Delta_1}}{A} \\
        \seq{\Gamma^*,f:(A\lolli B)^*,\Gamma'^*,z:B^*}{\vars{\Delta_2}\otimes z}{C^*}
        \end{array}
      }
\]
This follows from $\Gamma \splits \Delta_1,\Delta_2$, using associativity and commutativity
of $\otimes$.  
\end{itemize}

In the converse direction, we generalize slightly and prove 
\begin{quote}
If $\Gamma_0 \ge \Gamma$ and $\seq{\Gamma_0^*}{\vars{\Gamma}}{A^*}$ then
$\seqa{\Gamma}{A}$
\end{quote}
where $\Gamma_0 \ge \Gamma$ means that $\Gamma_0$ contains every
assumption $x:A$ that $\Gamma$ does and possibly more (i.e. $\Gamma_0$
is a weakening of an exchange of $\Gamma$).  That is, we allow the
actual context of the encoding to contain more assumptions than the
context descriptor, but all of those assumptions will be of a type that
is in the image of the translation of affine logic.  This is necessary
because of the contraction in \UL, where the function is removed from
the context descriptor but not from the context itself.

We need the following facts about the mode theory.  

First, every context descriptor is equal to one of the form $x_1 \otimes
\ldots \otimes x_n$ (for not necessarily distinct $x_i$), which we
abbreviate as $\vec{x_i}$.  This is by definition (the only constants
are $\otimes,1$) and the associativity/unit laws.

\begin{lemma} \label{lem:affine-mode-1}
If $\vec{x_i} \spr \vec{y_i}$ then $\vec{x_i} \deq \vec{z_i} \otimes
\vec{y_i}$---i.e. $\vec{y_i}$ is a sub-multiset of $\vec{x_i}$
\end{lemma}
\begin{proof}
In the case for weakening $\vec{x_i} \spr 1$ (the only axiom), take
$\vec{z_i} = \vec{x_i}$.  In the case for reflexivity take $\vec{z_i} =
1$.  In the case for transitivity, we have $\vec{x_i} \spr \vec{x_i'}
\spr \vec{y_i}$.  By the second inductive hypothesis, we have
$\vec{x_i'} \deq \vec{z_i'} \otimes \vec{y_i}$, so by the first we have
$\vec{x_i} \deq \vec{z_i''} \otimes (\vec{z_i'} \otimes \vec{y_i})$, so
take $\vec{z_i}$ to be $\vec{z_i'} \otimes \vec{z_i''}$.  In the case
for congruence, we have $\vec{x_i} \spr \vec{y_i}$, and $\vec{x_i'} \spr
\vec{y_i'}$, and we want to show that $\subst{\vec{x_i}}{\vec{x_i}'}{x}
\deq \vec{z_i} \otimes \subst{\vec{y_i}}{\vec{y_i'}}{x}$ for some
$\vec{z_i''}$. By the inductive hypotheses $x_i = \vec{z_i} \otimes
\vec{y_i}$ and $x_i' \deq \vec{z_i}' \otimes \vec{y_i'}$.  So $x_i[x_i'/x]
\deq (\vec{z_i} \otimes y_i)[\vec{z_i}' \otimes y_i'/x]$ and it remains to
show that $(\vec{z_i} \otimes \vec{y_i})[\vec{z_i}' \otimes
  \vec{y_i}'/x] = \vec{z_i}'' \otimes \subst{\vec{y_i}}{\vec{y_i'}}{x}$,
which is true by commuting any $\vec{z_i}'s$ in $(\vec{y_i})[\vec{z_i}'
  \otimes \vec{y_i}'/x]$ to the left.
\end{proof}

\begin{lemma} \label{lem:affine-mode-2}
If $\vars{\Gamma} \spr \alpha_1 \otimes \alpha_2$ then there exist
contexts $\Gamma_1$ and $\Gamma_2$ such that $\vars{\Gamma_1} \deq
\alpha_1$ and $\vars{\Gamma_2} \deq \alpha_2$ and $\Gamma \ge
\Gamma_1,\Gamma_2$.
\end{lemma}

\begin{proof}  TODO
\end{proof}

Moving on to the proof of the converse,

\begin{itemize}
\item In the case for the assumption rule, we have
\[
\infer{\seq{\Gamma_0^*}{\vars{\Gamma}}{P}}
      {x:P \in \Gamma_0 &
       \vars{\Gamma} \spr x}
\]

First, by Lemma~\ref{lem:affine-mode-1}, $\vars{\Gamma} \spr x$ implies
$\vars{\Gamma} \deq \vec{z} \otimes x$.  This in turn implies that $x$
is in $\Gamma$ (because the ACU equations do not change the elements, so
it is in $\vars{\Gamma}$, which only consists of variables from
$\Gamma$).  Because $\Gamma$ is a subcontext of $\Gamma_0$ and $x:P$ is
in $\Gamma_0$ and $x$ is also in $\Gamma$, $x$ must have type $P$ in
$\Gamma$, so we can apply the hypothesis rule of affine logic.

\item In the case where \UR\/ was used to derive
  \seq{\Gamma_0^*}{\vars{\Gamma}}{A^*}, $A$ must be $A_1 \lolli A_2$
  (because no other types encode to \Usymb), and the premise is
  \seq{\Gamma_0^*,x:A_1^*}{\vars{\Gamma}\otimes x}{A_2^*}.  Since
  ${\Gamma_0,x:A_1} \ge {\Gamma,x:A_1}$, the inductive hypothesis gives
  $\seqa{\Gamma,x:A_1}{A_2}$, and we can apply $\lolli$-right.

\item In the case where \FR\/ was used, the conclusion must be $(A_1
  \otimes A_2)$, and we have $\vars{\Gamma} \spr (\alpha_1 \otimes
  \alpha_2)$ with \seq{\Gamma_0^*}{\alpha_1}{A_1^*} and
  \seq{\Gamma_0^*}{\alpha_2}{A_1^*}.  Using
  Lemma~\ref{lem:affine-mode-2}, we have $\Gamma_0 \ge \Gamma \ge
  \Gamma_1,\Gamma_2$ and \seq{\Gamma_0^*}{\vars{\Gamma_1}}{A_1^*}
  \seq{\Gamma_0^*}{\vars{\Gamma_2}}{A_2^*}.  Thus, by the inductive
  hypotheses we get $\seqa{\Gamma_1}{A_1}$ and $\seqa{\Gamma_2}{A_2}$,
  so $\seqa{\Gamma_1,\Gamma_2}{A_1 \otimes A_2}$.  Then weakening and
  exchange on $\Gamma \ge \Gamma_1,\Gamma_2$ gives the result.  

\item In the case where \FL\/ was used, we know that every assumption even
  in the larger context $\Gamma_0$ is of the form $A^*$ for some $A$, so
  the formula under elimination must be $z:(A_1 \otimes A_2)^* \in \Gamma_0$. The
  premise is
  \seq{\Gamma_0^*-z,x:A_1^*,y:A_2^*}{\vars{\Gamma}[(x \otimes y)/z]}{C^*},
  and we want 
  \seqa{\Gamma}{C}.  

  We distinguish cases on whether $z$ (which is only known to be in
  $\Gamma_0$) is bound in $\Gamma$ or not.  

  If it is, then $\vars{\Gamma}[(x \otimes y)/z] \deq
  (\vars{\Gamma}-z)\otimes x \otimes y$, and since
  $\Gamma_0^*-z,x:A_1,y:A_2 \ge \Gamma-z,x:A_1,y:A_2$, we get
  \seqa{\Gamma-z,x:A_1,y:A_2}{C} by the inductive hypothesis, and can
  apply $\otimes$-left.

  If it is not, then $\vars{\Gamma}[(x \otimes y)/z] \deq
  \vars{\Gamma}$.  Because $\Gamma_0 \ge \Gamma$ by assumption,
  ${\Gamma_0-z,x:A_1,y:A_2} \ge \Gamma$ (it removes something not in
  $\Gamma$, and adds two things, and the premise is 
  \seq{\Gamma_0^*-z,x:A_1^*,y:A_2^*}{\vars{\Gamma}}{C^*},
  we get \seqa{\Gamma}{C} directly by the inductive hypothesis.  
  In this case, the given proof does a left rule on a ``0-use''
  variable that does not occur in the context descriptor, which adds
  some additional 0-use assumptions to the context.  The inductive
  hypothesis is compatible with this, and returning its result
  drops this step.    

\item In the case for \UL 

\end{itemize}
\end{proof}

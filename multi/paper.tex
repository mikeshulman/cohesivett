\documentclass[conference,compsoconf]{../drl-common/IEEEtran}
\IEEEoverridecommandlockouts

\usepackage{multicol}
\usepackage{mathptmx}
\usepackage{color}
\usepackage[cmex10]{amsmath}
\usepackage{amsthm}
\usepackage{amssymb}
\usepackage{stmaryrd}
\usepackage{../drl-common/proof}
\usepackage{../drl-common/typesit}
\usepackage{../drl-common/typescommon}
\usepackage{../drl-common/theorem-envs}
\usepackage[square,sort]{natbib}
%% \usepackage{arydshln}
\usepackage{graphics}
\usepackage{natbib}
\usepackage{url}
\usepackage{relsize}
\usepackage{tipa}

\usepackage{tikz}
\usetikzlibrary{decorations.pathmorphing}

\usepackage{fancyvrb}
\newcommand{\ttt}[1]{\texttt{#1}}


\newcommand\Bx[2]{\ensuremath{\Box_{#1} \, {#2}}}
\newcommand\Crc[2]{\ensuremath{\bigcirc_{#1} \, {#2}}}
\newcommand\Dia[2]{\ensuremath{\Diamond_{#1} \, {#2}}}
\newcommand\Flat[1]{\ensuremath{\flat \, {#1}}}
\newcommand\Sharp[1]{\ensuremath{\sharp \, {#1}}}
\newcommand{\sh}{\text{\textesh}}

\newcommand\magicwand{\mathrel{-\mkern-6mu*}}
\newcommand\mor[3]{\ensuremath{#2} \longrightarrow_#1 #3}
\newcommand\C{\ensuremath{\mathcal{C}}}
\newcommand\D{\ensuremath{\mathcal{D}}}
\newcommand\E{\ensuremath{\mathcal{E}}}
\newcommand\deq{\ensuremath{\equiv}}
\newcommand\spr{\ensuremath{\Rightarrow}} %% structural property/2-cell
\newcommand\seq[3]{\ensuremath{#1 \vdash_{#2} #3}}
\newcommand\F[2]{\ensuremath{\dsd{F}_{#1}(#2)}}
\newcommand\U[3]{\ensuremath{\dsd{U}_{#1}(#2 \mid #3)}}
\newcommand\Fsymb[0]{\dsd{F}}
\newcommand\Usymb[0]{\dsd{U}}
\newcommand\tsubst[2]{\ensuremath{#1[#2]}}
\renewcommand\subst[3]{\ensuremath{#1[#2/#3]}}
\newcommand\wftype[2]{\ensuremath{#1 \,\, \dsd{type}_{#2}}}
\renewcommand\wfctx[2]{\ensuremath{#1 \,\, \dsd{ctx}_{#2}}}
\newcommand\modeof[1]{\ensuremath{\hat{#1}}}
\newcommand\many[1]{\ensuremath{\overline{#1}}}
\renewcommand{\oftp}[3]{\ensuremath{#1 \, \vdash #2 \, \dcd{:} \, #3}}
\newcommand\FL{\dsd{FL}}
\newcommand\FR{\dsd{FR}}
\newcommand\UL{\dsd{UL}}
\newcommand\UR{\dsd{UR}}
\newcommand\lolli\multimap
\newcommand\la\dashv

\newcommand{\ignore}[1]{}

\begin{document}

\title{A Fibrational Framework for \\ Substructural and Modal Logics}

% author names and affiliations
% use a multiple column layout for up to three different
% affiliations
\author{\IEEEauthorblockN{Daniel R. Licata}
\IEEEauthorblockA{Wesleyan University\\
\url{dlicata@wesleyan.edu}}
\and
\IEEEauthorblockN{Michael Shulman}
\IEEEauthorblockA{University of San Diego \\
  \url{shulman@sandiego.edu}}

\thanks{
  ?
}

}

\maketitle

\begin{abstract}
Many intuitionistic substructural and modal logics can be seen as a
restriction on the allowed proofs in a basic structural logic or
$\lambda$-calculus.  For example, substructural logics remove structural
properties such as weakening, exchange, and/or contraction, while modal
logics place restrictions on positions in which certain kinds of
variables can be used.  We give a new sequent calculus that makes this
idea precise, describing a substructural or modal logic derivation as a
structural proof that obeys some constraints on the use of the context.
Because the sequent calculus is parametrized by a \emph{mode theory}
describing the constraints, a single generic proof of cut admissibility
can be instantiated to all of the above logics, as well as more complex
variants, including n-linear variables, bunched implications, and
subexponentials.  This codifies the common patterns in cut proofs as an
abstraction.  Semantically, the new sequent calculus corresponds to a
functor between 2-dimensional cartesian multicategories, and the logical
connectives make this functor into a bifibration.  The resulting
framework can be used both to understand logics from the literature and
to design new substructural and modal logics.
\end{abstract}


\section{Introduction}

In ordinary intuitionistic logic or $\lambda$-calculus, assumptions or
variables can go unused (weakening), be used in any order (exchange), be
used more than once (contraction), and be used in any position in a
term.  \emph{Substructural} logics, such as linear logic, ordered logic,
relevant logic, and affine logic, drop some of these structural
properties of weakening, exchange, and contraction, while \emph{modal
  logics} place restrictions on where variables may be used---e.g. a
formula $\Bx{} C$ can only be proved using assumptions of $\Bx{} A$,
while an assumption of $\Dia{}{A}$ can only be used when the conclusion
is $\Dia{}{C}$.  Substructural and modal logics have had many
applications to both functional and logic programming (modeling concepts
such state, staged computation, distribution, and concurrency, to name
just a few).

Substructural and modal logics can also be used as \emph{internal
  languages} of categories, where one uses an appropriate logical
language to do constructions ``inside'' a particular mathematical
setting, which often leads to shorter statements than working
``externally''.  For example, to define a function when working
``externally'' in domains, one must first define the underlying
set-theoretic function, and then prove that it is continuous.  But when
using untyped $\lambda$-calculus as an internal language of domains,
there is no need to prove that a function described by a $\lambda$-term
is continuous, because all terms are shown to denote continous functions
once and for all.  Substructural logics extend this idea to various
forms of monoidal categories, while modal logics describe monads and
comonads.  Recently,
\citet{schreibershulman12cohesive,shulman15realcohesion} proposed using
modal operators to add a notion of \emph{cohesion} to homotopy type
theory/univalent foundations~\citep{,voevodsky06homotopy,uf13hott-book}.
Without going into the precise details of this application, the idea is
to add a triple $\sh{} \la \Flat{} \la \Sharp{}$ of type operators,
where for example $\Sharp A$ is a monad (like a modal possibility
$\Diamond$ or $\bigcirc$), $\Flat A$ is a comonad (like a modal
necessity $\Box$), and there is an adjunction structure between them
(e.g. $\flat{A} \to B$ is the same as $A \to \Sharp{B}$).  This raised
the question of how to best add modalities with these properties to type
theory.

Because other similar applications would have different monads and
comonads with different properties, we would like general tools for
going from a semantic situation of interest to a ``nice'' type
theory/logic for it, e.g. one with cut and identity admissibility and
the subformula property. In previous work~\citep{ls16adjoint}, we
considered the special case of a single-assumption logic, building most
directly on the adjoint logics of
\citet{benton94mixed,bentonwadler96adjoint,reed09adjoint}.  Here we
extend this previous work to the multi-assumption case.  The resulting
framework is quite general and covers many existing intuitionistic
substructural and modal connectives: cartesian, linear, affine,
relevant, ordered, bunched~\citep{ohearnpym99bunched} and
non-associative products and implications; $n$-linear
variables~\citep{reed08namessubstructural}; the comonadic $\Box$ and
linear exponential $!$ and
subexponentials~\citep{nigammiller09subexponentials,danos+93subexponentials};
monadic $\Diamond$ and $\bigcirc$ modalities; and adjoint logic $F$ and
$G$~\citep{benton94mixed,bentonwadler96adjoint,reed09adjoint}, including
the single-assumption 2-categorical version from our previous
work~\citep{ls16adjoint}.  It also supports variations on these, such as
non-monoidal comonads and non-strong monads.  We show that a single,
simple structural~\citep{pfenning94cut} proof of cut (and identity)
admissibility applies to all of these logics, as well as any new logics
that can be described in the framework.
%% While it is not too surprising
%% that this is possible, given that cut proofs for these logics all follow
%% a similar template, it is nonetheless satisfying to codify this pattern
%% as an abstraction.

At a high level, the framework expresses the idea that all of the above
logics are a restriction on how variables can be used in ordinary
structural/cartesian proofs.  We express these restrictions using a
first layer of the logic, which is a simple type theory for what we will
call \emph{modes} and \emph{context descriptors}.  The modes are just a
collection of base types, which we write as $p,q,r$, while a context
descriptor is a term built from variables and constants.  The next layer
is the main logic.  Each proposition of the logic is assigned a mode,
and the basic sequent is \seq{x_1 : A_1, \ldots, x_n : A_n}{\alpha}{C},
where if $A_i$ has mode $p_i$, and $C$ has mode $q$, then $\oftp{x_1 :
  p_1,\ldots, x_n : p_n}{\alpha}{q}$.  
%% In a sequent
%% \seq{\Gamma}{\alpha}{A}, the idea is that $\Gamma$ binds some variables
%% for use both in $\alpha$ and in the derivation.  
$\Gamma$ itself behaves like an ordinary structural/cartesian context,
while the substructural and modal aspects are enforced by the
\emph{term} $\alpha$, which describes how the resources from $\Gamma$
are allowed to be used.  For example, in linear logic/ordered logic/BI,
the context is usually taken to be a multiset/list/tree (respectively).
We represent the multiset or list or tree using a pair of an ordinary
structural context $\Gamma$, together with a term $\alpha$ that
describes the multiset or list or tree structure, labeled with variables
from the ordinary context at the leaves.  We pronounce a sequent
\seq{\Gamma}{\alpha}{A} as ``$\Gamma$ proves $A$ {along,over} $\alpha$''
or ``$\Gamma$ structured according to $\alpha$ proves $A$''.

For example, suppose we have one mode $\dsd{n}$, together with a context
descriptor constant
\[
x : \dsd{n}, y:\dsd{n} \vdash x \odot y : \dsd{n}
\]
Then an example sequent
\[
\seq{x:A, y:B, z:C, w:D}{(y \odot x) \odot z}{E}
\]
should be read as saying that we must prove $E$ using the resources $y$
and $x$ and $z$ (but not $w$) according to the particular tree structure
${(y \odot x) \odot z}$.  If we say nothing else, the framework will
treat $\odot$ as describing a non-associative, linear, ordered context:
if we have a product-like type $A \odot B$ internalizing this context
operation,\footnote{We overload binary operations to refer both to
  context descriptors and propositional connectives, because it is clear
  from whether it is applied to variables $x,y,z$ or propositions
  $A,B,C$ which we mean.}  then we will \emph{not} be able to prove
associativity ($(A \odot B) \odot C \dashv\vdash A \odot (B \odot C)$)
or contraction ($A \vdash A \odot A$) or exchange ($A \odot B \vdash B
\odot A$) etc.

To get from this basic structure to linear or affine or relevant or
cartesian logic, we need to add some structural properties to the
context descriptor term $\alpha$.  We analyze structural properties as
\emph{equations}, or more generally \emph{directed transformations}, on
such terms.  For example, to specify linear logic, we will add a unit
element $1 : \dsd{n}$ together with equations making $(\odot,1)$ into a
commutative monoid:
\[
\begin{array}{c}
x \odot (y \odot z) = (x \odot y) \odot z\\
x \odot 1 = x = 1 \odot x\\
x \odot y = y \odot x
\end{array}
\]
so that the context descriptors ignore associativity and order.  To get
BI, we add an additional commutative monoid $(\times,\top)$ (with
weakening and contraction, as discussed below), so that a BI context
tree $(x:A,y:B);(z:C,w:D)$ can be represented by the ordinary context
$x:A,y:B,z:C,w:D$ with the term $(x \odot y) \times (z \odot w)$
describing the tree.  Because the context descriptors are themselves
ordinary structural/cartesian terms, the same variable can occur more
than once or not at all.  A descriptor such as $x \odot x$ captures the
idea that we can use the \emph{same} variable $x$ twice, expressing
$n$-linear types~\citep{reed08namessubstructural}.  Thus, we can express
contraction for a particular context descriptor $\odot$ as an equation
$x = x \odot x$ (one use of $x$ is the same as two, or $\odot$ is an
idempotent binary operation).  However, weakening cannot be represented
as an equation between context descriptors: an equation $x = 1$ would
trivialize the logic to ordinary intuitionistic logic.  Instead, to
express weakening, we use a directed transformation $x \spr 1$, which is
oriented to allow throwing away an allowed use of $x$, but not creating
an allowed use from nothing.  We refer to these as \emph{structural
  transformations}, to evoke their use in representing the structural
properties of object logics that are embedded in our framework.
Structural transformations are also used to describe relationships
between adjunctions~\citep{ls16adjoint}.

In summary, to specify a particular substructural or modal logic, one
gives constants generating context descriptors $\alpha$, with equations
$\alpha = \beta$ and transformations $\alpha \spr \beta$ expressing
structural properties.  The main sequent $\seq{\Gamma}{\alpha}{A}$
respects the specified structural properties in the sense that when
$\alpha = \beta$, we regard $\seq{\Gamma}{\alpha}{A}$ and
$\seq{\Gamma}{\beta}{A}$ as the same sequent, while when $\alpha \spr
\beta$, there will be an operation that takes a derivation of
\seq{\Gamma}{\beta}{A} to a derivation of \seq{\Gamma}{\alpha}{A}.

A guiding principle of the framework is a meta-level notion of
\emph{structurality over structurality}.  For example, we always have
\emph{weakening over weakening}: if \seq{\Gamma}{\alpha}{A} then
\seq{\Gamma,y:B}{\alpha}{A}, where $\alpha$ itself is weakened with $y$.
This does not prevent encodings of e.g. linear logic: it is permissible
to weaken a derivation of \seq{\Gamma}{x_1 \odot \ldots \odot x_n}{A}
(``use $x_1$ through $x_n$'') to a derivation of \seq{\Gamma,y:B}{x_1
  \odot \ldots \odot x_n}{A} because the (weakened) context descriptor
still disallows the use of $y$.  Similarly, we always have exchange over
exchange and contraction over contraction.  The identity and and cut
principles are analogous:
\[
\infer{\seq{\Gamma,x:A}{x}{A}}{}
\qquad
\infer{\seq{\Gamma}{\subst{\beta}{\alpha}{x}}{B}}
    {\seq{\Gamma,x:A}{\beta}{B} &
     \seq{\Gamma}{\alpha}{A}}
\]
The identity-over-identity principle says that we should be able to
prove $A$ using exactly an assumption $x:A$.  The cut principle says
that the context descriptor for the result of the cut is the
substitution of the context descriptor used to prove $A$ into the one
used to prove $B$.  For example, together with weakening-over-weakening,
this captures the usual cut principle of linear logic, which says that
cutting $\Gamma,x:A \vdash B$ and $\Delta \vdash A$ yields
$\Gamma,\Delta \vdash B$.  If $\Gamma$ binds $x_1,\ldots,x_n$ and
$\Delta$ binds $y_1,\ldots,y_n$, then we will represent the two
derivations to be cut together by sequents with
\[
\begin{array}{l}
\beta = x_1 \odot \ldots \odot x_n \odot x\\
\alpha = y_1 \odot \ldots \odot y_n
\end{array}
\]
so
\[
\beta[\alpha/x] = x_1 \odot \ldots \odot x_n \odot y_1 \odot \ldots \odot y_n
\]
correctly deletes $x$ and replaces it with the variables from $\Delta$.
Moreover, in more subtle situations such as BI, the substitution will
insert the resources used to prove the cut formula in the correct place
in the tree.

The framework has two main logical connectives.  The first,
\F{\alpha}{\Delta}, generalizes the \dsd{F} of adjoint
logic~\citep{bentonwadler96adjoint,reed09adjoint} and the tensor
($\otimes$) of linear logic.  The second, \U{x.\alpha}{\Delta}{A},
generalizes the $\dsd{G}/\dsd{U}$ of adjoint logic and the implication
$A \lolli B$ of linear logic.  Here $\Delta$ is a context of assumptions
$x_i:A_i$, and trivializing the context descriptors (i.e. adding an
equation $\alpha = \beta$ for all $\alpha$ and $\beta$) degenerates
$\F{\alpha}{\Delta}$ into the ordinary intuititionistic product $A_1
\times \ldots \times A_n$, while \U{x.\alpha}{\Delta}{A} becomes $A_1
\to \ldots \to A_n \to A$.  Though we do not give a full
polarized/focused proof theory in this paper, we do prove that \dsd{F}
is left-invertible and \dsd{U} is right-invertible, and we conjecture
that focusing works with the polarization that one would expect based on
these degeneracies ($\F{\alpha}{\Delta^{\mathord{+}}}^{\mathord{+}}$ and
$\U{x.\alpha}{\Delta^{\mathord{+}}}{A^{\mathord{-}}}^{\mathord{-}}$).
In linear logic terms, our \dsd{F} and \dsd{U} cover both the
multiplicatives and exponentials; additives can be added separately by
essentially the usual rules.

Being a very general theory, our framework treats the structural
properties in a general but na\"ive way, allowing an arbitrary
structural transformation to be applied at the non-invertible rules for
$\dsd{F}$ and $\dsd{U}$ and at the leaves of a derivation.  For specific
embedded logics, there will often be a more refined discipline that
suffices---e.g. for cartesian logic, always contract all assumptions at
in all premises, rather than choosing which assumptions to contract.  We
view our framework as a tool for bridging the gap between an intended
semantic situation such as the cohesion example mentioned above (``a
comonad and a monad which are themselves adjoint'') and a proof theory:
the framework gives \emph{some} proof theory for the semantics, and the
placement of structural rules can then be optimized purely in syntax.
To support this mode of use, we give an equational theory on sequent
derivations that identifies different placements of the same structural
rules.  This equational theory is used to prove correctness of such
optimizations not just at the level of provability, but also identity of
derivations---which matters for our intended applications to internal
languages.

Semantically, the logic corresponds to a functor between
\emph{2-dimensional cartesian multicategories} which is a fibration in
various senses.  Multicategories are a generalization of categories
which allow more than one object in the domain, and cartesianness means
that the multiple domain objects are treated structurally.  The
2-dimensionality supplies a notion of morphism between (multi)morphisms,
which correspond to the structural transformations.  The functor
specifies the mode of each proposition and the context descriptor of a
sequent.  The fibration conditions (similar to \citep{hermida,hormann})
specify respect for the structural transformations and the presence of
\dsd{F} and \dsd{U} types.

The remainder of this paper is organized as follows.  FIXME

FIXME: comparison with display logic, L/CLF, what else?  


\newcommand\wfsp[4]{\ensuremath{#1 \vdash #2 \spr_{#4} #3}}

\section{Sequent Calculus}
\label{sec:syntax}

\newcommand\wfsig[1]{\ensuremath{#1 \, \dsd{sig}}}
\newcommand\deqtms[5]{\ensuremath{#1 \vdash_{#2} #3 \deq #4 : #5}}
\newcommand\wfsps[5]{\ensuremath{#1 \vdash_{#2} #3 \spr_{#5} #4}}

\subsection{Mode Theories}

\begin{figure}
\begin{small}
\[
\begin{array}{l}
\framebox{Signatures \wfsig{\Sigma}}
\qquad
\infer{\wfsig{\cdot}}
      {}
\qquad
\infer{\wfsig{(\Sigma,p \, \dsd{mode})}}
      {\wfsig{\Sigma}}
\qquad
\infer{\wfsig{(\Sigma,c : \, p_1,\ldots,p_n \to q)}}
      {\wfsig{\Sigma} &
        (p_1 \, \dsd{mode},\ldots,p_n \, \dsd{mode},q \, \dsd{mode}) \in \Sigma
      }
\\ \\
\infer{\wfsig{(\Sigma, (\alpha \deq \alpha' : \psi \to p))}}
      {\wfsig{\Sigma} &
        \psi \dsd{ctx}_\Sigma & 
        p \, \dsd{mode} \in \Sigma &
        \oftps{\psi}{\Sigma}{\alpha}{p} & 
        \oftps{\psi}{\Sigma}{\alpha'}{p} 
      }
\qquad
\infer{\wfsig{(\Sigma, (\alpha \spr \alpha' : \psi \to p))}}
      {\wfsig{\Sigma} &
        \psi \dsd{ctx}_\Sigma & 
        p \, \dsd{mode} \in \Sigma &
        \oftps{\psi}{\Sigma}{\alpha}{p} & 
        \oftps{\psi}{\Sigma}{\alpha'}{p} 
      }
\\\\
\ifthenelse{\boolean{short}}{}
{\framebox{Mode contexts $\psi \, \dsd{ctx}_{\Sigma}$}
\qquad
\infer{\cdot \, \dsd{ctx}_{\Sigma}}{}
\qquad
\infer{(\psi, x:p) \, \dsd{ctx}_{\Sigma}}
      {\psi \, \dsd{ctx}_{\Sigma} & 
        p \, \dsd{mode} \in \Sigma
      }
\\\\
}
\framebox{Context descriptors \oftps{\psi}{\Sigma}{\alpha}{p},
  where $\psi \, \dsd{ctx}_\Sigma$ and $p \, \dsd{mode} \in \Sigma$}
\qquad
\infer{\oftps{\psi}{\Sigma}{x}{p}}
      {x:p \in \psi}
\quad
\infer{\oftps{\psi}{\Sigma}{\dsd{c}(\alpha_1,\ldots,\alpha_n)}{q}}
      {(\dsd{c} : p_1,\ldots,p_n \to q) \in \Sigma &
       \oftps{\psi}{\Sigma}{\alpha_i}{p_i}
      }
\\\\
\framebox{Mode Substitutions \oftps{\psi}{\Sigma}{\gamma}{\psi'}, where
  $\psi \, \dsd{ctx}_\Sigma$ and $\psi' \, \dsd{ctx}_\Sigma$ }
\qquad
\infer{\oftps{\psi}{\Sigma}{\cdot}{\cdot}}{}
\qquad
\infer{\oftps{\psi}{\Sigma}{\gamma,\alpha/x}{\psi',x:p}}
      {\oftps{\psi}{\Sigma}{\gamma}{\psi'} &
        \oftps{\psi}{\Sigma}{\alpha}{p}}

\ifthenelse{\boolean{short}}{}
{\framebox{Equalities of mode morphisms
  \deqtms{\psi}{\Sigma}{\alpha}{\alpha'}{p},
where $\psi \, \dsd{ctx}_\Sigma$ and $p \, \dsd{mode} \in \Sigma$
and \oftps{\psi}{\Sigma}{\alpha}{p}
and \oftps{\psi}{\Sigma}{\alpha'}{p}
}
\qquad
\infer{\deqtms{\psi}{\Sigma}{\alpha}{\alpha}{p}}{}
\qquad
\infer{\deqtms{\psi}{\Sigma}{\alpha_1}{\alpha_2}{p}}
      {\deqtms{\psi}{\Sigma}{\alpha_2}{\alpha_1}{p}}
\qquad
\infer{\deqtms{\psi}{\Sigma}{\alpha_1}{\alpha_3}{p}}
      {\deqtms{\psi}{\Sigma}{\alpha_1}{\alpha_2}{p} &
        \deqtms{\psi}{\Sigma}{\alpha_2}{\alpha_3}{p} &
      }
\\ \\
\infer{\deqtms{\psi,\psi'}{\Sigma}{\subst{\beta}{\alpha}{x}}{\subst{\beta'}{\alpha'}{x}}{q}}
      {\deqtms{\psi,x:p,\psi'}{\Sigma}{\beta}{\beta'}{q} &
        \deqtms{\psi,\psi'}{\Sigma}{\alpha}{\alpha'}{p}}
\qquad
\infer{\deqtms{\psi}{\Sigma}{\alpha}{\alpha'}{p}}
      {(\alpha \deq \alpha' : \psi \to p) \in \Sigma}
}
\\\\
\framebox{Structural transformations \wfsps{\psi}{\Sigma}{\alpha}{\alpha'}{p},
where \oftps{\psi}{\Sigma}{\alpha}{p}
and \oftps{\psi}{\Sigma}{\alpha'}{p}
}
\qquad
\infer{\wfsps{\psi}{\Sigma}{\alpha}{\alpha}{p}}{}
\\\\
\infer{\wfsps{\psi}{\Sigma}{\alpha_1}{\alpha_3}{p}}
      {\wfsps{\psi}{\Sigma}{\alpha_1}{\alpha_2}{p} &
       \wfsps{\psi}{\Sigma}{\alpha_2}{\alpha_3}{p} &
      }
\qquad
\infer{\wfsps{\psi,\psi'}{\Sigma}{\subst{\beta}{\alpha}{x}}{\subst{\beta'}{\alpha'}{x}}{q}}
      {\wfsps{\psi,x:p,\psi'}{\Sigma}{\beta}{\beta'}{q} &
       \wfsps{\psi,\psi'}{\Sigma}{\alpha}{\alpha'}{p}}
\qquad
\infer{\wfsps{\psi}{\Sigma}{\alpha}{\alpha'}{p}}
      {(\alpha \spr \alpha' : \psi \to p) \in \Sigma}
\end{array}
\]
\end{small}
\caption{Syntax for mode theories}
\label{fig:2multicategory}
\end{figure}

The first layer of our framework is a type theory whose types we will call
\emph{modes}, and whose terms we will call \emph{context descriptors} or
\emph{mode morphisms}.  The only modes are atomic/base types $p$.  A
term is either a variable (bound in a context $\psi$) or a typed $n$-ary
constant (function symbol) \dsd{c} applied to terms of the appropriate
types.

This is formalized in the notion of signature, or \emph{mode theory},
defined in Figure~\ref{fig:2multicategory}.  The judgement $\wfsig
\Sigma$ means that $\Sigma$ is a well-formed signature.  The top line
says that a signature is either empty, or a signature extended with a
new mode declaration, or a signature extended with a typed
constant/function symbol, all of whose modes are declared previously in
the signature.  The notation $p_1,\ldots,p_n \to q$ is not itself a
mode, but notation for declaring a function symbol in the signature (it
cannot occur on the right-hand side of a typing judgement).  For
example, the type and term constructors for a monoid $(\odot,1)$ are
represented by a signature $\dsd{p} \, \dsd{mode}, \dsd{\odot} :
(\dsd{p},\dsd{p} \to \dsd{p}), 1 : (\to \dsd{p})$.

\ifthenelse{\boolean{short}}{
We elide the rules for 
the judgement $\psi \, \dsd{ctx}_\Sigma$, which simply says that each
mode used in the 
context of variable declarations $\psi$ is declared in $\Sigma$.  
}
{
The judgement $\psi \, \dsd{ctx}_\Sigma$ defines well-formedness of a
context of variable declarations relative to a signature $\Sigma$: each
mode in the context must be declared in the signature.}  The judgement
$\oftps{\psi}{\Sigma}{\alpha}{p}$ defines well-typedness of context
descriptor terms, which are either a variable declared in the context,
or a constant declared in the signature applied to arguments of the
correct types.  The judgement $\oftps{\psi}{\Sigma}{\gamma}{\psi'}$
defines a substitution as a tuple of terms in the standard way.  The
context $\psi$ in these judgements enjoys the cartesian structural
properties (associativity, unit, weakening, exchange, contraction).
Simultaneous substitution into terms and substitutions is defined as
usual (e.g.  $x[\gamma,\alpha/x] := \alpha$ and
$\dsd{c}(\vec{\alpha_i})[\gamma] := \dsd{c}(\alpha_i[\gamma])$).

Returning to the top of the figure, the final two rules of the judgement
$\wfsig{\Sigma}$ permit two additional forms of signature declaration.
The first of these extends a signature with an equational axiom between
two terms $\alpha$ and $\alpha'$ that have the same mode $p$, in the
same context $\psi$, relative to the prior signature $\Sigma$.  These
equational axioms will be used to encode reversible object language
structural properties, such as associativity, commutativity, and unit
laws.  For example, to specify the right unit law for the above monoid
$(\odot,1)$, we add an axiom $(x \odot 1 \deq x : (x : \dsd{p}) \to
\dsd{p})$ to the signature, which can be read as ``$x \odot 1$ is equal
to $x$ as a morphism from $(x : \dsd{p})$ to \dsd{p}''.  The judgement
\deqtms{\psi}{\Sigma}{\alpha}{\alpha'}{p} (omitted from the figure; the
rules are the same as for $\spr$ plus symmetry) is the least congruence
closed under these axioms.

The second of these extends a signature with a directed structural
transformation axiom between two terms $\alpha$ and $\alpha'$ that have
the same mode $p$, in the same context $\psi$, relative to the prior
signature $\Sigma$.  As discussed above, these structural
transformations will be used to represent object language structural
properties such as weakening and contraction that are not invertible.
The judgement \wfsps{\psi}{\Sigma}{\alpha}{\alpha'}{p} defines these
transformations: it is the least precongruence (preorder compatible with
the term formers) closed under the axioms specified in the signature
$\Sigma$.  For example, to say that the above monoid $(\odot,1)$ is
affine, we add in $\Sigma$ a transformation axiom $(x \spr 1 : (x:\dsd{p}) \to
{\dsd{p}})$.
%% Then, using the rules in the figure, we can for example derive a
%% transformation $(x \odot y) \spr (1 \odot y) \spr y$ that, when
%% applied (contravariantly) to a sequent, will allow weakening $y$ to
%% $x \odot y$.
\ifthenelse{\boolean{short}}{}
{An alternative to including the judgement $\alpha \deq \alpha'$ would be
to present a desired equation $\alpha \deq \alpha'$ as an isomorphism,
with transformation axioms $s : \alpha \spr \alpha'$ and $s' : \alpha'
\spr \alpha$.  While this is conceptually and technically sufficient, we
have found it helpful in examples to use ``strict'' equality of context
descriptors.  This simplifies the description of some situations, though
the difference is important mainly at the level of identity of
derivations rather than provability---for example, we can make a binary
operation $\odot$ into a strict monoid, rather than adding associator
and unitor isomorphisms.
}

Because context descriptors
$\alpha$ and their equality $\alpha_1 \deq \alpha_2$ are defined prior
to the subsequent judgements, we suppress this equality by using
$\alpha$ to refer to a term-modulo-\deq---that is, we assume a
metatheory with quotient sets/types, and use meta-level equality for
object-level equality, as recently advocated by
\citet{altenkirchkaposi16qit}.  For example, because the judgement
\wfsp{\psi}{\alpha}{\beta}{p} is indexed by equivalence classes of
context descriptions, the reflexivity rule above implicitly means
$\alpha \deq \beta$ implies $\alpha \spr \beta$.
\ifthenelse{\boolean{short}}{}
{
As discussed in Section~\ref{sec:equational}, we will eventually need an
equational theory between two structural property derivations $s \deq s'
:: \wfsp{\psi}{\alpha}{\alpha'}{q}$.  Because this equational theory
does not influence provability in the sequent calculus, only identity of
proofs, we defer the details to that section.  

}
In examples, we will notate a signature declaration introducing a term
constant/function symbol by showing the function symbol applied to
variables, rather than writing the formal $\dsd{c} : p_1,\ldots,p_n \to
q$. For example, we write $x : \dsd{p}, y : \dsd{p} \vdash x \odot y :
\dsd{p}$ for $\odot : \dsd{p},\dsd{p} \to \dsd{p}$.  We also suppress
the signature $\Sigma$.

\subsection{Sequent Calculus Rules}

\begin{figure}
\begin{small}
\[
\begin{array}{l}
%% \begin{array}{llll}
%% \text{Types} & A & ::= & P \mid \F{\alpha}{\Delta} \mid \U{\alpha}{\Delta}{A} \\
%% \end{array}
%% \\ \\
\framebox{Types $A,B,C$ \quad \wftype{A}{p}}
\qquad
\infer{\wftype{P}{p}}{}
\qquad
\infer{\wftype{\F{\alpha}{\Delta}}{q}}
      {\oftp{\psi}{\alpha}{q} &
        \wfctx{\Delta}{\psi}}
\qquad
\infer{\wftype{\U{x.\alpha}{\Delta}{A}}{q}}
      {\oftp{\psi,x:q}{\alpha}{p} &
        \wfctx{\Delta}{\psi} &
        \wftype{A}{p}
      }
\\ \\
\framebox{Contexts $\Gamma,\Delta$ \quad \wfctx{\Gamma}{\psi}}
\qquad
\infer{\wfctx{\cdot}{\cdot}}{}
\qquad
\infer{\wfctx{\Gamma,x:A}{\psi,x:p}}
      {\wfctx{\Gamma}{\psi} &
        \wftype{A}{p}}
\\ \\
\framebox{\seq{\Gamma}{\alpha}{A} where $\wfctx{\Gamma}{\psi}$ and $\wftype{A}{q}$ and  $\oftp{\psi}{\alpha}{q}$}
\quad
\infer[\FL]{\seq{\Gamma,x:\F{\alpha}{\Delta},\Gamma'}{\beta}{C}}
      {\seq{\Gamma,\Gamma',\Delta}{\subst \beta {\alpha}{x}}{C}}
\quad
\infer[\FR]{\seq{\Gamma}{\beta}{\F{\alpha}{\Delta}}}
      {%% \modeof{\Gamma} \vdash \gamma : \modeof{\Delta} & 
        \beta \spr \tsubst{\alpha}{\gamma} &
        \seq{\Gamma}{\gamma}{\Delta} 
      }
%% \infer{\seq{\Gamma}{\beta}{C}}
%%       {{x}:{\F{\alpha}{\Delta}} \in \Gamma & 
%%         \oftp{\modeof{\Gamma},{x'} : {\modeof{\F{\alpha}{\Delta}}}}{\beta'}{\modeof{C}} &
%%         \beta \deq \tsubst{\beta'}{x/x'} &
%%         \seq{\Gamma,\Delta}{\subst {\beta'} {\alpha}{x'}}{C}}
\\ \\
\infer[\UL]{\seq{\Gamma}{\beta}{C}}
      {\begin{array}{llll}
          x:\U{x.\alpha}{\Delta}{A} \in \Gamma &
          \beta \spr \subst{\beta'}{\tsubst{\alpha}{\gamma}}{z} &
          \seq{\Gamma}{\gamma}{\Delta} &
          \seq{\Gamma,\tptm{z}{A}}{\beta'}{C}
       \end{array}
      }
\quad
\infer[\UR]{\seq{\Gamma}{\beta}{\U{x.\alpha}{\Delta}{A}}}
      {\seq{\Gamma,\Delta}{\subst{\alpha}{\beta}{x}}{A}}
\quad
\infer[\dsd{v}]{\seq{\Gamma}{\beta}{P}}
      {x:P \in \Gamma & \beta \spr x}
\\ \\
\framebox{\seq{\Gamma}{\gamma}{\Delta} where $\wfctx{\Gamma}{\psi}$ and $\wfctx{\Delta}{\psi'}$ and  $\oftp{\psi}{\gamma}{\psi'}$}
\qquad
\infer[\cdot]{\seq{\Gamma}{\cdot}{\cdot}}
      {}
\qquad
\infer[\_,\_]{\seq{\Gamma}{\gamma,\alpha/x}{\Delta,x:A}}
      {\seq{\Gamma}{\gamma}{\Delta} &
       \seq{\Gamma}{\alpha}{A}
      }
\end{array}
\]    
\caption{Sequent Calculus}
\label{fig:sequent}
\hrule
\end{small}
\end{figure}

For a fixed mode theory $\Sigma$, we define a second layer of judgements
in Figure~\ref{fig:sequent}.  The first judgement assigns each
proposition/type $A$ a mode $p$.  Encodings of non-modal logics will
generally only make use of one mode, while modal logics use different
modes to represent different notions of truth, such as the linear and
cartesian categories in the adjoint decomposition of linear
logic~\citep{benton94mixed,bentonwadler96adjoint} and the true/valid/lax
judgements in modal logic~\citep{pfenningdavies}.  The next judgement
assigns each context $\Gamma$ a mode context $\psi$.  Formally, we think
of contexts as ordered: we do not regard $x:A,y:B$ and $y:B,x:A$ as the
same context, though we will have an admissible exchange rule that
passes between derivations in one and the other.

The sequent judgement \seq{\Gamma}{\alpha}{A} relates a context
$\wfctx{\Gamma}{\psi}$ and a type $\wftype{A}{p}$ and context descriptor
\oftp{\psi}{\alpha}{p}, while the substitution judgement \seq{\Gamma}{\gamma}{\Delta} relates
$\wfctx{\Gamma}{\psi}$ and $\wfctx{\Delta}{\psi'}$ and
$\oftp{\psi}{\gamma}{\psi'}$. Because $\wfctx{\Gamma}{\psi}$ means that
each variable in $\Gamma$ is in $\psi$, where $x : A_i \in \Gamma$
implies $x : p_i$ in $\psi$ with \wftype{A_i}{p_i}, we think of $\Gamma$
as binding variable names both in $\alpha$ and for use in the
derivation.

\ifthenelse{\boolean{short}}{}{
As discussed in the introduction, a guiding principle is to make the
following rules admissible (see Section~\ref{sec:synprop-long} for
details), which express respect for structural transformations and
structurality-over-structurality:
\[
\begin{array}{c}
\infer[Lem~\ref{lem:respectspr}]{\seq{\Gamma}{\alpha}{A}}
      {\alpha \spr \beta &
       \seq{\Gamma}{\beta}{A}}
\qquad
\infer[Thm~\ref{thm:identity}]{\seq{\Gamma,x:A}{x}{A}}{}
\qquad
\infer[Thm~\ref{thm:cut}]{\seq{\Gamma}{\subst{\beta}{\alpha}{x}}{B}}
    {\seq{\Gamma,x:A}{\beta}{B} &
     \seq{\Gamma}{\alpha}{A}}
\\ \\
\infer[Lem~\ref{lem:weakening}]{\seq{\Gamma,y:A}{\alpha}{C}}
      {\seq{\Gamma}{\alpha}{C}}
\quad
\infer[Lem~\ref{lem:exchange}]{\seq{\Gamma,y:B,x:A}{\alpha}{C}}
      {\seq{\Gamma,x:A,y:B}{\alpha}{C}}
\qquad
\infer[Cor~\ref{cor:contraction}]{\seq{\Gamma,x:A}{\subst \alpha x y}{C}}
      {\seq{\Gamma,x:A,y:A}{\alpha}{C}}
\end{array}
\]
}

We now explain the rules for the sequent calculus; the reader may wish
to refer to the examples in Section~\ref{sec:exampleencodings} in
parallel with this abstract description.  We assume atomic propositions
$P$ are given a specified mode $p$, and state identity as a primitive
rule only for them with the \dsd{v} rule.  This says that
\seq{\Gamma,x:P}{x}{P}, and additionally composes with a structural
transformation $\beta \spr x$.  Using a structural property at a leaf of
a derivation is common in e.g. affine logic, where the derivation of
$\beta \spr x$ would use weakening to forget any additional resources
besides $x$.

Next, we consider the \F{\alpha}{\Delta} type, which ``internalizes''
the context operation $\alpha$ as a type/proposition.  Syntactically, we
view the context $\Delta = x_1:A_1,\ldots,x_n:A_n$ where
\wftype{A_i}{p_i} as binding the variables $x_i:p_i$ in $\alpha$, so for
example \F{\alpha}{x:A,y:B} and \F{\alpha[x \leftrightarrow
    x']}{x':A,y:B} are $\alpha$-equivalent types (in de Bruijn form we
would write \F{\alpha}{A_1,\ldots,A_n} and use indices in $\alpha$).
The type formation rule says that \dsd{F} moves covariantly along a mode
morphism $\alpha$, representing a ``product'' (in a loose sense) of the
types in $\Delta$ structured according to the context descriptor
$\alpha$. A typical binary instance of \dsd{F} is a multiplicative
product ($A \otimes B$ in linear logic), which, given a binary context
descriptor $\odot$ as in the introduction, is written \F{x \odot
  y}{x:A,y:B}.  A typical nullary instance is a unit (1 in linear
logic), written \F{1}{}.  A typical unary instance is the \dsd{F}
connective of adjoint logic, which for a unary context descriptor
constant $\dsd{f} : \dsd{p} \to \dsd{q}$ is written \F{\dsd{f}(x)}{x:A}.
We sometimes write \F{\dsd{f}}{A} in this case, eliding the variable
name, and similarly for a unary \dsd{U}.

The rules for our \dsd{F} connective capture a pattern common to all of
these examples.  The left $\FL$ rule says that \F{\alpha}{\Delta}
``decays'' into $\Delta$, but \emph{structuring the uses of resources in
  $\Delta$ with $\alpha$ by the substitution \subst{\beta}{\alpha}{x}}.
We assume that $\Delta$ is $\alpha$-renamed to avoid collision with
$\Gamma$ (the proof term here is a ``\dsd{split}'' that binds
variables for each position in $\Delta$).  The placement of $\Delta$ at
the right of the context is arbitrary (because we have
exchange-over-exchange), but we follow the convention that new variables
go on the right to emphasize that $\Gamma$ behaves mostly as in ordinary
cartesian logic.  The right \FR\/ rule says that you must rewrite (using
structural transformations) the context descriptor to have an $\alpha$
at the outside, with a mode substitution $\gamma$ that divides the
existing resources up between the positions in $\Delta$, and then prove
each formula in $\Delta$ using the specified resources.  We leave the
typing of $\gamma$ implicit, though there is officially a requirement
$\oftp{\psi}{\gamma}{\psi'}$ where $\wfctx{\Gamma}{\psi}$ and
$\wfctx{\Delta}{\psi'}$, as required for the second premise to be a
well-formed sequent.  Another way to understand this rule is to begin
with the ``axiomatic \FR'' instance 
$\FR^* :: {\seq{\Delta}{\alpha}{\F{\alpha}{\Delta}}}{}$
which says that there is a map from $\Delta$ to \F{\alpha}{\Delta} along
$\alpha$.  Then, in the same way that a typical right rule for
coproducts builds a precomposition into an ``axiomatic injection'' such
as $\dsd{inl} :: A \vdash A + B$, the \FR\/ rule builds a precomposition
with $\seq{\Gamma}{\gamma}{\Delta}$ and then an application of a
structural rule $\beta \spr \alpha[\gamma]$ into the ``axiomatic''
version, in order to make cut and respect for transformations
admissible.

Next, we turn to $\U{x.\alpha}{\Delta}{A}$.  As a first approximation,
if we ignore the context descriptors and structural properties,
\U{-}{\Delta}{A} behaves like $\Delta \to A$, and the \UL\/ and \UR\/
rules are an annotation of the usual structural/cartesian rules for
implication.  In a formula \U{x.\alpha}{\Delta}{A}, the context
descriptor $\alpha$ has access to the variables from $\Delta$ as well as
an extra variable $x$, whose mode is the same as the \emph{overall mode
  of \U{x.\alpha}{\Delta}{A}}, while the mode of $A$ itself is the mode
of the conclusion of $\alpha$---in terms of typing, \dsd{U} is
contravariant where \dsd{F} is covariant.  It is helpful to think of $x$
as standing for the context that will be used to prove
\U{x.\alpha}{\Delta}{A}.  For example, a typical function type $A \lolli
B$ is represented by \U{x.x \otimes y}{y:A}{B}, which says to extend the
``current context'' $x$ with a resource $y$.  In \UR, the context
descriptor $\beta$ being used to prove the \dsd{U} is substituted
\emph{for $x$} in $\alpha$ (dual to \FL, which substituted $\alpha$ into
$\beta$).  The ``axiomatic'' \UL\/ instance
$\UL^* :: {\seq{\Delta,x:\U{x.\alpha}{\Delta}{A}}{\alpha}{A}}$
says that \U{x.\alpha}{\Delta}{A} together with $\Delta$ has a map to
$A$ along $\alpha$.  (The bound $x$ in $x.\alpha$ subscript is tacitly
renamed to match the name of the assumption in the context, in the same
way that the typing rule for $\lambda x.e : \Pi x:A.B$ requires
coordination between two variables in different scopes).  The full rule
builds in precomposition with \seq{\Gamma}{\gamma}{\Delta},
postcomposition with \seq{\Gamma,z:A}{\beta'}{C}, and precomposition
with $\beta \spr \beta'[\alpha[\gamma]/z]$.

Finally, the rules for substitutions are pointwise.  In examples, we
will write the components of a substitution directly as multiple
premises of \FR\/ and \UL\/, rather than packaging them with 
$\_,\_$ and $\cdot$.

\ifthenelse{\boolean{short}}{}{
One subtle point about the $\FL$ rule is that there are two competing
principles: making the rules ``obviously'' structural-over-structural,
and reducing inessential non-determinism.  Here, we choose the later,
and treat the assumption of \F{\alpha}{\Delta} affinely, removing it
from the context when it is used.  It will turn out that the judgement
nonetheless enjoys contraction-over-contraction
(Corollary~\ref{cor:contraction}), because contraction
for negatives is built into the \UL-rule, and contraction for positives
follows from this and the fact that we can always reconstruct a positive
from what it decays to on the left (c.f. how purely positive formulas
have contraction in linear logic).
}

Additives can be added to this sequent calculus; e.g. a mode $p$ has
sums $\wftype{A_p + B_p}{p}$ if
\[
\begin{array}{c}
\infer{\seq{\Gamma}{\alpha}{A + B}}
      {\seq{\Gamma}{\alpha}{A}}
\quad
\infer{\seq{\Gamma}{\alpha}{A + B}}
      {\seq{\Gamma}{\alpha}{B}}
\quad
\infer{\seq{\Gamma,x:A+B,\Gamma'}{\beta}{C}}
      {\seq{\Gamma,\Gamma',y:A}{\subst \beta y x}{C} &
       \seq{\Gamma,\Gamma',z:B}{\subst \beta z x}{C} 
      }
%% \infer{\wftype{A \& B}{p}}
%%       {\wftype{A}{p} &
%%        \wftype{B}{p}}
%% \qquad
%% \infer{\seq{\Gamma,x:A \& B}{\alpha}{C}}
%%       {\seq{\Gamma,y:A}{\alpha[y/x]}{C}}
%% \quad
%% \infer{\seq{\Gamma}{\alpha}{A + B}}
%%       {\seq{\Gamma}{\alpha}{B}}
\end{array}
\]


\newcommand\truej[1]{#1 \,\, \dsd{true}}
\newcommand\possj[1]{#1 \,\, \dsd{poss}}
\newcommand\validj[1]{#1 \,\, \dsd{valid}}
\newcommand\crispj[1]{#1 \,\, \dsd{crisp}}
\newcommand\cohesivej[1]{#1 \,\, \dsd{coh}}

\section{Examples}
\label{sec:exampleencodings}

In this section, we give some examples of logical connectives that can
be represented by mode theories in this framework, and explain
informally why they have the desired behavior with respect to
provability.
%% An additional level to these embeddings is making the
%% equational theory of derivations of \seq{\Gamma}{\alpha}{A} match a
%% desired notion of equality of maps/morphisms, and for this it is often
%% necessary to add some additional equations between structural properties
%% $s \deq s' : \alpha \spr \beta$.

\subsection{Non-associative products}

A mode theory with one mode \dsd{m} and a constant
\[
\begin{array}{c}
\oftp{x : \dsd{m}, y : \dsd{m}}{x \odot y}{\dsd{m}}\\
\end{array}
\]
specifies a completely astructural context (no weakening, exchange,
contraction, associtivity).  If we write $A \odot B$ for \F{x \odot
  y}{x:A,y:B} we \emph{cannot}, for example, derive associativity $A
\odot (B \odot C) \vdash (A \odot B) \odot C$.

To attempt a derivation, we can (without loss of generality) begin by
applying the invertible (Lemma~\ref{lem:Finv}) \FL\/ rule twice, at
which point no further left rules are possible, so we must try to apply
\FR:

\begin{footnotesize}
\[
\infer[\FL]
{\seq{a:\F{x \odot p}{x:A,p:\F{y \odot z}{y:B,z:C}}}
  {a}
  {\F{q \odot z}{q:\F{x \odot y}{x:A,y:B},z:C}}}
{
  \infer[\FL]
        {\seq{x:A,p:\F{y \odot z}{y:B,z:C}}{x \times p}{{\F{q \odot z}{q:\F{x \odot y}{x:A,y:B},z:C}}}}
        {\infer[\FR]
          {\seq{x:A,y:B,z:C}{x \times (y \odot z)}{{\F{q \odot z}{q:\F{x \odot y}{x:A,y:B},z:C}}}}
          {\begin{array}{l}
              {x \odot (y \odot z)} \spr (q \odot z)[\alpha_1/q,\alpha_2/z] \\
              \seq{x:A,y:B,z:C}{\alpha_1}{\F{x \odot y}{x:A,y:B}}\\
              \seq{x:A,y:B,z:C}{\alpha_2}{C}\\
            \end{array}
        }}
}
\]
\end{footnotesize}

\noindent To apply \FR, we need to find a substitution for $\alpha_1/q$ and
$\alpha_2/z$ with a structural transformations as above.  In the absence of any
equational or transformation axioms, the only possible choice is $x/q, (y \odot
z)/z$, so we need to show
\[
\seq{x:A,y:B,z:C}{x}{A \odot B}
\qquad
\seq{x:A,y:B,z:C}{y \odot z}{C}
\]
This is not possible because the context is not divded correctly.  

\subsection{Ordered Products and Implications}

We extend the above mode theory with a constant $1 : \dsd{m}$ and
equations
\[
\begin{array}{c}
x \odot (y \odot z) \deq (x \odot y) \odot z\\
x \odot 1 \deq x \deq 1 \odot x
\end{array}
\]
making $(\odot,1)$ into a monoid.  This makes the context behave like
ordered logic, which has associativity but none of exchange, weakening,
and contraction---a monoidal product that is not symmetric monoidal.

We can complete the above proof of associativity of $\odot$: where we
need to find a substitution such that ${x \odot (y \odot z)} \spr (q
\odot z)[\alpha_1/q,\alpha_2/z]$, we can now choose $(x \odot y)/q,
z/z$ because
\[
{x \odot (y \odot z)} \deq {(x \odot y) \odot z} = (q \odot z)[x \odot y/q, z/z]
\]
Thus, the subgoals are
\[
\seq{x:A,y:B,z:C}{x \odot y}{A \odot B}
\qquad
\seq{x:A,y:B,z:C}{z}{C}
\]
The latter is identity-over-identity (Theorem~\ref{thm:identity}), and
the former is a further \FR\/ and then identities:
\[
\infer{\seq{x:A,y:B,z:C}{x \odot y}{\F{x' \odot y'}{x':A,y':B}}}
      { \begin{array}{l}
          x \otimes y \spr (x' \odot y')[x/x',y/y'] \\
          \seq{x:A,y:B,z:C}{x}{A} \\
          \seq{x:A,y:B,z:C}{y}{B} 
        \end{array}
      }
\]
However, we cannot prove commutativity:

\begin{small}
\[
\infer[\FL]{\seq{p:\F{x\odot y}{x:A,y:B}}{p}{\F{z\odot w}{z:B,w:A}}}
      {\infer[\FR]{\seq{x:A,y:B}{x \odot y}{\F{z\odot w}{z:B,w:A}}}
        {
            x \odot y \spr (z \odot w) [\alpha_1/z,\alpha_2/w] &
            \seq{x:A,y:B}{\alpha_1}{B} &
            \seq{x:A,y:B}{\alpha_2}{A} 
      }}
\]
\end{small}

\noindent because the only choice is $\alpha_1 = x$ and $\alpha_2 = y$, which
sends the wrong resource to each branch.  

Ordered logic has two different implications, one that adds to the left
of the context, and one that adds to the right; the expected rules are

\begin{small}
\[
\begin{array}{l}
\infer{\seql{\Gamma}{o}{ A \rightharpoonup B}}
      {\seql{\Gamma,A}{o}B}
~~
\infer{\seql{\Gamma,A \rightharpoonup B,\Delta,\Gamma'}{o}{C}}
      {\seql{\Delta}{o}{A} &
       \seql{\Gamma,B,\Gamma'}{o}{C}
      }
~~
\infer{\seql{\Gamma}{o}{A \leftharpoonup B}}
      {\seql{A,\Gamma}{o}{B}}
~~
\infer{\seql{\Gamma,\Delta,A \leftharpoonup B,\Gamma'}{o}{C}}
      {\seql{\Delta}{o}{A} &
        \seql{\Gamma,B,\Gamma'}{o}{C}
      }
\end{array}
\]
\end{small}

We represent these by 
\[
\begin{array}{ll}
A \rightharpoonup B := \U{c.c \odot x}{x:A}{B} &
A \leftharpoonup B := \U{c.x \odot c}{x:A}{B}
\end{array}
\]
These have the expected right rules, putting $x$ on the left or right of
the current context descriptor, by the substitution $\beta/c$ in \UR:
\[
\infer{\seq{\Gamma}{\beta}{\U{c.c \odot x}{x:A}{B}}}
      {\seq{\Gamma,x:A}{\beta \odot x}{B}}
\qquad
\infer{\seq{\Gamma}{\beta}{\U{c.x \odot c}{x:A}{B}}}
      {\seq{\Gamma,x:A}{x \odot \beta}{B}}
\]
The instances of \UL\/ are
\[
\begin{array}{l}
\infer{\seq{\Gamma} {\beta} {C}}
      {\begin{array}{l}
          c:\U{c.c \odot x}{x:A}{B} \in \Gamma \\
          \beta \spr \beta'[c \odot \alpha/z] \\
          \seq{\Gamma}{\alpha}{A} \\
          \seq{\Gamma,z:A}{\beta'}{C}
        \end{array}
      }
\qquad
\infer{\seq{\Gamma} {\beta} {C}}
      {\begin{array}{l}
          c:\U{c.x \odot c}{x:A}{B} \in \Gamma \\
          \beta \spr \beta'[\alpha \odot c/z] \\
          \seq{\Gamma}{\alpha}{A} \\
          \seq{\Gamma,z:A}{\beta'}{C}
       \end{array}
      }
\end{array}
\]
Suppose that $\beta$ is of the form $x_1 \odot \ldots c \ldots \odot
x_n$ for distinct variables $x_i$, and consider the rule on the left,
for $\rightharpoonup$.  Because the only structural transformations are
the associativity and unit equations, the transformation must
reassociate $\beta$ as $\beta_1 \odot (c \odot \alpha) \odot \beta_2$,
with $\beta' = \beta_1 \odot z \odot \beta_2$, for some $\beta_1$ and
$\beta_2$.  Here $\alpha$ plays the role of $\Delta$ in the ordered
logic rule---the resources used to prove $A$, which occur to the right
of the implication being eliminated.  Reading the substitution
backwards, the resources $\beta'$ used for the continuation are
``$\beta$ with $c \odot \alpha$ replaced by the result of the
implication,'' as desired.  While $c$ and any variables used in $\alpha$
are still in $\Gamma$, permission to use them has been removed from
$\beta'$---and there is no way to restore such permissions in this mode
theory.  The rule for $\leftharpoonup$ is the same, but with $\alpha$ on
the opposite side of $c$.

More formally, for an ordered logic formula built from $\odot
\leftharpoonup \rightharpoonup$ and atoms, write $A^*$ for the
translation to the above encodings, and extend this pointwise to
$\Gamma^*$ for an ordered logic context $\Gamma$.  Further, define
$\vars{x_1:A_1,\ldots,x_n:A_n} = x_1 \odot \ldots \odot x_n$.  Then the
encoding of ordered logic is adequate in the sense that
$\seql{\Gamma}{o}{A}$ iff $\seq{\Gamma^*}{\vars{\Gamma}}{A^*}$.  The
proof is very similar to the proof for affine logic discussed in
Example~\ref{sec:ex:affine}.  The analogous translation of types and
judgements and adequacy statement is used for
Examples~\ref{sec:ex:linear},\ref{sec:ex:affine},\ref{sec:ex:relevant-cartesian}.

\subsection{Linear products and implication}
\label{sec:ex:linear}

Linear logic is ordered logic with exchange, so to model this we add a
commutativity equation
\[
x \otimes y \deq y \otimes x
\]
(and switch notation from $\odot$ to $\otimes$).  For example, we can
derive {\seq{p : A \otimes B}{p}{B \otimes A}}:
\[
\infer[\FL]
      {\seq{p:\F{x\otimes y}{x:A,y:B}}{p}{\F{z\otimes w}{z:B,w:A}}}
      {\infer[\FR]{\seq{x:A,y:B}{x \otimes y}{\F{z\otimes w}{z:B,w:A}}}
        {
            x \otimes y \spr (z \otimes w) [y/z,x/w] &
            \seq{x:A,y:B}{y}{B} &
            \seq{x:A,y:B}{x}{A} 
      }}
\]
where the first premise is exactly $x \otimes y = y \otimes x$.

For this mode theory, \U{c.c \odot x}{x:A}{B} and \U{c.x \odot
  c}{x:A}{B} are equal types (because comutativity is an equation, and
types are parametrized by equivalence-classes of context descriptors),
and both represent $A \lolli B$.  

%% The linear logic $\multimap$-left rule (where contexts are implicitly
%% treated modulo exchange) is
%% \[
%% \infer{\Gamma,\Delta,A \multimap B \vdash C}
%%       {\Delta \vdash A &
%%        \Gamma,B \vdash C}
%% \]
%% We have
%% \[
%% \infer{\seq{\Gamma} {\beta} {C}}
%%       {c:\U{c.c \otimes x}{x:A}{B} \in \Gamma &
%%        \beta \deq \beta'[c \odot \alpha/z] &
%%        \seq{\Gamma}{\alpha}{A} &
%%        \seq{\Gamma,z:A}{\beta'}{C}
%%       }
%% \]

\subsection{Multi-use variables}

An $n$-use variable (see \citep{reed08namessubstructural} for example)
is like a linear variable, but instead of being used ``exactly once''
(modulo additives), it is used ``exactly $n$ times.''  In the above
work, $0$-use variables were used in an encoding of nominal techniques;
another application of $n$-use variables is static analysis of
functional programs\footnote{Andreas Abel, personal communication}
(e.g. counting how many times a variable occurs to decide whether it
will be efficient to unfold a substitution).

%% n-use functions [Wright, Momigliano]
%% • Other 0-use (“irrelevant”) functions [Pfenning, Ley-Wild]
%% • RLF [Ishtiaq, Pym]
%% • HLF
%% – Designed for statement of metatheorems for Linear LF.
%% – Does n-linear Πs above, and more (e.g. some of BI)
%% – Prototype implementation

We use the following sequent calculus rules for $n$-linear functions 
\begin{small}
\[
\infer{{0\cdot \Gamma,x:^1 P} \vdash {P}}
      {}
\qquad
\infer{\Gamma \vdash A \to^n B}
      {{\Gamma, x :^n A} \vdash {B}}
\qquad
\infer{\Gamma + f:^k A \to^n B + (nk \cdot \Delta) \vdash C}
      {\Delta \vdash A &
       {\Gamma, z :^k B} \vdash {C}}
\]
\end{small}%
\noindent where $\Gamma + \Delta$ acts pointwise by $x :^{n} A + x :^{m}
A = x :^{n+m} A$ and $n \cdot \Delta$ acts pointwise by $n \cdot x^{m} A
= x :^{nm} A$.  In the left rule, $\Gamma$ and $\Delta$ have the same
underlying variables and types (but potentially different counts), and
$f:^kA \to^n B$ abbreviates a context with the same variables and types
but $0$'s for all counts besides $f$'s.  The left rule says that if you
spend $k$ occurences of a function that takes $n$ occurences of an
argument, then you need $nk$ occurrences of whatever you use to
construct the argument, in order to get $k$ occurneces of the result.  

We can model this in the linear mode theory by using context descriptors
that are themselves non-linear:
\[
\begin{array}{rcl}
x^0 & := & 1 \\
x^{n+1} & := & x^n \otimes x \\
A \to^n B & := & \U{c.c \otimes (x^n)}{x:A}{B} \\
\end{array}
\]

This has the following instances of \UL{}{} and \UR{}: 
\[
\infer{\seq{\Gamma}{\beta}{A \to^n B}}
      {\seq{\Gamma, x:A}{\beta \otimes x^n}{B}}
\qquad
\infer{\seq{\Gamma}{\beta}{C}}
      {\begin{array}{l}
          f : \U{f.f \otimes x^n}{x : A}{B} \in \Gamma \\
          \beta \spr \beta'[f \otimes (\alpha)^n/z] \\
          \seq{\Gamma}{\alpha}{A} \\
          \seq{\Gamma, z:B}{\beta'}{C} 
       \end{array}
      }
\]
For this mode theory, the only transformations are the commutative
monoid equations, and we can commute $\beta'$ to the form $\beta''
\otimes z^k$ for some $\beta''$ not mentioning $z$ and $k$ because any context descriptor
is a polynomial of variables. Thus the premise is really of form $\beta
\deq (\beta'' \otimes z^k) [f \otimes (\alpha)^n/z]$, which is equal to
$\beta'' \otimes f^k \otimes (\alpha)^{nk}$.  Here $\beta''$
corresponds to the $\Gamma$ in the above left rule (the resources used
in the continuation, besides $z^k$) and $\alpha$ corresponds to $\Delta$.
Overall, we have $x_1:^{k_1} A_1,\ldots,x_n :^{k_n} A_n \vdash C$ iff
\seq{x_1:A_1^*,\ldots,x_n:A_n^*}{x_1^{k_1} \otimes \ldots \otimes
  x_n^{k_n}}{C^*} (where $A^*$ translates atoms to themselves and each
$A \to^n B$ as indicated above).

We can also consider an $n$-use product 
\[
A^n := \F{x^n}{x:A}
\]
as a positive type, which will decompose $A \to^n B$ as $A^n \lolli B$
(by Lemma~\ref{lem:fusion}).  This has a map \seq{p:A^n}{p}{A \otimes
  \ldots \otimes A} but not a converse map \seq{p:A \otimes \ldots \otimes
  A}{p}{A^n}.  For example,
\[
\infer {\seq{p:\F{x \otimes x}{x:A}}{p}{\F{x \otimes y}{x:A,y:A}}}
       {\infer {\seq {x:A}{x \otimes x}{\F{x \otimes y}{x:A,y:A}}}
               {x \otimes x \deq (y \otimes z)[x/y,x/z] &
                \seq{x:A}{x}{A} &
                \seq{x:A}{x}{A}}
       }
\]
the essence of which is the contraction in the substitution
$[x/y,x/z]$.  However,
\[
\infer {\seq{p:\F{y \otimes z}{y:A,z:A}}{p}{\F{x \otimes x}{x:A}}}
       {\infer {\seq {y:A,z:A}{y \otimes z}{\F{x \otimes x}{x:A}}}
               {y \otimes z \deq (x \otimes x)[?/x] &
                \ldots
               }
       }
\]
is not derivable, because there is no substitution into $x \otimes x$
that makes it equal to $y \otimes z$ for distinct $y$ and $z$.
Conceptually, we think of $A^2$ as expressing a notion of identity: it
is a \emph{single} $A$ that can be used twice, which is stronger than
having two potentially different $A$'s.
%% This might be useful for
%% applications of linear logic to imperative or concurrent programming,
%% where there is a notion of identity of memory cells or resources.  

\subsection{Affine products and implications}
\label{sec:ex:affine}

If we extend the linear logic mode theory with our first directed
structural transformation $\dsd{w} :: x \spr 1$ then we get weakening.
For example, we can define a projection
\[
\infer[\FL]{\seq{p : A \otimes B}{p}{A}}
           {\infer[Lemma~\ref{lem:respectspr}]
             {\seq{x:A,y:B}{x \otimes y}{A}}
             {\infer{x \otimes y \spr x}
                    {y \spr 1}
               &
               \infer[Theorem~\ref{thm:identity}]{\seq{x:A,y:B}{x}{A}}{}
             }}
\]

We give the full adequacy proof for this example in the appendix.
Inspecting this proof, we can see that the translation from a ``native''
sequent proof in affine logic to our framework and back is the identity
on cut-free derivations.  The other round-trip is not the identity,
because the framework allows two things that the native sequent calculus
does not.  First, the framework allows weakening at the non-invertible
rules, rather than pushing it to the leaves.  For example, we have
the following two derivations of $P,Q,R \vdash P \otimes R$.

\begin{footnotesize}
\[
\infer[\FR]
      {\seq{x:P,y:Q,z:R}{x \otimes y \otimes z}{\F{x' \otimes z'}{x':P,z':R}}}
      {x \otimes y \otimes z \spr ((x \otimes y) \otimes z) &
        \infer[\dsd{v}]
              {\seq{x:A,y:B,z:C}{x \otimes y}{C}}
              {(x \otimes y) \spr x} &
        \infer[\dsd{v}]
              {\seq{x:A,y:B,z:C}{z}{C}}
              {z \spr z}
      }
\]
\end{footnotesize}
\begin{footnotesize}
\[
\infer[\FR]
      {\seq{x:P,y:Q,z:R}{x \otimes y \otimes z}{\F{x' \otimes z'}{x':P,z':R}}}
      {x \otimes y \otimes z \spr (x \otimes z) &
        \infer[\dsd{v}]
              {\seq{x:A,y:B,z:C}{x}{C}}
              {x \spr x} &
        \infer[\dsd{v}]
              {\seq{x:A,y:B,z:C}{z}{C}}
              {z \spr z}
      }
\]
\end{footnotesize}%

\noindent The second is that a derivation may perform a left rule on a
$0$-linear (in the sense of the previous section) variable, i.e. one
that does not occur in the context descriptor.  Such variables arise
because \UL\/ ``removes a variable from the context'' by marking it as
0-use, not by actually removing it.  For this mode theory (and the other
ones we consider, besides the previous section), these left rules
produce only other 0-use variables, which ultimately cannot be used, and
can be strengthed away (see Lemma~\ref{lem:0-use-strengthening}).

\subsection{Relevant and Cartesian products and implications}
\label{sec:ex:relevant-cartesian}

Next, we consider a logic with contraction, which should result in a map
$A \vdash (A \otimes A)$.  We always have the left and right components
of the chain
\[
\begin{array}{llllllll}
A & \cong & \F{x}{x:A}  & \vdash^? & \F{x \otimes x}{x:A} & \vdash & \F{x \otimes y}{x:A, y : A}
\end{array}
\]
The left isomorphism is just \FL/\FR, while the right map was discussed
above.  So it suffices to give $\F{x}{x:A} \vdash \F{x
  \otimes x}{x:A}$.  Since \dsd{F} is covariant on structural properties
(Lemma~\ref{lem:typespr}), it suffices to add a structural transformation
$\dsd{c} :: x \spr x \otimes x$.  Then we have $A \vdash A^2 \vdash (A
\otimes A)$ but neither of the converses.

Moreover, if we have both $\dsd{w} :: x \spr 1$ and $\dsd{c} :: x \spr x
\otimes x$, then $x \otimes y$ will behave like a cartesian product in
the mode theory (with projections $x \otimes y \spr x$ and $x \otimes y
\spr y$ and pairing of $z \spr x$ and $z \spr y$ ad $z \spr x \otimes
y$), and consequently $A \otimes B$ will behave like a cartesian product
type, and $\U{c.c \otimes x}{x:A}{B}$ like the usual structural $A \to B$.
We refer to this mode theory as an
\emph{cartesian monoid} and write $(\times,\top)$ for it.

%% In this setting, we have both $A \vdash \F{x \otimes x}{x:A}$ (by
%% contraction) and $\F{x \otimes x}{x:A} \vdash A$ (by projection).  It
%% seems reasonable to make this an isomorphism, rather than just an
%% interprovability, expressing the idea that in a cartesian setting, a
%% one-use $A$ is exactly the same as a two-use $A$.  (On the other hand,
%% we do not want $A \cong (A \times A)$, because a single $A$ is not the
%% same as two potentially different $A$'s).  To accomplish this, we can
%% take contraction to be an equation $x \deq x \otimes x$ (idempotence of
%% $\otimes$) rather than a directed transformation.  This makes $A \cong
%% A^2 \dashv\vdash A \times A$.  We refer to the mode of theory for an
%% idempotent commutative monoid with a weakening transformation as an
%% \emph{idempotent cartesian monoid} and write $(\times,\top)$ for it.

\subsection{Bunched Implication (BI)} Bunched implication~\citep{ohearnpym99bunched}
has two context-forming operations $\Gamma,\Gamma'$ and
$\Gamma;\Gamma'$, along with corresponding products and implications.
Both are associative, unitial, and commutative, but $;$ has weakening
and contraction while $,$ does not.  A context is represented by a tree
such as $(x:A, y:B);(z : C, w : D)$ (considered modulo the laws), and
the notation $\Gamma[\Delta]$ is used to refer to a tree with a hole
$\Gamma[-]$ that has $\Delta$ as a subtree at the hole.  In sequent
calculus style, the rules for the product and implication corresponding
to $,$ are
\[
\begin{array}{l}
\infer{\Gamma[A * B] \vdash C}
      {\Gamma[A , B] \vdash C}
\quad
\infer{\Gamma,\Delta \vdash A * B}
      {\Gamma \vdash A &
       \Delta \vdash B}
\quad
\infer{\Gamma \vdash A \magicwand B}
      {\Gamma, A \vdash B}
\quad
\infer{\Gamma[A \magicwand B, \Delta] \vdash C}
      {\Delta \vdash A &
       \Gamma[B] \vdash C}
\end{array}
\]
There are similar rules for a product and implication for $;$ as well as
structural rules of weakening and contraction for it.

We can model BI by a mode \dsd{m} with a commutative monoid $(*,I)$ and
a cartesian monoid $(\times,\top)$.  
%% \[
%% \begin{array}{l}
%% x  : \dsd{m}, y  : \dsd{m} \vdash x \times y : \dsd{m} \\
%% \cdot \vdash \top : \dsd{m} \\
%% x  : \dsd{m}, y  : \dsd{m} \vdash x * y : \dsd{m} \\
%% \cdot \vdash \dsd{I} : \dsd{m} \\
%% \end{array}
%% \]
%% where both $(\times,\top)$ and $(*,I)$ are commutative monoids, $\times$
%% is idempotent, and $\top$ (but not $I$) is terminal ($x \spr \top$).  
We define the BI products and implications using the monoids:
\[
\begin{array}{ll}
A * B := \F{x * y}{x : A, y : B}  &
A \magicwand B := \U{c.c * x}{x : A}{B} \\
A \times B := \F{x \times y}{x : A, y : B} &
A \to B := \U{c.c \times x}{x : A}{B}\\
\end{array}
\]
A context descriptor such as $(x \times y) * (z \times w)$ captures
the ``bunched'' structure of a BI context, and substitution for a
variable models the hole-filling operation $\Gamma[\Delta]$.  The left
rule for $*$ (and similarly $\times$) acts on a leaf
\[
\infer{\seq{\Gamma,z:A*B,\Gamma'}{\beta}{C}}
      {\seq{\Gamma,\Gamma',x:A,y:B}{\subst{\beta}{x * y}{z}}{C}}
\]
and replaces the leaf where $z$ occurs in the tree $\beta$ with the
correct bunch $x*y$, The left rule for $\magicwand$ (and similarly for
$\to$)
\[
\infer{\seq{\Gamma}{\beta}{C}}
      {
        c : A \magicwand B \in \Gamma &
        \beta \spr \beta'[ c * \alpha / z] & 
        \seq{\Gamma}{\alpha}{A} &
        \seq{\Gamma,z:B}{\beta'}{C} 
      }
\]
isolates a subtree containing the implication $c$ and resources $*$'ed
with it, uses those resources to prove $A$, and then replaces the
subtree with the variable $z$ standing for the result of the
implication.

We assume the BI sequent is given as a judgement $\Gamma \vdash A$ where
$\Gamma$ is a tree and there are explicit equality premises for the
algebraic laws on bunches.  Then we define $\Gamma^*$ as an in-order
flattening of the tree into one of our contexts (e.g.  $x:A^* = x:A$ and
$(\Gamma,\Delta)^* = (\Gamma;\Delta)^*=\Gamma^*,\Delta^*$), while we
define $\vars{\Gamma}$ as a context descriptor that preserves the tree
structure (e.g. $x:A^* = x$ and $\vars{(\Gamma,\Delta)} =
\vars{\Gamma}*\vars{\Delta}$ and
$\vars{\Gamma;\Delta}=\vars{\Gamma}\times\vars{\Delta}$).  Then we have
the usual adequacy statement $\Gamma \vdash A$ iff
\seq{\Gamma^*}{\vars{\Gamma}}{A^*}.

\subsection{Adjoint decomposition of !}  
\label{sec:example:bang}

Following \citet{benton94mixed,bentonwadler96adjoint}, we decompose the
$!$ exponential of intuitionistic linear logic as the comonad of an
adjunction between ``linear'' and ``cartesian'' categories.  We start
with two modes \dsd{l} (linear) and \dsd{c} (cartesian), along with a
commutative monoid $(\otimes,1)$ on \dsd{l} and a cartesian
monoid $(\times,\top)$ on \dsd{c}.  Next, we add a context descriptor
from \dsd{c} to \dsd{l}:
\[
x : \dsd{c} \vdash \dsd{f}(x) : \dsd{l}
\]
that we think of as including a cartesian context in a linear context.
This generates types 
\[
\wftype {\F{\dsd{f}(x)}{x : A_{\dsd{c}}}}{\dsd{l}}
\qquad
\wftype {\U{x.\dsd{f}(x)}{\cdot}{A_{\dsd{l}}}}{\dsd{c}}
\]
which are adjoint $\F{\dsd{f}(x)}{x:-} \la
{\U{x.\dsd{f}(x)}{\cdot}{-}}$.  The bijection on hom-sets is defined
using \FL\/ and \FR\/ and their invertibility
(Corollary~\ref{cor:Uinv}, Lemma~\ref{lem:Finv}):
\[
\infer={\seq{p:\F{\dsd{f}(x)}{x:A}}{p}{B}}
       {\infer={\seq{x:A}{\dsd{f}(x)}{B}}
               {\seq{x:A}{x}{\U{x.\dsd{f}(x)}{\cdot}{B}}}}
\]
The comonad of the adjunction
\F{\dsd{f}(x)}{x:\U{c.\dsd{f}(c)}{\cdot}{A}} is the linear logic $!A$.

In the LNL models and sequent calculus~\citep{benton94mixed}, $F(A
\times B) \cong F(A) \otimes F(B)$ and $F(\top) \cong 1$, which we can
add to the mode theory by equations 
\[
\dsd{f}(x \times y) \deq \dsd{f}(x) \otimes \dsd{f}(y)
\qquad \dsd{f}(\top) \deq 1
\]
These equations then extend to isomorphisms using Lemma~\ref{lem:fusion}
because all of $F,\otimes,\times$ are represented by \Fsymb-types in our
framework.  These properties of \dsd{f} are necessary to prove that $!
A$ has weakening and contraction (with respect to $\otimes$) and $!A
\otimes !B \vdash !(A \otimes B)$, for example.  Omitting these
equations allows us to describe non-monoidal (or lax monoidal, if we add
only one direction) left adjoints.

In general, we translate $F(A)^* = \F{\dsd{f}(x)}{x:C^*}$ and $G(A)^* =
\U{x.\dsd{f}(x)}{\cdot}{A}$ and products and functions as usual.
Then a sequent $x_1:C_1,\ldots,x_n:C_n \vdash C$ in the cartesian
category is represented by a sequent
\seq{x_1:C_1^*,\ldots,x_n:C_n^*}{x_1 \times \ldots \times x_n}{C^*}, 
and 
a mixed sequent with cartesian and linear assumptions and a linear
conclusion  $x_1:C_1,\ldots,x_n:C_n;y_1:A_1,\ldots,y_m:A_m \vdash A$ 
by 
\seq{x_1:C_1^*,\ldots,y_1:A_1^*,\ldots}{\dsd{f}(x_1) \otimes\ldots\otimes
  \dsd{f}(x_n)  \otimes y_1 \otimes \ldots \otimes y_n}{A^*}.


%% The $ \dsd{f}(x) \otimes \dsd{f}(y) \spr \dsd{f}(x \times y)$
%% direction is used to prove the purely linear logic entailment $!A
%% \otimes !B \vdash !(A \otimes B)$, for example.

%% For example, for contraction we can begin
%% \[
%% \infer{\seq{p : ! A}{p}{! A \otimes ! A}}
%%       {\infer{\seq{{x:\U{c.\dsd{f}(c)}{\cdot}{A}}}{f(x)}{{! A \otimes ! A}}}
%%              {\begin{array}{l}
%%                  f(x) \spr (x' \otimes y') [\dsd{f}(x) / x' , \dsd{f}(x) / y'] \\
%%                  \seq{x:\U{c.\dsd{f}(c)}{\cdot}{A}}{\dsd{f}(x)}{! A} \\
%%                  \seq{x:\U{c.\dsd{f}(c)}{\cdot}{A}}{\dsd{f}(x)}{! A} 
%%                \end{array}
%%              }}
%% \]
%% and we can derive \seq{x:\U{c.\dsd{f}(c)}{\cdot}{A}}{\dsd{f}(x)}{! A}
%% by \FR\/ (it is of the ``axiomatic'' form
%% $\seq{x:C}{f(x)}{\F{\dsd{f}(x)}{x:C}}$).  The key point is that the first
%% premise, which reduces to
%% \[
%% \dsd{f}(x) \spr \dsd{f}(x) \otimes \dsd{f}(x)
%% \]
%% can be deduced as
%% \[
%% \dsd{f}(x) \deq \dsd{f}(x \times x) \deq \dsd{f}(x) \otimes \dsd{f}(x)
%% \]
%% by contraction for $\times$ \emph{if we add an axiom that \dsd{f}
%%   (strictly) preserves the monoidal product}
%% \[
%% \dsd{f}(x \times y) \deq \dsd{f}(x) \otimes \dsd{f}(y)
%% \]

%% Similarly, to get weakening we take $\dsd{f}(\top) \deq 1$.  

\subsection{Adjoint decomposition of $\Box$}  
\label{sec:example:box}

The modal S4 \Bx{}{} as in \citet{pfenningdavies} is similar to !.  We
call the two modes \dsd{t}ruth and \dsd{v}alidity and have 
cartesian monoids on both (we write $(\times,\top)$ for the \dsd{t} one
and $(\times_v,\top_v)$ for the \dsd{v} one) along with $x : \dsd{v}
\vdash \dsd{f}(x) : \dsd{t}$.  Here, following the analysis of \Bx{}{}
as a monoidal comonad~\citep{alechina+01categoricals4}, we have only lax
monoid-preservation axioms
\[
\dsd{f}(x) \times \dsd{f}(y) \spr \dsd{f}(x \times_v y) \\
\qquad
\top \spr \dsd{f}(\top_v)
\]
though the difference is only at the level of equality of
derivations.\footnote{Because the context monoids are cartesian
  products, there are always converse maps, e.g.  $\dsd{f}(x \times_v y)
  \spr \dsd{f}(x) \times \dsd{f}(y)$ defined by pairing, projection, and
  congruence.  However, in the equational theory of proofs in
  S4~\citep{pfenningdavies}, there is a section-retraction $(\Box A
  \times \Box B) \rightarrowtail \Box (A \times B) \twoheadrightarrow
  (\Box A \times \Box B)$ but not an isomorphism. If we had equalities
  above, they would generate type isomorphisms $\dsd{F}(A \times_v B)
  \cong \dsd{F}(A) \times \dsd{F}(B)$, and because the right-adjoint
  $\dsd{U}$ preserves products, we would have $\dsd{F} \dsd{U} (A \times
  B) \cong \dsd{F}(U A \times_v U B) \cong (\dsd{FU}(A) \times
  \dsd{FU}(B))$, which does not match the existing theory---though it is
  a reasonable alternative to consider.
%% I think it reduces to p : Box A =?= letbox x = p in letbox y = p in box(fst x, snd y)
}
We represent a sequent 
\[
 x_1:\validj{A_1},\ldots,x_n:\validj{A_n};y_1:\truej{B_1},\ldots \vdash \truej{C}
\]
by 
\[
\seq{x_1:\Uempty{\dsd{f}}{A_1^*},\ldots,x_n:\Uempty{\dsd{f}}{A_n^*};y_1:B_1^*,\ldots}
    {\dsd{f}(x_1) \times\ldots\times \dsd{f}(x_n) \times y_1 \times \ldots \times y_n}{C^*}
\]


\subsection{Subexponentials}

Subexponentials~\citep{danos+93subexponentials,nigammiller09subexponentials} extend linear logic with a family of
comonads $!_a A$.  All of the comonads are monoidal ($!_a A \otimes !_a
B \vdash !_a(A \otimes B)$ and $1 \vdash !_a A$), and there is a
preorder $a \le b$ such that $!_b A \vdash !_a A$.  Each $!_a$ is
allowed to have weakening and/or contraction, subject to the constraint
that when $a \le b$, $b$ must be at least as structural as $a$.

We illustrate the embedding on a specific example of the diamond
preorder generated by $\dsd i < \dsd j,\dsd k < \dsd m$.  Following
\citep[Example 4.3]{reed09adjoint}, we identify each subexponential $a$
with a mode, and have an additional mode \dsd{l} for basic linear truth,
all with commutative monoids $(\otimes_a,1_a)$.  We add context
descriptor constants \oftp{x : b}{ba(x)}{a} for each $a < b$ (so, in
this example, \dsd{mk}, \dsd{mj}, \dsd{ji}, \dsd{ki}), with an
additional \oftp{x:\dsd i}{\dsd{il}(x)}{\dsd l}.  These include each
``higher'' mode into the immediately ``lower'' ones, and the lowest ones
into \dsd{l}.  We add an equation $\dsd{ji}(\dsd{mj}(x)) \deq
\dsd{ki}(\dsd{mk}(x))$ that the diamond commutes.  Then $!_b A$ is the
comonad $\F{b{\dsd l}(x)} {x : \U{x.b\dsd{l}(x)}{\cdot}{A}}$ for the
unique $\oftp{x : b}{b{\dsd l}}{\dsd{l}}$ generated by these constants.
For example, $!_k$ is the comonad of $\oftp{x :
  \dsd{k}}{\dsd{il}(\dsd{ki}(x))}{\dsd{l}}$.

This mode theory is constructed so that every mode has a unique map to
\dsd{l}.  When $a \le b$, we have a morphism \oftp{x:b}{ab(x)}{a}, so
the morphism \oftp{x:b}{b\dsd{l}(x)}{\dsd{l}} is equal to
\oftp{x:b}{a\dsd{l}(ba(x))}{\dsd{l}}.  Thus, by Lemma~\ref{lem:fusion},
we have
\[
!_b A = \dsd{F}_{{b\dsd{l}}}(\dsd{U}_{{b\dsd{l}}} A) \cong \dsd{F}_{{a\dsd{l}}}\dsd{F}_{{ba}} \dsd{U}_{{ba}} \dsd{U}_{{a\dsd{l}}} A
\]
The map $!_b A \vdash !_a A$ can thus be defined as the conunit for the
$\dsd{F}_{{ba}} \dsd{U}_{{ba}} A \vdash A$ for the comonad in the middle.

We add equations $ba(x \otimes_b y) \deq ba(x) \otimes_a ba(y)$ and
$ba(1_b) \deq 1_a$ making each generator strictly monoidal.  This
ensures that each $!_b$ is monoidal and that $!_b A$ can be weakned or
contracted if $(\otimes_b,1_b)$ has weakening or contraction (and more
generally that \F{ba}{B} can be weakened or contracted for any $B$, not
just $\dsd{U}_{ba}{(A)}$).  Thus, we add weakening or contraction to
a particular subexponential $a$ by adding them to $(\otimes_a,1_a)$.
%% assuming only monoid axioms (not that one is cartesian), 
%% - !_b A \otimes !_b B \vdash !_b(A \otimes B) uses both directions of ba(x \otimes_b y) <=> ba(x) \otimes_a ba(y)
%% - 1 \vdash !_b(1) uses both directions of ba(1_b) \deq 1_a
%% - contraction uses f(a . b) => f(a) . f(b) direction
%% - weakening uses f(1) => 1 direction

Interestingly, when $a \le b$, it does not seem that we need a condition
that $(\otimes_b,1_b)$ has whatever structural properties
$(\otimes_a,1_a)$ has in order to get that $!_b A$ is at least as
structural as $!_a A$.  As argued above $!_b A$ factors into the form
$\dsd{F}_{a{\dsd l}}(C)$, which has whatever structural properties mode $a$ has.

%% Following \citep[Example 4.3]{reed09adjoint}, we identify each
%% subexponential $i$ with a mode.  We put a commutative monoid
%% $(\otimes_i,1_i)$ on each mode, and an additional mode \dsd{l} with
%% $(\otimes,1)$ for non-modal truth.  We assume that the relation ($i \le
%% j \cup (\forall k, \dsd{l} \le k)$ is the reflexive, transitive closure
%% of a simple graph, and add a context descriptor $\oftp{x :
%%   \dsd{j}}{\dsd{ji}(x)}{i}$ for each edge of this graph, with equations
%% $\alpha \deq \beta$ when $\alpha$ and $\beta$ both correspond to paths
%% between the same two nodes in the graph.  Then the subexponential $!_i
%% A$ is defined to be $\F{\alpha} {x : \U{x.\alpha}{\cdot}{A}}$ for the
%% $\oftp{x : i}{\alpha}{\dsd{l}}$.

\subsection{Monads}

Consider a \Dia{}{A} modality with rules in the style of
\citet{pfenningdavies}: 
\[
\infer{\Gamma \vdash \possj{A}}
      {\Gamma \vdash \truej{A}}
\qquad
\infer{\Gamma \vdash \truej{\Dia{}{A}}}
      {\Gamma \vdash \possj{A}}
\qquad
\infer{\Gamma,\truej{\Dia{}{A}} \vdash \possj{C}}
      {\truej{A} \vdash \possj{C}}
\]

We can model this using a mode theory with two modes \dsd{t} and \dsd{p}
and context descriptor \oftp{x:\dsd{t}}{\dsd{g}(x)}{\dsd{p}}, defining
the type $\Dia{}{A} :=
\U{c.\dsd{g}(c)}{\cdot}{\F{\dsd{g}(x)}{x:A}}$.  This is always a monad,
but it does not automatically have a tensorial strength, which
corresponds to the context-clearing in the left rule.

For example, if we have a monoid $(\otimes_\dsd{t},1_{\dsd t})$ on mode
\dsd{t} and try to derive
\begin{small}
\[
\infer[\UR]
      {\seq{x : A, y : \Dia{\dsd{g}}{B}}{x \otimes_{\dsd t} y}{\Dia{\dsd{g}}{(A \otimes_{\dsd t} B)}}}
      {\infer[\UL]
        {\seq{x : A, y : \Dia{\dsd{g}}{B}}{\dsd{g}(x \otimes_{\dsd t} y)}{\F{\dsd{g}}{A \otimes_{\dsd t} B}}}
        {\dsd{g}(x \otimes_{\dsd t} y) \spr \subst{\beta'}{\dsd{g}(y)}{z} &
          \seq{x:A,y : \Dia{\dsd{g}}{B},z:\F{\dsd{g}}{B}}{\beta'}{\F{\dsd{g}}{A \otimes_{\dsd t} B}}
        }}
\]
\end{small}
we are stuck, because there is no way to rewrite $\dsd{g}(x
\otimes_{\dsd t} y)$ as a term containing $\dsd{g}(y)$.  If
$(\otimes_t,1_t)$ is affine, then we can weaken away $x$ and take
$\beta' = z$---the context clearing of in the left rule---but then in
the right-hand premise we will only have access to $z$, not $x$, and
cannot complete the derivation.

In general, we translate all formulas at mode \dsd{t} and represent
\Dia{}{A} as above, and translate a sequent $\truej{A_1}, \ldots,
\truej{A_1} \vdash \truej{C}$ by
\seq{x_1:A_1^*,\ldots,x_1:A_n^*}{x_1\otimes\ldots\otimes x_n}{C^*} and a
sequent $\truej{A_1}, \ldots, \truej{A_b} \vdash \possj{C}$ by
\seq{x_1:A_1^*,\ldots,x_1:A_n^*}{\dsd{g}(x_1\otimes\ldots\otimes
  x_n)}{\F{\dsd{g}}{C^*}}.  Then the three ``native'' rules above are
\FR, \UR, and a composite of \UL\/ followed by \FL, respectively.

\newcommand\ttp[2]{#1 \otimes_{\dsd {tp}} #2}
\newcommand\tvp[2]{#1 \otimes_{\dsd {vp}} #2}

Some monads, such as the \Crc{}{A} of \citep{pfenningdavies} and those
used to encapsulate effects in functional programming are strong.  
One way to axiomatize the strength is using an asymmetric product of a
\dsd{t}-mode and \dsd{p}-mode context:
\[
\begin{array}{ll}
\oftp{x : \dsd{t}, y : \dsd{p}}{\ttp x y}{\dsd{p}}
& \dsd{g}(x \otimes_{\dsd t} y) \deq \ttp x {\dsd{g}(y)}\\
\ttp {(x \otimes_{\dsd t} y)} z \deq \ttp x {(\ttp y z)}
& \ttp {\dsd{1}} y \deq y
\end{array}
\]
The equations make this into a monoid action of the \dsd{t}-contexts on
the \dsd{p}-contexts, and allow for ``isolating'' any one $x_i$ in
$\dsd{g}(x_1 \otimes_{\dsd t} \ldots \otimes_{\dsd t} x_n)$ as the
designated variable under a \dsd{g}.  Using this (and switching notation
from \Dia{\dsd{g}}{A} to \Crc{\dsd{g}}{A}), we can prove

\begin{footnotesize}
\[
\infer[\UR]
      {\seq{x : A, y : \Crc{\dsd{g}}{B}}{x \otimes_{\dsd t} y}{\Crc{\dsd{g}}{(A \otimes_{\dsd t} B)}}}
      {\infer[\UL]
        {\seq{x : A, y : \Crc{\dsd{g}}{B}}{\dsd{g}(x \otimes_{\dsd t} y)}{\F{\dsd{g}}{A \otimes_{\dsd t} B}}}
        {\dsd{g}(x \otimes_{\dsd t} y) \spr \subst{(\ttp x z)}{\dsd{g}(y)}{z} &
          \infer[\FL]
                {\seq{x:A,y : \Crc{\dsd{g}}{B},z:\F{\dsd{g}}{B}}{\ttp x z}{\F{\dsd{g}}{A \otimes_{\dsd t} B}}}
                {\infer[\UL]
                       {\seq{x:A,y : \Crc{\dsd{g}}{B},z':B}{\ttp{x}{\dsd{g}(z')}}{\F{\dsd{g}}{A \otimes_{\dsd t} B}}}
                       { {\ttp{x}{\dsd{g}(z')}} \spr \dsd{g}(x \otimes_{\dsd t} z') & 
                         \infer[\FR]{\seq{\ldots}{x \otimes_{\dsd t} z'}{{A \otimes_{\dsd t} B}}}{}
                       }
        }}}
\]
\end{footnotesize}%


An analogous description can be given for the ``$\Box$-strong
$\Diamond$''~\citep{pfenningdavies,alechina+01categoricals4}, which has
a strength only for boxed formulas ($\Bx{} A \otimes \Dia{} B \vdash
\Dia{}(\Bx{} A \otimes B)$).  We use 3 modes \dsd{v},\dsd{t},\dsd{p} and
represent the $\Box$ as the comonad of a context descriptor
\oftp{x:\dsd{v}}{\dsd{f}(x)}{\dsd{t}} (with cartesian monoids
on \dsd{v} and \dsd{t} and \dsd{f} laxly monoidal as above), and the
$\Diamond$ as the monad of a \oftp{x:\dsd{t}}{\dsd{g}(x)}{\dsd{p}}.  We
have a mixed-mode product between \dsd{v} and \dsd{p}
\[
\begin{array}{ll}
\oftp{x : \dsd{v}, y : \dsd{p}}{\tvp x y}{\dsd{p}}
& \dsd{g}(\dsd{f}(x) \times_{\dsd t} y) \deq \tvp x {\dsd{g}(y)}\\
\ttp {(x \times_{\dsd v} y)} z \deq \ttp x {\tvp y z}
& \tvp {\dsd{1}} y \deq y
\end{array}
\]

We represent the truth-conclused sequent as in
Example~\ref{sec:example:box}, and $x_1:A_1 \dsd{valid},\ldots;y_1:B_1
\dsd{true},\ldots \vdash C \, \dsd{poss}$ by
\[
\seq{x_1:\Uempty{\dsd{f}}{A_1},\ldots,y_1:B_1,\ldots}
    {\dsd{g}(\dsd{f}(x_1) \times_{\dsd t} \dsd{f}(x_2) \times_{\dsd t} \ldots y_1 \times_{\dsd t} \ldots )}
    {\F{\dsd{g}}{C}}
\]
The left rule
\[
\infer{\Delta ; \Gamma, \truej{z:\Dia{}{A}} \vdash \possj C}
      {\Delta ; w':\truej A \vdash \possj C}
\]
that keeps the valid assumptions and discards the true ones is derivable
by
\begin{footnotesize}
\[
\infer%[\UL]
      {\seq{x_i:\Uempty{\dsd{f}}{A_i},y_i:B_i,z:\Dia{\dsd{g}}{A}}{\dsd{g}(\dsd{f}{(x_i)} \times y_i \times z)}{\F{\dsd{f}}{C}}}
      {
        \dsd{g}(\dsd{f}{(x_i)} \times y_i \times z) \spr \tvp{(x_1 \times_v \ldots x_n)} {\dsd{g}(z)} & 
        \infer%[\FL]
            {\seq{\ldots,w:\F{\dsd{g}}{A}}{(\tvp{(x_1 \times_v \ldots_v \times x_n)} {w})}{{\F{\dsd{f}}{C}}}}
            {\seq{\ldots,w':A}{(\tvp{(x_1 \times_v \ldots \times_v x_n)} {\dsd{g}(w')})}{{\F{\dsd{f}}{C}}}}
      }
\]
\end{footnotesize}
\noindent The transformation is given by weakening away $y_i$ and
using the monoidalness of \dsd{f} and the isolation equation:
\[
\begin{array}{ll}
& \dsd{g}(\dsd{f}(x_1) \times \ldots \times \dsd{f}(x_n) \times y_1 \times \ldots \times z)\\
\spr & \dsd{g}(\dsd{f}(x_1) \times \ldots \times \dsd{f}(x_n) \times z)\\
\deq & \dsd{g}(\dsd{f}(x_1 \times_v x_n) \times z)\\
\deq & \tvp {(x_1 \times_v x_n)} z
\end{array}
\]
The right-hand premise is the encoding of the premise of the rule, using
the isolation equation and monoidalness of \dsd{f} in the other
direction.  The restriction of the isolation equation to \dsd{f}
prevents keeping any additional \dsd{true}\/ variables in the premise.

\subsection{Spatial Type Theory}

The spatial type theory for cohesion~\citep{shulman15realcohesion}
(which motivated this work) has an adjoint pair $\flat \la \sharp$,
where $\flat$ is a comonad and $\sharp$ is a monad, with some additional
properties.  In the one-variable case~\citep{ls16adjoint}, we analyzed
this as arising from an idempotent comonad\footnote{There it was an
  idempotent monad; the variance of \dsd{F} and \dsd{U} has been flipped
  in paper.} in the mode theory: we have a mode \dsd{c} with a cartesian
monoid $(\times,\top)$ and a context descriptor
\oftp{x:\dsd{c}}{\dsd{r}(x)}{\dsd{c}} such that $\dsd{r}(\dsd{r}(x))
\deq \dsd{r}(x)$ and there is a directed transformation $\dsd{r}(x) \spr
x$.  Then we define $\flat A := \F{\dsd{r}}{A}$ and $\sharp A :=
\Uempty{\dsd{r}}{A}$. These are adjoint as discussed in
Example~\ref{sec:example:bang}, and the transformation gives the counit
$\F{\dsd{r}}{A} \vdash A$ and the unit $A \vdash \Uempty{\dsd{r}}{A}$ by
Lemma~\ref{lem:typespr}.  Now that we have a multi-assumptioned logic,
we can model the fact that $\flat{A}$ preserves products by the equation
equation $\dsd{r}(x \times y) \deq \dsd{r}(x) \times \dsd{r}(y)$.
Overall, we encode a simply-typed spatial type theory judgement $x_1 :
\crispj{A_1},\ldots;y_1:\cohesivej{B_1} \vdash \cohesivej{C}$ as
$\seq{x_1:A_1,\ldots,y_1:B_1,\ldots}{\dsd{r}(x_1)\times\ldots\times
  y_1\times\ldots}{C}$.  This corresponds to the following native rules:

\begin{footnotesize}
\[
\begin{array}{c}
\infer{\Delta;\Gamma \vdash C}
      {A \in \Delta &
       \Delta;\Gamma,A \vdash C}
\quad
\infer{\Delta; \Gamma \vdash {\Flat A}}
      {\Delta; \cdot \vdash {A}}
\quad
\infer{\Delta; \Gamma,\Flat{A} \vdash C}
      {\Delta,A; \Gamma \vdash C}
\quad
\infer{\Delta;\Gamma \vdash {\Sharp C}}
      {\Delta,\Gamma; \cdot \vdash C}
\quad
\infer{\Delta;\Gamma \vdash C}
      {\Sharp A \in \Delta &
        \Delta;\Gamma,A \vdash {C}}
\quad
\end{array}
\]
\end{footnotesize}%

\noindent In order, these correspond to (1) the action of the
contraction and $\dsd{r}(x) \spr x$ transformations; (2) \FR\/ with
weakening, using monoidalness of \dsd{r} in one direction; (3) \FL; (4)
\UR, using monoidalness of \dsd{r} in the other direction and
idempotence; (5) \UL, with contraction.  This provides a satisfying
explanation for the unusual features of these rules, such as promoting
all cohesive variables to crisp in \Sharp{}-right, and eliminating a
crisp \Sharp{} in \Sharp{}-left.  


\setlength{\bibsep}{-1pt} %% dirty trick: make this negative
{ \small
%% \linespread{0.70}
\bibliographystyle{abbrvnat}
\bibliography{../drl-common/cs}
}

\newpage
\clearpage


\section{Syntactic Properties}
\label{sec:synprop-long}

%% Here, we include proofs of the results from
%% Section~\ref{sec:synprop-short}.  

\subsection{Admissible Structural Rules}
We show that identity, cut, weakening, exchange, contraction, and
respect for transformations, are admissible.  We give the cases for the
rules in Figure~\ref{fig:sequent}, though the results readily extend to
additive sums and products.

Define the \emph{size} of a derivation of \seq{\Gamma}{\alpha}{A} or
\seq{\Gamma}{\gamma}{\Delta} to be the number of inference rules for
these judgements $(\dsd{v},\FL, \FR, \UL, \UR, \cdot, \_,\_)$ used in it
(i.e., the evidence that variables are in a context and the evidence for
structural transformations do not contribute to the size).  Sizes are
necessary for the cut proof, where we sometimes weaken or invert a
derivation before applying the inductive hypothesis.

\begin{lemma}[Respect for Transformations] ~ \label{lem:respectspr}
\begin{enumerate}
\item If \seq{\Gamma}{\beta}{A} and $\beta' \spr \beta$ then
  \seq{\Gamma}{\beta'}{A}, and the resulting derivation has the same
  size as the given one.
\item If \seq{\Gamma}{\gamma}{\Delta} and $\gamma' \spr \gamma$ then
  \seq{\Gamma}{\gamma'}{\Delta}, and the resulting derivation has the
  same size as the given one.
\end{enumerate}
\end{lemma}
\begin{proof}
Mutual induction on the given derivation.  The cases for \dsd{v} and
$\FR$ and $\UL$ are immediate (with no use of the inductive hypothesis)
by composing with the equality in the premise of the rule.  This does
not change the size of the derivation because the derivations of
structural transformations are ignored by the size.  The cases for $\FL$ and
$\UR$ use the inductive hypothesis, along with congruence for structural
transformations to show that $\subst{\beta}{\alpha}{x} \spr
\subst{\beta'}{\alpha}{x}$ or $\subst{\alpha}{\beta}{x} \spr
\subst{\alpha}{\beta'}{x}$.  The cases for substitutions rely on the
fact that no generating structural transformations for mode substitutions are
allowed, so if $\gamma' \spr \cdot$ then $\gamma'$ is literally $\cdot$,
and $(-,-)$ is injective (if $\gamma' \spr (\gamma_1,\alpha_2/x)$, then
$\gamma'$ is $(\gamma_1',\alpha_2'/x)$ with $\gamma_1' \spr \gamma_1$
and $\alpha_2' \spr \alpha_2$); this is enough to use the inductive
hypotheses in the cons case.
\end{proof}

\begin{lemma}[Weakening over weakening] ~ \label{lem:weakening} ~
\begin{enumerate}
\item If \seq{\Gamma,\Gamma'}{\alpha}{C} then
\seq{\Gamma,\tptm{z}{A},\Gamma'}{\alpha}{C}, and the resulting
derivation has the same size as the given one.  
\item If \seq{\Gamma,\Gamma'}{\gamma}{\Delta} then
\seq{\Gamma,\tptm{z}{A},\Gamma'}{\gamma}{\Delta}, and the resulting
derivation has the same size as the given one.  
\item If \seq{\Gamma,\Gamma''}{\alpha}{C} then
\seq{\Gamma,\Gamma',\Gamma''}{\alpha}{C}, and the resulting
derivation has the same size as the given one.  
\end{enumerate}
\end{lemma}
\begin{proof}
It is implicit that the mode morphism $\alpha$ is weakened with $z$ in
the conclusion.  Intuitively, weakening holds because the contexts
$\Gamma$ are treated like ordinary structural contexts in all of the
rules---they are fully general in every conclusion, and the premises
check membership or extend them---and because weakening holds for mode
morphisms and equalities of mode morphisms.  Formally, the first two
parts are proved by mutual induction; each case is either immediate
or follows from weakening for the mode morphisms, weakening for
transformations, and the inductive hypotheses.  The third
part is proved by induction over $\Gamma'$, repeatedly applying the
first part.  
%% The case for the hypothesis rule is immediate, because
%% $\Gamma$ may contain variables other than $x$.  The case for
%% \Fsymb-right follows from weakening for the mode morphisms, and
%% equations between mode morphisms, and the inductive hypothesis for
%% substitutions.  The case for \Fsymb-left follows from the inductive
%% hypothesis, as does the case for \Usymb-right.  
\end{proof}

\begin{lemma}[Exchange over exchange] \label{lem:exchange}
If \seq{\Gamma,x:A,y:B,\Gamma'}{\alpha}{C} then
\seq{\Gamma,y:B,x:A,\Gamma'}{\alpha}{C}, and the resulting derivation
has the same size as the given one.  (And similarly for substitutions,
and exchange can be iterated).  
\end{lemma}
\begin{proof} Analogous to weakening.  
\end{proof}

We sometimes write $\modeof{\Gamma}$ for the $\psi$ such that
\wfctx{\Gamma}{\psi} and similarly for $\modeof{A}$.

\begin{theorem}[Identity] ~ \label{thm:identity}
\begin{enumerate}
\item If $x:A \in \Gamma$ then $\seq{\Gamma}{x}{A}$.
\item If $\oftp{\modeof{\Gamma}}{\rho}{\modeof{\Delta}}$ is a
  variable-for-variable mode substitution such that $x:A \in \Delta$
  implies $\rho(x) : A \in \Gamma$, then $\seq{\Gamma}{\rho}{\Delta}$.
\end{enumerate}
\end{theorem}

\begin{proof}
The standard proof by induction on $A$ (mutually with $\Delta$) applies:
the case for atomic propositions is a rule, and for the other
connectives, apply the invertible and then non-invertible rule to reduce
the problem to the inductive hypotheses.  More specifically, identity
for $P$ is a rule.  In the case for \F{\alpha}{\Delta}, with $\Gamma =
\Gamma_1,x:\F{\alpha}{\Delta},\Gamma_2$, we reduce it to the inductive
hypothesis as follows:
\[
\infer[\FL]{\seq{\Gamma_1,x:\F{\alpha}{\Delta},\Gamma_2}{x}{\F{\alpha}{\Delta}}}
      {\infer[\FR]{\seq{\Gamma_1,\Gamma_2,\Delta}{\alpha}{\F{\alpha}{\Delta}}}
                        {\alpha \spr \tsubst{\alpha}{\vec{x/x}} &
                        \seq{\Gamma_1,\Gamma_2,\Delta}{\vec{x/x}}{\Delta}
                        }}
\]
In the second premise, the $\vec{x/x}$ substitution for each $x \in
\Delta$ is a variable-for-variable substitution, so the second part of
the inductive hypothesis applies.  
The case for \Usymb\/ is similar
\[
\infer[\UR]{\seq{\Gamma}{x}{\U{\alpha}{\Delta}{A}}}
      {\infer[\UL]{\seq{\Gamma,\Delta}{\alpha}{A}}
                        {\alpha \spr \subst{x}{\tsubst{\alpha}{\vec{x/x}}}{x} &
                        \seq{\Gamma,\Delta}{\vec{x/x}}{\Delta} &
                        \seq{\Gamma,x:A}{x}{A}
                        }}
\]

For the second part, the hypothesis of the lemma asks that every
variable in $\Delta$ is associated by $\rho$ with a variable of the same
type in $\Gamma$; this is enough to iterate the first part of the
lemma for each position in $\Delta$.  Specifically, the case where
$\Delta$ is the empty context $\cdot$ is a rule. In the case for a cons
$\Delta,y:A$, we have
\oftp{\modeof{\Gamma}}{\rho}{(\modeof{\Delta},y:\modeof{A})} which means
$\rho$ must be of the form $\rho',x/y$ where $x \in \modeof{\Gamma}$ and
$\rho'$ is a variable-for-variable substitution.  Because $\rho$ was
type-preserving, $x : A \in \Gamma$ and $\rho'$ is type-preserving, so
we obtain the result from the inductive hypotheses as follows:
\[
\infer{\seq{\Gamma}{\rho,x/y}{\Delta,y:A}}
      {\seq{\Gamma}{\rho}{\Delta} & 
       \seq{\Gamma}{x}{A}
      }
\]
\end{proof}

\begin{lemma}[Left-invertibility of \Fsymb] \label{lem:Finv}
If $\D :: \seq{\Gamma_1,x_0:\F{\alpha_0}{\Delta_0},\Gamma_2}{\beta}{C}$
and then there is a derivation $\D' ::
\seq{\Gamma_1,\Gamma_2,\Delta_0}{\subst{\beta}{\alpha_0}{x_0}}{C}$ and
$size(\D') \le size(\D)$ (and analogously for substitutions).
\end{lemma}

\begin{proof}
Intuitively, we find all of the places where \D ``splits'' $x_0$, delete
the \FL used to do the split, and reroute the variables to the ones in
the context of the result.  

Formally, we proceed by induction on \D.  We write $\Gamma$ for the
whole context $\Gamma_1,x_0:\F{\alpha_0}{\Delta_0},\Gamma_2$.

In the case for \dsd{v}, $x : P \in
\Gamma_1,x_0:\F{\alpha_0}{\Delta_0},\Gamma_2$ cannot be equal to $x_0 :
\F{\alpha_0}{\Delta_0}$ because the types conflict, so we can reapply
the \dsd{v} rule in $\Gamma_1,\Gamma_2,\Delta$.

In the case for $\FR$, we have
\[
\infer{\seq{\Gamma}{\beta}{\F{\alpha}{\Delta}}}
      {\beta \spr \tsubst{\alpha}{\gamma} &
        \seq{\Gamma}{\gamma}{\Delta} 
      }
\]
with $x_0 : \F{\alpha_0}{\Delta_0} \in \Gamma$.  By the inductive
hypothesis we get
\seq{\Gamma_1,\Gamma_2,\Delta_0}{\subst{\gamma}{\alpha_0}{x}}{\Delta}.  Because
$x_0$ is not free in $\alpha$,
$\subst{(\tsubst{\alpha}{\gamma})}{\alpha_0}{x_0} =
\tsubst{\alpha}{\subst{\gamma}{\alpha_0}{x_0}}$, so we can reapply \FR:
\[
\infer{\seq{\Gamma_1,\Gamma_2}{\subst{\beta}{\alpha_0}{x_0}}{\F{\alpha}{\Delta}}}
      {{\subst{\beta}{\alpha_0}{x_0}} \spr \tsubst{\alpha}{\subst{\gamma}{\alpha_0}{x_0}} &
        \seq{\Gamma_1,\Gamma_2,\Delta_0}{\subst{\gamma}{\alpha_0}{x}}{\Delta}
      }
\]
Both the input and the output have size 1 more than the size of their
subderivations, and the output subderivation is no bigger than the input
by the inductive hypothesis.

In the case for $\FL$
\[
\infer[\FL]{\seq{\Gamma_1',x:\F{\alpha}{\Delta},\Gamma_2'}{\beta}{C}}
      {\deduce{\seq{\Gamma_1',\Gamma_2',\Delta}{\subst \beta {\alpha}{x}}{C}}{\D}}
\]
with $\Gamma_1,x_0 : \F{\alpha_0}{\Delta_0},\Gamma_2 =
\Gamma_1',x:\F{\alpha}{\Delta},\Gamma_2'$, we distinguish cases on
whether $x = x_0$ or not.  If they are the same (i.e. we have hit a left
rule on $x_0$), then $\alpha_0 = \alpha$ and $\Delta_0 = \Delta$ and
\D\/ is the result, and the size is 1 less than the size of the input.
If they are different, then (because $x_0$ is somewhere in
$\Gamma_1',\Gamma_2'$) by the inductive hypothesis we have a derivation
\[
\D' :: {\seq{(\Gamma_1',\Gamma_2')-x_0,\Delta,\Delta_0}{\subst{\subst \beta {\alpha}{x}}{\alpha_0}{x_0}}{C}}
\]
that is no bigger than \D.  Because $x_0$ is from $\Gamma$ and not
$\Delta$, it does not occur in $\alpha$, so 
\[
{\subst{\subst \beta {\alpha}{x}}{\alpha_0}{x_0}} = 
{\subst{\subst \beta {\alpha_0}{x_0}}{\alpha}{x}}
\]
By (iterating) exchange, we get a derivation 
\[
\D'' :: {\seq{(\Gamma_1',\Gamma_2')-x_0,\Delta_0,\Delta}{\subst{\subst \beta {\alpha_0}{x_0}}{\alpha}{x}}{C}}
\]
whose size is the same as $\D'$ and so no bigger than $\D$.  Applying
$\FL$ to $\D''$ (using the fact that
$(\Gamma_1',x:\F{\alpha}{\Delta},\Gamma_2')-x_0 = \Gamma_1,\Gamma_2$)
derives $\seq{\Gamma_1,\Gamma_2}{\subst{\beta}{\alpha_0}{x_0}}{C}$, and
the size is no bigger than the size of the input.

In the case for $\UR$,
\[
\infer{\seq{\Gamma}{\beta}{\U{x.\alpha}{\Delta}{A}}}
      {\seq{\Gamma,\Delta}{\subst{\alpha}{\beta}{x}}{A}}
\]
the inductive hypothesis gives a
$\D' :: \seq{\Gamma_1,\Gamma_2,\Delta,\Delta_0}{\subst{\subst{\alpha}{\beta}{x}}{\alpha_0}{x_0}}{A}$
and (iterated) exchange gives 
$\D'' ::
\seq{\Gamma_1,\Gamma_2,\Delta_0,\Delta}{\subst{\subst{\alpha}{\beta}{x}}{\alpha_0}{x_0}}{A}$,
both no bigger than \D.  Because $x_0$ is in $\Gamma$ and not $\Delta$,
it is not free in $\alpha$, so 
\[
{\subst{\subst{\alpha}{\beta}{x}}{\alpha_0}{x_0}} = {\subst{\alpha}{\subst{\beta}{\alpha_0}{x_0}}{x}}
\]
Thus, we can derive
\[
\infer{\seq{\Gamma_1,\Gamma_2,\Delta_0}{\subst{\beta}{\alpha_0}{x_0}}{\U{x.\alpha}{\Delta}{A}}}
      {\deduce{\seq{\Gamma_1,\Gamma_2,\Delta_0,\Delta}{\subst{\alpha}{\subst{\beta}{\alpha_0}{x_0}}{x}}{A}}{\D''}}
\]

In the case for $\UL$, 
\[
\infer{\seq{\Gamma}{\beta}{C}}
      {x:\U{x.\alpha}{\Delta}{A} \in \Gamma & 
        \beta \spr \subst{\beta'}{\tsubst{\alpha}{\gamma}}{z} &
        \seq{\Gamma}{\gamma}{\Delta} &
        \seq{\Gamma,\tptm{z}{A}}{\beta'}{C}
      }
\]
we know that $x$ is different than $x_0$ because the types conflict.
The inductive hypotheses give no-bigger derivations of
\[
\seq{\Gamma_1,\Gamma_2\Delta_0}{\subst{\gamma}{\alpha_0}{x_0}}{\Delta} \qquad \seq{\Gamma_1,\Gamma_2,\tptm{z}{A},\Delta_0}{\subst{\beta'}{\alpha_0}{x_0}}{C}
\]
and the latter can be exchanged to
\[
\seq{\Gamma_1,\Gamma_2,\Delta_0,\tptm{z}{A}}{\subst{\beta'}{\alpha_0}{x_0}}{C}
\]
again without increasing the size.  Thus, we can produce
\[
\infer{\seq{\Gamma_1,\Gamma_2,\Delta_0}{\subst{\beta}{\alpha_0}{x}}{C}}
      {\begin{array}{l}
          x:\U{x.\alpha}{\Delta}{A} \in \Gamma_1,\Gamma_2,\Delta_0 \\
          {\subst{\beta}{\alpha_0}{x}} \spr \subst{{\subst{\beta'}{\alpha_0}{x_0}}}{\tsubst{\alpha}{{\subst{\gamma}{\alpha_0}{x_0}}}}{z}\\
          \seq{\Gamma_1,\Gamma_2,\Delta_0}{\subst{\gamma}{\alpha_0}{x_0}}{\Delta} \\
          \seq{\Gamma_1,\Gamma_2,\Delta_0,\tptm{z}{A}}{\subst{\beta'}{\alpha_0}{x_0}}{C}
        \end{array}
      }
\]
where the transformation is the composition of the
\subst{-}{\alpha_0}{x_0} substitution into the given transformation, and
rearranging the substitution (note that $x_0$ does not occur in
$\alpha$):
\[
\begin{array}{ll}
\subst{\beta}{\alpha_0}{x_0} & \spr
\subst{\subst{\beta'}{\tsubst{\alpha}{\gamma}}{z}}{\alpha_0}{x_0} 
= 
\subst{\subst{\beta'}{\alpha_0}{x_0}}{\subst{\tsubst{\alpha}{\gamma}}{\alpha_0}{x_0}}{z}
\\
& =
\subst{\subst{\beta'}{\alpha_0}{x_0}}{\tsubst{\alpha}{\subst{\gamma}{\alpha_0}{x_0}}}{z} 
\end{array}
\]

The case for $\cdot$ is immediate.  The case for $\_,\_$ follows from
the two inductive hypotheses, because
$\subst{(\gamma,\alpha/x)}{\alpha_0}{x_0} =
{(\subst{\gamma}{\alpha_0}{x_0},\subst{\alpha}{\alpha_0}{x_0}/x)}$.
\end{proof}


\begin{theorem}[Cut] ~ \label{thm:cut}
\begin{enumerate} 
\item  If $\seq{\Gamma,\Gamma'}{\alpha_0}{A_0}$ and $\seq{\Gamma,x_0:A_0,\Gamma'}{\beta}{B}$ 
then $\seq{\Gamma,\Gamma'}{\beta[\alpha_0/x_0]}{B}$ 
\item If $\seq{\Gamma,\Gamma'}{\alpha_0}{A_0}$ and $\seq{\Gamma,x_0:A_0,\Gamma'}{\gamma}{\Delta}$ 
then $\seq{\Gamma,\Gamma'}{\gamma[\alpha_0/x_0]}{\Delta}$ 
\item If $\seq{\Gamma}{\gamma}{\Delta}$ and 
\seq{\Gamma,\Delta}{\beta}{C}
then \seq{\Gamma}{\tsubst{\beta}{\gamma}}{C}.  
\end{enumerate}
\end{theorem}

\begin{proof}
The induction ordering is the usual one: First the cut formula, and then
the sizes of size of the two derivations.  More specifically, any part
can call another with a smaller cut formula ($A_0$ for part 1 and part
2, $\Delta$ for part 3).  Additionally, part 1 and part 2 call
themselves and each other with the same cut formula and smaller $\D$ or
$\E$.

For part part 1, there are 5 rules in the sequent calculus, so 25 pairs
of final rules.

\begin{itemize}
\item (5 pairs) Any rule and identity
\[
\deduce{\seq{\Gamma,\Gamma'}{\alpha_0}{A_0}}{\D} 
\qquad
\infer{\seq{\Gamma,x:A,\Gamma'}{\beta}{Q}}
      {z:Q \in (\Gamma,x_0:A_0,\Gamma') &
        \deduce{\beta \spr z}{s}}
\]
There two subcases, depending on whether the cut variable is $z$ or not.
If $z$ is $x_0$ and $A_0$ is $Q$, then \D\/ derives
\seq{\Gamma,\Gamma'}{\alpha_0}{Q} and we want a derivation of
\seq{\Gamma,\Gamma'}{\subst{\beta}{\alpha_0}{z}}{Q}.  By congruence on
$s$, $\subst{\beta}{\alpha_0}{z} \spr \subst{z}{\alpha_0}{z}$, so
Lemma~\ref{lem:respectspr} gives the result.  If $z$ is not $x_0$,
then $z:Q \in \Gamma,\Gamma'$.  We want a derivation of
\seq{\Gamma,\Gamma'}{\subst{\beta}{\alpha_0}{x_0}}{Q}, and substituting
into $s$ gives $\subst{\beta}{\alpha_0}{x_0} \spr z$ (because $z \neq
x_0$), so we can derive
\[
\infer{\seq{\Gamma,\Gamma'}{\subst \beta {\alpha_0}{x_0}}{Q}}
      {z:Q \in (\Gamma,\Gamma') &
        {\subst{\beta}{\alpha_0}{x_0} \spr z}}
\]

\item (5 pairs) Any rule and $\FR$ (right-commutative)
\[
\deduce{\seq{\Gamma,\Gamma'}{\alpha_0}{A_0}}{\D} \qquad
\infer{\seq{\Gamma,x_0:A_0,\Gamma'}{\beta}{\F{\alpha}{\Delta}}}
      {\beta \spr \tsubst{\alpha}{\gamma} &
        \deduce{\seq{\Gamma,x_0:A_0,\Gamma'}{\gamma}{\Delta}}{\E}
      }
\]
By the inductive hypothesis, cutting into \D\/ into \E\/ gives
\seq{\Gamma,\Gamma'}{\subst{\gamma}{\alpha_0}{x_0}}{\Delta}.  By
congruence, $\subst{\beta}{\alpha_0}{x_0} \spr
\subst{\tsubst{\alpha}{\gamma}}{\alpha_0}{x_0}$.  Since $\gamma$ is a
total substitution for all variables in \modeof{\Delta},
$\subst{\tsubst{\alpha}{\gamma}}{\alpha_0}{x_0} =
\tsubst{\alpha}{\subst{\gamma}{\alpha_0}{x_0}}$, so
$\subst{\beta}{\alpha_0}{x_0} \spr
\tsubst{\alpha}{\subst{\gamma}{\alpha_0}{x_0}}$.  Thus we can reapply
the $\FR$ rule to get
\seq{\Gamma,\Gamma'}{\subst{\beta}{\alpha_0}{x_0}}{\F{\alpha}{\Delta}}.

\item (5 pairs) Any rule and $\UR$ (right-commutative).    
\[
\deduce{\seq{\Gamma,\Gamma'}{\alpha_0}{A_0}}{\D} \qquad
\infer{\seq{\Gamma,x_0:A_0,\Gamma'}{\beta}{\U{x.\alpha}{\Delta}{A}}}
      {\deduce{\seq{\Gamma,x_0:A_0,\Gamma',\Delta}{\subst{\alpha}{\beta}{x}}{A}}{\E}}
\]
The inductive cut of \D\/ into \E\/ gives 
\[
\seq{\Gamma,\Gamma',\Delta}{\subst{\subst{\alpha}{\beta}{x}}{\alpha_0}{x_0}}{A}
\]
Because the variables from $\modeof{\Gamma},\modeof{\Gamma'}$ occur only
in $\beta$, not in $\alpha$, this substitution equals 
{\subst{\alpha}{\subst{\beta}{\alpha_0}{x_0}}{x}} so reapplying the
$\UR$ rule
derives 
{\seq{\Gamma,\Gamma'}{\subst{\beta}{\alpha_0}{x_0}}{\U{x.\alpha}{\Delta}{A}}}.   

\item (2 additional pairs, plus 3 overlapping with above) $\FL$ and
  any rule (left commutative).  

There is one subtlety in this case.  The usual strategy for a left rule
against an arbitrary \E\/ is to push $\E$ into the ``continuation'' of
the left rule.  However, as discussed above, our left rule for \Fsymb\/
eagerly inverts \emph{all} occurrences of $x$, while $\E$ itself also has
$x$ in scope.  Thus, we use Lemma~\ref{lem:Finv} to pull the
left-inversion to the bottom of \E, and then push that into \D.  On
proof terms, this corresponds to making all references to $x$ in \E\/
instead refer to the results of the ``split'' at the bottom of $\D$.
%% This subtlety could be avoided by building contraction into $\FL$, as
%% discussed above.

Formally, we have
\[
\begin{array}{c}
\infer{\seq{\Gamma,\Gamma'}{\alpha_0}{A_0}}
      {{x}:{\F{\alpha}{\Delta}} \in \Gamma,\Gamma' &
        \deduce{\seq{((\Gamma,\Gamma')-x),\Delta}{\subst {\alpha_0} {\alpha}{x}}{A_0}}{\D}}
\\ \\
\deduce{\seq{\Gamma,x_0:A_0,\Gamma'}{\beta}{C}}{\E}
\end{array}
\]

By left invertibility on \E, we obtain (note that $x \neq x_0$ because
$x_0$ only in scope in \E\/, not \D) a derivation $\E'$ of
{\seq{(\Gamma,x:A_0,\Gamma')-x,\Delta}{\subst{\beta}{\alpha}{x}}{C}}
that is no bigger than $\E$.  Because the cut formula is the same, and
$\E'$ is no bigger than \E\/, and \D\/ is smaller than the given
derivation of $A_0$, we can apply the inductive hypothesis to cut $\D$
and $\E'$ to get
\[
{\seq{(\Gamma,\Gamma')-x,\Delta}{\subst{\subst{\beta}{\alpha}{x}}{\subst{\alpha_0}{\alpha}{x}}{x_0}}{C}}.
\]
Commuting substitutions gives
\[
{\subst{{\subst{\beta}{\alpha}{x}}}{\subst{\alpha_0}{\alpha}{x}}{x_0}} = \subst {\beta[\alpha_0/x_0]}{\alpha}{x}
\]
so we can reapply $\FL$ 
\[
\infer{\seq{\Gamma,\Gamma'}{\beta[\alpha_0/x_0]}{C}}
      {\seq{((\Gamma,\Gamma')-x),\Delta}{\subst {(\beta[\alpha_0/x_0])} {\alpha}{x}}{C}}
\]

\item (2 additional pairs, plus 3 overlapping with above) $\UL$ and any rule (left commutative)
In this case, $x:\U{\alpha}{\Delta}{A} \in \Gamma,\Gamma'$ and
we have
\[
\begin{array}{c}
\infer{\seq{\Gamma,\Gamma'}{\alpha_0}{A_0}}
      {\deduce{\alpha_0 \spr \subst{\alpha_0'}{\tsubst{\alpha}{\gamma}}{z}}{s} &
       \deduce{\seq{\Gamma,\Gamma'}{\gamma}{\Delta}}{\D_1} &
       \deduce{\seq{\Gamma,\Gamma',z:A}{\alpha_0'}{A_0}}{\D_2}
      }
\\ \\
\deduce{\seq{\Gamma,x_0:A_0,\Gamma'}{\beta}{B}}{\E}
\end{array}
\]

Weakening \E\/ with $z$ and then cutting $\D_2$ and $\E$ by the inductive
hypothesis (which applies because $\D_2$ is smaller and weakening does
not change the size) gives
\[
\deduce{\seq{\Gamma,\Gamma',z:A}{\subst{\beta}{\alpha_0'}{x_0}}{B}}{\D_2'}
\]
Thus, we have the first, third, and fourth premises of
\[
\infer{\seq{\Gamma,\Gamma'}{\subst{\beta}{\alpha_0}{x_0}}{A_0}}
      {\begin{array}{l}
          x:\U{\alpha}{\Delta}{A} \in \Gamma,\Gamma' \\
          {\subst{\beta}{\alpha_0}{x_0}} \spr \subst{\subst{\beta}{\alpha_0'}{x_0}}{\tsubst{\alpha}{\gamma}}{z} \\
       {\seq{\Gamma,\Gamma'}{\gamma}{\Delta}} \\
       {\seq{\Gamma,\Gamma',z:A}{\subst{\beta}{\alpha_0'}{x_0}}{B}}
        \end{array}
      }
\]
The transformation premise is
\[
     {\subst{\beta}{\alpha_0}{x_0}} 
\spr \subst{\beta}{\subst{\alpha_0'}{\tsubst{\alpha}{\gamma}}{z}}{x0} = \subst{\subst{\beta}{\alpha_0'}{x_0}}{\tsubst{\alpha}{\gamma}}{z}
\]
where the first step is by congruence with $\beta$ on $s$, and the
second is by properties of substitution ($z$ is not free in $\beta$).

\item (3 pairs) Right rule or identity and $\FL$ (principal or
  right-commutative).  

Suppose the right-hand derivation ends with $\FL$, and the left-hand
derivation is either a right rule or identity (\dsd{v}) (the cases for
left-rules were covered above).  

We distinguish cases on whether the \FL\/ case-analyzes $x_0$ or not.
If it does, then, because $A_0$ is $\F{\alpha}{\Delta}$, the left-hand
derivation must be \FR, and we have a principal cut
\[
\infer{\seq{\Gamma,\Gamma'}{\alpha_0}{\F{\alpha}{\Delta}}}
      {  
        \deduce{\alpha_0 \spr \tsubst{\alpha}{\gamma}}{s} &
        \deduce{\seq{\Gamma,\Gamma'}{\gamma}{\Delta}}{\D}
      }
\qquad
\infer{\seq{\Gamma,x_0:\F{\alpha}{\Delta},\Gamma'}{\beta}{C}}
      {\deduce{\seq{\Gamma,\Gamma',\Delta}{\subst{\beta}{\alpha}{x_0}}{C}}
              {\E}}
\]
Using the inductive hypothesis part 3 to cut $\D$ and $\E$ ($\Delta$ is
a subformula of the original cut formula \F{\alpha}{\Delta}) gives
\[
\seq{\Gamma,\Gamma'}{\tsubst{\subst{\beta}{\alpha}{x_0}}{\gamma}}{C}
\]
By congruence on $s$ and because $\gamma$ substitutes only for
variables in $\modeof {\Delta}$,
\[
\subst{\beta}{\alpha_0}{x_0} \spr
{\subst{\beta}{\tsubst \alpha \gamma}{x_0}} =
{\tsubst{\subst{\beta}{\alpha}{x_0}}{\gamma}} 
\]
So applying Lemma~\ref{lem:respectspr} gives 
\seq{\Gamma,\Gamma'}{\subst{\beta}{\alpha_0}{x_0}}{C}.  

If the left rule is not on the cut variable, then we have
\[
\deduce{\seq{\Gamma,\Gamma'}{\alpha_0}{A_0}}{\D}
\quad
\infer{\seq{\Gamma,x_0:A_0,\Gamma'}{\beta}{C}}
      { x : \F{\alpha}{\Delta} \in \Gamma,\Gamma' &
        \deduce{\seq{((\Gamma,x_0:A_0,\Gamma')-x),\Delta}{\subst{\beta}{\alpha}{x}}{C}}{\E}}
\]

We are going to commute $\D$ under \FL\/ on $x$, so need to reroute uses
of $x$ to the bottom by the left-inversion lemma, which gives
\[
\D' :: {\seq{((\Gamma,\Gamma')-x),\Delta}{\subst{\alpha_0}{\alpha}{x}}{A_0}}
\]
and $\D'$ is no bigger than \D.

Cutting $\D'$ and $\E$ by the inductive hypothesis gives
\[
\seq{((\Gamma,\Gamma')-x),\Delta}{\subst{\subst{\beta}{\alpha}{x}}{\subst{\alpha_0}{\alpha}{x}}{x_0}}{C}
\]
Because $x_0$ is not free in $\alpha$, 
\[
  {\subst{\subst{\beta}{\alpha}{x}}{\subst{\alpha_0}{\alpha}{x}}{x_0}}
= {\subst{\subst{\beta}{\alpha_0}{x_0}}{\alpha}{x}}
\]
so we can apply \FL
\[
\infer{\seq{\Gamma,\Gamma'}{\subst{\beta}{\alpha_0}{x_0}}{C}}
      {\seq{(\Gamma,\Gamma'-x)}{\subst{\subst{\beta}{\alpha_0}{x_0}}{\alpha}{x}}{C}}
\]
%% \seq{((\Gamma,\Gamma')-x),\Delta}{\subst{\subst{\beta}{\alpha}{x}}{\alpha_0}{x_0}}{C}
%% \]

\item (3 pairs) Right rule or identity and $\UL$ (principal or
  right-commutative).

If $x_0$ is the variable used by \UL\/ in the right-hand derivation, then
the left-hand derivation must have been derived by $\UR$, and we have
\[
\begin{array}{l}
\D \quad = \quad \infer{\seq{\Gamma,\Gamma'}{\alpha_0}{\U{x_0.\alpha}{\Delta}{A}}}
   {  
     \deduce{\seq{\Gamma,\Gamma',\Delta}{\subst \alpha {\alpha_0}{x_0}}{A}}{\D'}
   }
\\ \\
\infer{\seq{\Gamma,x_0:\U{x_0.\alpha}{\Delta}{A},\Gamma'}{\beta}{C}}
      {
        \begin{array}{l}
        s :: \beta \spr \subst{\beta'}{\tsubst{\alpha}{\gamma}}{z} \\
        {\E_1 :: \seq{\Gamma,x_0:{\U{x_0.\alpha}{\Delta}{A}},\Gamma'}{\gamma}{\Delta}}\\
        {\E_2 :: \seq{\Gamma,x_0:{\U{x_0.\alpha}{\Delta}{A}},\Gamma',\tptm{z}{A}}{\beta'}{C}}
        \end{array}
      }
\end{array}
\]
First, cutting the original \D\/ and the smaller $\E_1$ and $\E_2$ gives 
\[
\deduce{{\seq{\Gamma,\Gamma'}{\subst{\gamma}{\alpha_0}{x_0}}{\Delta}}}{\E_1'}
\qquad 
\deduce{{\seq{\Gamma,\Gamma',\tptm{z}{A}}{\subst{\beta'}{\alpha_0}{x_0}}{C}}}{\E_2'}
\]
Cutting $\E_1'$ \emph{into} $\D'$ (the derivations have switched places,
so are not necessarily smaller, but the cut formula $\Delta$ is a
subformula of $\U{x_0.\alpha}{\Delta}{A}$) gives
\[
\deduce
{\seq{\Gamma,\Gamma'}{\tsubst{\alpha}{\subst{\gamma}{\alpha_0}{x}}}{A}} {\D_1'}
\]
Cutting $\D_1'$ into $\E_2'$ gives 
\[
\seq{\Gamma,\Gamma'}{\subst{\subst{\beta'}{\alpha_0}{x_0}}{\subst{\alpha}{\alpha_0}{x_0}}{z}}{A}
\]
But by using $s$ and commuting substitutions we have 
\[
\subst{\beta}{\alpha_0}{x_0} \spr
\subst{(\subst{\beta'}{\tsubst{\alpha}{\gamma}}{z})}{\alpha_0}{x_0} = 
{\subst{\subst{\beta'}{\alpha_0}{x_0}}{\tsubst{\alpha}{\subst{\gamma}{\alpha_0}{x_0}}}{z}}
\]
so Lemma~\ref{lem:respectspr} gives the result.  

On the other hand, if the \UL\/ is not on $x_0$, then we have
\[
\deduce{\seq{\Gamma,\Gamma'}{\alpha_0}{A_0}}
       {
         \D
       }
\quad
\infer{\seq{\Gamma,x_0:A_0,\Gamma'}{\beta}{C}}
      {
        \begin{array}{l}
          x : \U{x.\alpha}{\Delta}{A} \in \Gamma,\Gamma' \\
          \beta \spr \subst{\beta'}{\tsubst{\alpha}{\gamma}}{z} \\
          \seq{\Gamma,x_0:A_0,\Gamma'}{\gamma}{\Delta} \\
          \seq{\Gamma,x_0:A_0,\Gamma',\tptm{z}{A}}{\beta'}{C}
        \end{array}
      }
\]

By the inductive hypotheses we get 
\[
\seq{\Gamma,\Gamma'}{\subst{\gamma}{\alpha_0}{x_0}}{\Delta}
\qquad
\seq{\Gamma,\Gamma',z:A}{\subst{\beta'}{\alpha_0}{x_0}}{C}
\]
so we can derive
\[
\infer{\seq{\Gamma,x_0:A_0,\Gamma'}{\subst{\beta}{\alpha_0}{x_0}}{C}}
      {
        \begin{array}{l}
          x : \U{x.\alpha}{\Delta}{A} \in \Gamma,\Gamma' \\
          {\subst{\beta}{\alpha_0}{x_0}} \spr \subst{\subst{\beta'}{\alpha_0}{x_0}}{\tsubst{\alpha}{\subst{\gamma}{\alpha_0}{x_0}}}{z} \\
          \seq{\Gamma,\Gamma'}{\subst{\gamma}{\alpha_0}{x_0}}{\Delta} \\
          \seq{\Gamma,\Gamma',\tptm{z}{A}}{\subst{\beta'}{\alpha_0}{x_0}}{C}
        \end{array}
      }
\]
For the second premise, we get
\[
\subst{\beta}{\alpha_0}{x_0} \spr
\subst{\subst{\beta'}{\tsubst{\alpha}{\gamma}}{z}}{\alpha_0}{x_0}
\]
by congruence on the assumed transformation, and then commute substitutions.  

\end{itemize}

For part 2, there are just two right-commutative cases: For
\[
\seq{\Gamma,\Gamma'}{\alpha_0}{A_0}
\qquad
\seq{\Gamma,x_0:A_0,\Gamma'}{\cdot}{\cdot}
\]
we also have $\subst \cdot {\alpha_0}{x_0} = \cdot$ and
\seq{\Gamma,\Gamma'}{\cdot}{\cdot}.  For
\[
\seq{\Gamma,\Gamma'}{\alpha_0}{A_0}
\qquad
\infer{\seq{\Gamma,x_0:A_0,\Gamma'}{\gamma,\alpha/x}{\Delta,x:A}}
      {\seq{\Gamma,x_0:A_0,\Gamma'}{\gamma}{\Delta} &
        \seq{\Gamma,x_0:A_0,\Gamma'}{\alpha}{A}
      }
\]
we have $\subst{(\gamma,\alpha/x)}{\alpha_0}{x_0} 
= (\subst{\gamma}{\alpha_0}{x_0},\subst{\alpha}{\alpha_0}{x_0})$, and
 \[
\seq{\Gamma,\Gamma'}{\subst{\gamma}{\alpha_0}{x_0}}{\Delta} \quad
\seq{\Gamma,\Gamma'}{\subst{\alpha}{\alpha_0}{x_0}}{A}
\]
by the inductive hypotheses, so we can reapply the rule to conclude
\seq{\Gamma,\Gamma'}{\subst{(\gamma,\alpha/x)}{\alpha_0}{x_0}}{\Delta,x:A}.

For part 3, we induct on $\Delta$, reducing a simultaneous cut to
iterated single-variable cuts.  If $\Delta$ is empty, then we have
\[
\seq{\Gamma}{\cdot}{\cdot}
\qquad
\deduce{\seq{\Gamma,\cdot}{\beta}{C}}{\E}
\]
and we return \E, noting that $\tsubst{\beta}{\cdot} = \beta$.  Otherwise
we have
\[
\infer{\seq{\Gamma}{\gamma,\alpha/x}{\Delta,x:A}}
      {\deduce{\seq{\Gamma}{\gamma}{\Delta}}{\D_1} &
        \deduce{\seq{\Gamma}{\alpha}{A}}{\D_2}}
\qquad
\deduce{\seq{\Gamma,\Delta,x:A}{\beta}{C}}{\E}
\]
Using the inductive hypothesis to cut $\D_2$ into $\E$ ($A$ is smaller
than $\Delta,x:A$) gives
\[
\deduce{\seq{\Gamma,\Delta}{\subst{\beta}{\alpha}{x}}{C}}
       {\E'}
\]
Using the inductive hypothesis to cut $\D_1$ into $\E'$ ($\Delta$ is
smaller than $\Delta,x:A$) gives
\[
\seq{\Gamma}{\tsubst{\subst{\beta}{\alpha}{x}}{\gamma}}{C}
\]
Because $\gamma$ substitutes for $\hat \Delta$ (and not $\hat \Gamma$,
the free variables of $\alpha$),
\[
\tsubst{\beta}{\gamma,\alpha/x}
= {\tsubst{\subst{\beta}{\alpha}{x}}{\gamma}}
\]
\end{proof}

Using this, we have
\begin{corollary}[Contraction over contraction] \label{cor:contraction}
\item If
\seq{\Gamma,x:A,y:A,\Gamma'}{\alpha}{C}
then
\seq{\Gamma,z:A,\Gamma'}{\tsubst \alpha {z/x,z/y}}{C}
\end{corollary}

\begin{proof}  Contraction can be shown by cutting with a renaming substitution.
The mode substitution $z/x,z/y$ is a variable-for-variable substitution,
and is type-preserving between ${x:A,y:A}$ and ${\Gamma,z:A,\Gamma'}$.
Therefore, by identity (part 2),
\seq{\Gamma,z:A,\Gamma'}{z/x,z/y}{x:A,y:A}.  Thus, by cut (part 2), we
obtain the result.
\end{proof}

\begin{corollary}[Right-invertibility of \Usymb] \label{cor:Uinv}
If $\seq{\Gamma}{\beta}{\U{x.\alpha}{\Delta}{A}}$ then 
{\seq{\Gamma,\Delta}{\subst{\alpha}{\beta}{x}}{A}}.
\end{corollary}

\begin{proof}
$\UL$ with identities in both premises gives a derivation
\[
\infer{\seq{\Gamma,\Delta,x:{\U{x.\alpha}{\Delta}{A}}}{\alpha}{A}}
      {
        \alpha = z[\alpha[\vec{x/x}]/z] & 
        \seq{\Gamma,\Delta}{\vec{x/x}}{\Delta} &
        \seq{\Gamma,\Delta,x:{\U{x.\alpha}{\Delta}{A}},z:A}{z}{A}
      }
\]
Weakening the assumed derivation to 
\seq{\Gamma,\Delta}{\beta}{\U{x.\alpha}{\Delta}{A}}
and then cutting for $x$ in the above gives the result:

\[
\infer{\seq{\Gamma,\Delta}{\subst{\alpha}{\beta}{x}}{A}}
      {\seq{\Gamma,\Delta}{\beta}{\U{x.\alpha}{\Delta}{A}} & 
       \seq{\Gamma,\Delta,x:{\U{x.\alpha}{\Delta}{A}}}{\alpha}{A}
      }
\]
\end{proof}

\subsection{General Properties}

We give a couple of general constructions that were used in several
examples above.

The following ``fusion'' lemmas (which are isomorphisms, not just
interprovabilities) relate $\Fsymb$ and $\Usymb$.  Special cases
include: $A \times (B \times C)$ is isomorphic to a primitive triple
product $\{x:A,y:B,z:C\}$; currying; and associativity of $n$-ary
functions ($A_1,\ldots,A_n \to (B_1,\ldots,B_m \to C)$ is isomorphic to
$A_1,\ldots,A_n,B_1,\ldots,B_m \to C$).  The derivations are in
Figure~\ref{fig:fusion}.  We adopt the convention that an unlabeled leaf
$\alpha \spr \beta$ is proved by equality of context descriptors, and an
unlabeled sequent leaf is proved by identity
(Theorem~\ref{thm:identity}).  

\begin{lemma}[Fusion] ~ \label{lem:fusion}
\begin{enumerate} 

\item $\F{\alpha}{\Delta,x:\F{\beta}{\Delta'},\Delta''} \dashv \vdash
  \F{\subst{\alpha}{\beta}{x}}{\Delta,\Delta',\Delta''}$

\item $\U{x.\alpha}{\Delta,y:\F{\beta}{\Delta'},\Delta''}{A} \dashv \vdash
  \U{x.\subst{\alpha}{\beta}{y}}{\Delta,\Delta',\Delta''}{A}$

\item 
$\U{x.\alpha}{\Delta}{\U{y.\beta}{\Delta'}{A}} \dashv \vdash
 \U{x.\subst{\beta}{\alpha}{y}}{\Delta,\Delta'}{A}$

\end{enumerate}
\end{lemma}

\begin{figure}
\begin{small}
\[
\infer[\FL]{
  \seq{z:\F{\alpha}{\Delta,x:\F{\beta}{\Delta'},\Delta''}}
      {z}
      {\F{\subst{\alpha}{\beta}{x}}{\Delta,\Delta',\Delta''}}
}
{
  \infer[\FL]{\seq{\Delta,x:\F{\beta}{\Delta'},\Delta''}{\alpha}{\F{\subst{\alpha}{\beta}{x}}{\Delta,\Delta',\Delta''}}}
        {
          \infer[\FL]{\seq{\Delta,\Delta'',\Delta'}{\subst{\alpha}{\beta}{x}}{\F{\subst{\alpha}{\beta}{x}}{\Delta,\Delta',\Delta''}}}
                {\subst{\alpha}{\beta}{x} \spr \tsubst{\subst{\alpha}{\beta}{x}}{\vec{z/z}} &
                 \seq{\Delta,\Delta'',\Delta'}{\vec{z/z}}{\Delta,\Delta',\Delta''}
                }
        }
}
\]

\[
\infer{
  \seq{z:{\F{\subst{\alpha}{\beta}{x}}{\Delta,\Delta',\Delta''}}}
      {z}
      {\F{\alpha}{\Delta,x:\F{\beta}{\Delta'},\Delta''}}
}
{  
\infer[\FL]{\seq{\Delta,\Delta',\Delta''}
           {\subst{\alpha}{\beta}{x}}
           {\F{\alpha}{\Delta,x:\F{\beta}{\Delta'},\Delta''}}}
      {\alpha[\beta/x] \spr \alpha[\vec{y/y},\beta/x,\vec{z/z}] &
        \infer[\FR]{\seq{\Delta,\Delta',\Delta''}{\vec{y/y},\beta/x,\vec{z/z}} {\Delta,x:\F{\beta}{\Delta'},\Delta''}}
              {\seq{\Delta,\Delta',\Delta''}{\vec{y/y}}{\Delta} & 
               \infer[\FR]{\seq{\Delta,\Delta',\Delta''}{\beta}{\F{\beta}{\Delta'}}}
                     {\beta \spr \beta[\vec{w/w}] & \seq{\Delta,\Delta',\Delta''}{\vec{w/w}}{\Delta'}} &
               \seq{\Delta,\Delta',\Delta''}{\vec{z/z}}{\Delta''} }
      }
}
\]

\[
\infer[\UR]{\seq{x:\U{x.\alpha}{\Delta,y:\F{\beta}{\Delta'},\Delta''}{A}}{x}{\U{x.\subst{\alpha}{\beta}{y}}{\Delta,\Delta',\Delta''}{A}}}
      {\infer[\UL]
        {\seq{\Gamma := (x:\U{x.\alpha}{\Delta,y:\F{\beta}{\Delta'},\Delta''}{A}, {\Delta,\Delta',\Delta''})}{{\subst{\alpha}{\beta}{y}}}{A}} 
        {\subst{\alpha}{\beta}{y} \spr z[\tsubst{\alpha}{\vec{w/w},\beta/y}/z] &
          \seq{\Gamma}{\vec{w/w}}{\Delta,\Delta''} &
          \infer[\FR]{\seq{\Gamma}{\beta}{\F{\beta}{\Delta'}}}{} &
          \seq{\Gamma,z:A}{z}{A}
        }
      }
\]

\[
\infer[\UR]{\seq{x:\U{x.\subst{\alpha}{\beta}{y}}{\Delta,\Delta',\Delta''}{A}}{x}{\U{x.\alpha}{\Delta,y:\F{\beta}{\Delta'},\Delta''}{A}}}
      {\infer[\FL]
        {\seq{x:\U{x.\subst{\alpha}{\beta}{y}}{\Delta,\Delta',\Delta''}{A}, \Delta,y:\F{\beta}{\Delta'},\Delta''}{\alpha}{A}} 
        {\infer[\UL]{\seq{x:\U{x.\subst{\alpha}{\beta}{y}}{\Delta,\Delta',\Delta''}{A}, \Delta,\Delta'',\Delta'}{\subst{\alpha}{\beta}{y}}{A}}
          {\subst{\alpha}{\beta}{y} \spr z[\tsubst{\subst{\alpha}{\beta}{y}}{\vec{w/w}}] &
            \seq{\Delta,\Delta',\Delta''}{\vec{w/w}}{\Delta,\Delta',\Delta''} &
            \seq{\Delta,\Delta',\Delta'',z:A}{z}{A}
          }
        }}
\]

\[
\infer[\UR]
      {\seq{x:\U{x.\alpha}{\Delta}{\U{y.\beta}{\Delta'}{A}}}{x} {\U{x.\subst{\beta}{\alpha}{y}}{\Delta,\Delta'}{A}}}
      {\infer[\UL]
        {\seq{x:\U{x.\alpha}{\Delta}{\U{y.\beta}{\Delta'}{A}},\Delta,\Delta'}{\subst{\beta}{\alpha}{y}}{A}}
        {\subst{\beta}{\alpha}{y} \spr \beta[\alpha[\vec{w/w}]/y] &
          \seq{x:\U{x.\alpha}{\Delta}{\U{y.\beta}{\Delta'}{A}},\Delta,\Delta'}{\vec{w/w}}{\Delta} &
          \infer[\UL]{\seq{x:\U{x.\alpha}{\Delta}{\U{y.\beta}{\Delta'}{A}},\Delta,\Delta',y:\U{y.\beta}{\Delta'}{A}}{\beta}{A}}{}
        }}
\]

\[
\infer[\UR]
      {\seq{x:{\U{x.\subst{\beta}{\alpha}{y}}{\Delta,\Delta'}{A}}}{x}{\U{x.\alpha}{\Delta}{\U{y.\beta}{\Delta'}{A}}}}
      {\infer[\UR]
        {\seq{x:{\U{x.\subst{\beta}{\alpha}{y}}{\Delta,\Delta'}{A}},\Delta}{\alpha}{{\U{y.\beta}{\Delta'}{A}}}}
        {\infer[\UL]{\seq{x:{\U{x.\subst{\beta}{\alpha}{y}}{\Delta,\Delta'}{A}},\Delta,\Delta'}{\subst{\beta}{\alpha}{y}}{A}}
          {}}}
\]
\end{small}
\caption{Derivations of fusion lemmas}
\vspace{2in}
\label{fig:fusion}
\end{figure}

The types respect the structural transformations, covariantly for \Fsymb\/
and contravariantly for \Usymb\/.

\begin{lemma}[Types Respect Structural Transformations] ~ \label{lem:typespr}
\begin{enumerate}
\item 
 If $\alpha \spr \beta$ then $\F{\alpha}{\Delta} \vdash
  \F{\beta}{\Delta}$

\item If $\alpha \spr \beta$ then $\U{x.\beta}{\Delta}{A} \vdash
  \U{x.\alpha}{\Delta}{A}$
\end{enumerate}
\end{lemma}

\begin{proof}
\[
\infer[\FL]{\seq{x:\F{\alpha}{\Delta}}{x}{\F{\beta}{\Delta}}}
      {\infer[\FR]{\seq{\Delta}{\alpha}{\F{\beta}{\Delta}}}
        {\alpha \spr \beta[\vec{w/w}] &
          \seq{\Delta}{\vec{w/w}}{\Delta}
      }}
\]        
\[
\infer[\UR]
      {\seq{x:\U{x.\beta}{\Delta}{A}}{\U{x.\alpha}{\Delta}{A}}}
      {\infer[\UL]{ \seq{x:\U{x.\beta}{\Delta}{A},\Delta}{\alpha}{A}}
        {\alpha \spr z[\beta[\vec{w/w}]/z] &
          \seq{x:\U{x.\beta}{\Delta}{A},\Delta}{\vec{w/w}}{\Delta} &
          \seq{\ldots,z:A}{z}{A}
      }}
\]        
\end{proof}



\end{document}


\documentclass[conference,compsoconf]{../drl-common/IEEEtran}
\IEEEoverridecommandlockouts

\usepackage{multicol}
\usepackage{mathptmx}
\usepackage{color}
\usepackage[cmex10]{amsmath}
\usepackage{amsthm}
\usepackage{amssymb}
\usepackage{stmaryrd}
\usepackage{../drl-common/proof}
\usepackage{../drl-common/typesit}
\usepackage{../drl-common/typescommon}
\usepackage{../drl-common/theorem-envs}
\usepackage[sort]{natbib}
%% \usepackage{arydshln}
\usepackage{graphics}
\usepackage{natbib}
\usepackage{url}
\usepackage{relsize}
\usepackage{tipa}

\usepackage{tikz}
\usetikzlibrary{decorations.pathmorphing}

\usepackage{fancyvrb}
\newcommand{\ttt}[1]{\texttt{#1}}


\newcommand\Bx[2]{\ensuremath{\Box_{#1} \, {#2}}}
\newcommand\Crc[2]{\ensuremath{\bigcirc_{#1} \, {#2}}}
\newcommand\Dia[2]{\ensuremath{\Diamond_{#1} \, {#2}}}
\newcommand\Flat[1]{\ensuremath{\flat \, {#1}}}
\newcommand\Sharp[1]{\ensuremath{\sharp \, {#1}}}
\newcommand{\sh}{\text{\textesh}}

\newcommand\magicwand{\mathrel{-\mkern-6mu*}}
\newcommand\mor[3]{\ensuremath{#2} \longrightarrow_#1 #3}
\newcommand\C{\ensuremath{\mathcal{C}}}
\newcommand\D{\ensuremath{\mathcal{D}}}
\newcommand\E{\ensuremath{\mathcal{E}}}
\newcommand\deq{\ensuremath{\equiv}}
\newcommand\spr{\ensuremath{\Rightarrow}} %% structural property/2-cell
\newcommand\seq[3]{\ensuremath{#1 \vdash_{#2} #3}}
\newcommand\F[2]{\ensuremath{\dsd{F}_{#1}(#2)}}
\newcommand\U[3]{\ensuremath{\dsd{U}_{#1}(#2 \mid #3)}}
\newcommand\Fsymb[0]{\dsd{F}}
\newcommand\Usymb[0]{\dsd{U}}
\newcommand\tsubst[2]{\ensuremath{#1[#2]}}
\renewcommand\subst[3]{\ensuremath{#1[#2/#3]}}
\newcommand\wftype[2]{\ensuremath{#1 \,\, \dsd{type}_{#2}}}
\renewcommand\wfctx[2]{\ensuremath{#1 \,\, \dsd{ctx}_{#2}}}
\newcommand\modeof[1]{\ensuremath{\hat{#1}}}
\newcommand\many[1]{\ensuremath{\overline{#1}}}
\renewcommand{\oftp}[3]{\ensuremath{#1 \, \vdash #2 \, \dcd{:} \, #3}}
\newcommand\FL{\dsd{FL}}
\newcommand\FR{\dsd{FR}}
\newcommand\UL{\dsd{UL}}
\newcommand\UR{\dsd{UR}}
\newcommand\lolli\multimap
\newcommand\la\dashv

\newcommand{\ignore}[1]{}

\begin{document}

\title{A Fibrational Framework for \\ Substructural and Modal Logics}

% author names and affiliations
% use a multiple column layout for up to three different
% affiliations
\author{\IEEEauthorblockN{Daniel R. Licata}
\IEEEauthorblockA{Wesleyan University\\
\url{dlicata@wesleyan.edu}}
\and
\IEEEauthorblockN{Michael Shulman}
\IEEEauthorblockA{University of San Diego \\
  \url{shulman@sandiego.edu}}

\thanks{
  ?
}

}

\maketitle

\begin{abstract}
Many intuitionistic substructural and modal logics can be seen as a
restriction on the allowed proofs in a basic structural logic or
$\lambda$-calculus.  For example, substructural logics remove structural
properties such as weakening, exchange, and/or contraction, while modal
logics place restrictions on positions in which certain kinds of
variables can be used.  We give a new sequent calculus that makes this
idea precise, describing a substructural or modal logic derivation as a
structural proof that obeys some constraints on the use of the context.
Because the sequent calculus is parametrized by a \emph{mode theory}
describing the constraints, a single generic proof of cut admissibility
can be instantiated to all of the above logics, as well as more complex
variants, including n-linear variables, bunched implications, and
subexponentials.  This codifies the common patterns in cut proofs as an
abstraction.  Semantically, the new sequent calculus corresponds to a
functor between 2-dimensional cartesian multicategories, and the logical
connectives make this functor into a bifibration.  The resulting
framework can be used both to understand logics from the literature and
to design new substructural and modal logics.
\end{abstract}


\section{Introduction}

In ordinary intuitionistic logic or $\lambda$-calculus, assumptions or
variables can go unused (weakening), be used in any order (exchange), be
used more than once (contraction), and be used in any position in a
term.  \emph{Substructural} logics, such as linear logic, ordered logic,
relevant logic, and affine logic, omit some of these structural
properties of weakening, exchange, and contraction, while \emph{modal
  logics} place restrictions on where variables may be used---e.g. a
formula $\Bx{} C$ can only be proved using assumptions of $\Bx{} A$,
while an assumption of $\Dia{}{A}$ can only be used when the conclusion
is $\Dia{}{C}$.  Substructural and modal logics have had many
applications to both functional and logic programming (modeling concepts
such state, staging, distribution, and concurrency, to name just a few).
They are also used as \emph{internal languages} of categories: one uses
an appropriate logical language to do constructions ``inside'' a
particular mathematical setting, which often leads to shorter statements
than working ``externally''.  For example, to define a function
externally in domains, one must first define the underlying
set-theoretic function, and then prove that it is continuous; but when
using untyped $\lambda$-calculus as an internal language of domains,
there is no need to prove that a function described by a $\lambda$-term
is continuous, because all terms are shown to denote continuous
functions.  Substructural logics extend this idea to
various forms of monoidal categories, while modal logics describe monads
and comonads.  Recently,
\citet{schreibershulman12cohesive,shulman15realcohesion} proposed using
modal operators to add a notion of \emph{cohesion} to homotopy type
theory/univalent foundations~\citep{voevodsky06homotopy,uf13hott-book}.
Without going into the precise details of this application, the idea is
to add a triple $\sh{} \la \Flat{} \la \Sharp{}$ of type operators,
where for example $\Sharp{}$ is a monad (like a modal possibility
$\Diamond$ or $\bigcirc$), $\Flat{}$ is a comonad (like a modal
necessity $\Box$), and there is an adjunction structure between them
(e.g. $\flat{A} \to B$ is the same as $A \to \Sharp{B}$).  This raised
the question of how to best add modalities with these properties to type
theory.

Because other similar applications have different monads and comonads
with different properties, we would like general tools for going from a
semantic situation of interest to a well-behaved logic/type theory for
it, e.g. one with cut and identity admissibility (normalization,
$\eta$-expansion).  In previous work~\citep{ls16adjoint}, we considered
the special case of a single-assumption logic, building most directly on
the adjoint logics of
\citet{benton94mixed,bentonwadler96adjoint,reed09adjoint}.  Here we
extend this previous work to the multi-assumption case.  The resulting
framework is quite general and covers many existing intuitionistic
substructural and modal connectives: cartesian, linear, affine,
relevant, ordered, bunched~\citep{ohearnpym99bunched} and
non-associative products and implications; $n$-linear
variables~\citep{reed08namessubstructural,abel15modal,mcbride16nuttin};
the comonadic $\Box$ and linear exponential $!$ and
subexponentials~\citep{nigammiller09subexponentials,danos+93subexponentials};
monadic $\Diamond$ and $\bigcirc$ modalities; and adjoint logic $F$ and
$G$~\citep{benton94mixed,bentonwadler96adjoint,reed09adjoint}, including
the single-assumption 2-categorical version from our previous
work~\citep{ls16adjoint}.  It also supports variations on these, such as
non-monoidal comonads and non-strong monads.  A central syntactic result
is that cut and identity are admissible for our framework
itself, which implies cut admissibility for any logic that can be
described in the framework, including all of the above, as well as any
new logics that one designs using it.  When we view the derivations in
the framework as terms in a type theory, this gives an immediate
normalization (and $\eta$-expansion) result.
%% While it is not too surprising
%% that this is possible, given that cut proofs for these logics all follow
%% a similar template, it is nonetheless satisfying to codify this pattern
%% as an abstraction.

At a high level, the framework is based on the idea that all of the
above logics / type theories are a restriction on how variables can be
used in ordinary structural/cartesian proofs.  We express these
restrictions using a first layer, which is a simple type theory for what
we will call \emph{modes} and \emph{context descriptors}.  The modes are
just a collection of base types, which we write as $p,q,r$, while a
context descriptor $\alpha$ is a term built from variables and
constants.  The next layer is the main logic.  Each proposition/type is
assigned a mode, and the basic sequent is \seq{x_1 : A_1, \ldots, x_n :
  A_n}{\alpha}{C}, where if $A_i$ has mode $p_i$, and $C$ has mode $q$,
then $\oftp{x_1 : p_1,\ldots, x_n : p_n}{\alpha}{q}$.  We use a sequent
calculus to concisely describe cut-free derivations/normal forms, but
everything can be translated to natural deduction in the usual way.
$\Gamma$ itself behaves like an ordinary structural/cartesian context,
while the substructural and modal aspects are enforced by the
\emph{term} $\alpha$, which constrains how the resources from $\Gamma$
may be used.  For example, in linear logic/ordered logic/BI, the context
is usually taken to be a multiset/list/tree.  We represent this by a
pair of an ordinary structural context $\Gamma$, together with a term
$\alpha$ that describes the multiset or list or tree structure, labeled
with variables from the ordinary context at the leaves.  We pronounce a
sequent \seq{\Gamma}{\alpha}{A} as ``$\Gamma$ proves $A$ \{along,over\}
$\alpha$''.

For example, suppose we have one mode $\dsd{n}$, together with a context
descriptor constant $x : \dsd{n}, y:\dsd{n} \vdash x \odot y : \dsd{n}$.
Then an example sequent \seq{x:A, y:B, z:C, w:D}{(y \odot x) \odot z}{E}
should be read as saying that we must prove $E$ using the resources $y$
and $x$ and $z$ (but not $w$) according to the particular tree structure
${(y \odot x) \odot z}$.  If we say nothing else, the framework will
treat $\odot$ as describing a non-associative, linear, ordered context
as in Lambek calculus~\citep{lambek58calculus}: if we have a
product-like type $A \odot B$ internalizing this context
operation,\footnote{We overload binary operations to refer both to
  context descriptors and propositional connectives, because it is clear
  from whether it is applied to variables $x,y,z$ or propositions
  $A,B,C$ which we mean.}  then we will \emph{not} be able to prove
associativity ($(A \odot B) \odot C \dashv\vdash A \odot (B \odot C)$)
or exchange ($A \odot B \vdash B \odot A$) etc.  
\ifthenelse{\boolean{short}}{}{ 
%% para break
}
To get from this basic structure to a linear or affine or relevant or
cartesian system, we provide a way to add structural properties governing
the context descriptor term $\alpha$.  We analyze structural properties
as \emph{equations}, or more generally \emph{directed transformations},
on such terms.  For example, to specify linear logic, we will add a unit
element $1 : \dsd{n}$ together with equations making $(\odot,1)$ into a
commutative monoid ($x \odot (y \odot z) = (x \odot y) \odot z$ and 
$x \odot 1 = x = 1 \odot x$ and 
$x \odot y = y \odot x$)
so that the context descriptors ignore associativity and order.  To get
BI, we add an additional commutative monoid $(\times,\top)$ (with
weakening and contraction, as discussed below), so that a BI context
tree $(x:A,y:B);(z:C,w:D)$ can be represented by the ordinary context
$x:A,y:B,z:C,w:D$ with the term $(x \odot y) \times (z \odot w)$
describing the tree.  Because the context descriptors are themselves
ordinary structural/cartesian terms, the same variable can occur more
than once or not at all.  A descriptor such as $x \odot x$ captures the
idea that we can use the \emph{same} variable $x$ twice, expressing
$n$-linear types.  Thus, we can express contraction for a particular
context descriptor $\odot$ as a transformation $x \spr x \odot x$ (one
use of $x$ allows two).  Weakening, on the other hand, is represented by
a transformation $x \spr 1$, which is oriented to allow throwing away an
allowed use of $x$, but not creating an allowed use from nothing.  We
refer to these as \emph{structural transformations}, to evoke their use
in representing the structural properties of object logics that are
embedded in our framework.  The main sequent $\seq{\Gamma}{\alpha}{A}$
respects the specified structural properties in the sense that when
$\alpha = \beta$, we regard $\seq{\Gamma}{\alpha}{A}$ and
$\seq{\Gamma}{\beta}{A}$ as the same sequent, while when $\alpha \spr
\beta$, there will be an operation that takes a derivation of
\seq{\Gamma}{\beta}{A} to a derivation of \seq{\Gamma}{\alpha}{A}.

Modal logics will generally involve a mode theory with more than one
mode.  For example, a context descriptor $x : \dsd{c} \vdash \dsd{f}(x)
: \dsd{l}$ will generate an adjoint pair of functors between the two
modes, as in the adjoint syntax for linear logic's
$!$~\citep{bentonwadler96adjoint} or other modal
operators~\citep{reed09adjoint}.  Structural transformations are used to
describe how these modal operators interact with each other and with the
product structures, and in some cases~\citep{ls16adjoint} it is
important that there can be more than one structural transformation
between a given pair of context descriptors.

A guiding principle of the framework is a meta-level notion of
\emph{structurality over structurality}.  For example, we always have
\emph{weakening over weakening}: if \seq{\Gamma}{\alpha}{A} then
\seq{\Gamma,y:B}{\alpha}{A}, where $\alpha$ itself is weakened with $y$.
This does not prevent encodings of e.g. linear logic: it is permissible
to weaken a derivation of \seq{\Gamma}{x_1 \odot \ldots \odot x_n}{A}
(``use $x_1$ through $x_n$'') to a derivation of \seq{\Gamma,y:B}{x_1
  \odot \ldots \odot x_n}{A} because the (weakened) context descriptor
still disallows the use of $y$.  Similarly, we have exchange over
exchange and contraction over contraction.  The \emph{identity-over-identity}
principle says that we should be able to prove $A$ using exactly an
assumption $x:A$ ({\seq{\Gamma,x:A}{x}{A}}).  The cut principle says
that from \seq{\Gamma,x:A}{\beta}{B} and \seq{\Gamma}{\alpha}{A} we get
{\seq{\Gamma}{\subst{\beta}{\alpha}{x}}{B}}---the context descriptor for
the result of the cut is the substitution of the context descriptor used
to prove $A$ into the one used to prove $B$.  For example, together with
weakening-over-weakening, this captures the usual cut principle of
linear logic, which says that cutting $\Gamma,x:A \vdash B$ and $\Delta
\vdash A$ yields $\Gamma,\Delta \vdash B$.  If $\Gamma$ binds
$x_1,\ldots,x_n$ and $\Delta$ binds $y_1,\ldots,y_n$, then we will
represent the two derivations to be cut together by sequents with $\beta
= x_1 \odot \ldots \odot x_n \odot x$ and $\alpha = y_1 \odot \ldots
\odot y_n$ then $\beta[\alpha/x] = x_1 \odot \ldots \odot x_n \odot y_1
\odot \ldots \odot y_n$ correctly deletes $x$ and replaces it with the
variables from $\Delta$.  In more subtle situations such as BI, the
substitution will insert the resources used to prove the cut formula in
the correct place in the tree.

The framework has two main logical connectives / type constructors.  The
first, \F{\alpha}{\Delta}, generalizes the \dsd{F} of adjoint logic and
the multiplicative products (e.g. $\otimes$ of linear logic).  The
second, \U{x.\alpha}{\Delta}{A}, generalizes the $\dsd{G}/\dsd{U}$ of
adjoint logic and implication (e.g. $A \lolli B$ in linear logic).  Here
$\Delta$ is a context of assumptions $x_i:A_i$, and trivializing the
context descriptors (i.e. adding an equation $\alpha = \beta$ for all
$\alpha$ and $\beta$) degenerates $\F{\alpha}{\Delta}$ into the ordinary
intuitionistic product $A_1 \times \ldots \times A_n$, while
\U{x.\alpha}{\Delta}{A} becomes $A_1 \to \ldots \to A_n \to A$.  
As one would expect, \dsd{F} is left-invertible and \dsd{U} is right-invertible.
%%  and we conjecture that focusing works with the
%% polarization that one would expect based on these degeneracies
%% ($\F{\alpha}{\Delta^{\mathord{+}}}^{\mathord{+}}$ and
%% $\U{x.\alpha}{\Delta^{\mathord{+}}}{A^{\mathord{-}}}^{\mathord{-}}$).
In linear logic terms, our \dsd{F} and \dsd{U} cover both the
multiplicatives and exponentials; additives can be added separately by
the usual rules.  We discuss many examples of \emph{logical adequacy}
theorems, showing that a sequent can be proved in a standard sequent
calculus for a logic iff its embedding using these connectives can be
proved in the framework.

%% In summary, to specify a particular substructural or modal logic / type
%% theory, one gives constants generating context descriptors $\alpha$,
%% with equations $\alpha = \beta$ and transformations $\alpha \spr \beta$
%% expressing structural properties.  

Being a very general theory, our framework treats the object-logic
structural properties in a general but na\"ive way, allowing an
arbitrary structural transformation to be applied at the non-invertible
rules for $\dsd{F}$ and $\dsd{U}$ and at the leaves of a derivation.
For specific embedded logics, there is often a more refined discipline
that suffices---e.g. for cartesian logic, always contract all
assumptions in all premises, and only weaken at the leaves.  We view our
framework as a tool for bridging the gap between an intended semantic
situation (such as the cohesion example mentioned, ``a comonad and a
monad which are themselves adjoint'') and a proof theory: the framework
gives \emph{some} proof theory for the semantics, and the placement of
structural rules can then be optimized purely in syntax.  To support
this mode of use, we give an equational theory on sequent derivations
that identifies different placements of the same structural rules, which
can be used to prove correctness of such optimizations not just at the
level of provability, but also identity of derivations---which matters
for our intended applications to internal languages.  We discuss some
preliminary work on \emph{equational adequacy}, which extends the
logical correspondence to isomorphisms of definitional-equality-classes
of derivations.

Semantically, the logic corresponds to a functor between
\emph{2-dimensional cartesian multicategories} which is a fibration in
various senses.  Multicategories are a generalization of categories
which allow more than one object in the domain, and cartesianness means
that the multiple domain objects are treated structurally.  The
2-dimensionality supplies a notion of morphism between (multi)morphisms.
A \emph{mode theory} specifying context descriptors and structural
properties is analyzed as a cartesian 2-multicategory, with the
descriptors as 1-cells and the structural properties as 2-cells.  The
functor relates the sequent judgement to the mode theory, specifying the
mode of each proposition and the context descriptor of a sequent.  The
fibration conditions (similar to
\citep{hermida02fibrations,hormann15multicategories}) give respect for
the structural transformations and the presence of \dsd{F} and \dsd{U}
types.  We prove that the sequent calculus and the equational theory are
sound and complete for this semantics: the syntax can be interpreted in
any bifibration, and itself determines one.  This semantics shows that
an interesting class of type theories can be identified with a class of
more mathematical objects, fibrations of cartesian 2-multicategories,
thus providing some progress towards characterizing substructural
and modal type theories in mathematical terms.

\ifthenelse{\boolean{short}}{}{Our framework builds on many approaches to substructural and modal logic
in the literature.  Logical rules that act at a leaf of a
tree-structured context go back to the Lambek
calculus~\citep{lambek58calculus}.  A rich collection of context
structures that correspond to type constructors plays a central role in
display logic~\citep{belnap82display}.  \citet{atkey04separation}'s
$\lambda$-calculus for resource separation is similar to mode theories
with one mode, where there is at most one 2-cell between a given pair of
1-cells; at the logical level, our calculus is a unification of this
with the multimode adjoint logic of
\citet{reed09adjoint}.  Algebraic resource annotations on variables are
used to track modalities in Agda's implementation~\citep{abel15modal}
and in \citet{mcbride16nuttin}'s approach to linear dependent types.  LF
representations of modal or substructural logics work by restricting the
use of cartesian variables~\citep{crary10substructural}.  Relative to
all of these approaches, we believe that the analysis of the context
structures/resources as a \emph{term} in a base type theory, and the
fibrational structure of the derivations over them, is a new and useful
observation.  For example, rather than needing extra-logical conditions
on proof rules to ensure cut admissibility, as in display logic, the
conditions are encoded in the language of context descriptors and the
definition of types from them.  Moreover, none of these existing
approaches allow for proof-relevant 2-cells/structural rules, and their
presence (and the equational theory we give for them) is important for
our applications to extensions of homotopy type theory.  A point of
contrast with substructural logical
frameworks~\citep{cervesatopfenning02llf,watkins+03clf-tr,reed09thesis}
is that logics are ``embedded'' in our calculus (giving a type
translation such that provability in the object logic corresponds to
provability in ours), rather than ``encoding'' the structure of
derivations.  This way, we obtain cut elimination for object languages
as a corollary of framework cut elimination.
}

%% Similar semantic structures have come up recently
%% in~\citep{zeilberger,mellieszeilberger,johann}.  
 

\newcommand\wfsp[4]{\ensuremath{#1 \vdash #2 \spr_{#4} #3}}

\section{Sequent Calculus}

\subsection{Mode Theories}

\begin{figure}
\[
\begin{array}{llll}
\text{modes} & p & & (constants) \\
\text{mode contexts} & \psi & ::= & \cdot \mid \psi,\tptm{x}{p} \\
%% \text{context descriptors} & \alpha,\beta & ::= & x \mid \dsd{c}(\alpha_1,\ldots,\alpha_n) \\
%% \text{substitutions} & \gamma,\delta & ::= & \cdot \mid \gamma,\alpha/x\\
\end{array}
\]

\framebox{Context descriptors/mode morphisms \oftp{\psi}{\alpha}{p}}

\[
\infer{\oftp{\psi}{x}{p}}
      {x:p \in \psi}
\quad
\infer{\oftp{\psi}{\dsd{c}(\alpha_1,\ldots,\alpha_n)}{q}}
      {\dsd{c} : p_1,\ldots,p_n \to q \in \Sigma &
       \oftp{\psi}{\alpha_i}{p_i}
      }
\quad
\infer{\oftp{\psi}{\alpha_i/x_i}{\overline{x_i:q_i}}}
      {\oftp{\psi}{\alpha_i}{q_i}}
\]

\framebox{Structural properties/mode transformations \wfsp{\psi}{\alpha}{\alpha'}{p}}

\[
\begin{array}{c}
\infer{\wfsp{\psi}{\alpha}{\alpha}{p}}{}
\qquad
\infer{\wfsp{\psi}{\alpha_1}{\alpha_3}{p}}
      {\wfsp{\psi}{\alpha_1}{\alpha_2}{p} &
       \wfsp{\psi}{\alpha_2}{\alpha_3}{p} &
      }
\\ \\
\infer{\wfsp{\psi,\psi'}{\subst{\beta}{\alpha}{x}}{\subst{\beta'}{\alpha'}{x}}{q}}
      {\wfsp{\psi,x:p,\psi'}{\beta}{\beta'}{q} &
       \wfsp{\psi,\psi'}{\alpha}{\alpha'}{p}}
\qquad
\infer{\wfsp{\psi}{\alpha}{\alpha'}{p}}
      {\alpha \spr \alpha' \in \Sigma}
\end{array}
\]

\label{fig:2multicategory}
\caption{Syntax for mode theories}
\end{figure}

The first layer of the logic is a type theory whose types we will call
\emph{modes}, and whose terms we will call \emph{context descriptors} or
\emph{mode morphisms}.  We begin with a simple type theory with
variables and constants, as described in
Figure~\ref{fig:2multicategory}.  We assume a fixed signature $\Sigma$
of constants \dsd{c}, and we write $\oftp{\psi}{\alpha}{q}$ for mode
morphism well-formedness and $\oftp{\psi}{\gamma}{\psi'}$ for
substitution well-formedness.  The context $\psi$ here enjoys
(cartesian) structural properties (weakening, exchange, contraction).
Simultaneous substitution into terms is defined as usual.

Next, we need a notation for presenting object-logic structural
properties, which, as described above, will in general be directed
transformations between context descriptors.  For
\oftp{\psi}{\alpha,\alpha'}{p}, we define a judgement
     {\wfsp{\psi}{\alpha}{\alpha'}{p}} representing such
     transformations; it is the least precongruence closed under some
     axioms in the signature $\Sigma$.  For example, to say that a mode
     $p$ with a context monoid $(\odot,1)$ is affine, we would specify
     in $\Sigma$ a structural property \wfsp{x:p}{x}{1}{p}.  Then, using
     the rules in the figure, we can for example derive a transformation
     $(x \odot y) \spr (1 \odot y) \spr y$ that, when applied
     (contravariantly) to a sequent, will allow weakening $y$ to $x
     \odot y$.

In Section~\ref{sec:equalityofderiv}, where we discuss equality of
sequent calculus derivations, we will need an equational theory between
two structural property derivations $s \deq s' ::
\wfsp{\psi}{\alpha}{\alpha'}{q}$.  Because this equational theory does
not influence provability in the sequent calculus, only identity of
proofs, we defer the details.

One choice in the design of this language of context descriptors is how
to handle symmetric structural properties.  One possibility is to
present a desired equation $\alpha = \alpha'$ as an isomorphism, with
axioms $s : \alpha \spr \alpha'$ and $s' : \alpha' \spr \alpha$ (and,
writing $s_1;s_2$ for derivations by the above transitivity rule,
equations $s;s' = 1_{\alpha}$ and $s';s = 1_{\alpha'}$).  While this is
conceptually and technically sufficient, we have found it helpful in
examples to also use a ``strict'' equality of context descriptors which
is implicitly respected by all types and judgements.  This simplifies
the description of some situations, but the difference is important
mainly at the level of identity of derivations rather than
provability---for example, we can make a binary operation $\odot$ into a
strict monoid, rather than adding associator and unitor isomorphisms,
which requires more equations between structural properties.  To support
this, we allow $\Sigma$ to contain axioms for equations $\alpha \deq
\beta$ and define a judgement $\psi \vdash \alpha \deq \beta : p$ as the
least congruence containing these axioms (i.e. add symmetry to the rules
for transformations).  Because the context descriptors and their
equality are defined independently of any subsequent judgements, we take
the liberty of suppressing this equality using $\alpha$ to refer to a
context-descriptor-modulo-\deq---that is, we assume a metatheory with
quotient sets/types, and use meta-level equality for object-level
equality, as recently advocated by \citet{altenkirchkaposiXXpopl}.  For
example, because the judgement \wfsp{\psi}{\alpha}{\beta}{p} is indexed
by equivalence classes of context descriptions, the reflexivity rule
above implicitly means $\alpha \deq \beta$ implies $\alpha \spr \beta$.  

A signature $\Sigma$ consisting of some number of constants $c$,
equations $\alpha \deq \beta$, transformations $\alpha \spr \beta$ (and
equalities of transformations $s \deq s'$) specifies a \emph{mode theory}.

\subsection{Sequent Calculus}

\begin{figure}
\[
\begin{array}{l}
%% \begin{array}{llll}
%% \text{Types} & A & ::= & P \mid \F{\alpha}{\Delta} \mid \U{\alpha}{\Delta}{A} \\
%% \end{array}
%% \\ \\
\framebox{Types $A,B,C$ \quad \wftype{A}{p}}
\qquad
\infer{\wftype{P}{p}}{}
\\ \\
\infer{\wftype{\F{\alpha}{\Delta}}{q}}
      {\oftp{\psi}{\alpha}{q} &
        \wfctx{\Delta}{\psi}}
\qquad
\infer{\wftype{\U{x.\alpha}{\Delta}{A}}{q}}
      {\oftp{\psi,x:q}{\alpha}{p} &
        \wfctx{\Delta}{\psi} &
        \wftype{A}{p}
      }
\\ \\
\framebox{Contexts $\Gamma,\Delta$ \quad \wfctx{\Gamma}{\psi}}
\qquad
\infer{\wfctx{\cdot}{\cdot}}{}
\qquad
\infer{\wfctx{\Gamma,x:A}{\psi,x:p}}
      {\wfctx{\Gamma}{\psi} &
        \wftype{A}{p}}
\\ \\
\framebox{\seq{\Gamma}{\alpha}{A}}
\qquad
\infer[\dsd{v}]{\seq{\Gamma}{\beta}{P}}
      {x:P \in \Gamma & \beta \spr x}
\\ \\
\infer[\FR]{\seq{\Gamma}{\beta}{\F{\alpha}{\Delta}}}
      {%% \modeof{\Gamma} \vdash \gamma : \modeof{\Delta} & 
        \beta \spr \tsubst{\alpha}{\gamma} &
        \seq{\Gamma}{\gamma}{\Delta} 
      }
\quad
\infer[\FL]{\seq{\Gamma,x:\F{\alpha}{\Delta},\Gamma'}{\beta}{C}}
      {\seq{\Gamma,\Gamma',\Delta}{\subst \beta {\alpha}{x}}{C}}
%% \infer{\seq{\Gamma}{\beta}{C}}
%%       {{x}:{\F{\alpha}{\Delta}} \in \Gamma & 
%%         \oftp{\modeof{\Gamma},{x'} : {\modeof{\F{\alpha}{\Delta}}}}{\beta'}{\modeof{C}} &
%%         \beta \deq \tsubst{\beta'}{x/x'} &
%%         \seq{\Gamma,\Delta}{\subst {\beta'} {\alpha}{x'}}{C}}
\\ \\
\infer[\UR]{\seq{\Gamma}{\beta}{\U{x.\alpha}{\Delta}{A}}}
      {\seq{\Gamma,\Delta}{\subst{\alpha}{\beta}{x}}{A}}
\qquad
\infer[\UL]{\seq{\Gamma}{\beta}{C}}
      {\begin{array}{l}
          x:\U{x.\alpha}{\Delta}{A} \in \Gamma \\
          \beta \spr \subst{\beta'}{\tsubst{\alpha}{\gamma}}{z} \\
          \seq{\Gamma}{\gamma}{\Delta} \\
          \seq{\Gamma,\tptm{z}{A}}{\beta'}{C}
       \end{array}
      }
\\ \\
\framebox{\seq{\Gamma}{\gamma}{\Delta}}
\qquad
\infer[\cdot]{\seq{\Gamma}{\cdot}{\cdot}}
      {}
\qquad
\infer[\_,\_]{\seq{\Gamma}{\gamma,\alpha/x}{\Delta,x:A}}
      {\seq{\Gamma}{\gamma}{\Delta} &
       \seq{\Gamma}{\alpha}{A}
      }
\end{array}
\]    
\caption{Sequent Calculus}
\label{fig:sequent}
\hrule
\end{figure}

The whole sequent calculus is parametrized by a mode theory $\Sigma$,
which is an implicit argument to all of the definitions in
Figure~\ref{fig:sequent}.  The judgement assigns each proposition/type
$A$ a mode $p$.  Encodings of non-modal logics will generally only make
use of one mode, while modal logics use different modes to represent
different notions of truth, such as the linear and cartesian categories
in \citet{bentonwadler96adjoint} and the true/valid/lax judgements in
\citet{pfenningdavies}.  The next judgement assigns each context
$\Gamma$ a mode context $\psi$ pointwise.  Formally, we think of
contexts as ordered: we do not regard $x:A,y:B$ and $y:B,x:A$ and the
same context, though we will have an admissible exchange rule that
passes between derivations of one and the other.

The sequent judgement \seq{\Gamma}{\alpha}{A} relates a context
$\wfctx{\Gamma}{\psi}$ and a type $\wftype{A}{p}$ and context descriptor
\oftp{\psi}{\alpha}{p}.  Because $\wfctx{\Gamma}{\psi}$ means that each
variable in $\Gamma$ is in $\psi$, where $x : A_i \in \Gamma$ implies $x
: p_i$ in $\psi$ with \wftype{A_i}{p_i}, we think of $\Gamma$ as binding
variable names both in $\alpha$ and for use in the derivation.

As discussed in the introduction, a guiding principle is to make the
following rules admissible (see Section~\ref{sec:synmeta} for details),
which express respect for object-logic structural properties and
structurality-over-structurality:
\[
\begin{array}{c}
\infer{\seq{\Gamma}{\alpha}{A}}
      {\alpha \spr \beta &
       \seq{\Gamma}{\beta}{A}}
\qquad
\infer{\seq{\Gamma,x:A}{x}{A}}{}
\qquad
\infer{\seq{\Gamma}{\subst{\beta}{\alpha}{x}}{B}}
    {\seq{\Gamma,x:A}{\beta}{B} &
     \seq{\Gamma}{\alpha}{A}}
\\ \\
\infer{\seq{\Gamma,y:A}{\alpha}{C}}
      {\seq{\Gamma}{\alpha}{C}}
\quad
\infer{\seq{\Gamma,y:B,x:A}{\alpha}{C}}
      {\seq{\Gamma,x:A,y:B}{\alpha}{C}}
\qquad
\infer{\seq{\Gamma,x:A}{\subst \alpha x y}{C}}
      {\seq{\Gamma,x:A,y:A}{\alpha}{C}}
\end{array}
\]

We now explain the rules for the sequent calculus; the reader may wish
to refer to the examples in Section~\ref{sec:exampleencodings} in
parallel with this high-level description.

We assume atomic propositions $P$ are given a specified mode $p$, and
state the identity rule only for atoms, to check that the other
instances are admissible.  If we had states the identity rule as
\seq{\Gamma,x:P}{x}{P}, then respect for object-logic structural
properties would not be admissible, so we allow a $\beta \spr x$ premise
here.  Using a structural property at a leaf of a derivation is common
in e.g. affine logic, where the derivation of $\beta \spr x$ would use
weakening to forget any additional resources.  

Next, we consider the \F{\alpha}{\Delta} type.  Syntactically, we view
the context $\Delta = x_1:A_1,\ldots,x_n:A_n$ where \wftype{A_i}{p_i} as
binding the variables $x_i:p_i$ in $\alpha$, so for example
\F{\alpha}{x:A,y:B} and \F{\alpha[x \leftrightarrow x']}{x':A,y:B} are
$\alpha$-equivalent types (in de Bruijn form we would write
\F{\alpha}{A_1,\ldots,A_n} and use indices in $\alpha$).  The type
formation rule says that \dsd{F} moves covariantly along a mode morphism
$\alpha$, representing a combination of the types in $\Delta$ according
to the context descriptor $\alpha$.  A typical binary instance of
\dsd{F} is a multiplicative product ($A \otimes B$ in linear logic),
which, given a binary context descriptor $\odot$ as in the introduction,
is written \F{x \odot y}{x:A,y:B}.  A typical nullary instance is a unit
(1 in linear logic), written \F{1}{}.  A typical unary instance is the
\dsd{F} connective of adjoint logic, which for a unary context
descriptor $\dsd{f} : \dsd{p} \to \dsd{q}$ is written
\F{\dsd{f}(x)}{x:A}.

The rules for our \dsd{F} connective capture the common pattern between
all of these examples.  On the left, \F{\alpha}{\Delta} ``decays'' into
$\Delta$, but \emph{marking the uses of resources in $\Delta$ with
  $\alpha$ by the substitution \subst{\beta}{\alpha}{x}}.  We assume
that $\Delta$ is $\alpha$-renamed to avoid collision with $\Gamma$ (the
proof term here would be a ``\dsd{split}'' that binds variables for each
position in $\Delta$).  On the right, the \FR rule says that you must
rewrite (using structural properties) the context descriptor to have an
$\alpha$ at the outside, with a mode substitution $\gamma$ that divides
the exisitng resources up between the positions in $\Delta$, and then
prove each formula in $\Delta$ using the specified resources.  We leave
the typing of $\gamma$ implicit, though there is officially a
requirement $\oftp{\psi}{\gamma}{\psi'}$ where $\wfctx{\Gamma}{\psi}$
and $\wfctx{\Delta}{\psi'}$, as required for the second premise to be
well-formed.  Another way to understand this rule is to begin with
the ``axiomatic \FR'' 
\[
{\seq{\Delta}{\alpha}{\F{\alpha}{\Delta}}}
\]
which says that there is a map from $\Delta$ to \F{\alpha}{\Delta} along
$\alpha$.  Then, in the same way that a typical injection rule for
coproducts builds a precomposition into an ``axiomatic injection'' such
as $\dsd{inl} :: A \vdash A + B$, the \FR\/ rule builds a precomposition
with $\seq{\Gamma}{\gamma}{\Delta}$ and then a structural rule $\beta
\spr \alpha[\gamma]$ into the ``axiomatic'' version.  


Finally, the rules for substitutions are pointwise.

One subtle point about $\FL$ is that there are two competing principles:
making the rules ``obviously'' structural-over-structural, and reducing
inessential non-determinism.  Here, we choose the later, and treat the
assumption of \F{\alpha}{\Delta} affinely, removing it from the context
when it is used.  It will turn out that the judgement nonetheless enjoys
contraction-over-contraction
(Corollary~\ref{thm:contraction-over-contraction}), because contraction
for negatives is built into the \UL-rule, and contraction for positives
follows from this and the fact that we can always reconstruct a positive
from what it decays to on the left (c.f. how purely positive formulas
have contraction in linear logic).



TODO: move ../depdep-examples-short here


\setlength{\bibsep}{-1pt} %% dirty trick: make this negative
{ \small
%% \linespread{0.70}
\bibliographystyle{abbrvnat}
\bibliography{../drl-common/cs}
}

\newpage
\clearpage


\appendix[Syntactic Properties]
\label{sec:synprop-long}

Here, we include proofs of the results from
Section~\ref{sec:synprop-short}.  We show the cases for the rules in
Figure~\ref{fig:sequent}, though the results readily extend to additive
sums and products.

Define the \emph{size} of a derivation of \seq{\Gamma}{\alpha}{A} or
\seq{\Gamma}{\gamma}{\Delta} to be the number of inference rules for
these judgements $(\dsd{v},\FL, \FR, \UL, \UR, \cdot, \_,\_)$ used in it
(i.e., the evidence that variables are in a context and the evidence for
structural transformations do not contribute to the size).  Sizes are
necessary for the cut proof, where we sometimes weaken or invert a
derivation before applying the inductive hypothesis.

\begin{lemma}[Respect for Transformations] ~ %% \label{lem:respectspr}
\begin{enumerate}
\item If \seq{\Gamma}{\beta}{A} and $\beta' \spr \beta$ then
  \seq{\Gamma}{\beta'}{A}, and the resulting derivation has the same
  size as the given one.
\item If \seq{\Gamma}{\gamma}{\Delta} and $\gamma' \spr \gamma$ then
  \seq{\Gamma}{\gamma'}{\Delta}, and the resulting derivation has the
  same size as the given one.
\end{enumerate}
\end{lemma}
\begin{proof}
Mutual induction on the given derivation.  The cases for \dsd{v} and
$\FR$ and $\UL$ are immediate (with no use of the inductive hypothesis)
by composing with the equality in the premise of the rule.  This does
not change the size of the derivation because the derivations of
structural transformations are ignored by the size.  The cases for $\FL$ and
$\UR$ use the inductive hypothesis, along with congruence for structural
transformations to show that $\subst{\beta}{\alpha}{x} \spr
\subst{\beta'}{\alpha}{x}$ or $\subst{\alpha}{\beta}{x} \spr
\subst{\alpha}{\beta'}{x}$.  The cases for substitutions rely on the
fact that no generating structural transformations for mode substitutions are
allowed, so if $\gamma' \spr \cdot$ then $\gamma'$ is literally $\cdot$,
and $(-,-)$ is injective (if $\gamma' \spr (\gamma_1,\alpha_2/x)$, then
$\gamma'$ is $(\gamma_1',\alpha_2'/x)$ with $\gamma_1' \spr \gamma_1$
and $\alpha_2' \spr \alpha_2$); this is enough to use the inductive
hypotheses in the cons case.
\end{proof}

\begin{lemma}[Weakening over weakening] ~ %% \label{lem:weakening} ~
\begin{enumerate}
\item If \seq{\Gamma,\Gamma'}{\alpha}{C} then
\seq{\Gamma,\tptm{z}{A},\Gamma'}{\alpha}{C}, and the resulting
derivation has the same size as the given one.  
\item If \seq{\Gamma,\Gamma'}{\gamma}{\Delta} then
\seq{\Gamma,\tptm{z}{A},\Gamma'}{\gamma}{\Delta}, and the resulting
derivation has the same size as the given one.  
\item If \seq{\Gamma,\Gamma''}{\alpha}{C} then
\seq{\Gamma,\Gamma',\Gamma''}{\alpha}{C}, and the resulting
derivation has the same size as the given one.  
\end{enumerate}
\end{lemma}
\begin{proof}
It is implicit that the mode morphism $\alpha$ is weakend with $z$ in
the conclusion.  Intuitively, weakening holds because the contexts
$\Gamma$ are treated like ordinary structural contexts in all of the
rules---they are fully general in every conclusion, and the premises
check membership or extend them---and because weakening holds for mode
morphisms and equalities of mode morphisms.  Formally, the first two
parts are proved by mutual induction; each case is either immediate
or follows from weakening for the mode morphisms, weakening for
transformations, and the inductive hypotheses.  The third
part is proved by induction over $\Gamma'$, repeatedly applying the
first part.  
%% The case for the hypothesis rule is immediate, because
%% $\Gamma$ may contain variables other than $x$.  The case for
%% \Fsymb-right follows from weakening for the mode morphisms, and
%% equations between mode morphisms, and the inductive hypothesis for
%% substitutions.  The case for \Fsymb-left follows from the inductive
%% hypothesis, as does the case for \Usymb-right.  
\end{proof}

\begin{lemma}[Exchange over exchange] %% \label{lem:exchange}
If \seq{\Gamma,x:A,y:B,\Gamma'}{\alpha}{C} then
\seq{\Gamma,y:B,x:A,\Gamma'}{\alpha}{C}, and the resulting derivation
has the same size as the given one.  (And similarly for substitutions,
and exchange can be iterated).  
\end{lemma}
\begin{proof} Analogous to weakening.  
\end{proof}

We sometimes write $\modeof{\Gamma}$ for the $\psi$ such that
\wfctx{\Gamma}{\psi} and similarly for $\modeof{A}$.

\begin{theorem}[Identity] ~ %% \label{thm:identity}
\begin{enumerate}
\item If $x:A \in \Gamma$ then $\seq{\Gamma}{x}{A}$.
\item If $\oftp{\modeof{\Gamma}}{\rho}{\modeof{\Delta}}$ is a
  variable-for-variable mode substitution such that $x:A \in \Delta$
  implies $\rho(x) : A \in \Gamma$, then $\seq{\Gamma}{\rho}{\Delta}$.
\end{enumerate}
\end{theorem}

\begin{proof}
The standard proof by induction on $A$ (mutually with $\Delta$) applies:
the case for atomic propositions is a rule, and for the other
connectives, apply the invertible and then non-invertible rule to reduce
the problem to the inductive hypotheses.  More specifically, identity
for $P$ is a rule.  In the case for \F{\alpha}{\Delta}, with $\Gamma =
\Gamma_1,x:\F{\alpha}{\Delta},\Gamma_2$, we reduce it to the inductive
hypothesis as follows:
\[
\infer[\FL]{\seq{\Gamma_1,x:\F{\alpha}{\Delta},\Gamma_2}{x}{\F{\alpha}{\Delta}}}
      {\infer[\FR]{\seq{\Gamma_1,\Gamma_2,\Delta}{\alpha}{\F{\alpha}{\Delta}}}
                        {\alpha \spr \tsubst{\alpha}{\vec{x/x}} &
                        \seq{\Gamma_1,\Gamma_2,\Delta}{\vec{x/x}}{\Delta}
                        }}
\]
In the second premise, the $\vec{x/x}$ substitution for each $x \in
\Delta$ is a variable-for-variable substitution, so the second part of
the inductive hypothesis applies.  
The case for \Usymb\/ is similar
\[
\infer[\UR]{\seq{\Gamma}{x}{\U{\alpha}{\Delta}{A}}}
      {\infer[\UL]{\seq{\Gamma,\Delta}{\alpha}{A}}
                        {\alpha \spr \subst{x}{\tsubst{\alpha}{\vec{x/x}}}{x} &
                        \seq{\Gamma,\Delta}{\vec{x/x}}{\Delta} &
                        \seq{\Gamma,x:A}{x}{A}
                        }}
\]

For the second part, the hypothesis of the lemma asks that every
variable in $\Delta$ is associated by $\rho$ with a variable of the same
type in $\Gamma$; this is enough to iterate the first part of the
lemma for each position in $\Delta$.  Specifically, the case where
$\Delta$ is the empty context $\cdot$ is a rule. In the case for a cons
$\Delta,y:A$, we have
\oftp{\modeof{\Gamma}}{\rho}{(\modeof{\Delta},y:\modeof{A})} which means
$\rho$ must be of the form $\rho',x/y$ where $x \in \modeof{\Gamma}$ and
$\rho'$ is a variable-for-variable substitution.  Because $\rho$ was
type-preserving, $x : A \in \Gamma$ and $\rho'$ is type-preserving, so
we obtain the result from the inductive hypotheses as follows:
\[
\infer{\seq{\Gamma}{\rho,x/y}{\Delta,y:A}}
      {\seq{\Gamma}{\rho}{\Delta} & 
       \seq{\Gamma}{x}{A}
      }
\]
\end{proof}

\begin{lemma}[Left-invertibility of \Fsymb] %% \label{lem:Finv}
If $\D :: \seq{\Gamma_1,x_0:\F{\alpha_0}{\Delta_0},\Gamma_2}{\beta}{C}$
and then there is a derivation $\D' ::
\seq{\Gamma_1,\Gamma_2,\Delta_0}{\subst{\beta}{\alpha_0}{x_0}}{C}$ and
$size(\D') \le size(\D)$ (and analogously for substitutions).
\end{lemma}

\begin{proof}
Intuitively, we find all of the places where \D ``splits'' $x_0$, delete
the \FL used to do the split, and reroute the variables to the ones in
the context of the result.  

Formally, we proceed by induction on \D.  We write $\Gamma$ for the
whole context $\Gamma_1,x_0:\F{\alpha_0}{\Delta_0},\Gamma_2$.

In the case for \dsd{v}, $x : P \in
\Gamma_1,x_0:\F{\alpha_0}{\Delta_0},\Gamma_2$ cannot be equal to $x_0 :
\F{\alpha_0}{\Delta_0}$ because the types conflict, so we can reapply
the \dsd{v} rule in $\Gamma_1,\Gamma_2,\Delta$.

In the case for $\FR$, we have
\[
\infer{\seq{\Gamma}{\beta}{\F{\alpha}{\Delta}}}
      {\beta \spr \tsubst{\alpha}{\gamma} &
        \seq{\Gamma}{\gamma}{\Delta} 
      }
\]
with $x_0 : \F{\alpha_0}{\Delta_0} \in \Gamma$.  By the inductive
hypothesis we get
\seq{\Gamma_1,\Gamma_2,\Delta_0}{\subst{\gamma}{\alpha_0}{x}}{\Delta}.  Because
$x_0$ is not free in $\alpha$,
$\subst{(\tsubst{\alpha}{\gamma})}{\alpha_0}{x_0} =
\tsubst{\alpha}{\subst{\gamma}{\alpha_0}{x_0}}$, so we can reapply \FR:
\[
\infer{\seq{\Gamma_1,\Gamma_2}{\subst{\beta}{\alpha_0}{x_0}}{\F{\alpha}{\Delta}}}
      {{\subst{\beta}{\alpha_0}{x_0}} \spr \tsubst{\alpha}{\subst{\gamma}{\alpha_0}{x_0}} &
        \seq{\Gamma_1,\Gamma_2,\Delta_0}{\subst{\gamma}{\alpha_0}{x}}{\Delta}
      }
\]
Both the input and the output have size 1 more than the size of their
subderivations, and the output subderivation is no bigger than the input
by the inductive hypothesis.

In the case for $\FL$
\[
\infer[\FL]{\seq{\Gamma_1',x:\F{\alpha}{\Delta},\Gamma_2'}{\beta}{C}}
      {\deduce{\seq{\Gamma_1',\Gamma_2',\Delta}{\subst \beta {\alpha}{x}}{C}}{\D}}
\]
with $\Gamma_1,x_0 : \F{\alpha_0}{\Delta_0},\Gamma_2 =
\Gamma_1',x:\F{\alpha}{\Delta},\Gamma_2'$, we distinguish cases on
whether $x = x_0$ or not.  If they are the same (i.e. we have hit a left
rule on $x_0$), then $\alpha_0 = \alpha$ and $\Delta_0 = \Delta$ and
\D\/ is the result, and the size is 1 less than the size of the input.
If they are different, then (because $x_0$ is somewhere in
$\Gamma_1',\Gamma_2'$) by the inductive hypothesis we have a derivation
\[
\D' :: {\seq{(\Gamma_1',\Gamma_2')-x_0,\Delta,\Delta_0}{\subst{\subst \beta {\alpha}{x}}{\alpha_0}{x_0}}{C}}
\]
that is no bigger than \D.  Because $x_0$ is from $\Gamma$ and not
$\Delta$, it does not occur in $\alpha$, so 
\[
{\subst{\subst \beta {\alpha}{x}}{\alpha_0}{x_0}} = 
{\subst{\subst \beta {\alpha_0}{x_0}}{\alpha}{x}}
\]
By (iterating) exchange, we get a derivation 
\[
\D'' :: {\seq{(\Gamma_1',\Gamma_2')-x_0,\Delta_0,\Delta}{\subst{\subst \beta {\alpha_0}{x_0}}{\alpha}{x}}{C}}
\]
whose size is the same as $\D'$ and so no bigger than $\D$.  Applying
$\FL$ to $\D''$ (using the fact that
$(\Gamma_1',x:\F{\alpha}{\Delta},\Gamma_2')-x_0 = \Gamma_1,\Gamma_2$)
derives $\seq{\Gamma_1,\Gamma_2}{\subst{\beta}{\alpha_0}{x_0}}{C}$, and
the size is no bigger than the size of the input.

In the case for $\UR$,
\[
\infer{\seq{\Gamma}{\beta}{\U{x.\alpha}{\Delta}{A}}}
      {\seq{\Gamma,\Delta}{\subst{\alpha}{\beta}{x}}{A}}
\]
the inductive hypothesis gives a
$\D' :: \seq{\Gamma_1,\Gamma_2,\Delta,\Delta_0}{\subst{\subst{\alpha}{\beta}{x}}{\alpha_0}{x_0}}{A}$
and (iterated) exchange gives 
$\D'' ::
\seq{\Gamma_1,\Gamma_2,\Delta_0,\Delta}{\subst{\subst{\alpha}{\beta}{x}}{\alpha_0}{x_0}}{A}$,
both no bigger than \D.  Because $x_0$ is in $\Gamma$ and not $\Delta$,
it is not free in $\alpha$, so 
\[
{\subst{\subst{\alpha}{\beta}{x}}{\alpha_0}{x_0}} = {\subst{\alpha}{\subst{\beta}{\alpha_0}{x_0}}{x}}
\]
Thus, we can derive
\[
\infer{\seq{\Gamma_1,\Gamma_2,\Delta_0}{\subst{\beta}{\alpha_0}{x_0}}{\U{x.\alpha}{\Delta}{A}}}
      {\deduce{\seq{\Gamma_1,\Gamma_2,\Delta_0,\Delta}{\subst{\alpha}{\subst{\beta}{\alpha_0}{x_0}}{x}}{A}}{\D''}}
\]

In the case for $\UL$, 
\[
\infer{\seq{\Gamma}{\beta}{C}}
      {x:\U{x.\alpha}{\Delta}{A} \in \Gamma & 
        \beta \spr \subst{\beta'}{\tsubst{\alpha}{\gamma}}{z} &
        \seq{\Gamma}{\gamma}{\Delta} &
        \seq{\Gamma,\tptm{z}{A}}{\beta'}{C}
      }
\]
we know that $x$ is different than $x_0$ because the types conflict.
The inductive hypotheses give no-bigger derivations of
\[
\seq{\Gamma_1,\Gamma_2\Delta_0}{\subst{\gamma}{\alpha_0}{x_0}}{\Delta} \qquad \seq{\Gamma_1,\Gamma_2,\tptm{z}{A},\Delta_0}{\subst{\beta'}{\alpha_0}{x_0}}{C}
\]
and the latter can be exchanged to
\[
\seq{\Gamma_1,\Gamma_2,\Delta_0,\tptm{z}{A}}{\subst{\beta'}{\alpha_0}{x_0}}{C}
\]
again without increasing the size.  Thus, we can produce
\[
\infer{\seq{\Gamma_1,\Gamma_2,\Delta_0}{\subst{\beta}{\alpha_0}{x}}{C}}
      {\begin{array}{l}
          x:\U{x.\alpha}{\Delta}{A} \in \Gamma_1,\Gamma_2,\Delta_0 \\
          {\subst{\beta}{\alpha_0}{x}} \spr \subst{{\subst{\beta'}{\alpha_0}{x_0}}}{\tsubst{\alpha}{{\subst{\gamma}{\alpha_0}{x_0}}}}{z}\\
          \seq{\Gamma_1,\Gamma_2,\Delta_0}{\subst{\gamma}{\alpha_0}{x_0}}{\Delta} \\
          \seq{\Gamma_1,\Gamma_2,\Delta_0,\tptm{z}{A}}{\subst{\beta'}{\alpha_0}{x_0}}{C}
        \end{array}
      }
\]
where the transformation is the composition of the
\subst{-}{\alpha_0}{x_0} substitution into the given transformation, and
rearranging the substitution (note that $x_0$ does not occur in
$\alpha$):
\[
\begin{array}{ll}
\subst{\beta}{\alpha_0}{x_0} & \spr
\subst{\subst{\beta'}{\tsubst{\alpha}{\gamma}}{z}}{\alpha_0}{x_0} 
= 
\subst{\subst{\beta'}{\alpha_0}{x_0}}{\subst{\tsubst{\alpha}{\gamma}}{\alpha_0}{x_0}}{z}
\\
& =
\subst{\subst{\beta'}{\alpha_0}{x_0}}{\tsubst{\alpha}{\subst{\gamma}{\alpha_0}{x_0}}}{z} 
\end{array}
\]

The case for $\cdot$ is immediate.  The case for $\_,\_$ follows from
the two inductive hypotheses, because
$\subst{(\gamma,\alpha/x)}{\alpha_0}{x_0} =
{(\subst{\gamma}{\alpha_0}{x_0},\subst{\alpha}{\alpha_0}{x_0}/x)}$.
\end{proof}


\begin{theorem}[Cut] ~ %% \label{thm:cut}
\begin{enumerate} 
\item  If $\seq{\Gamma,\Gamma'}{\alpha_0}{A_0}$ and $\seq{\Gamma,x_0:A_0,\Gamma'}{\beta}{B}$ 
then $\seq{\Gamma,\Gamma'}{\beta[\alpha_0/x_0]}{B}$ 
\item If $\seq{\Gamma,\Gamma'}{\alpha_0}{A_0}$ and $\seq{\Gamma,x_0:A_0,\Gamma'}{\gamma}{\Delta}$ 
then $\seq{\Gamma,\Gamma'}{\gamma[\alpha_0/x_0]}{\Delta}$ 
\item If $\seq{\Gamma}{\gamma}{\Delta}$ and 
\seq{\Gamma,\Delta}{\beta}{C}
then \seq{\Gamma}{\tsubst{\beta}{\gamma}}{C}.  
\end{enumerate}
\end{theorem}

\begin{proof}
The induction ordering is the usual one: First the cut formula, and then
the sizes of size of the two derivations.  More specifically, any part
can call another with a smaller cut formula ($A_0$ for part 1 and part
2, $\Delta$ for part 3).  Additionally, part 1 and part 2 call
themselves and each other with the same cut formula and smaller $\D$ or
$\E$.

For part part 1, there are 5 rules in the sequent calculus, so 25 pairs
of final rules.

\begin{itemize}
\item (5 pairs) Any rule and identity
\[
\deduce{\seq{\Gamma,\Gamma'}{\alpha_0}{A_0}}{\D} 
\qquad
\infer{\seq{\Gamma,x:A,\Gamma'}{\beta}{Q}}
      {z:Q \in (\Gamma,x_0:A_0,\Gamma') &
        \deduce{\beta \spr z}{s}}
\]
There two subcases, depending on whether the cut variable is $z$ or not.
If $z$ is $x_0$ and $A_0$ is $Q$, then \D\/ derives
\seq{\Gamma,\Gamma'}{\alpha_0}{Q} and we want a derivation of
\seq{\Gamma,\Gamma'}{\subst{\beta}{\alpha_0}{z}}{Q}.  By congruence on
$s$, $\subst{\beta}{\alpha_0}{z} \spr \subst{z}{\alpha_0}{z}$, so
Lemma~\ref{lem:respectspr} gives the result.  If $z$ is not $x_0$,
then $z:Q \in \Gamma,\Gamma'$.  We want a derivation of
\seq{\Gamma,\Gamma'}{\subst{\beta}{\alpha_0}{x_0}}{Q}, and substituting
into $s$ gives $\subst{\beta}{\alpha_0}{x_0} \spr z$ (because $z \neq
x_0$), so we can derive
\[
\infer{\seq{\Gamma,\Gamma'}{\subst \beta {\alpha_0}{x_0}}{Q}}
      {z:Q \in (\Gamma,\Gamma') &
        {\subst{\beta}{\alpha_0}{x_0} \spr z}}
\]

\item (5 pairs) Any rule and $\FR$ (right-commutative)
\[
\deduce{\seq{\Gamma,\Gamma'}{\alpha_0}{A_0}}{\D} \qquad
\infer{\seq{\Gamma,x_0:A_0,\Gamma'}{\beta}{\F{\alpha}{\Delta}}}
      {\beta \spr \tsubst{\alpha}{\gamma} &
        \deduce{\seq{\Gamma,x_0:A_0,\Gamma'}{\gamma}{\Delta}}{\E}
      }
\]
By the inductive hypothesis, cutting into \D\/ into \E\/ gives
\seq{\Gamma,\Gamma'}{\subst{\gamma}{\alpha_0}{x_0}}{\Delta}.  By
congruence, $\subst{\beta}{\alpha_0}{x_0} \spr
\subst{\tsubst{\alpha}{\gamma}}{\alpha_0}{x_0}$.  Since $\gamma$ is a
total substitution for all variables in \modeof{\Delta},
$\subst{\tsubst{\alpha}{\gamma}}{\alpha_0}{x_0} =
\tsubst{\alpha}{\subst{\gamma}{\alpha_0}{x_0}}$, so
$\subst{\beta}{\alpha_0}{x_0} \spr
\tsubst{\alpha}{\subst{\gamma}{\alpha_0}{x_0}}$.  Thus we can reapply
the $\FR$ rule to get
\seq{\Gamma,\Gamma'}{\subst{\beta}{\alpha_0}{x_0}}{\F{\alpha}{\Delta}}.

\item (5 pairs) Any rule and $\UR$ (right-commutative).    
\[
\deduce{\seq{\Gamma,\Gamma'}{\alpha_0}{A_0}}{\D} \qquad
\infer{\seq{\Gamma,x_0:A_0,\Gamma'}{\beta}{\U{x.\alpha}{\Delta}{A}}}
      {\deduce{\seq{\Gamma,x_0:A_0,\Gamma',\Delta}{\subst{\alpha}{\beta}{x}}{A}}{\E}}
\]
The inductive cut of \D\/ into \E\/ gives 
\[
\seq{\Gamma,\Gamma',\Delta}{\subst{\subst{\alpha}{\beta}{x}}{\alpha_0}{x_0}}{A}
\]
Because the variables from $\modeof{\Gamma},\modeof{\Gamma'}$ occur only
in $\beta$, not in $\alpha$, this substitution equals 
{\subst{\alpha}{\subst{\beta}{\alpha_0}{x_0}}{x}} so reapplying the
$\UR$ rule
derives 
{\seq{\Gamma,\Gamma'}{\subst{\beta}{\alpha_0}{x_0}}{\U{x.\alpha}{\Delta}{A}}}.   

\item (2 additional pairs, plus 3 overlapping with above) $\FL$ and
  any rule (left commutative).  

There is one subtlety in this case.  The usual strategy for a left rule
against an arbitrary \E\/ is to push $\E$ into the ``continuation'' of
the left rule.  However, as discussed above, our left rule for \Fsymb\/
eagerly inverts \emph{all} occurences of $x$, while $\E$ itself also has
$x$ in scope.  Thus, we use Lemma~\ref{lem:Finv} to pull the
left-inversion to the bottom of \E, and then push that into \D.  On
proof terms, this corresponds to making all references to $x$ in \E\/
instead refer to the results of the ``split'' at the bottom of $\D$.
%% This subtlety could be avoided by building contraction into $\FL$, as
%% discussed above.

Formally, we have
\[
\begin{array}{c}
\infer{\seq{\Gamma,\Gamma'}{\alpha_0}{A_0}}
      {{x}:{\F{\alpha}{\Delta}} \in \Gamma,\Gamma' &
        \deduce{\seq{((\Gamma,\Gamma')-x),\Delta}{\subst {\alpha_0} {\alpha}{x}}{A_0}}{\D}}
\\ \\
\deduce{\seq{\Gamma,x_0:A_0,\Gamma'}{\beta}{C}}{\E}
\end{array}
\]

By left invertibility on \E, we obtain (note that $x \neq x_0$ because
$x_0$ only in scope in \E\/, not \D) a derivation $\E'$ of
{\seq{(\Gamma,x:A_0,\Gamma')-x,\Delta}{\subst{\beta}{\alpha}{x}}{C}}
that is no bigger than $\E$.  Because the cut formula is the same, and
$\E'$ is no bigger than \E\/, and \D\/ is smaller than the given
derivation of $A_0$, we can apply the inductive hypothesis to cut $\D$
and $\E'$ to get
\[
{\seq{(\Gamma,\Gamma')-x,\Delta}{\subst{\subst{\beta}{\alpha}{x}}{\subst{\alpha_0}{\alpha}{x}}{x_0}}{C}}.
\]
Commuting substitutions gives
\[
{\subst{{\subst{\beta}{\alpha}{x}}}{\subst{\alpha_0}{\alpha}{x}}{x_0}} = \subst {\beta[\alpha_0/x_0]}{\alpha}{x}
\]
so we can reapply $\FL$ 
\[
\infer{\seq{\Gamma,\Gamma'}{\beta[\alpha_0/x_0]}{C}}
      {\seq{((\Gamma,\Gamma')-x),\Delta}{\subst {(\beta[\alpha_0/x_0])} {\alpha}{x}}{C}}
\]

\item (2 additional pairs, plus 3 overlapping with above) $\UL$ and any rule (left commutative)
In this case, $x:\U{\alpha}{\Delta}{A} \in \Gamma,\Gamma'$ and
we have
\[
\begin{array}{c}
\infer{\seq{\Gamma,\Gamma'}{\alpha_0}{A_0}}
      {\deduce{\alpha_0 \spr \subst{\alpha_0'}{\tsubst{\alpha}{\gamma}}{z}}{s} &
       \deduce{\seq{\Gamma,\Gamma'}{\gamma}{\Delta}}{\D_1} &
       \deduce{\seq{\Gamma,\Gamma',z:A}{\alpha_0'}{A_0}}{\D_2}
      }
\\ \\
\deduce{\seq{\Gamma,x_0:A_0,\Gamma'}{\beta}{B}}{\E}
\end{array}
\]

Weakening \E\/ with $z$ and then cutting $\D_2$ and $\E$ by the inductive
hypothesis (which applies because $\D_2$ is smaller and weakening does
not change the size) gives
\[
\deduce{\seq{\Gamma,\Gamma',z:A}{\subst{\beta}{\alpha_0'}{x_0}}{B}}{\D_2'}
\]
Thus, we have the first, third, and fourth premises of
\[
\infer{\seq{\Gamma,\Gamma'}{\subst{\beta}{\alpha_0}{x_0}}{A_0}}
      {\begin{array}{l}
          x:\U{\alpha}{\Delta}{A} \in \Gamma,\Gamma' \\
          {\subst{\beta}{\alpha_0}{x_0}} \spr \subst{\subst{\beta}{\alpha_0'}{x_0}}{\tsubst{\alpha}{\gamma}}{z} \\
       {\seq{\Gamma,\Gamma'}{\gamma}{\Delta}} \\
       {\seq{\Gamma,\Gamma',z:A}{\subst{\beta}{\alpha_0'}{x_0}}{B}}
        \end{array}
      }
\]
The transformation premise is
\[
     {\subst{\beta}{\alpha_0}{x_0}} 
\spr \subst{\beta}{\subst{\alpha_0'}{\tsubst{\alpha}{\gamma}}{z}}{x0} = \subst{\subst{\beta}{\alpha_0'}{x_0}}{\tsubst{\alpha}{\gamma}}{z}
\]
where the first step is by congruence with $\beta$ on $s$, and the
second is by properties of substitution ($z$ is not free in $\beta$).

\item (3 pairs) Right rule or identity and $\FL$ (principal or
  right-commutative).  

Suppose the right-hand derivation ends with $\FL$, and the left-hand
derivation is either a right rule or identity (\dsd{v}) (the cases for
left-rules were covered above).  

We distinguish cases on whether the \FL\/ case-analyzes $x_0$ or not.
If it does, then, because $A_0$ is $\F{\alpha}{\Delta}$, the left-hand
derivation must be \FR, and we have a principal cut
\[
\infer{\seq{\Gamma,\Gamma'}{\alpha_0}{\F{\alpha}{\Delta}}}
      {  
        \deduce{\alpha_0 \spr \tsubst{\alpha}{\gamma}}{s} &
        \deduce{\seq{\Gamma,\Gamma'}{\gamma}{\Delta}}{\D}
      }
\qquad
\infer{\seq{\Gamma,x_0:\F{\alpha}{\Delta},\Gamma'}{\beta}{C}}
      {\deduce{\seq{\Gamma,\Gamma',\Delta}{\subst{\beta}{\alpha}{x_0}}{C}}
              {\E}}
\]
Using the inductive hypothesis part 3 to cut $\D$ and $\E$ ($\Delta$ is
a subformula of the original cut formula \F{\alpha}{\Delta}) gives
\[
\seq{\Gamma,\Gamma'}{\tsubst{\subst{\beta}{\alpha}{x_0}}{\gamma}}{C}
\]
By congruence on $s$ and because $\gamma$ substitutes only for
variables in $\modeof {\Delta}$,
\[
\subst{\beta}{\alpha_0}{x_0} \spr
{\subst{\beta}{\tsubst \alpha \gamma}{x_0}} =
{\tsubst{\subst{\beta}{\alpha}{x_0}}{\gamma}} 
\]
So applying Lemma~\ref{lem:respectspr} gives 
\seq{\Gamma,\Gamma'}{\subst{\beta}{\alpha_0}{x_0}}{C}.  

If the left rule is not on the cut variable, then we have
\[
\deduce{\seq{\Gamma,\Gamma'}{\alpha_0}{A_0}}{\D}
\quad
\infer{\seq{\Gamma,x_0:A_0,\Gamma'}{\beta}{C}}
      { x : \F{\alpha}{\Delta} \in \Gamma,\Gamma' &
        \deduce{\seq{((\Gamma,x_0:A_0,\Gamma')-x),\Delta}{\subst{\beta}{\alpha}{x}}{C}}{\E}}
\]

We are going to commute $\D$ under \FL\/ on $x$, so need to reroute uses
of $x$ to the bottom by the left-inversion lemma, which gives
\[
\D' :: {\seq{((\Gamma,\Gamma')-x),\Delta}{\subst{\alpha_0}{\alpha}{x}}{A_0}}
\]
and $\D'$ is no bigger than \D.

Cutting $\D'$ and $\E$ by the inductive hypothesis gives
\[
\seq{((\Gamma,\Gamma')-x),\Delta}{\subst{\subst{\beta}{\alpha}{x}}{\subst{\alpha_0}{\alpha}{x}}{x_0}}{C}
\]
Because $x_0$ is not free in $\alpha$, 
\[
  {\subst{\subst{\beta}{\alpha}{x}}{\subst{\alpha_0}{\alpha}{x}}{x_0}}
= {\subst{\subst{\beta}{\alpha_0}{x_0}}{\alpha}{x}}
\]
so we can apply \FL
\[
\infer{\seq{\Gamma,\Gamma'}{\subst{\beta}{\alpha_0}{x_0}}{C}}
      {\seq{(\Gamma,\Gamma'-x)}{\subst{\subst{\beta}{\alpha_0}{x_0}}{\alpha}{x}}{C}}
\]
%% \seq{((\Gamma,\Gamma')-x),\Delta}{\subst{\subst{\beta}{\alpha}{x}}{\alpha_0}{x_0}}{C}
%% \]

\item (3 pairs) Right rule or identity and $\UL$ (principal or
  right-commutative).

If $x_0$ is the variable used by \UL\/ in the right-hand derivation, then
the left-hand derivation must have been derived by $\UR$, and we have
\[
\begin{array}{l}
\D \quad = \quad \infer{\seq{\Gamma,\Gamma'}{\alpha_0}{\U{x_0.\alpha}{\Delta}{A}}}
   {  
     \deduce{\seq{\Gamma,\Gamma',\Delta}{\subst \alpha {\alpha_0}{x_0}}{A}}{\D'}
   }
\\ \\
\infer{\seq{\Gamma,x_0:\U{x_0.\alpha}{\Delta}{A},\Gamma'}{\beta}{C}}
      {
        \begin{array}{l}
        s :: \beta \spr \subst{\beta'}{\tsubst{\alpha}{\gamma}}{z} \\
        {\E_1 :: \seq{\Gamma,x_0:{\U{x_0.\alpha}{\Delta}{A}},\Gamma'}{\gamma}{\Delta}}\\
        {\E_2 :: \seq{\Gamma,x_0:{\U{x_0.\alpha}{\Delta}{A}},\Gamma',\tptm{z}{A}}{\beta'}{C}}
        \end{array}
      }
\end{array}
\]
First, cutting the original \D\/ and the smaller $\E_1$ and $\E_2$ gives 
\[
\deduce{{\seq{\Gamma,\Gamma'}{\subst{\gamma}{\alpha_0}{x_0}}{\Delta}}}{\E_1'}
\qquad 
\deduce{{\seq{\Gamma,\Gamma',\tptm{z}{A}}{\subst{\beta'}{\alpha_0}{x_0}}{C}}}{\E_2'}
\]
Cutting $\E_1'$ \emph{into} $\D'$ (the derivations have switched places,
so are not necessarily smaller, but the cut formula $\Delta$ is a
subformula of $\U{x_0.\alpha}{\Delta}{A}$) gives
\[
\deduce
{\seq{\Gamma,\Gamma'}{\tsubst{\alpha}{\subst{\gamma}{\alpha_0}{x}}}{A}} {\D_1'}
\]
Cutting $\D_1'$ into $\E_2'$ gives 
\[
\seq{\Gamma,\Gamma'}{\subst{\subst{\beta'}{\alpha_0}{x_0}}{\subst{\alpha}{\alpha_0}{x_0}}{z}}{A}
\]
But by using $s$ and commuting substitutions we have 
\[
\subst{\beta}{\alpha_0}{x_0} \spr
\subst{(\subst{\beta'}{\tsubst{\alpha}{\gamma}}{z})}{\alpha_0}{x_0} = 
{\subst{\subst{\beta'}{\alpha_0}{x_0}}{\tsubst{\alpha}{\subst{\gamma}{\alpha_0}{x_0}}}{z}}
\]
so Lemma~\ref{lem:respectspr} gives the result.  

On the other hand, if the \UL\/ is not on $x_0$, then we have
\[
\deduce{\seq{\Gamma,\Gamma'}{\alpha_0}{A_0}}
       {
         \D
       }
\quad
\infer{\seq{\Gamma,x_0:A_0,\Gamma'}{\beta}{C}}
      {
        \begin{array}{l}
          x : \U{x.\alpha}{\Delta}{A} \in \Gamma,\Gamma' \\
          \beta \spr \subst{\beta'}{\tsubst{\alpha}{\gamma}}{z} \\
          \seq{\Gamma,x_0:A_0,\Gamma'}{\gamma}{\Delta} \\
          \seq{\Gamma,x_0:A_0,\Gamma',\tptm{z}{A}}{\beta'}{C}
        \end{array}
      }
\]

By the inductive hypotheses we get 
\[
\seq{\Gamma,\Gamma'}{\subst{\gamma}{\alpha_0}{x_0}}{\Delta}
\qquad
\seq{\Gamma,\Gamma',z:A}{\subst{\beta'}{\alpha_0}{x_0}}{C}
\]
so we can derive
\[
\infer{\seq{\Gamma,x_0:A_0,\Gamma'}{\subst{\beta}{\alpha_0}{x_0}}{C}}
      {
        \begin{array}{l}
          x : \U{x.\alpha}{\Delta}{A} \in \Gamma,\Gamma' \\
          {\subst{\beta}{\alpha_0}{x_0}} \spr \subst{\subst{\beta'}{\alpha_0}{x_0}}{\tsubst{\alpha}{\subst{\gamma}{\alpha_0}{x_0}}}{z} \\
          \seq{\Gamma,\Gamma'}{\subst{\gamma}{\alpha_0}{x_0}}{\Delta} \\
          \seq{\Gamma,\Gamma',\tptm{z}{A}}{\subst{\beta'}{\alpha_0}{x_0}}{C}
        \end{array}
      }
\]
For the second premise, we get
\[
\subst{\beta}{\alpha_0}{x_0} \spr
\subst{\subst{\beta'}{\tsubst{\alpha}{\gamma}}{z}}{\alpha_0}{x_0}
\]
by congruence on the assumed transformation, and then commute substitutions.  

\end{itemize}

For part 2, there are just two right-commutative cases: For
\[
\seq{\Gamma,\Gamma'}{\alpha_0}{A_0}
\qquad
\seq{\Gamma,x_0:A_0,\Gamma'}{\cdot}{\cdot}
\]
we also have $\subst \cdot {\alpha_0}{x_0} = \cdot$ and
\seq{\Gamma,\Gamma'}{\cdot}{\cdot}.  For
\[
\seq{\Gamma,\Gamma'}{\alpha_0}{A_0}
\qquad
\infer{\seq{\Gamma,x_0:A_0,\Gamma'}{\gamma,\alpha/x}{\Delta,x:A}}
      {\seq{\Gamma,x_0:A_0,\Gamma'}{\gamma}{\Delta} &
        \seq{\Gamma,x_0:A_0,\Gamma'}{\alpha}{A}
      }
\]
we have $\subst{(\gamma,\alpha/x)}{\alpha_0}{x_0} 
= (\subst{\gamma}{\alpha_0}{x_0},\subst{\alpha}{\alpha_0}{x_0})$, and
 \[
\seq{\Gamma,\Gamma'}{\subst{\gamma}{\alpha_0}{x_0}}{\Delta} \quad
\seq{\Gamma,\Gamma'}{\subst{\alpha}{\alpha_0}{x_0}}{A}
\]
by the inductive hypotheses, so we can reapply the rule to conclude
\seq{\Gamma,\Gamma'}{\subst{(\gamma,\alpha/x)}{\alpha_0}{x_0}}{\Delta,x:A}.

For part 3, we induct on $\Delta$, reducing a simulatenous cut to
iterated single-variable cuts.  If $\Delta$ is empty, then we have
\[
\seq{\Gamma}{\cdot}{\cdot}
\qquad
\deduce{\seq{\Gamma,\cdot}{\beta}{C}}{\E}
\]
and we return \E, noting that $\tsubst{\beta}{\cdot} = \beta$.  Otherwise
we have
\[
\infer{\seq{\Gamma}{\gamma,\alpha/x}{\Delta,x:A}}
      {\deduce{\seq{\Gamma}{\gamma}{\Delta}}{\D_1} &
        \deduce{\seq{\Gamma}{\alpha}{A}}{\D_2}}
\qquad
\deduce{\seq{\Gamma,\Delta,x:A}{\beta}{C}}{\E}
\]
Using the inductive hypothesis to cut $\D_2$ into $\E$ ($A$ is smaller
than $\Delta,x:A$) gives
\[
\deduce{\seq{\Gamma,\Delta}{\subst{\beta}{\alpha}{x}}{C}}
       {\E'}
\]
Using the inductive hypothesis to cut $\D_1$ into $\E'$ ($\Delta$ is
smaller than $\Delta,x:A$) gives
\[
\seq{\Gamma}{\tsubst{\subst{\beta}{\alpha}{x}}{\gamma}}{C}
\]
Because $\gamma$ substitutes for $\hat \Delta$ (and not $\hat \Gamma$,
the free variables of $\alpha$),
\[
\tsubst{\beta}{\gamma,\alpha/x}
= {\tsubst{\subst{\beta}{\alpha}{x}}{\gamma}}
\]
\end{proof}

\begin{corollary}[Contraction over contraction] %% \label{cor:contraction}
\item If
\seq{\Gamma,x:A,y:A,\Gamma'}{\alpha}{C}
then
\seq{\Gamma,z:A,\Gamma'}{\tsubst \alpha {z/x,z/y}}{C}
\end{corollary}

\begin{proof}  Contraction can be shown by cutting with a renaming substitution.
The mode substitution $z/x,z/y$ is a variable-for-variable substitution,
and is type-preserving between ${x:A,y:A}$ and ${\Gamma,z:A,\Gamma'}$.
Therefore, by identity (part 2),
\seq{\Gamma,z:A,\Gamma'}{z/x,z/y}{x:A,y:A}.  Thus, by cut (part 2), we
obtain the result.
\end{proof}

\begin{corollary}[Right-invertibility of \Usymb] %% \label{cor:Uinv}
If $\seq{\Gamma}{\beta}{\U{x.\alpha}{\Delta}{A}}$ then 
{\seq{\Gamma,\Delta}{\subst{\alpha}{\beta}{x}}{A}}.
\end{corollary}

\begin{proof}
$\UL$ with identities in both premises gives a derivation
\[
\infer{\seq{\Gamma,\Delta,x:{\U{x.\alpha}{\Delta}{A}}}{\alpha}{A}}
      {
        \alpha = z[\alpha[\vec{x/x}]/z] & 
        \seq{\Gamma,\Delta}{\vec{x/x}}{\Delta} &
        \seq{\Gamma,\Delta,x:{\U{x.\alpha}{\Delta}{A}},z:A}{z}{A}
      }
\]
Weakening the assumed derivation to 
\seq{\Gamma,\Delta}{\beta}{\U{x.\alpha}{\Delta}{A}}
and then cutting for $x$ in the above gives the result:

\[
\infer{\seq{\Gamma,\Delta}{\subst{\alpha}{\beta}{x}}{A}}
      {\seq{\Gamma,\Delta}{\beta}{\U{x.\alpha}{\Delta}{A}} & 
       \seq{\Gamma,\Delta,x:{\U{x.\alpha}{\Delta}{A}}}{\alpha}{A}
      }
\]
\end{proof}

The following ``fusion'' lemmas (which are isomorphisms, not just
interprovabilities) relate $\Fsymb$ and $\Usymb$.  Special cases
include: $A \times (B \times C)$ is isomorphic to a primitive triple
product $\{x:A,y:B,z:C\}$; currying; and associativity of $n$-ary
functions ($A_1,\ldots,A_n \to (B_1,\ldots,B_m \to C)$ is isomorphic to
$A_1,\ldots,A_n,B_1,\ldots,B_m \to C$).  The derivations are in
Figure~\ref{fig:fusion}.  We adopt the convention that an unlabled leaf
$\alpha \spr \beta$ is proved by equality of context descriptors, and an
unlabeled sequent leaf is proved by identity
(Theorem~\ref{thm:identity}).  

\begin{lemma}[Fusion] ~ %% \label{lem:fusion}
\begin{enumerate} 

\item $\F{\alpha}{\Delta,x:\F{\beta}{\Delta'},\Delta''} \dashv \vdash
  \F{\subst{\alpha}{\beta}{x}}{\Delta,\Delta',\Delta''}$

\item $\U{x.\alpha}{\Delta,y:\F{\beta}{\Delta'},\Delta''}{A} \dashv \vdash
  \U{x.\subst{\alpha}{\beta}{y}}{\Delta,\Delta',\Delta''}{A}$

\item 
$\U{x.\alpha}{\Delta}{\U{y.\beta}{\Delta'}{A}} \dashv \vdash
 \U{x.\subst{\beta}{\alpha}{y}}{\Delta,\Delta'}{A}$

\end{enumerate}
\end{lemma}

\begin{figure*}
\begin{footnotesize}
\[
\infer[\FL]{
  \seq{z:\F{\alpha}{\Delta,x:\F{\beta}{\Delta'},\Delta''}}
      {z}
      {\F{\subst{\alpha}{\beta}{x}}{\Delta,\Delta',\Delta''}}
}
{
  \infer[\FL]{\seq{\Delta,x:\F{\beta}{\Delta'},\Delta''}{\alpha}{\F{\subst{\alpha}{\beta}{x}}{\Delta,\Delta',\Delta''}}}
        {
          \infer[\FL]{\seq{\Delta,\Delta'',\Delta'}{\subst{\alpha}{\beta}{x}}{\F{\subst{\alpha}{\beta}{x}}{\Delta,\Delta',\Delta''}}}
                {\subst{\alpha}{\beta}{x} \spr \tsubst{\subst{\alpha}{\beta}{x}}{\vec{z/z}} &
                 \seq{\Delta,\Delta'',\Delta'}{\vec{z/z}}{\Delta,\Delta',\Delta''}
                }
        }
}
\]

\[
\infer{
  \seq{z:{\F{\subst{\alpha}{\beta}{x}}{\Delta,\Delta',\Delta''}}}
      {z}
      {\F{\alpha}{\Delta,x:\F{\beta}{\Delta'},\Delta''}}
}
{  
\infer[\FL]{\seq{\Delta,\Delta',\Delta''}
           {\subst{\alpha}{\beta}{x}}
           {\F{\alpha}{\Delta,x:\F{\beta}{\Delta'},\Delta''}}}
      {\alpha[\beta/x] \spr \alpha[\vec{y/y},\beta/x,\vec{z/z}] &
        \infer[\FR]{\seq{\Delta,\Delta',\Delta''}{\vec{y/y},\beta/x,\vec{z/z}} {\Delta,x:\F{\beta}{\Delta'},\Delta''}}
              {\seq{\Delta,\Delta',\Delta''}{\vec{y/y}}{\Delta} & 
               \infer[\FR]{\seq{\Delta,\Delta',\Delta''}{\beta}{\F{\beta}{\Delta'}}}
                     {\beta \spr \beta[\vec{w/w}] & \seq{\Delta,\Delta',\Delta''}{\vec{w/w}}{\Delta'}} &
               \seq{\Delta,\Delta',\Delta''}{\vec{z/z}}{\Delta''} }
      }
}
\]

\[
\infer[\UR]{\seq{x:\U{x.\alpha}{\Delta,y:\F{\beta}{\Delta'},\Delta''}{A}}{x}{\U{x.\subst{\alpha}{\beta}{y}}{\Delta,\Delta',\Delta''}{A}}}
      {\infer[\UL, \text { with } \Gamma = {p:\U{x.\alpha}{\Delta,y:\F{\beta}{\Delta'},\Delta''}{A}, {\Delta,\Delta',\Delta''}}]
        {\seq{x:\U{x.\alpha}{\Delta,y:\F{\beta}{\Delta'},\Delta''}{A}, {\Delta,\Delta',\Delta''}}{{\subst{\alpha}{\beta}{y}}}{A}} 
        {\subst{\alpha}{\beta}{y} \spr z[\tsubst{\alpha}{\vec{w/w},\beta/y}/z] &
          \seq{\Gamma}{\vec{w/w}}{\Delta,\Delta''} &
          \infer[\FR]{\seq{\Gamma}{\beta}{\F{\beta}{\Delta'}}}{} &
          \seq{\Gamma,z:A}{z}{A}
        }
      }
\]

\[
\infer[\UR]{\seq{x:\U{x.\subst{\alpha}{\beta}{y}}{\Delta,\Delta',\Delta''}{A}}{x}{\U{x.\alpha}{\Delta,y:\F{\beta}{\Delta'},\Delta''}{A}}}
      {\infer[\FL]
        {\seq{x:\U{x.\subst{\alpha}{\beta}{y}}{\Delta,\Delta',\Delta''}{A}, \Delta,y:\F{\beta}{\Delta'},\Delta''}{\alpha}{A}} 
        {\infer[\UL]{\seq{x:\U{x.\subst{\alpha}{\beta}{y}}{\Delta,\Delta',\Delta''}{A}, \Delta,\Delta'',\Delta'}{\subst{\alpha}{\beta}{y}}{A}}
          {\subst{\alpha}{\beta}{y} \spr z[\tsubst{\subst{\alpha}{\beta}{y}}{\vec{w/w}}] &
            \seq{\Delta,\Delta',\Delta''}{\vec{w/w}}{\Delta,\Delta',\Delta''} &
            \seq{\Delta,\Delta',\Delta'',z:A}{z}{A}
          }
        }}
\]

\[
\infer[\UR]
      {\seq{x:\U{x.\alpha}{\Delta}{\U{y.\beta}{\Delta'}{A}}}{x} {\U{x.\subst{\beta}{\alpha}{y}}{\Delta,\Delta'}{A}}}
      {\infer[\UL]
        {\seq{x:\U{x.\alpha}{\Delta}{\U{y.\beta}{\Delta'}{A}},\Delta,\Delta'}{\subst{\beta}{\alpha}{y}}{A}}
        {\subst{\beta}{\alpha}{y} \spr \beta[\alpha[\vec{w/w}]/y] &
          \seq{x:\U{x.\alpha}{\Delta}{\U{y.\beta}{\Delta'}{A}},\Delta,\Delta'}{\vec{w/w}}{\Delta} &
          \infer[\UL]{\seq{x:\U{x.\alpha}{\Delta}{\U{y.\beta}{\Delta'}{A}},\Delta,\Delta',y:\U{y.\beta}{\Delta'}{A}}{\beta}{A}}{}
        }}
\]

\[
\infer[\UR]
      {\seq{x:{\U{x.\subst{\beta}{\alpha}{y}}{\Delta,\Delta'}{A}}}{x}{\U{x.\alpha}{\Delta}{\U{y.\beta}{\Delta'}{A}}}}
      {\infer[\UR]
        {\seq{x:{\U{x.\subst{\beta}{\alpha}{y}}{\Delta,\Delta'}{A}},\Delta}{\alpha}{{\U{y.\beta}{\Delta'}{A}}}}
        {\infer[\UL]{\seq{x:{\U{x.\subst{\beta}{\alpha}{y}}{\Delta,\Delta'}{A}},\Delta,\Delta'}{\subst{\beta}{\alpha}{y}}{A}}
          {}}}
\]

\end{footnotesize}
\caption{Derivations of fusion lemmas}
\label{fig:fusion}
\end{figure*}

The types respect the structural transformations, covariantly for \Fsymb\/
and contravariantly for \Usymb\/.

\begin{lemma}[Types Respect Structural Transformations] ~ %% \label{lem:typespr}
\begin{enumerate}
\item 
 If $\alpha \spr \beta$ then $\F{\alpha}{\Delta} \vdash
  \F{\beta}{\Delta}$

\item If $\alpha \spr \beta$ then $\U{x.\beta}{\Delta}{A} \vdash
  \U{x.\alpha}{\Delta}{A}$
\end{enumerate}
\end{lemma}

\begin{proof}
\[
\infer[\FL]{\seq{x:\F{\alpha}{\Delta}}{x}{\F{\beta}{\Delta}}}
      {\infer[\FR]{\seq{\Delta}{\alpha}{\F{\beta}{\Delta}}}
        {\alpha \spr \beta[\vec{w/w}] &
          \seq{\Delta}{\vec{w/w}}{\Delta}
      }}
\]        
\[
\infer[\UR]
      {\seq{x:\U{x.\beta}{\Delta}{A}}{\U{x.\alpha}{\Delta}{A}}}
      {\infer[\UL]{ \seq{x:\U{x.\beta}{\Delta}{A},\Delta}{\alpha}{A}}
        {\alpha \spr z[\beta[\vec{w/w}]/z] &
          \seq{x:\U{x.\beta}{\Delta}{A},\Delta}{\vec{w/w}}{\Delta} &
          \seq{\ldots,z:A}{z}{A}
      }}
\]        
\end{proof}



\end{document}


\documentclass[conference,compsoconf]{../drl-common/IEEEtran}
\IEEEoverridecommandlockouts

\usepackage{multicol}
\usepackage{mathptmx}
\usepackage{color}
\usepackage[cmex10]{amsmath}
\usepackage{amsthm}
\usepackage{amssymb}
\usepackage{stmaryrd}
\usepackage{../drl-common/proof}
\usepackage{../drl-common/typesit}
\usepackage{../drl-common/typescommon}
\usepackage{../drl-common/theorem-envs}
\usepackage[square,sort]{natbib}
%% \usepackage{arydshln}
\usepackage{graphics}
\usepackage{natbib}
\usepackage{url}
\usepackage{relsize}
\usepackage{tipa}

\usepackage{tikz}
\usetikzlibrary{decorations.pathmorphing}

\usepackage{fancyvrb}
\newcommand{\ttt}[1]{\texttt{#1}}


\newcommand\Bx[2]{\ensuremath{\Box_{#1} \, {#2}}}
\newcommand\Crc[2]{\ensuremath{\bigcirc_{#1} \, {#2}}}
\newcommand\Dia[2]{\ensuremath{\Diamond_{#1} \, {#2}}}
\newcommand\Flat[1]{\ensuremath{\flat \, {#1}}}
\newcommand\Sharp[1]{\ensuremath{\sharp \, {#1}}}
\newcommand{\sh}{\text{\textesh}}

\newcommand\magicwand{\mathrel{-\mkern-6mu*}}
\newcommand\mor[3]{\ensuremath{#2} \longrightarrow_#1 #3}
\newcommand\C{\ensuremath{\mathcal{C}}}
\newcommand\D{\ensuremath{\mathcal{D}}}
\newcommand\E{\ensuremath{\mathcal{E}}}
\newcommand\deq{\ensuremath{\equiv}}
\newcommand\spr{\ensuremath{\Rightarrow}} %% structural property/2-cell
\newcommand\seq[3]{\ensuremath{#1 \vdash_{#2} #3}}
\newcommand\F[2]{\ensuremath{\dsd{F}_{#1}(#2)}}
\newcommand\U[3]{\ensuremath{\dsd{U}_{#1}(#2 \mid #3)}}
\newcommand\Fsymb[0]{\dsd{F}}
\newcommand\Usymb[0]{\dsd{U}}
\newcommand\tsubst[2]{\ensuremath{#1[#2]}}
\renewcommand\subst[3]{\ensuremath{#1[#2/#3]}}
\newcommand\wftype[2]{\ensuremath{#1 \,\, \dsd{type}_{#2}}}
\renewcommand\wfctx[2]{\ensuremath{#1 \,\, \dsd{ctx}_{#2}}}
\newcommand\modeof[1]{\ensuremath{\hat{#1}}}
\newcommand\many[1]{\ensuremath{\overline{#1}}}
\renewcommand{\oftp}[3]{\ensuremath{#1 \, \vdash #2 \, \dcd{:} \, #3}}
\newcommand\FL{\dsd{FL}}
\newcommand\FR{\dsd{FR}}
\newcommand\UL{\dsd{UL}}
\newcommand\UR{\dsd{UR}}
\newcommand\lolli\multimap
\newcommand\la\dashv

\newcommand{\ignore}[1]{}

\begin{document}

\title{A Fibrational Framework for \\ Substructural and Modal Logics}

% author names and affiliations
% use a multiple column layout for up to three different
% affiliations
\author{\IEEEauthorblockN{Daniel R. Licata}
\IEEEauthorblockA{Wesleyan University\\
\url{dlicata@wesleyan.edu}}
\and
\IEEEauthorblockN{Michael Shulman}
\IEEEauthorblockA{University of San Diego \\
  \url{shulman@sandiego.edu}}

\thanks{
  ?
}

}

\maketitle

\begin{abstract}
Many intuitionistic substructural and modal logics can be seen as a
restriction on the allowed proofs in a basic structural logic or
$\lambda$-calculus.  For example, substructural logics remove structural
properties such as weakening, exchange, and/or contraction, while modal
logics place restrictions on positions in which certain kinds of
variables can be used.  We give a new sequent calculus that makes this
idea precise, describing a substructural or modal logic derivation as a
structural proof that obeys some constraints on the use of the context.
Because the sequent calculus is parametrized by a \emph{mode theory}
describing the constraints, a single generic proof of cut admissibility
can be instantiated to all of the above logics, as well as more complex
variants, including n-linear variables, bunched implications, and
subexponentials.  This codifies the common patterns in cut proofs as an
abstraction.  Semantically, the new sequent calculus corresponds to a
functor between 2-dimensional cartesian multicategories, and the logical
connectives make this functor into a bifibration.  The resulting
framework can be used both to understand logics from the literature and
to design new substructural and modal logics.
\end{abstract}


\section{Introduction}

In ordinary intuitionistic logic or $\lambda$-calculus, assumptions or
variables can go unused (weakening), be used in any order (exchange), be
used more than once (contraction), and be used in any position in a
term.  \emph{Substructural} logics, such as linear logic, ordered logic,
relevant logic, and affine logic, drop some of these structural
properties of weakening, exchange, and contraction, while \emph{modal
  logics} place restrictions on where variables may be used---e.g. a
formula $\Bx{} C$ can only be proved using assumptions of $\Bx{} A$,
while an assumption of $\Dia{}{A}$ can only be used when the conclusion
is $\Dia{}{C}$.  Substructural and modal logics have had many
applications to both functional and logic programming (modeling concepts
such state, staged computation, distribution, and concurrency, to name
just a few).

Substructural and modal logics can also be used as \emph{internal
  languages} of categories, where one uses an appropriate logical
language to do constructions ``inside'' a particular mathematical
setting, which often leads to shorter statements than working
``externally''.  For example, to define a function when working
``externally'' in domains, one must first define the underlying
set-theoretic function, and then prove that it is continuous.  But when
using untyped $\lambda$-calculus as an internal language of domains,
there is no need to prove that a function described by a $\lambda$-term
is continuous, because all terms are shown to denote continous functions
once and for all.  Substructural logics extend this idea to various
forms of monoidal categories, while modal logics describe monads and
comonads.  Recently,
\citet{schreibershulman12cohesive,shulman15realcohesion} proposed using
modal operators to add a notion of \emph{cohesion} to homotopy type
theory/univalent foundations~\citep{,voevodsky06homotopy,uf13hott-book}.
Without going into the precise details of this application, the idea is
to add a triple $\sh{} \la \Flat{} \la \Sharp{}$ of type operators,
where for example $\Sharp A$ is a monad (like a modal possibility
$\Diamond$ or $\bigcirc$), $\Flat A$ is a comonad (like a modal
necessity $\Box$), and there is an adjunction structure between them
(e.g. $\flat{A} \to B$ is the same as $A \to \Sharp{B}$).  This raised
the question of how to best add modalities with these properties to type
theory.

Because other similar applications would have different monads and
comonads with different properties, we would like general tools for
going from a semantic situation of interest to a ``nice'' type
theory/logic for it, e.g. one with cut and identity admissibility and
the subformula property. In previous work~\citep{ls16adjoint}, we
considered the special case of a single-assumption logic, building most
directly on the adjoint logics of
\citet{benton94mixed,bentonwadler96adjoint,reed09adjoint}.  Here we
extend this previous work to the multi-assumption case.  The resulting
framework is quite general and covers many existing intuitionistic
substructural and modal connectives: cartesian, linear, affine,
relevant, ordered, bunched~\citep{ohearnpym99bunched} and
non-associative products and implications; $n$-linear
variables~\citep{reed08namessubstructural}; the comonadic $\Box$ and
linear exponential $!$ and
subexponentials~\citep{nigammiller09subexponentials,danos+93subexponentials};
monadic $\Diamond$ and $\bigcirc$ modalities; and adjoint logic $F$ and
$G$~\citep{benton94mixed,bentonwadler96adjoint,reed09adjoint}, including
the single-assumption 2-categorical version from our previous
work~\citep{ls16adjoint}.  It also supports variations on these, such as
non-monoidal comonads and non-strong monads.  We show that a single,
simple structural~\citep{pfenning94cut} proof of cut (and identity)
admissibility applies to all of these logics, as well as any new logics
that can be described in the framework.
%% While it is not too surprising
%% that this is possible, given that cut proofs for these logics all follow
%% a similar template, it is nonetheless satisfying to codify this pattern
%% as an abstraction.

At a high level, the framework expresses the idea that all of the above
logics are a restriction on how variables can be used in ordinary
structural/cartesian proofs.  We express these restrictions using a
first layer of the logic, which is a simple type theory for what we will
call \emph{modes} and \emph{context descriptors}.  The modes are just a
collection of base types, which we write as $p,q,r$, while a context
descriptor is a term built from variables and constants.  The next layer
is the main logic.  Each proposition of the logic is assigned a mode,
and the basic sequent is \seq{x_1 : A_1, \ldots, x_n : A_n}{\alpha}{C},
where if $A_i$ has mode $p_i$, and $C$ has mode $q$, then $\oftp{x_1 :
  p_1,\ldots, x_n : p_n}{\alpha}{q}$.  
%% In a sequent
%% \seq{\Gamma}{\alpha}{A}, the idea is that $\Gamma$ binds some variables
%% for use both in $\alpha$ and in the derivation.  
$\Gamma$ itself behaves like an ordinary structural/cartesian context,
while the substructural and modal aspects are enforced by the
\emph{term} $\alpha$, which describes how the resources from $\Gamma$
are allowed to be used.  For example, in linear logic/ordered logic/BI,
the context is usually taken to be a multiset/list/tree (respectively).
We represent the multiset or list or tree using a pair of an ordinary
structural context $\Gamma$, together with a term $\alpha$ that
describes the multiset or list or tree structure, labeled with variables
from the ordinary context at the leaves.  We pronounce a sequent
\seq{\Gamma}{\alpha}{A} as ``$\Gamma$ proves $A$ {along,over} $\alpha$''
or ``$\Gamma$ structured according to $\alpha$ proves $A$''.

For example, suppose we have one mode $\dsd{n}$, together with a context
descriptor constant
\[
x : \dsd{n}, y:\dsd{n} \vdash x \odot y : \dsd{n}
\]
Then an example sequent
\[
\seq{x:A, y:B, z:C, w:D}{(y \odot x) \odot z}{E}
\]
should be read as saying that we must prove $E$ using the resources $y$
and $x$ and $z$ (but not $w$) according to the particular tree structure
${(y \odot x) \odot z}$.  If we say nothing else, the framework will
treat $\odot$ as describing a non-associative, linear, ordered context:
if we have a product-like type $A \odot B$ internalizing this context
operation,\footnote{We overload binary operations to refer both to
  context descriptors and propositional connectives, because it is clear
  from whether it is applied to variables $x,y,z$ or propositions
  $A,B,C$ which we mean.}  then we will \emph{not} be able to prove
associativity ($(A \odot B) \odot C \dashv\vdash A \odot (B \odot C)$)
or contraction ($A \vdash A \odot A$) or exchange ($A \odot B \vdash B
\odot A$) etc.

To get from this basic structure to linear or affine or relevant or
cartesian logic, we need to add some structural properties to the
context descriptor term $\alpha$.  We analyze structural properties as
\emph{equations}, or more generally \emph{directed transformations}, on
such terms.  For example, to specify linear logic, we will add a unit
element $1 : \dsd{n}$ together with equations making $(\odot,1)$ into a
commutative monoid:
\[
\begin{array}{c}
x \odot (y \odot z) = (x \odot y) \odot z\\
x \odot 1 = x = 1 \odot x\\
x \odot y = y \odot x
\end{array}
\]
so that the context descriptors ignore associativity and order.  To get
BI, we add an additional commutative monoid $(\times,\top)$ (with
weakening and contraction, as discussed below), so that a BI context
tree $(x:A,y:B);(z:C,w:D)$ can be represented by the ordinary context
$x:A,y:B,z:C,w:D$ with the term $(x \odot y) \times (z \odot w)$
describing the tree.  Because the context descriptors are themselves
ordinary structural/cartesian terms, the same variable can occur more
than once or not at all.  A descriptor such as $x \odot x$ captures the
idea that we can use the \emph{same} variable $x$ twice, expressing
$n$-linear types~\citep{reed08namessubstructural}.  Thus, we can express
contraction for a particular context descriptor $\odot$ as an equation
$x = x \odot x$ (one use of $x$ is the same as two, or $\odot$ is an
idempotent binary operation).  However, weakening cannot be represented
as an equation between context descriptors: an equation $x = 1$ would
trivialize the logic to ordinary intuitionistic logic.  Instead, to
express weakening, we use a directed transformation $x \spr 1$, which is
oriented to allow throwing away an allowed use of $x$, but not creating
an allowed use from nothing.  We refer to these as \emph{structural
  transformations}, to evoke their use in representing the structural
properties of object logics that are embedded in our framework.
Structural transformations are also used to describe relationships
between adjunctions~\citep{ls16adjoint}.

In summary, to specify a particular substructural or modal logic, one
gives constants generating context descriptors $\alpha$, with equations
$\alpha = \beta$ and transformations $\alpha \spr \beta$ expressing
structural properties.  The main sequent $\seq{\Gamma}{\alpha}{A}$
respects the specified structural properties in the sense that when
$\alpha = \beta$, we regard $\seq{\Gamma}{\alpha}{A}$ and
$\seq{\Gamma}{\beta}{A}$ as the same sequent, while when $\alpha \spr
\beta$, there will be an operation that takes a derivation of
\seq{\Gamma}{\beta}{A} to a derivation of \seq{\Gamma}{\alpha}{A}.

A guiding principle of the framework is a meta-level notion of
\emph{structurality over structurality}.  For example, we always have
\emph{weakening over weakening}: if \seq{\Gamma}{\alpha}{A} then
\seq{\Gamma,y:B}{\alpha}{A}, where $\alpha$ itself is weakened with $y$.
This does not prevent encodings of e.g. linear logic: it is permissible
to weaken a derivation of \seq{\Gamma}{x_1 \odot \ldots \odot x_n}{A}
(``use $x_1$ through $x_n$'') to a derivation of \seq{\Gamma,y:B}{x_1
  \odot \ldots \odot x_n}{A} because the (weakened) context descriptor
still disallows the use of $y$.  Similarly, we always have exchange over
exchange and contraction over contraction.  The identity and and cut
principles are analogous:
\[
\infer{\seq{\Gamma,x:A}{x}{A}}{}
\qquad
\infer{\seq{\Gamma}{\subst{\beta}{\alpha}{x}}{B}}
    {\seq{\Gamma,x:A}{\beta}{B} &
     \seq{\Gamma}{\alpha}{A}}
\]
The identity-over-identity principle says that we should be able to
prove $A$ using exactly an assumption $x:A$.  The cut principle says
that the context descriptor for the result of the cut is the
substitution of the context descriptor used to prove $A$ into the one
used to prove $B$.  For example, together with weakening-over-weakening,
this captures the usual cut principle of linear logic, which says that
cutting $\Gamma,x:A \vdash B$ and $\Delta \vdash A$ yields
$\Gamma,\Delta \vdash B$.  If $\Gamma$ binds $x_1,\ldots,x_n$ and
$\Delta$ binds $y_1,\ldots,y_n$, then we will represent the two
derivations to be cut together by sequents with
\[
\begin{array}{l}
\beta = x_1 \odot \ldots \odot x_n \odot x\\
\alpha = y_1 \odot \ldots \odot y_n
\end{array}
\]
so
\[
\beta[\alpha/x] = x_1 \odot \ldots \odot x_n \odot y_1 \odot \ldots \odot y_n
\]
correctly deletes $x$ and replaces it with the variables from $\Delta$.
Moreover, in more subtle situations such as BI, the substitution will
insert the resources used to prove the cut formula in the correct place
in the tree.

The framework has two main logical connectives.  The first,
\F{\alpha}{\Delta}, generalizes the \dsd{F} of adjoint
logic~\citep{bentonwadler96adjoint,reed09adjoint} and the tensor
($\otimes$) of linear logic.  The second, \U{x.\alpha}{\Delta}{A},
generalizes the $\dsd{G}/\dsd{U}$ of adjoint logic and the implication
$A \lolli B$ of linear logic.  Here $\Delta$ is a context of assumptions
$x_i:A_i$, and trivializing the context descriptors (i.e. adding an
equation $\alpha = \beta$ for all $\alpha$ and $\beta$) degenerates
$\F{\alpha}{\Delta}$ into the ordinary intuititionistic product $A_1
\times \ldots \times A_n$, while \U{x.\alpha}{\Delta}{A} becomes $A_1
\to \ldots \to A_n \to A$.  Though we do not give a full
polarized/focused proof theory in this paper, we do prove that \dsd{F}
is left-invertible and \dsd{U} is right-invertible, and we conjecture
that focusing works with the polarization that one would expect based on
these degeneracies ($\F{\alpha}{\Delta^{\mathord{+}}}^{\mathord{+}}$ and
$\U{x.\alpha}{\Delta^{\mathord{+}}}{A^{\mathord{-}}}^{\mathord{-}}$).
In linear logic terms, our \dsd{F} and \dsd{U} cover both the
multiplicatives and exponentials; additives can be added separately by
essentially the usual rules.

Being a very general theory, our framework treats the structural
properties in a general but na\"ive way, allowing an arbitrary
structural transformation to be applied at the non-invertible rules for
$\dsd{F}$ and $\dsd{U}$ and at the leaves of a derivation.  For specific
embedded logics, there will often be a more refined discipline that
suffices---e.g. for cartesian logic, always contract all assumptions at
in all premises, rather than choosing which assumptions to contract.  We
view our framework as a tool for bridging the gap between an intended
semantic situation such as the cohesion example mentioned above (``a
comonad and a monad which are themselves adjoint'') and a proof theory:
the framework gives \emph{some} proof theory for the semantics, and the
placement of structural rules can then be optimized purely in syntax.
To support this mode of use, we give an equational theory on sequent
derivations that identifies different placements of the same structural
rules.  This equational theory is used to prove correctness of such
optimizations not just at the level of provability, but also identity of
derivations---which matters for our intended applications to internal
languages.

Semantically, the logic corresponds to a functor between
\emph{2-dimensional cartesian multicategories} which is a fibration in
various senses.  Multicategories are a generalization of categories
which allow more than one object in the domain, and cartesianness means
that the multiple domain objects are treated structurally.  The
2-dimensionality supplies a notion of morphism between (multi)morphisms,
which correspond to the structural transformations.  The functor
specifies the mode of each proposition and the context descriptor of a
sequent.  The fibration conditions (similar to \citep{hermida,hormann})
specify respect for the structural transformations and the presence of
\dsd{F} and \dsd{U} types.

The remainder of this paper is organized as follows.  FIXME

FIXME: comparison with display logic, L/CLF, what else?  



\newcommand\mor[3]{\ensuremath{#2} \longrightarrow_#1 #3}
\newcommand\C{\ensuremath{\mathcal{C}}}
\newcommand\D{\ensuremath{\mathcal{D}}}
\newcommand\E{\ensuremath{\mathcal{E}}}
\newcommand\deq{\ensuremath{\equiv}}
\newcommand\seq[3]{\ensuremath{#1 \vdash_{#2} #3}}
\newcommand\F[2]{\ensuremath{\dsd{F}_{#1}(#2)}}
\newcommand\U[3]{\ensuremath{\dsd{U}_{#1}(#2 \mid #3)}}
\newcommand\Fsymb[0]{\dsd{F}}
\newcommand\Usymb[0]{\dsd{U}}
\newcommand\tsubst[2]{\ensuremath{#1[#2]}}
\renewcommand\subst[3]{\ensuremath{#1[#2/#3]}}
\newcommand\wftype[2]{\ensuremath{#1 \: \dsd{type}_{#2}}}
\renewcommand\wfctx[2]{\ensuremath{#1 \: \dsd{ctx}_{#2}}}
\newcommand\modeof[1]{\ensuremath{\hat{#1}}}
\newcommand\many[1]{\ensuremath{\overline{#1}}}
\renewcommand{\oftp}[3]{\ensuremath{#1 \, \vdash #2 \, \dcd{:} \, #3}}
\newcommand\FL{\dsd{FL}}
\newcommand\FR{\dsd{FR}}
\newcommand\UL{\dsd{UL}}
\newcommand\UR{\dsd{UR}}

\section{1-Multicategories}

\subsection{Syntax for Cartesian Multicategories}
In an ordinary category, each morphism has a single object as both its
domain and its range.  In a \emph{multicategory}, each morphism has a
list of objects as its domain, and a single object as its range.  In a
\emph{cartesian multicategory}, these multiple objects in the domain are
treated like cartesian (structural) variables. So for a cartesian
multicategory \C\/ we will describe a morphism
\[
\alpha \in \mor{\C}{p_1,\ldots,p_n}{q}
\]
by a term with variables $x_i:p_i$ free, built from generating constants
\dsd{c}:
\[
\begin{array}{llll}
\text{modes} & p & & (constants) \\
\text{mode contexts} & \psi & ::= & \cdot \mid \psi,\tptm{x}{p} \\
\text{mode morphisms} & \alpha,\beta & ::= & x \mid \dsd{c}(\alpha_1,\ldots,\alpha_n) \\
\text{mode substitutions} & \gamma,\delta & ::= & \cdot \mid \gamma,\alpha/x\\
\end{array}
\]
We write $\oftp{\psi}{\alpha}{q}$ for mode morphism well-formedness;
$\psi$ here enjoys (cartesian) structural properties (weakening,
exchange, contraction).  We write $\oftp{\psi}{\gamma}{\psi'}$ for a
familiar structural substitution in the mode multicategory, which
consists of a term $\alpha_i/x_i$ for each variable $\tptm{x_i}{p_i}$ in
$\psi'$, where each $\alpha_i$ satisfies $\oftp{\psi}{\alpha_i}{p_i}$.
Simultaneous substitution into terms is defined in the standard way.

The multicategories needed for this section can be presented by giving a
signature for the constants \dsd{c}, together with some generating
axioms $\deqtm{\psi}{\alpha}{\beta}{q}$ for two mode morphisms
\oftp{\psi}{\alpha,\beta}{q}.  The set of morphisms is defined to be the
quotient of the above terms by the least congruence containing these
generating equations.

\subsection{Logic over a Multicategory}

The rules for the logic are given in Figure~\ref{fig:1logic}.  

explain intended structural properties

explain $\alpha$-conversion

\emph{context descriptor} 

For $\FL$, there are two competing principles: making the rules
obviously structural, and reducing inessential non-determinism.  Here,
we choose the later, and treat the assumption of \F{\alpha}{\Delta}
affinely, removing it from the context when it is used.  It will turn
out that the judgement nonetheless enjoys contraction (over
contraction), because contraction for negatives is built in, and
contraction for positives follows from this and the fact that we can
always reconstruct a positive from what it decays to on the left
(c.f. how purely positive formulas have contraction in linear logic).  

\begin{figure}
\[
\begin{array}{c}
\begin{array}{llll}
\text{Types} & A & ::= & P \mid \F{\alpha}{\Delta} \mid \U{\alpha}{\Delta}{A} \\
\end{array}
\\ \\
\framebox{\wftype{A}{p}}
\\ \\
\infer{\wftype{P}{p}}{\text{(declared in signature)}}
\qquad
\infer{\wftype{\F{\alpha}{\Delta}}{q}}
      {\oftp{\psi}{\alpha}{q} &
        \wfctx{\Delta}{\psi}}
\\ \\
\infer{\wftype{\U{x.\alpha}{\Delta}{A}}{q}}
      {\oftp{\psi,x:q}{\alpha}{p} &
        \wfctx{\Delta}{\psi} &
        \wftype{A}{p}
      }
\\ \\
\framebox{\wfctx{\Gamma}{\psi}}
\\ \\
\infer{\wfctx{\cdot}{\cdot}}{}
\qquad
\infer{\wfctx{\Delta,x:A}{\psi,x:p}}
      {\wfctx{\Delta}{\psi} &
        \wftype{A}{p}}
\\ \\
\framebox{\seq{\Gamma}{\alpha}{A}}
\\ \\
\infer[\dsd{v}]{\seq{\Gamma}{\beta}{P}}
      {x:P \in \Gamma & \beta \equiv x}
\\ \\
\infer[\FR]{\seq{\Gamma}{\beta}{\F{\alpha}{\Delta}}}
      {%% \modeof{\Gamma} \vdash \gamma : \modeof{\Delta} & 
        \beta \deq \tsubst{\alpha}{\gamma} &
        \seq{\Gamma}{\gamma}{\Delta} 
      }
\quad
\infer[\FL]{\seq{\Gamma,x:\F{\Delta}{A},\Gamma'}{\beta}{C}}
      {\seq{\Gamma,\Gamma',\Delta}{\subst \beta {\alpha}{x}}{C}}
%% \infer{\seq{\Gamma}{\beta}{C}}
%%       {{x}:{\F{\alpha}{\Delta}} \in \Gamma & 
%%         \oftp{\modeof{\Gamma},{x'} : {\modeof{\F{\alpha}{\Delta}}}}{\beta'}{\modeof{C}} &
%%         \beta \deq \tsubst{\beta'}{x/x'} &
%%         \seq{\Gamma,\Delta}{\subst {\beta'} {\alpha}{x'}}{C}}
\\ \\
\infer[\UR]{\seq{\Gamma}{\beta}{\U{x.\alpha}{\Delta}{A}}}
      {\seq{\Gamma,\Delta}{\subst{\alpha}{\beta}{x}}{A}}
\\ \\
\infer[\UL]{\seq{\Gamma}{\beta}{C}}
      {x:\U{x.\alpha}{\Delta}{A} \in \Gamma & 
        \beta \deq \subst{\beta'}{\tsubst{\alpha}{\gamma}}{z} &
        \seq{\Gamma}{\gamma}{\Delta} &
        \seq{\Gamma,\tptm{z}{A}}{\beta'}{C}
      }
\\ \\
\framebox{\seq{\Gamma}{\gamma}{\Delta}}
\\ \\
\infer[\cdot]{\seq{\Gamma}{\cdot}{\cdot}}
      {}
\qquad
\infer[\_,\_]{\seq{\Gamma}{\gamma,\alpha/x}{\Delta,x:A}}
      {\seq{\Gamma}{\gamma}{\Delta} &
       \seq{\Gamma}{\alpha}{A}
      }
\end{array}
\]    
\caption{1-Multicategorical Logic}
\label{fig:1logic}
\hrule
\end{figure}

\subsection{Examples}

\subsubsection{Non-associative Logic}

For a mode theory with one mode \dsd{m}, constants
\[
\begin{array}{c}
\oftp{x : \dsd{m}, y : \dsd{m}}{x \odot y}{\dsd{m}}\\
\oftp{\cdot}{1}{\dsd{m}}
\end{array}
\]
and no equations, we get a non-associative context.  Writing $A \odot B$
for \F{x \odot y}{x:A,y:B}, we \emph{cannot}, for example, derive
\[
A \odot (B \odot C) \vdash (A \odot B) \odot C
\]
To derive 
\begin{footnotesize}
\[{\seq{a:\F{x \odot p}{x:A,p:\F{y \odot z}{y:B,z:C}}}
  {a}
  {\F{q \odot z}{q:\F{x \odot y}{x:A,y:B},z:C}}}
\]
\end{footnotesize}
we can start by apply \FL\/ twice to reduce the sequent to
\[
\seq{x:A,y:B,z:C}{x \times (y \odot z)}{{\F{q \odot z}{q:\F{x \odot y}{x:A,y:B},z:C}}}
\]
To apply \FR, we need to find a $\gamma$ such that ${x \times (y \odot
  z)} \deq (q \odot z)[\gamma]$.  In the absence of any equational
axioms, the only possible choice is $x/q, (y \odot z)/z$, so we need
to show
\[
\seq{x:A,y:B,z:C}{x}{A \odot B}
\qquad
\seq{x:A,y:B,z:C}{y \odot z}{C}
\]
which is not possible because the context is not divded correctly.  

\subsubsection{Ordered Logic}

If we add the axioms
\[
\begin{array}{c}
x \odot (y \odot z) \deq (x \odot y) \odot z\\
x \odot 1 \deq x \deq 1 \odot x
\end{array}
\]
then we get ordered logic (which has none of exchange, weakening, and
contraction---a monoidal but not symmetric monoidal product).  In
ordered logic, we can complete the above proof of associativity of
$\odot$: we need to find a $\gamma$ such that ${x \times (y \odot z)}
\deq (q \odot z)[\gamma]$, we can now choose $(x \odot y)/q, z/z)$, so
the subgoals are
\[
\seq{x:A,y:B,z:C}{x \odot y}{A \odot B}
\qquad
\seq{x:A,y:B,z:C}{z}{C}
\]
The latter is identity, and the former is a further \FR\/ and then
identities: 
\[
\infer{\seq{x:A,y:B,z:C}{x \odot y}{\F{x' \odot y'}{x':A,y':B}}}
      { \begin{array}{l}
          x \otimes y \deq (x' \odot y')[x/x',y/y'] \\
          \seq{x:A,y:B,z:C}{x}{A} \\
          \seq{x:A,y:B,z:C}{y}{B} 
        \end{array}
      }
\]

The expected rules for the left and right implications of ordered logic are
\[
\begin{array}{l}
\infer{\Gamma \vdash A \rightharpoonup B}
      {\Gamma,A \vdash B}
\qquad
\infer{\Gamma,A \rightharpoonup B,\Delta,\Gamma' \vdash C}
      {\Delta \vdash A &
       \Gamma,B,\Gamma' \vdash C
      }
\\ \\
\infer{\Gamma \vdash A \leftharpoonup B}
      {A,\Gamma \vdash B}
\qquad
\infer{\Gamma,\Delta,A \leftharpoonup B,\Gamma' \vdash C}
      {\Delta \vdash A &
       \Gamma,B,\Gamma' \vdash C
      }
\end{array}
\]

We can define these by 
\[
\begin{array}{l}
A \rightharpoonup B := \U{c.c \odot x}{x:A}{B}\\
A \leftharpoonup B := \U{c.x \odot c}{x:A}{B}
\end{array}
\]
which have the expected right rules, putting $x$ on the left or right of
the context descriptor:
\[
\infer{\seq{\Gamma}{\beta}{\U{c.c \odot x}{x:A}{B}}}
      {\seq{\Gamma,x:A}{\beta \odot x}{B}}
\qquad
\infer{\seq{\Gamma}{\beta}{\U{c.c \odot x}{x:A}{B}}}
      {\seq{\Gamma,x:A}{x \odot \beta}{B}}
\]
The left rules are
\[
\begin{array}{l}
\infer{\seq{\Gamma} {\beta} {C}}
      {c:\U{c.c \odot x}{x:A}{B} \in \Gamma &
       \beta \deq \beta'[c \odot \alpha/z] &
       \seq{\Gamma}{\alpha}{A} &
       \seq{\Gamma,z:A}{\beta'}{C}
      }
\\ \\ 
\infer{\seq{\Gamma} {\beta} {C}}
      {c:\U{c.x \odot c}{x:A}{B} \in \Gamma &
       \beta \deq \beta'[\alpha \odot c/z] &
       \seq{\Gamma}{\alpha}{A} &
       \seq{\Gamma,z:A}{\beta'}{C}
      }
\end{array}
\]

Suppose that $\beta$ is (up to associativity and unit) of the form $x_1
\odot \ldots \odot x_n$ for distinct variables $x_i$.  Then the
equality premise for the $\rightharpoonup$ rule will reassociate $\beta$
as $\beta_1 \odot (c \odot \alpha) \odot \beta_2$, where $\alpha$
plays the role of $\Delta$ in the ordered logic rule---the resources
used to prove $A$, which occur to the right of the implication being
eliminated.  The resources $\beta'$ used for the continuation are
``$\beta$ with $c \otimes \alpha$ replaced by the result of the
implication,'' as desired.  While $c$ and any variables used in $\alpha$
are still in $\Gamma$, permission to use them has been removed from
$\beta'$---and there is no way to restore such permissions in this mode
theory.  The rule for $\leftharpoonup$ is the same, but with $\alpha$ on
the opposite side of $c$.

FIXME: get more formal?

FIXME: comment on non-linear $\beta$?

\subsubsection{Linear Logic}

If extend the mode theory with the equation
\[
x \otimes y \deq y \otimes x
\]
(and switich notation from $\odot$ to $\otimes$) then we get linear
logic.  
For example, we can derive 
{\seq{p : A \otimes B}{p}{B \otimes A}} in the expected way:
\[
\infer[\FL]
      {\seq{p : \F{x \otimes y}{x:A,y:B}}{p}{\F{x \otimes y}{x:B,y:A}}}
      {\infer[\FR]
        {\seq{x:A,y:B}{x \otimes y}{\F{x' \otimes y'}{x':B,y':A}}}
        {x \otimes y \deq (x' \otimes y')[y/x',x/y'] &
         \seq{x:A,y:B}{y}{B} &
         \seq{x:A,y:B}{x}{A}
        }}
\]

For this mode theory, \U{c.c \odot x}{x:A}{B} and \U{c.c \odot
  x}{x:A}{B} are isomorphic, and both represent $A \multimap B$.

%% The linear logic $\multimap$-left rule (where contexts are implicitly
%% treated modulo exchange) is
%% \[
%% \infer{\Gamma,\Delta,A \multimap B \vdash C}
%%       {\Delta \vdash A &
%%        \Gamma,B \vdash C}
%% \]
%% We have
%% \[
%% \infer{\seq{\Gamma} {\beta} {C}}
%%       {c:\U{c.c \otimes x}{x:A}{B} \in \Gamma &
%%        \beta \deq \beta'[c \odot \alpha/z] &
%%        \seq{\Gamma}{\alpha}{A} &
%%        \seq{\Gamma,z:A}{\beta'}{C}
%%       }
%% \]

\subsubsection{Zero-use variables}

\subsubsection{Multi-use variables}

\subsubsection{Relevant Logic}

\subsubsection{``Non-cartesian BI''} Two monoids

\subsubsection{Adjoint Logic} Non-tensor-preserving left adjoint

\subsection{Properties}

Define the \emph{size} of a derivation of \seq{\Gamma}{\alpha}{A} or
\seq{\Gamma}{\gamma}{\Delta} to be the number of inference rules for
these judgements $(\dsd{v},\FL, \FR, \UL, \UR,
\cdot, \_,\_)$ used in it (i.e., the evidence that variables are in a
context and the evidence for equations does not contribute to the size).
Sizes are necessary for the cut proof, where we sometimes weaken or
invert a derivation before applying the inductive hypothesis.

\begin{lemma}[Respect for Equality] ~ \label{lemma:respecteq}
\begin{enumerate}
\item If \seq{\Gamma}{\beta}{A} and $\beta' \deq \beta$ then
\seq{\Gamma}{\beta'}{A}, and the resulting derivation has the same size
as the given one.
\item If \seq{\Gamma}{\gamma}{\Delta} and $\gamma' \deq \gamma$ then
  \seq{\Gamma}{\gamma'}{\Delta}, and the resulting derivation has the
  same size as the given one.
\end{enumerate}
\end{lemma}
\begin{proof}
Mutual induction on the given derivation.  The cases for \dsd{v} and
$\FR$ and $\UL$ are immediate (with no use of the inductive
hypothesis) by composing with the equality in the premise of the rule.
The cases for $\FL$ and $\UR$ use the inductive hypothesis,
along with congruence for equality of mode morphism to show that
$\subst{\beta}{\alpha}{x} \deq \subst{\beta'}{\alpha}{x}$ or
$\subst{\alpha}{\beta}{x} \deq \subst{\alpha}{\beta'}{x}$.  The cases
for substitutions rely on the fact that no generating equalities for
mode substitutions are allowed, so if $\gamma' \deq \cdot$ then
$\gamma'$ is literally $\cdot$, and $(-,-)$ is injective (if $\gamma'
\deq (\gamma_1,\alpha_2/x)$, then $\gamma'$ is $(\gamma_1',\alpha_2'/x)$
with $\gamma_1' \deq \gamma_1$ and $\alpha_2' = \alpha_2$); this is
enough to use the inductive hypotheses in the cons case.  
\end{proof}

\begin{lemma}[Weakening over weakening] \label{thm:weakening} ~
\begin{enumerate}
\item If \seq{\Gamma,\Gamma'}{\alpha}{C} then
\seq{\Gamma,\tptm{z}{A},\Gamma'}{\alpha}{C}, and the resulting
derivation has the same size as the given one.  
\item If \seq{\Gamma,\Gamma'}{\gamma}{\Delta} then
\seq{\Gamma,\tptm{z}{A},\Gamma'}{\gamma}{\Delta}, and the resulting
derivation has the same size as the given one.  
\item If \seq{\Gamma,\Gamma''}{\alpha}{C} then
\seq{\Gamma,\Gamma',\Gamma''}{\alpha}{C}, and the resulting
derivation has the same size as the given one.  
\end{enumerate}
\end{lemma}
\begin{proof}
It is implicit that the mode morphism $\alpha$ is weakend with $z$ in
the conclusion.  Intuitively, weakening holds because the contexts
$\Gamma$ are treated like ordinary structural contexts in all of the
rules---they are fully general in every conclusion, and the premises
check membership or extend them---and because weakening holds for mode
morphisms and equalities of mode morphisms.  Formally, the first two
parts are proved by mutual induction; each case is either immediate
or follows from weakening for the mode morphisms, weakening for
equalities of mode morphisms, and the inductive hypotheses.  The third
part is proved by induction over $\Gamma'$, repeatedly applying the
first part.  
%% The case for the hypothesis rule is immediate, because
%% $\Gamma$ may contain variables other than $x$.  The case for
%% \Fsymb-right follows from weakening for the mode morphisms, and
%% equations between mode morphisms, and the inductive hypothesis for
%% substitutions.  The case for \Fsymb-left follows from the inductive
%% hypothesis, as does the case for \Usymb-right.  
\end{proof}

\begin{lemma}[Exchange over exchange]
If \seq{\Gamma,x:A,y:B,\Gamma'}{\alpha}{C} then
\seq{\Gamma,y:B,x:A,\Gamma'}{\alpha}{C}, and the resulting derivation
has the same size as the given one.  (And similarly for substitutions,
and exchange can be iterated).  
\end{lemma}
\begin{proof} Analogous to weakening.  
\end{proof}

\begin{theorem}[Identity] ~
\begin{enumerate}
\item If $x:A \in \Gamma$ then $\seq{\Gamma}{x}{A}$.
\item If $\oftp{\modeof{\Gamma}}{\rho}{\modeof{\Delta}}$ is a
  variable-for-variable mode substitution such that $x:A \in \Delta$
  implies $\rho(x) : A \in \Gamma$, then $\seq{\Gamma}{\rho}{\Delta}$.
\end{enumerate}
\end{theorem}

\begin{proof}
The standard proof by induction on $A$ (mutually with $\Delta$) applies:
the case for atomic propositions is a rule, and for the other
connectives, apply the invertible and then non-invertible rule to reduce
the problem to the inductive hypotheses.  More specifically, identity
for $P$ is a rule.  In the case for \F{\alpha}{\Delta}, with $\Gamma =
\Gamma_1,x:\F{\alpha}{\Delta},\Gamma_2$, we reduce it to the inductive
hypothesis as follows:
\[
\infer[\FL]{\seq{\Gamma_1,x:\F{\alpha}{\Delta},\Gamma_2}{x}{\F{\alpha}{\Delta}}}
      {\infer[\FR]{\seq{\Gamma_1,\Gamma_2,\Delta}{\alpha}{\F{\alpha}{\Delta}}}
                        {\alpha \deq \tsubst{\alpha}{\vec{x/x}} &
                        \seq{\Gamma_1,\Gamma_2,\Delta}{\vec{x/x}}{\Delta}
                        }}
\]
In the second premise, the $\vec{x/x}$ substitution for each $x \in
\Delta$ is a variable-for-variable substitution, so the second part of
the inductive hypothesis applies.  
The case for \Usymb\/ is similar
\[
\infer[\UR]{\seq{\Gamma}{x}{\U{\alpha}{\Delta}{A}}}
      {\infer[\UL]{\seq{\Gamma,\Delta}{\alpha}{A}}
                        {\alpha \deq \subst{x}{\tsubst{\alpha}{\vec{x/x}}}{x} &
                        \seq{\Gamma,\Delta}{\vec{x/x}}{\Delta} &
                        \seq{\Gamma,x:A}{x}{A}
                        }}
\]

For the second part, the hypothesis of the lemma asks that every
variable in $\Delta$ is associated by $\rho$ with a variable of the same
type in $\Gamma$; this is enough to iterate the first part of the
lemma for each position in $\Delta$.  Specifically, the case where
$\Delta$ is the empty context $\cdot$ is a rule. In the case for a cons
$\Delta,y:A$, we have
\oftp{\modeof{\Gamma}}{\rho}{(\modeof{\Delta},y:\modeof{A})} which means
$\rho$ must be of the form $\rho',x/y$ where $x \in \modeof{\Gamma}$ and
$\rho'$ is a variable-for-variable substitution.  Because $\rho$ was
type-preserving, $x : A \in \Gamma$ and $\rho'$ is type-preserving, so
we obtain the result from the inductive hypotheses as follows:
\[
\infer{\seq{\Gamma}{\rho,x/y}{\Delta,y:A}}
      {\seq{\Gamma}{\rho}{\Delta} & 
       \seq{\Gamma}{x}{A}
      }
\]
\end{proof}

\begin{lemma}[Left-invertibility of \Fsymb] \label{lemma:Finv}
If $\D :: \seq{\Gamma_1,x_0:\F{\alpha_0}{\Delta_0},\Gamma_2}{\beta}{C}$
and then there is a derivation $D' ::
\seq{\Gamma_1,\Gamma_2,\Delta_0}{\subst{\beta}{\alpha_0}{x_0}}{C}$ and
$size(\D') \le size(\D)$ (and analogously for substitutions).
\end{lemma}

\begin{proof}
By induction on \D.  We write $\Gamma$ for
the whole context $\Gamma_1,x_0:\F{\alpha_0}{\Delta_0},\Gamma_2$.

In the case for \dsd{v}, $x : P \in
\Gamma_1,x_0:\F{\alpha_0}{\Delta_0},\Gamma_2$ cannot be equal to $x_0 :
\F{\alpha_0}{\Delta_0}$ because the types conflict, so we can reapply
the \dsd{v} rule in $\Gamma_1,\Gamma_2,\Delta$.

In the case for $\FR$, we have
\[
\infer{\seq{\Gamma}{\beta}{\F{\alpha}{\Delta}}}
      {\beta \deq \tsubst{\alpha}{\gamma} &
        \seq{\Gamma}{\gamma}{\Delta} 
      }
\]
with $x_0 : \F{\alpha_0}{\Delta_0} \in \Gamma$.  By the inductive
hypothesis we get
\seq{\Gamma_1,\Gamma_2,\Delta_0}{\subst{\gamma}{\alpha_0}{x}}{\Delta}.  Because
$x_0$ is not free in $\alpha$,
$\subst{(\tsubst{\alpha}{\gamma})}{\alpha_0}{x_0} =
\tsubst{\alpha}{\subst{\gamma}{\alpha_0}{x_0}}$, so we can reapply \FR:
\[
\infer{\seq{\Gamma_1,\Gamma_2}{\subst{\beta}{\alpha_0}{x_0}}{\F{\alpha}{\Delta}}}
      {{\subst{\beta}{\alpha_0}{x_0}} \deq \tsubst{\alpha}{\subst{\gamma}{\alpha_0}{x_0}} &
        \seq{\Gamma_1,\Gamma_2,\Delta_0}{\subst{\gamma}{\alpha_0}{x}}{\Delta}
      }
\]
Both the input and the output have size 1 more than the size of their
subderivations, and the output subderivation is no bigger than the input
by the inductive hypothesis.

In the case for $\FL$
\[
\infer[\FL]{\seq{\Gamma_1',x:\F{\alpha}{\Delta},\Gamma_2'}{\beta}{C}}
      {\deduce{\seq{\Gamma_1',\Gamma_2',\Delta}{\subst \beta {\alpha}{x}}{C}}{\D}}
\]
with $\Gamma_1,x_0 : \F{\alpha_0}{\Delta_0},\Gamma_2 =
\Gamma_1',x:\F{\alpha}{\Delta},\Gamma_2'$, we distinguish cases on
whether $x = x_0$ or not.  If they are the same (i.e. we have hit a left
rule on $x_0$), then $\alpha_0 = \alpha$ and $\Delta_0 = \Delta$ and
\D\/ is the result, and the size is 1 less than the size of the input.
If they are different, then (because $x_0$ is somewhere in
$\Gamma_1',\Gamma_2'$) by the inductive hypothesis we have a derivation
\[
\D' :: {\seq{(\Gamma_1',\Gamma_2')-x_0,\Delta,\Delta_0}{\subst{\subst \beta {\alpha}{x}}{\alpha_0}{x_0}}{C}}
\]
that is no bigger than \D.  Because $x_0$ is from $\Gamma$ and not
$\Delta$, it does not occur in $\alpha$, so 
\[
{\subst{\subst \beta {\alpha}{x}}{\alpha_0}{x_0}} = 
{\subst{\subst \beta {\alpha_0}{x_0}}{\alpha}{x}}
\]
By (iterating) exchange, we get a derivation 
\[
\D'' :: {\seq{(\Gamma_1',\Gamma_2')-x_0,\Delta_0,\Delta}{\subst{\subst \beta {\alpha_0}{x_0}}{\alpha}{x}}{C}}
\]
whose size is the same as $\D'$ and so no bigger than $\D$.  Applying
$\FL$ to $\D''$ (using the fact that
$(\Gamma_1',x:\F{\alpha}{\Delta},\Gamma_2')-x_0 = \Gamma_1,\Gamma_2$)
derives $\seq{\Gamma_1,\Gamma_2}{\subst{\beta}{\alpha_0}{x_0}}{C}$, and
the size is no bigger than the size of the input.

In the case for $\UR$,
\[
\infer{\seq{\Gamma}{\beta}{\U{x.\alpha}{\Delta}{A}}}
      {\seq{\Gamma,\Delta}{\subst{\alpha}{\beta}{x}}{A}}
\]
the inductive hypothesis gives a
$\D' :: \seq{\Gamma_1,\Gamma_2,\Delta,\Delta_0}{\subst{\subst{\alpha}{\beta}{x}}{\alpha_0}{x_0}}{A}$
and (iterated) exchange gives 
$\D'' ::
\seq{\Gamma_1,\Gamma_2,\Delta_0,\Delta}{\subst{\subst{\alpha}{\beta}{x}}{\alpha_0}{x_0}}{A}$,
both no bigger than \D.  Because $x_0$ is in $\Gamma$ and not $\Delta$,
it is not free in $\alpha$, so 
\[
{\subst{\subst{\alpha}{\beta}{x}}{\alpha_0}{x_0}} = {\subst{\alpha}{\subst{\beta}{\alpha_0}{x_0}}{x}}
\]
Thus, we can derive
\[
\infer{\seq{\Gamma_1,\Gamma_2,\Delta_0}{\subst{\beta}{\alpha_0}{x_0}}{\U{x.\alpha}{\Delta}{A}}}
      {\deduce{\seq{\Gamma_1,\Gamma_2,\Delta_0,\Delta}{\subst{\alpha}{\subst{\beta}{\alpha_0}{x_0}}{x}}{A}}{\D''}}
\]

In the case for $\UL$, 
\[
\infer{\seq{\Gamma}{\beta}{C}}
      {x:\U{x.\alpha}{\Delta}{A} \in \Gamma & 
        \beta \deq \subst{\beta'}{\tsubst{\alpha}{\gamma}}{z} &
        \seq{\Gamma}{\gamma}{\Delta} &
        \seq{\Gamma,\tptm{z}{A}}{\beta'}{C}
      }
\]
we know that $x$ is different that $x_0$ because the types conflict.
The inductive hypotheses give no-bigger derivations of
\[
\seq{\Gamma_1,\Gamma_2\Delta_0}{\subst{\gamma}{\alpha_0}{x_0}}{\Delta} \qquad \seq{\Gamma_1,\Gamma_2,\tptm{z}{A},\Delta_0}{\subst{\beta'}{\alpha_0}{x_0}}{C}
\]
and the latter can be exchanged to
\[
\seq{\Gamma_1,\Gamma_2,\Delta_0,\tptm{z}{A}}{\subst{\beta'}{\alpha_0}{x_0}}{C}
\]
again without increasing the size.  Thus, we can produce
\[
\infer{\seq{\Gamma_1,\Gamma_2,\Delta_0}{\subst{\beta}{\alpha_0}{x}}{C}}
      {\begin{array}{l}
          x:\U{x.\alpha}{\Delta}{A} \in \Gamma_1,\Gamma_2,\Delta_0 \\
          {\subst{\beta}{\alpha_0}{x}} \deq \subst{{\subst{\beta'}{\alpha_0}{x_0}}}{\tsubst{\alpha}{{\subst{\gamma}{\alpha_0}{x_0}}}}{z}\\
          \seq{\Gamma_1,\Gamma_2,\Delta_0}{\subst{\gamma}{\alpha_0}{x_0}}{\Delta} \\
          \seq{\Gamma_1,\Gamma_2,\Delta_0,\tptm{z}{A}}{\subst{\beta'}{\alpha_0}{x_0}}{C}
        \end{array}
      }
\]
where equality is the composition of the \subst{-}{\alpha_0}{x_0}
substitution into the given equality, and rearranging the substitution
(note that $x_0$ does not occur in $\alpha$):
\[
\begin{array}{ll}
\subst{\beta}{\alpha_0}{x_0} & \deq
\subst{\subst{\beta'}{\tsubst{\alpha}{\gamma}}{z}}{\alpha_0}{x_0} 
= 
\subst{\subst{\beta'}{\alpha_0}{x_0}}{\subst{\tsubst{\alpha}{\gamma}}{\alpha_0}{x_0}}{z}
\\
& =
\subst{\subst{\beta'}{\alpha_0}{x_0}}{\tsubst{\alpha}{\subst{\gamma}{\alpha_0}{x_0}}}{z} 
\end{array}
\]

The case for $\cdot$ is immediate.  The case for $\_,\_$ follows from
the two inductive hypotheses, because
$\subst{(\gamma,\alpha/x)}{\alpha_0}{x_0} =
{(\subst{\gamma}{\alpha_0}{x_0},\subst{\alpha}{\alpha_0}{x_0}/x)}$.
\end{proof}


\begin{theorem}[Cut] ~
\begin{enumerate} 
\item  If $\seq{\Gamma,\Gamma'}{\alpha_0}{A_0}$ and $\seq{\Gamma,x_0:A_0,\Gamma'}{\beta}{B}$ 
then $\seq{\Gamma,\Gamma'}{\beta[\alpha_0/x_0]}{B}$ 
\item If $\seq{\Gamma,\Gamma'}{\alpha_0}{A_0}$ and $\seq{\Gamma,x_0:A_0,\Gamma'}{\gamma}{\Delta}$ 
then $\seq{\Gamma,\Gamma'}{\gamma[\alpha_0/x_0]}{\Delta}$ 
\item If $\seq{\Gamma}{\gamma}{\Delta}$ and 
\seq{\Gamma,\Delta}{\beta}{C}
then \seq{\Gamma}{\tsubst{\beta}{\gamma}}{C}.  
\end{enumerate}
\end{theorem}

\begin{proof}
Induction ordering: Part 1 and 2: cut formula, then simultaneous on the
size of \D\/ and \E\/.  Part 3:

Part 1: There are 5 rules, so 25 pairs of final rules.  

\begin{itemize}
\item (5 pairs) Any rule and identity
\[
\deduce{\seq{\Gamma,\Gamma'}{\alpha_0}{A_0}}{\D} \qquad \infer{\seq{\Gamma,x:A,\Gamma'}{z}{Q}}{z:Q \in (\Gamma,x:A,\Gamma')}
\]
There two subcases, depending on whether the variable being cut for is
$z$ or not.  If $z$ is $x_0$ and $A_0$ is $Q$, then \D\/ has the desired
conclusion \seq{\Gamma}{\alpha}{Q}.  If not, then $z:Q \in \Gamma,\Gamma'$, so
the hypothesis rule applies to give \seq{\Gamma,\Gamma'}{z}{Q}.  

\item (5 pairs) Any rule and $\FR$ (right-commutative)
\[
\deduce{\seq{\Gamma,\Gamma'}{\alpha_0}{A_0}}{\D} \qquad
\infer{\seq{\Gamma,x_0:A_0,\Gamma'}{\beta}{\F{\alpha}{\Delta}}}
      {%% \modeof{\Gamma} \vdash \gamma : \modeof{\Delta} & 
        \beta \deq \tsubst{\alpha}{\gamma} &
        \deduce{\seq{\Gamma,x_0:A_0,\Gamma'}{\gamma}{\Delta}}{\E}
      }
\]
By the inductive hypothesis, cutting into \D\/ into \E\/ gives
\seq{\Gamma,\Gamma'}{\subst{\gamma}{\alpha_0}{x_0}}{\Delta}.  By
congruence, $\subst{\beta}{\alpha_0}{x_0} \deq
\subst{\tsubst{\alpha}{\gamma}}{\alpha_0}{x_0}$.  Since $\gamma$ is a
total substitution for all variables in \modeof{\Delta},
$\subst{\tsubst{\alpha}{\gamma}}{\alpha_0}{x} =
\tsubst{\alpha}{\subst{\gamma}{\alpha_0}{x}}$, so
\subst{\beta}{\alpha_0}{x_0} \deq
\tsubst{\alpha}{\subst{\gamma}{\alpha_0}{x}}.  Thus we can reapply the
$\FR$ rule to get
\seq{\Gamma,\Gamma'}{\subst{\beta}{\alpha_0}{x_0}}{\F{\alpha}{\Delta}}.

\item (5 pairs) Any rule and $\UR$ (right-commutative).    
\[
\deduce{\seq{\Gamma,\Gamma'}{\alpha_0}{A_0}}{\D} \qquad
\infer{\seq{\Gamma,x_0:A_0,\Gamma'}{\beta}{\U{x.\alpha}{\Delta}{A}}}
      {\deduce{\seq{\Gamma,x_0:A_0,\Gamma',\Delta}{\subst{\alpha}{\beta}{x}}{A}}{\E}}
\]
The inductive cut of \D\/ into \E\/ gives 
\[
\seq{\Gamma,\Gamma',\Delta}{\subst{\subst{\alpha}{\beta}{x}}{\alpha_0}{x_0}}{A}
\]
Because the variables from $\modeof{\Gamma},\modeof{\Gamma'}$ occur only
in $\beta$, not in $\alpha$, this substitution equals 
{\subst{\alpha}{\subst{\beta}{\alpha_0}{x_0}}{x}} so reapplying the
$\UR$ rule
derives 
{\seq{\Gamma,\Gamma'}{\subst{\beta}{\alpha_0}{x_0}}{\U{x.\alpha}{\Delta}{A}}}.   

\item (2 additional pairs, plus 3 overlapping with above) $\FL$ and
  any rule (left commutative).  

There is one subtlety in this case.  The usual strategy for a left rule
against an arbitrary \E is to push $\E$ into the ``continuation'' of the
\Fsymb-left on $x$.  However, as discussed above, our left rule for
\Fsymb eagerly inverts \emph{all} occurences of $x$, while $\E$ itself
also has $x$ in scope.  Thus, we use Lemma~\ref{lemma:Finv} to pull the
left-inversion to the bottom of \E, and then push that into \D.  On
proof terms, this corresponds to making all references to $x$ in \E
instead refer to the results of the case-analysis at the bottom of $\D$.
This subtlety could be avoided by building contraction into $\FL$, as
discussed above.

Formally, we have
\[
\begin{array}{c}
\infer{\seq{\Gamma,\Gamma'}{\alpha_0}{A_0}}
      {{x}:{\F{\alpha}{\Delta}} \in \Gamma,\Gamma' &
        \deduce{\seq{((\Gamma,\Gamma')-x),\Delta}{\subst {\alpha_0} {\alpha}{x}}{A_0}}{\D}}
\\ \\
\deduce{\seq{\Gamma,x_0:A_0,\Gamma'}{\beta}{C}}{\E}
\end{array}
\]

By left invertibility on \E, we obtain (note that $x \neq x_0$ because
$x_0$ is added to the context in the right-hand derivation) a derivation
$\E'$ of
{\seq{(\Gamma,x:A_0,\Gamma')-x,\Delta}{\subst{\beta}{\alpha}{x}}{C}} that is
no bigger than $\E$.  Because the cut formula is the same, and $\E'$ has
the same size as \E\/, and \D\/ is smaller than the given derivation of
$A_0$, we can apply the inductive hypothesis to cut $\D$ and $\E'$ to
get
\[
{\seq{(\Gamma,\Gamma')-x,\Delta}{\subst{\subst{\beta}{\alpha}{x}}{\subst{\alpha_0}{\alpha}{x}}{x_0}}{C}}.
\]
Commuting substitutions gives
\[
{\subst{{\subst{\beta}{\alpha}{x}}}{\subst{\alpha_0}{\alpha}{x}}{x_0}} = \subst {\beta[\alpha_0/x_0]}{\alpha}{x}
\]
so we can reapply $\FL$ to get
\[
\infer{\seq{\Gamma,\Gamma'}{\beta[\alpha_0/x_0]}{C}}
      {\seq{((\Gamma,\Gamma')-x),\Delta}{\subst {(\beta[\alpha_0/x_0])} {\alpha}{x}}{C}}
\]


\item (2 additional pairs, plus 3 overlapping with above) $\UL$ and any rule (left commutative)
In this case, $x:\U{\alpha}{\Delta}{A} \in \Gamma,\Gamma'$ and
we have
\[
\begin{array}{c}
\infer{\seq{\Gamma,\Gamma'}{\alpha_0}{A_0}}
      {\alpha_0 \deq \subst{\alpha_0'}{\tsubst{\alpha}{\gamma}}{z} &
       \deduce{\seq{\Gamma,\Gamma'}{\gamma}{\Delta}}{\D_1} &
       \deduce{\seq{\Gamma,\Gamma',z:A}{\alpha_0'}{A_0}}{\D_2}
      }
\\ \\
\deduce{\seq{\Gamma,x_0:A_0,\Gamma'}{\beta}{B}}{\E}
\end{array}
\]

Weakening \E with $z$ and then cutting $\D_2$ and $\E$ by the inductive
hypothesis (which applies because $\D_2$ is smaller and weakening does
not change the size) gives
\[
\deduce{\seq{\Gamma,\Gamma',z:A}{\subst{\beta}{\alpha_0'}{x_0}}{B}}{\D_2'}
\]
Thus, we have the first, third, and fourth premises of
\[
\infer{\seq{\Gamma,\Gamma'}{\subst{\beta}{\alpha_0}{x_0}}{A_0}}
      {\begin{array}{l}
          x:\U{\alpha}{\Delta}{A} \in \Gamma,\Gamma' \\
          {\subst{\beta}{\alpha_0}{x_0}} \deq \subst{\subst{\beta}{\alpha_0'}{x_0}}{\tsubst{\alpha}{\gamma}}{z} \\
       {\seq{\Gamma,\Gamma'}{\gamma}{\Delta}} \\
       {\seq{\Gamma,\Gamma',z:A}{\subst{\beta}{\alpha_0'}{x_0}}{B}}
        \end{array}
      }
\]
The equation is proved by
\[
     {\subst{\beta}{\alpha_0}{x_0}} 
\deq \subst{\beta}{\subst{\alpha_0'}{\tsubst{\alpha}{\gamma}}{z}}{x0} = \subst{\subst{\beta}{\alpha_0'}{x_0}}{\tsubst{\alpha}{\gamma}}{z}
\]
where the first step is by congruence with $\beta$ on the 
equality about $\alpha_0$ assumed for the case, and the second is by
properties of substitution ($z$ is not free in $\beta$).  

\item (3 pairs) Right rule or identity and $\FL$ (principal or
  right-commutative).  

Suppose the right-hand derivation ends with $\FL$, and the left-hand
derivation is either a right rule or identity (\dsd{v}) (the cases for
left-rules were covered above).  

We distinguish cases on whether the \FL\/ case-analyzes $x_0$ or not.  If
it does, then, because $A_0 is \F{\alpha}{\Delta}$, the left-hand
derivation must be \FR, and we have a principal cut
\[
\infer{\seq{\Gamma,\Gamma'}{\alpha_0}{\F{\alpha}{\Delta}}}
      {  
        \alpha_0 \deq \tsubst{\alpha}{\gamma} &
        \deduce{\seq{\Gamma,\Gamma'}{\gamma}{\Delta}}{\D}
      }
\qquad
\infer{\seq{\Gamma,x_0:\F{\alpha}{\Delta},\Gamma'}{\beta}{C}}
      {\deduce{\seq{\Gamma,\Gamma',\Delta}{\subst{\beta}{\alpha}{x_0}}{C}}
              {\E}}
\]
Using the inductive hypothesis part 3 to cut \D and \E ($\Delta$ is a
subformula of the original cut formula \F{\alpha}{\Delta}) gives
\[
\seq{\Gamma,\Gamma'}{\tsubst{\subst{\beta}{\alpha}{x_0}}{\gamma}}{C}
\]
By congruence with $\beta$ and because $\gamma$ substitutes only for
variables in $\modeof {\Delta}$,
\[
\subst{\beta}{\alpha_0}{x_0} \deq 
{\subst{\beta}{\tsubst \alpha \gamma}{x_0}} =
{\tsubst{\subst{\beta}{\alpha}{x_0}}{\gamma}} 
\]
So applying Lemma~\ref{lemma:respecteq} gives 
\seq{\Gamma,\Gamma'}{\subst{\beta}{\alpha_0}{x_0}}{C}.  

If not, then we have
\[
\deduce{\seq{\Gamma,\Gamma'}{\alpha_0}{A_0}}{\D}
\quad
\infer{\seq{\Gamma,x_0:A_0,\Gamma'}{\beta}{C}}
      { x : \F{\alpha}{\Delta} \in \Gamma,\Gamma' &
        \deduce{\seq{((\Gamma,x_0:A_0,\Gamma')-x),\Delta}{\subst{\beta}{\alpha}{x}}{C}}{\E}}
\]

We are going to commute $\D$ under \FL on $x$, so need to reroute uses
of $x$ to here by the left-inversion lemma
\[
\D' :: {\seq{((\Gamma,\Gamma')-x),\Delta}{\subst{\alpha_0}{\alpha}{x}}{A_0}}
\]
and $\D'$ is no bigger than \D.

Cutting $\D'$ and $\E$ by the inductive hypothesis gives
\[
\seq{((\Gamma,\Gamma')-x),\Delta}{\subst{\subst{\beta}{\alpha}{x}}{\subst{\alpha_0}{\alpha}{x}}{x_0}}{C}
\]
Because $x_0$ is not free in $\alpha$, 
\[
  {\subst{\subst{\beta}{\alpha}{x}}{\subst{\alpha_0}{\alpha}{x}}{x_0}}
= {\subst{\subst{\beta}{\alpha_0}{x_0}}{\alpha}{x}}
\]
so we can apply \FL
\[
\infer{\seq{\Gamma,\Gamma'}{\subst{\beta}{\alpha_0}{x_0}}{C}}
      {\seq{(\Gamma,\Gamma'-x)}{\subst{\subst{\beta}{\alpha_0}{x_0}}{\alpha}{x}}{C}}
\]
%% \seq{((\Gamma,\Gamma')-x),\Delta}{\subst{\subst{\beta}{\alpha}{x}}{\alpha_0}{x_0}}{C}
%% \]

\item (3 pairs) Right rule or identity and $\UL$ (principal or
  right-commutative).

If $x_0$ is the variable used in the left rule in the right-hand
derivation, then the left-hand derivation must have been derived by
$\UR$, and we have
\[
\begin{array}{l}
\D \quad = \quad \infer{\seq{\Gamma,\Gamma'}{\alpha_0}{\U{x_0.\alpha}{\Delta}{A}}}
   {  
     \deduce{\seq{\Gamma,\Gamma',\Delta}{\subst \alpha {\alpha_0}{x_0}}{A}}{\D'}
   }
\\ \\
\infer{\seq{\Gamma,x_0:\U{x_0.\alpha}{\Delta}{A},\Gamma'}{\beta}{C}}
      {
        \begin{array}{l}
        \beta \deq \subst{\beta'}{\tsubst{\alpha}{\gamma}}{z} \\
        {\E_1 :: \seq{\Gamma,x_0:{\U{x_0.\alpha}{\Delta}{A}},\Gamma'}{\gamma}{\Delta}}\\
        {\E_2 :: \seq{\Gamma,x_0:{\U{x_0.\alpha}{\Delta}{A}},\Gamma',\tptm{z}{A}}{\beta'}{C}}
        \end{array}
      }
\end{array}
\]
First, cutting the original \D and the smaller $\E_1$ and $\E_2$ gives 
\[
\deduce{{\seq{\Gamma,\Gamma'}{\subst{\gamma}{\alpha_0}{x_0}}{\Delta}}}{\E_1'}
\qquad 
\deduce{{\seq{\Gamma,\Gamma',\tptm{z}{A}}{\subst{\beta'}{\alpha_0}{x_0}}{C}}}{\E_2'}
\]
Cutting $\E_1'$ \emph{into} $\D'$ (the cut formula $\Delta$ is a
subformula of $\U{x_0.\alpha}{\Delta}{A}$, so it is okay that the derivations are
not known to be smaller) gives
\[
\deduce
{\seq{\Gamma,\Gamma'}{\tsubst{\alpha}{\subst{\gamma}{\alpha_0}{x}}}{A}} {\D_1'}
\]
Cutting $\D_1'$ into $\E_2'$ gives 
\[
\seq{\Gamma,\Gamma'}{\subst{\subst{\beta'}{\alpha_0}{x_0}}{\subst{\alpha}{\alpha_0}{x_0}}{z}}{A}
\]
But we have 
\[
\subst{\beta}{\alpha_0}{x_0} \deq 
\subst{(\subst{\beta'}{\tsubst{\alpha}{\gamma}}{z})}{\alpha_0}{x_0} = 
{\subst{\subst{\beta'}{\alpha_0}{x_0}}{\tsubst{\alpha}{\subst{\gamma}{\alpha_0}{x_0}}}{z}}
\]
by commuting substitutions, which gives the result.  

On the other hand, if the subject of the left rule $x$ is not equal to
$x_0$, then we have
\[
\deduce{\seq{\Gamma,\Gamma'}{\alpha_0}{A_0}}
       {
         \D
       }
\quad
\infer{\seq{\Gamma,x_0:A_0,\Gamma'}{\beta}{C}}
      {
        \begin{array}{l}
          x : \U{x.\alpha}{\Delta}{A} \in \Gamma,\Gamma' \\
          \beta \deq \subst{\beta'}{\tsubst{\alpha}{\gamma}}{z} \\
          \seq{\Gamma,x_0:A_0,\Gamma'}{\gamma}{\Delta} \\
          \seq{\Gamma,x_0:A_0,\Gamma',\tptm{z}{A}}{\beta'}{C}
        \end{array}
      }
\]

By the inductive hypotheses we get 
\[
\seq{\Gamma,\Gamma'}{\subst{\gamma}{\alpha_0}{x_0}}{\Delta}
\qquad
\seq{\Gamma,\Gamma',z:A}{\subst{\beta'}{\alpha_0}{x_0}}{C}
\]
so we can derive
\[
\infer{\seq{\Gamma,x_0:A_0,\Gamma'}{\subst{\beta}{\alpha_0}{x_0}}{C}}
      {
        \begin{array}{l}
          x : \U{x.\alpha}{\Delta}{A} \in \Gamma,\Gamma' \\
          {\subst{\beta}{\alpha_0}{x_0}} \deq \subst{\subst{\beta'}{\alpha_0}{x_0}}{\tsubst{\alpha}{\subst{\gamma}{\alpha_0}{x_0}}}{z} \\
          \seq{\Gamma,\Gamma'}{\subst{\gamma}{\alpha_0}{x_0}}{\Delta} \\
          \seq{\Gamma,\Gamma',\tptm{z}{A}}{\subst{\beta'}{\alpha_0}{x_0}}{C}
        \end{array}
      }
\]
For the equation, we get
\[
\subst{\beta}{\alpha_0}{x_0} \deq
\subst{\subst{\beta'}{\tsubst{\alpha}{\gamma}}{z}}{\alpha_0}{x_0}
\]
by congruence on the assumed equation, and then commute substitutions.  

Part 3: 

\end{itemize}

For part 2, there are just two right-commutative cases: For
\[
\seq{\Gamma,\Gamma'}{\alpha_0}{A_0}
\qquad
\seq{\Gamma,x_0:A_0,\Gamma'}{\cdot}{\cdot}
\]
we also have $\subst \cdot {\alpha_0}{x_0} \deq \cdot$ and
\seq{\Gamma,\Gamma'}{\cdot}{\cdot}.  For
\[
\seq{\Gamma,\Gamma'}{\alpha_0}{A_0}
\qquad
\infer{\seq{\Gamma,x_0:A_0,\Gamma'}{\gamma,\alpha/x}{\Delta,x:A}}
      {\seq{\Gamma,x_0:A_0,\Gamma'}{\gamma}{\Delta} &
        \seq{\Gamma,x_0:A_0,\Gamma'}{\alpha}{A}
      }
\]
we have $\subst{(\gamma,\alpha/x)}{\alpha_0}{x_0} 
= (\subst{\gamma}{\alpha_0}{x_0},\subst{\alpha}{\alpha_0}{x_0})$, and
 \[
\seq{\Gamma,\Gamma'}{\subst{\gamma}{\alpha_0}{x_0}}{\Delta} \quad
\seq{\Gamma,\Gamma'}{\subst{\alpha}{\alpha_0}{x_0}}{A}
\]
by the inductive hypotheses, so we can reapply the rule to conclude
\seq{\Gamma,\Gamma'}{\subst{(\gamma,\alpha/x)}{\alpha_0}{x_0}}{\Delta,x:A}.

For part 3, we induct on $\Delta$, reducing a simulatenous cut to
iterated single-variable cuts.  If $\Delta$ is empty, then we have
\[
\seq{\Gamma}{\cdot}{\cdot}
\qquad
\deduce{\seq{\Gamma,\cdot}{\beta}{C}}{\E}
\]
and we return \E, noting that $\subst{\beta}{\cdot} = \beta$.  Otherwise
we have
\[
\infer{\seq{\Gamma}{\gamma,\alpha/x}{\Delta,x:A}}
      {\deduce{\seq{\Gamma}{\gamma}{\Delta}}{\D_1} &
        \deduce{\seq{\Gamma}{\alpha}{A}}{\D_2}}
\qquad
\deduce{\seq{\Gamma,\Delta,x:A}{\beta}{C}}{\E}
\]
Using the inductive hypothesis to cut $\D_2$ into $\E$ ($A$ is smaller
than $\Delta,x:A$) gives
\[
\deduce{\seq{\Gamma,\Delta}{\subst{\beta}{\alpha}{x}}{C}}
       {\E'}
\]
Using the inductive hypothesis to cut $\D_1$ into $\E'$ ($\Delta$ is
smaller than $\Delta$) gives
\[
\seq{\Gamma}{\tsubst{\subst{\beta}{\alpha}{x}}{\gamma}}{C}
\]
Because $\gamma$ substitutes for $\Delta$ and not $\Gamma$,
\[
\tsubst{\beta}{\gamma,\alpha/x}
= {\tsubst{\subst{\beta}{\alpha}{x}}{\gamma}}
\]
\end{proof}

\begin{corollary}[Contraction over contraction]
\item If
\seq{\Gamma,x:A,y:A,\Gamma'}{\alpha}{C}
then
\seq{\Gamma,z:A,\Gamma'}{\tsubst \alpha {z/x,z/y}}{C}
\end{corollary}

\begin{proof}  Contraction can be shown by cutting with a renaming substitution.
The mode substitution $z/x,z/y$ is a variable-for-variable substitution,
and is type-preserving between ${x:A,y:A}$ and ${\Gamma,z:A,\Gamma'}$.
Therefore, by identity (part 2),
\seq{\Gamma,z:A,\Gamma'}{z/x,z/y}{x:A,y:A}.  Thus, by cut (part 2), we
obtain the result.
\end{proof}

\begin{corollary}[Right-invertibility of \Usymb] \label{lemma:Uinv}
If $\seq{\Gamma}{\beta}{\U{x.\alpha}{\Delta}{A}}$ then 
{\seq{\Gamma,\Delta}{\subst{\alpha}{\beta}{x}}{A}}.
\end{corollary}

\begin{proof}
$\UL$ with identities in both premises gives a derivation
\[
\infer{\seq{\Gamma,\Delta,x:{\U{x.\alpha}{\Delta}{A}}}{\alpha}{A}}
      {
        \alpha = z[\alpha[\vec{x/x}]/z] & 
        \seq{\Gamma,\Delta}{\vec{x/x}}{\Delta} &
        \seq{\Gamma,\Delta,x:{\U{x.\alpha}{\Delta}{A}},z:A}{z}{A}
      }
\]
Weakening the assumed derivation to 
\seq{\Gamma,\Delta}{\beta}{\U{x.\alpha}{\Delta}{A}}
and then cutting for $x$ in the above gives the result.  

\[
\infer{\seq{\Gamma,\Delta}{\subst{\alpha}{\beta}{x}}{A}}
      {\seq{\Gamma,\Delta}{\beta}{\U{x.\alpha}{\Delta}{A}} & 
       \seq{\Gamma,\Delta,x:{\U{x.\alpha}{\Delta}{A}}}{\alpha}{A}
      }
\]

\end{proof}

\begin{theorem}[Fusion] ~
\begin{enumerate} 

\item $\F{\alpha}{\Delta,x:\F{\beta}{\Delta'},\Delta''} \dashv \vdash
  \F{\subst{\alpha}{\beta}{x}}{\Delta,\Delta',\Delta''}$

\item $\U{x.\alpha}{\Delta,y:\F{\beta}{\Delta'},\Delta''}{A} \dashv \vdash
  \U{x.\subst{\alpha}{\beta}{y}}{\Delta,\Delta',\Delta''}{A}$

\item 
$\U{x.\alpha}{\Delta}{\U{y.\beta}{\Delta'}{A}} \dashv \vdash
 \U{x.\subst{\beta}{\alpha}{y}}{\Delta,\Delta'}{A}$

\end{enumerate}
\end{theorem}

\begin{proof}

\[
\infer{
  \seq{z:\F{\alpha}{\Delta,x:\F{\beta}{\Delta'},\Delta''}}
      {z}
      {\F{\subst{\alpha}{\beta}{x}}{\Delta,\Delta',\Delta''}}
}
{
  \infer{\seq{\Delta,x:\F{\beta}{\Delta'},\Delta''}{\alpha}{\F{\subst{\alpha}{\beta}{x}}{\Delta,\Delta',\Delta''}}}
        {
          \infer{\seq{\Delta,\Delta'',\Delta'}{\subst{\alpha}{\beta}{x}}{\F{\subst{\alpha}{\beta}{x}}{\Delta,\Delta',\Delta''}}}
                {\subst{\alpha}{\beta}{x} \deq \tsubst{\subst{\alpha}{\beta}{x}}{\vec{z/z}} & 
                 \seq{\Delta,\Delta'',\Delta'}{\vec{z/z}}{\Delta,\Delta',\Delta''}
                }
        }
}
\]

FIXME: obvious append lemma for substitutions

\[
\infer{
  \seq{z:{\F{\subst{\alpha}{\beta}{x}}{\Delta,\Delta',\Delta''}}}
      {z}
      {\F{\alpha}{\Delta,x:\F{\beta}{\Delta'},\Delta''}}
}
{  
\infer{\seq{\Delta,\Delta',\Delta''}
           {\subst{\alpha}{\beta}{x}}
           {\F{\alpha}{\Delta,x:\F{\beta}{\Delta'},\Delta''}}}
      {\alpha[\beta/x] = \alpha[\vec{y/y},\beta/x,\vec{z/z}] &
        \infer{\seq{\Delta,\Delta',\Delta''}{\vec{y/y},\beta/x,\vec{z/z}} {\Delta,x:\F{\beta}{\Delta'},\Delta''}}
              {\seq{\Delta,\Delta',\Delta''}{\vec{y/y}}{\Delta} & 
               \infer{\seq{\Delta,\Delta',\Delta''}{\beta}{\F{\beta}{\Delta'}}}
                     {\beta = \beta[\vec{w/w}] & \seq{\Delta,\Delta',\Delta''}{\vec{w/w}}{\Delta'}} &
               \seq{\Delta,\Delta',\Delta''}{\vec{z/z}}{\Delta''} }
      }
}
\]

\end{proof}

\subsection{Equational Theory of Proofs}



\subsection{Categorical Semantics}

TODO: Mitchell


\setlength{\bibsep}{-1pt} %% dirty trick: make this negative
{ \small
%% \linespread{0.70}
\bibliographystyle{abbrvnat}
\bibliography{../drl-common/cs}
}

\end{document}


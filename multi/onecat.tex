
\section{1-Multicategories}

\subsection{Syntax for Cartesian Multicategories}
In an ordinary category, each morphism has a single object as both its
domain and its range.  In a \emph{multicategory}, each morphism has a
list of objects as its domain, and a single object as its range.  In a
\emph{cartesian multicategory}, these multiple objects in the domain are
treated like cartesian (structural) variables. So for a cartesian
multicategory \C\/ we will describe a morphism
\[
\alpha \in \mor{\C}{p_1,\ldots,p_n}{q}
\]
by a term with variables $x_i:p_i$ free, built from generating constants
\dsd{c}:
\[
\begin{array}{llll}
\text{modes} & p & & (constants) \\
\text{mode contexts} & \psi & ::= & \cdot \mid \psi,\tptm{x}{p} \\
\text{mode morphisms} & \alpha,\beta & ::= & x \mid \dsd{c}(\alpha_1,\ldots,\alpha_n) \\
\text{mode substitutions} & \gamma,\delta & ::= & \cdot \mid \gamma,\alpha/x\\
\end{array}
\]
We write $\oftp{\psi}{\alpha}{q}$ for mode morphism well-formedness;
$\psi$ here enjoys (cartesian) structural properties (weakening,
exchange, contraction).  We write $\oftp{\psi}{\gamma}{\psi'}$ for a
familiar structural substitution in the mode multicategory, which
consists of a term $\alpha_i/x_i$ for each variable $\tptm{x_i}{p_i}$ in
$\psi'$, where each $\alpha_i$ satisfies $\oftp{\psi}{\alpha_i}{p_i}$.
Simultaneous substitution into terms is defined in the standard way.

The multicategories needed for this section can be presented by giving a
signature for the constants \dsd{c}, together with some generating
axioms $\deqtm{\psi}{\alpha}{\beta}{q}$ for two mode morphisms
\oftp{\psi}{\alpha,\beta}{q}.  The set of morphisms is defined to be the
quotient of the above terms by the least congruence containing these
generating equations.

\subsection{Logic over a Multicategory}

The rules for the logic are given in Figure~\ref{fig:1logic}.  

explain intended structural properties

explain $\alpha$-conversion

\emph{context descriptor} 

For $\FL$, there are two competing principles: making the rules
obviously structural, and reducing inessential non-determinism.  Here,
we choose the later, and treat the assumption of \F{\alpha}{\Delta}
affinely, removing it from the context when it is used.  It will turn
out that the judgement nonetheless enjoys contraction (over
contraction), because contraction for negatives is built in, and
contraction for positives follows from this and the fact that we can
always reconstruct a positive from what it decays to on the left
(c.f. how purely positive formulas have contraction in linear logic).  

\begin{figure}
\[
\begin{array}{c}
\begin{array}{llll}
\text{Types} & A & ::= & P \mid \F{\alpha}{\Delta} \mid \U{\alpha}{\Delta}{A} \\
\end{array}
\\ \\
\framebox{\wftype{A}{p}}
\\ \\
\infer{\wftype{P}{p}}{\text{(declared in signature)}}
\qquad
\infer{\wftype{\F{\alpha}{\Delta}}{q}}
      {\oftp{\psi}{\alpha}{q} &
        \wfctx{\Delta}{\psi}}
\\ \\
\infer{\wftype{\U{x.\alpha}{\Delta}{A}}{q}}
      {\oftp{\psi,x:q}{\alpha}{p} &
        \wfctx{\Delta}{\psi} &
        \wftype{A}{p}
      }
\\ \\
\framebox{\wfctx{\Gamma}{\psi}}
\\ \\
\infer{\wfctx{\cdot}{\cdot}}{}
\qquad
\infer{\wfctx{\Delta,x:A}{\psi,x:p}}
      {\wfctx{\Delta}{\psi} &
        \wftype{A}{p}}
\\ \\
\framebox{\seq{\Gamma}{\alpha}{A}}
\\ \\
\infer[\dsd{v}]{\seq{\Gamma}{\beta}{P}}
      {x:P \in \Gamma & \beta \equiv x}
\\ \\
\infer[\FR]{\seq{\Gamma}{\beta}{\F{\alpha}{\Delta}}}
      {%% \modeof{\Gamma} \vdash \gamma : \modeof{\Delta} & 
        \beta \deq \tsubst{\alpha}{\gamma} &
        \seq{\Gamma}{\gamma}{\Delta} 
      }
\quad
\infer[\FL]{\seq{\Gamma,x:\F{\Delta}{A},\Gamma'}{\beta}{C}}
      {\seq{\Gamma,\Gamma',\Delta}{\subst \beta {\alpha}{x}}{C}}
%% \infer{\seq{\Gamma}{\beta}{C}}
%%       {{x}:{\F{\alpha}{\Delta}} \in \Gamma & 
%%         \oftp{\modeof{\Gamma},{x'} : {\modeof{\F{\alpha}{\Delta}}}}{\beta'}{\modeof{C}} &
%%         \beta \deq \tsubst{\beta'}{x/x'} &
%%         \seq{\Gamma,\Delta}{\subst {\beta'} {\alpha}{x'}}{C}}
\\ \\
\infer[\UR]{\seq{\Gamma}{\beta}{\U{x.\alpha}{\Delta}{A}}}
      {\seq{\Gamma,\Delta}{\subst{\alpha}{\beta}{x}}{A}}
\\ \\
\infer[\UL]{\seq{\Gamma}{\beta}{C}}
      {x:\U{x.\alpha}{\Delta}{A} \in \Gamma & 
        \beta \deq \subst{\beta'}{\tsubst{\alpha}{\gamma}}{z} &
        \seq{\Gamma}{\gamma}{\Delta} &
        \seq{\Gamma,\tptm{z}{A}}{\beta'}{C}
      }
\\ \\
\framebox{\seq{\Gamma}{\gamma}{\Delta}}
\\ \\
\infer[\cdot]{\seq{\Gamma}{\cdot}{\cdot}}
      {}
\qquad
\infer[\_,\_]{\seq{\Gamma}{\gamma,\alpha/x}{\Delta,x:A}}
      {\seq{\Gamma}{\gamma}{\Delta} &
       \seq{\Gamma}{\alpha}{A}
      }
\end{array}
\]    
\caption{1-Multicategorical Logic}
\label{fig:1logic}
\hrule
\end{figure}

\subsection{Examples}

\subsubsection{Non-associative Logic}

For a mode theory with one mode \dsd{m}, constants
\[
\begin{array}{c}
\oftp{x : \dsd{m}, y : \dsd{m}}{x \odot y}{\dsd{m}}\\
\oftp{\cdot}{1}{\dsd{m}}
\end{array}
\]
and no equations, we get a non-associative context.  Writing $A \odot B$
for \F{x \odot y}{x:A,y:B}, we \emph{cannot}, for example, derive
\[
A \odot (B \odot C) \vdash (A \odot B) \odot C
\]
To derive 
\begin{footnotesize}
\[{\seq{a:\F{x \odot p}{x:A,p:\F{y \odot z}{y:B,z:C}}}
  {a}
  {\F{q \odot z}{q:\F{x \odot y}{x:A,y:B},z:C}}}
\]
\end{footnotesize}
we can start by apply \FL\/ twice to reduce the sequent to
\[
\seq{x:A,y:B,z:C}{x \times (y \odot z)}{{\F{q \odot z}{q:\F{x \odot y}{x:A,y:B},z:C}}}
\]
To apply \FR, we need to find a $\gamma$ such that ${x \times (y \odot
  z)} \deq (q \odot z)[\gamma]$.  In the absence of any equational
axioms, the only possible choice is $x/q, (y \odot z)/z$, so we need
to show
\[
\seq{x:A,y:B,z:C}{x}{A \odot B}
\qquad
\seq{x:A,y:B,z:C}{y \odot z}{C}
\]
which is not possible because the context is not divded correctly.  

\subsubsection{Ordered Logic}

If we add the axioms
\[
\begin{array}{c}
x \odot (y \odot z) \deq (x \odot y) \odot z\\
x \odot 1 \deq x \deq 1 \odot x
\end{array}
\]
then we get ordered logic (which has none of exchange, weakening, and
contraction---a monoidal but not symmetric monoidal product).  In
ordered logic, we can complete the above proof of associativity of
$\odot$: we need to find a $\gamma$ such that ${x \times (y \odot z)}
\deq (q \odot z)[\gamma]$, we can now choose $(x \odot y)/q, z/z)$, so
the subgoals are
\[
\seq{x:A,y:B,z:C}{x \odot y}{A \odot B}
\qquad
\seq{x:A,y:B,z:C}{z}{C}
\]
The latter is identity, and the former is a further \FR\/ and then
identities: 
\[
\infer{\seq{x:A,y:B,z:C}{x \odot y}{\F{x' \odot y'}{x':A,y':B}}}
      { \begin{array}{l}
          x \otimes y \deq (x' \odot y')[x/x',y/y'] \\
          \seq{x:A,y:B,z:C}{x}{A} \\
          \seq{x:A,y:B,z:C}{y}{B} 
        \end{array}
      }
\]

The expected rules for the left and right implications of ordered logic are
\[
\begin{array}{l}
\infer{\Gamma \vdash A \rightharpoonup B}
      {\Gamma,A \vdash B}
\qquad
\infer{\Gamma,A \rightharpoonup B,\Delta,\Gamma' \vdash C}
      {\Delta \vdash A &
       \Gamma,B,\Gamma' \vdash C
      }
\\ \\
\infer{\Gamma \vdash A \leftharpoonup B}
      {A,\Gamma \vdash B}
\qquad
\infer{\Gamma,\Delta,A \leftharpoonup B,\Gamma' \vdash C}
      {\Delta \vdash A &
       \Gamma,B,\Gamma' \vdash C
      }
\end{array}
\]

We can define these by 
\[
\begin{array}{l}
A \rightharpoonup B := \U{c.c \odot x}{x:A}{B}\\
A \leftharpoonup B := \U{c.x \odot c}{x:A}{B}
\end{array}
\]
which have the expected right rules, putting $x$ on the left or right of
the context descriptor:
\[
\infer{\seq{\Gamma}{\beta}{\U{c.c \odot x}{x:A}{B}}}
      {\seq{\Gamma,x:A}{\beta \odot x}{B}}
\qquad
\infer{\seq{\Gamma}{\beta}{\U{c.c \odot x}{x:A}{B}}}
      {\seq{\Gamma,x:A}{x \odot \beta}{B}}
\]
The left rules are
\[
\begin{array}{l}
\infer{\seq{\Gamma} {\beta} {C}}
      {c:\U{c.c \odot x}{x:A}{B} \in \Gamma &
       \beta \deq \beta'[c \odot \alpha/z] &
       \seq{\Gamma}{\alpha}{A} &
       \seq{\Gamma,z:A}{\beta'}{C}
      }
\\ \\ 
\infer{\seq{\Gamma} {\beta} {C}}
      {c:\U{c.x \odot c}{x:A}{B} \in \Gamma &
       \beta \deq \beta'[\alpha \odot c/z] &
       \seq{\Gamma}{\alpha}{A} &
       \seq{\Gamma,z:A}{\beta'}{C}
      }
\end{array}
\]

Suppose that $\beta$ is (up to associativity and unit) of the form $x_1
\odot \ldots \odot x_n$ for distinct variables $x_i$.  Then the
equality premise for the $\rightharpoonup$ rule will reassociate $\beta$
as $\beta_1 \odot (c \odot \alpha) \odot \beta_2$, where $\alpha$
plays the role of $\Delta$ in the ordered logic rule---the resources
used to prove $A$, which occur to the right of the implication being
eliminated.  The resources $\beta'$ used for the continuation are
``$\beta$ with $c \otimes \alpha$ replaced by the result of the
implication,'' as desired.  While $c$ and any variables used in $\alpha$
are still in $\Gamma$, permission to use them has been removed from
$\beta'$---and there is no way to restore such permissions in this mode
theory.  The rule for $\leftharpoonup$ is the same, but with $\alpha$ on
the opposite side of $c$.

FIXME: get more formal?

FIXME: comment on non-linear $\beta$?

\subsubsection{Linear Logic}

If extend the mode theory with the equation
\[
x \otimes y \deq y \otimes x
\]
(and switich notation from $\odot$ to $\otimes$) then we get linear
logic.  
For example, we can derive 
{\seq{p : A \otimes B}{p}{B \otimes A}} in the expected way:
\[
\infer[\FL]
      {\seq{p : \F{x \otimes y}{x:A,y:B}}{p}{\F{x \otimes y}{x:B,y:A}}}
      {\infer[\FR]
        {\seq{x:A,y:B}{x \otimes y}{\F{x' \otimes y'}{x':B,y':A}}}
        {x \otimes y \deq (x' \otimes y')[y/x',x/y'] &
         \seq{x:A,y:B}{y}{B} &
         \seq{x:A,y:B}{x}{A}
        }}
\]

For this mode theory, \U{c.c \odot x}{x:A}{B} and \U{c.c \odot
  x}{x:A}{B} are isomorphic, and both represent $A \multimap B$.

%% The linear logic $\multimap$-left rule (where contexts are implicitly
%% treated modulo exchange) is
%% \[
%% \infer{\Gamma,\Delta,A \multimap B \vdash C}
%%       {\Delta \vdash A &
%%        \Gamma,B \vdash C}
%% \]
%% We have
%% \[
%% \infer{\seq{\Gamma} {\beta} {C}}
%%       {c:\U{c.c \otimes x}{x:A}{B} \in \Gamma &
%%        \beta \deq \beta'[c \odot \alpha/z] &
%%        \seq{\Gamma}{\alpha}{A} &
%%        \seq{\Gamma,z:A}{\beta'}{C}
%%       }
%% \]

\subsubsection{Multi-use variables}

Jcreed names and substructurality, and related work cited there
Abel relevance tracking
Channels? 

\[
\infer{\Gamma \vdash A \to^n B}
      {{\Gamma, x :^n A} \vdash {B}}
\qquad
\infer{\Gamma + f: A \to^n B + (n \cdot \Delta) \vdash C}
      {\Delta \vdash A &
       {\Gamma, B} \vdash {C}}
\]
where $\Gamma + \Delta$ acts pointwise by $x :^{n} A + x :^{m} A = x
:^{n+m} A$ and $n \cdot \Delta$ act pointwise by $n \cdot x^{m} A = x
:^{nm} A$.

In the linear logic mode theory, we use
\[
\begin{array}{c}
\mathord{\otimes}^0(x) = 1 \\
\mathord{\otimes}^{1+n}(x) = \otimes^{n}(x) \otimes x \\
A \to^n B := \U{c.c \otimes (\otimes^n x)}{x:A}{B} \\
\end{array}
\]

\[
\infer{\seq{\Gamma}{\beta}{A \to^n B}}
      {\seq{\Gamma, x:A}{\beta \otimes \mathord{\otimes}\otimes^n(x)}{B}}
\]

\[
\infer{\seq{\Gamma}{\beta}{C}}
      {\begin{array}{l}
          f : \U{f.f \otimes \mathord{\otimes}^n(x)}{x : A}{B} \in \Gamma \\
          \beta \deq \beta'[f \otimes \mathord{\otimes}^n(\alpha)/z] \\
          \seq{\Gamma}{\alpha}{A} \\
          \seq{\Gamma, z:B}{\beta'}{C} 
       \end{array}
      }
\]

We can also consider an $n$-use product 
\[
A^n := \F{\mathord{\otimes}^n(x)}{x:A}
\]
as a positive type, which will decompose $A \to^n B$ as $A^n \lolli B$
(by Theorem~\ref{thm:fusion}).

This has 
\[
\seq{p:A^n}{p}{A \otimes \ldots \otimes A}
\]
but not
\[
\seq{A \otimes \ldots \otimes A}{p}{p:A^n}
\]

For example,
\[
\infer {\seq{p:\F{x \otimes x}{x:A}}{p}{\F{x \otimes y}{x:A,y:A}}}
       {\infer {\seq {x:A}{x \otimes x}{\F{x \otimes y}{x:A,y:A}}}
               {x \otimes x \deq (y \otimes z)[x/y,x/z] &
                \seq{x:A}{x}{A} &
                \seq{x:A}{x}{A}}
       }
\]
but
\[
\infer {\seq{p:\F{y \otimes z}{y:A,z:A}}{p}{\F{x \otimes x}{x:A}}}
       {\infer {\seq {y:A,z:A}{y \otimes z}{\F{x \otimes x}{x:A}}}
               {y \otimes z \deq (x \otimes x)[?/x] &
                \ldots
               }
       }
\]
is not derivable, because there is no substitution into $x \otimes x$
that makes it equal to $y \otimes z$ for distinct $y$ and $z$.  

Conceptually, we think of $A^2$ as expressing a notion of identity: it is
a \emph{single} $A$ that can be used twice, which is a restriction on
having two potentially different $A$'s.  

\subsubsection{Relevant Logic}

If we start from the mode theory for linear logic and make the context
operation idempotent, in the sense that for all $x$,
\[
x \deq x \otimes x 
\]
(two uses of $x$ are the same as one) then we get relevant logic,
i.e. exchange and contraction but not weakening.

Given (with $x$ and $y$ not occuring in $\alpha$), we can prove
contraction as follows:
\[
\infer[Lemma~\ref{lemma:respecteq}]
      {\seq{\Gamma,z:A}{\alpha \otimes z}{C}}
      {\infer[Corollary~\ref{cor:controver}]{\seq{\Gamma,z:A}{\alpha \otimes z \otimes z}{C}}
                                    {\seq{\Gamma,x:A,y:A}{\alpha \otimes x \otimes y}{C}}}
\]

We discuss weakening in Section~\ref{sec:?} below.  

\subsubsection{``Relevant BI''} Bunched implication~\citep{ohearnpym}
has two context-forming operations $\Gamma,\Gamma'$ and
$\Gamma;\Gamma'$, along with corresponding products and implications.
Both are associative, unitial, and commutative, but $;$ has weakening
and contraction while $,$ does not.  A context is represented by a tree
such as $(x:A, y:B);(z : C, w : D)$ (considered modulo the laws), and
the notation $\Gamma[\Delta]$ is used to refer to a tree with a hole
$\Gamma[-]$ that has $\Delta$ as a subtree at the hole.  In sequent
calculus style, the rules for the product and implication corresponding
to $,$ --- written $*$ and $\magicwand$ --- are
\[
\begin{array}{l}
\infer{\Gamma[A * B] \vdash C}
      {\Gamma[A , B] \vdash C}
\qquad
\infer{\Gamma,\Delta \vdash A * B}
      {\Gamma \vdash A &
       \Delta \vdash B}
\\ \\
\infer{\Gamma \vdash A \magicwand B}
      {\Gamma, A \vdash B}
\qquad
\infer{\Gamma[A \magicwand B, \Delta] \vdash C}
      {\Delta \vdash A &
       \Gamma[B] \vdash C}
\end{array}
\]
There are similar rules for a product and implication based on $;$, and 
structural rules of weakening and contraction for $;$.  

To model the basic structure of two context-forming operations, we can
consider a mode theory with two commutative monoids
\[
\begin{array}{l}
x  : \dsd{m}, y  : \dsd{m} \vdash x \otimes y : \dsd{m} \\
\cdot \vdash 1 : m \\
x \otimes (y \otimes z) \deq (x \otimes y) \otimes z\\
x \otimes 1 \deq x \deq 1 \otimes x\\
x \otimes y \deq y \otimes x \\
x  : \dsd{m}, y  : \dsd{m} \vdash x * y : \dsd{m} \\
\cdot \vdash \dsd{I} : \dsd{m} \\
x * (y * z) \deq (x * y) * z\\
x * \dsd{I} \deq x \deq \dsd{I} * x\\
x * y \deq y * x \\
\end{array}
\]
and define products and implications using the monoids:
\[
\begin{array}{l}
A * B := \F{x * y}{x : A, y : B} \\
A \magicwand B := \U{c.c * x}{x : A}{B} \\
A \otimes B := \F{x \otimes y}{x : A, y : B}\\
A \lolli B := \U{c.c \otimes x}{x : A}{B}\\
\end{array}
\]
A context descriptor such as $(x \otimes y) * (z \otimes w)$ captures
the ``bunched'' structure of a BI context, and substitution for a
variable models the hole-filling operation $\Gamma[\Delta]$.  The left
rules act on a leaf or subtree: the left rule for $*$
\[
\infer{\seq{\Gamma,z:A*B,\Gamma'}{\beta}{C}}
      {\seq{\Gamma,\Gamma',x:A,y:B}{\subst{\beta}{x * y}{z}}{C}}
\]
replaces the leaf where $z$ occurs in the tree $\beta$ with $x*y$, while
the left rule for $\magicwand$
\[
\infer{\seq{\Gamma}{\beta}{C}}
      {
        c : A \magicwand B \in \Gamma &
        \beta \deq \beta'[ c * \alpha / z] & 
        \seq{\Gamma}{\alpha}{A} &
        \seq{\Gamma,z:B}{\beta'}{C} 
      }
\]
isolates a subtree containing the implication $c$ and resources $*$'ed
with it, uses those resources to prove $A$, and then replaces the
subtree with the variable $z$ standing for the result of the
implication.

If we additionally make $\otimes$ idempotent ($x \otimes x \deq x$),
this encodes the contraction for $;$ that is present in BI; we discuss
weakening below.

\subsubsection{Exponentials}  As an example of a mode theory with more
than one mode, we can consider a pair of adjoint functors between two
symmetric monoidal categories.  First, we have
\[
x  : \dsd{m}, y  : \dsd{m} \vdash x \otimes y : \dsd{m} \\
\cdot \vdash 1_{\dsd m} : m \\
x  : \dsd{n}, y  : \dsd{n} \vdash x \odot y : \dsd{n} \\
\cdot \vdash 1_{\dsd n} : \dsd{n} \\
\]
with associativity, unit, and commutativity equations for both.  
Next, we have
\[
x : \dsd{m} \vdash \dsd{f}(x) : \dsd{n}
\]
This creates types 
\[
\wftype {\F{f(x)}{x : A_{\dsd{m}}}}{\dsd{n}}
\qquad
\wftype {\U{x.f(x)}{\cdot}{A_{\dsd{n}}}}{\dsd{m}}
\]
which satisfy
\[
\F{f(x)}{x:-} \la {\U{x.f(x)}{\cdot}{-}}
\]
In terms of provability (deferring discussion of equality of proofs
until below), this follows from \FL\/ and \FR\/ and their invertibility
(Corollary~\ref{cor:Uinvertibility}, Lemma~\ref{lemma:finvert}):
\[
\infer={\seq{p:\F{f(x)}{x:A}}{p}{B}}
       {\infer={\seq{x:A}{f(x)}{B}}
               {\seq{x:A}{x}{\U{x.f(x)}{\cdot}{B}}}}
\]

Analogously to the adjoint decomposition of linear
logic~\citep{bentonwadler96adjoint}, we will view \dsd{m} as a mode of
linear propositions, and \dsd{n} as a mode of relevant proposition, and
add contraction for $\odot$:
\[
(x \odot x) \deq x
\]
The comonad of the adjunction
\[
!^{\dsd{c}} A := \F{\dsd{f(x)}}{x:\U{c.\dsd{f}(c)}{\cdot}{A}}
\]
both takes and produces a linear proposition, which should be
contractibile, in the same way that \citet{bentonwadler96adjoint} obtain
both weakening and contraction for $!$ from cartesianness the analogous
mode in their system.  
The derivation begins
\[
\infer{\seq{p : !^{\dsd{c}} A}{p}{!^{\dsd{c}} A \otimes !^{\dsd{c}} A}}
      {\infer{\seq{{x:\U{c.\dsd{f}(c)}{\cdot}{A}}}{f(x)}{{!^{\dsd{c}} A \otimes !^{\dsd{c}} A}}}
             {\begin{array}{l}
                 f(x) \deq (x' \otimes y') [f(x) / x' , f(x) / y'] \\
                 \seq{x:\U{c.\dsd{f}(c)}{\cdot}{A}}{f(x)}{!^{\dsd c} A} \\
                 \seq{x:\U{c.\dsd{f}(c)}{\cdot}{A}}{f(x)}{!^{\dsd c} A} 
               \end{array}
             }}
\]
and we can derive \seq{x:\U{c.\dsd{f}(c)}{\cdot}{A}}{f(x)}{!^{\dsd c} A}
by \FR\/ with identity in the premise (it is of the form
$\seq{x:C}{f(x)}{\F{f(x)}{x:C}}$).  The key point is that the first
premise, which reduces to
\[
\dsd{f}(x) \deq \dsd{f}(x) \otimes \dsd{f}(x)
\]
can be deduced from 
\[
\dsd{f}(x) \deq \dsd{f}(x \odot x) \deq \dsd{f}(x) \otimes \dsd{f}(x)
\]
by contraction for $\odot$ \emph{if we have an axiom that \dsd{f}
  preserves the monoidal product}
\[
\dsd{f}(x \odot y) \deq \dsd{f}(x) \otimes \dsd{f}(y)
\]
The fact that \dsd{F} preserves the context ``comma'' is implicit in the
treatment of the context in \citep{bentonwadler96adjoint,reed09adjoint},
but an explicit choice here, to allow mode theories with non-monoidal
left adjoints.  

Note that there is nothing specific to \dsd{U} in this proof; in general
we have contraction for \F{f(x)}{x:C_\dsd{n}} for any $C$ of mode
\dsd{n}.

\subsection{Properties}

Define the \emph{size} of a derivation of \seq{\Gamma}{\alpha}{A} or
\seq{\Gamma}{\gamma}{\Delta} to be the number of inference rules for
these judgements $(\dsd{v},\FL, \FR, \UL, \UR,
\cdot, \_,\_)$ used in it (i.e., the evidence that variables are in a
context and the evidence for equations does not contribute to the size).
Sizes are necessary for the cut proof, where we sometimes weaken or
invert a derivation before applying the inductive hypothesis.

\begin{lemma}[Respect for Equality] ~ \label{lemma:respecteq}
\begin{enumerate}
\item If \seq{\Gamma}{\beta}{A} and $\beta' \deq \beta$ then
\seq{\Gamma}{\beta'}{A}, and the resulting derivation has the same size
as the given one.
\item If \seq{\Gamma}{\gamma}{\Delta} and $\gamma' \deq \gamma$ then
  \seq{\Gamma}{\gamma'}{\Delta}, and the resulting derivation has the
  same size as the given one.
\end{enumerate}
\end{lemma}
\begin{proof}
Mutual induction on the given derivation.  The cases for \dsd{v} and
$\FR$ and $\UL$ are immediate (with no use of the inductive
hypothesis) by composing with the equality in the premise of the rule.
The cases for $\FL$ and $\UR$ use the inductive hypothesis,
along with congruence for equality of mode morphism to show that
$\subst{\beta}{\alpha}{x} \deq \subst{\beta'}{\alpha}{x}$ or
$\subst{\alpha}{\beta}{x} \deq \subst{\alpha}{\beta'}{x}$.  The cases
for substitutions rely on the fact that no generating equalities for
mode substitutions are allowed, so if $\gamma' \deq \cdot$ then
$\gamma'$ is literally $\cdot$, and $(-,-)$ is injective (if $\gamma'
\deq (\gamma_1,\alpha_2/x)$, then $\gamma'$ is $(\gamma_1',\alpha_2'/x)$
with $\gamma_1' \deq \gamma_1$ and $\alpha_2' = \alpha_2$); this is
enough to use the inductive hypotheses in the cons case.  
\end{proof}

\begin{lemma}[Weakening over weakening] \label{thm:weakening} ~
\begin{enumerate}
\item If \seq{\Gamma,\Gamma'}{\alpha}{C} then
\seq{\Gamma,\tptm{z}{A},\Gamma'}{\alpha}{C}, and the resulting
derivation has the same size as the given one.  
\item If \seq{\Gamma,\Gamma'}{\gamma}{\Delta} then
\seq{\Gamma,\tptm{z}{A},\Gamma'}{\gamma}{\Delta}, and the resulting
derivation has the same size as the given one.  
\item If \seq{\Gamma,\Gamma''}{\alpha}{C} then
\seq{\Gamma,\Gamma',\Gamma''}{\alpha}{C}, and the resulting
derivation has the same size as the given one.  
\end{enumerate}
\end{lemma}
\begin{proof}
It is implicit that the mode morphism $\alpha$ is weakend with $z$ in
the conclusion.  Intuitively, weakening holds because the contexts
$\Gamma$ are treated like ordinary structural contexts in all of the
rules---they are fully general in every conclusion, and the premises
check membership or extend them---and because weakening holds for mode
morphisms and equalities of mode morphisms.  Formally, the first two
parts are proved by mutual induction; each case is either immediate
or follows from weakening for the mode morphisms, weakening for
equalities of mode morphisms, and the inductive hypotheses.  The third
part is proved by induction over $\Gamma'$, repeatedly applying the
first part.  
%% The case for the hypothesis rule is immediate, because
%% $\Gamma$ may contain variables other than $x$.  The case for
%% \Fsymb-right follows from weakening for the mode morphisms, and
%% equations between mode morphisms, and the inductive hypothesis for
%% substitutions.  The case for \Fsymb-left follows from the inductive
%% hypothesis, as does the case for \Usymb-right.  
\end{proof}

\begin{lemma}[Exchange over exchange]
If \seq{\Gamma,x:A,y:B,\Gamma'}{\alpha}{C} then
\seq{\Gamma,y:B,x:A,\Gamma'}{\alpha}{C}, and the resulting derivation
has the same size as the given one.  (And similarly for substitutions,
and exchange can be iterated).  
\end{lemma}
\begin{proof} Analogous to weakening.  
\end{proof}

\begin{theorem}[Identity] ~
\begin{enumerate}
\item If $x:A \in \Gamma$ then $\seq{\Gamma}{x}{A}$.
\item If $\oftp{\modeof{\Gamma}}{\rho}{\modeof{\Delta}}$ is a
  variable-for-variable mode substitution such that $x:A \in \Delta$
  implies $\rho(x) : A \in \Gamma$, then $\seq{\Gamma}{\rho}{\Delta}$.
\end{enumerate}
\end{theorem}

\begin{proof}
The standard proof by induction on $A$ (mutually with $\Delta$) applies:
the case for atomic propositions is a rule, and for the other
connectives, apply the invertible and then non-invertible rule to reduce
the problem to the inductive hypotheses.  More specifically, identity
for $P$ is a rule.  In the case for \F{\alpha}{\Delta}, with $\Gamma =
\Gamma_1,x:\F{\alpha}{\Delta},\Gamma_2$, we reduce it to the inductive
hypothesis as follows:
\[
\infer[\FL]{\seq{\Gamma_1,x:\F{\alpha}{\Delta},\Gamma_2}{x}{\F{\alpha}{\Delta}}}
      {\infer[\FR]{\seq{\Gamma_1,\Gamma_2,\Delta}{\alpha}{\F{\alpha}{\Delta}}}
                        {\alpha \deq \tsubst{\alpha}{\vec{x/x}} &
                        \seq{\Gamma_1,\Gamma_2,\Delta}{\vec{x/x}}{\Delta}
                        }}
\]
In the second premise, the $\vec{x/x}$ substitution for each $x \in
\Delta$ is a variable-for-variable substitution, so the second part of
the inductive hypothesis applies.  
The case for \Usymb\/ is similar
\[
\infer[\UR]{\seq{\Gamma}{x}{\U{\alpha}{\Delta}{A}}}
      {\infer[\UL]{\seq{\Gamma,\Delta}{\alpha}{A}}
                        {\alpha \deq \subst{x}{\tsubst{\alpha}{\vec{x/x}}}{x} &
                        \seq{\Gamma,\Delta}{\vec{x/x}}{\Delta} &
                        \seq{\Gamma,x:A}{x}{A}
                        }}
\]

For the second part, the hypothesis of the lemma asks that every
variable in $\Delta$ is associated by $\rho$ with a variable of the same
type in $\Gamma$; this is enough to iterate the first part of the
lemma for each position in $\Delta$.  Specifically, the case where
$\Delta$ is the empty context $\cdot$ is a rule. In the case for a cons
$\Delta,y:A$, we have
\oftp{\modeof{\Gamma}}{\rho}{(\modeof{\Delta},y:\modeof{A})} which means
$\rho$ must be of the form $\rho',x/y$ where $x \in \modeof{\Gamma}$ and
$\rho'$ is a variable-for-variable substitution.  Because $\rho$ was
type-preserving, $x : A \in \Gamma$ and $\rho'$ is type-preserving, so
we obtain the result from the inductive hypotheses as follows:
\[
\infer{\seq{\Gamma}{\rho,x/y}{\Delta,y:A}}
      {\seq{\Gamma}{\rho}{\Delta} & 
       \seq{\Gamma}{x}{A}
      }
\]
\end{proof}

\begin{lemma}[Left-invertibility of \Fsymb] \label{lemma:Finv}
If $\D :: \seq{\Gamma_1,x_0:\F{\alpha_0}{\Delta_0},\Gamma_2}{\beta}{C}$
and then there is a derivation $D' ::
\seq{\Gamma_1,\Gamma_2,\Delta_0}{\subst{\beta}{\alpha_0}{x_0}}{C}$ and
$size(\D') \le size(\D)$ (and analogously for substitutions).
\end{lemma}

\begin{proof}
By induction on \D.  We write $\Gamma$ for
the whole context $\Gamma_1,x_0:\F{\alpha_0}{\Delta_0},\Gamma_2$.

In the case for \dsd{v}, $x : P \in
\Gamma_1,x_0:\F{\alpha_0}{\Delta_0},\Gamma_2$ cannot be equal to $x_0 :
\F{\alpha_0}{\Delta_0}$ because the types conflict, so we can reapply
the \dsd{v} rule in $\Gamma_1,\Gamma_2,\Delta$.

In the case for $\FR$, we have
\[
\infer{\seq{\Gamma}{\beta}{\F{\alpha}{\Delta}}}
      {\beta \deq \tsubst{\alpha}{\gamma} &
        \seq{\Gamma}{\gamma}{\Delta} 
      }
\]
with $x_0 : \F{\alpha_0}{\Delta_0} \in \Gamma$.  By the inductive
hypothesis we get
\seq{\Gamma_1,\Gamma_2,\Delta_0}{\subst{\gamma}{\alpha_0}{x}}{\Delta}.  Because
$x_0$ is not free in $\alpha$,
$\subst{(\tsubst{\alpha}{\gamma})}{\alpha_0}{x_0} =
\tsubst{\alpha}{\subst{\gamma}{\alpha_0}{x_0}}$, so we can reapply \FR:
\[
\infer{\seq{\Gamma_1,\Gamma_2}{\subst{\beta}{\alpha_0}{x_0}}{\F{\alpha}{\Delta}}}
      {{\subst{\beta}{\alpha_0}{x_0}} \deq \tsubst{\alpha}{\subst{\gamma}{\alpha_0}{x_0}} &
        \seq{\Gamma_1,\Gamma_2,\Delta_0}{\subst{\gamma}{\alpha_0}{x}}{\Delta}
      }
\]
Both the input and the output have size 1 more than the size of their
subderivations, and the output subderivation is no bigger than the input
by the inductive hypothesis.

In the case for $\FL$
\[
\infer[\FL]{\seq{\Gamma_1',x:\F{\alpha}{\Delta},\Gamma_2'}{\beta}{C}}
      {\deduce{\seq{\Gamma_1',\Gamma_2',\Delta}{\subst \beta {\alpha}{x}}{C}}{\D}}
\]
with $\Gamma_1,x_0 : \F{\alpha_0}{\Delta_0},\Gamma_2 =
\Gamma_1',x:\F{\alpha}{\Delta},\Gamma_2'$, we distinguish cases on
whether $x = x_0$ or not.  If they are the same (i.e. we have hit a left
rule on $x_0$), then $\alpha_0 = \alpha$ and $\Delta_0 = \Delta$ and
\D\/ is the result, and the size is 1 less than the size of the input.
If they are different, then (because $x_0$ is somewhere in
$\Gamma_1',\Gamma_2'$) by the inductive hypothesis we have a derivation
\[
\D' :: {\seq{(\Gamma_1',\Gamma_2')-x_0,\Delta,\Delta_0}{\subst{\subst \beta {\alpha}{x}}{\alpha_0}{x_0}}{C}}
\]
that is no bigger than \D.  Because $x_0$ is from $\Gamma$ and not
$\Delta$, it does not occur in $\alpha$, so 
\[
{\subst{\subst \beta {\alpha}{x}}{\alpha_0}{x_0}} = 
{\subst{\subst \beta {\alpha_0}{x_0}}{\alpha}{x}}
\]
By (iterating) exchange, we get a derivation 
\[
\D'' :: {\seq{(\Gamma_1',\Gamma_2')-x_0,\Delta_0,\Delta}{\subst{\subst \beta {\alpha_0}{x_0}}{\alpha}{x}}{C}}
\]
whose size is the same as $\D'$ and so no bigger than $\D$.  Applying
$\FL$ to $\D''$ (using the fact that
$(\Gamma_1',x:\F{\alpha}{\Delta},\Gamma_2')-x_0 = \Gamma_1,\Gamma_2$)
derives $\seq{\Gamma_1,\Gamma_2}{\subst{\beta}{\alpha_0}{x_0}}{C}$, and
the size is no bigger than the size of the input.

In the case for $\UR$,
\[
\infer{\seq{\Gamma}{\beta}{\U{x.\alpha}{\Delta}{A}}}
      {\seq{\Gamma,\Delta}{\subst{\alpha}{\beta}{x}}{A}}
\]
the inductive hypothesis gives a
$\D' :: \seq{\Gamma_1,\Gamma_2,\Delta,\Delta_0}{\subst{\subst{\alpha}{\beta}{x}}{\alpha_0}{x_0}}{A}$
and (iterated) exchange gives 
$\D'' ::
\seq{\Gamma_1,\Gamma_2,\Delta_0,\Delta}{\subst{\subst{\alpha}{\beta}{x}}{\alpha_0}{x_0}}{A}$,
both no bigger than \D.  Because $x_0$ is in $\Gamma$ and not $\Delta$,
it is not free in $\alpha$, so 
\[
{\subst{\subst{\alpha}{\beta}{x}}{\alpha_0}{x_0}} = {\subst{\alpha}{\subst{\beta}{\alpha_0}{x_0}}{x}}
\]
Thus, we can derive
\[
\infer{\seq{\Gamma_1,\Gamma_2,\Delta_0}{\subst{\beta}{\alpha_0}{x_0}}{\U{x.\alpha}{\Delta}{A}}}
      {\deduce{\seq{\Gamma_1,\Gamma_2,\Delta_0,\Delta}{\subst{\alpha}{\subst{\beta}{\alpha_0}{x_0}}{x}}{A}}{\D''}}
\]

In the case for $\UL$, 
\[
\infer{\seq{\Gamma}{\beta}{C}}
      {x:\U{x.\alpha}{\Delta}{A} \in \Gamma & 
        \beta \deq \subst{\beta'}{\tsubst{\alpha}{\gamma}}{z} &
        \seq{\Gamma}{\gamma}{\Delta} &
        \seq{\Gamma,\tptm{z}{A}}{\beta'}{C}
      }
\]
we know that $x$ is different that $x_0$ because the types conflict.
The inductive hypotheses give no-bigger derivations of
\[
\seq{\Gamma_1,\Gamma_2\Delta_0}{\subst{\gamma}{\alpha_0}{x_0}}{\Delta} \qquad \seq{\Gamma_1,\Gamma_2,\tptm{z}{A},\Delta_0}{\subst{\beta'}{\alpha_0}{x_0}}{C}
\]
and the latter can be exchanged to
\[
\seq{\Gamma_1,\Gamma_2,\Delta_0,\tptm{z}{A}}{\subst{\beta'}{\alpha_0}{x_0}}{C}
\]
again without increasing the size.  Thus, we can produce
\[
\infer{\seq{\Gamma_1,\Gamma_2,\Delta_0}{\subst{\beta}{\alpha_0}{x}}{C}}
      {\begin{array}{l}
          x:\U{x.\alpha}{\Delta}{A} \in \Gamma_1,\Gamma_2,\Delta_0 \\
          {\subst{\beta}{\alpha_0}{x}} \deq \subst{{\subst{\beta'}{\alpha_0}{x_0}}}{\tsubst{\alpha}{{\subst{\gamma}{\alpha_0}{x_0}}}}{z}\\
          \seq{\Gamma_1,\Gamma_2,\Delta_0}{\subst{\gamma}{\alpha_0}{x_0}}{\Delta} \\
          \seq{\Gamma_1,\Gamma_2,\Delta_0,\tptm{z}{A}}{\subst{\beta'}{\alpha_0}{x_0}}{C}
        \end{array}
      }
\]
where equality is the composition of the \subst{-}{\alpha_0}{x_0}
substitution into the given equality, and rearranging the substitution
(note that $x_0$ does not occur in $\alpha$):
\[
\begin{array}{ll}
\subst{\beta}{\alpha_0}{x_0} & \deq
\subst{\subst{\beta'}{\tsubst{\alpha}{\gamma}}{z}}{\alpha_0}{x_0} 
= 
\subst{\subst{\beta'}{\alpha_0}{x_0}}{\subst{\tsubst{\alpha}{\gamma}}{\alpha_0}{x_0}}{z}
\\
& =
\subst{\subst{\beta'}{\alpha_0}{x_0}}{\tsubst{\alpha}{\subst{\gamma}{\alpha_0}{x_0}}}{z} 
\end{array}
\]

The case for $\cdot$ is immediate.  The case for $\_,\_$ follows from
the two inductive hypotheses, because
$\subst{(\gamma,\alpha/x)}{\alpha_0}{x_0} =
{(\subst{\gamma}{\alpha_0}{x_0},\subst{\alpha}{\alpha_0}{x_0}/x)}$.
\end{proof}


\begin{theorem}[Cut] ~
\begin{enumerate} 
\item  If $\seq{\Gamma,\Gamma'}{\alpha_0}{A_0}$ and $\seq{\Gamma,x_0:A_0,\Gamma'}{\beta}{B}$ 
then $\seq{\Gamma,\Gamma'}{\beta[\alpha_0/x_0]}{B}$ 
\item If $\seq{\Gamma,\Gamma'}{\alpha_0}{A_0}$ and $\seq{\Gamma,x_0:A_0,\Gamma'}{\gamma}{\Delta}$ 
then $\seq{\Gamma,\Gamma'}{\gamma[\alpha_0/x_0]}{\Delta}$ 
\item If $\seq{\Gamma}{\gamma}{\Delta}$ and 
\seq{\Gamma,\Delta}{\beta}{C}
then \seq{\Gamma}{\tsubst{\beta}{\gamma}}{C}.  
\end{enumerate}
\end{theorem}

\begin{proof}
Induction ordering: Part 1 and 2: cut formula, then simultaneous on the
size of \D\/ and \E\/.  Part 3:

Part 1: There are 5 rules, so 25 pairs of final rules.  

\begin{itemize}
\item (5 pairs) Any rule and identity
\[
\deduce{\seq{\Gamma,\Gamma'}{\alpha_0}{A_0}}{\D} \qquad \infer{\seq{\Gamma,x:A,\Gamma'}{z}{Q}}{z:Q \in (\Gamma,x:A,\Gamma')}
\]
There two subcases, depending on whether the variable being cut for is
$z$ or not.  If $z$ is $x_0$ and $A_0$ is $Q$, then \D\/ has the desired
conclusion \seq{\Gamma}{\alpha}{Q}.  If not, then $z:Q \in \Gamma,\Gamma'$, so
the hypothesis rule applies to give \seq{\Gamma,\Gamma'}{z}{Q}.  

\item (5 pairs) Any rule and $\FR$ (right-commutative)
\[
\deduce{\seq{\Gamma,\Gamma'}{\alpha_0}{A_0}}{\D} \qquad
\infer{\seq{\Gamma,x_0:A_0,\Gamma'}{\beta}{\F{\alpha}{\Delta}}}
      {%% \modeof{\Gamma} \vdash \gamma : \modeof{\Delta} & 
        \beta \deq \tsubst{\alpha}{\gamma} &
        \deduce{\seq{\Gamma,x_0:A_0,\Gamma'}{\gamma}{\Delta}}{\E}
      }
\]
By the inductive hypothesis, cutting into \D\/ into \E\/ gives
\seq{\Gamma,\Gamma'}{\subst{\gamma}{\alpha_0}{x_0}}{\Delta}.  By
congruence, $\subst{\beta}{\alpha_0}{x_0} \deq
\subst{\tsubst{\alpha}{\gamma}}{\alpha_0}{x_0}$.  Since $\gamma$ is a
total substitution for all variables in \modeof{\Delta},
$\subst{\tsubst{\alpha}{\gamma}}{\alpha_0}{x} =
\tsubst{\alpha}{\subst{\gamma}{\alpha_0}{x}}$, so
\subst{\beta}{\alpha_0}{x_0} \deq
\tsubst{\alpha}{\subst{\gamma}{\alpha_0}{x}}.  Thus we can reapply the
$\FR$ rule to get
\seq{\Gamma,\Gamma'}{\subst{\beta}{\alpha_0}{x_0}}{\F{\alpha}{\Delta}}.

\item (5 pairs) Any rule and $\UR$ (right-commutative).    
\[
\deduce{\seq{\Gamma,\Gamma'}{\alpha_0}{A_0}}{\D} \qquad
\infer{\seq{\Gamma,x_0:A_0,\Gamma'}{\beta}{\U{x.\alpha}{\Delta}{A}}}
      {\deduce{\seq{\Gamma,x_0:A_0,\Gamma',\Delta}{\subst{\alpha}{\beta}{x}}{A}}{\E}}
\]
The inductive cut of \D\/ into \E\/ gives 
\[
\seq{\Gamma,\Gamma',\Delta}{\subst{\subst{\alpha}{\beta}{x}}{\alpha_0}{x_0}}{A}
\]
Because the variables from $\modeof{\Gamma},\modeof{\Gamma'}$ occur only
in $\beta$, not in $\alpha$, this substitution equals 
{\subst{\alpha}{\subst{\beta}{\alpha_0}{x_0}}{x}} so reapplying the
$\UR$ rule
derives 
{\seq{\Gamma,\Gamma'}{\subst{\beta}{\alpha_0}{x_0}}{\U{x.\alpha}{\Delta}{A}}}.   

\item (2 additional pairs, plus 3 overlapping with above) $\FL$ and
  any rule (left commutative).  

There is one subtlety in this case.  The usual strategy for a left rule
against an arbitrary \E is to push $\E$ into the ``continuation'' of the
\Fsymb-left on $x$.  However, as discussed above, our left rule for
\Fsymb eagerly inverts \emph{all} occurences of $x$, while $\E$ itself
also has $x$ in scope.  Thus, we use Lemma~\ref{lemma:Finv} to pull the
left-inversion to the bottom of \E, and then push that into \D.  On
proof terms, this corresponds to making all references to $x$ in \E
instead refer to the results of the case-analysis at the bottom of $\D$.
This subtlety could be avoided by building contraction into $\FL$, as
discussed above.

Formally, we have
\[
\begin{array}{c}
\infer{\seq{\Gamma,\Gamma'}{\alpha_0}{A_0}}
      {{x}:{\F{\alpha}{\Delta}} \in \Gamma,\Gamma' &
        \deduce{\seq{((\Gamma,\Gamma')-x),\Delta}{\subst {\alpha_0} {\alpha}{x}}{A_0}}{\D}}
\\ \\
\deduce{\seq{\Gamma,x_0:A_0,\Gamma'}{\beta}{C}}{\E}
\end{array}
\]

By left invertibility on \E, we obtain (note that $x \neq x_0$ because
$x_0$ is added to the context in the right-hand derivation) a derivation
$\E'$ of
{\seq{(\Gamma,x:A_0,\Gamma')-x,\Delta}{\subst{\beta}{\alpha}{x}}{C}} that is
no bigger than $\E$.  Because the cut formula is the same, and $\E'$ has
the same size as \E\/, and \D\/ is smaller than the given derivation of
$A_0$, we can apply the inductive hypothesis to cut $\D$ and $\E'$ to
get
\[
{\seq{(\Gamma,\Gamma')-x,\Delta}{\subst{\subst{\beta}{\alpha}{x}}{\subst{\alpha_0}{\alpha}{x}}{x_0}}{C}}.
\]
Commuting substitutions gives
\[
{\subst{{\subst{\beta}{\alpha}{x}}}{\subst{\alpha_0}{\alpha}{x}}{x_0}} = \subst {\beta[\alpha_0/x_0]}{\alpha}{x}
\]
so we can reapply $\FL$ to get
\[
\infer{\seq{\Gamma,\Gamma'}{\beta[\alpha_0/x_0]}{C}}
      {\seq{((\Gamma,\Gamma')-x),\Delta}{\subst {(\beta[\alpha_0/x_0])} {\alpha}{x}}{C}}
\]


\item (2 additional pairs, plus 3 overlapping with above) $\UL$ and any rule (left commutative)
In this case, $x:\U{\alpha}{\Delta}{A} \in \Gamma,\Gamma'$ and
we have
\[
\begin{array}{c}
\infer{\seq{\Gamma,\Gamma'}{\alpha_0}{A_0}}
      {\alpha_0 \deq \subst{\alpha_0'}{\tsubst{\alpha}{\gamma}}{z} &
       \deduce{\seq{\Gamma,\Gamma'}{\gamma}{\Delta}}{\D_1} &
       \deduce{\seq{\Gamma,\Gamma',z:A}{\alpha_0'}{A_0}}{\D_2}
      }
\\ \\
\deduce{\seq{\Gamma,x_0:A_0,\Gamma'}{\beta}{B}}{\E}
\end{array}
\]

Weakening \E with $z$ and then cutting $\D_2$ and $\E$ by the inductive
hypothesis (which applies because $\D_2$ is smaller and weakening does
not change the size) gives
\[
\deduce{\seq{\Gamma,\Gamma',z:A}{\subst{\beta}{\alpha_0'}{x_0}}{B}}{\D_2'}
\]
Thus, we have the first, third, and fourth premises of
\[
\infer{\seq{\Gamma,\Gamma'}{\subst{\beta}{\alpha_0}{x_0}}{A_0}}
      {\begin{array}{l}
          x:\U{\alpha}{\Delta}{A} \in \Gamma,\Gamma' \\
          {\subst{\beta}{\alpha_0}{x_0}} \deq \subst{\subst{\beta}{\alpha_0'}{x_0}}{\tsubst{\alpha}{\gamma}}{z} \\
       {\seq{\Gamma,\Gamma'}{\gamma}{\Delta}} \\
       {\seq{\Gamma,\Gamma',z:A}{\subst{\beta}{\alpha_0'}{x_0}}{B}}
        \end{array}
      }
\]
The equation is proved by
\[
     {\subst{\beta}{\alpha_0}{x_0}} 
\deq \subst{\beta}{\subst{\alpha_0'}{\tsubst{\alpha}{\gamma}}{z}}{x0} = \subst{\subst{\beta}{\alpha_0'}{x_0}}{\tsubst{\alpha}{\gamma}}{z}
\]
where the first step is by congruence with $\beta$ on the 
equality about $\alpha_0$ assumed for the case, and the second is by
properties of substitution ($z$ is not free in $\beta$).  

\item (3 pairs) Right rule or identity and $\FL$ (principal or
  right-commutative).  

Suppose the right-hand derivation ends with $\FL$, and the left-hand
derivation is either a right rule or identity (\dsd{v}) (the cases for
left-rules were covered above).  

We distinguish cases on whether the \FL\/ case-analyzes $x_0$ or not.  If
it does, then, because $A_0 is \F{\alpha}{\Delta}$, the left-hand
derivation must be \FR, and we have a principal cut
\[
\infer{\seq{\Gamma,\Gamma'}{\alpha_0}{\F{\alpha}{\Delta}}}
      {  
        \alpha_0 \deq \tsubst{\alpha}{\gamma} &
        \deduce{\seq{\Gamma,\Gamma'}{\gamma}{\Delta}}{\D}
      }
\qquad
\infer{\seq{\Gamma,x_0:\F{\alpha}{\Delta},\Gamma'}{\beta}{C}}
      {\deduce{\seq{\Gamma,\Gamma',\Delta}{\subst{\beta}{\alpha}{x_0}}{C}}
              {\E}}
\]
Using the inductive hypothesis part 3 to cut \D and \E ($\Delta$ is a
subformula of the original cut formula \F{\alpha}{\Delta}) gives
\[
\seq{\Gamma,\Gamma'}{\tsubst{\subst{\beta}{\alpha}{x_0}}{\gamma}}{C}
\]
By congruence with $\beta$ and because $\gamma$ substitutes only for
variables in $\modeof {\Delta}$,
\[
\subst{\beta}{\alpha_0}{x_0} \deq 
{\subst{\beta}{\tsubst \alpha \gamma}{x_0}} =
{\tsubst{\subst{\beta}{\alpha}{x_0}}{\gamma}} 
\]
So applying Lemma~\ref{lemma:respecteq} gives 
\seq{\Gamma,\Gamma'}{\subst{\beta}{\alpha_0}{x_0}}{C}.  

If not, then we have
\[
\deduce{\seq{\Gamma,\Gamma'}{\alpha_0}{A_0}}{\D}
\quad
\infer{\seq{\Gamma,x_0:A_0,\Gamma'}{\beta}{C}}
      { x : \F{\alpha}{\Delta} \in \Gamma,\Gamma' &
        \deduce{\seq{((\Gamma,x_0:A_0,\Gamma')-x),\Delta}{\subst{\beta}{\alpha}{x}}{C}}{\E}}
\]

We are going to commute $\D$ under \FL on $x$, so need to reroute uses
of $x$ to here by the left-inversion lemma
\[
\D' :: {\seq{((\Gamma,\Gamma')-x),\Delta}{\subst{\alpha_0}{\alpha}{x}}{A_0}}
\]
and $\D'$ is no bigger than \D.

Cutting $\D'$ and $\E$ by the inductive hypothesis gives
\[
\seq{((\Gamma,\Gamma')-x),\Delta}{\subst{\subst{\beta}{\alpha}{x}}{\subst{\alpha_0}{\alpha}{x}}{x_0}}{C}
\]
Because $x_0$ is not free in $\alpha$, 
\[
  {\subst{\subst{\beta}{\alpha}{x}}{\subst{\alpha_0}{\alpha}{x}}{x_0}}
= {\subst{\subst{\beta}{\alpha_0}{x_0}}{\alpha}{x}}
\]
so we can apply \FL
\[
\infer{\seq{\Gamma,\Gamma'}{\subst{\beta}{\alpha_0}{x_0}}{C}}
      {\seq{(\Gamma,\Gamma'-x)}{\subst{\subst{\beta}{\alpha_0}{x_0}}{\alpha}{x}}{C}}
\]
%% \seq{((\Gamma,\Gamma')-x),\Delta}{\subst{\subst{\beta}{\alpha}{x}}{\alpha_0}{x_0}}{C}
%% \]

\item (3 pairs) Right rule or identity and $\UL$ (principal or
  right-commutative).

If $x_0$ is the variable used in the left rule in the right-hand
derivation, then the left-hand derivation must have been derived by
$\UR$, and we have
\[
\begin{array}{l}
\D \quad = \quad \infer{\seq{\Gamma,\Gamma'}{\alpha_0}{\U{x_0.\alpha}{\Delta}{A}}}
   {  
     \deduce{\seq{\Gamma,\Gamma',\Delta}{\subst \alpha {\alpha_0}{x_0}}{A}}{\D'}
   }
\\ \\
\infer{\seq{\Gamma,x_0:\U{x_0.\alpha}{\Delta}{A},\Gamma'}{\beta}{C}}
      {
        \begin{array}{l}
        \beta \deq \subst{\beta'}{\tsubst{\alpha}{\gamma}}{z} \\
        {\E_1 :: \seq{\Gamma,x_0:{\U{x_0.\alpha}{\Delta}{A}},\Gamma'}{\gamma}{\Delta}}\\
        {\E_2 :: \seq{\Gamma,x_0:{\U{x_0.\alpha}{\Delta}{A}},\Gamma',\tptm{z}{A}}{\beta'}{C}}
        \end{array}
      }
\end{array}
\]
First, cutting the original \D and the smaller $\E_1$ and $\E_2$ gives 
\[
\deduce{{\seq{\Gamma,\Gamma'}{\subst{\gamma}{\alpha_0}{x_0}}{\Delta}}}{\E_1'}
\qquad 
\deduce{{\seq{\Gamma,\Gamma',\tptm{z}{A}}{\subst{\beta'}{\alpha_0}{x_0}}{C}}}{\E_2'}
\]
Cutting $\E_1'$ \emph{into} $\D'$ (the cut formula $\Delta$ is a
subformula of $\U{x_0.\alpha}{\Delta}{A}$, so it is okay that the derivations are
not known to be smaller) gives
\[
\deduce
{\seq{\Gamma,\Gamma'}{\tsubst{\alpha}{\subst{\gamma}{\alpha_0}{x}}}{A}} {\D_1'}
\]
Cutting $\D_1'$ into $\E_2'$ gives 
\[
\seq{\Gamma,\Gamma'}{\subst{\subst{\beta'}{\alpha_0}{x_0}}{\subst{\alpha}{\alpha_0}{x_0}}{z}}{A}
\]
But we have 
\[
\subst{\beta}{\alpha_0}{x_0} \deq 
\subst{(\subst{\beta'}{\tsubst{\alpha}{\gamma}}{z})}{\alpha_0}{x_0} = 
{\subst{\subst{\beta'}{\alpha_0}{x_0}}{\tsubst{\alpha}{\subst{\gamma}{\alpha_0}{x_0}}}{z}}
\]
by commuting substitutions, which gives the result.  

On the other hand, if the subject of the left rule $x$ is not equal to
$x_0$, then we have
\[
\deduce{\seq{\Gamma,\Gamma'}{\alpha_0}{A_0}}
       {
         \D
       }
\quad
\infer{\seq{\Gamma,x_0:A_0,\Gamma'}{\beta}{C}}
      {
        \begin{array}{l}
          x : \U{x.\alpha}{\Delta}{A} \in \Gamma,\Gamma' \\
          \beta \deq \subst{\beta'}{\tsubst{\alpha}{\gamma}}{z} \\
          \seq{\Gamma,x_0:A_0,\Gamma'}{\gamma}{\Delta} \\
          \seq{\Gamma,x_0:A_0,\Gamma',\tptm{z}{A}}{\beta'}{C}
        \end{array}
      }
\]

By the inductive hypotheses we get 
\[
\seq{\Gamma,\Gamma'}{\subst{\gamma}{\alpha_0}{x_0}}{\Delta}
\qquad
\seq{\Gamma,\Gamma',z:A}{\subst{\beta'}{\alpha_0}{x_0}}{C}
\]
so we can derive
\[
\infer{\seq{\Gamma,x_0:A_0,\Gamma'}{\subst{\beta}{\alpha_0}{x_0}}{C}}
      {
        \begin{array}{l}
          x : \U{x.\alpha}{\Delta}{A} \in \Gamma,\Gamma' \\
          {\subst{\beta}{\alpha_0}{x_0}} \deq \subst{\subst{\beta'}{\alpha_0}{x_0}}{\tsubst{\alpha}{\subst{\gamma}{\alpha_0}{x_0}}}{z} \\
          \seq{\Gamma,\Gamma'}{\subst{\gamma}{\alpha_0}{x_0}}{\Delta} \\
          \seq{\Gamma,\Gamma',\tptm{z}{A}}{\subst{\beta'}{\alpha_0}{x_0}}{C}
        \end{array}
      }
\]
For the equation, we get
\[
\subst{\beta}{\alpha_0}{x_0} \deq
\subst{\subst{\beta'}{\tsubst{\alpha}{\gamma}}{z}}{\alpha_0}{x_0}
\]
by congruence on the assumed equation, and then commute substitutions.  

Part 3: 

\end{itemize}

For part 2, there are just two right-commutative cases: For
\[
\seq{\Gamma,\Gamma'}{\alpha_0}{A_0}
\qquad
\seq{\Gamma,x_0:A_0,\Gamma'}{\cdot}{\cdot}
\]
we also have $\subst \cdot {\alpha_0}{x_0} \deq \cdot$ and
\seq{\Gamma,\Gamma'}{\cdot}{\cdot}.  For
\[
\seq{\Gamma,\Gamma'}{\alpha_0}{A_0}
\qquad
\infer{\seq{\Gamma,x_0:A_0,\Gamma'}{\gamma,\alpha/x}{\Delta,x:A}}
      {\seq{\Gamma,x_0:A_0,\Gamma'}{\gamma}{\Delta} &
        \seq{\Gamma,x_0:A_0,\Gamma'}{\alpha}{A}
      }
\]
we have $\subst{(\gamma,\alpha/x)}{\alpha_0}{x_0} 
= (\subst{\gamma}{\alpha_0}{x_0},\subst{\alpha}{\alpha_0}{x_0})$, and
 \[
\seq{\Gamma,\Gamma'}{\subst{\gamma}{\alpha_0}{x_0}}{\Delta} \quad
\seq{\Gamma,\Gamma'}{\subst{\alpha}{\alpha_0}{x_0}}{A}
\]
by the inductive hypotheses, so we can reapply the rule to conclude
\seq{\Gamma,\Gamma'}{\subst{(\gamma,\alpha/x)}{\alpha_0}{x_0}}{\Delta,x:A}.

For part 3, we induct on $\Delta$, reducing a simulatenous cut to
iterated single-variable cuts.  If $\Delta$ is empty, then we have
\[
\seq{\Gamma}{\cdot}{\cdot}
\qquad
\deduce{\seq{\Gamma,\cdot}{\beta}{C}}{\E}
\]
and we return \E, noting that $\subst{\beta}{\cdot} = \beta$.  Otherwise
we have
\[
\infer{\seq{\Gamma}{\gamma,\alpha/x}{\Delta,x:A}}
      {\deduce{\seq{\Gamma}{\gamma}{\Delta}}{\D_1} &
        \deduce{\seq{\Gamma}{\alpha}{A}}{\D_2}}
\qquad
\deduce{\seq{\Gamma,\Delta,x:A}{\beta}{C}}{\E}
\]
Using the inductive hypothesis to cut $\D_2$ into $\E$ ($A$ is smaller
than $\Delta,x:A$) gives
\[
\deduce{\seq{\Gamma,\Delta}{\subst{\beta}{\alpha}{x}}{C}}
       {\E'}
\]
Using the inductive hypothesis to cut $\D_1$ into $\E'$ ($\Delta$ is
smaller than $\Delta$) gives
\[
\seq{\Gamma}{\tsubst{\subst{\beta}{\alpha}{x}}{\gamma}}{C}
\]
Because $\gamma$ substitutes for $\Delta$ and not $\Gamma$,
\[
\tsubst{\beta}{\gamma,\alpha/x}
= {\tsubst{\subst{\beta}{\alpha}{x}}{\gamma}}
\]
\end{proof}

\begin{corollary}[Contraction over contraction] \label{cor:controver}
\item If
\seq{\Gamma,x:A,y:A,\Gamma'}{\alpha}{C}
then
\seq{\Gamma,z:A,\Gamma'}{\tsubst \alpha {z/x,z/y}}{C}
\end{corollary}

\begin{proof}  Contraction can be shown by cutting with a renaming substitution.
The mode substitution $z/x,z/y$ is a variable-for-variable substitution,
and is type-preserving between ${x:A,y:A}$ and ${\Gamma,z:A,\Gamma'}$.
Therefore, by identity (part 2),
\seq{\Gamma,z:A,\Gamma'}{z/x,z/y}{x:A,y:A}.  Thus, by cut (part 2), we
obtain the result.
\end{proof}

\begin{corollary}[Right-invertibility of \Usymb] \label{lemma:Uinv}
If $\seq{\Gamma}{\beta}{\U{x.\alpha}{\Delta}{A}}$ then 
{\seq{\Gamma,\Delta}{\subst{\alpha}{\beta}{x}}{A}}.
\end{corollary}

\begin{proof}
$\UL$ with identities in both premises gives a derivation
\[
\infer{\seq{\Gamma,\Delta,x:{\U{x.\alpha}{\Delta}{A}}}{\alpha}{A}}
      {
        \alpha = z[\alpha[\vec{x/x}]/z] & 
        \seq{\Gamma,\Delta}{\vec{x/x}}{\Delta} &
        \seq{\Gamma,\Delta,x:{\U{x.\alpha}{\Delta}{A}},z:A}{z}{A}
      }
\]
Weakening the assumed derivation to 
\seq{\Gamma,\Delta}{\beta}{\U{x.\alpha}{\Delta}{A}}
and then cutting for $x$ in the above gives the result.  

\[
\infer{\seq{\Gamma,\Delta}{\subst{\alpha}{\beta}{x}}{A}}
      {\seq{\Gamma,\Delta}{\beta}{\U{x.\alpha}{\Delta}{A}} & 
       \seq{\Gamma,\Delta,x:{\U{x.\alpha}{\Delta}{A}}}{\alpha}{A}
      }
\]

\end{proof}

\begin{theorem}[Fusion] ~
\begin{enumerate} 

\item $\F{\alpha}{\Delta,x:\F{\beta}{\Delta'},\Delta''} \dashv \vdash
  \F{\subst{\alpha}{\beta}{x}}{\Delta,\Delta',\Delta''}$

\item $\U{x.\alpha}{\Delta,y:\F{\beta}{\Delta'},\Delta''}{A} \dashv \vdash
  \U{x.\subst{\alpha}{\beta}{y}}{\Delta,\Delta',\Delta''}{A}$

\item 
$\U{x.\alpha}{\Delta}{\U{y.\beta}{\Delta'}{A}} \dashv \vdash
 \U{x.\subst{\beta}{\alpha}{y}}{\Delta,\Delta'}{A}$

\end{enumerate}
\end{theorem}

\begin{proof}

\[
\infer{
  \seq{z:\F{\alpha}{\Delta,x:\F{\beta}{\Delta'},\Delta''}}
      {z}
      {\F{\subst{\alpha}{\beta}{x}}{\Delta,\Delta',\Delta''}}
}
{
  \infer{\seq{\Delta,x:\F{\beta}{\Delta'},\Delta''}{\alpha}{\F{\subst{\alpha}{\beta}{x}}{\Delta,\Delta',\Delta''}}}
        {
          \infer{\seq{\Delta,\Delta'',\Delta'}{\subst{\alpha}{\beta}{x}}{\F{\subst{\alpha}{\beta}{x}}{\Delta,\Delta',\Delta''}}}
                {\subst{\alpha}{\beta}{x} \deq \tsubst{\subst{\alpha}{\beta}{x}}{\vec{z/z}} & 
                 \seq{\Delta,\Delta'',\Delta'}{\vec{z/z}}{\Delta,\Delta',\Delta''}
                }
        }
}
\]

FIXME: obvious append lemma for substitutions

\[
\infer{
  \seq{z:{\F{\subst{\alpha}{\beta}{x}}{\Delta,\Delta',\Delta''}}}
      {z}
      {\F{\alpha}{\Delta,x:\F{\beta}{\Delta'},\Delta''}}
}
{  
\infer{\seq{\Delta,\Delta',\Delta''}
           {\subst{\alpha}{\beta}{x}}
           {\F{\alpha}{\Delta,x:\F{\beta}{\Delta'},\Delta''}}}
      {\alpha[\beta/x] = \alpha[\vec{y/y},\beta/x,\vec{z/z}] &
        \infer{\seq{\Delta,\Delta',\Delta''}{\vec{y/y},\beta/x,\vec{z/z}} {\Delta,x:\F{\beta}{\Delta'},\Delta''}}
              {\seq{\Delta,\Delta',\Delta''}{\vec{y/y}}{\Delta} & 
               \infer{\seq{\Delta,\Delta',\Delta''}{\beta}{\F{\beta}{\Delta'}}}
                     {\beta = \beta[\vec{w/w}] & \seq{\Delta,\Delta',\Delta''}{\vec{w/w}}{\Delta'}} &
               \seq{\Delta,\Delta',\Delta''}{\vec{z/z}}{\Delta''} }
      }
}
\]

\end{proof}

\subsection{Equational Theory of Proofs}



\subsection{Categorical Semantics}

TODO: Mitchell



\section{Equational Theory on Derivations}
\label{sec:equational}

In this section, an equational theory describing $\beta\eta$-equality of
derivations.  We use this equational theory in the categorical semantics
below, and to reason about terms in encoded languages (for example, to
prove that a pair of entailments is an isomorphism, we show that the
maps compose to the identity up to these equations).

First, we need a notation for derivations of the $\alpha \spr \beta$
judgement in Figure~\ref{fig:2multicategory}.  We assume names for
constants are given in the signature $\Sigma$, and write $1_\alpha$ for
reflexivity, $s_1;s_2$ for transitivity (in diagramatic order), and
$s_1[s_2/x]$ for congruence.  We extend the signature $\Sigma$ to allow
axioms for equality of transformations $s_1 \deq s_2$ (for two
derivations of the same judgement $s_1,s_2 ::
\wfsp{\psi}{\alpha}{\beta}{p}$), and define equality to be the least
congruence closed under those axioms and some associativity, unit, and
interchange laws, which are the 2-category axioms extended to the
multicategorical case (see the extended version for details).  As with
equality of context descriptors, we think of all definitions as being
parametrized by \deq-equivalence-classes of transformations, not raw
syntax.

To simplify the axiomatic description of equality, we use a notation for
derivations where the admissible transformation, identity, and cut rules
are internalized as explicit rules---so the calculus has the flavor of
an explicit substitution one.
%% \[
%% \begin{array}{rcl}
%% \D & ::= & \Ident{x} \mid \Trd{s}{\D} \mid \Cut{\D_1}{\D_2}{x} \mid
%%  \FLd{x}{\Delta}{\D} \mid \FRd{\gamma}{s}{\vec{\D_i/x_i}} \mid \ULd{x}{}{s}{\vec{\D_i/x_i}}{z}{\D} \mid \URd{\Delta}{\D} 
%% \end{array}
%% \]
We write proof terms for these plus the 4 \Usymb/\Fsymb\, rules (the
hypothesis rule for atoms is derivable from these) as follows.
\begin{small}
\[
\begin{array}{c}
\infer{{\Gamma,x:A} \vdash_{x} x : {A}}{}
\qquad
\infer{{\Gamma} \vdash_{\alpha} \Trd{s}{d} : {A}}
      {s :: \alpha \spr \beta &
        {\Gamma} \vdash_{\beta} d : {A}}
\qquad
\infer{{\Gamma} \vdash_{\subst{\beta}{\alpha}{x}} \Cut{e}{d}{x} : {B}}
      {{\Gamma,x:A} \vdash_{\beta} e : {B} &
        {\Gamma} \vdash_{\alpha} d : {A}}
\\\\ 
\infer{{\Gamma,x:\F{\alpha}{\Delta},\Gamma'} \vdash_{\beta} (\FLd{x}{\Delta}{d}) : {C}}
      {{\Gamma,\Gamma',\Delta} \vdash_{\subst \beta {\alpha}{x}} d : {C}}
\quad
\infer{{\Gamma} \vdash_{\beta} \FRd{}{s}{\vec{d_i/x_i}} : {\F{\alpha}{\Delta}}}
      {%% \modeof{\Gamma} \vdash \gamma : \modeof{\Delta} & 
        s :: \beta \spr \tsubst{\alpha}{\gamma} &
        {\Gamma} \vdash_{\gamma} \vec{d_i/x_i} : {\Delta} 
      }
\\\\
\infer{{\Gamma} \vdash_{\beta} \ULd{x}{}{s}{\vec{d_i/x_i}}{z}{d} : {C}}
      {
        x:\U{x.\alpha}{\Delta}{A} \in \Gamma &
        s :: \beta \spr \subst{\beta'}{\tsubst{\alpha}{\gamma}}{z} &
        {\Gamma} \vdash_{\gamma} {\vec{d_i/x_i}} : {\Delta} &
        {\Gamma,\tptm{z}{A}} \vdash_{\beta'} d' : {C}
      }
\quad
\infer{{\Gamma} \vdash_{\beta} \URd{\Delta}{d} : {\U{x.\alpha}{\Delta}{A}}}
      {{\Gamma,\Delta} \vdash_{\subst{\alpha}{\beta}{x}} d : {A}}
\end{array}
\]
\end{small}

%% We write \FRs\/ for $\FRd{\vec{x/x}}{1_\alpha}{\Ident{x}/x} ::
%% \seq{\Gamma}{\alpha}{\F{\alpha}{\Delta}}$ when $\Delta \subseteq \Gamma$
%% and we write and \ULs{x} for $\ULd{x}{\vec{x/x}}{1_\alpha}{\Ident{x}/x}{z.z} ::
%% \seq{\Gamma}{\alpha}{A}$ when $x:\U{x.\alpha}{\Delta}{A} \in \Gamma$ and
%% $\Delta \subseteq \Gamma$.  

The equational theory of derivations is the least congruence containing
the following equations.  
\begin{small}
\[
\begin{array}{rcll} 
\Cut{\D}{\Ident{x}}{x} & \deq & \D \\
\Cut{\Ident{x}}{\D}{x} & \deq & \D \\
\Cut{\D_1}{\D_2}{x} & \deq & \D_1 \text{ if $x \# \D_1$}\\
\Cut{(\Cut{\D_1}{\D_2}{x})}{\D_3}{y} & \deq & \Cut{(\Cut{\D_1}{\D_3}{y})}{\Cut{\D_2}{\D_3}{y}}{x}\\
\end{array}
\qquad
\begin{array}{rcll}
\Trd{1}{\D} & \deq & \D\\
\Trd{(s_1;s_2)}{\D} & \deq & \Trd{{s_1}}{\Trd{{s_2}}{\D}} \\
\Trd{(\subst{s_2}{s_1}{x})}{\Cut{\D_2}{\D_1}{x}} & \deq & \Cut{\Trd{{s_2}}{\D_2}}{\Trd{{s_1}}{\D_1}}{x} \\
\end{array}
\]
\[
\begin{array}{rcll}
\Cut{(\FLd{x_0}{\Delta}{\D})}{\FRd{}{s}{\vec{\D_i/x_i}}}{x_0} & \deq & \Trd{(1_\beta[s/x_0])}{\D[\vec{\D_i/x_i}]} & \dsd{F\beta} \\
\Cut{(\ULd{x_0}{}{s}{\vec{\D_i}/x_i}{z}{\D'})}{\URd{\Delta}{\D}}{x_0} & \deq & \Trd{(s[1_{\alpha}/{x_0}])}{\Cut{\D'}{(\D[{\vec{d_i}/x_i}])}{z}} & \dsd{U\beta} \\
\D :: \seq{\Gamma,x:\F{\alpha}{\Delta},\Gamma'}{\beta}{C} & \deq &
\FLd{x}{\Delta}{\Cut{\D}{\FRd{}{1}{\Delta/\Delta}}{x}} & \dsd{F\eta}\\
\D :: \seq{\Gamma}{\beta}{\U{x.\alpha}{\Delta}{A}} & \deq & \URd{\Delta}{\Cut{(\ULd{x}{}{1}{\Delta/\Delta}{z}{z})}{\D}{x}} & \dsd{U\eta}\\
\end{array}
\]
\end{small}

In the top-left, the first two equations say that identity is a unit for
cut.  The third says that non-occurence of a variable is a projection.
The fourth is functoriality of cut.  In the top-right, the first two
rules say that the action of a transformation is functorial, and the
third says that it commutes with cut.  The typing in the third rule is
$\D_1 :: \seq{\Gamma}{\alpha'}{A}$ and $\D_2 ::
\seq{\Gamma,x:A}{\beta'}{C}$ and $s_1 :: \alpha \spr \alpha'$ and $s_2
:: \beta \spr \beta'$, so both sides are derivations of as derivations
of \seq{\Gamma}{\subst{\beta}{\alpha}{x}}{C}.  Finally, we have the
$\beta\eta$-laws for \dsd{F} and \dsd{U}.  The $\beta$ laws are the
principal cut cases from our cut elimination proof.  The $\eta$ laws
witness left-invertibility of \Fsymb\, and right-invertibility of
\Usymb.



\section{Equational Theory on Derivations}
\label{sec:equational}

In this section, we give our framework a proof term notation and
equational theory describing $\beta\eta$-equality.  We use this
equational theory in the categorical semantics below, and to reason
about terms in encoded logics (for example, to prove that a pair of
entailments is an isomorphism, we show that the maps compose to the
identity up to these equations).  First, we need a notation for
derivations of the $\alpha \spr \beta$ judgement in
Figure~\ref{fig:2multicategory}.  We assume names for constants are
given in the signature $\Sigma$, and write $1_\alpha$ for reflexivity,
$s_1;s_2$ for transitivity (in diagramatic order), and $s_1[s_2/x]$ for
congruence.  We extend the signature $\Sigma$ to allow axioms for
equality of transformations $s_1 \deq s_2$ (for two derivations of the
same judgement $s_1,s_2 :: \wfsp{\psi}{\alpha}{\beta}{p}$), and define
equality to be the least congruence closed under those axioms and some
associativity, unit, and interchange laws, which are the 2-category
axioms extended to the multicategorical case (see the extended version
for details).  As with equality of context descriptors, we think of all
definitions as being parametrized by \deq-equivalence-classes of
transformations, not raw syntax.

To simplify the axiomatic description of equality, we use a notation for
derivations where the admissible transformation, identity, and cut rules
are internalized as explicit rules---so the calculus has the flavor of
an explicit substitution one:  
\[
\begin{array}{rcl}
\D & ::= & \Ident{x} \mid \Trd{s}{\D} \mid \Cut{\D_1}{\D_2}{x} \mid
 \FLd{x}{\Delta}{\D} \mid \FRd{\gamma}{s}{\vec{\D_i/x_i}} \mid \ULd{x}{}{s}{\vec{\D_i/x_i}}{z}{\D} \mid \URd{\Delta}{\D} 
\end{array}
\]
We omit the primitive hypothesis rule for atoms (it is derivable), and
write $x$ for identity and \Trd{s}{\D} for respect for transformations
(identity for atoms combines these) and \Cut{\D_1}{\D_2}{x} for cut.
The next 4 terms correspond to the 4 \Usymb/\Fsymb\, rules from
Figure~\ref{fig:sequent}.  We write $d :: \seq{\Gamma}{\alpha}{A}$ to
mean $d$ is a derivation of that sequent; we use this slightly unusual
notation to call attention to the fact that we think of the terms as a
linear representation of the derivations themselves, rather than as a
``raw'' term syntax.  
\[
\begin{array}{c}
\infer{x :: \seq{\Gamma,x:A}{x}{A}}{}
\qquad
\infer{\Trd{s}{d} :: \seq{\Gamma}{\alpha}{A}}
      {s :: \alpha \spr \beta &
       d :: \seq{\Gamma}{\beta}{A}}
\qquad
\infer{\Cut{e}{d}{x} :: \seq{\Gamma}{\subst{\beta}{\alpha}{x}}{B}}
      {e :: \seq{\Gamma,x:A}{\beta}{B} &
       d :: \seq{\Gamma}{\alpha}{A}}
\\\\ 
\infer{(\FLd{x}{\Delta}{d}) :: \seq{\Gamma,x:\F{\alpha}{\Delta},\Gamma'}{\beta}{C}}
      {d :: \seq{\Gamma,\Gamma',\Delta}{\subst \beta {\alpha}{x}}{C}}
\quad
\infer{\FRd{}{s}{\vec{d_i/x_i}} :: \seq{\Gamma}{\beta}{\F{\alpha}{\Delta}}}
      {%% \modeof{\Gamma} \vdash \gamma : \modeof{\Delta} & 
        s :: \beta \spr \tsubst{\alpha}{\gamma} &
        \vec{d_i/x_i} :: \seq{\Gamma}{\gamma}{\Delta} 
      }
\\\\
\infer{\ULd{x}{}{s}{\vec{d_i/x_i}}{z}{d} :: \seq{\Gamma}{\beta}{C}}
      {
        x:\U{x.\alpha}{\Delta}{A} \in \Gamma &
        s :: \beta \spr \subst{\beta'}{\tsubst{\alpha}{\gamma}}{z} &
        {\vec{d_i/x_i}} :: \seq{\Gamma}{\gamma}{\Delta} &
        d' :: \seq{\Gamma,\tptm{z}{A}}{\beta'}{C}
      }
\quad
\infer{\URd{\Delta}{d} :: \seq{\Gamma}{\beta}{\U{x.\alpha}{\Delta}{A}}}
      {d :: \seq{\Gamma,\Delta}{\subst{\alpha}{\beta}{x}}{A}}
\end{array}
\]
We elide weakenings and exchanges in the terms.

%% We write \FRs\/ for $\FRd{\vec{x/x}}{1_\alpha}{\Ident{x}/x} ::
%% \seq{\Gamma}{\alpha}{\F{\alpha}{\Delta}}$ when $\Delta \subseteq \Gamma$
%% and we write and \ULs{x} for $\ULd{x}{\vec{x/x}}{1_\alpha}{\Ident{x}/x}{z.z} ::
%% \seq{\Gamma}{\alpha}{A}$ when $x:\U{x.\alpha}{\Delta}{A} \in \Gamma$ and
%% $\Delta \subseteq \Gamma$.  

The equational theory of derivations is the least congruence containing
the following equations.  
\[
\begin{array}{rcll} 
\Cut{\D}{\Ident{x}}{x} & \deq & \D \\
\Cut{\Ident{x}}{\D}{x} & \deq & \D \\
\Cut{\D_1}{\D_2}{x} & \deq & \D_1 \text{ if $x \# \D_1$}\\
\Cut{(\Cut{\D_1}{\D_2}{x})}{\D_3}{y} & \deq & \Cut{(\Cut{\D_1}{\D_3}{y})}{\Cut{\D_2}{\D_3}{y}}{x}\\
\end{array}
\qquad
\begin{array}{rcll}
\Trd{1}{\D} & \deq & \D\\
\Trd{(s_1;s_2)}{\D} & \deq & \Trd{{s_1}}{\Trd{{s_2}}{\D}} \\
\Trd{(\subst{s_2}{s_1}{x})}{\Cut{\D_2}{\D_1}{x}} & \deq & \Cut{\Trd{{s_2}}{\D_2}}{\Trd{{s_1}}{\D_1}}{x} \\
\end{array}
\]
\[
\begin{array}{rcll}
\Cut{(\FLd{x_0}{\Delta}{\D})}{\FRd{}{s}{\vec{\D_i/x_i}}}{x_0} & \deq & \Trd{(1_\beta[s/x_0])}{\D[\vec{\D_i/x_i}]} & \dsd{F\beta} \\
\Cut{(\ULd{x_0}{}{s}{\vec{\D_i}/x_i}{z}{\D'})}{\URd{\Delta}{\D}}{x_0} & \deq & \Trd{(s[1_{\alpha}/{x_0}])}{\Cut{\D'}{(\D[{\vec{d_i}/x_i}])}{z}} & \dsd{U\beta} \\
\D :: \seq{\Gamma,x:\F{\alpha}{\Delta},\Gamma'}{\beta}{C} & \deq &
\FLd{x}{\Delta}{\Cut{\D}{\FRd{}{1}{\Delta/\Delta}}{x}} & \dsd{F\eta}\\
\D :: \seq{\Gamma}{\beta}{\U{x.\alpha}{\Delta}{A}} & \deq & \URd{\Delta}{\Cut{(\ULd{x}{}{1}{\Delta/\Delta}{z}{z})}{\D}{x}} & \dsd{U\eta}\\
\end{array}
\]

In the top-left, the first two equations say that identity is a unit for
cut.  The third says that non-occurence of a variable is a projection.
The fourth is functoriality of cut.  In the next group, the first two
rules say that the action of a transformation is functorial, and the
third says that it commutes with cut.  The typing in the third rule is
$\D_1 :: \seq{\Gamma}{\alpha'}{A}$ and $\D_2 ::
\seq{\Gamma,x:A}{\beta'}{C}$ and $s_1 :: \alpha \spr \alpha'$ and $s_2
:: \beta \spr \beta'$, so both sides are derivations of as derivations
of \seq{\Gamma}{\subst{\beta}{\alpha}{x}}{C}.  Finally, we have the
$\beta\eta$-laws for \dsd{F} and \dsd{U}.  The $\beta$ laws are the
principal cut cases from our cut elimination proof.  The $\eta$ laws
witness left-invertibility of \Fsymb\, and right-invertibility of
\Usymb.



\newcommand\wfsp[4]{\ensuremath{#1 \vdash #2 \spr_{#4} #3}}

\section{Sequent Calculus}

\subsection{Mode Theories}

\begin{figure}
\[
\begin{array}{llll}
\text{modes} & p & & (constants) \\
\text{mode contexts} & \psi & ::= & \cdot \mid \psi,\tptm{x}{p} \\
%% \text{context descriptors} & \alpha,\beta & ::= & x \mid \dsd{c}(\alpha_1,\ldots,\alpha_n) \\
%% \text{substitutions} & \gamma,\delta & ::= & \cdot \mid \gamma,\alpha/x\\
\end{array}
\]

\framebox{Context descriptors/mode morphisms \oftp{\psi}{\alpha}{p}}

\[
\infer{\oftp{\psi}{x}{p}}
      {x:p \in \psi}
\quad
\infer{\oftp{\psi}{\dsd{c}(\alpha_1,\ldots,\alpha_n)}{q}}
      {\dsd{c} : p_1,\ldots,p_n \to q \in \Sigma &
       \oftp{\psi}{\alpha_i}{p_i}
      }
\quad
\infer{\oftp{\psi}{\alpha_i/x_i}{\overline{x_i:q_i}}}
      {\oftp{\psi}{\alpha_i}{q_i}}
\]

\framebox{Structural properties/mode transformations \wfsp{\psi}{\alpha}{\alpha'}{p}}

\[
\begin{array}{c}
\infer{\wfsp{\psi}{\alpha}{\alpha}{p}}{}
\qquad
\infer{\wfsp{\psi}{\alpha_1}{\alpha_3}{p}}
      {\wfsp{\psi}{\alpha_1}{\alpha_2}{p} &
       \wfsp{\psi}{\alpha_2}{\alpha_3}{p} &
      }
\\ \\
\infer{\wfsp{\psi,\psi'}{\subst{\beta}{\alpha}{x}}{\subst{\beta'}{\alpha'}{x}}{q}}
      {\wfsp{\psi,x:p,\psi'}{\beta}{\beta'}{q} &
       \wfsp{\psi,\psi'}{\alpha}{\alpha'}{p}}
\qquad
\infer{\wfsp{\psi}{\alpha}{\alpha'}{p}}
      {\alpha \spr \alpha' \in \Sigma}
\end{array}
\]

\label{fig:2multicategory}
\caption{Syntax for mode theories}
\end{figure}

The first layer of the logic is a type theory whose types we will call
\emph{modes}, and whose terms we will call \emph{context descriptors} or
\emph{mode morphisms}.  We begin with a simple type theory with
variables and constants, as described in
Figure~\ref{fig:2multicategory}.  We assume a fixed signature $\Sigma$
of constants \dsd{c}, and we write $\oftp{\psi}{\alpha}{q}$ for context
descriptor well-formedness and $\oftp{\psi}{\gamma}{\psi'}$ for
substitution well-formedness.  The context $\psi$ here enjoys the
cartesian structural properties (weakening, exchange, contraction).
Simultaneous substitution into terms is defined as usual.

Next, we need a notation for presenting object-logic structural
properties, which, as described above, will in general be directed
transformations between context descriptors.  For
\oftp{\psi}{\alpha,\alpha'}{p}, we define a judgement
     {\wfsp{\psi}{\alpha}{\alpha'}{p}} representing such
     transformations; it is the least precongruence closed under axioms
     specified in a signature $\Sigma$.  For example, to say that a mode
     $p$ with a context monoid $(\odot,1)$ is affine, we would specify
     in $\Sigma$ a transformation \wfsp{x:p}{x}{1}{p}.  Then, using the
     rules in the figure, we can for example derive a transformation $(x
     \odot y) \spr (1 \odot y) \spr y$ that, when applied
     (contravariantly) to a sequent, will allow weakening $y$ to $x
     \odot y$.

In Section~\ref{sec:equalityofderiv}, where we discuss equality of
sequent calculus derivations, we will need an equational theory between
two structural property derivations $s \deq s' ::
\wfsp{\psi}{\alpha}{\alpha'}{q}$.  Because this equational theory does
not influence provability in the sequent calculus, only identity of
proofs, we defer the details.

One choice in the design of this language of context descriptors is how
to handle symmetric structural properties.  One possibility is to
present a desired equation $\alpha = \alpha'$ as an isomorphism, with
axioms $s : \alpha \spr \alpha'$ and $s' : \alpha' \spr \alpha$ (and,
writing $s_1;s_2$ for derivations by the above transitivity rule,
equations $s;s' = 1_{\alpha}$ and $s';s = 1_{\alpha'}$).  While this is
conceptually and technically sufficient, we have found it helpful in
examples to also use a ``strict'' equality of context descriptors which
is implicitly respected by all types and judgements.  This simplifies
the description of some situations, though the difference is important
mainly at the level of identity of derivations rather than
provability---for example, we can make a binary operation $\odot$ into a
strict monoid, rather than adding associator and unitor isomorphisms,
which requires more equations between structural transformations.  To
support this, we allow $\Sigma$ to contain axioms for equations $\alpha
\deq \beta$ and define a judgement $\psi \vdash \alpha \deq \beta : p$
as the least congruence containing these axioms (i.e. add symmetry to
the rules for transformations).  Because the context descriptors and
their equality are defined independently of any subsequent judgements,
we take the liberty of suppressing this equality by using $\alpha$ to
refer to a context-descriptor-modulo-\deq---that is, we assume a
metatheory with quotient sets/types, and use meta-level equality for
object-level equality, as recently advocated by
\citet{altenkirchkaposi16qit}.  For example, because the judgement
\wfsp{\psi}{\alpha}{\beta}{p} is indexed by equivalence classes of
context descriptions, the reflexivity rule above implicitly means
$\alpha \deq \beta$ implies $\alpha \spr \beta$.

A signature $\Sigma$ consisting of some number of constants $c$,
equations $\alpha \deq \beta$, transformations $\alpha \spr \beta$ (and
equalities of transformations $s \deq s'$) specifies a \emph{mode theory}.

\subsection{Sequent Calculus Rules}

\begin{figure}
\[
\begin{array}{l}
%% \begin{array}{llll}
%% \text{Types} & A & ::= & P \mid \F{\alpha}{\Delta} \mid \U{\alpha}{\Delta}{A} \\
%% \end{array}
%% \\ \\
\framebox{Types $A,B,C$ \quad \wftype{A}{p}}
\qquad
\infer{\wftype{P}{p}}{}
\\ \\
\infer{\wftype{\F{\alpha}{\Delta}}{q}}
      {\oftp{\psi}{\alpha}{q} &
        \wfctx{\Delta}{\psi}}
\qquad
\infer{\wftype{\U{x.\alpha}{\Delta}{A}}{q}}
      {\oftp{\psi,x:q}{\alpha}{p} &
        \wfctx{\Delta}{\psi} &
        \wftype{A}{p}
      }
\\ \\
\framebox{Contexts $\Gamma,\Delta$ \quad \wfctx{\Gamma}{\psi}}
\qquad
\infer{\wfctx{\cdot}{\cdot}}{}
\qquad
\infer{\wfctx{\Gamma,x:A}{\psi,x:p}}
      {\wfctx{\Gamma}{\psi} &
        \wftype{A}{p}}
\\ \\
\framebox{\seq{\Gamma}{\alpha}{A} where $\wfctx{\Gamma}{\psi}$ and $\wftype{A}{q}$ and  $\oftp{\psi}{\alpha}{q}$}
\\ \\
\infer[\dsd{v}]{\seq{\Gamma}{\beta}{P}}
      {x:P \in \Gamma & \beta \spr x}
\\ \\
\infer[\FR]{\seq{\Gamma}{\beta}{\F{\alpha}{\Delta}}}
      {%% \modeof{\Gamma} \vdash \gamma : \modeof{\Delta} & 
        \beta \spr \tsubst{\alpha}{\gamma} &
        \seq{\Gamma}{\gamma}{\Delta} 
      }
\quad
\infer[\FL]{\seq{\Gamma,x:\F{\alpha}{\Delta},\Gamma'}{\beta}{C}}
      {\seq{\Gamma,\Gamma',\Delta}{\subst \beta {\alpha}{x}}{C}}
%% \infer{\seq{\Gamma}{\beta}{C}}
%%       {{x}:{\F{\alpha}{\Delta}} \in \Gamma & 
%%         \oftp{\modeof{\Gamma},{x'} : {\modeof{\F{\alpha}{\Delta}}}}{\beta'}{\modeof{C}} &
%%         \beta \deq \tsubst{\beta'}{x/x'} &
%%         \seq{\Gamma,\Delta}{\subst {\beta'} {\alpha}{x'}}{C}}
\\ \\
\infer[\UR]{\seq{\Gamma}{\beta}{\U{x.\alpha}{\Delta}{A}}}
      {\seq{\Gamma,\Delta}{\subst{\alpha}{\beta}{x}}{A}}
\qquad
\infer[\UL]{\seq{\Gamma}{\beta}{C}}
      {\begin{array}{l}
          x:\U{x.\alpha}{\Delta}{A} \in \Gamma \\
          \beta \spr \subst{\beta'}{\tsubst{\alpha}{\gamma}}{z} \\
          \seq{\Gamma}{\gamma}{\Delta} \\
          \seq{\Gamma,\tptm{z}{A}}{\beta'}{C}
       \end{array}
      }
\\ \\
\framebox{\seq{\Gamma}{\gamma}{\Delta} where $\wfctx{\Gamma}{\psi}$ and $\wfctx{\Delta}{\psi'}$ and  $\oftp{\psi}{\gamma}{\psi'}$}
\\ \\
\infer[\cdot]{\seq{\Gamma}{\cdot}{\cdot}}
      {}
\qquad
\infer[\_,\_]{\seq{\Gamma}{\gamma,\alpha/x}{\Delta,x:A}}
      {\seq{\Gamma}{\gamma}{\Delta} &
       \seq{\Gamma}{\alpha}{A}
      }
\end{array}
\]    
\caption{Sequent Calculus}
\label{fig:sequent}
\hrule
\end{figure}

The whole sequent calculus is parametrized by a mode theory $\Sigma$,
which is an implicit argument to all of the definitions in
Figure~\ref{fig:sequent}.  The first judgement assigns each
proposition/type $A$ a mode $p$.  Encodings of non-modal logics will
generally only make use of one mode, while modal logics use different
modes to represent different notions of truth, such as the linear and
cartesian categories in \citet{bentonwadler96adjoint} and the
true/valid/lax judgements in \citet{pfenningdavies}.  The next judgement
assigns each context $\Gamma$ a mode context $\psi$.  Formally, we think
of contexts as ordered: we do not regard $x:A,y:B$ and $y:B,x:A$ and the
same context, though we will have an admissible exchange rule that
passes between derivations in one and the other.

The sequent judgement \seq{\Gamma}{\alpha}{A} relates a context
$\wfctx{\Gamma}{\psi}$ and a type $\wftype{A}{p}$ and context descriptor
\oftp{\psi}{\alpha}{p}, while the substitution relates
$\wfctx{\Gamma}{\psi}$ and $\wfctx{\Delta}{\psi'}$ and
$\oftp{\psi}{\gamma}{\psi'}$. Because $\wfctx{\Gamma}{\psi}$ means that
each variable in $\Gamma$ is in $\psi$, where $x : A_i \in \Gamma$
implies $x : p_i$ in $\psi$ with \wftype{A_i}{p_i}, we think of $\Gamma$
as binding variable names both in $\alpha$ and for use in the
derivation.

As discussed in the introduction, a guiding principle is to make the
following rules admissible (see Section~\ref{sec:synprop-long} for
details), which express respect for structural transformations and
structurality-over-structurality:
\[
\begin{array}{c}
\infer{\seq{\Gamma}{\alpha}{A}}
      {\alpha \spr \beta &
       \seq{\Gamma}{\beta}{A}}
\qquad
\infer{\seq{\Gamma,x:A}{x}{A}}{}
\qquad
\infer{\seq{\Gamma}{\subst{\beta}{\alpha}{x}}{B}}
    {\seq{\Gamma,x:A}{\beta}{B} &
     \seq{\Gamma}{\alpha}{A}}
\\ \\
\infer{\seq{\Gamma,y:A}{\alpha}{C}}
      {\seq{\Gamma}{\alpha}{C}}
\quad
\infer{\seq{\Gamma,y:B,x:A}{\alpha}{C}}
      {\seq{\Gamma,x:A,y:B}{\alpha}{C}}
\qquad
\infer{\seq{\Gamma,x:A}{\subst \alpha x y}{C}}
      {\seq{\Gamma,x:A,y:A}{\alpha}{C}}
\end{array}
\]

We now explain the rules for the sequent calculus; the reader may wish
to refer to the examples in Section~\ref{sec:exampleencodings} in
parallel with this abstract description.

We assume atomic propositions $P$ are given a specified mode $p$, and
state identity as a primitive rule only for them.  If we had instead
stated the identity rule as \seq{\Gamma,x:P}{x}{P}, then respect for
structural transformation would not be admissible, so we allow a $\beta
\spr x$ premise here.  Using a structural property at a leaf of a
derivation is common in e.g. affine logic, where the derivation of
$\beta \spr x$ would use weakening to forget any additional resources.

Next, we consider the \F{\alpha}{\Delta} type, which ``internalizes''
the context operation $\alpha$ as a type/proposition.  Syntactically, we
view the context $\Delta = x_1:A_1,\ldots,x_n:A_n$ where
\wftype{A_i}{p_i} as binding the variables $x_i:p_i$ in $\alpha$, so for
example \F{\alpha}{x:A,y:B} and \F{\alpha[x \leftrightarrow
    x']}{x':A,y:B} are $\alpha$-equivalent types (in de Bruijn form we
would write \F{\alpha}{A_1,\ldots,A_n} and use indices in $\alpha$).
The type formation rule says that \dsd{F} moves covariantly along a mode
morphism $\alpha$, representing a ``product'' (in a loose sense) of
types in $\Delta$ structured according to the context descriptor
$\alpha$. A typical binary instance of \dsd{F} is a multiplicative
product ($A \otimes B$ in linear logic), which, given a binary context
descriptor $\odot$ as in the introduction, is written \F{x \odot
  y}{x:A,y:B}.  A typical nullary instance is a unit (1 in linear
logic), written \F{1}{}.  A typical unary instance is the \dsd{F}
connective of adjoint logic, which for a unary context descriptor
constant $\dsd{f} : \dsd{p} \to \dsd{q}$ is written \F{\dsd{f}(x)}{x:A}.
We sometimes write \F{\dsd{f}}{A} in this case, eliding the variable
name, and similarly for a unary \dsd{U}.

The rules for our \dsd{F} connective capture a pattern common to all of
these examples.  On the left, \F{\alpha}{\Delta} ``decays'' into
$\Delta$, but \emph{structuring the uses of resources in $\Delta$ with
  $\alpha$ by the substitution \subst{\beta}{\alpha}{x}}.  We assume
that $\Delta$ is $\alpha$-renamed to avoid collision with $\Gamma$ (the
proof term here would be a ``\dsd{split}'' that binds variables for each
position in $\Delta$).  On the right, the \FR\/ rule says that you must
rewrite (using structural transformations) the context descriptor to
have an $\alpha$ at the outside, with a mode substitution $\gamma$ that
divides the exisitng resources up between the positions in $\Delta$, and
then prove each formula in $\Delta$ using the specified resources.  We
leave the typing of $\gamma$ implicit, though there is officially a
requirement $\oftp{\psi}{\gamma}{\psi'}$ where $\wfctx{\Gamma}{\psi}$
and $\wfctx{\Delta}{\psi'}$, as required for the second premise to be a
well-formed sequent.  Another way to understand this rule is to begin
with the ``axiomatic \FR''
\[
\infer{\seq{\Delta}{\alpha}{\F{\alpha}{\Delta}}}{}
\]
which says that there is a map from $\Delta$ to \F{\alpha}{\Delta} along
$\alpha$.  Then, in the same way that a typical injection rule for
coproducts builds a precomposition into an ``axiomatic injection'' such
as $\dsd{inl} :: A \vdash A + B$, the \FR\/ rule builds a precomposition
with $\seq{\Gamma}{\gamma}{\Delta}$ and then an application of a
structural rule $\beta \spr \alpha[\gamma]$ into the ``axiomatic''
version, in order to make cut and respect for transformations
admissible.

Next, we turn to $\U{x.\alpha}{\Delta}{A}$.  As a first approximation,
if we ignore the context descriptors and structural properties,
\U{-}{\Delta}{A} behaves like $\Delta \to A$, and the \UL\/ and \UR\/
rules are an annotation of the usual structural/cartesian rules for
implication.  In a formula \U{x.\alpha}{\Delta}{A}, the context
descriptor $\alpha$ has access to the variables from $\Delta$ as well as
an extra variable $x$, whose mode is the same as the \emph{overall mode
  of \U{x.\alpha}{\Delta}{A}}, while the mode of $A$ itself is the mode
of the conclusion of $\alpha$---in terms of typing, \dsd{U} is
contravariant where \dsd{F} is covariant.  It is helpful to think of $x$
as standing for the context that will be used to prove
\U{x.\alpha}{\Delta}{A}.  For example, a typical function type $A \lolli
B$ represented by \U{x.x \otimes y}{y:A}{B}, which says to extend the
``current context'' $x$ with a resource $y$.  In \UR, the context
descriptor $\beta$ being used to prove the \dsd{U} is substituted
\emph{for $x$} in $\alpha$ (dual to \FL, which substituted $\alpha$ into
$\beta$).  The ``axiomatic'' \UL\/ instance
\[
\infer{\seq{\Delta,x:\U{x.\alpha}{\Delta}{A}}{\alpha}{A}}{}
\]
says that \U{x.\alpha}{\Delta}{A} together with $\Delta$ has a map to
$A$ along $\alpha$.  (The bound $x$ in $x.\alpha$ subscript is tacitly
renamed to match the name of the assumption in the context, in the same
way that the typing rule for $\lambda x.e : \Pi x:A.B$ requires
coordination between two variables in different scopes).  The full rule
builds in precomposition with \seq{\Gamma}{\gamma}{\Delta},
postcomposition with \seq{\Gamma,z:A}{\beta'}{C}, and precomposition
with a structural transformation $\beta \spr \beta'[\alpha[\gamma]/z]$.

Finally, the rules for substitutions are pointwise.  In examples, we
often will write the components of a substitution directly as multiple
premises of \FR\/ and \UL\/, eliding packaging them into a list with the
$\_,\_$ and $\cdot$ rules.

One subtle point about the $\FL$ rule is that there are two competing
principles: making the rules ``obviously'' structural-over-structural,
and reducing inessential non-determinism.  Here, we choose the later,
and treat the assumption of \F{\alpha}{\Delta} affinely, removing it
from the context when it is used.  It will turn out that the judgement
nonetheless enjoys contraction-over-contraction
(Corollary~\ref{cor:contraction}), because contraction
for negatives is built into the \UL-rule, and contraction for positives
follows from this and the fact that we can always reconstruct a positive
from what it decays to on the left (c.f. how purely positive formulas
have contraction in linear logic).

Additives can be added to this sequent calculus; e.g. a mode $p$ has
sums if
\[
\begin{array}{c}
\infer{\wftype{A+B}{p}}
      {\wftype{A}{p} &
       \wftype{B}{p}}
\quad
\infer{\seq{\Gamma}{\alpha}{A + B}}
      {\seq{\Gamma}{\alpha}{A}}
\quad
\infer{\seq{\Gamma}{\alpha}{A + B}}
      {\seq{\Gamma}{\alpha}{B}}
\\ \\
\infer{\seq{\Gamma,x:A+B,\Gamma'}{\beta}{C}}
      {\seq{\Gamma,\Gamma',y:A}{\subst \beta y x}{C} &
       \seq{\Gamma,\Gamma',z:B}{\subst \beta z x}{C} 
      }
%% \infer{\wftype{A \& B}{p}}
%%       {\wftype{A}{p} &
%%        \wftype{B}{p}}
%% \qquad
%% \infer{\seq{\Gamma,x:A \& B}{\alpha}{C}}
%%       {\seq{\Gamma,y:A}{\alpha[y/x]}{C}}
%% \quad
%% \infer{\seq{\Gamma}{\alpha}{A + B}}
%%       {\seq{\Gamma}{\alpha}{B}}
\end{array}
\]

\subsection{Syntactic Metatheory}
\label{sec:synprop-short}

We summarize some syntactic properties; see
Appendix~\ref{sec:synprop-long} for proofs.  Define the \emph{size} of a
derivation of \seq{\Gamma}{\alpha}{A} or \seq{\Gamma}{\gamma}{\Delta} to
be the number of inference rules for these judgements $(\dsd{v},\FL,
\FR, \UL, \UR, \cdot, \_,\_)$ used in it (i.e., the evidence that
variables are in a context and derivations of structural transformations
do not contribute to the size).  Sizes are necessary for the cut proof,
where we sometimes weaken or invert a derivation before applying the
inductive hypothesis.

\begin{lemma}[Respect for Transformations] ~ \label{lem:respectspr}
\begin{enumerate}
\item If \seq{\Gamma}{\beta}{A} and $\beta' \spr \beta$ then
  \seq{\Gamma}{\beta'}{A}, and the resulting derivation has the same
  size as the given one.
\item If \seq{\Gamma}{\gamma}{\Delta} and $\gamma' \spr \gamma$ then
  \seq{\Gamma}{\gamma'}{\Delta}, and the resulting derivation has the
  same size as the given one.
\end{enumerate}
\end{lemma}

\begin{lemma}[Weakening over weakening] \label{lem:weakening} ~
\begin{enumerate}
\item If \seq{\Gamma,\Gamma'}{\alpha}{C} then
\seq{\Gamma,\tptm{z}{A},\Gamma'}{\alpha}{C}, and the resulting
derivation has the same size as the given one.  
\item If \seq{\Gamma,\Gamma'}{\gamma}{\Delta} then
\seq{\Gamma,\tptm{z}{A},\Gamma'}{\gamma}{\Delta}, and the resulting
derivation has the same size as the given one.  
\item If \seq{\Gamma,\Gamma''}{\alpha}{C} then
\seq{\Gamma,\Gamma',\Gamma''}{\alpha}{C}, and the resulting
derivation has the same size as the given one.  
\end{enumerate}
\end{lemma}

\begin{lemma}[Exchange over exchange] \label{lem:exchange}
If \seq{\Gamma,x:A,y:B,\Gamma'}{\alpha}{C} then
\seq{\Gamma,y:B,x:A,\Gamma'}{\alpha}{C}, and the resulting derivation
has the same size as the given one.  (And similarly for substitutions,
and exchange can be iterated).  
\end{lemma}

\begin{theorem}[Identity] ~ \label{thm:identity}
\begin{enumerate}
\item If $x:A \in \Gamma$ then $\seq{\Gamma}{x}{A}$.
\item If $\oftp{\modeof{\Gamma}}{\rho}{\modeof{\Delta}}$ is a
  variable-for-variable mode substitution such that $x:A \in \Delta$
  implies $\rho(x) : A \in \Gamma$, then $\seq{\Gamma}{\rho}{\Delta}$.
\end{enumerate}
\end{theorem}

\begin{lemma}[Left-invertibility of \Fsymb] \label{lem:Finv}
If $\D :: \seq{\Gamma_1,x_0:\F{\alpha_0}{\Delta_0},\Gamma_2}{\beta}{C}$
and then there is a derivation $D' ::
\seq{\Gamma_1,\Gamma_2,\Delta_0}{\subst{\beta}{\alpha_0}{x_0}}{C}$ and
$size(\D') \le size(\D)$ (and analogously for substitutions).
\end{lemma}

\begin{theorem}[Cut] ~ \label{thm:cut}
\begin{enumerate} 
\item  If $\seq{\Gamma,\Gamma'}{\alpha_0}{A_0}$ and $\seq{\Gamma,x_0:A_0,\Gamma'}{\beta}{B}$ 
then $\seq{\Gamma,\Gamma'}{\beta[\alpha_0/x_0]}{B}$ 
\item If $\seq{\Gamma,\Gamma'}{\alpha_0}{A_0}$ and $\seq{\Gamma,x_0:A_0,\Gamma'}{\gamma}{\Delta}$ 
then $\seq{\Gamma,\Gamma'}{\gamma[\alpha_0/x_0]}{\Delta}$ 
\item If $\seq{\Gamma}{\gamma}{\Delta}$ and 
\seq{\Gamma,\Delta}{\beta}{C}
then \seq{\Gamma}{\tsubst{\beta}{\gamma}}{C}.  
\end{enumerate}
\end{theorem}

\begin{corollary}[Contraction over contraction] \label{cor:contraction}
\item If
\seq{\Gamma,x:A,y:A,\Gamma'}{\alpha}{C}
then
\seq{\Gamma,z:A,\Gamma'}{\tsubst \alpha {z/x,z/y}}{C}
\end{corollary}

\begin{corollary}[Right-invertibility of \Usymb] \label{cor:Uinv}
If $\seq{\Gamma}{\beta}{\U{x.\alpha}{\Delta}{A}}$ then 
{\seq{\Gamma,\Delta}{\subst{\alpha}{\beta}{x}}{A}}.
\end{corollary}

\subsection{General Constructions}

Next, we give a couple of general constructions inside the logic that
are helpful in many examples.  The following ``fusion'' lemmas (which
are additionally type isomorphisms, not just interprovabilities) relate
$\Fsymb$ and $\Usymb$.  Special cases include: $A \times (B \times C)$
is isomorphic to a primitive triple product $\{x:A,y:B,z:C\}$; currying;
and associativity of $n$-ary functions ($A_1,\ldots,A_n \to
(B_1,\ldots,B_m \to C)$ is isomorphic to $A_1,\ldots,A_n,B_1,\ldots,B_m
\to C$).  The derivations are in Appendix~\ref{sec:synprop-long}.

\begin{lemma}[Fusion] ~ \label{lem:fusion}
\begin{enumerate} 

\item $\F{\alpha}{\Delta,x:\F{\beta}{\Delta'},\Delta''} \dashv \vdash
  \F{\subst{\alpha}{\beta}{x}}{\Delta,\Delta',\Delta''}$

\item $\U{x.\alpha}{\Delta,y:\F{\beta}{\Delta'},\Delta''}{A} \dashv \vdash
  \U{x.\subst{\alpha}{\beta}{y}}{\Delta,\Delta',\Delta''}{A}$

\item 
$\U{x.\alpha}{\Delta}{\U{y.\beta}{\Delta'}{A}} \dashv \vdash
 \U{x.\subst{\beta}{\alpha}{y}}{\Delta,\Delta'}{A}$

\end{enumerate}
\end{lemma}

The types respect the structural transformations, covariantly for
\Fsymb\/ and contravariantly for \Usymb\/.

\begin{lemma}[Types Respect Structural Transformations] ~ \label{lem:typespr}
\begin{enumerate}
\item 
 If $\alpha \spr \beta$ then $\F{\alpha}{\Delta} \vdash
 \F{\beta}{\Delta}$

\item If $\alpha \spr \beta$ then $\U{x.\beta}{\Delta}{A} \vdash
  \U{x.\alpha}{\Delta}{A}$
\end{enumerate}
\end{lemma}

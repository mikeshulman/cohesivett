\newcommand\wfsp[4]{\ensuremath{#1 \vdash #2 \spr_{#4} #3}}

\section{Sequent Calculus}
\label{sec:syntax}

\newcommand\wfsig[1]{\ensuremath{#1 \, \dsd{sig}}}
\newcommand\deqtms[5]{\ensuremath{#1 \vdash_{#2} #3 \deq #4 : #5}}
\newcommand\wfsps[5]{\ensuremath{#1 \vdash_{#2} #3 \spr_{#5} #4}}

\subsection{Mode Theories}

\begin{figure}
\[
\begin{array}{l}
\framebox{Signatures \wfsig{\Sigma}}
\qquad
\infer{\wfsig{\cdot}}
      {}
\qquad
\infer{\wfsig{(\Sigma,p \, \dsd{mode})}}
      {\wfsig{\Sigma}}
\qquad
\infer{\wfsig{(\Sigma,c : \, p_1,\ldots,p_n \to q)}}
      {\wfsig{\Sigma} &
        (p_1 \, \dsd{mode},\ldots,p_n \, \dsd{mode},q \, \dsd{mode}) \in \Sigma
      }
\\ \\
\infer{\wfsig{(\Sigma, (\alpha \deq \alpha' : \psi \to p))}}
      {\wfsig{\Sigma} &
        \psi \dsd{ctx}_\Sigma & 
        p \, \dsd{mode} \in \Sigma &
        \oftps{\psi}{\Sigma}{\alpha}{p} & 
        \oftps{\psi}{\Sigma}{\alpha'}{p} 
      }
\qquad
\infer{\wfsig{(\Sigma, (\alpha \spr \alpha' : \psi \to p))}}
      {\wfsig{\Sigma} &
        \psi \dsd{ctx}_\Sigma & 
        p \, \dsd{mode} \in \Sigma &
        \oftps{\psi}{\Sigma}{\alpha}{p} & 
        \oftps{\psi}{\Sigma}{\alpha'}{p} 
      }
\\\\
\ifthenelse{\boolean{short}}{}
{\framebox{Mode contexts $\psi \, \dsd{ctx}_{\Sigma}$}
\qquad
\infer{\cdot \, \dsd{ctx}_{\Sigma}}{}
\qquad
\infer{(\psi, x:p) \, \dsd{ctx}_{\Sigma}}
      {\psi \, \dsd{ctx}_{\Sigma} & 
        p \, \dsd{mode} \in \Sigma
      }
\\\\
}
\framebox{Context descriptors \oftps{\psi}{\Sigma}{\alpha}{p},
  where $\psi \, \dsd{ctx}_\Sigma$ and $p \, \dsd{mode} \in \Sigma$}
\qquad
\infer{\oftps{\psi}{\Sigma}{x}{p}}
      {x:p \in \psi}
\quad
\infer{\oftps{\psi}{\Sigma}{\dsd{c}(\alpha_1,\ldots,\alpha_n)}{q}}
      {(\dsd{c} : p_1,\ldots,p_n \to q) \in \Sigma &
       \oftps{\psi}{\Sigma}{\alpha_i}{p_i}
      }
\\\\
\framebox{Mode Substitutions \oftps{\psi}{\Sigma}{\gamma}{\psi'}, where
  $\psi \, \dsd{ctx}_\Sigma$ and $\psi' \, \dsd{ctx}_\Sigma$ }
\qquad
\infer{\oftps{\psi}{\Sigma}{\cdot}{\cdot}}{}
\qquad
\infer{\oftps{\psi}{\Sigma}{\gamma,\alpha/x}{\psi',x:p}}
      {\oftps{\psi}{\Sigma}{\gamma}{\psi'} &
        \oftps{\psi}{\Sigma}{\alpha}{p}}

\ifthenelse{\boolean{short}}{}
{\framebox{Equalities of mode morphisms
  \deqtms{\psi}{\Sigma}{\alpha}{\alpha'}{p},
where $\psi \, \dsd{ctx}_\Sigma$ and $p \, \dsd{mode} \in \Sigma$
and \oftps{\psi}{\Sigma}{\alpha}{p}
and \oftps{\psi}{\Sigma}{\alpha'}{p}
}
\qquad
\infer{\deqtms{\psi}{\Sigma}{\alpha}{\alpha}{p}}{}
\qquad
\infer{\deqtms{\psi}{\Sigma}{\alpha_1}{\alpha_2}{p}}
      {\deqtms{\psi}{\Sigma}{\alpha_2}{\alpha_1}{p}}
\qquad
\infer{\deqtms{\psi}{\Sigma}{\alpha_1}{\alpha_3}{p}}
      {\deqtms{\psi}{\Sigma}{\alpha_1}{\alpha_2}{p} &
        \deqtms{\psi}{\Sigma}{\alpha_2}{\alpha_3}{p} &
      }
\\ \\
\infer{\deqtms{\psi,\psi'}{\Sigma}{\subst{\beta}{\alpha}{x}}{\subst{\beta'}{\alpha'}{x}}{q}}
      {\deqtms{\psi,x:p,\psi'}{\Sigma}{\beta}{\beta'}{q} &
        \deqtms{\psi,\psi'}{\Sigma}{\alpha}{\alpha'}{p}}
\qquad
\infer{\deqtms{\psi}{\Sigma}{\alpha}{\alpha'}{p}}
      {(\alpha \deq \alpha' : \psi \to p) \in \Sigma}
}
\\\\
\framebox{Structural transformations \wfsps{\psi}{\Sigma}{\alpha}{\alpha'}{p},
where \oftps{\psi}{\Sigma}{\alpha}{p}
and \oftps{\psi}{\Sigma}{\alpha'}{p}
}
\qquad
\infer{\wfsps{\psi}{\Sigma}{\alpha}{\alpha}{p}}{}
\\\\
\infer{\wfsps{\psi}{\Sigma}{\alpha_1}{\alpha_3}{p}}
      {\wfsps{\psi}{\Sigma}{\alpha_1}{\alpha_2}{p} &
       \wfsps{\psi}{\Sigma}{\alpha_2}{\alpha_3}{p} &
      }
\qquad
\infer{\wfsps{\psi,\psi'}{\Sigma}{\subst{\beta}{\alpha}{x}}{\subst{\beta'}{\alpha'}{x}}{q}}
      {\wfsps{\psi,x:p,\psi'}{\Sigma}{\beta}{\beta'}{q} &
       \wfsps{\psi,\psi'}{\Sigma}{\alpha}{\alpha'}{p}}
\qquad
\infer{\wfsps{\psi}{\Sigma}{\alpha}{\alpha'}{p}}
      {(\alpha \spr \alpha' : \psi \to p) \in \Sigma}
\end{array}
\]

\caption{Syntax for mode theories}
\label{fig:2multicategory}
\end{figure}

The first layer of our framework is a type theory whose types we will call
\emph{modes}, and whose terms we will call \emph{context descriptors} or
\emph{mode morphisms}.  The only modes are atomic/base types $p$.  A
term is either a variable (bound in a context $\psi$) or a typed $n$-ary
constant (function symbol) \dsd{c} applied to terms of the appropriate
types.

This is formalized in the notion of signature, or \emph{mode theory},
defined in Figure~\ref{fig:2multicategory}.  The judgement $\wfsig
\Sigma$ means that $\Sigma$ is a well-formed signature.  The top line
says that a signature is either empty, or a signature extended with a
new mode declaration, or a signature extended with a typed
constant/function symbol, all of whose modes are declared previously in
the signature.  The notation $p_1,\ldots,p_n \to q$ is not itself a
mode, but notation for declaring a function symbol in the signature (it
cannot occur on the right-hand side of a typing judgement).  For
example, the type and term constructors for a monoid $(\odot,1)$ are
represented by a signature $\dsd{p} \, \dsd{mode}, \dsd{\odot} :
(\dsd{p},\dsd{p} \to \dsd{p}), 1 : (\to \dsd{p})$.

\ifthenelse{\boolean{short}}{
We elide the rules for 
the judgement $\psi \, \dsd{ctx}_\Sigma$, which simply says that each
mode used in the 
context of variable declarations $\psi$ is declared in $\Sigma$.  
}
{
The judgement $\psi \, \dsd{ctx}_\Sigma$ defines well-formedness of a
context of variable declarations relative to a signature $\Sigma$: each
mode in the context must be declared in the signature.}  The judgement
$\oftps{\psi}{\Sigma}{\alpha}{p}$ defines well-typedness of context
descriptor terms, which are either a variable declared in the context,
or a constant declared in the signature applied to arguments of the
correct types.  The judgement $\oftps{\psi}{\Sigma}{\gamma}{\psi'}$
defines a substitution as a tuple of terms in the standard way.  The
context $\psi$ in these judgements enjoys the cartesian structural
properties (associativity, unit, weakening, exchange, contraction).
Simultaneous substitution into terms and substitutions is defined as
usual (e.g.  $x[\rho,\alpha/x] := \alpha$ and
$\dsd{c}(\vec{\alpha_i})[\gamma] := \dsd{c}(\alpha_i[\gamma])$).

Returning to the top of the figure, the final two rules of the judgement
$\wfsig{\Sigma}$ permit two additional forms of signature declaration.
The first of these extends a signature with an equational axiom between
two terms $\alpha$ and $\alpha'$ that have the same mode $p$, in the
same context $\psi$, relative to the prior signature $\Sigma$.  These
equational axioms will be used to encode reversible object language
structural properties, such as associativity, commutativity, and unit
laws.  For example, to specify the right unit law for the above monoid
$(\odot,1)$, we add an axiom $(x \odot 1 \deq x : (x : \dsd{p}) \to
\dsd{p})$ to the signature, which can be read as ``$x \odot 1$ is equal
to $x$ as a morphism from $(x : \dsd{p})$ to \dsd{p}''.  The judgement
\deqtms{\psi}{\Sigma}{\alpha}{\alpha'}{p} (omitted from the figure; the
rules are the same as for $\spr$ plus symmetry) is the least congruence
closed under these axioms.

The second of these extends a signature with a directed structural
transformation axiom between two terms $\alpha$ and $\alpha'$ that have
the same mode $p$, in the same context $\psi$, relative to the prior
signature $\Sigma$.  As discussed above, these structural
transformations will be used to represent object language structural
properties such as weakening and contraction that are not invertible.
The judgement \wfsps{\psi}{\Sigma}{\alpha}{\alpha'}{p} defines these
transformations: it is the least precongruence (preorder compatible with
the term formers) closed under the axioms specified in the signature
$\Sigma$.  For example, to say that the above monoid $(\odot,1)$ is
affine, we add in $\Sigma$ a transformation axiom $(x \spr 1 : (x:\dsd{p}) \to
{\dsd{p}})$.
%% Then, using the rules in the figure, we can for example derive a
%% transformation $(x \odot y) \spr (1 \odot y) \spr y$ that, when
%% applied (contravariantly) to a sequent, will allow weakening $y$ to
%% $x \odot y$.
\ifthenelse{\boolean{short}}{}
{An alternative to including the judgement $\alpha \deq \alpha'$ would be
to present a desired equation $\alpha \deq \alpha'$ as an isomorphism,
with transformation axioms $s : \alpha \spr \alpha'$ and $s' : \alpha'
\spr \alpha$.  While this is conceptually and technically sufficient, we
have found it helpful in examples to use ``strict'' equality of context
descriptors.  This simplifies the description of some situations, though
the difference is important mainly at the level of identity of
derivations rather than provability---for example, we can make a binary
operation $\odot$ into a strict monoid, rather than adding associator
and unitor isomorphisms.

}
Because context descriptors
$\alpha$ and their equality $\alpha_1 \deq \alpha_2$ are defined prior
to the subsequent judgements, we suppress this equality by using
$\alpha$ to refer to a term-modulo-\deq---that is, we assume a
metatheory with quotient sets/types, and use meta-level equality for
object-level equality, as recently advocated by
\citet{altenkirchkaposi16qit}.  For example, because the judgement
\wfsp{\psi}{\alpha}{\beta}{p} is indexed by equivalence classes of
context descriptions, the reflexivity rule above implicitly means
$\alpha \deq \beta$ implies $\alpha \spr \beta$.
\ifthenelse{\boolean{short}}{}
{
As discussed in Section~\ref{sec:equational}, we will eventually need an
equational theory between two structural property derivations $s \deq s'
:: \wfsp{\psi}{\alpha}{\alpha'}{q}$.  Because this equational theory
does not influence provability in the sequent calculus, only identity of
proofs, we defer the details to that section.  

}
In examples, we will notate a signature declaration introducing a term
constant/function symbol by showing the function symbol applied to
variables, rather than writing the formal $\dsd{c} : p_1,\ldots,p_n \to
q$. For example, we write $x : \dsd{p}, y : \dsd{p} \vdash x \odot y :
\dsd{p}$ for $\odot : \dsd{p},\dsd{p} \to \dsd{p}$.  We also suppress
the signature $\Sigma$.

\subsection{Sequent Calculus Rules}

\begin{figure}
\begin{small}
\[
\begin{array}{l}
%% \begin{array}{llll}
%% \text{Types} & A & ::= & P \mid \F{\alpha}{\Delta} \mid \U{\alpha}{\Delta}{A} \\
%% \end{array}
%% \\ \\
\framebox{Types $A,B,C$ \quad \wftype{A}{p}}
\qquad
\infer{\wftype{P}{p}}{}
\qquad
\infer{\wftype{\F{\alpha}{\Delta}}{q}}
      {\oftp{\psi}{\alpha}{q} &
        \wfctx{\Delta}{\psi}}
\qquad
\infer{\wftype{\U{x.\alpha}{\Delta}{A}}{q}}
      {\oftp{\psi,x:q}{\alpha}{p} &
        \wfctx{\Delta}{\psi} &
        \wftype{A}{p}
      }
\\ \\
\framebox{Contexts $\Gamma,\Delta$ \quad \wfctx{\Gamma}{\psi}}
\qquad
\infer{\wfctx{\cdot}{\cdot}}{}
\qquad
\infer{\wfctx{\Gamma,x:A}{\psi,x:p}}
      {\wfctx{\Gamma}{\psi} &
        \wftype{A}{p}}
\\ \\
\framebox{\seq{\Gamma}{\alpha}{A} where $\wfctx{\Gamma}{\psi}$ and $\wftype{A}{q}$ and  $\oftp{\psi}{\alpha}{q}$}
\quad
\infer[\FL]{\seq{\Gamma,x:\F{\alpha}{\Delta},\Gamma'}{\beta}{C}}
      {\seq{\Gamma,\Gamma',\Delta}{\subst \beta {\alpha}{x}}{C}}
\quad
\infer[\FR]{\seq{\Gamma}{\beta}{\F{\alpha}{\Delta}}}
      {%% \modeof{\Gamma} \vdash \gamma : \modeof{\Delta} & 
        \beta \spr \tsubst{\alpha}{\gamma} &
        \seq{\Gamma}{\gamma}{\Delta} 
      }
%% \infer{\seq{\Gamma}{\beta}{C}}
%%       {{x}:{\F{\alpha}{\Delta}} \in \Gamma & 
%%         \oftp{\modeof{\Gamma},{x'} : {\modeof{\F{\alpha}{\Delta}}}}{\beta'}{\modeof{C}} &
%%         \beta \deq \tsubst{\beta'}{x/x'} &
%%         \seq{\Gamma,\Delta}{\subst {\beta'} {\alpha}{x'}}{C}}
\\ \\
\infer[\UL]{\seq{\Gamma}{\beta}{C}}
      {\begin{array}{llll}
          x:\U{x.\alpha}{\Delta}{A} \in \Gamma &
          \beta \spr \subst{\beta'}{\tsubst{\alpha}{\gamma}}{z} &
          \seq{\Gamma}{\gamma}{\Delta} &
          \seq{\Gamma,\tptm{z}{A}}{\beta'}{C}
       \end{array}
      }
\quad
\infer[\UR]{\seq{\Gamma}{\beta}{\U{x.\alpha}{\Delta}{A}}}
      {\seq{\Gamma,\Delta}{\subst{\alpha}{\beta}{x}}{A}}
\quad
\infer[\dsd{v}]{\seq{\Gamma}{\beta}{P}}
      {x:P \in \Gamma & \beta \spr x}
\\ \\
\framebox{\seq{\Gamma}{\gamma}{\Delta} where $\wfctx{\Gamma}{\psi}$ and $\wfctx{\Delta}{\psi'}$ and  $\oftp{\psi}{\gamma}{\psi'}$}
\qquad
\infer[\cdot]{\seq{\Gamma}{\cdot}{\cdot}}
      {}
\qquad
\infer[\_,\_]{\seq{\Gamma}{\gamma,\alpha/x}{\Delta,x:A}}
      {\seq{\Gamma}{\gamma}{\Delta} &
       \seq{\Gamma}{\alpha}{A}
      }
\end{array}
\]    
\caption{Sequent Calculus}
\label{fig:sequent}
\hrule
\end{small}
\end{figure}

The whole sequent calculus is parametrized by a mode theory $\Sigma$,
which is an implicit argument to all of the definitions in
Figure~\ref{fig:sequent}.  The first judgement assigns each
proposition/type $A$ a mode $p$.  Encodings of non-modal logics will
generally only make use of one mode, while modal logics use different
modes to represent different notions of truth, such as the linear and
cartesian categories in the adjoint decomposition of linear
logic~\citep{benton94mixed,bentonwadler96adjoint} and the true/valid/lax
judgements in modal logic~\citep{pfenningdavies}.  The next judgement
assigns each context $\Gamma$ a mode context $\psi$.  Formally, we think
of contexts as ordered: we do not regard $x:A,y:B$ and $y:B,x:A$ as the
same context, though we will have an admissible exchange rule that
passes between derivations in one and the other.

The sequent judgement \seq{\Gamma}{\alpha}{A} relates a context
$\wfctx{\Gamma}{\psi}$ and a type $\wftype{A}{p}$ and context descriptor
\oftp{\psi}{\alpha}{p}, while the substitution relates
$\wfctx{\Gamma}{\psi}$ and $\wfctx{\Delta}{\psi'}$ and
$\oftp{\psi}{\gamma}{\psi'}$. Because $\wfctx{\Gamma}{\psi}$ means that
each variable in $\Gamma$ is in $\psi$, where $x : A_i \in \Gamma$
implies $x : p_i$ in $\psi$ with \wftype{A_i}{p_i}, we think of $\Gamma$
as binding variable names both in $\alpha$ and for use in the
derivation.

\ifthenelse{\boolean{short}}{}{
As discussed in the introduction, a guiding principle is to make the
following rules admissible (see Section~\ref{sec:synprop-long} for
details), which express respect for structural transformations and
structurality-over-structurality:
\[
\begin{array}{c}
\infer[Lem~\ref{lem:respectspr}]{\seq{\Gamma}{\alpha}{A}}
      {\alpha \spr \beta &
       \seq{\Gamma}{\beta}{A}}
\qquad
\infer[Thm~\ref{thm:identity}]{\seq{\Gamma,x:A}{x}{A}}{}
\qquad
\infer[Thm~\ref{thm:cut}]{\seq{\Gamma}{\subst{\beta}{\alpha}{x}}{B}}
    {\seq{\Gamma,x:A}{\beta}{B} &
     \seq{\Gamma}{\alpha}{A}}
\\ \\
\infer[Lem~\ref{lem:weakening}]{\seq{\Gamma,y:A}{\alpha}{C}}
      {\seq{\Gamma}{\alpha}{C}}
\quad
\infer[Lem~\ref{lem:exchange}]{\seq{\Gamma,y:B,x:A}{\alpha}{C}}
      {\seq{\Gamma,x:A,y:B}{\alpha}{C}}
\qquad
\infer[Cor~\ref{cor:contraction}]{\seq{\Gamma,x:A}{\subst \alpha x y}{C}}
      {\seq{\Gamma,x:A,y:A}{\alpha}{C}}
\end{array}
\]
}

We now explain the rules for the sequent calculus; the reader may wish
to refer to the examples in Section~\ref{sec:exampleencodings} in
parallel with this abstract description.  We assume atomic propositions
$P$ are given a specified mode $p$, and state identity as a primitive
rule only for them.  If we had instead stated the identity rule as
\seq{\Gamma,x:P}{x}{P}, then respect for structural transformation would
not be admissible, so we allow a $\beta \spr x$ premise here.  Using a
structural property at a leaf of a derivation is common in e.g. affine
logic, where the derivation of $\beta \spr x$ would use weakening to
forget any additional resources.

Next, we consider the \F{\alpha}{\Delta} type, which ``internalizes''
the context operation $\alpha$ as a type/proposition.  Syntactically, we
view the context $\Delta = x_1:A_1,\ldots,x_n:A_n$ where
\wftype{A_i}{p_i} as binding the variables $x_i:p_i$ in $\alpha$, so for
example \F{\alpha}{x:A,y:B} and \F{\alpha[x \leftrightarrow
    x']}{x':A,y:B} are $\alpha$-equivalent types (in de Bruijn form we
would write \F{\alpha}{A_1,\ldots,A_n} and use indices in $\alpha$).
The type formation rule says that \dsd{F} moves covariantly along a mode
morphism $\alpha$, representing a ``product'' (in a loose sense) of the
types in $\Delta$ structured according to the context descriptor
$\alpha$. A typical binary instance of \dsd{F} is a multiplicative
product ($A \otimes B$ in linear logic), which, given a binary context
descriptor $\odot$ as in the introduction, is written \F{x \odot
  y}{x:A,y:B}.  A typical nullary instance is a unit (1 in linear
logic), written \F{1}{}.  A typical unary instance is the \dsd{F}
connective of adjoint logic, which for a unary context descriptor
constant $\dsd{f} : \dsd{p} \to \dsd{q}$ is written \F{\dsd{f}(x)}{x:A}.
We sometimes write \F{\dsd{f}}{A} in this case, eliding the variable
name, and similarly for a unary \dsd{U}.

The rules for our \dsd{F} connective capture a pattern common to all of
these examples.  The left \FL rule says that \F{\alpha}{\Delta}
``decays'' into $\Delta$, but \emph{structuring the uses of resources in
  $\Delta$ with $\alpha$ by the substitution \subst{\beta}{\alpha}{x}}.
We assume that $\Delta$ is $\alpha$-renamed to avoid collision with
$\Gamma$ (the proof term here would be a ``\dsd{split}'' that binds
variables for each position in $\Delta$).  The placement of $\Delta$ at
the right of the context is arbitrary (because we have
exchange-over-exchange), but we follow the convention that new variables
go on the right to emphasize that $\Gamma$ behaves mostly as in ordinary
cartesian logic.  The right \FR\/ rule says that you must rewrite (using
structural transformations) the context descriptor to have an $\alpha$
at the outside, with a mode substitution $\gamma$ that divides the
existing resources up between the positions in $\Delta$, and then prove
each formula in $\Delta$ using the specified resources.  We leave the
typing of $\gamma$ implicit, though there is officially a requirement
$\oftp{\psi}{\gamma}{\psi'}$ where $\wfctx{\Gamma}{\psi}$ and
$\wfctx{\Delta}{\psi'}$, as required for the second premise to be a
well-formed sequent.  Another way to understand this rule is to begin
with the ``axiomatic \FR'' instance 
$\FR^* :: {\seq{\Delta}{\alpha}{\F{\alpha}{\Delta}}}{}$
which says that there is a map from $\Delta$ to \F{\alpha}{\Delta} along
$\alpha$.  Then, in the same way that a typical right rule for
coproducts builds a precomposition into an ``axiomatic injection'' such
as $\dsd{inl} :: A \vdash A + B$, the \FR\/ rule builds a precomposition
with $\seq{\Gamma}{\gamma}{\Delta}$ and then an application of a
structural rule $\beta \spr \alpha[\gamma]$ into the ``axiomatic''
version, in order to make cut and respect for transformations
admissible.

Next, we turn to $\U{x.\alpha}{\Delta}{A}$.  As a first approximation,
if we ignore the context descriptors and structural properties,
\U{-}{\Delta}{A} behaves like $\Delta \to A$, and the \UL\/ and \UR\/
rules are an annotation of the usual structural/cartesian rules for
implication.  In a formula \U{x.\alpha}{\Delta}{A}, the context
descriptor $\alpha$ has access to the variables from $\Delta$ as well as
an extra variable $x$, whose mode is the same as the \emph{overall mode
  of \U{x.\alpha}{\Delta}{A}}, while the mode of $A$ itself is the mode
of the conclusion of $\alpha$---in terms of typing, \dsd{U} is
contravariant where \dsd{F} is covariant.  It is helpful to think of $x$
as standing for the context that will be used to prove
\U{x.\alpha}{\Delta}{A}.  For example, a typical function type $A \lolli
B$ is represented by \U{x.x \otimes y}{y:A}{B}, which says to extend the
``current context'' $x$ with a resource $y$.  In \UR, the context
descriptor $\beta$ being used to prove the \dsd{U} is substituted
\emph{for $x$} in $\alpha$ (dual to \FL, which substituted $\alpha$ into
$\beta$).  The ``axiomatic'' \UL\/ instance
$\UL^* :: {\seq{\Delta,x:\U{x.\alpha}{\Delta}{A}}{\alpha}{A}}$
says that \U{x.\alpha}{\Delta}{A} together with $\Delta$ has a map to
$A$ along $\alpha$.  (The bound $x$ in $x.\alpha$ subscript is tacitly
renamed to match the name of the assumption in the context, in the same
way that the typing rule for $\lambda x.e : \Pi x:A.B$ requires
coordination between two variables in different scopes).  The full rule
builds in precomposition with \seq{\Gamma}{\gamma}{\Delta},
postcomposition with \seq{\Gamma,z:A}{\beta'}{C}, and precomposition
with $\beta \spr \beta'[\alpha[\gamma]/z]$.

Finally, the rules for substitutions are pointwise.  In examples, we
will write the components of a substitution directly as multiple
premises of \FR\/ and \UL\/, eliding packaging them into a list with the
$\_,\_$ and $\cdot$ rules.

\ifthenelse{\boolean{short}}{}{
One subtle point about the $\FL$ rule is that there are two competing
principles: making the rules ``obviously'' structural-over-structural,
and reducing inessential non-determinism.  Here, we choose the later,
and treat the assumption of \F{\alpha}{\Delta} affinely, removing it
from the context when it is used.  It will turn out that the judgement
nonetheless enjoys contraction-over-contraction
(Corollary~\ref{cor:contraction}), because contraction
for negatives is built into the \UL-rule, and contraction for positives
follows from this and the fact that we can always reconstruct a positive
from what it decays to on the left (c.f. how purely positive formulas
have contraction in linear logic).
}

Additives can be added to this sequent calculus; e.g. a mode $p$ has
sums $\wftype{A_p + B_p}{p}$ if
\[
\begin{array}{c}
\infer{\seq{\Gamma}{\alpha}{A + B}}
      {\seq{\Gamma}{\alpha}{A}}
\quad
\infer{\seq{\Gamma}{\alpha}{A + B}}
      {\seq{\Gamma}{\alpha}{B}}
\quad
\infer{\seq{\Gamma,x:A+B,\Gamma'}{\beta}{C}}
      {\seq{\Gamma,\Gamma',y:A}{\subst \beta y x}{C} &
       \seq{\Gamma,\Gamma',z:B}{\subst \beta z x}{C} 
      }
%% \infer{\wftype{A \& B}{p}}
%%       {\wftype{A}{p} &
%%        \wftype{B}{p}}
%% \qquad
%% \infer{\seq{\Gamma,x:A \& B}{\alpha}{C}}
%%       {\seq{\Gamma,y:A}{\alpha[y/x]}{C}}
%% \quad
%% \infer{\seq{\Gamma}{\alpha}{A + B}}
%%       {\seq{\Gamma}{\alpha}{B}}
\end{array}
\]


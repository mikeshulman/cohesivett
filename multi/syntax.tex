\newcommand\wfsp[4]{\ensuremath{#1 \vdash #2 \spr_{#4} #3}}

\section{Sequent Calculus}

\subsection{Mode Theories}

\begin{figure}
\[
\begin{array}{llll}
\text{modes} & p & & (constants) \\
\text{mode contexts} & \psi & ::= & \cdot \mid \psi,\tptm{x}{p} \\
%% \text{context descriptors} & \alpha,\beta & ::= & x \mid \dsd{c}(\alpha_1,\ldots,\alpha_n) \\
%% \text{substitutions} & \gamma,\delta & ::= & \cdot \mid \gamma,\alpha/x\\
\end{array}
\]

\framebox{Context descriptors/mode morphisms \oftp{\psi}{\alpha}{p}}

\[
\infer{\oftp{\psi}{x}{p}}
      {x:p \in \psi}
\quad
\infer{\oftp{\psi}{\dsd{c}(\alpha_1,\ldots,\alpha_n)}{q}}
      {\dsd{c} : p_1,\ldots,p_n \to q \in \Sigma &
       \oftp{\psi}{\alpha_i}{p_i}
      }
\quad
\infer{\oftp{\psi}{\alpha_i/x_i}{\overline{x_i:q_i}}}
      {\oftp{\psi}{\alpha_i}{q_i}}
\]

\framebox{Structural properties/mode transformations \wfsp{\psi}{\alpha}{\alpha'}{p}}

\[
\begin{array}{c}
\infer{\wfsp{\psi}{\alpha}{\alpha}{p}}{}
\qquad
\infer{\wfsp{\psi}{\alpha_1}{\alpha_3}{p}}
      {\wfsp{\psi}{\alpha_1}{\alpha_2}{p} &
       \wfsp{\psi}{\alpha_2}{\alpha_3}{p} &
      }
\\ \\
\infer{\wfsp{\psi,\psi'}{\subst{\beta}{\alpha}{x}}{\subst{\beta'}{\alpha'}{x}}{q}}
      {\wfsp{\psi,x:p,\psi'}{\beta}{\beta'}{q} &
       \wfsp{\psi,\psi'}{\alpha}{\alpha'}{p}}
\qquad
\infer{\wfsp{\psi}{\alpha}{\alpha'}{p}}
      {\alpha \spr \alpha' \in \Sigma}
\end{array}
\]

\label{fig:2multicategory}
\caption{Syntax for mode theories}
\end{figure}

The first layer of the logic is a type theory whose types we will call
\emph{modes}, and whose terms we will call \emph{context descriptors} or
\emph{mode morphisms}.  We begin with a simple type theory with
variables and constants, as described in
Figure~\ref{fig:2multicategory}.  We assume a fixed signature $\Sigma$
of constants \dsd{c}, and we write $\oftp{\psi}{\alpha}{q}$ for mode
morphism well-formedness and $\oftp{\psi}{\gamma}{\psi'}$ for
substitution well-formedness.  The context $\psi$ here enjoys
(cartesian) structural properties (weakening, exchange, contraction).
Simultaneous substitution into terms is defined as usual.

Next, we need a notation for presenting object-logic structural
properties, which, as described above, will in general be directed
transformations between context descriptors.  For
\oftp{\psi}{\alpha,\alpha'}{p}, we define a judgement
     {\wfsp{\psi}{\alpha}{\alpha'}{p}} representing such
     transformations; it is the least precongruence closed under some
     axioms in the signature $\Sigma$.  For example, to say that a mode
     $p$ with a context monoid $(\odot,1)$ is affine, we would specify
     in $\Sigma$ a structural property \wfsp{x:p}{x}{1}{p}.  Then, using
     the rules in the figure, we can for example derive a transformation
     $(x \odot y) \spr (1 \odot y) \spr y$ that, when applied
     (contravariantly) to a sequent, will allow weakening $y$ to $x
     \odot y$.

In Section~\ref{sec:equalityofderiv}, where we discuss equality of
sequent calculus derivations, we will need an equational theory between
two structural property derivations $s \deq s' ::
\wfsp{\psi}{\alpha}{\alpha'}{q}$.  Because this equational theory does
not influence provability in the sequent calculus, only identity of
proofs, we defer the details.

One choice in the design of this language of context descriptors is how
to handle symmetric structural properties.  One possibility is to
present a desired equation $\alpha = \alpha'$ as an isomorphism, with
axioms $s : \alpha \spr \alpha'$ and $s' : \alpha' \spr \alpha$ (and,
writing $s_1;s_2$ for derivations by the above transitivity rule,
equations $s;s' = 1_{\alpha}$ and $s';s = 1_{\alpha'}$).  While this is
conceptually and technically sufficient, we have found it helpful in
examples to also use a ``strict'' equality of context descriptors which
is implicitly respected by all types and judgements.  This simplifies
the description of some situations, but the difference is important
mainly at the level of identity of derivations rather than
provability---for example, we can make a binary operation $\odot$ into a
strict monoid, rather than adding associator and unitor isomorphisms,
which requires more equations between structural properties.  To support
this, we allow $\Sigma$ to contain axioms for equations $\alpha \deq
\beta$ and define a judgement $\psi \vdash \alpha \deq \beta : p$ as the
least congruence containing these axioms (i.e. add symmetry to the rules
for transformations).  Because the context descriptors and their
equality are defined independently of any subsequent judgements, we take
the liberty of suppressing this equality using $\alpha$ to refer to a
context-descriptor-modulo-\deq---that is, we assume a metatheory with
quotient sets/types, and use meta-level equality for object-level
equality, as recently advocated by \citet{altenkirchkaposiXXpopl}.  For
example, because the judgement \wfsp{\psi}{\alpha}{\beta}{p} is indexed
by equivalence classes of context descriptions, the reflexivity rule
above implicitly means $\alpha \deq \beta$ implies $\alpha \spr \beta$.  

A signature $\Sigma$ consisting of some number of constants $c$,
equations $\alpha \deq \beta$, transformations $\alpha \spr \beta$ (and
equalities of transformations $s \deq s'$) specifies a \emph{mode theory}.

\subsection{Sequent Calculus}

\begin{figure}
\[
\begin{array}{l}
%% \begin{array}{llll}
%% \text{Types} & A & ::= & P \mid \F{\alpha}{\Delta} \mid \U{\alpha}{\Delta}{A} \\
%% \end{array}
%% \\ \\
\framebox{Types $A,B,C$ \quad \wftype{A}{p}}
\qquad
\infer{\wftype{P}{p}}{}
\\ \\
\infer{\wftype{\F{\alpha}{\Delta}}{q}}
      {\oftp{\psi}{\alpha}{q} &
        \wfctx{\Delta}{\psi}}
\qquad
\infer{\wftype{\U{x.\alpha}{\Delta}{A}}{q}}
      {\oftp{\psi,x:q}{\alpha}{p} &
        \wfctx{\Delta}{\psi} &
        \wftype{A}{p}
      }
\\ \\
\framebox{Contexts $\Gamma,\Delta$ \quad \wfctx{\Gamma}{\psi}}
\qquad
\infer{\wfctx{\cdot}{\cdot}}{}
\qquad
\infer{\wfctx{\Gamma,x:A}{\psi,x:p}}
      {\wfctx{\Gamma}{\psi} &
        \wftype{A}{p}}
\\ \\
\framebox{\seq{\Gamma}{\alpha}{A}}
\qquad
\infer[\dsd{v}]{\seq{\Gamma}{\beta}{P}}
      {x:P \in \Gamma & \beta \spr x}
\\ \\
\infer[\FR]{\seq{\Gamma}{\beta}{\F{\alpha}{\Delta}}}
      {%% \modeof{\Gamma} \vdash \gamma : \modeof{\Delta} & 
        \beta \spr \tsubst{\alpha}{\gamma} &
        \seq{\Gamma}{\gamma}{\Delta} 
      }
\quad
\infer[\FL]{\seq{\Gamma,x:\F{\alpha}{\Delta},\Gamma'}{\beta}{C}}
      {\seq{\Gamma,\Gamma',\Delta}{\subst \beta {\alpha}{x}}{C}}
%% \infer{\seq{\Gamma}{\beta}{C}}
%%       {{x}:{\F{\alpha}{\Delta}} \in \Gamma & 
%%         \oftp{\modeof{\Gamma},{x'} : {\modeof{\F{\alpha}{\Delta}}}}{\beta'}{\modeof{C}} &
%%         \beta \deq \tsubst{\beta'}{x/x'} &
%%         \seq{\Gamma,\Delta}{\subst {\beta'} {\alpha}{x'}}{C}}
\\ \\
\infer[\UR]{\seq{\Gamma}{\beta}{\U{x.\alpha}{\Delta}{A}}}
      {\seq{\Gamma,\Delta}{\subst{\alpha}{\beta}{x}}{A}}
\qquad
\infer[\UL]{\seq{\Gamma}{\beta}{C}}
      {\begin{array}{l}
          x:\U{x.\alpha}{\Delta}{A} \in \Gamma \\
          \beta \spr \subst{\beta'}{\tsubst{\alpha}{\gamma}}{z} \\
          \seq{\Gamma}{\gamma}{\Delta} \\
          \seq{\Gamma,\tptm{z}{A}}{\beta'}{C}
       \end{array}
      }
\\ \\
\framebox{\seq{\Gamma}{\gamma}{\Delta}}
\qquad
\infer[\cdot]{\seq{\Gamma}{\cdot}{\cdot}}
      {}
\qquad
\infer[\_,\_]{\seq{\Gamma}{\gamma,\alpha/x}{\Delta,x:A}}
      {\seq{\Gamma}{\gamma}{\Delta} &
       \seq{\Gamma}{\alpha}{A}
      }
\end{array}
\]    
\caption{Sequent Calculus}
\label{fig:sequent}
\hrule
\end{figure}

The whole sequent calculus is parametrized by a mode theory $\Sigma$,
which is an implicit argument to all of the definitions in
Figure~\ref{fig:sequent}.  The judgement assigns each proposition/type
$A$ a mode $p$.  Encodings of non-modal logics will generally only make
use of one mode, while modal logics use different modes to represent
different notions of truth, such as the linear and cartesian categories
in \citet{bentonwadler96adjoint} and the true/valid/lax judgements in
\citet{pfenningdavies}.  The next judgement assigns each context
$\Gamma$ a mode context $\psi$ pointwise.  Formally, we think of
contexts as ordered: we do not regard $x:A,y:B$ and $y:B,x:A$ and the
same context, though we will have an admissible exchange rule that
passes between derivations of one and the other.

The sequent judgement \seq{\Gamma}{\alpha}{A} relates a context
$\wfctx{\Gamma}{\psi}$ and a type $\wftype{A}{p}$ and context descriptor
\oftp{\psi}{\alpha}{p}.  Because $\wfctx{\Gamma}{\psi}$ means that each
variable in $\Gamma$ is in $\psi$, where $x : A_i \in \Gamma$ implies $x
: p_i$ in $\psi$ with \wftype{A_i}{p_i}, we think of $\Gamma$ as binding
variable names both in $\alpha$ and for use in the derivation.

As discussed in the introduction, a guiding principle is to make the
following rules admissible (see Section~\ref{sec:synmeta} for details),
which express respect for object-logic structural properties and
structurality-over-structurality:
\[
\begin{array}{c}
\infer{\seq{\Gamma}{\alpha}{A}}
      {\alpha \spr \beta &
       \seq{\Gamma}{\beta}{A}}
\qquad
\infer{\seq{\Gamma,x:A}{x}{A}}{}
\qquad
\infer{\seq{\Gamma}{\subst{\beta}{\alpha}{x}}{B}}
    {\seq{\Gamma,x:A}{\beta}{B} &
     \seq{\Gamma}{\alpha}{A}}
\\ \\
\infer{\seq{\Gamma,y:A}{\alpha}{C}}
      {\seq{\Gamma}{\alpha}{C}}
\quad
\infer{\seq{\Gamma,y:B,x:A}{\alpha}{C}}
      {\seq{\Gamma,x:A,y:B}{\alpha}{C}}
\qquad
\infer{\seq{\Gamma,x:A}{\subst \alpha x y}{C}}
      {\seq{\Gamma,x:A,y:A}{\alpha}{C}}
\end{array}
\]

We now explain the rules for the sequent calculus; the reader may wish
to refer to the examples in Section~\ref{sec:exampleencodings} in
parallel with this high-level description.

We assume atomic propositions $P$ are given a specified mode $p$, and
state the identity rule only for atoms, to check that the other
instances are admissible.  If we had states the identity rule as
\seq{\Gamma,x:P}{x}{P}, then respect for object-logic structural
properties would not be admissible, so we allow a $\beta \spr x$ premise
here.  Using a structural property at a leaf of a derivation is common
in e.g. affine logic, where the derivation of $\beta \spr x$ would use
weakening to forget any additional resources.  

Next, we consider the \F{\alpha}{\Delta} type.  Syntactically, we view
the context $\Delta = x_1:A_1,\ldots,x_n:A_n$ where \wftype{A_i}{p_i} as
binding the variables $x_i:p_i$ in $\alpha$, so for example
\F{\alpha}{x:A,y:B} and \F{\alpha[x \leftrightarrow x']}{x':A,y:B} are
$\alpha$-equivalent types (in de Bruijn form we would write
\F{\alpha}{A_1,\ldots,A_n} and use indices in $\alpha$).  The type
formation rule says that \dsd{F} moves covariantly along a mode morphism
$\alpha$, representing a combination of the types in $\Delta$ according
to the context descriptor $\alpha$.  A typical binary instance of
\dsd{F} is a multiplicative product ($A \otimes B$ in linear logic),
which, given a binary context descriptor $\odot$ as in the introduction,
is written \F{x \odot y}{x:A,y:B}.  A typical nullary instance is a unit
(1 in linear logic), written \F{1}{}.  A typical unary instance is the
\dsd{F} connective of adjoint logic, which for a unary context
descriptor $\dsd{f} : \dsd{p} \to \dsd{q}$ is written
\F{\dsd{f}(x)}{x:A}.

The rules for our \dsd{F} connective capture the common pattern between
all of these examples.  On the left, \F{\alpha}{\Delta} ``decays'' into
$\Delta$, but \emph{marking the uses of resources in $\Delta$ with
  $\alpha$ by the substitution \subst{\beta}{\alpha}{x}}.  We assume
that $\Delta$ is $\alpha$-renamed to avoid collision with $\Gamma$ (the
proof term here would be a ``\dsd{split}'' that binds variables for each
position in $\Delta$).  On the right, the \FR rule says that you must
rewrite (using structural properties) the context descriptor to have an
$\alpha$ at the outside, with a mode substitution $\gamma$ that divides
the exisitng resources up between the positions in $\Delta$, and then
prove each formula in $\Delta$ using the specified resources.  We leave
the typing of $\gamma$ implicit, though there is officially a
requirement $\oftp{\psi}{\gamma}{\psi'}$ where $\wfctx{\Gamma}{\psi}$
and $\wfctx{\Delta}{\psi'}$, as required for the second premise to be
well-formed.  Another way to understand this rule is to begin with
the ``axiomatic \FR'' 
\[
{\seq{\Delta}{\alpha}{\F{\alpha}{\Delta}}}
\]
which says that there is a map from $\Delta$ to \F{\alpha}{\Delta} along
$\alpha$.  Then, in the same way that a typical injection rule for
coproducts builds a precomposition into an ``axiomatic injection'' such
as $\dsd{inl} :: A \vdash A + B$, the \FR\/ rule builds a precomposition
with $\seq{\Gamma}{\gamma}{\Delta}$ and then a structural rule $\beta
\spr \alpha[\gamma]$ into the ``axiomatic'' version.  


Finally, the rules for substitutions are pointwise.

One subtle point about $\FL$ is that there are two competing principles:
making the rules ``obviously'' structural-over-structural, and reducing
inessential non-determinism.  Here, we choose the later, and treat the
assumption of \F{\alpha}{\Delta} affinely, removing it from the context
when it is used.  It will turn out that the judgement nonetheless enjoys
contraction-over-contraction
(Corollary~\ref{thm:contraction-over-contraction}), because contraction
for negatives is built into the \UL-rule, and contraction for positives
follows from this and the fact that we can always reconstruct a positive
from what it decays to on the left (c.f. how purely positive formulas
have contraction in linear logic).


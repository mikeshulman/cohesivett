
I'm thinking of assoc/unit/comm as equalities, so the non-identity
2-cells are generated by

w : x ⇒ 1
c : x ⇒ x * x

I'll write α for a 1-cell x1*...*xn where all variables are distinct,
and β for a general one.

For this mode theory, there is at most one 2-cell α ⇒ β for any (all
distinct) α and any β --- contract x to xⁿ for whatever xⁿ occurs in β,
which works as long as everything in β is in α, and weaken everything
else.

The standard cartesian sequent calculus for * sits inside adjoint logic
as the following instances of the rules.  The parenthesized premises are
how the 2-cells for the general rules are chosen; these can't fail to
exist.  For convenience in stating the theorem I'm going to allow Γ to
have assumptions that are not marked in α.  However, this mode theory
has admissible strengthening of variables in Γ that are not in α:

If Γ,x:A [α]⊢ B and ¬(α ⇒ x) then Γ [α]⊢ B

so I don't think that's essential.

x:P ∈ Γ    α => x
------------------
Γ [α]⊢ P

(c_α : α ⇒ α * α)
Γ [α]⊢ A
Γ [α]⊢ B
------------
Γ [α]⊢ A * B

x : A*B ∈ Γ
(c_x : α ⇒ (α*w)[x/w])
Γ, y:A, z:B [ α*y*z ]⊢ C
-------------------------------
Γ [α]⊢ C

Then I think the following goes through:

If e : α ⇒ β and
   D : Γ [β]⊢ C then
   E : Γ [α]⊢ C and
   transport e D == E

(where == is some equational theory for derivations like we had in the
1-var case).  I didn't check the equality in full, but the right rule
case seemed plausible using some of the same reasoning that you use to
show that a monoidal category with c and w really has products, plus a
rule like the FR2 equation from the one-var paper.

Separating e and D is handy for making the induction go through.

I'll highlight what we need to know about the mode theory as we go
through.  

== hypothesis ==

e : α ⇒ β

x:P ∈ Γ 
e' : β ⇒ x
--------
Γ [β]⊢ P

Result is immediate by composing e and e' (and transport e D = E' by definition).  

== * right ==

e : α ⇒ β

e' : β ⇒ β1 * β2 
Γ [β1]⊢ A1
Γ [β2]⊢ A2
-------------
Γ [β]⊢ A1 * A2

We have

e;e';fst : α ⇒ β1 
e;e';snd : α ⇒ β2

so by the IH we get 

Γ [α]⊢ A1
Γ [α]⊢ A2

For the equality, I think the equation we'll need for an FR2-like move is 

c_α ; (e;e';fst  *  e;e';snd) = e;e' 

== * left ==

e : α ⇒ β

e' : β ⇒ β'[x/w]  (where ψ,x,w ⊢ β' -- remember the left rule builds in a contraction)
Γ, x:A*B, y:A, z:B [β'[y*z/w]]⊢ C
----------------------------------
Γ, x:A*B [β]⊢ C

This one is a little messy, because in the specialized rule, I'm
insisting (by way of the fact that contraction on x exists) that x is
actually in α, but here we might have done an elimination on an x that
is not in β.  This gives two cases:

(1) ¬(α ⇒ x)

By composition with e and e', this means that 
¬(β ⇒ x) and 
¬(β'[x/w] ⇒ x), 
so by some little inductive lemma, this means w doesn't occur in β'.

Therefore β'[anything/w] = β', and so e;e' : α ⇒ β'[y*z/w].  

Thus, we use the IH to get Γ, x:A*B, y:A,z:B [α]⊢ C, 
and then strengthening gives 
Γ, x:A*B [α]⊢ C

I haven't thought about the equality part here.  

(2) α ⇒ x

Observe that α ⇒ β'[x/w], so by some little inductive lemma, 
α*y*z ⇒ β'[y*z/w]
(contract y and z as many times as w occurs, and cut any x's that were
turned into w's and therefore y*z's).  

Thus, the IH gives

Γ, x:A*B, y:A, z:B [α*y*z]⊢ C

and since x is in α, the specialized rule applies.  

I haven't thought about the equality part here.  


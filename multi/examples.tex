\section{Examples}
\label{sec:exampleencodings}

FIXME introduce F/U abbrevations

In this section, we give some examples of logical connectives that can
be represented by mode theories in this framework, and explain
informally why they have the desired behavior with respect to
provability.  An additional level to these encodings is making the
equational theory of derivations of \seq{\Gamma}{\alpha}{A} match a
desired notion of equality of maps/morphisms, and for this it is often
necessary to add some additional equations between structural properties
$s \deq s' : \alpha \spr \beta$, but we defer discussion about identity
of structural properties and adequacy proofs to
Section~\ref{sec:adequacy}.

\subsection{Magma products}

A mode theory with one mode \dsd{m} and a constant
\[
\begin{array}{c}
\oftp{x : \dsd{m}, y : \dsd{m}}{x \odot y}{\dsd{m}}\\
\end{array}
\]
specifies a completely astructural context (no weakening, exchange,
contraction, associtivity).  If we write $A \odot B$ for \F{x \odot
  y}{x:A,y:B} we \emph{cannot}, for example, derive associativity $A
\odot (B \odot C) \vdash (A \odot B) \odot C$.

To attempt a derivation, we can (without loss of generality) begin by
applying the invertible (Lemma~\ref{lemma:Finv}) \FL\/ rule twice, at
which point no further left rules are possible, so we must try to apply
\FR:

\begin{footnotesize}
\[
\infer[\FL]
{\seq{a:\F{x \odot p}{x:A,p:\F{y \odot z}{y:B,z:C}}}
  {a}
  {\F{q \odot z}{q:\F{x \odot y}{x:A,y:B},z:C}}}
{
  \infer[\FL]
        {\seq{x:A,p:\F{y \odot z}{y:B,z:C}}{x \times p}{{\F{q \odot z}{q:\F{x \odot y}{x:A,y:B},z:C}}}}
        {\infer[\FR]
          {\seq{x:A,y:B,z:C}{x \times (y \odot z)}{{\F{q \odot z}{q:\F{x \odot y}{x:A,y:B},z:C}}}}
          {\begin{array}{l}
              {x \odot (y \odot z)} \spr (q \odot z)[\alpha_1/q,\alpha_2/z] \\
              \seq{x:A,y:B,z:C}{\alpha_1}{\F{x \odot y}{x:A,y:B}}\\
              \seq{x:A,y:B,z:C}{\alpha_2}{C}\\
            \end{array}
        }}
}
\]
\end{footnotesize}

To apply \FR, we need to find a substitution for $\alpha_1/q$ and
$\alpha_2/z$ with a structural property as above.  In the absence of any
equations or transformations, the only possible choice is $x/q, (y \odot
z)/z$, so we need to show
\[
\seq{x:A,y:B,z:C}{x}{A \odot B}
\qquad
\seq{x:A,y:B,z:C}{y \odot z}{C}
\]
This is not possible because the context is not divded correctly.  

\subsection{Ordered Products and Implications}

We extend the above mode theory with a constant $1 : \dsd{m}$ and
equations
\[
\begin{array}{c}
x \odot (y \odot z) \deq (x \odot y) \odot z\\
x \odot 1 \deq x \deq 1 \odot x
\end{array}
\]
making $(\odot,1)$ into a monoid.  This makes the context behave like
ordered logic, which has associativity but none of exchange, weakening,
and contraction---a monoidal but not symmetric monoidal product.

We can complete the above proof of associativity of $\odot$: where ne
needed we need to find a substitution such that ${x \odot (y \odot z)}
\spr (q \odot z)[\alpha_1/q,\alpha_2/z]$, we can now choose $(x \odot
y)/q, z/z)$ because
\[
{x \odot (y \odot z)} \deq {(x \odot y) \odot z} = (q \odot z)[x \odot y/q, z/z]
\]
Thus, the subgoals are
\[
\seq{x:A,y:B,z:C}{x \odot y}{A \odot B}
\qquad
\seq{x:A,y:B,z:C}{z}{C}
\]
The latter is identity-over-identity
(Theorem~\ref{thm:identity-over-identity}), and the former is a further
\FR\/ and then identities:
\[
\infer{\seq{x:A,y:B,z:C}{x \odot y}{\F{x' \odot y'}{x':A,y':B}}}
      { \begin{array}{l}
          x \otimes y \spr (x' \odot y')[x/x',y/y'] \\
          \seq{x:A,y:B,z:C}{x}{A} \\
          \seq{x:A,y:B,z:C}{y}{B} 
        \end{array}
      }
\]
However, we get stuck trying to prove commutativity:
\[
\infer[\FL]{\seq{p:\F{x\odot y}{x:A,y:B}}{p}{\F{z\odot w}{z:B,w:A}}}
      {\infer[\FR]{\seq{x:A,y:B}{x \odot y}{\F{z\odot w}{z:B,w:A}}}
        {\begin{array}{l}
            x \odot y \spr (z \odot w) [\alpha_1/z,\alpha_2/w] \\
            \seq{x:A,y:B}{\alpha_1}{B} \qquad 
            \seq{x:A,y:B}{\alpha_2}{A} 
          \end{array}
      }}
\]
because the only choice is $\alpha_1 = x$ and $\alpha_2 = y$, which
sends the wrong resource to each branch.  

Ordered logic has two different implications, one that adds to the left
of the context, and one that adds to the right; the expected rules are
\[
\begin{array}{l}
\infer{\Gamma \vdash A \rightharpoonup B}
      {\Gamma,A \vdash B}
\qquad
\infer{\Gamma,A \rightharpoonup B,\Delta,\Gamma' \vdash C}
      {\Delta \vdash A &
       \Gamma,B,\Gamma' \vdash C
      }
\\ \\
\infer{\Gamma \vdash A \leftharpoonup B}
      {A,\Gamma \vdash B}
\qquad
\infer{\Gamma,\Delta,A \leftharpoonup B,\Gamma' \vdash C}
      {\Delta \vdash A &
       \Gamma,B,\Gamma' \vdash C
      }
\end{array}
\]
We represent these by 
\[
\begin{array}{l}
A \rightharpoonup B := \U{c.c \odot x}{x:A}{B}\\
A \leftharpoonup B := \U{c.x \odot c}{x:A}{B}
\end{array}
\]
These have the expected right rules, putting $x$ on the left or right of
the context descriptor, by the substitution $\beta/c$ in \UR:
\[
\infer{\seq{\Gamma}{\beta}{\U{c.c \odot x}{x:A}{B}}}
      {\seq{\Gamma,x:A}{\beta \odot x}{B}}
\qquad
\infer{\seq{\Gamma}{\beta}{\U{c.x \odot c}{x:A}{B}}}
      {\seq{\Gamma,x:A}{x \odot \beta}{B}}
\]
The instances of \UL\/ are
\[
\begin{array}{l}
\infer{\seq{\Gamma} {\beta} {C}}
      {\begin{array}{l}
          c:\U{c.c \odot x}{x:A}{B} \in \Gamma \\
          \beta \spr \beta'[c \odot \alpha/z] \\
          \seq{\Gamma}{\alpha}{A} \\
          \seq{\Gamma,z:A}{\beta'}{C}
        \end{array}
      }
\qquad
\infer{\seq{\Gamma} {\beta} {C}}
      {\begin{array}{l}
          c:\U{c.x \odot c}{x:A}{B} \in \Gamma \\
          \beta \spr \beta'[\alpha \odot c/z] \\
          \seq{\Gamma}{\alpha}{A} \\
          \seq{\Gamma,z:A}{\beta'}{C}
       \end{array}
      }
\end{array}
\]
Suppose that $\beta$ is of the form $x_1 \odot \ldots c \ldots \odot
x_n$ for distinct variables $x_i$.  Because the only structural rules
are the associativity and unit equations, the structural rule for the
$\rightharpoonup$ rule can only reassociate $\beta$ as $\beta_1 \odot (c
\odot \alpha) \odot \beta_2$, with $\beta' = \beta_1 \odot z \odot
\beta_2$.  Here $\alpha$ plays the role of $\Delta$ in the ordered logic
rule---the resources used to prove $A$, which occur to the right of the
implication being eliminated.  Reading the substitution backwards, the
resources $\beta'$ used for the continuation are ``$\beta$ with $c
\otimes \alpha$ replaced by the result of the implication,'' as desired.
While $c$ and any variables used in $\alpha$ are still in $\Gamma$,
permission to use them has been removed from $\beta'$---and there is no
way to restore such permissions in this mode theory.  The rule for
$\leftharpoonup$ is the same, but with $\alpha$ on the opposite side of
$c$.

\subsection{Linear products and implication}

Linear logic is ordered logic with exchange, so to model this we add a
commutativity equation
\[
x \otimes y \deq y \otimes x
\]
(and switch notation from $\odot$ to $\otimes$).  For example, we can
derive {\seq{p : A \otimes B}{p}{B \otimes A}}:
\[
\infer[\FL]
      {\seq{p:\F{x\otimes y}{x:A,y:B}}{p}{\F{z\otimes w}{z:B,w:A}}}
      {\infer[\FR]{\seq{x:A,y:B}{x \otimes y}{\F{z\otimes w}{z:B,w:A}}}
        {\begin{array}{l}
            x \otimes y \spr (z \otimes w) [y/z,x/w] \\
            \seq{x:A,y:B}{y}{B} \qquad 
            \seq{x:A,y:B}{x}{A} 
          \end{array}
      }}
\]
where the first premise is exactly $x \otimes y = y \otimes x$

For this mode theory, \U{c.c \odot x}{x:A}{B} and \U{c.x \odot
  c}{x:A}{B} are equal types (because comutativity is an equation, and
types are parametrized by equivalence-classes of context descriptors),
and both represent $A \lolli B$.

%% The linear logic $\multimap$-left rule (where contexts are implicitly
%% treated modulo exchange) is
%% \[
%% \infer{\Gamma,\Delta,A \multimap B \vdash C}
%%       {\Delta \vdash A &
%%        \Gamma,B \vdash C}
%% \]
%% We have
%% \[
%% \infer{\seq{\Gamma} {\beta} {C}}
%%       {c:\U{c.c \otimes x}{x:A}{B} \in \Gamma &
%%        \beta \deq \beta'[c \odot \alpha/z] &
%%        \seq{\Gamma}{\alpha}{A} &
%%        \seq{\Gamma,z:A}{\beta'}{C}
%%       }
%% \]

\subsection{Multi-use variables}

An $n$-use variable (see \citep{reed08namessubstructural} for example)
is like a linear variable, but instead of being used ``exactly once''
(modulo additives), it is used ``exactly $n$ times.''  In the above
work, $0$-use variables were used in an encoding of nominal techniques;
another application of $n$-use variables is static analysis of
functional programs\footnote{Andreas Abel, personal communication}
(e.g. counting how many times a variable occurs to decide whether it
will be efficient to unfold a substitution).

%% n-use functions [Wright, Momigliano]
%% • Other 0-use (“irrelevant”) functions [Pfenning, Ley-Wild]
%% • RLF [Ishtiaq, Pym]
%% • HLF
%% – Designed for statement of metatheorems for Linear LF.
%% – Does n-linear Πs above, and more (e.g. some of BI)
%% – Prototype implementation

The rules in \citet{reed08namessubstructural} for an $n$-linear
functions are
\[
\infer{\Gamma \vdash A \to^n B}
      {{\Gamma, x :^n A} \vdash {B}}
\qquad
\infer{\Gamma + f: A \to^n B + (n \cdot \Delta) \vdash C}
      {\Delta \vdash A &
       {\Gamma, B} \vdash {C}}
\]
where $\Gamma + \Delta$ acts pointwise by $x :^{n} A + x :^{m} A = x
:^{n+m} A$ and $n \cdot \Delta$ act pointwise by $n \cdot x^{m} A = x
:^{nm} A$.

We can model this in the linear mode theory by using context descriptors
that are themselves non-linear:
\[
\begin{array}{c}
x^0 = 1 \\
x^{n+1} = x^n \otimes x \\
A \to^n B := \U{c.c \otimes (x^n)}{x:A}{B} \\
\end{array}
\]

This has the following instances of \UL{}{} and \UR{}: 
\[
\infer{\seq{\Gamma}{\beta}{A \to^n B}}
      {\seq{\Gamma, x:A}{\beta \otimes x^n}{B}}
\qquad
\infer{\seq{\Gamma}{\beta}{C}}
      {\begin{array}{l}
          f : \U{f.f \otimes x^n}{x : A}{B} \in \Gamma \\
          \beta \deq \beta'[f \otimes (\alpha)^n/z] \\
          \seq{\Gamma}{\alpha}{A} \\
          \seq{\Gamma, z:B}{\beta'}{C} 
       \end{array}
      }
\]
That is, if we use resources $\alpha$ to prove $A$, we must have $n$
copies of $\alpha$ in $\beta$.  

We can also consider an $n$-use product 
\[
A^n := \F{\mathord{\otimes}^n(x)}{x:A}
\]
as a positive type, which will decompose $A \to^n B$ as $A^n \lolli B$
(by Theorem~\ref{thm:fusion}).  This has a map \seq{p:A^n}{p}{A \otimes
  \ldots \otimes A} but not a converse \seq{A \otimes \ldots \otimes
  A}{p}{p:A^n}.  For example,
\[
\infer {\seq{p:\F{x \otimes x}{x:A}}{p}{\F{x \otimes y}{x:A,y:A}}}
       {\infer {\seq {x:A}{x \otimes x}{\F{x \otimes y}{x:A,y:A}}}
               {x \otimes x \deq (y \otimes z)[x/y,x/z] &
                \seq{x:A}{x}{A} &
                \seq{x:A}{x}{A}}
       }
\]
the essence of which is the contraction in the substitution
$[x/y,x/z]$.  However,
\[
\infer {\seq{p:\F{y \otimes z}{y:A,z:A}}{p}{\F{x \otimes x}{x:A}}}
       {\infer {\seq {y:A,z:A}{y \otimes z}{\F{x \otimes x}{x:A}}}
               {y \otimes z \deq (x \otimes x)[?/x] &
                \ldots
               }
       }
\]
is not derivable, because there is no substitution into $x \otimes x$
that makes it equal to $y \otimes z$ for distinct $y$ and $z$.
Conceptually, we think of $A^2$ as expressing a notion of identity: it
is a \emph{single} $A$ that can be used twice, which is stronger than
having two potentially different $A$'s.
%% This might be useful for
%% applications of linear logic to imperative or concurrent programming,
%% where there is a notion of identity of memory cells or resources.  

\subsection{Relevant products and implications}

Next, we consider a logic with contraction, which should result in a map
$A \vdash (A \otimes A)$.  We always have the left and right components
of the chain
\[
\begin{array}{llllllll}
A & \cong & \F{x}{x:A}  & \vdash^? & \F{x \otimes x}{x:A} & \vdash & \F{x \otimes y}{x:A, y : A}
\end{array}
\]
(the left isomorphism is just \FL/\FR, while the right map was discussed
in the previous section), so we would like to add a middle map
$\F{x}{x:A} \vdash \F{x \otimes x}{x:A}$.  Since \dsd{F} is covariant on
structural properties (Lemma~\ref{lem:respectspr-types}), it suffices to
add a directed structural rule $\dsd{c} :: x \spr x \otimes x$.  Then we
have $A \vdash A^2 \vdash (A \otimes A)$ but neither of the converses.

%% Given {\seq{\Gamma,x:A,y:A}{\alpha \otimes x \otimes y}{C}} (with $x$
%% and $y$ not occuring in $\alpha$), by contraction over contraction
%% (Corollary~\ref{cor:controver}), we have
%% {\seq{\Gamma,z:A}{\alpha \otimes z \otimes z}{C}}, which is equal to 
%% {\seq{\Gamma,z:A}{\alpha z}{C}}

\subsection{Affine products and implications}

Alternatively, if we extend the linear logic mode theory with a directed
structural rule $\dsd{w} :: x \spr 1$ then we get weakening.  For
example, we can define a projection
\[
\infer[\FL]{\seq{p : A \otimes B}{p}{A}}
           {\infer[Lemma~\ref{lemma:respectsp}]
             {\seq{x:A,y:B}{x \otimes y}{A}}
             {\infer{x \otimes y \spr x}
                    {y \spr 1}
               &
               \seq{x:A,y:B}{x}{A}
             }}
\]

\subsection{Cartesian products and implications}

If we take $(\otimes,1)$ to be a commutative monoid (as in ordered
logic) with both $\dsd{w} :: x \spr 1$ and $\dsd{c} :: x \spr x \otimes
x$, then $A \otimes B$ will behave like a cartesian product (in terms of
provability---recall that we are deferring discussion of equality of
structural properties to Section~\ref{sec:adequacy}), and $\U{c.c
  \otimes x}{x:A}{B}$ like the usual structural $A \to B$.

In this setting, we have both $A \vdash \F{x \times x}{x:A}$ (by
contraction) and $\F{x \times x}{x:A} \vdash A$ (by projection).  It
seems reasonable to make this an isomorphism, rather than just an
interprovability, expressing the idea that in a cartesian setting, a
one-use $A$ is exactly the same as a two-use $A$.  (On the other hand,
we do not want $A \cong (A \times A)$, because a single $A$ is not the
same as two potentially different $A$'s).  To accomplish this, we can
take contraction to be an equation $x \deq x \otimes x$ (idempotence of
$\otimes$) rather than a directed transformation.  This makes $A \cong
A^2 \dashv\vdash A \times A$.  We refer to the mode of theory for an
idempotent commutative monoid with a weakening transformation as an
\emph{idempotent cartesian monoid} and write $(\times,\top)$ for it.

\subsection{BI} Bunched implication~\citep{ohearnpym}
has two context-forming operations $\Gamma,\Gamma'$ and
$\Gamma;\Gamma'$, along with corresponding products and implications.
Both are associative, unitial, and commutative, but $;$ has weakening
and contraction while $,$ does not.  A context is represented by a tree
such as $(x:A, y:B);(z : C, w : D)$ (considered modulo the laws), and
the notation $\Gamma[\Delta]$ is used to refer to a tree with a hole
$\Gamma[-]$ that has $\Delta$ as a subtree at the hole.  In sequent
calculus style, the rules for the product and implication corresponding
to $,$ are
\[
\begin{array}{l}
\infer{\Gamma[A * B] \vdash C}
      {\Gamma[A , B] \vdash C}
\quad
\infer{\Gamma,\Delta \vdash A * B}
      {\Gamma \vdash A &
       \Delta \vdash B}
\quad
\infer{\Gamma \vdash A \magicwand B}
      {\Gamma, A \vdash B}
\quad
\infer{\Gamma[A \magicwand B, \Delta] \vdash C}
      {\Delta \vdash A &
       \Gamma[B] \vdash C}
\end{array}
\]
There are similar rules for a product and implication for $;$ as well as
structural rules of weakening and contraction for it.

We can model BI by a mode \dsd{m} with a commutative monoid $(*,I)$ and
an idempotent cartesian monoid $(\times,\top)$.  
%% \[
%% \begin{array}{l}
%% x  : \dsd{m}, y  : \dsd{m} \vdash x \times y : \dsd{m} \\
%% \cdot \vdash \top : \dsd{m} \\
%% x  : \dsd{m}, y  : \dsd{m} \vdash x * y : \dsd{m} \\
%% \cdot \vdash \dsd{I} : \dsd{m} \\
%% \end{array}
%% \]
%% where both $(\times,\top)$ and $(*,I)$ are commutative monoids, $\times$
%% is idempotent, and $\top$ (but not $I$) is terminal ($x \spr \top$).  
We define the BI products and implications using the monoids:
\[
\begin{array}{ll}
A * B := \F{x * y}{x : A, y : B}  &
A \magicwand B := \U{c.c * x}{x : A}{B} \\
A \times B := \F{x \times y}{x : A, y : B} &
A \to B := \U{c.c \times x}{x : A}{B}\\
\end{array}
\]
A context descriptor such as $(x \times y) * (z \times w)$ captures
the ``bunched'' structure of a BI context, and substitution for a
variable models the hole-filling operation $\Gamma[\Delta]$.  The left
rule for $*$ (and similarly $\times$) acts on a leaf
\[
\infer{\seq{\Gamma,z:A*B,\Gamma'}{\beta}{C}}
      {\seq{\Gamma,\Gamma',x:A,y:B}{\subst{\beta}{x * y}{z}}{C}}
\]
and replaces the leaf where $z$ occurs in the tree $\beta$ with the
correct bunch $x*y$, The left rule for $\magicwand$ (and similarly for
$\to$)
\[
\infer{\seq{\Gamma}{\beta}{C}}
      {
        c : A \magicwand B \in \Gamma &
        \beta \spr \beta'[ c * \alpha / z] & 
        \seq{\Gamma}{\alpha}{A} &
        \seq{\Gamma,z:B}{\beta'}{C} 
      }
\]
isolates a subtree containing the implication $c$ and resources $*$'ed
with it, uses those resources to prove $A$, and then replaces the
subtree with the variable $z$ standing for the result of the
implication.

\subsection{Adjoint decomposition of !}  

Following \citet{benton94mixed,bentonwadler96adjoint}, we decompose the
$!$ exponential of intuitionistic linear logic as the comonad of an
adjunction between ``linear'' and ``cartesian'' categories.  We start
with two modes \dsd{l} (linear) and \dsd{c} (cartesian), along with a
commutative monoid $(\otimes,1)$ on \dsd{l} and an idempotent cartesian
monoid $(\times,\top)$ on \dsd{c}.  Next, we add a context descriptor
from \dsd{c} to \dsd{l}:
\[
x : \dsd{c} \vdash \dsd{f}(x) : \dsd{l}
\]
that we think of as including a cartesian context in a linear context.
This generates types 
\[
\wftype {\F{\dsd{f}(x)}{x : A_{\dsd{c}}}}{\dsd{l}}
\qquad
\wftype {\U{x.\dsd{f}(x)}{\cdot}{A_{\dsd{l}}}}{\dsd{c}}
\]
which are adjoint $\F{\dsd{f}(x)}{x:-} \la
{\U{x.\dsd{f}(x)}{\cdot}{-}}$.  The bijection on hom-sets is defined
using \FL\/ and \FR\/ and their invertibility
(Corollary~\ref{cor:Uinvertibility}, Lemma~\ref{lemma:finvert}):
\[
\infer={\seq{p:\F{\dsd{f}(x)}{x:A}}{p}{B}}
       {\infer={\seq{x:A}{\dsd{f}(x)}{B}}
               {\seq{x:A}{x}{\U{x.\dsd{f}(x)}{\cdot}{B}}}}
\]
The comonad of the adjunction
\F{\dsd{f}(x)}{x:\U{c.\dsd{f}(c)}{\cdot}{A}} is the linear logic $!A$.

In the LNL models and sequent calculus~\citep{benton94mixed}, $F(A
\times B) \cong F(A) \otimes F(B)$ and $F(\top) \cong 1$, which we can
add to the mode theory by equations 
\[
\dsd{f}(x \times y) \deq \dsd{f}(x) \otimes \dsd{f}(y)
\qquad \dsd{f}(\top) \deq 1
\]
These equations then extend to isomorphism using
Theorem~\ref{thm:fusion}.  These equations are used to prove that $! A$
has weakening and contraction (with respect to $\otimes$) and $!A
\otimes !B \vdash !(A \otimes B)$, for example.  Omitting these
equations allows us to describe non-monoidal (or lax monoidal, if we add
only one direction) left adjoints.

%% The $ \dsd{f}(x) \otimes \dsd{f}(y) \spr \dsd{f}(x \times y)$
%% direction is used to prove the purely linear logic entailment $!A
%% \otimes !B \vdash !(A \otimes B)$, for example.

%% For example, for contraction we can begin
%% \[
%% \infer{\seq{p : ! A}{p}{! A \otimes ! A}}
%%       {\infer{\seq{{x:\U{c.\dsd{f}(c)}{\cdot}{A}}}{f(x)}{{! A \otimes ! A}}}
%%              {\begin{array}{l}
%%                  f(x) \spr (x' \otimes y') [\dsd{f}(x) / x' , \dsd{f}(x) / y'] \\
%%                  \seq{x:\U{c.\dsd{f}(c)}{\cdot}{A}}{\dsd{f}(x)}{! A} \\
%%                  \seq{x:\U{c.\dsd{f}(c)}{\cdot}{A}}{\dsd{f}(x)}{! A} 
%%                \end{array}
%%              }}
%% \]
%% and we can derive \seq{x:\U{c.\dsd{f}(c)}{\cdot}{A}}{\dsd{f}(x)}{! A}
%% by \FR\/ (it is of the ``axiomatic'' form
%% $\seq{x:C}{f(x)}{\F{\dsd{f}(x)}{x:C}}$).  The key point is that the first
%% premise, which reduces to
%% \[
%% \dsd{f}(x) \spr \dsd{f}(x) \otimes \dsd{f}(x)
%% \]
%% can be deduced as
%% \[
%% \dsd{f}(x) \deq \dsd{f}(x \times x) \deq \dsd{f}(x) \otimes \dsd{f}(x)
%% \]
%% by contraction for $\times$ \emph{if we add an axiom that \dsd{f}
%%   (strictly) preserves the monoidal product}
%% \[
%% \dsd{f}(x \times y) \deq \dsd{f}(x) \otimes \dsd{f}(y)
%% \]

%% Similarly, to get weakening we take $\dsd{f}(\top) \deq 1$.  

\subsection{Adjoint decomposition of $\Box$}  

The modal S4 \Bx{}{} as in \citet{pfenningdavies} is similar to !.  We
call the two modes \dsd{t}ruth and \dsd{v}alidity and have idempotent
cartesian monoids on both (we write $(\times,\top)$ for the \dsd{t} one
and $(\times_v,\top_v)$ for the \dsd{v} one) along with $x : \dsd{v}
\vdash \dsd{f}(x) : \dsd{t}$.  Here, following the analysis of \Bx{}{}
as a monoidal comonad~\citep{alechina+01categoricals4}, we have only lax
monoid-preservation axioms
\[
\dsd{f}(x) \times \dsd{f}(y) \spr \dsd{f}(x \times_v y) \\
\qquad
\top \spr \dsd{f}(\top_v)
\]
though the difference is only at the level of equality of
derivations.\footnote{Because the context monoids are cartesian
  products, there are always converse maps, e.g.  $\dsd{f}(x \times_v y)
  \spr \dsd{f}(x) \times \dsd{f}(y)$ defined by pairing, projection, and
  congruence.  However, in the equational theory of proofs in
  S4~\citep{pfenningdavies}, there is a section-retraction $(\Box A
  \times \Box B) \rightarrowtail \Box (A \times B) \twoheadrightarrow
  (\Box A \times \Box B)$ but not an isomorphism. If we had equalities
  above, they would generate type isomorphisms $\dsd{F}(A \times_v B)
  \cong \dsd{F}(A) \times \dsd{F}(B)$, and because the right-adjoint
  $\dsd{U}$ preserves products, we would have $\dsd{F} \dsd{U} (A \times
  B) \cong \dsd{F}(U A \times_v U B) \cong (\dsd{FU}(A) \times
  \dsd{FU}(B))$, which does not match the existing theory---though it is
  a reasonable alternative to consider.
%% I think it reduces to p : Box A =?= letbox x = p in letbox y = p in box(fst x, snd y)
}

\subsection{Subexponentials}

Subexponentials~\citep{danos,nigam} extend linear logic with a family of
comonads $!_a A$.  All of the comonads are monoidal ($!_a A \otimes !_a
B \vdash !_a(A \otimes B)$ and $1 \vdash !_a A$), and there is a
preorder $a \le b$ such that $!_b A \vdash !_a A$.  Each $!_a$ is
allowed to have weakening and/or contraction, subject to the constraint
that when $a \le b$, $b$ must be at least as structural as $a$.

We illustrate the encoding on a specific example of the diamond preorder
generated by $\dsd i < \dsd j,\dsd k < \dsd m$.  Following
\citep[Example 4.3]{reed09adjoint}, we identify each subexponential $a$
with a mode, and have an additional mode \dsd{l} for basic linear truth,
all with commutative monoids $(\otimes_a,1_a)$.  We generate context
descriptors \oftp{x : b}{\dsd{ba}(x)}{a} for each $a < b$ (so, \dsd{mk},
\dsd{mj}, \dsd{ji}, \dsd{ki}), with an additional \oftp{x:\dsd
  i}{\dsd{il}(x)}{\dsd l}.  These include each ``higher'' mode into the
immediately ``lower'' ones.  We add an equation $\dsd{ji}(\dsd{mj}(x))
\deq \dsd{ki}(\dsd{mk}(x))$ that the diamond commutes.  Then $!_b A$ is
$\F{b{\dsd l}(x)} {x : \U{x.b\dsd{l}(x)}{\cdot}{A}}$ for the unique
$\oftp{x : b}{b{\dsd l}}{\dsd{l}}$ generated by these constants.  For
example, $!_k$ is the comonad of $\oftp{x :
  \dsd{k}}{\dsd{il}(\dsd{ki}(x))}{\dsd{l}}$.

This mode theory is constructed so that \dsd{l} is final, and when $a
\le b$, we have a morphism \oftp{x:b}{ab(x)}{a} and the morphism
\oftp{x:b}{b\dsd{l}(x)}{\dsd{l}} is equal to
\oftp{x:b}{a\dsd{l}(ba(x))}{\dsd{l}}.  Thus, by
Theorem~\ref{thm:fusion}, we have
\[
!_b A = \dsd{F}_{{b\dsd{l}}}(\dsd{U}_{{b\dsd{l}}} A) \cong \dsd{F}_{{a\dsd{l}}}\dsd{F}_{{ba}} \dsd{U}_{{ba}} \dsd{U}_{{a\dsd{l}}} A
\]
The map $!_b A \vdash !_a A$ can thus be defined as the conunit for the
$\dsd{F}_{{ba}} \dsd{U}_{{ba}}$ comonad in the middle.

We add equations $ba(x \otimes_b y) \deq ba(x) \otimes_a ba(y)$ and
$ba(1_b) \deq 1_a$ making each generator strictly monoidal.  This
ensures that each $!_b$ is monoidal and that $!_b A$ can be weakned or
contracted if $(\otimes_b,1_b)$ has weakening or contraction (and more
generally that \F{ba}{B} can be weakened or contracted for any $B$, not
just $\dsd{U}_{ba}{(A)}$).  Thus, we give a particular subexponential
$a$ weakening or contraction by adding it to $(\times_a,1_a)$.  
%% assuming only monoid axioms (not that one is cartesian), 
%% - !_b A \otimes !_b B \vdash !_b(A \otimes B) uses both directions of ba(x \otimes_b y) <=> ba(x) \otimes_a ba(y)
%% - 1 \vdash !_b(1) uses both directions of ba(1_b) \deq 1_a
%% - contraction uses f(a . b) => f(a) . f(b) direction
%% - weakening uses f(1) => 1 direction

Interestingly, when $a \le b$, it does not seem that we need a condition
that $(\otimes_b,1_b)$ has whatever structural properties
$(\otimes_a,1_a)$ has in order to get that $!_b A$ is at least as
structural as $!_a A$.  As argued above $!_b A$ factors into the form
$\dsd{F}_{al}(C)$, which has whatever structural properties mode $a$ has.

%% Following \citep[Example 4.3]{reed09adjoint}, we identify each
%% subexponential $i$ with a mode.  We put a commutative monoid
%% $(\otimes_i,1_i)$ on each mode, and an additional mode \dsd{l} with
%% $(\otimes,1)$ for non-modal truth.  We assume that the relation ($i \le
%% j \cup (\forall k, \dsd{l} \le k)$ is the reflexive, transitive closure
%% of a simple graph, and add a context descriptor $\oftp{x :
%%   \dsd{j}}{\dsd{ji}(x)}{i}$ for each edge of this graph, with equations
%% $\alpha \deq \beta$ when $\alpha$ and $\beta$ both correspond to paths
%% between the same two nodes in the graph.  Then the subexponential $!_i
%% A$ is defined to be $\F{\alpha} {x : \U{x.\alpha}{\cdot}{A}}$ for the
%% $\oftp{x : i}{\alpha}{\dsd{l}}$.

\subsection{Monads}

When we have two modes \dsd{t} and \dsd{p} with a context descriptor
\oftp{x:\dsd{t}}{\dsd{g}(x)}{\dsd{p}}, the type $\Dia{\dsd{g}}{A} :=
\U{c.\dsd{g}(c)}{\cdot}{\F{\dsd{g}(x)}{x:A}}$ will be a monad.  It does
not automatically have a tensorial strength: if we have a commutative
monoid $(\otimes_\dsd{t},1_{\dsd t})$ and try to derive
\[
\infer[\UR]
      {\seq{x : A, y : \Dia{\dsd{g}}{B}}{x \otimes_{\dsd t} y}{\Dia{\dsd{g}}{(A \otimes_{\dsd t} B)}}}
      {\infer[\UL]
        {\seq{x : A, y : \Dia{\dsd{g}}{B}}{\dsd{g}(x \otimes_{\dsd t} y)}{\F{\dsd{g}}{A \otimes_{\dsd t} B}}}
        {\dsd{g}(x \otimes_{\dsd t} y) \spr \subst{\beta'}{\dsd{g}(y)}{z} &
          \seq{x:A,y : \Dia{\dsd{g}}{B},z:\F{\dsd{g}}{B}}{\beta'}{\F{\dsd{g}}{A \otimes_{\dsd t} B}}
        }}
\]
we are stuck, because there is no way to rewrite $\dsd{g}(x
\otimes_{\dsd t} y)$ as a term containing $\dsd{g}(y)$.  If
$(\otimes_t,1_t)$ is affine, then we can weaken away $x$ and take
$\beta' = z$---the \UL\/ rule thus captures the context-clearing of the
\Dia{}{}-left rule in \citep{pfenningdavies}---but then in the
right-hand premise we will only have access to $z$, not $x$, and cannot
complete the derivation.  

\newcommand\ttp[2]{#1 \otimes_{\dsd {tp}} #2}
\newcommand\tvp[2]{#1 \otimes_{\dsd {vp}} #2}

One way to add the strength is to add an asymmtric product of a
\dsd{t}-mode and \dsd{p}-mode context:
\[
\begin{array}{ll}
\oftp{x : \dsd{t}, y : \dsd{p}}{\ttp x y}{\dsd{p}}
& \dsd{g}(x \otimes_{\dsd t} y) \deq \ttp x {\dsd{g}(y)}\\\\
\ttp {x \otimes_{\dsd t} y} z \deq \ttp x {\ttp y z}
& \ttp {\dsd{1}} y \deq y
\end{array}
\]
The equations make this into a monoid action of the \dsd{t}-contexts on
the \dsd{p}-contexts, and allow for ``isolating'' any one $x_i$ in
$\dsd{g}(x_1 \otimes_{\dsd t} \ldots \otimes_{\dsd t} x_n)$ as the
designated variable under a \dsd{g}.  Using this (and switching notation
from \Dia{\dsd{g}}{A} to the \Crc{\dsd{g}}{A} that is typically used for
strong monads), we can prove
\begin{footnotesize}
\[
\D = \infer[\FL]
                {\seq{x:A,y : \Crc{\dsd{g}}{B},z:\F{\dsd{g}}{B}}{\ttp x z}{\F{\dsd{g}}{A \otimes_{\dsd t} B}}}
                {\infer[\UL]
                       {\seq{x:A,y : \Crc{\dsd{g}}{B},z':B}{\ttp{x}{\dsd{g}(z')}}{\F{\dsd{g}}{A \otimes_{\dsd t} B}}}
                       {{\ttp{x}{\dsd{g}(z')}} \spr \dsd{g}(x \otimes_{\dsd t} z') &
                         \infer[\FR]{\seq{x:A,y : \Crc{\dsd{g}}{B},z':B}{x \otimes_{\dsd t} z'}{{A \otimes_{\dsd t} B}}}{}
                       }
        }
\]
\[
\infer[\UR]
      {\seq{x : A, y : \Crc{\dsd{g}}{B}}{x \otimes_{\dsd t} y}{\Crc{\dsd{g}}{(A \otimes_{\dsd t} B)}}}
      {\infer[\UL]
        {\seq{x : A, y : \Crc{\dsd{g}}{B}}{\dsd{g}(x \otimes_{\dsd t} y)}{\F{\dsd{g}}{A \otimes_{\dsd t} B}}}
        {\dsd{g}(x \otimes_{\dsd t} y) \spr \subst{(\ttp x z)}{\dsd{g}(y)}{z} &
          \D}}
\]
\end{footnotesize}

An advantage of this description is that it scales to the
``$\Box$-strong
$\Diamond$''~\citep{pfenningdavies,alechina+01categoricals4}, which has
a strength only for boxed formulas ($\Bx{} A \otimes \Dia{} B \vdash
\Dia{}(\Bx{} A \otimes B)$).  We use 3 modes \dsd{v},\dsd{t},\dsd{p} and
represent the $\Box$ as the comonad of a context descriptor
\oftp{x:\dsd{v}}{\dsd{f}(x)}{\dsd{t}} (with idempotent cartesian monoids
on \dsd{v} and \dsd{t} and \dsd{f} laxly monoidal as above), and the
$\Diamond$ as the monad of a \oftp{x:\dsd{t}}{\dsd{g}(x)}{\dsd{p}}.  We
have a mixed-mode product between \dsd{v} and \dsd{p}
\[
\begin{array}{ll}
\oftp{x : \dsd{v}, y : \dsd{p}}{\tvp x y}{\dsd{p}}
& \dsd{g}(\dsd{f}(x) \times_{\dsd t} y) \deq \tvp x {\dsd{g}(y)}\\
\ttp {(x \times_{\dsd v} y)} z \deq \ttp x {\tvp y z}
& \tvp {\dsd{1}} y \deq y
\end{array}
\]
Thus, if we have a \dsd{t}-context 
\[
\dsd{g}(\dsd{f}(x_1) \times_{\dsd t} \dsd{f}(x_2) \times_{\dsd t} \ldots y_1 \times_{\dsd t} \ldots )
\]
(representing a context with valid/boxed $x$'s and true $y$'s) we can
rewrite it to
\[
\tvp {(x_1 \times_{\dsd v} \ldots \times_{\dsd v} x_n)} \dsd{g}(y)
\]
by monoidalness of \dsd{f}, weakening away the other $y$'s, and the
isolation equation.  Then we can apply \UL\/ to $y_i$, as in the
$\Diamond$-left rule~\citep{pfenningdavies}.  

FIXME: probably should show, not tell

\subsection{Spatial Type Theory}



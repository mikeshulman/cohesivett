\section{Examples}
\label{sec:exampleencodings}

In this section, we give some examples of logical connectives that can
be represented by a mode theory in this framework, and explain
informally why they have the desired behavior.  We discuss adequacy
proofs more formally in Section~\ref{sec:adequacy}.  

\subsection{Magma products}

A mode theory with one mode \dsd{m} and a constant
\[
\begin{array}{c}
\oftp{x : \dsd{m}, y : \dsd{m}}{x \odot y}{\dsd{m}}\\
\end{array}
\]
(often called a \emph{magma}) specifies a completely astructural context
(no weakening, exchange, contraction, associtivity).  If we write $A \odot
B$ for \F{x \odot y}{x:A,y:B}, the type ``internalizing'' the context
operation,  we \emph{cannot}, for example, derive associativity
$A \odot (B \odot C) \vdash (A \odot B) \odot C$.  

To attempt a derivation, we can (without loss of generality) begin by
applying the invertible (Lemma~\ref{lemma:Finv}) \FL\/ rule twice, at
which point no further left rules are possible, so we must try to apply
\FR:

\begin{footnotesize}
\[
\infer[\FL]
{\seq{a:\F{x \odot p}{x:A,p:\F{y \odot z}{y:B,z:C}}}
  {a}
  {\F{q \odot z}{q:\F{x \odot y}{x:A,y:B},z:C}}}
{
  \infer[\FL]
        {\seq{x:A,p:\F{y \odot z}{y:B,z:C}}{x \times p}{{\F{q \odot z}{q:\F{x \odot y}{x:A,y:B},z:C}}}}
        {\infer[\FR]
          {\seq{x:A,y:B,z:C}{x \times (y \odot z)}{{\F{q \odot z}{q:\F{x \odot y}{x:A,y:B},z:C}}}}
          {\begin{array}{l}
              {x \odot (y \odot z)} \spr (q \odot z)[\alpha_1/q,\alpha_2/z] \\
              \seq{x:A,y:B,z:C}{\alpha_1}{\F{x \odot y}{x:A,y:B}}\\
              \seq{x:A,y:B,z:C}{\alpha_2}{C}\\
            \end{array}
        }}
}
\]
\end{footnotesize}

To apply \FR, we need to find a substitution for $\alpha_1/q$ and
$\alpha_2/z$ with a structural property as above.  In the absence of any
equations or transformations, the only possible choice is $x/q, (y \odot
z)/z$, so we need to show
\[
\seq{x:A,y:B,z:C}{x}{A \odot B}
\qquad
\seq{x:A,y:B,z:C}{y \odot z}{C}
\]
This is not possible because the context is not divded correctly.  

\subsection{Ordered Products and Implications}

Next, we extend the above mode theory with a constant $1 : \dsd{m}$ and
equations
\[
\begin{array}{c}
x \odot (y \odot z) \deq (x \odot y) \odot z\\
x \odot 1 \deq x \deq 1 \odot x
\end{array}
\]
making $(\odot,1)$ into a monoid.  Using this, we can define connectives
that behave like ordered logic, which has associativity but none of
exchange, weakening, and contraction---a monoidal but not symmetric
monoidal product.  

In ordered logic, we can complete the above proof of associativity of
$\odot$: where ne needed we need to find a substitution such that ${x \odot
  (y \odot z)} \spr (q \odot z)[\alpha_1/q,\alpha_2/z]$, we can now
choose $(x \odot y)/q, z/z)$ because
\[
{x \odot (y \odot z)} = {(x \odot y) \odot z} = (q \odot z)[x \odot y/q, z/z]
\]
Thus, the subgoals are
\[
\seq{x:A,y:B,z:C}{x \odot y}{A \odot B}
\qquad
\seq{x:A,y:B,z:C}{z}{C}
\]
The latter is identity-over-identity
(Theorem~\ref{thm:identity-over-identity}), and the former is a further
\FR\/ and then identities:
\[
\infer{\seq{x:A,y:B,z:C}{x \odot y}{\F{x' \odot y'}{x':A,y':B}}}
      { \begin{array}{l}
          x \otimes y \spr (x' \odot y')[x/x',y/y'] \\
          \seq{x:A,y:B,z:C}{x}{A} \\
          \seq{x:A,y:B,z:C}{y}{B} 
        \end{array}
      }
\]
However, we get stuck trying to prove commutativity:
\[
\infer[\FL]{\seq{p:\F{x\odot y}{x:A,y:B}}{p}{\F{z\odot w}{z:B,w:A}}}
      {\infer[\FR]{\seq{x:A,y:B}{x \odot y}{\F{z\odot w}{z:B,w:A}}}
        {\begin{array}{l}
            x \odot y \spr (z \odot w) [\alpha_1/z,\alpha_2/w] \\
            \seq{x:A,y:B}{\alpha_1}{B} \qquad 
            \seq{x:A,y:B}{\alpha_2}{A} 
          \end{array}
      }}
\]
because the only choice is $\alpha_1 = x$ and $\alpha_2 = y$, which
sends the wrong resource to each branch.  

Ordered logic has two different implications, one that adds to the left
of the context, and one that adds to the right; the expected rules are
\[
\begin{array}{l}
\infer{\Gamma \vdash A \rightharpoonup B}
      {\Gamma,A \vdash B}
\qquad
\infer{\Gamma,A \rightharpoonup B,\Delta,\Gamma' \vdash C}
      {\Delta \vdash A &
       \Gamma,B,\Gamma' \vdash C
      }
\\ \\
\infer{\Gamma \vdash A \leftharpoonup B}
      {A,\Gamma \vdash B}
\qquad
\infer{\Gamma,\Delta,A \leftharpoonup B,\Gamma' \vdash C}
      {\Delta \vdash A &
       \Gamma,B,\Gamma' \vdash C
      }
\end{array}
\]

We represent these by 
\[
\begin{array}{l}
A \rightharpoonup B := \U{c.c \odot x}{x:A}{B}\\
A \leftharpoonup B := \U{c.x \odot c}{x:A}{B}
\end{array}
\]
These have the expected right rules, putting $x$ on the left or right of
the context descriptor, by the substitution $\beta/c$ in \UR:
\[
\infer{\seq{\Gamma}{\beta}{\U{c.c \odot x}{x:A}{B}}}
      {\seq{\Gamma,x:A}{\beta \odot x}{B}}
\qquad
\infer{\seq{\Gamma}{\beta}{\U{c.x \odot c}{x:A}{B}}}
      {\seq{\Gamma,x:A}{x \odot \beta}{B}}
\]
The instances of \UL\/ are
\[
\begin{array}{l}
\infer{\seq{\Gamma} {\beta} {C}}
      {\begin{array}{l}
          c:\U{c.c \odot x}{x:A}{B} \in \Gamma \\
          \beta \spr \beta'[c \odot \alpha/z] \\
          \seq{\Gamma}{\alpha}{A} \\
          \seq{\Gamma,z:A}{\beta'}{C}
        \end{array}
      }
\qquad
\infer{\seq{\Gamma} {\beta} {C}}
      {\begin{array}{l}
          c:\U{c.x \odot c}{x:A}{B} \in \Gamma \\
          \beta \spr \beta'[\alpha \odot c/z] \\
          \seq{\Gamma}{\alpha}{A} \\
          \seq{\Gamma,z:A}{\beta'}{C}
       \end{array}
      }
\end{array}
\]
Suppose that $\beta$ is of the form $x_1 \odot \ldots c \ldots \odot
x_n$ for distinct variables $x_i$.  Because the only structural rules
are the associativity and unit equations, the structural rule for the
$\rightharpoonup$ rule can only reassociate $\beta$ as $\beta_1 \odot (c
\odot \alpha) \odot \beta_2$, with $\beta' = \beta_1 \odot z \odot
\beta_2$.  Here $\alpha$ plays the role of $\Delta$ in the ordered logic
rule---the resources used to prove $A$, which occur to the right of the
implication being eliminated.  Reading the substitution backwards, the
resources $\beta'$ used for the continuation are ``$\beta$ with $c
\otimes \alpha$ replaced by the result of the implication,'' as desired.
While $c$ and any variables used in $\alpha$ are still in $\Gamma$,
permission to use them has been removed from $\beta'$---and there is no
way to restore such permissions in this mode theory.  The rule for
$\leftharpoonup$ is the same, but with $\alpha$ on the opposite side of $c$.

\subsection{Linear products and implication}

To add exchange, we extend the mode theory with the equation
\[
x \otimes y \deq y \otimes x
\]
(and switich notation from $\odot$ to $\otimes$).  For example, we can
derive {\seq{p : A \otimes B}{p}{B \otimes A}}:
\[
\infer[\FL]
      {\seq{p:\F{x\otimes y}{x:A,y:B}}{p}{\F{z\otimes w}{z:B,w:A}}}
      {\infer[\FR]{\seq{x:A,y:B}{x \otimes y}{\F{z\otimes w}{z:B,w:A}}}
        {\begin{array}{l}
            x \otimes y \spr (z \otimes w) [y/z,x/w] \\
            \seq{x:A,y:B}{y}{B} \qquad 
            \seq{x:A,y:B}{x}{A} 
          \end{array}
      }}
\]
where the first premise is exactly $x \otimes y = y \otimes x$

For this mode theory, \U{c.c \odot x}{x:A}{B} 
and \U{c.x \odot c}{x:A}{B} are equal types (because we add comutativity as an
equation, and types are parametrized by equivalence-classes of context
descriptors), and both represent $A \lolli B$.

%% The linear logic $\multimap$-left rule (where contexts are implicitly
%% treated modulo exchange) is
%% \[
%% \infer{\Gamma,\Delta,A \multimap B \vdash C}
%%       {\Delta \vdash A &
%%        \Gamma,B \vdash C}
%% \]
%% We have
%% \[
%% \infer{\seq{\Gamma} {\beta} {C}}
%%       {c:\U{c.c \otimes x}{x:A}{B} \in \Gamma &
%%        \beta \deq \beta'[c \odot \alpha/z] &
%%        \seq{\Gamma}{\alpha}{A} &
%%        \seq{\Gamma,z:A}{\beta'}{C}
%%       }
%% \]

\subsection{Multi-use variables}

An $n$-use variable (see \citep{reed08namessubstructural} and the
references cited there) is like a linear variable, but instead of being
used ``exactly once'' (modulo additives), it is used ``exactly $n$
times.''  In the above work, $0$-use variables were used in an encoding
of nominal techniques; another application is static analysis of
functional programs\footnote{Andreas Abel, personal communication}
(e.g. counting how many times a variable occurs to decide whether it
will be efficient to unfold a substitution).

%% n-use functions [Wright, Momigliano]
%% • Other 0-use (“irrelevant”) functions [Pfenning, Ley-Wild]
%% • RLF [Ishtiaq, Pym]
%% • HLF
%% – Designed for statement of metatheorems for Linear LF.
%% – Does n-linear Πs above, and more (e.g. some of BI)
%% – Prototype implementation

Rules for an $n$-linear functions are
\[
\infer{\Gamma \vdash A \to^n B}
      {{\Gamma, x :^n A} \vdash {B}}
\qquad
\infer{\Gamma + f: A \to^n B + (n \cdot \Delta) \vdash C}
      {\Delta \vdash A &
       {\Gamma, B} \vdash {C}}
\]
where $\Gamma + \Delta$ acts pointwise by $x :^{n} A + x :^{m} A = x
:^{n+m} A$ and $n \cdot \Delta$ act pointwise by $n \cdot x^{m} A = x
:^{nm} A$.

We can model this without changing the linear logic mode theory by using
context descriptors that are themselves non-linear:
\[
\begin{array}{c}
x^0 = 1 \\
x^{n+1} = x^n \otimes x \\
A \to^n B := \U{c.c \otimes (x^n)}{x:A}{B} \\
\end{array}
\]

This has the following instances of \UL{}{} and \UR{}: 
\[
\infer{\seq{\Gamma}{\beta}{A \to^n B}}
      {\seq{\Gamma, x:A}{\beta \otimes x^n}{B}}
\qquad
\infer{\seq{\Gamma}{\beta}{C}}
      {\begin{array}{l}
          f : \U{f.f \otimes x^n}{x : A}{B} \in \Gamma \\
          \beta \deq \beta'[f \otimes (\alpha)^n/z] \\
          \seq{\Gamma}{\alpha}{A} \\
          \seq{\Gamma, z:B}{\beta'}{C} 
       \end{array}
      }
\]
That is, if we use resources $\alpha$ to prove $A$, we must have $n$
copies of $\alpha$ in $\beta$.  

We can also consider an $n$-use product 
\[
A^n := \F{\mathord{\otimes}^n(x)}{x:A}
\]
as a positive type, which will decompose $A \to^n B$ as $A^n \lolli B$
(by Theorem~\ref{thm:fusion}).
This satisfies 
\[
\seq{p:A^n}{p}{A \otimes \ldots \otimes A}
\]
but not
\[
\seq{A \otimes \ldots \otimes A}{p}{p:A^n}
\]
For example,
\[
\infer {\seq{p:\F{x \otimes x}{x:A}}{p}{\F{x \otimes y}{x:A,y:A}}}
       {\infer {\seq {x:A}{x \otimes x}{\F{x \otimes y}{x:A,y:A}}}
               {x \otimes x \deq (y \otimes z)[x/y,x/z] &
                \seq{x:A}{x}{A} &
                \seq{x:A}{x}{A}}
       }
\]
but
\[
\infer {\seq{p:\F{y \otimes z}{y:A,z:A}}{p}{\F{x \otimes x}{x:A}}}
       {\infer {\seq {y:A,z:A}{y \otimes z}{\F{x \otimes x}{x:A}}}
               {y \otimes z \deq (x \otimes x)[?/x] &
                \ldots
               }
       }
\]
is not derivable, because there is no substitution into $x \otimes x$
that makes it equal to $y \otimes z$ for distinct $y$ and $z$.
Conceptually, we think of $A^2$ as expressing a notion of identity: it
is a \emph{single} $A$ that can be used twice, which is stronger than
having two potentially different $A$'s.  This might be useful for
applications of linear logic to imperative or concurrent programming,
where there is a notion of identity of memory cells or resources.  

\subsection{Relevant products and implications}

If we extend the linear logic mode theory by making $\otimes$
idempotent, in the sense that
\[
x \deq x \otimes x 
\]
then we get relevant logic---exchange and contraction but not
weakening---because permission to use $x$ once is equal to permission to
use it as many times as desired.

Given {\seq{\Gamma,x:A,y:A}{\alpha \otimes x \otimes y}{C}} (with $x$
and $y$ not occuring in $\alpha$), by contraction over contraction
(Corollary~\ref{cor:controver}), we have
{\seq{\Gamma,z:A}{\alpha \otimes z \otimes z}{C}}, which is equal to 
{\seq{\Gamma,z:A}{\alpha z}{C}}

\subsection{Affine products and implications}

If we extend the linear logic mode theory with our first directed
structural rule $\dsd{w} :: x \spr 1$ then we get affine products and
implications---exchange and weakening but not contraction.  For example,
we can define a projection
\[
\infer[\FL]{\seq{p : A \otimes B}{p}{A}}
           {\infer[Lemma~\ref{lemma:respectsp}]
             {\seq{x:A,y:B}{x \otimes y}{A}}
             {\infer{x \otimes y \spr x}
                    {y \spr 1}
               &
               \seq{x:A,y:B}{x}{A}
             }}
\]
To have the right equational theory of proofs, we also will need some
equations of structural properties governing \dsd{w}, as discussed in
Section~\ref{sec:equational}.

\subsection{Cartesian products and implications}

Finally, if we put all of the above ingredients (associativity, unit,
commutativity, idempotence equations and a weakening transformation)
together, we can make $\otimes$ into an ordinary cartiesian product,
\F{x \otimes y}{x:A,y:B} a cartesian product $A \times B$ and \U{c.c
  \otimes x}{x:A}{B} a structural implication $A \to B$.

\subsection{BI} Bunched implication~\citep{ohearnpym}
has two context-forming operations $\Gamma,\Gamma'$ and
$\Gamma;\Gamma'$, along with corresponding products and implications.
Both are associative, unitial, and commutative, but $;$ has weakening
and contraction while $,$ does not.  A context is represented by a tree
such as $(x:A, y:B);(z : C, w : D)$ (considered modulo the laws), and
the notation $\Gamma[\Delta]$ is used to refer to a tree with a hole
$\Gamma[-]$ that has $\Delta$ as a subtree at the hole.  In sequent
calculus style, the rules for the product and implication corresponding
to $,$ are
\[
\begin{array}{l}
\infer{\Gamma[A * B] \vdash C}
      {\Gamma[A , B] \vdash C}
\qquad
\infer{\Gamma,\Delta \vdash A * B}
      {\Gamma \vdash A &
       \Delta \vdash B}
\\ \\
\infer{\Gamma \vdash A \magicwand B}
      {\Gamma, A \vdash B}
\qquad
\infer{\Gamma[A \magicwand B, \Delta] \vdash C}
      {\Delta \vdash A &
       \Gamma[B] \vdash C}
\end{array}
\]
There are similar rules for a product and implication for $;$ as well as
structural rules of weakening and contraction for it.

We can model BI by a mode with two monoids
\[
\begin{array}{l}
x  : \dsd{m}, y  : \dsd{m} \vdash x \times y : \dsd{m} \\
\cdot \vdash \top : \dsd{m} \\
x  : \dsd{m}, y  : \dsd{m} \vdash x * y : \dsd{m} \\
\cdot \vdash \dsd{I} : \dsd{m} \\
\end{array}
\]
where both $(\times,\top)$ and $(*,I)$ are commutative monoids, $\times$
is idempotent, and $\top$ (but not $I$) is terminal ($x \spr \top$).  We
define the BI products and implications using the monoids:
\[
\begin{array}{l}
A * B := \F{x * y}{x : A, y : B} \\
A \magicwand B := \U{c.c * x}{x : A}{B} \\
A \times B := \F{x \times y}{x : A, y : B}\\
A \to B := \U{c.c \times x}{x : A}{B}\\
\end{array}
\]
A context descriptor such as $(x \times y) * (z \times w)$ captures
the ``bunched'' structure of a BI context, and substitution for a
variable models the hole-filling operation $\Gamma[\Delta]$.  The left
rule for $*$ (and similarly $\times$) acts on a leaf
\[
\infer{\seq{\Gamma,z:A*B,\Gamma'}{\beta}{C}}
      {\seq{\Gamma,\Gamma',x:A,y:B}{\subst{\beta}{x * y}{z}}{C}}
\]
and replaces the leaf where $z$ occurs in the tree $\beta$ with the
correct bunch $x*y$, The left rule for $\magicwand$ (and similarly for
$\to$)
\[
\infer{\seq{\Gamma}{\beta}{C}}
      {
        c : A \magicwand B \in \Gamma &
        \beta \spr \beta'[ c * \alpha / z] & 
        \seq{\Gamma}{\alpha}{A} &
        \seq{\Gamma,z:B}{\beta'}{C} 
      }
\]
isolates a subtree containing the implication $c$ and resources $*$'ed
with it, uses those resources to prove $A$, and then replaces the
subtree with the variable $z$ standing for the result of the
implication.

\subsection{Adjoint decomposition of !}  

FIXME: make up a term for cartesian-product-generating-monoid

Following \citet{bentonwadler96adjoint}, we decompose the ! exponential
of intuitionistic linear logic as the comonad of an adjunction between
``linear'' and ``cartesian'' categories.  We start with two modes
\dsd{l} (linear) and \dsd{c} (cartesian), along with a commutative
monoid $(\otimes,1)$ on \dsd{l} and a
cartesian-product-generating-monoid $(\times,\top)$ on \dsd{c}.  Next,
we add a context descriptor from \dsd{c} to \dsd{l}:
\[
x : \dsd{c} \vdash \dsd{b}(x) : \dsd{l}
\]
that we think of as including a cartesian context in a linear context.
This creates types 
\[
\wftype {\F{\dsd{b}(x)}{x : A_{\dsd{c}}}}{\dsd{l}}
\qquad
\wftype {\U{x.\dsd{b}(x)}{\cdot}{A_{\dsd{l}}}}{\dsd{c}}
\]
which are adjoint $\F{\dsd{b}(x)}{x:-} \la
{\U{x.\dsd{b}(x)}{\cdot}{-}}$.  The bijection on hom-sets is defined
using \FL\/ and \FR\/ and their invertibility
(Corollary~\ref{cor:Uinvertibility}, Lemma~\ref{lemma:finvert}):
\[
\infer={\seq{p:\F{\dsd{b}(x)}{x:A}}{p}{B}}
       {\infer={\seq{x:A}{\dsd{b}(x)}{B}}
               {\seq{x:A}{x}{\U{x.\dsd{b}(x)}{\cdot}{B}}}}
\]

The comonad of the adjunction
\[
! A := \F{\dsd{b}(x)}{x:\U{c.\dsd{b}(c)}{\cdot}{A}}
\]
both takes and produces a linear proposition, which should inherit
weakening and contraction from the cartesian product in the other mode.
For example, for contraction we can begin
\[
\infer{\seq{p : ! A}{p}{! A \otimes ! A}}
      {\infer{\seq{{x:\U{c.\dsd{b}(c)}{\cdot}{A}}}{f(x)}{{! A \otimes ! A}}}
             {\begin{array}{l}
                 f(x) \spr (x' \otimes y') [\dsd{b}(x) / x' , \dsd{b}(x) / y'] \\
                 \seq{x:\U{c.\dsd{b}(c)}{\cdot}{A}}{\dsd{b}(x)}{! A} \\
                 \seq{x:\U{c.\dsd{b}(c)}{\cdot}{A}}{\dsd{b}(x)}{! A} 
               \end{array}
             }}
\]
and we can derive \seq{x:\U{c.\dsd{b}(c)}{\cdot}{A}}{\dsd{b}(x)}{! A}
by \FR\/ (it is of the ``axiomatic'' form
$\seq{x:C}{f(x)}{\F{\dsd{b}(x)}{x:C}}$).  The key point is that the first
premise, which reduces to
\[
\dsd{b}(x) \spr \dsd{b}(x) \otimes \dsd{b}(x)
\]
can be deduced as
\[
\dsd{b}(x) \deq \dsd{b}(x \times x) \deq \dsd{b}(x) \otimes \dsd{b}(x)
\]
by contraction for $\times$ \emph{if we add an axiom that \dsd{b}
  (strictly) preserves the monoidal product}
\[
\dsd{b}(x \times y) \deq \dsd{b}(x) \otimes \dsd{b}(y)
\]
In fact, we only need $\dsd{b}(x \times y) \spr \dsd{b}(x) \otimes
\dsd{b}(y)$ for this example, but in the LNL models of
\citet{benton94mixed} $F(A \times B) \cong F(A) \times F(B)$, and adding
the full equation will generate this isomorphism.  The $ \dsd{b}(x)
\otimes \dsd{b}(y) \spr \dsd{b}(x \times y)$ direction is used to prove
the purely linear logic entailment $!A \otimes !B \vdash !(A \otimes
B)$, for example.

Similarly, to get weakening we take $\dsd{b}(\top) \deq 1$.  

Thus, these equations, which are implicit in the syntactic treatment of
the context in \citep{bentonwadler96adjoint,reed09adjoint}, are an
explicit choice here, so we can also describe non-monoidal or
lax-monoidal left adjoints.

\subsection{Adjoint decomposition of $\Box$}  

The modal S4 \Bx{}{} as in \citet{pfenningdavies} is similar to !, but
with both context monoids cartesian (we write $(\times,\top)$ for
``truth'' and $(\times_v,\top_v)$ for ``validity''), and with only lax
monoid-preservation axioms (rather than the equality for $!$):
\[
\begin{array}{l}
\dsd{b}(x) \times \dsd{b}(y) \spr \dsd{b}(x \times_v y) \\
\top \spr \dsd{b}(\top_v)
\end{array}
\]
This follows the analysis of \Bx{}{} as a monoidal
comonad~\citep{alechina+01categoricals4}.  Because the context monoids
are cartesian products, there are always maps in the other direction.
However, in the equational theory of proofs in S4, there is a
section-retraction $(\Box A \times \Box B) \rightarrowtail \Box (A
\times B) \twoheadrightarrow (\Box A \times \Box B)$ but they are not
isomorphic. If we had equalities above, they would generate type
isomorphisms $\dsd{F}(A \times_v B) \cong \dsd{F}(A) \times \dsd{F}(B)$,
so because $\dsd{U}$ preserves products (it is a right adjoint), we
would incorrectly have $\dsd{F} \dsd{U} (A \times B) \cong \dsd{F}(U A
\times_v U B) \cong (\dsd{FU}(A) \times \dsd{FU}(B))$.

FIXME: equations on proofs

\subsection{Subexponentials}

\subsection{Monads}



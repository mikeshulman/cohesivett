%% FIXME: discussion of how the theories of context descriptors for the
%% various systems related to any semantic structure in existing models
%% of those systems.

\newcommand\truej[1]{#1 \,\, \dsd{true}}
\newcommand\possj[1]{#1 \,\, \dsd{poss}}
\newcommand\validj[1]{#1 \,\, \dsd{valid}}
\newcommand\crispj[1]{#1 \,\, \dsd{crisp}}
\newcommand\cohesivej[1]{#1 \,\, \dsd{coh}}

\section{Examples}
\label{sec:exampleencodings}

In the extended version, we give a more leisurely discussion of these
examples, and additionally consider subexponentials, modal S4 $\Box$,
and strong/$\Box$-strong monads.

\subsection{Products and Implications}

A mode theory with one mode \dsd{m} and a constant \oftp{x : \dsd{m}, y
  : \dsd{m}}{x \odot y}{\dsd{m}} specifies a completely astructural
context (no weakening, exchange, contraction, associativity), as in
non-associative Lambek calculus~\citep{lambek58calculus}.  To pass to
\emph{ordered logic} (associativity and unit laws but none of exchange,
weakening, and contraction), we add a constant $1 : \dsd{m}$ and
equational axioms $x \odot (y \odot z) \deq (x \odot y) \odot z$ and $x
\odot 1 \deq x \deq 1 \odot x$---i.e. $(\odot,1)$ is a monoid.  Then
associativity $A \odot (B \odot C) \vdash (A \odot B) \odot C)$ is
\begin{small}
\[
\infer[\FL]
{\seq{a:\F{x \odot p}{x:A,p:\F{y \odot z}{y:B,z:C}}}
  {a}
  {\F{q \odot z}{q:\F{x \odot y}{x:A,y:B},z:C}}}
{
  \infer[\FL]
        {\seq{x:A,p:\F{y \odot z}{y:B,z:C}}{x \times p}{{\F{q \odot z}{q:\F{x \odot y}{x:A,y:B},z:C}}}}
        {\infer[\FR]
          {\seq{x:A,y:B,z:C}{x \times (y \odot z)}{{\F{q \odot z}{q:\F{x \odot y}{x:A,y:B},z:C}}}}
          { {x \odot (y \odot z)} \spr (q \odot z)[x \odot y /q, z/z] &
            {\seq{x:A,y:B,z:C}{x \odot y}{\F{x \odot y}{x:A,y:B}}}
                  %% {x \otimes y \spr (x' \odot y')[x/x',y/y'] &
                  %%   \infer{\seq{x:A,y:B,z:C}{x}{A}}{} &
                  %%   \infer{\seq{x:A,y:B,z:C}{y}{B}}{} 
                  %% } 
            &
            \infer{\seq{x:A,y:B,z:C}{z}{C}}{}
        }}
}
\]
\end{small}
where ${\seq{x:A,y:B,z:C}{x \odot y}{\F{x \odot y}{x:A,y:B}}}$ is the
axiomatic \FR\/ instance.  Without the associativity axiom, the premise
${x \odot (y \odot z)} \spr ((x \odot y) \odot z)$ is not provable.  

Ordered logic has two different implications, one that adds to the left
of the context, and one that adds to the right; the expected rules are
\[
\begin{array}{l}
\infer{\seql{\Gamma}{o}{ A \rightharpoonup B}}
      {\seql{\Gamma,A}{o}B}
\qquad
\infer{\seql{\Gamma,A \rightharpoonup B,\Delta,\Gamma'}{o}{C}}
      {\seql{\Delta}{o}{A} &
       \seql{\Gamma,B,\Gamma'}{o}{C}
      }
\qquad
\infer{\seql{\Gamma}{o}{A \leftharpoonup B}}
      {\seql{A,\Gamma}{o}{B}}
\qquad
\infer{\seql{\Gamma,\Delta,A \leftharpoonup B,\Gamma'}{o}{C}}
      {\seql{\Delta}{o}{A} &
        \seql{\Gamma,B,\Gamma'}{o}{C}
      }
\end{array}
\]

We represent these by \Usymb-types: $A \rightharpoonup B := \U{c.c \odot
  x}{x:A}{B}$ and $A \leftharpoonup B := \U{c.x \odot c}{x:A}{B}$.
\begin{small}
\[
\infer{\seq{\Gamma}{\beta}{\U{c.c \odot x}{x:A}{B}}}
      {\seq{\Gamma,x:A}{\beta \odot x}{B}}
\qquad
\infer{\seq{\Gamma} {\beta} {C}}
      {\begin{array}{l}
          c:\U{c.c \odot x}{x:A}{B} \in \Gamma \\
          \beta \spr \beta'[c \odot \alpha/z] \\
          \seq{\Gamma}{\alpha}{A} \\
          \seq{\Gamma,z:A}{\beta'}{C}
        \end{array}
      }
\qquad
\infer{\seq{\Gamma}{\beta}{\U{c.x \odot c}{x:A}{B}}}
      {\seq{\Gamma,x:A}{x \odot \beta}{B}}
\qquad
\infer{\seq{\Gamma} {\beta} {C}}
      {\begin{array}{l}
          c:\U{c.x \odot c}{x:A}{B} \in \Gamma \\
          \beta \spr \beta'[\alpha \odot c/z] \\
          \seq{\Gamma}{\alpha}{A} \\
          \seq{\Gamma,z:A}{\beta'}{C}
       \end{array}
      }
\]
\end{small}
These have the expected right rules, putting $x$ on the left or right of
the current context descriptor, by the substitution $\beta/c$ in \UR.
Suppose that $\beta$ is of the form $x_1 \odot \ldots c \ldots \odot
x_n$ for distinct variables $x_i$, and consider the left rule for, for
$\rightharpoonup$.  Because the only structural transformations are the
associativity and unit equations, the transformation must reassociate
$\beta$ as $\beta_1 \odot (c \odot \alpha) \odot \beta_2$, with $\beta'
= \beta_1 \odot z \odot \beta_2$, for some $\beta_1$ and $\beta_2$.
Here $\alpha$ plays the role of $\Delta$ in the ordered logic rule---the
resources used to prove $A$, which occur to the right of the implication
being eliminated.  Reading the substitution backwards, the resources
$\beta'$ used for the continuation are ``$\beta$ with $c \odot \alpha$
replaced by the result of the implication,'' as desired.  While $c$ and
any variables used in $\alpha$ are still in $\Gamma$, permission to use
them has been removed from $\beta'$---and there is no way to restore
such permissions in this mode theory.  The rule for $\leftharpoonup$ is
the same, but with $\alpha$ on the opposite side of $c$.

Linear logic is ordered logic with exchange, so to model this we add a
commutativity equation $x \odot y \deq y \odot x$ making $\odot$ into a
commutative monoid.  %% For example, we can
%% derive {\seq{p : A \odot B}{p}{B \odot A}}:
%% \[
%% \infer[\FL]
%%       {\seq{p:\F{x\odot y}{x:A,y:B}}{p}{\F{z\odot w}{z:B,w:A}}}
%%       {\infer[\FR]{\seq{x:A,y:B}{x \odot y}{\F{z\odot w}{z:B,w:A}}}
%%         {
%%             x \odot y \spr (z \odot w) [y/z,x/w] &
%%             \seq{x:A,y:B}{y}{B} &
%%             \seq{x:A,y:B}{x}{A} 
%%       }}
%% \]
%% where the first premise is exactly $x \odot y \deq y \odot x$.  For
For this mode theory, \U{c.c \odot x}{x:A}{B} and \U{c.x \odot
  c}{x:A}{B} are equal types (because commutativity is an equation, and
types are parametrized by equivalence-classes of context descriptors),
and both represent $A \lolli B$.  If we add a directed structural
transformation $\dsd{w} :: x \spr 1$ then we get weakening (affine
logic), and with $\dsd{c} :: x \spr x \odot x$ we get contraction
(relevant logic):
\[
\infer[\FL]{\seq{z : \F{x \odot y}{x:A,y:B}}{z}{A}}
           {
             \infer{\seq{x:A,y:B}{x \odot y}{A}}
             {\infer{x \odot y \spr x \odot 1 \deq x}
                    {w :: y \spr 1}
               &
               \infer{\seq{x:A,y:B}{x}{A}}{}
           }}
\qquad
\infer[\FR]{\seq{z : A}{z}{\F{x \odot y}{x:A,y:A}}}
           {c :: z \spr (x \odot y)[z/x,z/y] &
            \infer{\seq{z:A}{z}{A}}{}
           }
\]
If we have both $\dsd{w} :: x \spr 1$ and $\dsd{c} :: x \spr x \odot
x$, then $x \odot y$ is a cartesian product in the mode theory, and
consequently $A \odot B$ will behave like a cartesian product type,
and $\U{c.c \otimes x}{x:A}{B}$ like the usual structural $A \to B$.  We
refer to this mode theory as an \emph{cartesian monoid} and write
$(\times,\top)$ for it.

These encodings are adequate in the following sense:
\begin{theorem}[Logical Adequacy]
Write $A^*$ for the encoding of a type as above and extend this
pointwise to contexts $\Gamma^*$.  Further, define
$\vars{x_1:A_1,\ldots,x_n:A_n} = x_1 \odot \ldots \odot x_n$.  Then
$\seql{\Gamma}{}{A}$ in the standard sequent calculus iff
$\seq{\Gamma^*}{\vars{\Gamma}}{A^*}$.
\end{theorem}
\begin{proof}
Detailed proofs for ordered logic (products), affine logic, and
cartesian logic are in the extended version. Encoding an object-language
derivation is straightforward, because the mode theory is chosen to make
each rule derivable.  The back-translation from the framework relies on
cut-freeness (so that we only need to translate normal forms), and a
lemma that, for these mode theories, left-rules on variables that are in
the framework context $\Gamma$ but do not occur in the context
descriptor $\alpha$ can be strengthened away.
\end{proof}

This approach extends to contexts with more than one type of tree node,
as in bunched implication~\citep{ohearnpym99bunched}, which has two
context-forming operations $\Gamma,\Gamma'$ and $\Gamma;\Gamma'$, along
with corresponding products and implications.  Both are associative,
unital, and commutative, but $;$ has weakening and contraction while $,$
does not.  A context is represented by a tree such as $(x:A, y:B);(z :
C, w : D)$ (considered modulo the laws), and the notation
$\Gamma[\Delta]$ is used to refer to a tree with a hole $\Gamma[-]$ that
has $\Delta$ as a subtree at the hole.  In sequent calculus style, the
rules for the product and implication corresponding to $,$ are
\[
\begin{array}{l}
\infer{\Gamma[A * B] \vdash C}
      {\Gamma[A , B] \vdash C}
\quad
\infer{\Gamma,\Delta \vdash A * B}
      {\Gamma \vdash A &
       \Delta \vdash B}
\quad
\infer{\Gamma \vdash A \magicwand B}
      {\Gamma, A \vdash B}
\quad
\infer{\Gamma[A \magicwand B, \Delta] \vdash C}
      {\Delta \vdash A &
       \Gamma[B] \vdash C}
\end{array}
\]
There are similar connectives $\times,\to$ for $;$ and structural rules
of weakening and contraction for it.

We model BI by a mode \dsd{m} with a commutative monoid $(*,I)$ and a
cartesian monoid $(\times,\top)$.  We define the BI products and
implications using the monoids as above (e.g.  $A * B := \F{x * y}{x :
  A, y : B}$ and $A \magicwand B := \U{c.c * x}{x : A}{B}$; replace $*$
with $\times$ for $\times$ and $\to$).  A context descriptor such as $(x
\times y) * (z \times w)$ captures the ``bunched'' structure of a BI
context, and substitution for a variable models the hole-filling
operation $\Gamma[\Delta]$.  The left rules are
\[
\infer{\seq{\Gamma,z:A*B,\Gamma'}{\beta}{C}}
      {\seq{\Gamma,\Gamma',x:A,y:B}{\subst{\beta}{x * y}{z}}{C}}
\qquad
\infer{\seq{\Gamma}{\beta}{C}}
      {
        c : A \magicwand B \in \Gamma &
        \beta \spr \beta'[ c * \alpha / z] & 
        \seq{\Gamma}{\alpha}{A} &
        \seq{\Gamma,z:B}{\beta'}{C} 
      }
\]
The rule for $*$ (and similarly $\times$) acts on a leaf and replaces
the leaf where $z$ occurs in the tree $\beta$ with the correct bunch
$x*y$. The left rule for $\magicwand$ (and similarly for $\to$) isolates
a subtree containing the implication $c$ and resources $*$'ed with it,
uses those resources to prove $A$, and then replaces the subtree with
the variable $z$ standing for the result of the implication.

%% We assume the BI sequent is given as a judgement $\Gamma \vdash A$ where
%% $\Gamma$ is a tree and there are explicit equality premises for the
%% algebraic laws on bunches.  Then we define $\Gamma^*$ as an in-order
%% flattening of the tree into one of our contexts (e.g.  $(x:A)^* = x:A^*$ and
%% $(\Gamma,\Delta)^* = (\Gamma;\Delta)^*=\Gamma^*,\Delta^*$), while we
%% define $\vars{\Gamma}$ as a context descriptor that preserves the tree
%% structure (e.g. $\vars{x:A} = x$ and $\vars{(\Gamma,\Delta)} =
%% \vars{\Gamma}*\vars{\Delta}$ and
%% $\vars{\Gamma;\Delta}=\vars{\Gamma}\times\vars{\Delta}$).  Then we have
%% the usual adequacy statement $\Gamma \vdash A$ iff
%% \seq{\Gamma^*}{\vars{\Gamma}}{A^*}.

\subsection{Multi-use variables}
\label{sec:ex:nlinear}

An $n$-use
variable~\citep{reed08namessubstructural,abel15modal,mcbride16nuttin} is
like a linear variable, but instead of being used ``exactly once''
(modulo additives), it is used ``exactly $n$ times.''  We use the
following sequent calculus rules for $n$-linear functions
\[
\infer{{0\cdot \Gamma,x:^1 P} \vdash {P}}
      {}
\qquad
\infer{\Gamma \vdash A \to^n B}
      {{\Gamma, x :^n A} \vdash {B}}
\qquad
\infer{\Gamma + f:^k A \to^n B + (nk \cdot \Delta) \vdash C}
      {\Delta \vdash A &
       {\Gamma, z :^k B} \vdash {C}}
\]
\noindent where $\Gamma + \Delta$ acts pointwise by $x :^{n} A + x :^{m}
A = x :^{n+m} A$ and $n \cdot \Delta$ acts pointwise by $n \cdot x^{m} A
= x :^{nm} A$.  In the left rule, $\Gamma$ and $\Delta$ have the same
underlying variables and types (but potentially different counts), and
$f:^kA \to^n B$ abbreviates a context with the same variables and types
but $0$'s for all counts besides $f$'s.  The left rule says that if you
spend $k$ ``uses'' of a function that takes $n$ uses of an
argument, then you need $nk$ uses of whatever you use to
construct the argument, in order to get $k$ uses of the result.  

We can model this in the mode theory of a commutative monoid by using
context descriptors that are themselves non-linear: we define $A \to^n B
:= \U{c.c \odot (x^n)}{x:A}{B}$ where $x^n := x \odot x \odot \ldots
\odot x$ ($n$ times).  This has the following instances of \UL{}{} and
\UR{}:
\[
\infer{\seq{\Gamma}{\beta}{A \to^n B}}
      {\seq{\Gamma, x:A}{\beta \odot x^n}{B}}
\qquad
\infer{\seq{\Gamma}{\beta}{C}}
      {f : \U{f.f \odot x^n}{x : A}{B} \in \Gamma &
        \beta \spr \beta'[f \odot (\alpha)^n/z] &
        \seq{\Gamma}{\alpha}{A} &
        \seq{\Gamma, z:B}{\beta'}{C} 
      }
\]
The only transformations are the commutative monoid equations, so we can
commute $\beta'$ to the form $\beta'' \odot z^k$ for some $k$ and
$\beta''$ not mentioning $z$ because any context descriptor is a
polynomial of variables. Thus the premise of $\UL$ is really of form
$\beta \deq (\beta'' \odot z^k) [f \odot (\alpha)^n/z]$, which is equal
to $\beta'' \odot f^k \odot (\alpha)^{nk}$.  Here $\beta''$ corresponds
to the $\Gamma$ in the above left rule (the resources of the
continuation, besides $z^k$) and $\alpha$ corresponds to $\Delta$.  The
full proof of adequacy is in the extended version:
\begin{theorem}[Logical adequacy for $n$-use variables]
$x_1:^{k_1} A_1,\ldots,x_n :^{k_n} A_n \vdash C$ iff
  \seq{x_1:A_1^*,\ldots,x_n:A_n^*}{x_1^{k_1} \odot \ldots \odot
    x_n^{k_n}}{C^*}, where $A^*$ translates $A \to^n B$ to 
$\U{c.c \odot (x^n)}{x:A^*}{B^*}$
\end{theorem}

\subsection{Comonads}  
\label{sec:example:bang}

Following \citet{benton94mixed,bentonwadler96adjoint}, we decompose the
$!$ exponential of intuitionistic linear logic as the comonad of an
adjunction between ``linear'' and ``cartesian'' categories.  We start
with two modes \dsd{l} (linear) and \dsd{c} (cartesian), along with a
commutative monoid $(\otimes,1)$ on \dsd{l} and a cartesian monoid
$(\times,\top)$ on \dsd{c}.  Next, we add a context descriptor from
\dsd{c} to \dsd{l} ($x : \dsd{c} \vdash \dsd{f}(x) : \dsd{l}$) that we
think of as including a cartesian context in a linear context.  This
generates types \wftype {\F{\dsd{f}(x)}{x : A_{\dsd{c}}}}{\dsd{l}} and
\wftype {\U{x.\dsd{f}(x)}{\cdot}{A_{\dsd{l}}}}{\dsd{c}} which are
adjoint $\F{\dsd{f}(x)}{x:-} \la {\U{x.\dsd{f}(x)}{\cdot}{-}}$.  The
bijection on hom-sets is defined using \FL\/ and \FR\/ and their
invertibility.  The comonad of the adjunction
\F{\dsd{f}(x)}{x:\U{c.\dsd{f}(c)}{\cdot}{A}} is the linear logic $!A$.

In the LNL models and sequent calculus~\citep{benton94mixed}, $F(A
\times B) \cong F(A) \otimes F(B)$ and $F(\top) \cong 1$, which we can
add to the mode theory by equations $\dsd{f}(x \times y) \deq \dsd{f}(x)
\otimes \dsd{f}(y)$ and $\dsd{f}(\top) \deq 1$. These equations then
extend to isomorphisms because all of $F,\otimes,\times$ are represented
by \Fsymb-types in our framework.  These properties of \dsd{f} are
necessary to prove that $!  A$ has weakening and contraction (with
respect to $\otimes$) and $!A \otimes !B \vdash !(A \otimes B)$, for
example.  Omitting these equations allows us to describe non-monoidal
(or lax monoidal, if we add only one direction) left adjoints;
in the extended version, we consider S4 $\Box$.  

\begin{theorem}[Logical adequacy for Adjoint $!$]
Translate $F(A)^* = \F{\dsd{f}(x)}{x:A^*}$ and $G(A)^* =
\U{x.\dsd{f}(x)}{\cdot}{A}$ and products and functions as usual.  Then
$x_1:C_1,\ldots,x_n:C_n \vdash C$ in the cartesian category iff
\seq{x_1:C_1^*,\ldots,x_n:C_n^*}{x_1 \times \ldots \times x_n}{C^*}, and
a mixed sequent with cartesian and linear assumptions and a linear
conclusion $x_1:C_1,\ldots,x_n:C_n;y_1:A_1,\ldots,y_m:A_m \vdash A$ iff
\seq{x_1:C_1^*,\ldots,y_1:A_1^*,\ldots}{\dsd{f}(x_1)
  \otimes\ldots\otimes \dsd{f}(x_n) \otimes y_1 \otimes \ldots \otimes
  y_n}{A^*}.
\end{theorem}

\subsection{Monads}
\label{sec:example:monad}

We model a \Dia{}{A} modality with rules in the style of
\citet{pfenningdavies}
%% \[
%% \infer{\Gamma \vdash \possj{A}}
%%       {\Gamma \vdash \truej{A}}
%% \qquad
%% \infer{\Gamma \vdash \truej{\Dia{}{A}}}
%%       {\Gamma \vdash \possj{A}}
%% \qquad
%% \infer{\Gamma,\truej{\Dia{}{A}} \vdash \possj{C}}
%%       {\truej{A} \vdash \possj{C}}
%% \]
by a mode theory with two modes \dsd{t} and \dsd{p}
and context descriptor \oftp{x:\dsd{t}}{\dsd{g}(x)}{\dsd{p}}, defining
the type $\Dia{}{A} := \U{c.\dsd{g}(c)}{\cdot}{\F{\dsd{g}(x)}{x:A}}$.
This is always a monad, but it does not automatically have a tensorial
strength.  For example, if we have a monoid $(\otimes,1)$ on mode
\dsd{t} and try to derive strength
\[
\infer[\UR]
      {\seq{x : A, y : \Dia{\dsd{g}}{B}}{x \otimes y}{\Dia{\dsd{g}}{(A \otimes B)}}}
      {\infer[\UL]
        {\seq{x : A, y : \Dia{\dsd{g}}{B}}{\dsd{g}(x \otimes y)}{\F{\dsd{g}}{A \otimes B}}}
        {\dsd{g}(x \otimes y) \spr \subst{\beta'}{\dsd{g}(y)}{z} &
          \seq{x:A,y : \Dia{\dsd{g}}{B},z:\F{\dsd{g}}{B}}{\beta'}{\F{\dsd{g}}{A \otimes B}}
        }}
\]
\noindent we are stuck, because there is no way to rewrite $\dsd{g}(x
\otimes_{\dsd t} y)$ as a term containing $\dsd{g}(y)$.  If $\otimes$ is
affine, then we can weaken away $x$ and take $\beta' =
z$---corresponding to the context-clearing in the left rule for
$\Dia{}{A}$ in \citet{pfenningdavies}---but then in the right-hand
premise we will only have access to $z$, not $x$, so we cannot complete
the derivation.  Thus, we can express non-strong monads cleanly in our
framework.  In the extended version, we prove adequacy for this and
extend the mode theory to express strong monads.

\begin{theorem}[Logical adequacy for a monad]
We translate all types at mode \dsd{t}, representing
\Dia{}{A} as above. Then $\truej{A_1}, \ldots,
\truej{A_1} \vdash \truej{C}$ iff
\seq{x_1:A_1^*,\ldots,x_1:A_n^*}{x_1\otimes\ldots\otimes x_n}{C^*}, and 
$\truej{A_1}, \ldots, \truej{A_b} \vdash \possj{C}$ iff 
\seq{x_1:A_1^*,\ldots,x_1:A_n^*}{\dsd{g}(x_1\otimes\ldots\otimes
  x_n)}{\F{\dsd{g}}{C^*}}.  
%% The three ``native'' rules above are
%% \FR, \UR, and a composite of \UL\/ followed by \FL, respectively.
\end{theorem}

\subsection{Spatial Type Theory}

The spatial type theory for cohesion~\citep{shulman15realcohesion}
(which motivated this work) has an adjoint pair $\flat \la \sharp$,
where $\flat$ is a comonad and $\sharp$ is a monad, with some additional
properties.  In the one-variable case~\citep{ls16adjoint}, we analyzed
this as arising from an idempotent comonad\footnote{There it was an
  idempotent monad; the variance of \dsd{F} and \dsd{U} has been flipped
  in paper.} in the mode theory: we have a mode \dsd{c} with a cartesian
monoid $(\times,\top)$ and a context descriptor
\oftp{x:\dsd{c}}{\dsd{r}(x)}{\dsd{c}} such that $\dsd{r}(\dsd{r}(x))
\deq \dsd{r}(x)$ and there is a directed transformation $\dsd{r}(x) \spr
x$.  Then we define $\flat A := \F{\dsd{r}}{A}$ and $\sharp A :=
\Uempty{\dsd{r}}{A}$. These are adjoint, and the transformation gives
the counit $\F{\dsd{r}}{A} \vdash A$ and the unit $A \vdash
\Uempty{\dsd{r}}{A}$.  Now that we have a multi-assumptioned logic, we
can model the fact that $\flat{A}$ preserves products by the equational
axiom $\dsd{r}(x \times y) \deq \dsd{r}(x) \times \dsd{r}(y)$.  Overall,
we encode a simply-typed spatial type theory judgement $x_1 :
\crispj{A_1},\ldots;y_1:\cohesivej{B_1} \vdash \cohesivej{C}$ as
$\seq{x_1:A_1,\ldots,y_1:B_1,\ldots}{\dsd{r}(x_1)\times\ldots\times
  y_1\times\ldots}{C}$.  As a sequent calculus, the rules
from~\citep{shulman15realcohesion} are
\[
\begin{array}{c}
\infer{\Delta;\Gamma \vdash C}
      {A \in \Delta &
       \Delta;\Gamma,A \vdash C}
\quad
\infer{\Delta; \Gamma \vdash {\Flat A}}
      {\Delta; \cdot \vdash {A}}
\quad
\infer{\Delta; \Gamma,\Flat{A} \vdash C}
      {\Delta,A; \Gamma \vdash C}
\quad
\infer{\Delta;\Gamma \vdash {\Sharp C}}
      {\Delta,\Gamma; \cdot \vdash C}
\quad
\infer{\Delta;\Gamma \vdash C}
      {\Sharp A \in \Delta &
        \Delta;\Gamma,A \vdash {C}}
\quad
\end{array}
\]
In order, these correspond to (1) the action of the contraction and
$\dsd{r}(x) \spr x$ transformations; (2) \FR\/ with weakening, using
monoidalness of \dsd{r} in one direction; (3) \FL; (4) \UR, using
monoidalness of \dsd{r} in the other direction and idempotence; (5) \UL,
with contraction.  This provides a satisfying explanation for the
unusual features of these rules, such as promoting all cohesive
variables to crisp in \Sharp{}-right, and eliminating a crisp \Sharp{}
in \Sharp{}-left, and illustrates how our framework can be used in
investigating extensions of homotopy type theory.

%% \subsection{Non-adjoints}

%% TODO E.g. the graded effects stuff, modalities in Lambek calculus  

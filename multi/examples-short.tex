%% FIXME: discussion of how the theories of context descriptors for the
%% various systems related to any semantic structure in existing models
%% of those systems.

\newcommand\truej[1]{#1 \,\, \dsd{true}}
\newcommand\possj[1]{#1 \,\, \dsd{poss}}
\newcommand\validj[1]{#1 \,\, \dsd{valid}}
\newcommand\crispj[1]{#1 \,\, \dsd{crisp}}
\newcommand\cohesivej[1]{#1 \,\, \dsd{coh}}

\section{Examples}
\label{sec:exampleencodings}

\subsection{Products and Implications}

First, we show how to encode substructural products and implications
with various structural properties.  A mode theory with one mode \dsd{m}
and a constant \oftp{x : \dsd{m}, y : \dsd{m}}{x \odot y}{\dsd{m}}
specifies a completely astructural context (no weakening, exchange,
contraction, associativity), as in non-associative Lambek
calculus~\citep{lambek58calculus}.  To pass to \emph{ordered logic}
(associativity and unit laws but none of exchange, weakening, and
contraction), we add a constant $1 : \dsd{m}$ and equational axioms $x
\odot (y \odot z) \deq (x \odot y) \odot z$ and $x \odot 1 \deq x \deq 1
\odot x$---i.e. $(\odot,1)$ is a monoid.  To get linear logic, we
additionally add commutativity $x \odot y \deq y \odot x$.  As a first
example of using the sequent calculus, we show how commutativity of
$\odot$ in the mode theory for linear logic generates commutativity of
the corresponding $A \otimes B$ type, which is represented by $\F{x
  \odot y}{x:A,y:B}$:
\begin{small}
\[
\infer[\FL]
      {\seq{q:\F{x\odot y}{x:A,y:B}}{q}{\F{z\odot w}{z:B,w:A}}}
      {\infer[\FR]{\seq{x:A,y:B}{x \odot y}{\F{z\odot w}{z:B,w:A}}}
        {
            x \odot y \spr (z \odot w) [y/z,x/w] &
            \seq{x:A,y:B}{y}{B} &
            \seq{x:A,y:B}{x}{A} 
      }}
\]
\end{small}%
First, we use \FL\/ to split the product type on the left up, obtaining
permission to use its pieces by substituting $(x \odot y)$ for the
variable $q$ we began with.  Next, to use \FR\/, we must transform the
current context descriptor $x \odot y$ into a substitution instance of
the one from the type $z \odot w$---dividing our resources in the form
dictated by the type.  We take $y/z,x/w$, which requires a
transformation $x \odot y \spr y \odot x$, which is given by reflexivity
because of the commutativity axiom in the mode theory.  Then we can
prove each of $A$ and $B$ by identity, because we have the correct
resources in each branch.  In the mode theory for ordered logic, without
commutativity, the only possible division is $x/z,y/w$, and with
permission only to use $x$ the first premise and $y$ in the second, the
derivation fails.

Returning to the mode theory of a non-symmetric $\odot$, we show how the
two implications of ordered logic are modeled by \Usymb-types; the
expected rules are
\begin{small}
\[
\begin{array}{l}
\infer{\seql{\Gamma}{o}{ A \rightharpoonup B}}
      {\seql{\Gamma,A}{o}B}
\qquad
\infer{\seql{\Gamma,A \rightharpoonup B,\Delta,\Gamma'}{o}{C}}
      {\seql{\Delta}{o}{A} &
       \seql{\Gamma,B,\Gamma'}{o}{C}
      }
\qquad
\infer{\seql{\Gamma}{o}{A \leftharpoonup B}}
      {\seql{A,\Gamma}{o}{B}}
\qquad
\infer{\seql{\Gamma,\Delta,A \leftharpoonup B,\Gamma'}{o}{C}}
      {\seql{\Delta}{o}{A} &
        \seql{\Gamma,B,\Gamma'}{o}{C}
      }
\end{array}
\]
\end{small}%
We represent these by the \Usymb-types $A \rightharpoonup B := \U{c.c
  \odot x}{x:A}{B}$ and $A \leftharpoonup B := \U{c.x \odot c}{x:A}{B}$.
The \UL\/ and \UR\/ rules specialize as follows:
\begin{small}
\[
\infer{\seq{\Gamma}{\beta}{\U{c.c \odot x}{x:A}{B}}}
      {\seq{\Gamma,x:A}{\beta \odot x}{B}}
\qquad
\infer{\seq{\Gamma} {\beta} {C}}
      {\begin{array}{l}
          c:\U{c.c \odot x}{x:A}{B} \in \Gamma \\
          \beta \spr \beta'[c \odot \alpha/z] \\
          \seq{\Gamma}{\alpha}{A} \\
          \seq{\Gamma,z:A}{\beta'}{C}
        \end{array}
      }
\qquad
\infer{\seq{\Gamma}{\beta}{\U{c.x \odot c}{x:A}{B}}}
      {\seq{\Gamma,x:A}{x \odot \beta}{B}}
\qquad
\infer{\seq{\Gamma} {\beta} {C}}
      {\begin{array}{l}
          c:\U{c.x \odot c}{x:A}{B} \in \Gamma \\
          \beta \spr \beta'[\alpha \odot c/z] \\
          \seq{\Gamma}{\alpha}{A} \\
          \seq{\Gamma,z:A}{\beta'}{C}
       \end{array}
      }
\]
\end{small}%
The \UR\, instances put $x$ on the left or right of the current context
descriptor $\beta$, by the substitution $\beta/c$ in \UR.  Consider the
left rule for $\rightharpoonup$/\U{c.c \odot x}{x:A}{B}, and suppose
that the $\beta$ in the conclusion is of the form $x_1 \odot \ldots c
\ldots \odot x_n$ for distinct variables $x_i$.  Because the only
structural transformations are the associativity and unit equations, the
transformation must reassociate $\beta$ as $\beta_1 \odot (c \odot
\alpha) \odot \beta_2$, with $\beta' = \beta_1 \odot z \odot \beta_2$,
for some $\beta_1$ and $\beta_2$.  Here $\alpha$ plays the role of
$\Delta$ in the ordered logic rule---the resources used to prove $A$,
which occur to the right of the implication being eliminated.  Reading
the substitution backwards, the resources $\beta'$ used for the
continuation are ``$\beta$ with $c \odot \alpha$ replaced by the result
of the implication,'' as desired.  While $c$ and any variables used in
$\alpha$ are still in $\Gamma$, permission to use them has been removed
from $\beta'$---and there is no way to restore such permissions in this
mode theory.  The rule for $\leftharpoonup$ is the same, but with
$\alpha$ on the opposite side of $c$.  For the linear logic mode theory,
\U{c.c \odot x}{x:A}{B} and \U{c.x \odot c}{x:A}{B} are equal types
(because commutativity is an equation, and types are parametrized by
equivalence-classes of context descriptors), and both represent $A
\lolli B$.

Weakening (affine logic) is modeled by adding a directed structural
transformation $\dsd{w} :: x \spr 1$, while contraction (relevant logic)
is modeled by $\dsd{c} :: x \spr x \odot x$.  These transformations in
the mode theory induce sequents $A \odot B \vdash A$ and $A \vdash A \odot A$:
\begin{small}
\[
\infer[\FL]{\seq{z : \F{x \odot y}{x:A,y:B}}{z}{A}}
           {
             \infer{\seq{x:A,y:B}{x \odot y}{A}}
             {\infer{x \odot y \spr x \odot 1 \deq x}
                    {w :: y \spr 1}
               &
               \infer{\seq{x:A,y:B}{x}{A}}{}
           }}
\qquad
\infer[\FR]{\seq{z : A}{z}{\F{x \odot y}{x:A,y:A}}}
           {c :: z \spr (x \odot y)[z/x,z/y] &
            \infer{\seq{z:A}{z}{A}}{}
           }
\]
\end{small}%
If we have both $\dsd{w} :: x \spr 1$ and $\dsd{c} :: x \spr x \odot x$
(with some equations relating them), then $x \odot y$ is a cartesian
product in the mode theory, and consequently $A \odot B$ will behave
like a cartesian product type, and $\U{c.c \otimes x}{x:A}{B}$ like the
usual structural $A \to B$.  We refer to this mode theory as an
\emph{cartesian monoid} and write $(\times,\top)$ for it.

These encodings are adequate in the following sense:
\begin{theorem}[Logical Adequacy for Products and Implications]
Write $A^*$ for the encoding of a type as above and extend this
pointwise to contexts $\Gamma^*$.  Further, define
$\vars{x_1:A_1,\ldots,x_n:A_n} = x_1 \odot \ldots \odot x_n$.  Then
$\seql{\Gamma}{}{A}$ in the standard sequent calculus iff
$\seq{\Gamma^*}{\vars{\Gamma}}{A^*}$.
\end{theorem}
\begin{proof}
Proofs for ordered logic (products), affine logic, and cartesian logic
are in the extended version. Encoding an object-language derivation is
straightforward, because the mode theory is chosen to make each rule
derivable.  The back-translation from the framework relies on
cut-freeness (so that we only need to back-translate normal forms), and
a lemma that, for these mode theories, left-rules on variables that are
in the framework context $\Gamma$ but do not occur in the context
descriptor $\alpha$ can be strengthened away.
\end{proof}

This approach extends to contexts with more than one type of tree node,
as in bunched implication~\citep{ohearnpym99bunched}, which has two
context-forming operations $\Gamma,\Gamma'$ and $\Gamma;\Gamma'$, along
with corresponding products and implications.  Both are associative,
unital, and commutative, but $;$ has weakening and contraction while $,$
does not.  A context is represented by a tree such as $(x:A, y:B);(z :
C, w : D)$ (considered modulo the laws), and the notation
$\Gamma[\Delta]$ is used to refer to a tree with a hole $\Gamma[-]$ that
has $\Delta$ as a subtree at the hole.  In sequent calculus style, the
rules for the product and implication corresponding to $,$ are
\begin{small}
\[
\begin{array}{l}
\infer{\Gamma[A * B] \vdash C}
      {\Gamma[A , B] \vdash C}
\quad
\infer{\Gamma,\Delta \vdash A * B}
      {\Gamma \vdash A &
       \Delta \vdash B}
\quad
\infer{\Gamma \vdash A \magicwand B}
      {\Gamma, A \vdash B}
\quad
\infer{\Gamma[A \magicwand B, \Delta] \vdash C}
      {\Delta \vdash A &
       \Gamma[B] \vdash C}
\end{array}
\]
\end{small}%
We model BI by a mode \dsd{m} with both a commutative monoid $(*,I)$ and
a cartesian monoid $(\times,\top)$.  We define the BI products and
implications using the monoids as above: $A * B := \F{x * y}{x : A, y :
  B}$ and $A \times B := \F{x \times y}{x:A,y:B}$ and $A \magicwand B :=
\U{c.c * x}{x : A}{B}$ and $A \to B := \U{c.c \times x}{x : A}{B}$.  A
context descriptor such as $(x \times y) * (z \times w)$ captures the
``bunched'' structure of a BI context, and substitution for a variable
models the hole-filling operation $\Gamma[\Delta]$.  The derived left
rules for $*$ and $\magicwand$ are
\begin{small}
\[
\infer{\seq{\Gamma,z:A*B,\Gamma'}{\beta}{C}}
      {\seq{\Gamma,\Gamma',x:A,y:B}{\subst{\beta}{x * y}{z}}{C}}
\qquad
\infer{\seq{\Gamma}{\beta}{C}}
      {
        c : A \magicwand B \in \Gamma &
        \beta \spr \beta'[ c * \alpha / z] & 
        \seq{\Gamma}{\alpha}{A} &
        \seq{\Gamma,z:B}{\beta'}{C} 
      }
\]
\end{small}%
The rule for $*$ (and similarly $\times$) acts on a leaf $z$ and replaces
the leaf where $z$ occurs in the tree $\beta$ with the correct bunch
$x*y$. The left rule for $\magicwand$ (and similarly for $\to$) isolates
a subtree containing the implication $c$ and resources $*$'ed with it,
uses those resources to prove $A$, and then replaces the subtree with
the variable $z$ standing for the result of the implication.

%% We assume the BI sequent is given as a judgement $\Gamma \vdash A$ where
%% $\Gamma$ is a tree and there are explicit equality premises for the
%% algebraic laws on bunches.  Then we define $\Gamma^*$ as an in-order
%% flattening of the tree into one of our contexts (e.g.  $(x:A)^* = x:A^*$ and
%% $(\Gamma,\Delta)^* = (\Gamma;\Delta)^*=\Gamma^*,\Delta^*$), while we
%% define $\vars{\Gamma}$ as a context descriptor that preserves the tree
%% structure (e.g. $\vars{x:A} = x$ and $\vars{(\Gamma,\Delta)} =
%% \vars{\Gamma}*\vars{\Delta}$ and
%% $\vars{\Gamma;\Delta}=\vars{\Gamma}\times\vars{\Delta}$).  Then we have
%% the usual adequacy statement $\Gamma \vdash A$ iff
%% \seq{\Gamma^*}{\vars{\Gamma}}{A^*}.

\subsection{Multi-use variables}
\label{sec:ex:nlinear}

An $n$-use
variable~\citep{reed08namessubstructural,abel15modal,mcbride16nuttin} is
a variable that is used ``exactly $n$ times'' (modulo additives), as
expressed by the following sequent calculus rules for $n$-use functions
\begin{small}
\[
\infer{{0\cdot \Gamma,x:^1 P} \vdash {P}}
      {}
\qquad
\infer{\Gamma \vdash A \to^n B}
      {{\Gamma, x :^n A} \vdash {B}}
\qquad
\infer{\Gamma + f:^k A \to^n B + (nk \cdot \Delta) \vdash C}
      {\Delta \vdash A &
       {\Gamma, z :^k B} \vdash {C}}
\]
\end{small}%
where $\Gamma + \Delta$ acts pointwise by $x :^{n} A + x :^{m}
A = x :^{n+m} A$ and $n \cdot \Delta$ acts pointwise by $n \cdot x^{m} A
= x :^{nm} A$.  In the left rule, $\Gamma$ and $\Delta$ have the same
underlying variables and types (but potentially different counts), and
$f:^kA \to^n B$ abbreviates a context with the same variables and types
but $0$'s for all counts besides $f$'s.  The left rule says that if you
spend $k$ ``uses'' of a function that takes $n$ uses of an
argument, then you need $nk$ uses of whatever you use to
construct the argument, in order to get $k$ uses of the result.  

We can model this in the mode theory of a commutative monoid by using
context descriptors that are themselves non-linear: we define $A \to^n B
:= \U{c.c \odot (x^n)}{x:A}{B}$ where $x^n := x \odot x \odot \ldots
\odot x$ ($n$ times).  This has the following instances of \UL{}{} and
\UR{}:
\begin{small}
\[
\infer{\seq{\Gamma}{\beta}{A \to^n B}}
      {\seq{\Gamma, x:A}{\beta \odot x^n}{B}}
\qquad
\infer{\seq{\Gamma}{\beta}{C}}
      {f : \U{f.f \odot x^n}{x : A}{B} \in \Gamma &
        \beta \spr \beta'[f \odot (\alpha)^n/z] &
        \seq{\Gamma}{\alpha}{A} &
        \seq{\Gamma, z:B}{\beta'}{C} 
      }
\]
\end{small}%
In the left rule, $\beta'$ must be equal to some term $\beta'' \odot
z^k$ for some $k$ and $\beta''$ not mentioning $z$ (for this mode
theory, any term is a polynomial of variables), and the only structural
transformations are the commutative monoid equations, so the premise is
$\beta \deq (\beta'' \odot z^k) [f \odot (\alpha)^n/z] \deq \beta''
\odot f^k \odot (\alpha)^{nk}$.  Here $\beta''$ corresponds to the
$\Gamma$ in the above left rule (the resources of the continuation,
besides $z^k$) and $\alpha$ corresponds to $\Delta$.  The full proof of
adequacy is in the extended version:
\begin{theorem}[Logical adequacy for $n$-use variables]
$x_1:^{k_1} A_1,\ldots,x_n :^{k_n} A_n \vdash C$ iff
  \seq{x_1:A_1^*,\ldots,x_n:A_n^*}{x_1^{k_1} \odot \ldots \odot
    x_n^{k_n}}{C^*}, where $A^*$ translates $A \to^n B$ to 
$\U{c.c \odot (x^n)}{x:A^*}{B^*}$
\end{theorem}

\subsection{Comonads}  
\label{sec:example:bang}

Following \citet{benton94mixed,bentonwadler96adjoint}, we decompose the
$!$ exponential of intuitionistic linear logic as the comonad of an
adjunction between ``linear'' and ``cartesian'' categories.  We start
with two modes \dsd{l} (linear) and \dsd{c} (cartesian), along with a
commutative monoid $(\otimes,1)$ on \dsd{l} and a cartesian monoid
$(\times,\top)$ on \dsd{c}.  Next, we add a context descriptor from
\dsd{c} to \dsd{l} ($x : \dsd{c} \vdash \dsd{f}(x) : \dsd{l}$) that we
think of as including a cartesian context in a linear context.  This
generates types \wftype {\F{\dsd{f}(x)}{x : A_{\dsd{c}}}}{\dsd{l}} and
\wftype {\U{x.\dsd{f}(x)}{\cdot}{A_{\dsd{l}}}}{\dsd{c}} which are
adjoint $\F{\dsd{f}(x)}{x:-} \la {\U{x.\dsd{f}(x)}{\cdot}{-}}$.  The
bijection on hom-sets is defined using \FL\/ and \UR\/ and their
invertibility.  The comonad of the adjunction
\F{\dsd{f}(x)}{x:\U{c.\dsd{f}(c)}{\cdot}{A}} is the linear logic $!A$.

In LNL~\citep{benton94mixed}, $F(A \times B) \cong F(A) \otimes F(B)$
and $F(\top) \cong 1$ (these properties of $F$ are necessary to
prove that $!  A$ has weakening and contraction with respect to
$\otimes$, for example), which we can add to the mode theory by
equations $\dsd{f}(x \times y) \deq \dsd{f}(x) \otimes \dsd{f}(y)$ and
$\dsd{f}(\top) \deq 1$. By Theorem~\ref{lem:fusion-respect}, these
equations induce type isomorphisms because all of $F,\otimes,\times$ are
represented by \Fsymb-types in our framework.  For example, $F(A \times
B) \vdash F(A) \otimes F(B)$ is derived as follows:
\begin{small}
\[
\infer[\FL]{\seq{q:\F{\dsd{\dsd{f}(x)}}{x:\F{y \times z}{y:A,z:B}}}{q}{\F{z \otimes w}{z:\F{\dsd{f}(x)}{x:A},w:\F{\dsd{f}(x)}{x:B}}}}
      {\infer[\FL]{\seq{x:{\F{y \times z}{y:A,z:B}}}{\dsd{f}{(x)}}{\F{z \otimes w}{z:\F{\dsd{f}(x)}{x:A},w:\F{\dsd{f}(x)}{x:B}}}}
        {\infer[\FR]{\seq{y:A,z:B}{\dsd{f}{(y \times z)}}{\F{z \otimes w}{z:\F{\dsd{f}(x)}{x:A},w:\F{\dsd{f}(x)}{x:B}}}}
          {\dsd{f}{(y \times z)} \deq \dsd{f}(y) \otimes \dsd{f}(z) &
            \infer[\FR^*]{\seq{y:A,z:B}{\dsd{f}{(y)}}{\F{x.\dsd{f}(x)}{x:A}}}{} & 
            \infer[\FR^*]{\seq{y:A,z:B}{\dsd{f}{(z)}}{\F{x.\dsd{f}(x)}{x:B}}}{} & 
          }}}
\]
\end{small}%
Omitting these equations allows us to describe non-monoidal (or lax
monoidal, if we add only one direction) left adjoints: in the extended
version, we consider S4 $\Box$, and prove adequacy for it.

%% \begin{theorem}[Logical adequacy for Adjoint $!$]
%% Translate $F(A)^* = \F{\dsd{f}(x)}{x:A^*}$ and $G(A)^* =
%% \U{x.\dsd{f}(x)}{\cdot}{A}$ and products and functions as usual.  Then
%% $x_1:C_1,\ldots,x_n:C_n \vdash C$ in the cartesian category iff
%% \seq{x_1:C_1^*,\ldots,x_n:C_n^*}{x_1 \times \ldots \times x_n}{C^*}, and
%% a mixed sequent with cartesian and linear assumptions and a linear
%% conclusion $x_1:C_1,\ldots,x_n:C_n;y_1:A_1,\ldots,y_m:A_m \vdash A$
%% holds iff
%% \seq{x_1:C_1^*,\ldots,y_1:A_1^*,\ldots}{\dsd{f}(x_1)
%%   \otimes\ldots\otimes \dsd{f}(x_n) \otimes y_1 \otimes \ldots \otimes
%%   y_n}{A^*}.
%% \end{theorem}

\subsection{Monads}
\label{sec:example:monad}

We model a \Dia{}{A} modality with rules in the style of
\citet{pfenningdavies}
%% \[
%% \infer{\Gamma \vdash \possj{A}}
%%       {\Gamma \vdash \truej{A}}
%% \qquad
%% \infer{\Gamma \vdash \truej{\Dia{}{A}}}
%%       {\Gamma \vdash \possj{A}}
%% \qquad
%% \infer{\Gamma,\truej{\Dia{}{A}} \vdash \possj{C}}
%%       {\truej{A} \vdash \possj{C}}
%% \]
by a mode theory with two modes \dsd{t} and \dsd{p}
and context descriptor \oftp{x:\dsd{t}}{\dsd{g}(x)}{\dsd{p}}; we define
$\Dia{}{A} := \U{c.\dsd{g}(c)}{\cdot}{\F{\dsd{g}(x)}{x:A}}$.
This is always a monad, but it does not automatically have a tensorial
strength.  For example, if we have a monoid $(\otimes,1)$ on mode
\dsd{t} and try to derive strength
\[
\infer[\UR]
      {\seq{x : A, y : \Dia{\dsd{g}}{B}}{x \otimes y}{\Dia{\dsd{g}}{(A \otimes B)}}}
      {\infer[\UL]
        {\seq{x : A, y : \Dia{\dsd{g}}{B}}{\dsd{g}(x \otimes y)}{\F{\dsd{g}}{A \otimes B}}}
        {\dsd{g}(x \otimes y) \spr \subst{\beta'}{\dsd{g}(y)}{z} &
          \seq{x:A,y : \Dia{\dsd{g}}{B},z:\F{\dsd{g}}{B}}{\beta'}{\F{\dsd{g}}{A \otimes B}}
        }}
\]
\noindent we are stuck, because there is no way to rewrite $\dsd{g}(x
\otimes y)$ as a term containing $\dsd{g}(y)$.  If $\otimes$ is affine,
then we can weaken away $x$ and take $\beta' = z$---corresponding to the
context-clearing in the left rule for $\Dia{}{A}$ in
\citet{pfenningdavies}---but then in the right-hand premise we will only
have access to $z$, not $x$, so $\Diamond$ correctly represents a
non-strong monad in this setting.  In the extended version, we prove
adequacy for this and extend the mode theory to express strong monads.

\begin{theorem}[Logical adequacy for a monad]
We translate all types at mode \dsd{t}, representing
\Dia{}{A} as above. Then $\truej{A_1}, \ldots,
\truej{A_1} \vdash \truej{C}$ iff
\seq{x_1:A_1^*,\ldots,x_1:A_n^*}{x_1\otimes\ldots\otimes x_n}{C^*}, and 
$\truej{A_1}, \ldots, \truej{A_n} \vdash \possj{C}$ iff 
\seq{x_1:A_1^*,\ldots,x_1:A_n^*}{\dsd{g}(x_1\otimes\ldots\otimes
  x_n)}{\F{\dsd{g}}{C^*}}.  
%% The three ``native'' rules above are
%% \FR, \UR, and a composite of \UL\/ followed by \FL, respectively.
\end{theorem}

\subsection{Spatial Type Theory}

The spatial type theory for cohesion~\citep{shulman15realcohesion}
which motivated this work has an adjoint pair $\flat \la \sharp$,
where $\flat$ is a comonad and $\sharp$ is a monad, with some additional
properties.  In the one-variable case~\citep{ls16adjoint}, we analyzed
this as arising from an idempotent comonad\footnote{There it was an
  idempotent monad; the variance of \dsd{F} and \dsd{U} has been flipped
  in paper.} in the mode theory: we have a mode \dsd{c} with a cartesian
monoid $(\times,\top)$ and a context descriptor
\oftp{x:\dsd{c}}{\dsd{r}(x)}{\dsd{c}} such that $\dsd{r}(\dsd{r}(x))
\deq \dsd{r}(x)$ and there is a directed transformation $\dsd{r}(x) \spr
x$.  Then we define $\flat A := \F{\dsd{r}}{A}$ and $\sharp A :=
\Uempty{\dsd{r}}{A}$. These are adjoint, and the transformation gives
the counit $\F{\dsd{r}}{A} \vdash A$ and the unit $A \vdash
\Uempty{\dsd{r}}{A}$.  Now that we have a multi-assumptioned logic, we
can model the fact that $\flat{A}$ preserves products by the equational
axiom $\dsd{r}(x \times y) \deq \dsd{r}(x) \times \dsd{r}(y)$.  Overall,
we encode a simply-typed spatial type theory judgement $x_1 :
\crispj{A_1},\ldots;y_1:\cohesivej{B_1} \vdash \cohesivej{C}$ as
$\seq{x_1:A_1,\ldots,y_1:B_1,\ldots}{\dsd{r}(x_1)\times\ldots\times
  y_1\times\ldots}{C}$.  As a sequent calculus, the rules
from~\citep{shulman15realcohesion} are
\begin{small}
\[
\begin{array}{c}
\infer{\Delta;\Gamma \vdash C}
      {A \in \Delta &
       \Delta;\Gamma,A \vdash C}
\quad
\infer{\Delta; \Gamma \vdash {\Flat A}}
      {\Delta; \cdot \vdash {A}}
\quad
\infer{\Delta; \Gamma,\Flat{A} \vdash C}
      {\Delta,A; \Gamma \vdash C}
\quad
\infer{\Delta;\Gamma \vdash {\Sharp C}}
      {\Delta,\Gamma; \cdot \vdash C}
\quad
\infer{\Delta;\Gamma \vdash C}
      {\Sharp A \in \Delta &
        \Delta;\Gamma,A \vdash {C}}
\quad
\end{array}
\]
\end{small}
In order, these correspond to (1) the action of the contraction and
$\dsd{r}(x) \spr x$ transformations; (2) \FR\/ with weakening, using
monoidalness of \dsd{r} in one direction; (3) \FL; (4) \UR, using
monoidalness of \dsd{r} in the other direction and idempotence; (5) \UL,
with contraction.  This provides a satisfying explanation for the
unusual features of these rules, such as promoting all cohesive
variables to crisp in \Sharp{}-right, and eliminating a crisp \Sharp{}
in \Sharp{}-left, and illustrates how our framework can be used in
investigating extensions of homotopy type theory.

%% \subsection{Non-adjoints}

%% TODO E.g. the graded effects stuff, modalities in Lambek calculus  

\section{Equational Adequacy}
\label{sec:adequacy-equational}

\subsection{Template}

We would like to show that the logical adequacy proofs in the previous
section are in fact an isomorphism of derivations modulo definitional
equality.  One approach we could take to this is to look for something
like a full and faithful functor between object-logic sequents $J$ and
their translations $J^*$ in the frmaework.  This form of adequacy is
something like a conservative extension result: it says that no matter
what instances of \Fsymb\, and \Usymb\, types we have in the framework, equality
of derivations sequents in the image of the translation corresponds to
object-logic equality.  It seems likely to us that this is true, though
the proof gets complex enough that to be sure we would want to mechanize
it.  The reason is that the ``back translation'' of a framework
derivation to the source uses cut elimination (because a cut formula
might not come from the source), so to prove that this is faithful we
need to show that all of the cut, identity, left inversion steps can be
translated back to the source.  

A alternative is to fix the collection of \Fsymb\, and \Usymb\, types to
correspond exactly to the source (this may not be possible in all cases,
if a source connective is translated as a composition of \Fsymb's and
\Usymb's), at which point \emph{any} derivation, not just cut-free ones,
can be translated back to the source.  Semantically, this corresponds to
fixing a collection of mode morphisms $(\vec{\mu},\vec{\mu'})$ for which
we have $\mu$-products and $\mu'$-homs, rather than asserting that we
have all products and homs.  This simplifies the proof because the
equational axioms on the source and the framework are quite similar to
each other.  We show one example of this approach here.

\newcommand\backtrf[1]{\ensuremath{#1^{\leftarrow}}}
\newcommand\backtr[1]{\ensuremath{#1^{\Leftarrow}}}
\newcommand\str[2]{\ensuremath{\dsd{str}_{#1}(#2)}}

\begin{definition}[Components of an equational adequacy proof] ~
\begin{enumerate}
\item The translation from types to types ($A^*$) and sequents to
  sequents ($J^*$).

\item For each source inference rule, a \emph{derivation} $d^*$ from the
  translated premises to the translated conclusion (not just an
  admissibility: each rule will be defined by a composition of framework
  inference rules).

\item A proof that equality axioms are preserved: for each
  connective-specific equality axiom (typically $\beta\eta$) and for the
  cut and identity associativity and unit laws, $d_1 \deq d_2$ implies
  $d_1^* \deq d_2^*$.

\item A function \backtrf{-} from derivations $e : J^*$ to source
  derivations of $J$.  If the output does not use the cut rule, or
  identity at non-base-types, except in the translation of cut and
  identity, this gives cut and identity elimination for the source as a
  corollary of the corresponding principles for the framework.

\item A proof that $e,e' : J^*$, $e \deq e'$ imples
  $\backtrf{e} \deq \backtrf{e'}$.  

\item A proof that ${\backtrf{e}}^* \deq e$.  

\item A proof that $\backtrf{{d^*}} \deq d$.  
\end{enumerate}
\end{definition}

This shows that derivations of object-language derivations of $J/\deq$
are isomorphic to framework derivations of $J^*/\deq$.  First, the
function from object derivations to framework is given in part 2, and
the fact that it preserves object $\deq$ follows from part 3.  The cases
for reflexivity, symmetry, transitivity, and congruence are essentially
automatic, because each object-language rule is derived in the
framework, so we can use the corresponding congruence rules in the
framework.  The function in the opposite direction is given on raw
derivations in part 4, and shown to respect equality in part 5.  Parts 6
and 7 show that the composites are the identity on raw derivations.  

We will use the following lemmas:

\begin{lemma}[Equational 0-use Strengthing] \label{lem:0-use-strengthening-eq}
Under the conditions of Lemma~\ref{lem:0-use-strengthening}, write
\str{\vec{x}}{\D} for the derivation produced by the lemma.  Then
$\str{\vec{x}}{\D}$ (weakened with $\vec{x}$) $\deq \D$.  
%% Moreover, if $\vec{x}$ do not occur in the
%% derivation $d$ (i.e. at an identity rule or as the principal argument of
%% a left rule), then $\str{\vec{x}}{d}$ is syntactically equal to $d$.
\end{lemma}
\begin{proof}
Inspecting the proof, we have the following reductions:
\[
\begin{array}{rcll}
\str{\vec{x}}{\Trd{\beta}{y}} & := & {\Trd{\beta}{y}}\\
\str{\vec{x}}{\FRd{}{s}{\vec{d}}} & := & \FRd{}{s}{\str{\vec{x}}{\vec{d}}}\\
\str{\vec{x}}{\FLd{x}{\Delta.d}} & := & \str{\vec{x},x,\Delta}{d} & \text{ if } x \in \vec{x}\\
\str{\vec{x}}{\FLd{x}{\Delta.d}} & := & \FLd{x}{\Delta.\str{\vec{x}}{d}} & \text{ if } x \not\in \vec{x}\\
\str{\vec{x}}{\URd{\Delta.d}} & := & \URd{\Delta.\str{\vec{x}}{d}}\\
\str{\vec{x}}{\ULd{x}{s}{\vec{d}}{z.d'}} & := & \Trd{s}{\str{\vec{x,z}}{d'}} & \text{ if } z \# \beta'\\
\str{\vec{x}}{\ULd{x}{}{s}{\vec{d}}{z.d'}} & := & \ULd{x}{}{s}{\str{\vec{x}}{d}}{z.\str{\vec{x}}{d'}} & \text{ if } z \in \beta', x \notin \vec{x}\\
\end{array}
\]
%% If $\vec{x}$ do not occur in the derivation, then no rules are deleted,
%% so we can prove that the term is unchanged by the inductive hypothesis
%% in each case.  
Most cases follow from the inductive hypothesis; the interesting ones
are where a rule is deleted.  

\begin{itemize}

\item Suppose we began with $\FLd{x}{\Delta.d} :
  \seq{\Gamma,x:\F{\alpha}{\Delta}}{\beta}{B}$ with $x \in \vec{x}$ and
  produced $\str{\vec{x},x,\Delta}{d} : \seq{\Gamma-\vec{x}}{\beta}{B}$
  (because neither $x$ nor $\Delta$ occur in $\beta$).  Then weakening
  with $x,\Delta$ gives $\str{\vec{x},x,\Delta}{d} :
  \seq{\Gamma,x:\F{\alpha}{\Delta},\Delta}{\beta}{B}$, and the inductive
  hypothesis gives that this is equal to $d$.  

  Thus, it suffices to show that for any $e :
  \seq{\Gamma,x:\F{\alpha}{\Delta}}{\beta}{B}$ where $x$ doesn't occur
  in $\beta$ or $e$, then $e$ is equal to \FLd{x}{\Delta.e}---i.e. we
  can introduce a ``dead branch'' on $x$.  But by the $\eta$ rule we
  have $e \deq \FLd{x}{\Cut{e}{\FR}{x}}$, and the cut cancels
  because $x$ does not occur.
  
\item Suppose we began with $\ULd{x}{}{s}{\vec{d}}{z.d'} :
  \seq{\Gamma,x:\U{\alpha}{\Delta}{A}}{\beta}{B}$ and $z$ does not occur
  in $\beta'$, the resources of $d'$, and we produce
  \Trd{s}{\str{\vec{x},z}{d'}}, and want to show this is equal to
  $\ULd{x}{}{s}{\vec{d}}{z.d'}$.  By the IH, weakening
  $\str{\vec{x},z}{d'}$ is equal to $d'$.

  Thus, it suffices to show that $\ULd{x}{}{s}{\vec{d}}{z.d'} \deq
  \Trd{s}{d'}$ when $z$ does not occur in the \emph{derivation} $d'$.
  This is immediate from the semantic expansion of \UL{} 
  \[
  \ULd{x}{}{s}{\vec{d/y}}{z.d'} \deq \Trd{s}{(d'[\ULs{x}[\vec{d/y}]/z])}
  \]
  because in this the substitution for $z$, which does not occur,
  cancels.

\end{itemize}

\end{proof}
%% We say that a formula $\F{\alpha}{\Delta}$ and \U{c.\alpha}{\Delta}{A}
%% is relevant if every variable from $\Delta$ (and $c$ for \Usymb) occurs
%% at least once in $\alpha$.

%% Suppose the mode theory has the property that for all $x$, $\alpha$,
%% $\beta$, if $\alpha \spr \beta$ and $x \# \alpha$ then $x \# \beta$ (in
%% particular, equations must have the same variables on both sides).
%% Suppose additionally a sequent \seq{\Gamma}{\alpha}{A} such that every
%% \Fsymb/\Usymb\/ subformula of $\Gamma,A$ is relevant.

%% Then if $\D :: \seq{\Gamma}{\alpha}{A}$ and $\vec{x}$ are variables such
%% that $\vec{x} \# \alpha$ then there is a $\D' ::
%% \seq{\Gamma-\vec{x}}{\alpha}{A}$ and $size(\D') \le size(\D)$ 
%% $\D' \deq \D$ (when $\D'$ is weakened to reintroduce $\vec{x}$).  

\begin{lemma}[Properties of Strengthening] \label{lem:strengthening-eq-properties} ~ 
\begin{itemize}
\item $\str{\Gamma}{\str{\Delta}{e}} = \str{\Gamma,\Delta}{e}$
\item $\str{\Delta,x}{d} = \str{\Delta,x,\Delta'}{\Linv{d}{\Delta'}{x}}$
\item $\str{\Delta}{\Linv{d}{\Delta'}{x}} = \Linv{\str{\Delta}{d}}{\Delta'}{x}$ if $x \# \Delta$
\item $\str{\vec{x}}{\Ident{y}} = \Ident{y}$ if $y \not \in x$.  
\item $\str{\vec{x}}{d} = d$ if $\vec{x} \# d$ and for every $\UL$ in $d$,
  $z$ occurs in $\beta'$.    
\end{itemize}
\end{lemma}

\begin{proof}
For the first part, deleting rules in two passes is the same one, as
long as all of the same variables are deleted overall.  

For the second part, deleting all uses of $x$ is the same as first
inverting $x$ (which deletes all left rules on it) and then deleting all
uses of the variables that are produced by the inversion (which are
deleted when strengthening hits $\FL^{x}$).

For the third, left inversion commutes with deleting variables other
than the ones being deleted.  

For the fourth, the $\eta$-expanded identity on $y$ contains no variables
beside $y$ and those introduced by rules in it, so $\str{\vec{x}}$
deletes nothing.  

For the fifth, strengthening only deletes (1) left rules on $x$ and
subsidiaries, and (2) unneeded occurences of $\UL$, and the premises say
that there are none of these.  
\end{proof}

\begin{lemma}[Cut does not introduce left rules] \label{lem:cut-doesnt-intro}
If $\FL^{x}$ or $\UL^{x}$ occus in $\elim{d}$, then it occurs in $d$
\end{lemma}

\begin{proof}
By induction on the execution trace of cut elimination: for each rule in
Figure~\ref{fig:admissible-rule-equations}, $\FL^{x}$ or $\UL^{x}$ 
occurs on the right only if it occurs on the left.  
\end{proof}

\subsection{Ordered Logic (Product Only)}

Consider ordered logic with only $A \odot B$:
\[
\infer{\seql{A}{o}{A}}{}
\quad
\infer{\seql{\Gamma,\Delta,\Gamma'}{o}{C}}
      {\seql{\Gamma,A,\Gamma'}{o}{C} &
        \seql{\Delta}{o}{A}}
\quad
\infer{\seql{\Gamma,A \odot B,\Gamma'}{o}{C}}
      {\seql{\Gamma,A,B,\Gamma'}{o}{C}}
\quad
\infer{\seql{\Gamma,\Delta}{o}{A \odot B}}
      {\seql{\Gamma}{o}{A} &
        \seql{\Delta}{o}{B}}
\]

Source equality is the least congruence containing
\[
\begin{array}{c}
\Cut{d}{x}{x} \deq d \deq \Cut{x}{d}{x} \\
\Cut{\Cut{d_1}{d_2}{x}}{d_3}{y} \deq \Cut{\Cut{d_1}{d_3}{y}}{d_2}{x} \text{ if } x \in d_1\\
\Cut{\Cut{d_1}{d_2}{x}}{d_3}{y} \deq \Cut{d_1}{\Cut{d_2}{d_3}{y}}{x} \text{ if } x \in d_2\\
\Cut{\dotLd{z}{x,y.d}}{\dotRd{d_1}{d_2}}{z} \deq \Cut{\Cut{d}{d_1}{x}}{d_2}{y}\\
d : \seql{\Gamma,z:A \odot B,\Gamma'}{o}{C} \deq \dotLd{z}{x,y.\Cut{d}{\dotRd{x}{y}}{z}}\\
\end{array}
\]
In the associativity rule, the equality depends on whether $x$ occurs in $d_2$

We use a mode theory with a monoid $(\odot,1)$, so the only
transformation axioms are equality axioms for associativity and unit.
We restrict to sequents with $\mu$-products for $\mu = x,y \vdash x
\otimes y$---i.e. every type occuring in a sequent (context or
conclusion) is either an atom or $\F{x \odot y}{x:A,y:B}$, where $A$ and
$B$ also have this property.  

\begin{enumerate}
\item The type translation is given by $P^* := P$ and $(A \odot B)^* :=
  \F{x \odot y}{x:A^*,y:B^*}$.  A context $(x_1:A_1,\ldots,x_n:A_n)^* :=
  x_1:A_1^*,\ldots,x_n:A_n^*$.  Writing $\vars{x_1:A_1,\ldots,x_n:A_n}
  := x_1 \odot \ldots \odot x_n$, a sequent $\seql{\Gamma}{o}{A}$ is
  translated to \seq{\Gamma^*}{\vars{\Gamma}}{A^*}.

  Observe that, under the restriction above, every type in the framework
  is the image of exactly one source type--$P$ comes from $P$, and $\F{x
    \odot y}{x:A,y:B} = (A' \odot B')^*$, where inductively ${A'}^* = A$
  and $B'^* = B$.  

We use the following properties of the mode theory:
\begin{itemize}
\item If ${\vars{\Gamma^*}} \deq {x}$ then $\Gamma$ is $x:Q$ for some
  $Q$.  
\item If $\vars{\Gamma} \deq \alpha_1 \odot \alpha_2$, then there exist
  $\Gamma_1,\Gamma_2$ such that $\Gamma = \Gamma_1,\Gamma_2$ and
  $\vars{\Gamma_1} \deq \alpha_1$ and $\vars{\Gamma_2} \deq \alpha_2$.
\item $A^*$ and $\Gamma^*$ are relevant propositions, and the monoid
  axioms preserve variables, so by Lemma~\ref{lem:0-use-strengthening} we can
  strengthen away any variables that are not in the context descriptor.  
\end{itemize}

\item As discussed in Section~\ref{sec:adequacy:ordered-logical}, the
  inference rules for $\odot$ are derived as follows:
\[
\infer[\FL]{\seq{\Gamma^*,z:\F{x \odot y}{x:A^*,y:B^*},{\Gamma'}^*}{\vars{\Gamma}\odot z \odot \vars{\Gamma'}}{C}}
      {\infer[Lemma~\ref{lem:exchange}]
        {\seq{\Gamma^*,{\Gamma'}^*,x:A,y:B}{\vars{\Gamma}\odot x \odot y \odot \vars{\Gamma'}}{C}}
        {\seq{\Gamma^*,x:A,y:B,{\Gamma'}^*}{\vars{\Gamma}\odot x \odot y \odot \vars{\Gamma'}}{C}}}
\]

\[
\infer{\seq{\Gamma^*,\Delta^*}{\vars{\Gamma} \odot \vars{\Delta}}{\F{x \odot y}{x:A,y:B}}}
      {{\vars{\Gamma} \odot \vars{\Delta}} \spr (x \odot y)[\vars{\Gamma}/x,\vars{\Delta}/y]
        \infer[Lemma~\ref{lem:weakening}]
              {\seql{\Gamma^*,\Delta^*}{\vars{\Gamma}}{A}}
              {\seql{\Gamma^*}{\vars{\Gamma}}{A}} &
        \infer[Lemma~\ref{lem:weakening}]
              {\seql{\Gamma^*,\Delta^*}{\vars{\Gamma}}{A}}
              {\seql{\Delta^*}{\vars{\Delta}}{B}}}
\]

Identity and cut are

\[
\infer[Thm~\ref{thm:identity}]
      {\seq{x:A^*}{x}{A}}
      {}
\qquad
\infer[Thm~\ref{thm:cut}]
      {\seq{{\Gamma}^*,{\Delta}^*,{\Gamma'}^*}{\vars{\Gamma}\odot \vars{\Delta} \odot \vars{\Gamma'}}{C}}
      {\infer[Lem~\ref{lem:weakening}]
        {\seq{\Gamma^*,\Delta^*,x:A^*,{\Gamma'}^*}{\vars{\Gamma}\odot x \odot \vars{\Gamma'}}{C}}
        {\seq{\Gamma^*,x:A^*,{\Gamma'}^*}{\vars{\Gamma}\odot x \odot \vars{\Gamma'}}{C}} &
        \infer[Lem~\ref{lem:weakening}]{\seq{\Gamma^*,\Delta^*,x:A^*,{\Gamma'}^*}{\vars{\Delta}}{A^*}}
             {{\seq{\Delta^*}{\vars{\Delta}}{A^*}}}}
\]

Since we do not notate weakening and exchange, we can summarize these
as:
\[
\begin{array}{rcl}
(\dotLd{z}{x,y.d})^* & := & \FLd{z}{x,y.d^*}\\
(\dotRd{d_1}{d_2})^* & := & \FRd{}{1}{(d_1^*/x,d_2^*/y)}\\
x^* & := & x\\
(\Cut{e}{d}{x})^* & := & \Cut{e^*}{d^*}{x}
\end{array}
\]

\item The $\beta\eta$ axioms for $\odot$ translate almost exactly to the
  corresponding axioms for $\F{x \odot y}{x:A^*,y:B^*}$: for $\beta$, we
  also use the fact that $\Trd{1}{-}$ is the identity.  Associativity
  and unit for cut are modeled by the corresponding rules.  Reflexivity,
  symmetry, transitivity, and congruence (e.g. if $d \deq d'$ then
  $\dotLd{z}{x,y.d} \deq \dotLd{z}{x,y,d'}$) are translated to the
  corresponding rules in the framework.

\item For the back-translation on normal derivations, suppose we have a
  normal derivation of \seq{\Gamma^*}{\vars{\Gamma}}{A^*}.  The only
  possible rules are hypothesis and the \Fsymb-rules.

\begin{itemize}
\item For identity
\[
\infer{\seq{\Gamma^*}{\vars{\Gamma}}{P}}
      {{\vars{\Gamma^*}} \spr {x} &
        x : P \in \Gamma^*}
\]
Because the only structural transformation axioms are equalities for
associativity and unit, we have ${\vars{\Gamma^*}} \deq {x}$, which in
turn implies that $\Gamma$ is $x:Q$ for some $Q$ (because if $\Gamma$ is
empty, does not contain $x$, or contains anything else, \vars{\Gamma}
will not equal $x$).  By definition, this implies $Q = P$, so $\Gamma$
is $x:P$.  Therefore the identity rule applies.

\item For \FR, because the only type that encodes to \Fsymb is $\odot$,
  we have
\[
\infer{\seq{\Gamma^*}{\vars{\Gamma}}{\F{x \otimes y}{x:A_1^*,y:A_2^*}}}
      {\vars{\Gamma} \deq \alpha_1 \odot \alpha_2 &
       \seq{\Gamma^*}{\alpha_1}{A_1^*} &
       \seq{\Gamma^*}{\alpha_2}{A_2^*}
      }
\]
By properties of the mode theory, $\Gamma = \Gamma_1,\Gamma_2$ with
$\vars{\Gamma_i} \deq \alpha_i$, so we have derivations of
\seq{\Gamma^*}{\vars{\Gamma_i}}{A_i^*}.  Because 0-use strengthening
applies, we can strengthen these to
\seq{\Gamma_i^*}{\vars{\Gamma_i}}{A_i^*}.  Then the inductive hypothesis
gives \seql{\Gamma_i}{A_i}, so applying the $\odot$ right rule gives the
result.

\item For \FL, because the only type encoding to $\Fsymb$ is $A \odot
  B$, we have
\[
\infer{\seq{\Gamma^*,z:\F{x \odot y}{x:A^*,y:B^*},{\Gamma'}^*}{\vars{\Gamma} \otimes z \otimes \vars{\Gamma'}}{C^*}}
      {\seq{\Gamma^*,{\Gamma'}^*,x:A^*,y:B^*}{\vars{\Gamma} \otimes (x \otimes y) \otimes \vars{\Gamma'}}{C^*}}
\]
By exchange (Lemma~\ref{lem:exchange}), we have a no-bigger derivation
of
{\seq{\Gamma^*,x:A^*,y:B^*,{\Gamma'}^*}{\vars{\Gamma} \otimes (x \otimes y) \otimes \vars{\Gamma'}}{C^*}} 
so applying the IH gives 
\seql{{\Gamma,x:A,y:B,{\Gamma'}}}{o}{C}, and then $\odot$-left gives the result.

\item We extend this to the internalized cut, identity, and
  transformation rules as follows.  For a general identity
\[
\infer{\seq{\Gamma^*}{x}{A^*}}
      {x : A^* \in \Gamma^*}
\]
If $\vars{\Gamma}$ is $x$, then $\Gamma$ is $x:A$, so the identity rule
applies.  

\item For a general action of a structural transformation
\[
\infer{\seq{\Gamma^*}{\vars{\Gamma}}{A}}
      {\vars{\Gamma} \spr \beta & 
        \seq{\Gamma^*}{\beta}{A}}
\]
Because the only structural transformation axioms are equalities for
associativity and unit, we have ${\vars{\Gamma^*}} \deq {\beta}$, so we
can translate the premise.  

\item For a cut 
\[
\infer{\seq{\Gamma^*}{\vars{\Gamma}}{B^*}}
      {\seq{\Gamma^*}{\alpha}{A_0} &
       \seq{\Gamma^*,x:A_0}{\beta}{B^*} &
       \vars{\Gamma} \deq { \beta[\alpha/x]}
      }
\]
There are two issues to address.  The first is that $A_0$ is a priori
not in the image of the translation, because the cut rule does not have
the subformula property.  However, we argued above that every type in
this restricted fragment comes from a source type, so we can without
loss of generality assume $A_0 = A^*$.  Moreover, because $\beta$ is a
tuple of variables, one of the following two cases applies:
\begin{itemize}
\item $x$ occurs once, in which case there exist
  $\Gamma_1,\Gamma_2,\Delta$ such that $\beta = \vars{\Gamma_1} \odot x
  \odot \vars{\Gamma_2}$, $\alpha = \vars{\Delta}$, and $\Gamma =
  \Gamma_1,\Delta,\Gamma_2$.
  
  In this case, we can strengthen $\Delta^*$ from the left-hand premise,
  $(\Gamma_1,\Gamma_2)^*$ from the right, and apply the inductive
  hypotheses to get $\seql{\Gamma_1,x:A,\Gamma_2}{o}{B}$ and
  $\seql{\Delta}{o}{A}$, and then cut in the source.

\item $x$ does not occur in $\beta$, so $\vars{\Gamma} = \beta$.  In
  this case, we can strengthen $x$ from the premise, and inductively
  translate it.

\item $x$ occurs more than once in $\beta$, in which case $\alpha$ must
  be empty, or else any variable in it would occur more than once in
  $\vars{\Gamma}$.  FIXME: what now? contract? generalize?

\end{itemize}

\end{itemize}

That is,
\[
\begin{array}{rcl}
\backtrf{\Trd{1}{x}} & := & x\\
\backtrf{\FRd{}{1}{e_1/x,e_2/y}} & := & \dotRd{\backtrf{\str{}{e_1}}}{\backtrf{\str{}{e_2}}}\\
\backtrf{\FLd{z}{x,y.e}} & := & \dotLd{z}{x,y.\backtrf{e}}\\
\backtrf{x} & := & x\\
\backtrf{\Trd{1}{d}} & := & \backtrf{d} \\
\backtrf{\Cut{d}{e}{x}} & := \backtrf{\str{x}{e}} \text{ if } x \# \beta \\
\backtrf{\Cut{d}{e}{x}} & := \Cut{\backtrf{\str{\Delta}{e}}}{\backtrf{\str{\Gamma_1,\Gamma_2}{d}}}{x} \text{ if } x \in \beta\\
\end{array}
\]
where $\str{x}{e_i}$ is the result of Lemma~\ref{lem:0-use-strengthening}.  

\item Next, we show that for $e : \seq{\Gamma^*}{\vars{\Gamma}}{A^*}$,
  ${\backtrf{e}}^* \deq e$.

In the case for the hypothesis rule for atoms, we have
\[
{\backtrf{\Trd{1}{x}}}^* = x^* = x \deq \Trd{1}{x}
\]

In the case for \FL, we have 
\[
(\backtrf{\FLd{z}{x,y.e}})^* = (\dotLd{z}{x,y.\backtrf{e}})^* =
{\FLd{z}{x,y.{\backtrf{e}}^*}}
\]
so the result follows from the inductive hypothesis.  

In the case for \FR, we have
\[
(\backtrf{\FRd{}{1}{e_1/x,e_2/y}})^* = (\dotRd{\backtrf{\str{}{e_1}}}{\backtrf{\str{}{e_2}}})^* =
{\FRd{}{1}{({\backtrf{{\str{}{e_1}}}}^*/x,{\backtrf{\str{}{e_2}}}^*/y)}}
\]
By the inductive hypothesis, we have 
${\backtrf{\str{}{e_i}}}^* \deq \str{}{e_i}$, but we have $\str{}{e_i} \deq e_i$ by
Lemma~\ref{lem:0-use-strengthening}.  

FIXME: identity and cut

\item Next, we show that $\backtrf{{\elim{d^*}}} \deq d$.

The proof is by induction on $d$.  

\begin{itemize}
\item In the case for $\dotLd{}{}$, expanding definitions, we have
\[
\backtrf{\elim{(\dotLd{z}{x,y.d})^*}} \deq
\backtrf{\elim{\FLd{z}{x,y.d^*}}} \deq 
\backtrf{\FLd{z}{x,y.\elim{d^*}}} \deq
{\dotLd{z}{x,y.\backtrf{\elim{d^*}}}}
\]
so the inductive hypothesis gives the result.  

\item In the case for $\dotRd{}{}$, we have
\[
\begin{array}{ll}
& \backtrf{\elim{(\dotRd{d_1}{d_2})^*}} \\
\deq & \backtrf{\elim{\FRd{}{1}{(d_1^*/x,d_2^*/y)}}} \\
\deq & \backtrf{\FRd{}{1}{(\elim{d_1^*}/x,\elim{d_2^*}/y)}} \\
\deq & \dotRd{\backtrf{\str{}{\elim{d_1^*}}}/x}{\backtrf{\str{}{\elim{d_2^*}}}/y}
\end{array}
\]
Note that the forward translation weakens $d_i^*$ when it constructs 
${\FRd{}{1}{(d_1^*/x,d_2^*/y)}}$, and cut elimination does not introduce
left rules on variables that are not case-analyzed somewhere in the
proof by Lemma~\ref{lem:cut-doesnt-intro}.  
So by Lemma~\ref{lem:strengthening-eq-properties}, strengthening
$\str{}{\elim{d_1}^*}$ will simply undo the weakening done in
constructing the term, and \backtrf{\str{}{\elim{d_1^*}}} equals
\backtrf{{\elim{d_1^*}}}.  Therefore the inductive hypotheses give the
result.

\item In the case for a variable, we have
\[
\backtrf{\elim{x^*}} \deq \backtrf{\elim{x}} \deq \backtrf{\Identa{x}}
\]
so the above part gives the result.  

\item In the case for cut, we have
\[
\backtrf{\elim{(\Cut{e}{d}{x})^*}} \deq \backtrf{\elim{\Cut{e^*}{d^*}{x}}}
\deq \backtrf{(\Cuta{\elim{e^*}}{\elim{d^*}}{x})}
\]
By the previous part, this is
\Cut{\backtrf{\elim{e^*}}}{\backtrf{\elim{d^*}}}{x} (the strengthening
only undoes the weakening that the forward translation puts in), so the
inductive hypothesis gives the result.
\end{itemize}

%% \item For normal derivations $e,e' :
%%   \seq{\Gamma^*}{\vars{\Gamma}}{A^*}$, if $e \deqp e'$, then
%%   $\backtrf{e} \deq \backtrf{e'}$.

%% We generalize slightly and prove that for 
%% $e,e' : \seq{\Gamma^*,\Delta^*}{\vars{\Gamma}}{A^*}$, 
%% if $e \deqp e'$, then $\backtrf{\str{\Delta}{e}} \deq \backtrf{\str{\Delta}{e'}}$.

%% \begin{itemize}

%% \item The cases for reflexivity, symmetry, and transitivity follow by
%%   induction, using the the correponding rules of source-language
%%   equalty.

%% \item The cases for compatibility all have a similar structure.  For
%%   example, suppose we have $\FRd{}{1}{e_1,e_2} \deqp
%%   \FRd{}{1}{e_1',e_2}$ because $e_1 \deqp e_1'$.
%%   Then by the inductive
%%   hypothesis we get $\backtrf{\str{\Delta,\Gamma_2}{e_1}} \deq
%%   \backtrf{\str{\Delta,\Gamma_2}{e_1'}}$.
%%   We need to show that $\backtrf{\str{\Delta}{\FRd{}{1}{e_1,e_2}}} \deq 
%%   \backtrf{\str{\Delta}{\FRd{}{1}{e_1,e_2}}}$.  By expanding definitions
%%   on both sides, it suffices to show
%%   \[
%%     \backtrf{\FRd{}{1}{\str{\Delta}{e_1},\str{\Delta}{e_2}}}
%%     \deq 
%%     \backtrf{\FRd{}{1}{\str{\Delta}{e_1'},\str{\Delta}{e_2}}}
%%   \]
%%   and so
%%   \[
%%     \dotRd{\backtrf{\str{\Gamma_2}{\str{\Delta}{e_1}}}}{\backtrf{\str{\Gamma_1}{\str{\Delta}{e_2}}}}
%%     \deq 
%%     \dotRd{\backtrf{\str{\Gamma_2}{\str{\Delta}{e_1'}}}}{\backtrf{\str{\Gamma_1}{\str{\Delta}{e_2}}}}
%%   \]
%%   By Lemma~\ref{lem:strengthening-eq-properties}, ${\str{\Gamma_2}{\str{\Delta}{e_1'}}} =
%%   {\str{\Gamma_2,\Delta}{e_1'}}$, so the inductive hypothesis gives the
%%   result.

%% \item For the axiom

%% \[
%% \FRd{}{s}{\D_1/x_1,\ldots,\Trd{{s_i}}{\D_i}/x_i,\ldots} \deqp \FRd{}{s;(1_{\alpha[\alpha_1/x_1,\ldots,\alpha_{i-1}/x_{i-1},\ldots]}[s_i/x_i])}{\vec{\D_j/x_j}}\\
%% \]
%% since there are only identity structural transformations in this mode
%% theory, $e$ and $e'$ must be syntactically identical terms.

%% \item For 
%% \[
%% \D \deqp \FLd{x}{y,z.\Linv{\D}{(y,z)}{x}}
%% \]
%% we need to show that
%% \[
%% \backtrf{\str{\Delta}{d}}
%% \deq 
%% \backtrf{\str{\Delta}{\FLd{x}{y,z.\Linv{\D}{(y,z)}{x}}}}
%% \]

%% We distinguish cases on whether $x \in \Delta$ (so the strengthening
%% deletes the \FL) or not.

%% If it does, we need to show that 
%% \[
%% \backtrf{\str{\Delta}{d}}
%% \deq 
%% \backtrf{\str{\Delta,x,y,z}{\Linv{\D}{(y,z)}{x}}}
%% \]
%% This follows from Lemma~\ref{lem:strengthening-eq-properties}.  

%% If it is not, we need to show that 
%% \[
%% \backtrf{\str{\Delta}{d}}
%% \deq 
%% \dotLd{x}{y,z.\backtrf{\str{\Delta}{\Linv{\D}{(y,z)}{x}}}}
%% \]
%% and by $\eta$-expanding the left-hand side gives
%% $\dotLd{x}{y,z.\Cut{\backtrf{\str{\Delta}{d}}}{\dotRd{y}{z}}{x}}$.
%% But in this case 
%% $\str{\Delta}{\Linv{\D}{(y,z)}{x}} = \Linv{\str{\Delta}{d}}{(y,z)}{x}$,
%% so the fact that back-translation preserves left inversion gives the
%% result.  

\end{enumerate}

Our framework builds on many approaches to substructural and modal logic
in the literature.  Logical rules that act at a leaf of a
tree-structured context go back to the Lambek
calculus~\citep{lambek58calculus}.  A rich collection of context
structures that correspond to type constructors plays a central role in
display logic~\citep{belnap82display}.  \citet{atkey04separation}'s
$\lambda$-calculus for resource separation is similar to mode theories
with one mode, where there is at most one 2-cell between a given pair of
1-cells; at the logical level, our calculus is a unification of this
with the multimode adjoint logic of
\citet{reed09adjoint}.  Algebraic resource annotations on variables are
used to track modalities in Agda's implementation~\citep{abel15modal}
and in \citet{mcbride16nuttin}'s approach to linear dependent types.  LF
representations of modal or substructural logics work by restricting the
use of cartesian variables~\citep{crary10substructural}.  Relative to
all of these approaches, we believe that the analysis of the context
structures/resources as a \emph{term} in a base type theory, and the
fibrational structure of the derivations over them, is a new and useful
observation.  For example, rather than needing extra-logical conditions
on proof rules to ensure cut admissibility, as in display logic, the
conditions are encoded in the language of context descriptors and the
definition of types from them.  Moreover, none of these existing
approaches allow for proof-relevant 2-cells/structural rules, and their
presence (and the equational theory we give for them) is important for
our applications to extensions of homotopy type theory.  A point of
contrast with substructural logical
frameworks~\citep{cervesatopfenning02llf,watkins+03clf-tr,reed09thesis}
is that logics are ``embedded'' in our calculus (giving a type
translation such that provability in the object logic corresponds to
provability in ours), rather than ``encoding'' the structure of
derivations.  This way, we obtain cut elimination for object languages
as a corollary of framework cut elimination.

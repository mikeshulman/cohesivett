
\section{Syntactic Properties}
\label{sec:synprop-long}

%% Here, we include proofs of the results from
%% Section~\ref{sec:synprop-short}.  

\subsection{Admissible Structural Rules}
We show that identity, cut, weakening, exchange, contraction, and
respect for transformations, are admissible.  We give the cases for the
rules in Figure~\ref{fig:sequent}, though the results readily extend to
additive sums and products.

Define the \emph{size} of a derivation of \seq{\Gamma}{\alpha}{A} or
\seq{\Gamma}{\gamma}{\Delta} to be the number of inference rules for
these judgements $(\dsd{v},\FL, \FR, \UL, \UR, \cdot, \_,\_)$ used in it
(i.e., the evidence that variables are in a context and the evidence for
structural transformations do not contribute to the size).  Sizes are
necessary for the cut proof, where we sometimes weaken or invert a
derivation before applying the inductive hypothesis.

\begin{lemma}[Respect for Transformations] ~ \label{lem:respectspr}
\begin{enumerate}
\item If \seq{\Gamma}{\beta}{A} and $\beta' \spr \beta$ then
  \seq{\Gamma}{\beta'}{A}, and the resulting derivation has the same
  size as the given one.
\item If \seq{\Gamma}{\gamma}{\Delta} and $\gamma' \spr \gamma$ then
  \seq{\Gamma}{\gamma'}{\Delta}, and the resulting derivation has the
  same size as the given one.
\end{enumerate}
\end{lemma}
\begin{proof}
Mutual induction on the given derivation.  The cases for \dsd{v} and
$\FR$ and $\UL$ are immediate (with no use of the inductive hypothesis)
by composing with the equality in the premise of the rule.  This does
not change the size of the derivation because the derivations of
structural transformations are ignored by the size.  The cases for $\FL$ and
$\UR$ use the inductive hypothesis, along with congruence for structural
transformations to show that $\subst{\beta}{\alpha}{x} \spr
\subst{\beta'}{\alpha}{x}$ or $\subst{\alpha}{\beta}{x} \spr
\subst{\alpha}{\beta'}{x}$.  The cases for substitutions rely on the
fact that no generating structural transformations for mode substitutions are
allowed, so if $\gamma' \spr \cdot$ then $\gamma'$ is literally $\cdot$,
and $(-,-)$ is injective (if $\gamma' \spr (\gamma_1,\alpha_2/x)$, then
$\gamma'$ is $(\gamma_1',\alpha_2'/x)$ with $\gamma_1' \spr \gamma_1$
and $\alpha_2' \spr \alpha_2$); this is enough to use the inductive
hypotheses in the cons case.
\end{proof}

\begin{lemma}[Weakening over weakening] ~ \label{lem:weakening} ~
\begin{enumerate}
\item If \seq{\Gamma,\Gamma'}{\alpha}{C} then
\seq{\Gamma,\tptm{z}{A},\Gamma'}{\alpha}{C}, and the resulting
derivation has the same size as the given one.  
\item If \seq{\Gamma,\Gamma'}{\gamma}{\Delta} then
\seq{\Gamma,\tptm{z}{A},\Gamma'}{\gamma}{\Delta}, and the resulting
derivation has the same size as the given one.  
\item If \seq{\Gamma,\Gamma''}{\alpha}{C} then
\seq{\Gamma,\Gamma',\Gamma''}{\alpha}{C}, and the resulting
derivation has the same size as the given one.  
\end{enumerate}
\end{lemma}
\begin{proof}
It is implicit that the mode morphism $\alpha$ is weakened with $z$ in
the conclusion.  Intuitively, weakening holds because the contexts
$\Gamma$ are treated like ordinary structural contexts in all of the
rules---they are fully general in every conclusion, and the premises
check membership or extend them---and because weakening holds for mode
morphisms and equalities of mode morphisms.  Formally, the first two
parts are proved by mutual induction; each case is either immediate
or follows from weakening for the mode morphisms, weakening for
transformations, and the inductive hypotheses.  The third
part is proved by induction over $\Gamma'$, repeatedly applying the
first part.  
%% The case for the hypothesis rule is immediate, because
%% $\Gamma$ may contain variables other than $x$.  The case for
%% \Fsymb-right follows from weakening for the mode morphisms, and
%% equations between mode morphisms, and the inductive hypothesis for
%% substitutions.  The case for \Fsymb-left follows from the inductive
%% hypothesis, as does the case for \Usymb-right.  
\end{proof}

\begin{lemma}[Exchange over exchange] \label{lem:exchange}
If \seq{\Gamma,x:A,y:B,\Gamma'}{\alpha}{C} then
\seq{\Gamma,y:B,x:A,\Gamma'}{\alpha}{C}, and the resulting derivation
has the same size as the given one.  (And similarly for substitutions,
and exchange can be iterated).  
\end{lemma}
\begin{proof} Analogous to weakening.  
\end{proof}

We sometimes write $\modeof{\Gamma}$ for the $\psi$ such that
\wfctx{\Gamma}{\psi} and similarly for $\modeof{A}$.

\begin{theorem}[Identity] ~ \label{thm:identity}
\begin{enumerate}
\item If $x:A \in \Gamma$ then $\seq{\Gamma}{x}{A}$.
\item If $\oftp{\modeof{\Gamma}}{\rho}{\modeof{\Delta}}$ is a
  variable-for-variable mode substitution such that $x:A \in \Delta$
  implies $\rho(x) : A \in \Gamma$, then $\seq{\Gamma}{\rho}{\Delta}$.
\end{enumerate}
\end{theorem}

\begin{proof}
The standard proof by induction on $A$ (mutually with $\Delta$) applies:
the case for atomic propositions is a rule, and for the other
connectives, apply the invertible and then non-invertible rule to reduce
the problem to the inductive hypotheses.  More specifically, identity
for $P$ is a rule.  In the case for \F{\alpha}{\Delta}, with $\Gamma =
\Gamma_1,x:\F{\alpha}{\Delta},\Gamma_2$, we reduce it to the inductive
hypothesis as follows:
\[
\infer[\FL]{\seq{\Gamma_1,x:\F{\alpha}{\Delta},\Gamma_2}{x}{\F{\alpha}{\Delta}}}
      {\infer[\FR]{\seq{\Gamma_1,\Gamma_2,\Delta}{\alpha}{\F{\alpha}{\Delta}}}
                        {\alpha \spr \tsubst{\alpha}{\vec{x/x}} &
                        \seq{\Gamma_1,\Gamma_2,\Delta}{\vec{x/x}}{\Delta}
                        }}
\]
In the second premise, the $\vec{x/x}$ substitution for each $x \in
\Delta$ is a variable-for-variable substitution, so the second part of
the inductive hypothesis applies.  
The case for \Usymb\/ is similar
\[
\infer[\UR]{\seq{\Gamma}{x}{\U{\alpha}{\Delta}{A}}}
      {\infer[\UL]{\seq{\Gamma,\Delta}{\alpha}{A}}
                        {\alpha \spr \subst{x}{\tsubst{\alpha}{\vec{x/x}}}{x} &
                        \seq{\Gamma,\Delta}{\vec{x/x}}{\Delta} &
                        \seq{\Gamma,x:A}{x}{A}
                        }}
\]

For the second part, the hypothesis of the lemma asks that every
variable in $\Delta$ is associated by $\rho$ with a variable of the same
type in $\Gamma$; this is enough to iterate the first part of the
lemma for each position in $\Delta$.  Specifically, the case where
$\Delta$ is the empty context $\cdot$ is a rule. In the case for a cons
$\Delta,y:A$, we have
\oftp{\modeof{\Gamma}}{\rho}{(\modeof{\Delta},y:\modeof{A})} which means
$\rho$ must be of the form $\rho',x/y$ where $x \in \modeof{\Gamma}$ and
$\rho'$ is a variable-for-variable substitution.  Because $\rho$ was
type-preserving, $x : A \in \Gamma$ and $\rho'$ is type-preserving, so
we obtain the result from the inductive hypotheses as follows:
\[
\infer{\seq{\Gamma}{\rho,x/y}{\Delta,y:A}}
      {\seq{\Gamma}{\rho}{\Delta} & 
       \seq{\Gamma}{x}{A}
      }
\]
\end{proof}

\begin{lemma}[Left-invertibility of \Fsymb] \label{lem:Finv}
If $\D :: \seq{\Gamma_1,x_0:\F{\alpha_0}{\Delta_0},\Gamma_2}{\beta}{C}$
then there is a derivation $\D' ::
\seq{\Gamma_1,\Gamma_2,\Delta_0}{\subst{\beta}{\alpha_0}{x_0}}{C}$ and
$size(\D') \le size(\D)$ (and analogously for substitutions).
\end{lemma}

\begin{proof}
Intuitively, we find all of the places where $\D$ ``splits'' $x_0$, delete
the $\FL$ used to do the split, and reroute the variables to the ones in
the context of the result.  

Formally, we proceed by induction on \D.  We write $\Gamma$ for the
whole context $\Gamma_1,x_0:\F{\alpha_0}{\Delta_0},\Gamma_2$.

In the case for \dsd{v}, $x : P \in
\Gamma_1,x_0:\F{\alpha_0}{\Delta_0},\Gamma_2$ cannot be equal to $x_0 :
\F{\alpha_0}{\Delta_0}$ because the types conflict, so we can reapply
the \dsd{v} rule in $\Gamma_1,\Gamma_2,\Delta$.

In the case for $\FR$, we have
\[
\infer{\seq{\Gamma}{\beta}{\F{\alpha}{\Delta}}}
      {\beta \spr \tsubst{\alpha}{\gamma} &
        \seq{\Gamma}{\gamma}{\Delta} 
      }
\]
with $x_0 : \F{\alpha_0}{\Delta_0} \in \Gamma$.  By the inductive
hypothesis we get
\seq{\Gamma_1,\Gamma_2,\Delta_0}{\subst{\gamma}{\alpha_0}{x}}{\Delta}.  Because
$x_0$ is not free in $\alpha$,
$\subst{(\tsubst{\alpha}{\gamma})}{\alpha_0}{x_0} =
\tsubst{\alpha}{\subst{\gamma}{\alpha_0}{x_0}}$, so we can reapply \FR:
\[
\infer{\seq{\Gamma_1,\Gamma_2}{\subst{\beta}{\alpha_0}{x_0}}{\F{\alpha}{\Delta}}}
      {{\subst{\beta}{\alpha_0}{x_0}} \spr \tsubst{\alpha}{\subst{\gamma}{\alpha_0}{x_0}} &
        \seq{\Gamma_1,\Gamma_2,\Delta_0}{\subst{\gamma}{\alpha_0}{x}}{\Delta}
      }
\]
Both the input and the output have size 1 more than the size of their
subderivations, and the output subderivation is no bigger than the input
by the inductive hypothesis.

In the case for $\FL$
\[
\infer[\FL]{\seq{\Gamma_1',x:\F{\alpha}{\Delta},\Gamma_2'}{\beta}{C}}
      {\deduce{\seq{\Gamma_1',\Gamma_2',\Delta}{\subst \beta {\alpha}{x}}{C}}{\D}}
\]
with $\Gamma_1,x_0 : \F{\alpha_0}{\Delta_0},\Gamma_2 =
\Gamma_1',x:\F{\alpha}{\Delta},\Gamma_2'$, we distinguish cases on
whether $x = x_0$ or not.  If they are the same (i.e. we have hit a left
rule on $x_0$), then $\alpha_0 = \alpha$ and $\Delta_0 = \Delta$ and
\D\/ is the result, and the size is 1 less than the size of the input.
If they are different, then (because $x_0$ is somewhere in
$\Gamma_1',\Gamma_2'$) by the inductive hypothesis we have a derivation
\[
\D' :: {\seq{(\Gamma_1',\Gamma_2')-x_0,\Delta,\Delta_0}{\subst{\subst \beta {\alpha}{x}}{\alpha_0}{x_0}}{C}}
\]
that is no bigger than \D.  Because $x_0$ is from $\Gamma$ and not
$\Delta$, it does not occur in $\alpha$, so 
\[
{\subst{\subst \beta {\alpha}{x}}{\alpha_0}{x_0}} = 
{\subst{\subst \beta {\alpha_0}{x_0}}{\alpha}{x}}
\]
By (iterating) exchange, we get a derivation 
\[
\D'' :: {\seq{(\Gamma_1',\Gamma_2')-x_0,\Delta_0,\Delta}{\subst{\subst \beta {\alpha_0}{x_0}}{\alpha}{x}}{C}}
\]
whose size is the same as $\D'$ and so no bigger than $\D$.  Applying
$\FL$ to $\D''$ (using the fact that
$(\Gamma_1',x:\F{\alpha}{\Delta},\Gamma_2')-x_0 = \Gamma_1,\Gamma_2$)
derives $\seq{\Gamma_1,\Gamma_2}{\subst{\beta}{\alpha_0}{x_0}}{C}$, and
the size is no bigger than the size of the input.

In the case for $\UR$,
\[
\infer{\seq{\Gamma}{\beta}{\U{x.\alpha}{\Delta}{A}}}
      {\seq{\Gamma,\Delta}{\subst{\alpha}{\beta}{x}}{A}}
\]
the inductive hypothesis gives a
$\D' :: \seq{\Gamma_1,\Gamma_2,\Delta,\Delta_0}{\subst{\subst{\alpha}{\beta}{x}}{\alpha_0}{x_0}}{A}$
and (iterated) exchange gives 
$\D'' ::
\seq{\Gamma_1,\Gamma_2,\Delta_0,\Delta}{\subst{\subst{\alpha}{\beta}{x}}{\alpha_0}{x_0}}{A}$,
both no bigger than \D.  Because $x_0$ is in $\Gamma$ and not $\Delta$,
it is not free in $\alpha$, so 
\[
{\subst{\subst{\alpha}{\beta}{x}}{\alpha_0}{x_0}} = {\subst{\alpha}{\subst{\beta}{\alpha_0}{x_0}}{x}}
\]
Thus, we can derive
\[
\infer{\seq{\Gamma_1,\Gamma_2,\Delta_0}{\subst{\beta}{\alpha_0}{x_0}}{\U{x.\alpha}{\Delta}{A}}}
      {\deduce{\seq{\Gamma_1,\Gamma_2,\Delta_0,\Delta}{\subst{\alpha}{\subst{\beta}{\alpha_0}{x_0}}{x}}{A}}{\D''}}
\]

In the case for $\UL$, 
\[
\infer{\seq{\Gamma}{\beta}{C}}
      {x:\U{x.\alpha}{\Delta}{A} \in \Gamma & 
        \beta \spr \subst{\beta'}{\tsubst{\alpha}{\gamma}}{z} &
        \seq{\Gamma}{\gamma}{\Delta} &
        \seq{\Gamma,\tptm{z}{A}}{\beta'}{C}
      }
\]
we know that $x$ is different than $x_0$ because the types conflict.
The inductive hypotheses give no-bigger derivations of
\[
\seq{\Gamma_1,\Gamma_2\Delta_0}{\subst{\gamma}{\alpha_0}{x_0}}{\Delta} \qquad \seq{\Gamma_1,\Gamma_2,\tptm{z}{A},\Delta_0}{\subst{\beta'}{\alpha_0}{x_0}}{C}
\]
and the latter can be exchanged to
\[
\seq{\Gamma_1,\Gamma_2,\Delta_0,\tptm{z}{A}}{\subst{\beta'}{\alpha_0}{x_0}}{C}
\]
again without increasing the size.  Thus, we can produce
\[
\infer{\seq{\Gamma_1,\Gamma_2,\Delta_0}{\subst{\beta}{\alpha_0}{x}}{C}}
      {\begin{array}{l}
          x:\U{x.\alpha}{\Delta}{A} \in \Gamma_1,\Gamma_2,\Delta_0 \\
          {\subst{\beta}{\alpha_0}{x}} \spr \subst{{\subst{\beta'}{\alpha_0}{x_0}}}{\tsubst{\alpha}{{\subst{\gamma}{\alpha_0}{x_0}}}}{z}\\
          \seq{\Gamma_1,\Gamma_2,\Delta_0}{\subst{\gamma}{\alpha_0}{x_0}}{\Delta} \\
          \seq{\Gamma_1,\Gamma_2,\Delta_0,\tptm{z}{A}}{\subst{\beta'}{\alpha_0}{x_0}}{C}
        \end{array}
      }
\]
where the transformation is the composition of the
\subst{-}{\alpha_0}{x_0} substitution into the given transformation, and
rearranging the substitution (note that $x_0$ does not occur in
$\alpha$):
\[
\begin{array}{ll}
\subst{\beta}{\alpha_0}{x_0} & \spr
\subst{\subst{\beta'}{\tsubst{\alpha}{\gamma}}{z}}{\alpha_0}{x_0} 
= 
\subst{\subst{\beta'}{\alpha_0}{x_0}}{\subst{\tsubst{\alpha}{\gamma}}{\alpha_0}{x_0}}{z}
\\
& =
\subst{\subst{\beta'}{\alpha_0}{x_0}}{\tsubst{\alpha}{\subst{\gamma}{\alpha_0}{x_0}}}{z} 
\end{array}
\]

The case for $\cdot$ is immediate.  The case for $\_,\_$ follows from
the two inductive hypotheses, because
$\subst{(\gamma,\alpha/x)}{\alpha_0}{x_0} =
{(\subst{\gamma}{\alpha_0}{x_0},\subst{\alpha}{\alpha_0}{x_0}/x)}$.
\end{proof}


\begin{theorem}[Cut] ~ \label{thm:cut}
\begin{enumerate} 
\item  If $\seq{\Gamma,\Gamma'}{\alpha_0}{A_0}$ and $\seq{\Gamma,x_0:A_0,\Gamma'}{\beta}{B}$ 
then $\seq{\Gamma,\Gamma'}{\beta[\alpha_0/x_0]}{B}$ 
\item If $\seq{\Gamma,\Gamma'}{\alpha_0}{A_0}$ and $\seq{\Gamma,x_0:A_0,\Gamma'}{\gamma}{\Delta}$ 
then $\seq{\Gamma,\Gamma'}{\gamma[\alpha_0/x_0]}{\Delta}$ 
\item If $\seq{\Gamma}{\gamma}{\Delta}$ and 
\seq{\Gamma,\Delta}{\beta}{C}
then \seq{\Gamma}{\tsubst{\beta}{\gamma}}{C}.  
\end{enumerate}
\end{theorem}

\begin{proof}
We write $\D$ for the derivation of $A_0$ and $\E$ for the derivation
from $A_0$.  The induction ordering is the usual one: First the cut
formula, and then the sizes of size of the two derivations.  More
specifically, any part can call another with a smaller cut formula
($A_0$ for part 1 and part 2, $\Delta$ for part 3).  Additionally, part
1 and part 2 call themselves and each other with the same cut formula
and smaller $\D$ or $\E$.

For part 1, there are 5 rules in the sequent calculus, so 25 pairs of
final rules.  We use the following case analysis on $\D/\E$ to cover
them all:
\begin{enumerate} 
\item any rule and \dsd{v}
\item \FR\/ and $\FL^{x_0}$ (principal)
\item \UR\/ and $\UR^{x_0}$ (principal)
\item any rule and \FR\/ (right-commutative)
\item any rule and \UR\/ (right-commutative)
\item any rule and $\FL^{x \neq x_0}$ (right-commutative)
\item any rule and $\UL^{x \neq x_0}$ (right-commutative)
\item $\FL$ and any rule (left-commutative)
\item $\UL$ and any rule (left-commutative)
\end{enumerate}
To see that this is exhaustive, cases 1 and 4-7 cover all pairs except
when $\E$ is a left rule on the cut variable $x_0$.  In these cases,
$\D$ must be either a left rule or a right rule for the cut formula (the
right rules for other types and identity do not have the appropriate
conclusion formula).  If $\D$ is a right rule, then it is a principal
cut; if it is a left rule, then the left-commutative cases apply.  

The left- and right-commutative cases overlap when $\D$ is a left-rule
and $\E$ is not a left rule on the cut variable.  We resolve this
arbitrarily, prioritizing right-commutative over left-commutative.

\begin{enumerate}
\item Any rule and variable
\[
\deduce{\seq{\Gamma,\Gamma'}{\alpha_0}{A_0}}{\D} 
\qquad
\infer{\seq{\Gamma,x:A,\Gamma'}{\beta}{Q}}
      {z:Q \in (\Gamma,x_0:A_0,\Gamma') &
        \deduce{\beta \spr z}{s}}
\]
There two subcases, depending on whether the cut variable is $z$ or not.
If $z$ is $x_0$ and $A_0$ is $Q$, then \D\/ derives
\seq{\Gamma,\Gamma'}{\alpha_0}{Q} and we want a derivation of
\seq{\Gamma,\Gamma'}{\subst{\beta}{\alpha_0}{z}}{Q}.  By congruence on
$s$, $\subst{\beta}{\alpha_0}{z} \spr \subst{z}{\alpha_0}{z}$, so
Lemma~\ref{lem:respectspr} gives the result.  If $z$ is not $x_0$,
then $z:Q \in \Gamma,\Gamma'$.  We want a derivation of
\seq{\Gamma,\Gamma'}{\subst{\beta}{\alpha_0}{x_0}}{Q}, and substituting
into $s$ gives $\subst{\beta}{\alpha_0}{x_0} \spr z$ (because $z \neq
x_0$), so we can derive
\[
\infer{\seq{\Gamma,\Gamma'}{\subst \beta {\alpha_0}{x_0}}{Q}}
      {z:Q \in (\Gamma,\Gamma') &
        {\subst{\beta}{\alpha_0}{x_0} \spr z}}
\]


\item $\FR$ and $\FL$ (principal)

\[
\infer{\seq{\Gamma,\Gamma'}{\alpha_0}{\F{\alpha}{\Delta}}}
      {  
        \deduce{\alpha_0 \spr \tsubst{\alpha}{\gamma}}{s} &
        \deduce{\seq{\Gamma,\Gamma'}{\gamma}{\Delta}}{\D}
      }
\qquad
\infer{\seq{\Gamma,x_0:\F{\alpha}{\Delta},\Gamma'}{\beta}{C}}
      {\deduce{\seq{\Gamma,\Gamma',\Delta}{\subst{\beta}{\alpha}{x_0}}{C}}
              {\E}}
\]
Using the inductive hypothesis part 3 to cut $\D$ and $\E$ ($\Delta$ is
a subformula of the original cut formula \F{\alpha}{\Delta}) gives
\[
\seq{\Gamma,\Gamma'}{\tsubst{\subst{\beta}{\alpha}{x_0}}{\gamma}}{C}
\]
By congruence on $s$ and because $\gamma$ substitutes only for
variables in $\modeof {\Delta}$,
\[
\subst{\beta}{\alpha_0}{x_0} \spr
{\subst{\beta}{\tsubst \alpha \gamma}{x_0}} =
{\tsubst{\subst{\beta}{\alpha}{x_0}}{\gamma}} 
\]
So applying Lemma~\ref{lem:respectspr} gives 
\seq{\Gamma,\Gamma'}{\subst{\beta}{\alpha_0}{x_0}}{C}.  

\item $\UR$ and $\UL$ (principal).  

We have
\[
\begin{array}{l}
\D \quad = \quad \infer{\seq{\Gamma,\Gamma'}{\alpha_0}{\U{x_0.\alpha}{\Delta}{A}}}
   {  
     \deduce{\seq{\Gamma,\Gamma',\Delta}{\subst \alpha {\alpha_0}{x_0}}{A}}{\D'}
   }
\\ \\
\infer{\seq{\Gamma,x_0:\U{x_0.\alpha}{\Delta}{A},\Gamma'}{\beta}{C}}
      {
        \begin{array}{l}
        s :: \beta \spr \subst{\beta'}{\tsubst{\alpha}{\gamma}}{z} \\
        {\E_1 :: \seq{\Gamma,x_0:{\U{x_0.\alpha}{\Delta}{A}},\Gamma'}{\gamma}{\Delta}}\\
        {\E_2 :: \seq{\Gamma,x_0:{\U{x_0.\alpha}{\Delta}{A}},\Gamma',\tptm{z}{A}}{\beta'}{C}}
        \end{array}
      }
\end{array}
\]
First, cutting the original \D\/ and the smaller $\E_1$ and $\E_2$ gives 
\[
\deduce{{\seq{\Gamma,\Gamma'}{\subst{\gamma}{\alpha_0}{x_0}}{\Delta}}}{\E_1'}
\qquad 
\deduce{{\seq{\Gamma,\Gamma',\tptm{z}{A}}{\subst{\beta'}{\alpha_0}{x_0}}{C}}}{\E_2'}
\]
Cutting $\E_1'$ \emph{into} $\D'$ (the derivations have switched places,
so are not necessarily smaller, but the cut formula $\Delta$ is a
subformula of $\U{x_0.\alpha}{\Delta}{A}$) gives
\[
\deduce
{\seq{\Gamma,\Gamma'}{\tsubst{\alpha}{\subst{\gamma}{\alpha_0}{x}}}{A}} {\D_1'}
\]
Cutting $\D_1'$ into $\E_2'$ gives 
\[
\seq{\Gamma,\Gamma'}{\subst{\subst{\beta'}{\alpha_0}{x_0}}{\subst{\alpha}{\alpha_0}{x_0}}{z}}{A}
\]
But by using $s$ and commuting substitutions we have 
\[
\subst{\beta}{\alpha_0}{x_0} \spr
\subst{(\subst{\beta'}{\tsubst{\alpha}{\gamma}}{z})}{\alpha_0}{x_0} = 
{\subst{\subst{\beta'}{\alpha_0}{x_0}}{\tsubst{\alpha}{\subst{\gamma}{\alpha_0}{x_0}}}{z}}
\]
so Lemma~\ref{lem:respectspr} gives the result.  


\item Any rule and $\FR$ (right-commutative)
\[
\deduce{\seq{\Gamma,\Gamma'}{\alpha_0}{A_0}}{\D} \qquad
\infer{\seq{\Gamma,x_0:A_0,\Gamma'}{\beta}{\F{\alpha}{\Delta}}}
      {\beta \spr \tsubst{\alpha}{\gamma} &
        \deduce{\seq{\Gamma,x_0:A_0,\Gamma'}{\gamma}{\Delta}}{\E}
      }
\]
By the inductive hypothesis, cutting into \D\/ into \E\/ gives
\seq{\Gamma,\Gamma'}{\subst{\gamma}{\alpha_0}{x_0}}{\Delta}.  By
congruence, $\subst{\beta}{\alpha_0}{x_0} \spr
\subst{\tsubst{\alpha}{\gamma}}{\alpha_0}{x_0}$.  Since $\gamma$ is a
total substitution for all variables in \modeof{\Delta},
$\subst{\tsubst{\alpha}{\gamma}}{\alpha_0}{x_0} =
\tsubst{\alpha}{\subst{\gamma}{\alpha_0}{x_0}}$, so
$\subst{\beta}{\alpha_0}{x_0} \spr
\tsubst{\alpha}{\subst{\gamma}{\alpha_0}{x_0}}$.  Thus we can reapply
the $\FR$ rule to get
\seq{\Gamma,\Gamma'}{\subst{\beta}{\alpha_0}{x_0}}{\F{\alpha}{\Delta}}.

\item Any rule and $\UR$ (right-commutative).    
\[
\deduce{\seq{\Gamma,\Gamma'}{\alpha_0}{A_0}}{\D} \qquad
\infer{\seq{\Gamma,x_0:A_0,\Gamma'}{\beta}{\U{x.\alpha}{\Delta}{A}}}
      {\deduce{\seq{\Gamma,x_0:A_0,\Gamma',\Delta}{\subst{\alpha}{\beta}{x}}{A}}{\E}}
\]
The inductive cut of \D\/ into \E\/ gives 
\[
\seq{\Gamma,\Gamma',\Delta}{\subst{\subst{\alpha}{\beta}{x}}{\alpha_0}{x_0}}{A}
\]
Because the variables from $\modeof{\Gamma},\modeof{\Gamma'}$ occur only
in $\beta$, not in $\alpha$, this substitution equals 
{\subst{\alpha}{\subst{\beta}{\alpha_0}{x_0}}{x}} so reapplying the
$\UR$ rule
derives 
{\seq{\Gamma,\Gamma'}{\subst{\beta}{\alpha_0}{x_0}}{\U{x.\alpha}{\Delta}{A}}}.   

\item Any rule and $\FL^{x\neq x_0}$ (right commutative)

If the left rule is not on the cut variable, then we have
\[
\deduce{\seq{\Gamma,\Gamma'}{\alpha_0}{A_0}}{\D}
\quad
\infer{\seq{\Gamma,x_0:A_0,\Gamma'}{\beta}{C}}
      { x : \F{\alpha}{\Delta} \in \Gamma,\Gamma' &
        \deduce{\seq{((\Gamma,x_0:A_0,\Gamma')-x),\Delta}{\subst{\beta}{\alpha}{x}}{C}}{\E}}
\]

We are going to commute $\D$ under \FL\/ on $x$, so need to reroute uses
of $x$ to the bottom by the left-inversion lemma, which gives
\[
\D' :: {\seq{((\Gamma,\Gamma')-x),\Delta}{\subst{\alpha_0}{\alpha}{x}}{A_0}}
\]
and $\D'$ is no bigger than \D.

Cutting $\D'$ and $\E$ by the inductive hypothesis gives
\[
\seq{((\Gamma,\Gamma')-x),\Delta}{\subst{\subst{\beta}{\alpha}{x}}{\subst{\alpha_0}{\alpha}{x}}{x_0}}{C}
\]
Because $x_0$ is not free in $\alpha$, 
\[
  {\subst{\subst{\beta}{\alpha}{x}}{\subst{\alpha_0}{\alpha}{x}}{x_0}}
= {\subst{\subst{\beta}{\alpha_0}{x_0}}{\alpha}{x}}
\]
so we can apply \FL
\[
\infer{\seq{\Gamma,\Gamma'}{\subst{\beta}{\alpha_0}{x_0}}{C}}
      {\seq{(\Gamma,\Gamma'-x)}{\subst{\subst{\beta}{\alpha_0}{x_0}}{\alpha}{x}}{C}}
\]
%% \seq{((\Gamma,\Gamma')-x),\Delta}{\subst{\subst{\beta}{\alpha}{x}}{\alpha_0}{x_0}}{C}
%% \]

\item Any rule and $\UL^{x\neq{x_0}}$ (right commutative)
\[
\deduce{\seq{\Gamma,\Gamma'}{\alpha_0}{A_0}}
       {
         \D
       }
\quad
\infer{\seq{\Gamma,x_0:A_0,\Gamma'}{\beta}{C}}
      {
        \begin{array}{l}
          x : \U{x.\alpha}{\Delta}{A} \in \Gamma,\Gamma' \\
          \beta \spr \subst{\beta'}{\tsubst{\alpha}{\gamma}}{z} \\
          \seq{\Gamma,x_0:A_0,\Gamma'}{\gamma}{\Delta} \\
          \seq{\Gamma,x_0:A_0,\Gamma',\tptm{z}{A}}{\beta'}{C}
        \end{array}
      }
\]

By the inductive hypotheses we get 
\[
\seq{\Gamma,\Gamma'}{\subst{\gamma}{\alpha_0}{x_0}}{\Delta}
\qquad
\seq{\Gamma,\Gamma',z:A}{\subst{\beta'}{\alpha_0}{x_0}}{C}
\]
so we can derive
\[
\infer{\seq{\Gamma,x_0:A_0,\Gamma'}{\subst{\beta}{\alpha_0}{x_0}}{C}}
      {
        \begin{array}{l}
          x : \U{x.\alpha}{\Delta}{A} \in \Gamma,\Gamma' \\
          {\subst{\beta}{\alpha_0}{x_0}} \spr \subst{\subst{\beta'}{\alpha_0}{x_0}}{\tsubst{\alpha}{\subst{\gamma}{\alpha_0}{x_0}}}{z} \\
          \seq{\Gamma,\Gamma'}{\subst{\gamma}{\alpha_0}{x_0}}{\Delta} \\
          \seq{\Gamma,\Gamma',\tptm{z}{A}}{\subst{\beta'}{\alpha_0}{x_0}}{C}
        \end{array}
      }
\]
For the second premise, we get
\[
\subst{\beta}{\alpha_0}{x_0} \spr
\subst{\subst{\beta'}{\tsubst{\alpha}{\gamma}}{z}}{\alpha_0}{x_0}
\]
by congruence on the assumed transformation, and then commute substitutions.  


\item $\FL$ and any rule (left commutative).

There is one subtlety in this case.  The usual strategy for a left rule
against an arbitrary \E\/ is to push $\E$ into the ``continuation'' of
the left rule.  However, as discussed above, our left rule for \Fsymb\/
eagerly inverts \emph{all} occurrences of $x$, while $\E$ itself also has
$x$ in scope.  Thus, we use Lemma~\ref{lem:Finv} to pull the
left-inversion to the bottom of \E, and then push that into \D.  On
proof terms, this corresponds to making all references to $x$ in \E\/
instead refer to the results of the ``split'' at the bottom of $\D$.
%% This subtlety could be avoided by building contraction into $\FL$, as
%% discussed above.

Formally, we have
\[
\begin{array}{c}
\infer{\seq{\Gamma,\Gamma'}{\alpha_0}{A_0}}
      {{x}:{\F{\alpha}{\Delta}} \in \Gamma,\Gamma' &
        \deduce{\seq{((\Gamma,\Gamma')-x),\Delta}{\subst {\alpha_0} {\alpha}{x}}{A_0}}{\D}}
\\ \\
\deduce{\seq{\Gamma,x_0:A_0,\Gamma'}{\beta}{C}}{\E}
\end{array}
\]

By left invertibility on \E, we obtain (note that $x \neq x_0$ because
$x_0$ only in scope in \E\/, not \D) a derivation $\E'$ of
{\seq{(\Gamma,x:A_0,\Gamma')-x,\Delta}{\subst{\beta}{\alpha}{x}}{C}}
that is no bigger than $\E$.  Because the cut formula is the same, and
$\E'$ is no bigger than \E\/, and \D\/ is smaller than the given
derivation of $A_0$, we can apply the inductive hypothesis to cut $\D$
and $\E'$ to get
\[
{\seq{(\Gamma,\Gamma')-x,\Delta}{\subst{\subst{\beta}{\alpha}{x}}{\subst{\alpha_0}{\alpha}{x}}{x_0}}{C}}.
\]
Commuting substitutions gives
\[
{\subst{{\subst{\beta}{\alpha}{x}}}{\subst{\alpha_0}{\alpha}{x}}{x_0}} = \subst {\beta[\alpha_0/x_0]}{\alpha}{x}
\]
so we can reapply $\FL$ 
\[
\infer{\seq{\Gamma,\Gamma'}{\beta[\alpha_0/x_0]}{C}}
      {\seq{((\Gamma,\Gamma')-x),\Delta}{\subst {(\beta[\alpha_0/x_0])} {\alpha}{x}}{C}}
\]

\item $\UL$ and any rule (left commutative)
In this case, $x:\U{\alpha}{\Delta}{A} \in \Gamma,\Gamma'$ and
we have
\[
\begin{array}{c}
\infer{\seq{\Gamma,\Gamma'}{\alpha_0}{A_0}}
      {\deduce{\alpha_0 \spr \subst{\alpha_0'}{\tsubst{\alpha}{\gamma}}{z}}{s} &
       \deduce{\seq{\Gamma,\Gamma'}{\gamma}{\Delta}}{\D_1} &
       \deduce{\seq{\Gamma,\Gamma',z:A}{\alpha_0'}{A_0}}{\D_2}
      }
\\ \\
\deduce{\seq{\Gamma,x_0:A_0,\Gamma'}{\beta}{B}}{\E}
\end{array}
\]

Weakening \E\/ with $z$ and then cutting $\D_2$ and $\E$ by the inductive
hypothesis (which applies because $\D_2$ is smaller and weakening does
not change the size) gives
\[
\deduce{\seq{\Gamma,\Gamma',z:A}{\subst{\beta}{\alpha_0'}{x_0}}{B}}{\D_2'}
\]
Thus, we have the first, third, and fourth premises of
\[
\infer{\seq{\Gamma,\Gamma'}{\subst{\beta}{\alpha_0}{x_0}}{A_0}}
      {\begin{array}{l}
          x:\U{\alpha}{\Delta}{A} \in \Gamma,\Gamma' \\
          {\subst{\beta}{\alpha_0}{x_0}} \spr \subst{\subst{\beta}{\alpha_0'}{x_0}}{\tsubst{\alpha}{\gamma}}{z} \\
       {\seq{\Gamma,\Gamma'}{\gamma}{\Delta}} \\
       {\seq{\Gamma,\Gamma',z:A}{\subst{\beta}{\alpha_0'}{x_0}}{B}}
        \end{array}
      }
\]
The transformation premise is
\[
     {\subst{\beta}{\alpha_0}{x_0}} 
\spr \subst{\beta}{\subst{\alpha_0'}{\tsubst{\alpha}{\gamma}}{z}}{x0} = \subst{\subst{\beta}{\alpha_0'}{x_0}}{\tsubst{\alpha}{\gamma}}{z}
\]
where the first step is by congruence with $\beta$ on $s$, and the
second is by properties of substitution ($z$ is not free in $\beta$).
\end{enumerate}

For part 2, there are just two right-commutative cases: For
\[
\seq{\Gamma,\Gamma'}{\alpha_0}{A_0}
\qquad
\seq{\Gamma,x_0:A_0,\Gamma'}{\cdot}{\cdot}
\]
we also have $\subst \cdot {\alpha_0}{x_0} = \cdot$ and
\seq{\Gamma,\Gamma'}{\cdot}{\cdot}.  For
\[
\seq{\Gamma,\Gamma'}{\alpha_0}{A_0}
\qquad
\infer{\seq{\Gamma,x_0:A_0,\Gamma'}{\gamma,\alpha/x}{\Delta,x:A}}
      {\seq{\Gamma,x_0:A_0,\Gamma'}{\gamma}{\Delta} &
        \seq{\Gamma,x_0:A_0,\Gamma'}{\alpha}{A}
      }
\]
we have $\subst{(\gamma,\alpha/x)}{\alpha_0}{x_0} 
= (\subst{\gamma}{\alpha_0}{x_0},\subst{\alpha}{\alpha_0}{x_0})$, and
 \[
\seq{\Gamma,\Gamma'}{\subst{\gamma}{\alpha_0}{x_0}}{\Delta} \quad
\seq{\Gamma,\Gamma'}{\subst{\alpha}{\alpha_0}{x_0}}{A}
\]
by the inductive hypotheses, so we can reapply the rule to conclude
\seq{\Gamma,\Gamma'}{\subst{(\gamma,\alpha/x)}{\alpha_0}{x_0}}{\Delta,x:A}.

For part 3, we induct on $\Delta$, reducing a simultaneous cut to
iterated single-variable cuts.  If $\Delta$ is empty, then we have
\[
\seq{\Gamma}{\cdot}{\cdot}
\qquad
\deduce{\seq{\Gamma,\cdot}{\beta}{C}}{\E}
\]
and we return \E, noting that $\tsubst{\beta}{\cdot} = \beta$.  Otherwise
we have
\[
\infer{\seq{\Gamma}{\gamma,\alpha/x}{\Delta,x:A}}
      {\deduce{\seq{\Gamma}{\gamma}{\Delta}}{\D_1} &
        \deduce{\seq{\Gamma}{\alpha}{A}}{\D_2}}
\qquad
\deduce{\seq{\Gamma,\Delta,x:A}{\beta}{C}}{\E}
\]
Using the inductive hypothesis to cut $\D_2$ into $\E$ ($A$ is smaller
than $\Delta,x:A$) gives
\[
\deduce{\seq{\Gamma,\Delta}{\subst{\beta}{\alpha}{x}}{C}}
       {\E'}
\]
Using the inductive hypothesis to cut $\D_1$ into $\E'$ ($\Delta$ is
smaller than $\Delta,x:A$) gives
\[
\seq{\Gamma}{\tsubst{\subst{\beta}{\alpha}{x}}{\gamma}}{C}
\]
Because $\gamma$ substitutes for $\hat \Delta$ (and not $\hat \Gamma$,
the free variables of $\alpha$),
\[
\tsubst{\beta}{\gamma,\alpha/x}
= {\tsubst{\subst{\beta}{\alpha}{x}}{\gamma}}
\]
\end{proof}

Using this, we have
\begin{corollary}[Contraction over contraction] \label{cor:contraction}
\item If
\seq{\Gamma,x:A,y:A,\Gamma'}{\alpha}{C}
then
\seq{\Gamma,z:A,\Gamma'}{\tsubst \alpha {z/x,z/y}}{C}
\end{corollary}

\begin{proof}  Contraction can be shown by cutting with a renaming substitution.
The mode substitution $z/x,z/y$ is a variable-for-variable substitution,
and is type-preserving between ${x:A,y:A}$ and ${\Gamma,z:A,\Gamma'}$.
Therefore, by identity (part 2),
\seq{\Gamma,z:A,\Gamma'}{z/x,z/y}{x:A,y:A}.  Thus, by cut (part 2), we
obtain the result.
\end{proof}

\begin{corollary}[Right-invertibility of \Usymb] \label{cor:Uinv}
If $\seq{\Gamma}{\beta}{\U{x.\alpha}{\Delta}{A}}$ then 
{\seq{\Gamma,\Delta}{\subst{\alpha}{\beta}{x}}{A}}.
\end{corollary}

\begin{proof}
$\UL$ with identities in both premises gives a derivation
\[
\infer{\seq{\Gamma,\Delta,x:{\U{x.\alpha}{\Delta}{A}}}{\alpha}{A}}
      {
        \alpha = z[\alpha[\vec{x/x}]/z] & 
        \seq{\Gamma,\Delta}{\vec{x/x}}{\Delta} &
        \seq{\Gamma,\Delta,x:{\U{x.\alpha}{\Delta}{A}},z:A}{z}{A}
      }
\]
Weakening the assumed derivation to 
\seq{\Gamma,\Delta}{\beta}{\U{x.\alpha}{\Delta}{A}}
and then cutting for $x$ in the above gives the result:

\[
\infer{\seq{\Gamma,\Delta}{\subst{\alpha}{\beta}{x}}{A}}
      {\seq{\Gamma,\Delta}{\beta}{\U{x.\alpha}{\Delta}{A}} & 
       \seq{\Gamma,\Delta,x:{\U{x.\alpha}{\Delta}{A}}}{\alpha}{A}
      }
\]
\end{proof}

\subsection{Natural Deduction Rules}

We show that natural-deduction-style rules are interderivable with the
sequent calculus rules presented above (a sharper result would be to
compare cut-free proofs with normal/neutral natural deduction).  In a
natural deduction style, the hypothesis and right/intro rules are
unchanged, except the hypothesis rule is not restricted to atoms:
\[
\infer[\dsd{v}]{\seq{\Gamma}{\beta}{A}}
      {x:A \in \Gamma & \beta \spr x}
\quad
\infer[\Fsymb I]{\seq{\Gamma}{\beta}{\F{\alpha}{\Delta}}}
      {%% \modeof{\Gamma} \vdash \gamma : \modeof{\Delta} & 
        \beta \spr \tsubst{\alpha}{\gamma} &
        \seq{\Gamma}{\gamma}{\Delta} 
      }
\qquad
\infer[\Usymb I]{\seq{\Gamma}{\beta}{\U{x.\alpha}{\Delta}{A}}}
      {\seq{\Gamma,\Delta}{\subst{\alpha}{\beta}{x}}{A}}
\]
The extended hypothesis rule is justified in the sequent calculus by
Theorem~\ref{thm:identity} and Lemma~\ref{lem:respectspr}, and clearly
includes the atom-restricted sequent rule as a special case.  

We build in a cut to the \FL-rule to obtain the \Fsymb-elimination rule:
\[
\infer[\Fsymb E]{\seq{\Gamma}{\beta}{C}}
      {\beta \spr \beta_2[\beta_1/x] &
        \seq{\Gamma}{\beta_1}{\F{\alpha}{\Delta}} &
        \seq{\Gamma,\Delta}{\beta_2[\alpha/x]}{C}}
\]
and build in a pre-composition and remove the post-composition from the
\UL-rule to obtain the \Usymb-elimination rule:
\[
\infer[\Usymb E]{\seq{\Gamma}{\beta}{A}}
      {\begin{array}{llll}
          \beta \spr \alpha[\gamma,\beta_1/c] &
          \seq{\Gamma}{\beta_1}{\U{c.\alpha}{\Delta}{A}} &
          \seq{\Gamma}{\gamma}{\Delta} 
       \end{array}
      }
\]

These are implemented in the sequent calculus as follows:
\[
\infer[Lem~\ref{lem:respectspr}]
      {\seq{\Gamma}{\beta}{C}}
      {\beta \spr \beta_2[\beta_1/x] &
        \infer[Thm~\ref{thm:cut}]{\seq{\Gamma}{\beta_2[\beta_1/x]}{C}}
              {\seq{\Gamma}{\beta_1}{\F{\alpha}{\Delta}} &
                \infer[\FL]{\seq{\Gamma,z:\F{\alpha}{\Delta}}{\beta_2}{C}}
                      {\seq{\Gamma,\Delta}{\beta_2[\alpha/x]}{C}}}}
\]
and 
\[
\infer[Lem~\ref{lem:respectspr}]
      {\seq{\Gamma}{\beta}{A}}
      {\beta \spr \alpha[\gamma,\beta_1/c] &
        \infer[Thm~\ref{thm:cut}]{\seq{\Gamma}{\alpha[\gamma,\beta_1/c]}{A}}
              {\seq{\Gamma}{\beta_1}{\U{c.\alpha}{\Delta}{A}} &
                \infer[\UL]{\seq{\Gamma,c:\U{c.\alpha}{\Delta}{A}}{\alpha[\gamma]}{A}}
                      { 
                        \alpha[\gamma] \spr z[\alpha[\gamma]/z] &
                        \seq{\Gamma}{\gamma}{\Delta} &
                        \infer[Thm~\ref{thm:identity}]{\seq{\Gamma,z:A}{z}{A}}{}
                      }}}
\]

For the natural deduction calculus consisting of primitive rules
$\dsd{v},\Fsymb I, \Fsymb E, \Usymb I, \Usymb E$, the analogues of
respect for transformations (Lemma~\ref{lem:respectspr}) and cut
(Theorem~\ref{thm:cut}) hold by induction on derivations, though the
latter has a much weaker meaning --- simple replacement rather than
normaliation --- because we allow non-normal forms in these rules.

Conversely, translating sequent calculus to natural deduction, \FL\/ is
the special case of $\Fsymb E$ where the first premise is the identity
and the second premise is a variable. For \UL\/, we take the special
case of $\Usymb E$ where the first premise is the identity and the
second premise is a variable, and then use the admissible substitution
and structural transformation principles of natural deduction to compose
with the third and then first premises of \UL.  

\subsection{General Properties}

We give a couple of general constructions that were used in several
examples above.

The following ``fusion'' lemmas (which are isomorphisms, not just
interprovabilities) relate $\Fsymb$ and $\Usymb$.  Special cases
include: $A \times (B \times C)$ is isomorphic to a primitive triple
product $\{x:A,y:B,z:C\}$; currying; and associativity of $n$-ary
functions---i.e. that $A_1,\ldots,A_n \to (B_1,\ldots,B_m \to C)$ is isomorphic to
$A_1,\ldots,A_n,B_1,\ldots,B_m \to C$.  The derivations are in
Figure~\ref{fig:fusion}.  We adopt the convention that an unlabeled leaf
$\alpha \spr \beta$ is proved by equality of context descriptors, and an
unlabeled sequent leaf is proved by identity
(Theorem~\ref{thm:identity}).  

\begin{lemma}[Fusion] ~ \label{lem:fusion}
\begin{enumerate} 

\item $\F{\alpha}{\Delta,x:\F{\beta}{\Delta'},\Delta''} \dashv \vdash
  \F{\subst{\alpha}{\beta}{x}}{\Delta,\Delta',\Delta''}$

\item $\U{x.\alpha}{\Delta,y:\F{\beta}{\Delta'},\Delta''}{A} \dashv \vdash
  \U{x.\subst{\alpha}{\beta}{y}}{\Delta,\Delta',\Delta''}{A}$

\item 
$\U{x.\alpha}{\Delta}{\U{y.\beta}{\Delta'}{A}} \dashv \vdash
 \U{x.\subst{\beta}{\alpha}{y}}{\Delta,\Delta'}{A}$

\end{enumerate}
\end{lemma}

\begin{figure}
\begin{small}
\[
\infer[\FL]{
  \seq{z:\F{\alpha}{\Delta,x:\F{\beta}{\Delta'},\Delta''}}
      {z}
      {\F{\subst{\alpha}{\beta}{x}}{\Delta,\Delta',\Delta''}}
}
{
  \infer[\FL]{\seq{\Delta,x:\F{\beta}{\Delta'},\Delta''}{\alpha}{\F{\subst{\alpha}{\beta}{x}}{\Delta,\Delta',\Delta''}}}
        {
          \infer[\FL]{\seq{\Delta,\Delta'',\Delta'}{\subst{\alpha}{\beta}{x}}{\F{\subst{\alpha}{\beta}{x}}{\Delta,\Delta',\Delta''}}}
                {\subst{\alpha}{\beta}{x} \spr \tsubst{\subst{\alpha}{\beta}{x}}{\vec{z/z}} &
                 \seq{\Delta,\Delta'',\Delta'}{\vec{z/z}}{\Delta,\Delta',\Delta''}
                }
        }
}
\]

\[
\infer{
  \seq{z:{\F{\subst{\alpha}{\beta}{x}}{\Delta,\Delta',\Delta''}}}
      {z}
      {\F{\alpha}{\Delta,x:\F{\beta}{\Delta'},\Delta''}}
}
{  
\infer[\FL]{\seq{\Delta,\Delta',\Delta''}
           {\subst{\alpha}{\beta}{x}}
           {\F{\alpha}{\Delta,x:\F{\beta}{\Delta'},\Delta''}}}
      {\alpha[\beta/x] \spr \alpha[\vec{y/y},\beta/x,\vec{z/z}] &
        \infer[\FR]{\seq{\Delta,\Delta',\Delta''}{\vec{y/y},\beta/x,\vec{z/z}} {\Delta,x:\F{\beta}{\Delta'},\Delta''}}
              {\seq{\Delta,\Delta',\Delta''}{\vec{y/y}}{\Delta} & 
               \infer[\FR]{\seq{\Delta,\Delta',\Delta''}{\beta}{\F{\beta}{\Delta'}}}
                     {\beta \spr \beta[\vec{w/w}] & \seq{\Delta,\Delta',\Delta''}{\vec{w/w}}{\Delta'}} &
               \seq{\Delta,\Delta',\Delta''}{\vec{z/z}}{\Delta''} }
      }
}
\]

\[
\infer[\UR]{\seq{x:\U{x.\alpha}{\Delta,y:\F{\beta}{\Delta'},\Delta''}{A}}{x}{\U{x.\subst{\alpha}{\beta}{y}}{\Delta,\Delta',\Delta''}{A}}}
      {\infer[\UL]
        {\seq{\Gamma := (x:\U{x.\alpha}{\Delta,y:\F{\beta}{\Delta'},\Delta''}{A}, {\Delta,\Delta',\Delta''})}{{\subst{\alpha}{\beta}{y}}}{A}} 
        {\subst{\alpha}{\beta}{y} \spr z[\tsubst{\alpha}{\vec{w/w},\beta/y}/z] &
          \seq{\Gamma}{\vec{w/w}}{\Delta,\Delta''} &
          \infer[\FR]{\seq{\Gamma}{\beta}{\F{\beta}{\Delta'}}}{} &
          \seq{\Gamma,z:A}{z}{A}
        }
      }
\]

\[
\infer[\UR]{\seq{x:\U{x.\subst{\alpha}{\beta}{y}}{\Delta,\Delta',\Delta''}{A}}{x}{\U{x.\alpha}{\Delta,y:\F{\beta}{\Delta'},\Delta''}{A}}}
      {\infer[\FL]
        {\seq{x:\U{x.\subst{\alpha}{\beta}{y}}{\Delta,\Delta',\Delta''}{A}, \Delta,y:\F{\beta}{\Delta'},\Delta''}{\alpha}{A}} 
        {\infer[\UL]{\seq{x:\U{x.\subst{\alpha}{\beta}{y}}{\Delta,\Delta',\Delta''}{A}, \Delta,\Delta'',\Delta'}{\subst{\alpha}{\beta}{y}}{A}}
          {\subst{\alpha}{\beta}{y} \spr z[\tsubst{\subst{\alpha}{\beta}{y}}{\vec{w/w}}] &
            \seq{\Delta,\Delta',\Delta''}{\vec{w/w}}{\Delta,\Delta',\Delta''} &
            \seq{\Delta,\Delta',\Delta'',z:A}{z}{A}
          }
        }}
\]

\[
\infer[\UR]
      {\seq{x:\U{x.\alpha}{\Delta}{\U{y.\beta}{\Delta'}{A}}}{x} {\U{x.\subst{\beta}{\alpha}{y}}{\Delta,\Delta'}{A}}}
      {\infer[\UL]
        {\seq{x:\U{x.\alpha}{\Delta}{\U{y.\beta}{\Delta'}{A}},\Delta,\Delta'}{\subst{\beta}{\alpha}{y}}{A}}
        {\subst{\beta}{\alpha}{y} \spr \beta[\alpha[\vec{w/w}]/y] &
          \seq{x:\U{x.\alpha}{\Delta}{\U{y.\beta}{\Delta'}{A}},\Delta,\Delta'}{\vec{w/w}}{\Delta} &
          \infer[\UL]{\seq{x:\U{x.\alpha}{\Delta}{\U{y.\beta}{\Delta'}{A}},\Delta,\Delta',y:\U{y.\beta}{\Delta'}{A}}{\beta}{A}}{}
        }}
\]

\[
\infer[\UR]
      {\seq{x:{\U{x.\subst{\beta}{\alpha}{y}}{\Delta,\Delta'}{A}}}{x}{\U{x.\alpha}{\Delta}{\U{y.\beta}{\Delta'}{A}}}}
      {\infer[\UR]
        {\seq{x:{\U{x.\subst{\beta}{\alpha}{y}}{\Delta,\Delta'}{A}},\Delta}{\alpha}{{\U{y.\beta}{\Delta'}{A}}}}
        {\infer[\UL]{\seq{x:{\U{x.\subst{\beta}{\alpha}{y}}{\Delta,\Delta'}{A}},\Delta,\Delta'}{\subst{\beta}{\alpha}{y}}{A}}
          {}}}
\]
\end{small}
\caption{Derivations of fusion lemmas}
\vspace{2in}
\label{fig:fusion}
\end{figure}

The types respect the structural transformations, covariantly for \Fsymb\/
and contravariantly for \Usymb\/.

\begin{lemma}[Types Respect Structural Transformations] ~ \label{lem:typespr}
\begin{enumerate}
\item 
 If $\alpha \spr \beta$ then $\F{\alpha}{\Delta} \vdash
  \F{\beta}{\Delta}$

\item If $\alpha \spr \beta$ then $\U{x.\beta}{\Delta}{A} \vdash
  \U{x.\alpha}{\Delta}{A}$
\end{enumerate}
\end{lemma}

\begin{proof}
\[
\infer[\FL]{\seq{x:\F{\alpha}{\Delta}}{x}{\F{\beta}{\Delta}}}
      {\infer[\FR]{\seq{\Delta}{\alpha}{\F{\beta}{\Delta}}}
        {\alpha \spr \beta[\vec{w/w}] &
          \seq{\Delta}{\vec{w/w}}{\Delta}
      }}
\]        
\[
\infer[\UR]
      {\seq{x:\U{x.\beta}{\Delta}{A}}{\U{x.\alpha}{\Delta}{A}}}
      {\infer[\UL]{ \seq{x:\U{x.\beta}{\Delta}{A},\Delta}{\alpha}{A}}
        {\alpha \spr z[\beta[\vec{w/w}]/z] &
          \seq{x:\U{x.\beta}{\Delta}{A},\Delta}{\vec{w/w}}{\Delta} &
          \seq{\ldots,z:A}{z}{A}
      }}
\]        
\end{proof}


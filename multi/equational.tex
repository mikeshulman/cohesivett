\section{Equational Theory}
\label{sec:equational}

\subsection{Equations on Structural Transformations}
\label{sec:equational-transformations}

First, we need a notation for derivations of the $\alpha \spr \beta$
judgement in Figure~\ref{fig:2multicategory}.  We assume names for
constants are given in the signature $\Sigma$, and write $1_\alpha$ for
reflexivity, $s_1;s_2$ for transitivity (in diagramatic order), and
$s_1[s_2/x]$ for congruence.  We allow the signature $\Sigma$ to provide
some axioms for equality of transformations $s_1 \deq s_2$ (for two
derivations of the same judgement $s_1,s_2 ::
\wfsp{\psi}{\alpha}{\beta}{p}$), and define equality to be the least
congruence closed under those axioms and the following associativity,
unit, and interchange laws:

\begin{itemize}
\item $1_\alpha;s \deq s \deq s;1_\beta \text{ for } s :: \alpha \spr \beta$
\item $(s_1;s_2);s_3 \deq s_1;(s_2;s_3)$
\item $s[1_x/x] \deq s \deq 1_x[s/x] \text{ for } s :: \wfsp{\psi,x,\psi'}{\alpha}{\beta}{r}$
\item $\subst{\subst{s_1}{s_2}{x}}{s_3}{y} \deq
\subst{\subst{s_1}{s_3}{y}}{\subst{s_2}{s_3}{y}}{x}$ as transformations
${\subst{\subst{\alpha_1}{\alpha_2}{x}}{\alpha_3}{y}} \spr
{\subst{\subst{\beta_1}{\beta_2}{x}}{\beta_3}{y}}$ for $s_1 :: (\wfsp{\psi,x:p,y:q}{\alpha_1}{\beta_1}{r}), 
 s_2 :: (\wfsp{\psi,y:q}{\alpha_2}{\beta_2}{p}), 
 s_3 :: (\wfsp{\psi}{\alpha_3}{\beta_3}{q})$ 
\item $s_1[t_1/x];s_2[t_2/x] \deq (s_1;s_2)[(t_1;t_2)/x]$
as morphisms $\alpha_1[\beta_1/x] \spr \alpha_3[\beta_3/x]$
for $s_1 :: (\wfsp{\psi,x:p,\psi'}{\alpha_1}{\alpha_2}{r})$,
 $s_2 :: (\wfsp{\psi,x:p,\psi'}{\alpha_2}{\alpha_3}{r})$
 $t_1 :: (\wfsp{\psi,\psi'}{\beta_1}{\beta_2}{p})$
 $t_2 :: (\wfsp{\psi,\psi'}{\beta_2}{\beta_3}{p})$
\item $1_\alpha[s/x] \deq 1_\alpha$ if $y \# \alpha$
\item $\subst{1_\alpha}{1_\beta}{x} \deq 1_{\subst{\alpha}{\beta}{x}}$
  FIXME follows from others?
\end{itemize}

\noindent These are the 2-category axioms extended to the
multicategorical case.  The associativity and unit laws for
congruence/horizontal composition $(s[s'/x])$ require the analogous
associativity for $(\alpha[\alpha'/x])$ to type check.

As we did with equality of context descriptors, we think of all
definitions as being parametrized by \deq-equivalence-classes of
transformations, not raw syntax.

\subsection{Equations on Derivations}

\newcommand\FLd[2]{\ensuremath{\FL_{#1}(#2)}}
\newcommand\FRd[2]{\ensuremath{\FR(#1,#2)}}
\newcommand\ULd[4]{\ensuremath{\UL_{#1}(#2,#3,#4)}}
\newcommand\URd[1]{\ensuremath{\UR(#1)}}
\newcommand\Trd[2]{\ensuremath{#1_*(#2)}}
\newcommand\Ident[1]{\ensuremath{\dsd{ident}_{#1}}}
\newcommand\Cut[3]{\ensuremath{#1[#2/#3]}}

We use the following notation for derivations:
\[
\begin{array}{rcl}
\D & ::= & \Trd{s}{x} \mid \FLd{x}{\Delta.\D} \mid \FRd{s}{\vec{\D_i/x_i}} \\
& & \ULd{x}{s}{\vec{\D_i/x_i}}{\D} \mid \URd{\Delta.\D} \mid \Trd{s}{\D} \mid
\Ident{x} \mid \Cut{\D_1}{\D_2}{x}
\end{array}
\]
where \Trd{s}{\D} is Lemma~\ref{lem:respectspr}, \Ident{x} is
Theorem~\ref{thm:identity}, and \Cut{\D_1}{\D_2}{x} is
Theorem~\ref{thm:cut}.

The equational theory of derivations is the least congruence (on all of
the above derivation formers) containing the definitions of
$\Trd{s}{\D}$ and $\Ident{x}$ and $\Cut{\D_1}{\D_2}{x}$ given by the
proofs in Appendix~\ref{sec:synprop-long} as well as
\begin{itemize}
\item The unit laws $\Cut{\D}{\Ident{x}}{x} \deq \D$ and $\Cut{\Ident{x}}{\D}{x} \deq \D$.  
\item A projection law $\Cut{\D_1}{\D_2}{x} \deq \D_1$ if $x$ is not used
  in $\D_1$ (in the usual sense that it does not occur in a free
  position, i.e. the hypothesis rule or the principal argument of a left
  rule).
\item Associativity of cut (in a multicategorical sense):
  $\Cut{(\Cut{\D_1}{\D_2}{x})}{\D_3}{y} \deq
  \Cut{(\Cut{\D_1}{\D_3}{y})}{\Cut{\D_2}{\D_3}{y}}{x}$
\item Functoriality of the action of structural transformations:
  $\Trd{1}{\D} \deq \D$ and $\Trd{(s_1;s_2)}{\D} \deq
  \Trd{{s_1}}{\Trd{{s_2}}{\D}}$

\item The action of a transformation commutes with cut: if
$\D_1 :: \seq{\Gamma}{\alpha'}{A}$ and
$\D_2 :: \seq{\Gamma,x:A}{\beta'}{C}$
and $s_1 :: \alpha \spr \alpha'$ 
and $s_2 :: \beta \spr \beta'$, then
$\Cut{\Trd{{s_2}}{\D_2}}{\Trd{{s_1}}{\D_1}}{x} \deq \Trd{(\subst{s_2}{s_1}{x})}{\Cut{\D_2}{\D_1}{x}}$
as derivations of \seq{\Gamma}{\subst{\beta}{\alpha}{x}}{C}.  

\item An $\eta$ law for \Usymb: $\D ::
  \seq{\Gamma}{\beta}{\U{x.\alpha}{\Delta}{A}}$ is equal to
\[
\infer[\UR]
      {\seq{\Gamma}{\alpha}{\U{x.\alpha}{\Delta}{A}}}
      {\infer[\dsd{cut}]
             {\seq{\Gamma,\Delta}{\subst{\beta}{\alpha}{x}}{\U{x.\alpha}{\Delta}{A}}}
             {\deduce{\seq{\Gamma,\Delta}{\beta}{\U{x.\alpha}{\Delta}{A}}}{\D} &
               \infer[\UL*]{\seq{\Gamma,\Delta,x:\U{x.\alpha}{\Delta}{A}}{\alpha}{A}}
                           {}
      }}
\]
(or $\URd{\Delta.\Cut{\ULd{x}{1}{\Ident{\Delta}}{z.z}}{\D}{x}}$).  The
$\beta$ law for \Usymb\/ is a defining equation of cut (the principal
case).

\item An $\eta$ law for \Fsymb: $\D ::
  \seq{\Gamma,x:\F{\alpha}{\Delta},\Gamma'}{\beta}{C}$ is equal to
\[
\infer[\FL]{\seq{\Gamma,x:\F{\alpha}{\Delta},\Gamma'}{\beta}{C}}
      {\infer[\dsd{cut}]
        {\seq{\Gamma,\Gamma',\Delta}{\subst{\beta}{\alpha}{x}}{C}}
        {\infer[\FR^*]{\seq{\Gamma,\Gamma',\Delta}{\alpha}{\F{\alpha}{\Delta}}}{} &
          \D}}
\]
(or $\FLd{x}{\Delta.\Cut{\D}{\FRd{1}{\Ident \Delta}}{x}}$).  
The $\beta$ law is a defining equation of cut.
\end{itemize}
In the one-variable case~\citep{ls16adjoint}, we reduced these laws to
some more specific axioms, which bridge the gap between the defining
equations of identity, cut, and respect for transformations and these
abstract properties, but we have not yet attempted such a reduction
here.

\paragraph{Example}

Consider an affine product (commutative monoid $(\otimes,1)$ with
$\dsd{w} :: x \spr 1$) and the following two derivations of $P,Q,R
\vdash P \otimes R$:

\begin{footnotesize}
\[
\infer[\FR]
      {\seq{x:P,y:Q,z:R}{x \otimes y \otimes z}{\F{x' \otimes z'}{x':P,z':R}}}
      {x \otimes y \otimes z \spr ((x \otimes y) \otimes z) &
        \infer[\dsd{v}]
              {\seq{x:A,y:B,z:C}{x \otimes y}{C}}
              {(x \otimes y) \spr x} &
        \infer[\dsd{v}]
              {\seq{x:A,y:B,z:C}{z}{C}}
              {z \spr z}
      }
\]
\end{footnotesize}
\begin{footnotesize}
\[
\infer[\FR]
      {\seq{x:P,y:Q,z:R}{x \otimes y \otimes z}{\F{x' \otimes z'}{x':P,z':R}}}
      {x \otimes y \otimes z \spr (x \otimes z) &
        \infer[\dsd{v}]
              {\seq{x:A,y:B,z:C}{x}{C}}
              {x \spr x} &
        \infer[\dsd{v}]
              {\seq{x:A,y:B,z:C}{z}{C}}
              {z \spr z}
      }
\]
\end{footnotesize}

These differ by the placement of weakening: our rules allow us to weaken
away $y$ either at a leaf (there is a third derivation that weakens in
the $z$ leaf instead), as is typical in affine logic, or at the context
division in $\FR$.  However, these two derivations are equal in the
above equational theory:
\[
\FRd{1}{\Trd{(1_{x \otimes y}[\dsd{w}/y])}{x},\Trd{1}{z}}
\deq
\FRd{1_{x \otimes y \otimes z}[\dsd{w}/y]}{\Trd{1}{x},\Trd{1}{z}}
\]
By associativity, unit and projection laws, we have that a general
instance of \FR\/ is equal to an iterated cut (in any order) and then a
structural transformation on the ``axiomatic'' $\FR^* ::
\seq{\Gamma,\Delta}{\alpha}{\F{\alpha}{\Delta}}$
\[
\FRd{\alpha}{\vec{\D_i/x_i}} \deq \Trd{\alpha}{\FR^*[\vec{\D_i/x_i}]}
\]
So applying this to both sides it suffices to show
\[
\begin{array}{rcl}
& &  \Trd{{(1_{x \otimes y \otimes z})}}{\FR^*[\Trd{1}{z}/z'][\Trd{(1_{x \otimes y}[\dsd{w}/y])}{x}/x']}\\
& \deq & \Trd{{(1_{x \otimes y \otimes z}[\dsd{w}/y])}}{\FR^*[\Trd{1}{z}/z'][\Trd{1}{x}/x']} 
\end{array}
\]
Here we have
\[
\begin{array}{rcl}
\FR^*[\Trd{1}{z}/z'] & :: & \seq{x:P,y:Q,z:R,x':P}{x' \otimes z}{P \otimes R}\\
\Trd{1}{x} & :: & \seq{x:P,y:Q,z:R}{x}{P}\\
{(1_{x \otimes y}[\dsd{w}/y])} & :: & x \otimes y \spr y\\
1_{x'\otimes z} & :: & x' \otimes z
\end{array}
\]
so the transformation-on-cut equation gives
\[
\begin{array}{rl}
     & \Cut{(\Trd{({1_{x'\otimes z}})}{\FR^*[\Trd{1}{z}/z']})}{\Trd{{(1_{x \otimes y}[\dsd{w}/y])}}{\Trd{1}{x}}}{x'}\\
 \deq & \Trd{ (\subst{({1_{x'\otimes z}})}{(1_{x \otimes y}[\dsd{w}/y])}{x'})  }{\Cut{\FR^*[\Trd{1}{z}/z']}{\Trd{1}{x}}{x'}}
\end{array}
\]
The left-hand side is equal to the left-hand side of the goal by
functoriality of \Trd{-}{-}, and the right-hand side is equal to the
right-hand side of the goal by associativity of horizontal composition
and $1_{x' \otimes z}[1_{x \otimes y/x'}] \deq 1_{x \otimes y \otimes
  z}$.  

%% \begin{array}{rcll}
%% & &  \FRd{1}{\Trd{(1_{x \otimes y}[\dsd{w}/y])}{x},\Trd{1}{z}} \\
%% & \deq & \Trd{{(1_{x \otimes y \otimes z})}}{\FR^*[\Trd{1}{z}/z'][\Trd{(1_{x \otimes y}[\dsd{w}/y])}{x}/x']}\\
%% & \deq & ? \\
%% & \deq & \Trd{{(1_{x \otimes y \otimes z}[\dsd{w}/y])}}{\FR^*[\Trd{1}{z}/z'][\Trd{1}{x}/x']} \\
%% & \deq & \FRd{1_{x \otimes y \otimes z}[\dsd{w}/y]}{\Trd{1}{x},\Trd{1}{z}} \\
%% \end{array}
%% \]

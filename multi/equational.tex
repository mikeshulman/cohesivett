\section{Equational Theory}
\label{sec:equational}

\subsection{Equations on Structural Transformations}
\label{sec:equational-transformations}

First, we need a notation for derivations of the $\alpha \spr \beta$
judgement in Figure~\ref{fig:2multicategory}.  We assume names for
constants are given in the signature $\Sigma$, and write $1_\alpha$ for
reflexivity, $s_1;s_2$ for transitivity (in diagramatic order), and
$s_1[s_2/x]$ for congruence.  We allow the signature $\Sigma$ to provide
some axioms for equality of transformations $s_1 \deq s_2$ (for two
derivations of the same judgement $s_1,s_2 ::
\wfsp{\psi}{\alpha}{\beta}{p}$), and define equality to be the least
congruence closed under those axioms and the following associativity,
unit, and interchange laws:

\begin{itemize}
\item $1_\alpha;s \deq s \deq s;1_\beta \text{ for } s :: \alpha \spr \beta$
\item $(s_1;s_2);s_3 \deq s_1;(s_2;s_3)$
\item $s[1_x/x] \deq s \deq 1_x[s/x] \text{ for } s :: \wfsp{\psi,x,\psi'}{\alpha}{\beta}{r}$
\item $\subst{\subst{s_1}{s_2}{x}}{s_3}{y} \deq
\subst{\subst{s_1}{s_3}{y}}{\subst{s_2}{s_3}{y}}{x}$ as transformations
${\subst{\subst{\alpha_1}{\alpha_2}{x}}{\alpha_3}{y}} \spr
{\subst{\subst{\beta_1}{\beta_2}{x}}{\beta_3}{y}}$ for $s_1 :: (\wfsp{\psi,x:p,y:q}{\alpha_1}{\beta_1}{r}), 
 s_2 :: (\wfsp{\psi,y:q}{\alpha_2}{\beta_2}{p}), 
 s_3 :: (\wfsp{\psi}{\alpha_3}{\beta_3}{q})$ 
\item $s_1[t_1/x];s_2[t_2/x] \deq (s_1;s_2)[(t_1;t_2)/x]$
as morphisms $\alpha_1[\beta_1/x] \spr \alpha_3[\beta_3/x]$
for $s_1 :: (\wfsp{\psi,x:p,\psi'}{\alpha_1}{\alpha_2}{r})$,
 $s_2 :: (\wfsp{\psi,x:p,\psi'}{\alpha_2}{\alpha_3}{r})$
 $t_1 :: (\wfsp{\psi,\psi'}{\beta_1}{\beta_2}{p})$
 $t_2 :: (\wfsp{\psi,\psi'}{\beta_2}{\beta_3}{p})$
\item $\subst{1_\alpha}{1_\beta}{x} \deq 1_{\subst{\alpha}{\beta}{x}}$
\item $1_\alpha[s/x] \deq 1_\alpha$ if $y \# \alpha$
\end{itemize}

\noindent These are the 2-category axioms extended to the
multicategorical case.  The first two rules are associativity and unit
for both kinds of compositions; the next two are interchange; the final
is because terms with variables that do not occur are an implicit
notation for product projections.  The associativity and unit laws for
congruence/horizontal composition $(s[s'/x])$ require the analogous
associativity for composition $(\alpha[\alpha'/x])$ (which is true
syntactically) to type check.

As we did with equality of context descriptors, we think of all
definitions as being parametrized by \deq-equivalence-classes of
transformations, not raw syntax.

\subsection{Equations on Derivations}

To simplify the axiomatic description of equality, we give a notation
for derivations in a sequent calculus where the admissible
transformation, identity, and cut rules are internalized as explicit
rules---so the calculus has the flavor of an explicit substitution one.
We use the following notation for derivations:
\[
\begin{array}{rcl}
\D & ::= & \Ident{x} \mid \Trd{s}{\D} \mid \Cut{\D_1}{\D_2}{x} \mid
 \FLd{x}{\Delta.\D} \mid \FRd{\gamma}{s}{\vec{\D_i/x_i}} \mid \ULd{x}{\gamma}{s}{\vec{\D_i/x_i}}{\D} \mid \URd{\Delta.\D} 
\end{array}
\]
We omit the primitive hypothesis rule for atoms (it will be derivable),
write $x$ for identity (Theorem~\ref{thm:identity}), \Trd{s}{\D} for
respect for transformations (Lemma~\ref{lem:respectspr}---identity for
atoms combines this and identity) and \Cut{\D_1}{\D_2}{x} for cut
(Theorem~\ref{thm:cut}).  The next 4 terms correspond to the 4
\Usymb/\Fsymb rules.  from Figure~\ref{fig:sequent}.  
We do not notate weakenings or exchanges in these terms.

\newcommand\FRs{\ensuremath{\FR^*}}
\newcommand\ULs[1]{\ensuremath{\UL^*_{#1}}}

We write \FRs\/ for $\FRd{\vec{x/x}}{1}{\Ident{x}/x} ::
\seq{\Gamma}{\alpha}{\F{\alpha}{\Delta}}$ when $\Delta \subseteq \Gamma$
and we write and \ULs{x} for $\ULd{x}{\vec{x/x}}{1}{\Ident{x}/x}{z.z} ::
\seq{\Gamma}{\alpha}{A}$ when $x:\U{x.\alpha}{\Delta}{A} \in \Gamma$ and
$\Delta \subseteq \Gamma$.  

The equational theory of derivations is the least congruence containing
the equations in Figure~\ref{fig:equality-of-derivations}.

\begin{figure}
\[
\begin{array}{rcll} 
\Cut{\D}{\Ident{x}}{x} & \deq & \D \\
\Cut{\Ident{x}}{\D}{x} & \deq & \D \\
\Cut{\D_1}{\D_2}{x} & \deq & \D_1 & \text{if $x \# \D_1$}\\
\Cut{(\Cut{\D_1}{\D_2}{x})}{\D_3}{y} & \deq & \Cut{(\Cut{\D_1}{\D_3}{y})}{\Cut{\D_2}{\D_3}{y}}{x}\\
\\
\Trd{1}{\D} & \deq & \D\\
\Trd{(s_1;s_2)}{\D} & \deq & \Trd{{s_1}}{\Trd{{s_2}}{\D}} \\
\Trd{(\subst{s_2}{s_1}{x})}{\Cut{\D_2}{\D_1}{x}} & \deq & \Cut{\Trd{{s_2}}{\D_2}}{\Trd{{s_1}}{\D_1}}{x} \\
\\
\Cut{\FLd{x_0}{\D}}{\FRd{}{s}{\vec{\D_i/x_i}}}{x_0} & \deq & \Trd{(1_\beta[s_0/x])}{\D[{\D_i/x_i}]} \\
%% special case \Cut{\FLd{x_0}{\D}}{\FRs}{x_0} & \deq & \D \\
\Cut{\ULd{x_0}{}{s}{\vec{\D_i}/x_i}{z.\D'}}{\URd{\Delta.\D}}{x_0} & \deq & \Trd{(s[1_{\alpha_0}])}{\Cut{\D'}{(\D[{\vec{d_i}/x_i}])}{z}} \\
\D :: \seq{\Gamma}{\beta}{\U{x.\alpha}{\Delta}{A}} & \deq & \URd{\Delta.\Cut{\ULs x}{\D}{x}}\\
\D :: \seq{\Gamma,x:\F{\alpha}{\Delta},\Gamma'}{\beta}{C} & \deq & \FLd{x}{\Delta.\Cut{\D}{\FRs}{x}}\\
\end{array}
\]
\caption{Equality of Derivations}
\label{fig:equality-of-derivations}
\end{figure}

The first two equations say that identity is a unit for cut.  The third
says that non-occurence of a variable is a projection (with more
explicit weakening, the notation $x \# \D_1$ means that $\D_1$ is the
weakening of some derivation $\seq{\Gamma,\Gamma'}{\alpha}{C}$ to
$\seq{\Gamma,x:A,\Gamma'}{\alpha}{C}$, and the equation says that we
return that original derivation).  The fourth is functoriality of cut
(when phrased for single-variable substitutions, the equation
$\D[\theta][\theta'] \deq \D[\theta[\theta']]$ becomes a rule for
commuting substitutions).

In the next group, the first two rules say that the action of a
transformation is functorial, and commutes with cut.  The typing in the
third rule is $\D_1 :: \seq{\Gamma}{\alpha'}{A}$ and $\D_2 ::
\seq{\Gamma,x:A}{\beta'}{C}$ and $s_1 :: \alpha \spr \alpha'$ and $s_2
:: \beta \spr \beta'$, so both sides are derivations of as derivations
of \seq{\Gamma}{\subst{\beta}{\alpha}{x}}{C}.

In the next group, we have the $\beta\eta$-laws for \dsd{F} and \dsd{U}.  The $\beta$ laws are the
principal cut cases given above.  By the composition law for cut, the
simultaneous substitution can be defined as iterated substitution in any
order.  The $\eta$ law for \Usymb\/ equates any derivation to
\[
\infer[\UR]
      {\seq{\Gamma}{\alpha}{\U{x.\alpha}{\Delta}{A}}}
      {\infer[\dsd{cut}]
             {\seq{\Gamma,\Delta}{\subst{\beta}{\alpha}{x}}{\U{x.\alpha}{\Delta}{A}}}
             {\deduce{\seq{\Gamma,\Delta}{\beta}{\U{x.\alpha}{\Delta}{A}}}{\D} &
               \infer[\UL*]{\seq{\Gamma,\Delta,x:\U{x.\alpha}{\Delta}{A}}{\alpha}{A}}
                           {}
      }}
\]
The $\eta$ law for \Fsymb\/ equations any derivation to 
\[
\infer[\FL]{\seq{\Gamma,x:\F{\alpha}{\Delta},\Gamma'}{\beta}{C}}
      {\infer[\dsd{cut}]
        {\seq{\Gamma,\Gamma',\Delta}{\subst{\beta}{\alpha}{x}}{C}}
        {\infer[\FR^*]{\seq{\Gamma,\Gamma',\Delta}{\alpha}{\F{\alpha}{\Delta}}}{} &
          \D}}
\]

In these rules, we have prioritized a concise axiomatization over
separating ``definitional'' and ``propositional'' equalities.  For
example, rather than giving the defining equations of transformation,
identity, and cut, we have stated their algebraic properties, including
associativity, unit, functoriality and $\eta$ laws, and the definitions
given in Section~\ref{sec:synprop-long} can be deduced from these.  In
the one-variable case~\citep{ls16adjoint}, we took the opposite
approach, using the defining equations plus to some more specific axioms
to prove these algebraic laws; we have not yet attempted such a
reduction here.

\subsection{Useful equations}

FIXME: derive definitional clauses


\documentclass[a4paper,USenglish,numberwithinsect]{lipics-v2016}
\usepackage{microtype}%if unwanted, comment out or use option "draft"

\usepackage{authblk}

\usepackage[all]{xy}
\usepackage{multicol}
\usepackage{mathptmx}
\usepackage{color}
\usepackage{amsthm}
\usepackage{amssymb}
\usepackage{stmaryrd}
\usepackage{../drl-common/proof}
\usepackage{../drl-common/typesit}
\usepackage{../drl-common/typescommon}
\usepackage{graphics}

\usepackage{url}
\usepackage{relsize}
\usepackage{tipa}

\usepackage{tikz}
\usetikzlibrary{decorations.pathmorphing}

\usepackage{fancyvrb}
\newcommand{\ttt}[1]{\texttt{#1}}


\newcommand\Bx[2]{\ensuremath{\Box_{#1} \, {#2}}}
\newcommand\Crc[2]{\ensuremath{\bigcirc_{#1} \, {#2}}}
\newcommand\Dia[2]{\ensuremath{\Diamond_{#1} \, {#2}}}
\newcommand\Flat[1]{\ensuremath{\flat \, {#1}}}
\newcommand\Sharp[1]{\ensuremath{\sharp \, {#1}}}
\newcommand{\sh}{\text{\textesh}}

\newcommand\D{\ensuremath{d}} %% originally was mathcal{D} but switched notation for derivations
\newcommand\E{\ensuremath{e}}

\newcommand\magicwand{\mathrel{-\mkern-6mu*}}
\newcommand\mor[3]{\ensuremath{#2} \longrightarrow_#1 #3}
\newcommand\C{\ensuremath{\mathcal{C}}}
\newcommand\deq{\ensuremath{\equiv}}
\newcommand\spr{\ensuremath{\Rightarrow}} %% structural property/2-cell
\newcommand\seq[3]{\ensuremath{#1 \vdash_{#2} #3}}
\newcommand\seql[3]{\ensuremath{#1 \vdash^{\dsd{#2}} #3}}
\newcommand\F[2]{\ensuremath{\dsd{F}_{#1}(#2)}}
\newcommand\U[3]{\ensuremath{\dsd{U}_{#1}(#2 \mid #3)}}
\newcommand\Uempty[2]{\ensuremath{\dsd{U}_{#1}(#2)}}
\newcommand\Fsymb[0]{\dsd{F}}
\newcommand\Usymb[0]{\dsd{U}}
\newcommand\tsubst[2]{\ensuremath{#1[#2]}}
\renewcommand\subst[3]{\ensuremath{#1[#2/#3]}}
\newcommand\wftype[2]{\ensuremath{#1 \,\, \dsd{type}_{#2}}}
\renewcommand\wfctx[2]{\ensuremath{#1 \,\, \dsd{ctx}_{#2}}}
\newcommand\modeof[1]{\ensuremath{\hat{#1}}}
\newcommand\many[1]{\ensuremath{\overline{#1}}}
\renewcommand{\oftp}[3]{\ensuremath{#1 \, \vdash #2 \, \dcd{:} \, #3}}
\newcommand\FL{\dsd{FL}}
\newcommand\FR{\dsd{FR}}
\newcommand\UL{\dsd{UL}}
\newcommand\UR{\dsd{UR}}
\newcommand\lolli\multimap
\newcommand\la\dashv

\def\M{\mathcal{M}}
\def\toiso{\xrightarrow{\sim}}
\let\To\Rightarrow
%\newcommand\compo[2]{\ensuremath{#1 \circ #2}}
\newcommand\compv[2]{\ensuremath{#1 \cdot #2}}
\newcommand\comph[2]{\ensuremath{#1 \mathbin{\circ_2} #2}}

\def\llb{\llbracket}
\def\rrb{\rrbracket}

\newcommand\seqa[2]{\seql{#1}{a}{#2}}
\newcommand\seqc[2]{\seql{#1}{c}{#2}}
\newcommand\splits{\rightrightarrows}
\newcommand\vars[1]{\ensuremath{\overline{#1}}}

\newcommand{\ignore}[1]{}

\newcommand\FLd[3]{\ensuremath{\dsd{split} \, #2 \, = \, {#1} \, \dsd{in} \, #3}}
\newcommand\FRd[3]{\ensuremath{\Trd{#2}{#3}}}
\newcommand\ULd[6]{\ensuremath{\Trd{#3}{\dsd{let} \, #5 \, = \, #1(#4) \, \dsd{in} \, #6 }}}
\newcommand\URd[2]{\ensuremath{\lambda #1.#2}}
\newcommand\Trd[2]{\ensuremath{#1_*(#2)}}
\newcommand\Ident[1]{\ensuremath{{#1}}}
\newcommand\Cut[3]{\ensuremath{#1[#2/#3]}}
\newcommand\Cuta[3]{\ensuremath{#1\{#2/#3\}}}
\newcommand\Cutta[2]{\ensuremath{#1\{#2\}}}
\newcommand\Trda[2]{\ensuremath{#1_*\{#2\}}}
\newcommand\Identa[1]{\ensuremath{{\dsd{id}\{#1\}}}}
\newcommand\FRs{\ensuremath{\FR^*}}
\newcommand\ULs[1]{\ensuremath{\UL^*_{#1}}}

\newcommand\elim[1]{#1\mathord{\downarrow}}
\newcommand\deqp{\deq_{\dsd p}}
\newcommand\Linv[3]{\ensuremath{\dsd{linv}(#1,#2,#3)}}
\newcommand\Rinv[2]{\ensuremath{\dsd{rinv}(#1,#2)}}

\begin{document}

\title{A Fibrational Framework for \hspace{4in} Substructural and Modal Logics}
\titlerunning{A Fibrational Framework for Substructural and Modal Logics}

\author[1]{{Daniel R. Licata}}
\author[2]{{Michael Shulman}}
\author[1]{{Mitchell Riley}}
\affil[1]{Wesleyan University,\thanks{This material is based on research
    sponsored by The United States Air Force Research Laboratory under
    agreement number FA9550-15-1-0053 and FA9550-16-1-0292.  The
    U.S. Government is authorized to reproduce and distribute reprints
    for Governmental purposes notwithstanding any copyright notation
    thereon.  The views and conclusions contained herein are those of
    the authors and should not be interpreted as necessarily
    representing the official policies or endorsements, either expressed
    or implied, of the United States Air Force Research Laboratory, the
    U.S. Government, or Carnegie Mellon University.} Middletown, CT, USA}
\affil[2]{University of San Diego,$^*$ San Diego, CA, USA}

\Copyright{Daniel R. Licata, Michael Shulman, Mitchell Riley}

\authorrunning{D. Licata and M. Shulman and M. Riley} 

\subjclass{F.4.1 Mathematical Logic}
\keywords{type theory, modal logic, substructural logic, homotopy type theory}

\EventEditors{Dale Miller}
\EventNoEds{1}
\EventLongTitle{2nd International Conference on Formal Structures for Computation and Deduction (FSCD 2017)}
\EventShortTitle{FSCD 2017}
\EventAcronym{FSCD}
\EventYear{2017}
\EventDate{September 3--9, 2017}
\EventLocation{Oxford, UK}
\EventLogo{}
\SeriesVolume{84}
\ArticleNo{24} 

\maketitle

\begin{abstract}
We define a general framework that abstracts the common features of many
intuitionistic substructural and modal logics / type theories.  The
framework is a sequent calculus / normal-form type theory parametrized
by a \emph{mode theory}, which is used to describe the structure of
contexts and the structural properties they obey.  In this sequent
calculus, the context itself obeys standard structural properties, while
a term, drawn from the mode theory, constrains how the context can be
used.  Product types, implications, and modalities are defined as
instances of two general connectives, one positive and one negative,
that manipulate these terms.  Specific mode theories can express a range
of substructural and modal connectives, including non-associative,
ordered, linear, affine, relevant, and cartesian products and
implications; monoidal and non-monoidal functors, (co)monads and
adjunctions; n-linear variables; and bunched implications.  We prove cut
(and identity) admissibility independently of the mode theory, obtaining
it for many different logics at once. Further, we give a general
equational theory on derivations / terms that, in addition to the usual
$\beta\eta$-rules, characterizes when two derivations differ only by the
placement of structural rules.  Additionally, we give an equivalent
semantic presentation of these ideas, in which a mode theory corresponds
to a 2-dimensional cartesian multicategory, the framework corresponds to
another such multicategory with a functor to the mode theory, and the
logical connectives make this into a bifibration.  Finally, we show how
the framework can be used both to encode existing existing logics / type
theories and to design new ones.
\end{abstract}

%% ----------------------------------------------------------------------

\section{Introduction}

In ordinary intuitionistic logic or $\lambda$-calculus, assumptions or
variables can go unused (weakening), be used in any order (exchange), be
used more than once (contraction), and be used in any position in a
term.  \emph{Substructural} logics, such as linear logic, ordered logic,
relevant logic, and affine logic, omit some of these structural
properties of weakening, exchange, and contraction, while \emph{modal
  logics} place restrictions on where variables may be used---e.g. a
formula $\Bx{} C$ can only be proved using assumptions of $\Bx{} A$,
while an assumption of $\Dia{}{A}$ can only be used when the conclusion
is $\Dia{}{C}$.  Substructural and modal logics have had many
applications to both functional and logic programming, modeling concepts
such state, staging, distribution, and concurrency.  They are also used
as \emph{internal languages} of categories, where one uses an
appropriate logical language to do constructions ``inside'' a particular
mathematical setting, which often results in shorter statements than
working ``externally''.  For example, to define a function externally in
domains, one must first define the underlying set-theoretic function,
and then prove that it is continuous; when using untyped
$\lambda$-calculus as an internal language of domains, one writes what
looks like only the function part, and continuity follows from a general
theorem about the language itself.  Substructural logics extend this
idea to various forms of monoidal categories, while modal logics
describe monads and comonads.  Recent
work~\cite{schreibershulman12cohesive,shulman15realcohesion} proposed
using modal operators to add a notion of \emph{cohesion} to homotopy
type theory/univalent
foundations~\cite{voevodsky06homotopy,uf13hott-book}.  Without going
into the precise details, the idea is to add a triple $\sh{} \la \Flat{}
\la \Sharp{}$ of type operators, where for example $\Sharp{}$ and
$\sh{}$ are monads (like a modal possibility $\Diamond$ or $\bigcirc$),
$\Flat{}$ is a comonad (like a modal necessity $\Box$), and there is an
adjunction structure between them ($\flat{A} \to B$ is the same as $A
\to \Sharp{B}$).  This raised the question of how to best add modalities
with these properties to type theory.

Because other similar applications rely on functors with different
properties, we would like general tools for going from a semantic
situation of interest to a well-behaved logic/type theory for
it---e.g. one with cut admissibility / normalization and identity
admissibility / $\eta$-expansion.  In previous work~\cite{ls16adjoint},
we considered the special case of a single-assumption logic, building
most directly on adjoint
logic~\cite{benton94mixed,bentonwadler96adjoint,reed09adjoint}.  Here we
extend this previous work to the multi-assumption case.  The resulting
framework is quite general and covers many existing intuitionistic
substructural and modal connectives: non-associative, ordered, linear,
affine, relevant, and cartesian products and implications; combinations
thereof such as bunched logic~\cite{ohearnpym99bunched} and resource
separation~\cite{atkey04separation}; $n$-linear
variables~\cite{reed08namessubstructural,abel15modal,mcbride16nuttin};
the comonadic $\Box$ and linear exponential $!$ and
subexponentials~\cite{nigammiller09subexponentials,danos+93subexponentials};
monadic $\Diamond$ and $\bigcirc$ modalities; and adjoint logic $F$ and
$G$~\cite{benton94mixed,bentonwadler96adjoint,reed09adjoint}, including
the single-assumption 2-categorical version from our previous
work~\cite{ls16adjoint}.  A central syntactic result is that cut and
identity are admissible for our framework itself, and this implies cut
admissibility for any logic that can be described in the framework,
including all of the above, as well as any new logics that one designs
using it.  When we view the derivations in the framework as terms in a
type theory, this gives an immediate normalization (and
$\eta$-expansion) result.  Our focus here is on propositional,
single-conclusioned substructural and modal logics, leaving extensions
to quantifiers, multi-conclusioned logics, and dependent types to future
work.

At a high level, the framework makes use of the fact that all of the
above logics / type theories are a restriction on how variables can be
used in ordinary structural/cartesian proofs.  We express these
restrictions using a first layer, which is a simple type theory for what
we will call \emph{modes} and \emph{context descriptors}.  The modes are
just a collection of base types, which we write as $p,q,r$, while a
context descriptor $\alpha$ is a first-order term built from variables
and function symbols.  The next layer is the main logic.  Each
proposition/type is assigned a mode, and the basic sequent is \seq{x_1 :
  A_1, \ldots, x_n : A_n}{\alpha}{C}, where if $A_i$ has mode $p_i$, and
$C$ has mode $q$, then $\oftp{x_1 : p_1,\ldots, x_n : p_n}{\alpha}{q}$.
We use a sequent calculus to concisely describe cut-free
derivations/normal forms, but everything can be translated to natural
deduction.  We write ${\Gamma}$ for $x_1 : A_1, \ldots, x_n : A_n$, and
$\Gamma$ itself behaves like an ordinary structural/cartesian context,
while the substructural and modal aspects are enforced by the
\emph{term} $\alpha$, which constrains how the resources from $\Gamma$
may be used.  For example, in linear logic/ordered logic/BI, the context
is usually taken to be a multiset/list/tree.  We represent this by a
pair of an ordinary structural context $\Gamma$, together with a term
$\alpha$ that describes the multiset or list or tree structure, labeled
with variables from the ordinary context at the leaves.  We pronounce
\seq{\Gamma}{\alpha}{A} as ``$\Gamma$ proves $A$ \{along,over,using\}
$\alpha$''.

For example, if we have a mode $\dsd{n}$, together with a context
descriptor constant $x : \dsd{n}, y:\dsd{n} \vdash x \odot y : \dsd{n}$,
then an example sequent \seq{x:A, y:B, z:C, w:D}{(y \odot x) \odot z}{E}
should be read as saying that we must prove $E$ using the resources $y$
and $x$ and $z$ (but not $w$) according to the particular tree structure
${(y \odot x) \odot z}$.  If we say nothing else, the framework will
treat $\odot$ as describing a non-associative, linear, ordered
context~\cite{lambek58calculus}: if we have a product-like type $A \odot B$ 
internalizing this context operation,\footnote{We overload
  binary operations to refer both to context descriptors and
  propositional connectives, relying on metavariables ($\alpha_1 \odot
  \alpha_2$ vs. $A_1 \odot A_2$) to distinguish them.}
then we will \emph{not} be able to prove associativity ($(A \odot B)
\odot C \dashv\vdash A \odot (B \odot C)$) or exchange ($A \odot B
\vdash B \odot A$) etc.  
To get from this basic structure to a linear or affine or relevant or
cartesian system, we provide a way to add structural properties governing
the context descriptor term $\alpha$.  We analyze structural properties
as \emph{equations}, or more generally \emph{directed transformations},
on such terms.  For example, to specify linear logic, we will add a unit
element $1 : \dsd{n}$ together with equations making $(\odot,1)$ into a
commutative monoid ($x \odot (y \odot z) = (x \odot y) \odot z$ and 
$x \odot 1 = x = 1 \odot x$ and 
$x \odot y = y \odot x$)
so that the context descriptors ignore associativity and order.  To get
BI, we add an additional commutative monoid $(\times,\top)$ (with
weakening and contraction, as discussed below), so that a BI context
tree $(x:A,y:B);(z:C,w:D)$ can be represented by the ordinary context
$x:A,y:B,z:C,w:D$ with the term $(x \odot y) \times (z \odot w)$
describing the tree.  Because the context descriptors are themselves
ordinary structural/cartesian terms, the same variable can occur more
than once or not at all.  A descriptor such as $x \odot x$ captures the
idea that we can use the \emph{same} variable $x$ twice, expressing
$n$-linear types.  Thus, we can express contraction for a particular
context descriptor $\odot$ as a transformation $x \spr x \odot x$ (one
use of $x$ allows two).  Weakening, on the other hand, is represented by
a transformation $x \spr 1$, which is oriented to allow throwing away an
allowed use of $x$, but not creating an allowed use from nothing.  We
refer to these as \emph{structural transformations}, to evoke their use
in representing the structural properties of object logics that are
embedded in our framework.  The main sequent $\seq{\Gamma}{\alpha}{A}$
respects the specified structural properties in the sense that when
$\alpha = \beta$, we regard $\seq{\Gamma}{\alpha}{A}$ and
$\seq{\Gamma}{\beta}{A}$ as the same sequent (so a derivation of one is
a derivation of the other), while when $\alpha \spr
\beta$, there will be an operation that takes a derivation of
\seq{\Gamma}{\beta}{A} to a derivation of
\seq{\Gamma}{\alpha}{A}---i.e. uses of transformations are explicitly
marked in the term.  

Modal logics will generally involve a mode theory with more than one
mode.  For example, a context descriptor $x : \dsd{c} \vdash \dsd{f}(x)
: \dsd{l}$ will generate an adjoint pair of functors between the two
modes, as in the adjoint syntax for linear logic's
$!$~\cite{bentonwadler96adjoint} or other modal
operators~\cite{reed09adjoint}.  Using this, a context descriptor
$\dsd{f}(x) \odot y$ expresses permission to use $x$ in a cartesian way
and $y$ in a linear way.  Structural transformations are used to
describe how these modal operators interact with each other and with the
products, and for some systems~\cite{ls16adjoint} it is important that
there can be more than one transformation between a given pair of
context descriptors.

A guiding principle of the framework is a meta-level notion of
\emph{structurality over structurality}.  For example, we always have
\emph{weakening over weakening}: if \seq{\Gamma}{\alpha}{A} then
\seq{\Gamma,y:B}{\alpha}{A}, where $\alpha$ itself is weakened with $y$.
This does not prevent encodings of relevant logics: though we might
weaken a derivation of \seq{\Gamma}{x_1 \odot \ldots \odot x_n}{A}
(``use $x_1$ through $x_n$'') to a derivation of \seq{\Gamma,y:B}{x_1
  \odot \ldots \odot x_n}{A}, the (weakened) context descriptor does not
allow the use of $y$.  Similarly, we have exchange over exchange and
contraction over contraction.  The \emph{identity-over-identity}
principle says that we should be able to prove $A$ using exactly an
assumption $x:A$ ({\seq{\Gamma,x:A}{x}{A}}).  The cut principle says
that from \seq{\Gamma,x:A}{\beta}{B} and \seq{\Gamma}{\alpha}{A} we get
{\seq{\Gamma}{\subst{\beta}{\alpha}{x}}{B}}---the context descriptor for
the result of the cut is the substitution of the context descriptor used
to prove $A$ into the one used to prove $B$.  For example, together with
weakening-over-weakening, this captures the usual cut principle of
linear logic, which says that cutting $\Gamma,x:A \vdash B$ and $\Delta
\vdash A$ yields $\Gamma,\Delta \vdash B$: if $\Gamma$ binds
$x_1,\ldots,x_n$ and $\Delta$ binds $y_1,\ldots,y_n$, then we will
represent the two derivations to be cut together by sequents with $\beta
= x_1 \odot \ldots \odot x_n \odot x$ and $\alpha = y_1 \odot \ldots
\odot y_n$, so $\beta[\alpha/x] = x_1 \odot \ldots \odot x_n \odot y_1
\odot \ldots \odot y_n$ correctly deletes $x$ and replaces it with the
variables from $\Delta$.  In more subtle situations such as BI, the
substitution will insert the resources used to prove the cut formula in
the correct place in the tree.  Our cut algorithm follows cut
admissibility for structural (cartesian) intuitionistic
logic~\cite{pfenning94cut}, and applies the same cut reductions to all
mode theories.  In substructural situations, certain portions of this
general algorithm are unnecessary, but not harmful.  For example, in a
setting without contraction, our algorithm will recursively cut into all
premises of a rule, even though the variable can only occur in one
premise---but the extra recursive cuts for variables that do not occur
will leave the derivation unchanged, as desired.

The framework has two main logical connectives / type constructors.  The
first, \F{\alpha}{\Delta}, generalizes the left-adjoint \dsd{F} of
adjoint logic and the multiplicative products (e.g. $\otimes$ of linear
logic).  The second, \U{x.\alpha}{\Delta}{A}, generalizes the
right-adjoint $\dsd{G}/\dsd{U}$ of adjoint logic and implication
(e.g. $A \lolli B$ in linear logic).  Here $\Delta$ is a context of
assumptions $x_i:A_i$, and trivializing the context descriptors
(i.e. adding an equation $\alpha = \beta$ for all $\alpha$ and $\beta$)
degenerates $\F{\alpha}{\Delta}$ into the ordinary intuitionistic
product $A_1 \times \ldots \times A_n$, while \U{x.\alpha}{\Delta}{A}
becomes $A_1 \to \ldots \to A_n \to A$.  As one would expect, \dsd{F} is
left-invertible and \dsd{U} is right-invertible.  In linear logic terms,
our \dsd{F} and \dsd{U} cover both the multiplicatives and exponentials;
additives can be defined separately by the usual rules.  Moreover, though
\dsd{F} and \dsd{U} form an adjoint pair, the subset of derivations that
use either $\dsd{F}_\alpha$ or $\dsd{U}_\alpha$ but not both (for a
particular $\alpha$) describe constructions on functors that do not have
an adjoint.  We discuss many examples of \emph{logical adequacy}
theorems, showing that a sequent can be proved in a standard sequent
calculus for a logic iff its embedding using these connectives can be
proved in the framework.

Being a very general theory, our framework treats the object-logic
structural properties in a general but na\"ive way, allowing an
arbitrary structural transformation to be applied at the non-invertible
rules for $\dsd{F}$ and $\dsd{U}$ and at the leaves of a derivation.
For specific embedded logics, there is often a more refined discipline
that suffices---e.g. for cartesian logic, always contract all
assumptions in all premises, and only weaken at the leaves.  We view our
framework as a tool for bridging the gap between an intended semantic
situation (in the cohesion example mentioned, ``a comonad and a
monad which are themselves adjoint'') and a proof theory: the framework
gives \emph{some} proof theory for the semantics, and the placement of
structural rules can then be optimized purely in syntax.  To support
this mode of use, we give an equational theory on derivations/terms
that identifies different placements of the same structural rules.  This
can be used to prove correctness of such optimizations not just at the
level of provability, but also identity of derivations---which matters
for our intended applications to internal languages.  We discuss some
preliminary work on \emph{equational adequacy}, which extends the
logical correspondence to isomorphisms of definitional-equality-classes
of derivations.

Semantically, the logic corresponds to a functor between
\emph{2-dimensional cartesian multicategories} which is a fibration in
various senses.  Multicategories are a generalization of categories
which allow more than one object in the domain of morphisms, and
cartesianness means that the multiple domain objects are treated
structurally.  The 2-dimensionality supplies a notion of morphism
between (multi)morphisms.  A \emph{mode theory} specifying context
descriptors and structural properties is analyzed as a cartesian
2-multicategory, with the descriptors as 1-cells and the structural
properties as 2-cells.  The functor relates the sequent judgement to the
mode theory, specifying the mode of each proposition and the context
descriptor of a sequent.  The fibration conditions (similar to
\cite{hermida02fibrations,hormann15multicategories}) give respect for
the structural transformations and the presence of \dsd{F} and \dsd{U}
types.  We prove that the sequent calculus and the equational theory are
sound and complete for this semantics: the syntax can be interpreted in
any bifibration, and itself determines one.  This semantics shows that
an interesting class of type theories can be identified with a class of
more mathematical objects, fibrations of cartesian 2-multicategories,
thus providing some progress towards characterizing substructural and
modal type theories in mathematical terms.

In Section~\ref{sec:syntax}, we present the syntax of the framework.  In
Section~\ref{sec:exampleencodings}, we discuss how a number of logics
are represented.  In Section~\ref{sec:equational}, we give the
$\beta\eta$-equational theory on derivations.  In
Section~\ref{sec:semantics}, we discuss the framework's categorical
semantics.  Proofs (of cut and identity admissibility, soundness and
completeness of the semantics, and adequacy of encodings) and additional
examples (subexponentials, modal S4 $\Box$, and strong/$\Box$-strong
monads) are available in an extended version of this paper~\cite{lsr17multi-extended}.

%% ----------------------------------------------------------------------

\newcommand\wfsp[4]{\ensuremath{#1 \vdash #2 \spr_{#4} #3}}

\section{Sequent Calculus}
\label{sec:syntax}

\newcommand\wfsig[1]{\ensuremath{#1 \, \dsd{sig}}}
\newcommand\deqtms[5]{\ensuremath{#1 \vdash_{#2} #3 \deq #4 : #5}}
\newcommand\wfsps[5]{\ensuremath{#1 \vdash_{#2} #3 \spr_{#5} #4}}

\subsection{Mode Theories}

\begin{figure}
\begin{small}
\[
\begin{array}{l}
\framebox{Signatures \wfsig{\Sigma}}
\qquad
\infer{\wfsig{\cdot}}
      {}
\qquad
\infer{\wfsig{(\Sigma,p \, \dsd{mode})}}
      {\wfsig{\Sigma}}
\qquad
\infer{\wfsig{(\Sigma,c : \, p_1,\ldots,p_n \to q)}}
      {\wfsig{\Sigma} &
        (p_1 \, \dsd{mode},\ldots,p_n \, \dsd{mode},q \, \dsd{mode}) \in \Sigma
      }
\\ \\
\infer{\wfsig{(\Sigma, (\alpha \deq \alpha' : \psi \to p))}}
      {\wfsig{\Sigma} &
        \vdash_\Sigma  \psi \, \dsd{ctx} & 
        p \, \dsd{mode} \in \Sigma &
        \oftps{\psi}{\Sigma}{\alpha}{p} & 
        \oftps{\psi}{\Sigma}{\alpha'}{p} 
      }
\qquad
\infer{\wfsig{(\Sigma, (\alpha \spr \alpha' : \psi \to p))}}
      {\wfsig{\Sigma} &
        \vdash_\Sigma \psi \, \dsd{ctx} & 
        p \, \dsd{mode} \in \Sigma &
        \oftps{\psi}{\Sigma}{\alpha}{p} & 
        \oftps{\psi}{\Sigma}{\alpha'}{p} 
      }
\\\\
\framebox{Context descriptors \oftps{\psi}{\Sigma}{\alpha}{p},
  where $\vdash_\Sigma  \psi \, \dsd{ctx}$ and $p \, \dsd{mode} \in \Sigma$}
\qquad
\infer{\oftps{\psi}{\Sigma}{x}{p}}
      {x:p \in \psi}
\quad
\infer{\oftps{\psi}{\Sigma}{\dsd{c}(\alpha_1,\ldots,\alpha_n)}{q}}
      {(\dsd{c} : p_1,\ldots,p_n \to q) \in \Sigma &
       \oftps{\psi}{\Sigma}{\alpha_i}{p_i}
      }
\\\\
\framebox{Mode Substitutions \oftps{\psi}{\Sigma}{\gamma}{\psi'}, where
  $\vdash_\Sigma  \psi \, \dsd{ctx}$ and $\vdash_\Sigma \psi' \, \dsd{ctx}$ }
\qquad
\infer{\oftps{\psi}{\Sigma}{\cdot}{\cdot}}{}
\qquad
\infer{\oftps{\psi}{\Sigma}{\gamma,\alpha/x}{\psi',x:p}}
      {\oftps{\psi}{\Sigma}{\gamma}{\psi'} &
        \oftps{\psi}{\Sigma}{\alpha}{p}}
\\\\
\framebox{Structural transformations \wfsps{\psi}{\Sigma}{\alpha}{\alpha'}{p},
where \oftps{\psi}{\Sigma}{\alpha}{p}
and \oftps{\psi}{\Sigma}{\alpha'}{p}
}
\qquad
\infer{\wfsps{\psi}{\Sigma}{\alpha}{\alpha}{p}}{}
\\\\
\infer{\wfsps{\psi}{\Sigma}{\alpha_1}{\alpha_3}{p}}
      {\wfsps{\psi}{\Sigma}{\alpha_1}{\alpha_2}{p} &
       \wfsps{\psi}{\Sigma}{\alpha_2}{\alpha_3}{p} &
      }
\qquad
\infer{\wfsps{\psi,\psi'}{\Sigma}{\subst{\beta}{\alpha}{x}}{\subst{\beta'}{\alpha'}{x}}{q}}
      {\wfsps{\psi,x:p,\psi'}{\Sigma}{\beta}{\beta'}{q} &
       \wfsps{\psi,\psi'}{\Sigma}{\alpha}{\alpha'}{p}}
\qquad
\infer{\wfsps{\psi}{\Sigma}{\alpha}{\alpha'}{p}}
      {(\alpha \spr \alpha' : \psi \to p) \in \Sigma}
\end{array}
\]
\end{small}
\caption{Syntax for mode theories}
\label{fig:2multicategory}
\end{figure}

The first layer of our framework is a type theory whose types we will
call \emph{modes}, and whose terms we will call \emph{context
  descriptors} or \emph{mode morphisms}.  To a first approximation,
context descriptors are multi-sorted first-order terms with equality and
``less than or equal to'' relations.  The only modes are atomic/base
types $p$.  A term is either a variable (bound in a context $\psi$) or a
typed $n$-ary constant (function symbol) \dsd{c} applied to terms of the
appropriate types.  Two terms may be equal, written $\alpha \deq \beta$,
or $\alpha$ may be stronger than $\beta$, written $\alpha \spr \beta$.  

This is formalized in the notion of signature, or \emph{mode theory},
defined in Figure~\ref{fig:2multicategory}.  The judgement $\wfsig
\Sigma$ means that $\Sigma$ is a well-formed signature.  The top line
says that a signature is either empty, or a signature extended with a
new mode declaration, or a signature extended with a typed
constant/function symbol, all of whose modes are declared previously in
the signature.  The notation $p_1,\ldots,p_n \to q$ is not itself a
mode, but notation for declaring a function symbol in the signature (it
cannot occur on the right-hand side of a typing judgement).  For
example, the type and term constructors for a monoid $(\odot,1)$ are
represented by a signature $\dsd{p} \, \dsd{mode}, \dsd{\odot} :
(\dsd{p},\dsd{p} \to \dsd{p}), 1 : (\to \dsd{p})$.

We elide the rules for the judgement $\vdash_\Sigma \psi \, \dsd{ctx}$, which
simply says that each mode used in the context of variable declarations
$\psi$ is declared in $\Sigma$.  The judgement
$\oftps{\psi}{\Sigma}{\alpha}{p}$ defines well-typedness of context
descriptor terms, which are either a variable declared in the context,
or a constant declared in the signature applied to arguments of the
correct types.  The judgement $\oftps{\psi}{\Sigma}{\gamma}{\psi'}$
defines a substitution as a tuple of terms in the standard way.  The
context $\psi$ in these judgements enjoys the cartesian structural
properties (associativity, unit, weakening, exchange, contraction).
Simultaneous substitution into terms and substitutions is defined as
usual (e.g.  $x[\gamma,\alpha/x] := \alpha$ and
$\dsd{c}(\vec{\alpha_i})[\gamma] := \dsd{c}(\vec{\alpha_i[\gamma]})$).

Returning to the top of the figure, the final two rules of the judgement
$\wfsig{\Sigma}$ permit two additional forms of signature declaration.
The first of these extends a signature with an equational axiom between
two terms $\alpha$ and $\alpha'$ that have the same mode $p$, in the
same context $\psi$, relative to the prior signature $\Sigma$.  These
equational axioms will be used to encode reversible object language
structural properties, such as associativity, commutativity, and unit
laws.  For example, to specify the right unit law for the above monoid
$(\odot,1)$, we add an axiom $(x \odot 1 \deq x : (x : \dsd{p}) \to
\dsd{p})$ to the signature, which can be read as ``$x \odot 1$ is equal
to $x$ as a morphism from $(x : \dsd{p})$ to \dsd{p}''.  The judgement
\deqtms{\psi}{\Sigma}{\alpha}{\alpha'}{p} (omitted from the figure; the
rules are the same as for $\spr$ plus symmetry) is the least congruence
closed under these axioms.

The second of these extends a signature with a directed structural
transformation axiom between two terms $\alpha$ and $\alpha'$ that have
the same mode $p$, in the same context $\psi$, relative to the prior
signature $\Sigma$.  As discussed above, these structural
transformations will be used to represent object language structural
properties such as weakening and contraction that are not invertible.
The judgement \wfsps{\psi}{\Sigma}{\alpha}{\alpha'}{p} defines these
transformations: it is the least precongruence (preorder compatible with
the term formers) closed under the axioms specified in the signature
$\Sigma$.  For example, to say that the above monoid $(\odot,1)$ is
affine, we add in $\Sigma$ a transformation axiom $(x \spr 1 : (x:\dsd{p}) \to
{\dsd{p}})$.

Because context descriptors $\alpha$ and their equality $\alpha_1 \deq
\alpha_2$ are defined prior to the subsequent judgements, we suppress
this equality by using $\alpha$ to refer to a term-modulo-\deq---that
is, we assume a metatheory with quotient sets/types, and use meta-level
equality for object-level equality~\cite{altenkirchkaposi16qit}.  For
example, because the judgement \wfsp{\psi}{\alpha}{\beta}{p} is indexed
by equivalence classes of context descriptions, the reflexivity rule
above implicitly means $\alpha \deq \beta$ implies $\alpha \spr \beta$.
In examples, we will notate a signature declaration introducing a term
constant/function symbol by showing the function symbol applied to
variables, rather than writing the formal $\dsd{c} : p_1,\ldots,p_n \to
q$. For example, we write $x : \dsd{p}, y : \dsd{p} \vdash x \odot y :
\dsd{p}$ for $\odot : \dsd{p},\dsd{p} \to \dsd{p}$.  We also suppress
the signature $\Sigma$.

\subsection{Sequent Calculus Rules}

\begin{figure}
\begin{small}
\[
\begin{array}{l}
%% \begin{array}{llll}
%% \text{Types} & A & ::= & P \mid \F{\alpha}{\Delta} \mid \U{\alpha}{\Delta}{A} \\
%% \end{array}
%% \\ \\
\framebox{Types $A,B,C$ \quad \wftype{A}{p}}
\qquad
\infer{\wftype{P}{p}}{}
\qquad
\infer{\wftype{\F{\alpha}{\Delta}}{q}}
      {\oftp{\psi}{\alpha}{q} &
        \wfctx{\Delta}{\psi}}
\qquad
\infer{\wftype{\U{x.\alpha}{\Delta}{A}}{q}}
      {\oftp{\psi,x:q}{\alpha}{p} &
        \wfctx{\Delta}{\psi} &
        \wftype{A}{p}
      }
\\ \\
\framebox{Contexts $\Gamma,\Delta$ \quad \wfctx{\Gamma}{\psi}}
\qquad
\infer{\wfctx{\cdot}{\cdot}}{}
\qquad
\infer{\wfctx{\Gamma,x:A}{\psi,x:p}}
      {\wfctx{\Gamma}{\psi} &
        \wftype{A}{p}}
\\ \\
\framebox{\seq{\Gamma}{\alpha}{A} where $\wfctx{\Gamma}{\psi}$ and $\wftype{A}{q}$ and  $\oftp{\psi}{\alpha}{q}$}
\quad
\infer[\FL]{\seq{\Gamma,x:\F{\alpha}{\Delta},\Gamma'}{\beta}{C}}
      {\seq{\Gamma,\Gamma',\Delta}{\subst \beta {\alpha}{x}}{C}}
\quad
\infer[\FR]{\seq{\Gamma}{\beta}{\F{\alpha}{\Delta}}}
      {%% \modeof{\Gamma} \vdash \gamma : \modeof{\Delta} & 
        \beta \spr \tsubst{\alpha}{\gamma} &
        \seq{\Gamma}{\gamma}{\Delta} 
      }
%% \infer{\seq{\Gamma}{\beta}{C}}
%%       {{x}:{\F{\alpha}{\Delta}} \in \Gamma & 
%%         \oftp{\modeof{\Gamma},{x'} : {\modeof{\F{\alpha}{\Delta}}}}{\beta'}{\modeof{C}} &
%%         \beta \deq \tsubst{\beta'}{x/x'} &
%%         \seq{\Gamma,\Delta}{\subst {\beta'} {\alpha}{x'}}{C}}
\\ \\
\infer[\UL]{\seq{\Gamma}{\beta}{C}}
      {\begin{array}{llll}
          x:\U{x.\alpha}{\Delta}{A} \in \Gamma &
          \beta \spr \subst{\beta'}{\tsubst{\alpha}{\gamma}}{z} &
          \seq{\Gamma}{\gamma}{\Delta} &
          \seq{\Gamma,\tptm{z}{A}}{\beta'}{C}
       \end{array}
      }
\quad
\infer[\UR]{\seq{\Gamma}{\beta}{\U{x.\alpha}{\Delta}{A}}}
      {\seq{\Gamma,\Delta}{\subst{\alpha}{\beta}{x}}{A}}
\quad
\infer[\dsd{v}]{\seq{\Gamma}{\beta}{P}}
      {x:P \in \Gamma & \beta \spr x}
\\ \\
\framebox{\seq{\Gamma}{\gamma}{\Delta} where $\wfctx{\Gamma}{\psi}$ and $\wfctx{\Delta}{\psi'}$ and  $\oftp{\psi}{\gamma}{\psi'}$}
\qquad
\infer[\cdot]{\seq{\Gamma}{\cdot}{\cdot}}
      {}
\qquad
\infer[\_,\_]{\seq{\Gamma}{\gamma,\alpha/x}{\Delta,x:A}}
      {\seq{\Gamma}{\gamma}{\Delta} &
       \seq{\Gamma}{\alpha}{A}
      }
\end{array}
\]    
\caption{Sequent Calculus}
\label{fig:sequent}
\hrule
\end{small}
\end{figure}

For a fixed mode theory $\Sigma$, we define a second layer of judgements
in Figure~\ref{fig:sequent}.  The first judgement assigns each
proposition/type $A$ a mode $p$.  Encodings of non-modal logics will
generally only make use of one mode, while modal logics use different
modes to represent different notions of truth, such as the linear and
cartesian categories in the adjoint decomposition of linear
logic~\cite{benton94mixed,bentonwadler96adjoint} and the true/valid/lax
judgements in modal logic~\cite{pfenningdavies}.  The next judgement
assigns each context $\Gamma$ a mode context $\psi$.  Formally, we think
of contexts as ordered: we do not regard $x:A,y:B$ and $y:B,x:A$ as the
same context, though we will have an admissible exchange rule that
passes between derivations in one and the other.

The sequent judgement \seq{\Gamma}{\alpha}{A} relates a context
$\wfctx{\Gamma}{\psi}$ and a type $\wftype{A}{p}$ and context descriptor
\oftp{\psi}{\alpha}{p}, while the substitution judgement \seq{\Gamma}{\gamma}{\Delta} relates
$\wfctx{\Gamma}{\psi}$ and $\wfctx{\Delta}{\psi'}$ and
$\oftp{\psi}{\gamma}{\psi'}$. Because $\wfctx{\Gamma}{\psi}$ means that
each variable in $\Gamma$ is in $\psi$, where $x : A_i \in \Gamma$
implies $x : p_i$ in $\psi$ with \wftype{A_i}{p_i}, we think of $\Gamma$
as binding variable names both in $\alpha$ and for use in the
derivation.

We now explain the rules for the sequent calculus; the reader may wish
to refer to the examples in Section~\ref{sec:exampleencodings} in
parallel with this abstract description.  We assume atomic propositions
$P$ are given a specified mode $p$, and state identity as a primitive
rule only for them with the \dsd{v} rule.  This says that
\seq{\Gamma,x:P}{x}{P}, and additionally composes with a structural
transformation $\beta \spr x$.  Using a structural property at a leaf of
a derivation is common in e.g. affine logic, where the derivation of
$\beta \spr x$ would use weakening to forget any additional resources
besides $x$.

Next, we consider the \F{\alpha}{\Delta} type, which ``internalizes''
the context operation $\alpha$ as a type/proposition.  Syntactically, we
view the context $\Delta = x_1:A_1,\ldots,x_n:A_n$ where
\wftype{A_i}{p_i} as binding the variables $x_i:p_i$ in $\alpha$, so for
example \F{\alpha}{x:A,y:B} and \F{\alpha[x \leftrightarrow
    x']}{x':A,y:B} are $\alpha$-equivalent types (in de Bruijn form we
would write \F{\alpha}{A_1,\ldots,A_n} and use indices in $\alpha$).
The type formation rule says that \dsd{F} moves covariantly along a mode
morphism $\alpha$, representing a ``product'' (in a loose sense) of the
types in $\Delta$ structured according to the context descriptor
$\alpha$. A typical binary instance of \dsd{F} is a multiplicative
product ($A \otimes B$ in linear logic), which, given a binary context
descriptor $\odot$ as in the introduction, is written \F{x \odot
  y}{x:A,y:B}.  A typical nullary instance is a unit (1 in linear
logic), written \F{1}{}.  A typical unary instance is the \dsd{F}
connective of adjoint logic, which for a unary context descriptor
constant $\dsd{f} : \dsd{p} \to \dsd{q}$ is written \F{\dsd{f}(x)}{x:A}.
We sometimes write \F{\dsd{f}}{A} in this case, eliding the variable
name, and similarly for a unary \dsd{U}.

The rules for our \dsd{F} connective capture a pattern common to all of
these examples.  The left $\FL$ rule says that \F{\alpha}{\Delta}
``decays'' into $\Delta$, but structuring the uses of resources in
  $\Delta$ with $\alpha$ by the substitution \subst{\beta}{\alpha}{x}.
We assume that $\Delta$ is $\alpha$-renamed to avoid collision with
$\Gamma$ (the proof term here is a ``\dsd{split}'' that binds
variables for each position in $\Delta$).  The placement of $\Delta$ at
the right of the context is arbitrary (because we have
exchange-over-exchange), but we follow the convention that new variables
go on the right to emphasize that $\Gamma$ behaves mostly as in ordinary
cartesian logic.  The right \FR\/ rule says that you must rewrite (using
structural transformations) the context descriptor to have an $\alpha$
at the outside, with a mode substitution $\gamma$ that divides the
existing resources up between the positions in $\Delta$, and then prove
each formula in $\Delta$ using the specified resources.  We leave the
typing of $\gamma$ implicit, though there is officially a requirement
$\oftp{\psi}{\gamma}{\psi'}$ where $\wfctx{\Gamma}{\psi}$ and
$\wfctx{\Delta}{\psi'}$, as required for the second premise to be a
well-formed sequent.  Another way to understand this rule is to begin
with the ``axiomatic \FR'' instance 
$\FR^* :: {\seq{\Delta}{\alpha}{\F{\alpha}{\Delta}}}{}$
which says that there is a map from $\Delta$ to \F{\alpha}{\Delta} along
$\alpha$.  Then, in the same way that a typical right rule for
coproducts builds a precomposition into an ``axiomatic injection'' such
as $\dsd{inl} :: A \vdash A + B$, the \FR\/ rule builds a precomposition
with $\seq{\Gamma}{\gamma}{\Delta}$ and then an application of a
structural rule $\beta \spr \alpha[\gamma]$ into the ``axiomatic''
version, in order to make cut and respect for transformations
admissible.

Next, we turn to $\U{x.\alpha}{\Delta}{A}$.  As a first approximation,
if we ignore the context descriptors and structural properties,
\U{-}{\Delta}{A} behaves like $\Delta \to A$, and the \UL\/ and \UR\/
rules are an annotation of the usual structural/cartesian rules for
implication.  In a formula \U{x.\alpha}{\Delta}{A}, the context
descriptor $\alpha$ has access to the variables from $\Delta$ as well as
an extra variable $x$, whose mode is the same as the \emph{overall mode
  of \U{x.\alpha}{\Delta}{A}}, while the mode of $A$ itself is the mode
of the conclusion of $\alpha$---in terms of typing, \dsd{U} is
contravariant where \dsd{F} is covariant.  It is helpful to think of $x$
as standing for the context that will be used to prove
\U{x.\alpha}{\Delta}{A}.  For example, a typical function type $A \lolli
B$ is represented by \U{x.x \otimes y}{y:A}{B}, which says to extend the
``current context'' $x$ with a resource $y$.  In \UR, the context
descriptor $\beta$ being used to prove the \dsd{U} is substituted
\emph{for $x$} in $\alpha$ (dual to \FL, which substituted $\alpha$ into
$\beta$).  The ``axiomatic'' \UL\/ instance
$\UL^* :: {\seq{\Delta,x:\U{x.\alpha}{\Delta}{A}}{\alpha}{A}}$
says that \U{x.\alpha}{\Delta}{A} together with $\Delta$ has a map to
$A$ along $\alpha$.  (The bound $x$ in $x.\alpha$ subscript is tacitly
renamed to match the name of the assumption in the context, in the same
way that the typing rule for $\lambda x.e : \Pi x:A.B$ requires
coordination between two variables in different scopes).  The full rule
builds in precomposition with \seq{\Gamma}{\gamma}{\Delta},
postcomposition with \seq{\Gamma,z:A}{\beta'}{C}, and precomposition
with $\beta \spr \beta'[\alpha[\gamma]/z]$.

Finally, the rules for substitutions are pointwise.  In examples, we
will write the components of a substitution directly as multiple
premises of \FR\/ and \UL\/, rather than packaging them with 
$\_,\_$ and $\cdot$.

For additives, the context descriptor is not modified; for example, a
coproduct/disjunction $\wftype{A_p + B_p}{p}$ for a mode $p$ is given by
the following rules:
\[
\begin{array}{c}
\infer{\seq{\Gamma}{\alpha}{A + B}}
      {\seq{\Gamma}{\alpha}{A}}
\quad
\infer{\seq{\Gamma}{\alpha}{A + B}}
      {\seq{\Gamma}{\alpha}{B}}
\quad
\infer{\seq{\Gamma,x:A+B,\Gamma'}{\beta}{C}}
      {\seq{\Gamma,\Gamma',y:A}{\subst \beta y x}{C} &
       \seq{\Gamma,\Gamma',z:B}{\subst \beta z x}{C} 
      }
\end{array}
\]

%% ----------------------------------------------------------------------

Our framework enjoys the following admissible structural rules:

\begin{theorem}[Admissibility of cut, identity,
    structurality-over-structurality, and respect for 2-cells]
The following rules are admissible:
\[
\begin{array}{c}
\infer{\seq{\Gamma}{\subst{\beta}{\alpha}{x}}{B}}
    {\seq{\Gamma,x:A}{\beta}{B} &
     \seq{\Gamma}{\alpha}{A}}
\quad
\infer{\seq{\Gamma,x:A}{x}{A}}{}
\quad
\infer{\seq{\Gamma,y:A}{\alpha}{C}}
      {\seq{\Gamma}{\alpha}{C}}
\quad
\infer{\seq{\Gamma,y:B,x:A}{\alpha}{C}}
      {\seq{\Gamma,x:A,y:B}{\alpha}{C}}
\quad
\infer{\seq{\Gamma}{\alpha}{A}}
      {\alpha \spr \beta &
       \seq{\Gamma}{\beta}{A}}
\end{array}
\]
\end{theorem}

The following general constructions can be helpful for understanding how
the types behave.  We write $A \vdash B$ for $\seq{x:A}{x}{B}$.  The
three ``fusion'' rules on the left (which are type isomorphisms, not
just interprovabilities) relate $\Fsymb$ and $\Usymb$.  Special cases
include: $A \times (B \times C)$ is isomorphic to a primitive triple
product $\{x:A,y:B,z:C\}$; currying; and associativity of $n$-ary
functions ($A_1,\ldots,A_n \to (B_1,\ldots,B_m \to C)$ is isomorphic to
$A_1,\ldots,A_n,B_1,\ldots,B_m \to C$).  Second, the types respect a
transformation covariantly for \Fsymb\/ and contravariantly for
\Usymb\/.
\begin{theorem}[Fusion and Respect Laws]~\label{lem:fusion-respect}
\begin{small}
\[
\begin{array}{rcl}
\F{\alpha}{\Delta,x:\F{\beta}{\Delta'},\Delta''} & \dashv \vdash & \F{\subst{\alpha}{\beta}{x}}{\Delta,\Delta',\Delta''}\\
\U{x.\alpha}{\Delta,y:\F{\beta}{\Delta'},\Delta''}{A} & \dashv \vdash & \U{x.\subst{\alpha}{\beta}{y}}{\Delta,\Delta',\Delta''}{A}\\
\U{x.\alpha}{\Delta}{\U{y.\beta}{\Delta'}{A}} & \dashv \vdash & \U{x.\subst{\beta}{\alpha}{y}}{\Delta,\Delta'}{A}\\
\end{array}
\qquad
\begin{array}{rcll}
\F{\alpha}{\Delta} & \vdash & \F{\beta}{\Delta} & \text { if } \alpha \spr \beta \\
\U{x.\beta}{\Delta}{A} & \vdash & \U{x.\alpha}{\Delta}{A} & \text { if } \alpha \spr \beta \\
\end{array}
\]
\end{small}
\end{theorem}

%% ----------------------------------------------------------------------

\newcommand\truej[1]{#1 \,\, \dsd{true}}
\newcommand\possj[1]{#1 \,\, \dsd{poss}}
\newcommand\validj[1]{#1 \,\, \dsd{valid}}
\newcommand\crispj[1]{#1 \,\, \dsd{crisp}}
\newcommand\cohesivej[1]{#1 \,\, \dsd{coh}}

\section{Examples}
\label{sec:exampleencodings}

\subsection{Products and Implications}

First, we show how to encode substructural products and implications
with various structural properties.  A mode theory with one mode \dsd{m}
and a constant \oftp{x : \dsd{m}, y : \dsd{m}}{x \odot y}{\dsd{m}}
specifies a completely astructural context (no weakening, exchange,
contraction, associativity), as in non-associative Lambek
calculus~\cite{lambek58calculus}.  To pass to \emph{ordered logic}
(associativity and unit laws but none of exchange, weakening, and
contraction), we add a constant $1 : \dsd{m}$ and equational axioms $x
\odot (y \odot z) \deq (x \odot y) \odot z$ and $x \odot 1 \deq x \deq 1
\odot x$---i.e. $(\odot,1)$ is a monoid.  To get linear logic, we
additionally add commutativity $x \odot y \deq y \odot x$.  As a first
example of using the sequent calculus, we show how commutativity of
$\odot$ in the mode theory for linear logic generates commutativity of
the corresponding $A \otimes B$ type, which is represented by $\F{x
  \odot y}{x:A,y:B}$:
\begin{small}
\[
\infer[\FL]
      {\seq{q:\F{x\odot y}{x:A,y:B}}{q}{\F{z\odot w}{z:B,w:A}}}
      {\infer[\FR]{\seq{x:A,y:B}{x \odot y}{\F{z\odot w}{z:B,w:A}}}
        {
            x \odot y \spr (z \odot w) [y/z,x/w] &
            \seq{x:A,y:B}{y}{B} &
            \seq{x:A,y:B}{x}{A} 
      }}
\]
\end{small}%
First, we use \FL\/ to split the product type on the left up, obtaining
permission to use its pieces by substituting $(x \odot y)$ for the
variable $q$ we began with.  Next, to use \FR\/, we must transform the
current context descriptor $x \odot y$ into a substitution instance of
the one from the type $z \odot w$---dividing our resources in the form
dictated by the type.  We take $y/z,x/w$, which requires a
transformation $x \odot y \spr y \odot x$, which is given by reflexivity
because of the commutativity axiom in the mode theory.  Then we can
prove each of $A$ and $B$ by identity, because we have the correct
resources in each branch.  In the mode theory for ordered logic, without
commutativity, the only possible division is $x/z,y/w$, and with
permission only to use $x$ the first premise and $y$ in the second, the
derivation fails.

Returning to the mode theory of a non-symmetric $\odot$, we show how the
two implications/residuations of ordered logic are modeled by
\Usymb-types; the expected rules are
\begin{small}
\[
\begin{array}{l}
\infer{\seql{\Gamma}{o}{ A \rightharpoonup B}}
      {\seql{\Gamma,A}{o}B}
\qquad
\infer{\seql{\Gamma,A \rightharpoonup B,\Delta,\Gamma'}{o}{C}}
      {\seql{\Delta}{o}{A} &
       \seql{\Gamma,B,\Gamma'}{o}{C}
      }
\qquad
\infer{\seql{\Gamma}{o}{A \leftharpoonup B}}
      {\seql{A,\Gamma}{o}{B}}
\qquad
\infer{\seql{\Gamma,\Delta,A \leftharpoonup B,\Gamma'}{o}{C}}
      {\seql{\Delta}{o}{A} &
        \seql{\Gamma,B,\Gamma'}{o}{C}
      }
\end{array}
\]
\end{small}%
We represent these by the \Usymb-types $A \rightharpoonup B := \U{c.c
  \odot x}{x:A}{B}$ and $A \leftharpoonup B := \U{c.x \odot c}{x:A}{B}$.
The \UL\/ and \UR\/ rules specialize as follows:
\begin{small}
\[
\infer{\seq{\Gamma}{\beta}{\U{c.c \odot x}{x:A}{B}}}
      {\seq{\Gamma,x:A}{\beta \odot x}{B}}
\qquad
\infer{\seq{\Gamma} {\beta} {C}}
      {\begin{array}{l}
          c:\U{c.c \odot x}{x:A}{B} \in \Gamma \\
          \beta \spr \beta'[c \odot \alpha/z] \\
          \seq{\Gamma}{\alpha}{A} \\
          \seq{\Gamma,z:A}{\beta'}{C}
        \end{array}
      }
\qquad
\infer{\seq{\Gamma}{\beta}{\U{c.x \odot c}{x:A}{B}}}
      {\seq{\Gamma,x:A}{x \odot \beta}{B}}
\qquad
\infer{\seq{\Gamma} {\beta} {C}}
      {\begin{array}{l}
          c:\U{c.x \odot c}{x:A}{B} \in \Gamma \\
          \beta \spr \beta'[\alpha \odot c/z] \\
          \seq{\Gamma}{\alpha}{A} \\
          \seq{\Gamma,z:A}{\beta'}{C}
       \end{array}
      }
\]
\end{small}%
The \UR\, instances put $x$ on the left or right of the current context
descriptor $\beta$, by the substitution $\beta/c$ in \UR.  Consider the
left rule for $\rightharpoonup$/\U{c.c \odot x}{x:A}{B}, and suppose
that the $\beta$ in the conclusion is of the form $x_1 \odot \ldots c
\ldots \odot x_n$ for distinct variables $x_i$.  Because the only
structural transformations are the associativity and unit equations, the
transformation must reassociate $\beta$ as $\beta_1 \odot (c \odot
\alpha) \odot \beta_2$, with $\beta' = \beta_1 \odot z \odot \beta_2$,
for some $\beta_1$ and $\beta_2$.  Here $\alpha$ plays the role of
$\Delta$ in the ordered logic rule---the resources used to prove $A$,
which occur to the right of the implication being eliminated.  Reading
the substitution backwards, the resources $\beta'$ used for the
continuation are ``$\beta$ with $c \odot \alpha$ replaced by the result
of the implication,'' as desired.  While $c$ and any variables used in
$\alpha$ are still in $\Gamma$, permission to use them has been removed
from $\beta'$---and there is no way to restore such permissions in this
mode theory.  The rule for $\leftharpoonup$ is the same, but with
$\alpha$ on the opposite side of $c$.  For the linear logic mode theory,
\U{c.c \odot x}{x:A}{B} and \U{c.x \odot c}{x:A}{B} are equal types
(because commutativity is an equation, and types are parametrized by
equivalence-classes of context descriptors), and both represent $A
\lolli B$.

Weakening (affine logic) is modeled by adding a directed structural
transformation $\dsd{w} :: x \spr 1$, while contraction (relevant logic)
is modeled by $\dsd{c} :: x \spr x \odot x$.  As an example use of
weakening, we can show $A \odot B \vdash A$ (formally {\seq{z : \F{x
      \odot y}{x:A,y:B}}{z}{A}}); and as an example of contraction we
can show $A \vdash A \odot A$ (formally {\seq{z : A}{z}{\F{x \odot
      y}{x:A,y:A}}}):
\begin{small}
\[
\infer[\FL]{\seq{z : \F{x \odot y}{x:A,y:B}}{z}{A}}
           {
             \infer{\seq{x:A,y:B}{x \odot y}{A}}
             {\infer{x \odot y \spr x \odot 1 \deq x}
                    {\dsd{w} :: y \spr 1}
               &
               \infer{\seq{x:A,y:B}{x}{A}}{}
           }}
\qquad
\infer[\FR]{\seq{z : A}{z}{\F{x \odot y}{x:A,y:A}}}
           {\dsd{c} :: z \spr (x \odot y)[z/x,z/y] &
            \infer{\seq{z:A}{z}{A}}{}
           }
\]
\end{small}%
If we have both $\dsd{w} :: x \spr 1$ and $\dsd{c} :: x \spr x \odot x$
(with some equations relating them), then $x \odot y$ is a cartesian
product in the mode theory, and consequently the type $\F{x \odot
  y}{x:A,y:B}$ will behave like a cartesian product type $A \times B$,
and $\U{c.c \odot x}{x:A}{B}$ like the usual structural $A \to B$.  We
refer to this mode theory as a \emph{cartesian monoid} and write
$(\times,\top)$ for it.

These encodings are adequate in the following sense:
\begin{theorem}[Logical Adequacy for Products and Implications]
Write $A^*$ for the encoding of a type as above and extend this
pointwise to contexts $\Gamma^*$.  Further, define
$\vars{x_1:A_1,\ldots,x_n:A_n} = x_1 \odot \ldots \odot x_n$.  Then
$\seql{\Gamma}{}{A}$ in the standard sequent calculus iff
$\seq{\Gamma^*}{\vars{\Gamma}}{A^*}$.
\end{theorem}
\begin{proof}
Proofs for ordered logic (products), affine logic, and cartesian logic
are in the extended version. Encoding an object-language derivation is
straightforward, because the mode theory is chosen to make each rule
derivable.  The back-translation from the framework relies on
cut-freeness (so that we only need to back-translate normal forms), and
a lemma that, for these mode theories, left-rules on variables that are
in the framework context $\Gamma$ but do not occur in the context
descriptor $\alpha$ can be strengthened away.
\end{proof}

This approach extends to contexts with more than one type of tree node,
as in bunched implication~\cite{ohearnpym99bunched}, which has two
context-forming operations $\Gamma,\Gamma'$ and $\Gamma;\Gamma'$, along
with corresponding products and implications.  Both are associative,
unital, and commutative, but $;$ has weakening and contraction while $,$
does not.  A context is represented by a tree such as $(x:A, y:B);(z :
C, w : D)$ (considered modulo the laws), and the notation
$\Gamma[\Delta]$ is used to refer to a tree with a hole $\Gamma[-]$ that
has $\Delta$ as a subtree at the hole.  In sequent calculus style, the
rules for the product and implication corresponding to $,$ are
\begin{small}
\[
\begin{array}{l}
\infer{\Gamma[A * B] \vdash C}
      {\Gamma[A , B] \vdash C}
\quad
\infer{\Gamma,\Delta \vdash A * B}
      {\Gamma \vdash A &
       \Delta \vdash B}
\quad
\infer{\Gamma \vdash A \magicwand B}
      {\Gamma, A \vdash B}
\quad
\infer{\Gamma[A \magicwand B, \Delta] \vdash C}
      {\Delta \vdash A &
       \Gamma[B] \vdash C}
\end{array}
\]
\end{small}%
We model BI by a mode \dsd{m} with both a commutative monoid $(*,I)$ and
a cartesian monoid $(\times,\top)$.  We define the BI products and
implications using the monoids as above: $A * B := \F{x * y}{x : A, y :
  B}$ and $A \times B := \F{x \times y}{x:A,y:B}$ and $A \magicwand B :=
\U{c.c * x}{x : A}{B}$ and $A \to B := \U{c.c \times x}{x : A}{B}$.  A
context descriptor such as $(x \times y) * (z \times w)$ captures the
``bunched'' structure of a BI context, and substitution for a variable
models the hole-filling operation $\Gamma[\Delta]$.  The derived left
rules for $*$ and $\magicwand$ are
\begin{small}
\[
\infer{\seq{\Gamma,z:A*B,\Gamma'}{\beta}{C}}
      {\seq{\Gamma,\Gamma',x:A,y:B}{\subst{\beta}{x * y}{z}}{C}}
\qquad
\infer{\seq{\Gamma}{\beta}{C}}
      {
        c : A \magicwand B \in \Gamma &
        \beta \spr \beta'[ c * \alpha / z] & 
        \seq{\Gamma}{\alpha}{A} &
        \seq{\Gamma,z:B}{\beta'}{C} 
      }
\]
\end{small}%
The rule for $*$ (and similarly $\times$) acts on a leaf $z$ and replaces
the leaf where $z$ occurs in the tree $\beta$ with the correct bunch
$x*y$. The left rule for $\magicwand$ (and similarly for $\to$) isolates
a subtree containing the implication $c$ and resources $*$'ed with it,
uses those resources to prove $A$, and then replaces the subtree with
the variable $z$ standing for the result of the implication.

\subsection{Multi-use variables}
\label{sec:ex:nlinear}

An $n$-use
variable~\cite{reed08namessubstructural,abel15modal,mcbride16nuttin} is
a variable that is used ``exactly $n$ times'' (modulo additives), as
expressed by the following sequent calculus rules for $n$-use functions
\begin{small}
\[
\infer{{0\cdot \Gamma,x:^1 P} \vdash {P}}
      {}
\qquad
\infer{\Gamma \vdash A \to^n B}
      {{\Gamma, x :^n A} \vdash {B}}
\qquad
\infer{\Gamma + f:^k A \to^n B + (nk \cdot \Delta) \vdash C}
      {\Delta \vdash A &
       {\Gamma, z :^k B} \vdash {C}}
\]
\end{small}%
where $\Gamma + \Delta$ acts pointwise by $x :^{n} A + x :^{m}
A = x :^{n+m} A$ and $n \cdot \Delta$ acts pointwise by $n \cdot x^{m} A
= x :^{nm} A$.  In the left rule, $\Gamma$ and $\Delta$ have the same
underlying variables and types (but potentially different counts), and
$f:^kA \to^n B$ abbreviates a context with the same variables and types
but $0$'s for all counts besides $f$'s.  The left rule says that if you
spend $k$ ``uses'' of a function that takes $n$ uses of an
argument, then you need $nk$ uses of whatever you use to
construct the argument, in order to get $k$ uses of the result.  

We can model this in the mode theory of a commutative monoid by using
context descriptors that are themselves non-linear: we define $A \to^n B
:= \U{c.c \odot (x^n)}{x:A}{B}$ where $x^n := x \odot x \odot \ldots
\odot x$ ($n$ times).  This has the following instances of \UL{}{} and
\UR{}:
\begin{small}
\[
\infer{\seq{\Gamma}{\beta}{A \to^n B}}
      {\seq{\Gamma, x:A}{\beta \odot x^n}{B}}
\qquad
\infer{\seq{\Gamma}{\beta}{C}}
      {f : \U{f.f \odot x^n}{x : A}{B} \in \Gamma &
        \beta \spr \beta'[f \odot (\alpha)^n/z] &
        \seq{\Gamma}{\alpha}{A} &
        \seq{\Gamma, z:B}{\beta'}{C} 
      }
\]
\end{small}%
In the left rule, $\beta'$ must be equal to some term $\beta'' \odot
z^k$ for some $k$ and $\beta''$ not mentioning $z$ (for this mode
theory, any term is a polynomial of variables), and the only structural
transformations are the commutative monoid equations, so the premise is
$\beta \deq (\beta'' \odot z^k) [f \odot (\alpha)^n/z] \deq \beta''
\odot f^k \odot (\alpha)^{nk}$.  Here $\beta''$ corresponds to the
$\Gamma$ in the above left rule (the resources of the continuation,
besides $z^k$) and $\alpha$ corresponds to $\Delta$.  The full proof of
adequacy is in the extended version:
\begin{theorem}[Logical adequacy for $n$-use variables]
$x_1:^{k_1} A_1,\ldots,x_n :^{k_n} A_n \vdash C$ iff
  \seq{x_1:A_1^*,\ldots,x_n:A_n^*}{x_1^{k_1} \odot \ldots \odot
    x_n^{k_n}}{C^*}, where $A^*$ translates $A \to^n B$ to 
$\U{c.c \odot (x^n)}{x:A^*}{B^*}$
\end{theorem}

\subsection{Comonads}  
\label{sec:example:bang}

Following linear-nonlinear
logic~\cite{benton94mixed,bentonwadler96adjoint}, we decompose the $!$
exponential of intuitionistic linear logic as the comonad of an
adjunction between ``linear'' and ``cartesian'' categories.  We start
with two modes \dsd{l} (linear) and \dsd{c} (cartesian), along with a
commutative monoid $(\otimes,1)$ on \dsd{l} and a cartesian monoid
$(\times,\top)$ on \dsd{c}.  Next, we add a context descriptor from
\dsd{c} to \dsd{l} ($x : \dsd{c} \vdash \dsd{f}(x) : \dsd{l}$) that we
think of as including a cartesian context in a linear context.  This
generates types \wftype {\F{\dsd{f}(x)}{x : A_{\dsd{c}}}}{\dsd{l}} and
\wftype {\U{x.\dsd{f}(x)}{\cdot}{A_{\dsd{l}}}}{\dsd{c}} which are
adjoint $\F{\dsd{f}(x)}{x:-} \la {\U{x.\dsd{f}(x)}{\cdot}{-}}$.  The
bijection on hom-sets is defined using \FL\/ and \UR\/ and their
invertibility.  The comonad of the adjunction
\F{\dsd{f}(x)}{x:\U{c.\dsd{f}(c)}{\cdot}{A}} is the linear logic $!A$.

In LNL~\cite{benton94mixed}, $F(A \times B) \cong F(A) \otimes F(B)$
and $F(\top) \cong 1$ (these properties of $F$ are necessary to
prove that $!  A$ has weakening and contraction with respect to
$\otimes$, for example), which we can add to the mode theory by
equations $\dsd{f}(x \times y) \deq \dsd{f}(x) \otimes \dsd{f}(y)$ and
$\dsd{f}(\top) \deq 1$. By Theorem~\ref{lem:fusion-respect}, these
equations induce type isomorphisms because all of $F,\otimes,\times$ are
represented by \Fsymb-types in our framework.  For example, $F(A \times
B) \vdash F(A) \otimes F(B)$ is derived as follows:
\begin{small}
\[
\infer[\FL]{\seq{q:\F{\dsd{\dsd{f}(x)}}{x:\F{y \times z}{y:A,z:B}}}{q}{\F{z \otimes w}{z:\F{\dsd{f}(x)}{x:A},w:\F{\dsd{f}(x)}{x:B}}}}
      {\infer[\FL]{\seq{x:{\F{y \times z}{y:A,z:B}}}{\dsd{f}{(x)}}{\F{z \otimes w}{z:\F{\dsd{f}(x)}{x:A},w:\F{\dsd{f}(x)}{x:B}}}}
        {\infer[\FR]{\seq{y:A,z:B}{\dsd{f}{(y \times z)}}{\F{z \otimes w}{z:\F{\dsd{f}(x)}{x:A},w:\F{\dsd{f}(x)}{x:B}}}}
          {\dsd{f}{(y \times z)} \deq \dsd{f}(y) \otimes \dsd{f}(z) &
            \infer[\FR^*]{\seq{y:A,z:B}{\dsd{f}{(y)}}{\F{x.\dsd{f}(x)}{x:A}}}{} & 
            \infer[\FR^*]{\seq{y:A,z:B}{\dsd{f}{(z)}}{\F{x.\dsd{f}(x)}{x:B}}}{} & 
          }}}
\]
\end{small}%
Omitting these equations allows us to describe non-monoidal (or lax
monoidal, if we add only one direction) left adjoints: in the extended
version, we consider S4
$\Box$~\cite{pfenningdavies,biermandepaiva00modal}, and prove adequacy
for it.

%% \begin{theorem}[Logical adequacy for Adjoint $!$]
%% Translate $F(A)^* = \F{\dsd{f}(x)}{x:A^*}$ and $G(A)^* =
%% \U{x.\dsd{f}(x)}{\cdot}{A}$ and products and functions as usual.  Then
%% $x_1:C_1,\ldots,x_n:C_n \vdash C$ in the cartesian category iff
%% \seq{x_1:C_1^*,\ldots,x_n:C_n^*}{x_1 \times \ldots \times x_n}{C^*}, and
%% a mixed sequent with cartesian and linear assumptions and a linear
%% conclusion $x_1:C_1,\ldots,x_n:C_n;y_1:A_1,\ldots,y_m:A_m \vdash A$
%% holds iff
%% \seq{x_1:C_1^*,\ldots,y_1:A_1^*,\ldots}{\dsd{f}(x_1)
%%   \otimes\ldots\otimes \dsd{f}(x_n) \otimes y_1 \otimes \ldots \otimes
%%   y_n}{A^*}.
%% \end{theorem}

\subsection{Monads}
\label{sec:example:monad}

We model a \Dia{}{A} modality~\cite{biermandepaiva00modal,pfenningdavies} with rules
\[
\infer{\Gamma \vdash \possj{A}}
      {\Gamma \vdash \truej{A}}
\qquad
\infer{\Gamma \vdash \truej{\Dia{}{A}}}
      {\Gamma \vdash \possj{A}}
\qquad
\infer{\Gamma,\truej{\Dia{}{A}} \vdash \possj{C}}
      {\truej{A} \vdash \possj{C}}
\]
by a mode theory with two modes \dsd{t} and \dsd{p}
and context descriptor \oftp{x:\dsd{t}}{\dsd{g}(x)}{\dsd{p}}; we define
$\Dia{\dsd{g}}{A} := \U{c.\dsd{g}(c)}{\cdot}{\F{\dsd{g}(x)}{x:A}}$.
This is always a monad, but it does not automatically have a tensorial
strength.  For example, if we have a monoid $(\otimes,1)$ on mode
\dsd{t} and try to derive strength
\[
\infer[\UR]
      {\seq{x : A, y : \Dia{\dsd{g}}{B}}{x \otimes y}{\Dia{\dsd{g}}{(A \otimes B)}}}
      {\infer[\UL]
        {\seq{x : A, y : \Dia{\dsd{g}}{B}}{\dsd{g}(x \otimes y)}{\F{\dsd{g}}{A \otimes B}}}
        {\dsd{g}(x \otimes y) \spr \subst{\beta'}{\dsd{g}(y)}{z} &
          \seq{x:A,y : \Dia{\dsd{g}}{B},z:\F{\dsd{g}}{B}}{\beta'}{\F{\dsd{g}}{A \otimes B}}
        }}
\]
\noindent we are stuck, because there is no way to rewrite $\dsd{g}(x
\otimes y)$ as a term containing $\dsd{g}(y)$.  If $\otimes$ is affine,
then we can weaken away $x$ and take $\beta' = z$---corresponding to the
context-clearing in the left rule for $\Dia{}{A}$---but then in the
right-hand premise we will only have access to $z$, not $x$, so
$\Diamond$ correctly represents a non-strong monad in this setting.  In
the extended version, we prove adequacy for this and extend the mode
theory to express strong monads.

\begin{theorem}[Logical adequacy for a monad]
We translate all types at mode \dsd{t}, representing
\Dia{}{A} as above. Then $\truej{A_1}, \ldots,
\truej{A_1} \vdash \truej{C}$ iff
\seq{x_1:A_1^*,\ldots,x_1:A_n^*}{x_1\otimes\ldots\otimes x_n}{C^*}, and 
$\truej{A_1}, \ldots, \truej{A_n} \vdash \possj{C}$ iff 
\seq{x_1:A_1^*,\ldots,x_1:A_n^*}{\dsd{g}(x_1\otimes\ldots\otimes
  x_n)}{\F{\dsd{g}}{C^*}}.  
The three ``native'' rules above are
\FR, \UR, and a composite of \UL\/ followed by \FL, respectively.
\end{theorem}

\subsection{Spatial Type Theory}

The spatial type theory for cohesion~\cite{shulman15realcohesion} which
motivated this work has an adjoint pair $\flat \la \sharp$, where
$\flat$ is a comonad and $\sharp$ is a monad, with some additional
properties.  In the one-variable case~\cite{ls16adjoint}, we analyzed
this as arising from an idempotent comonad\footnote{In
  \cite{ls16adjoint}, the mode theory was actually an idempotent
  \emph{monad}. The multicategorical generalization prompted changes in
  the variance of \dsd{F} and \dsd{U}; for example, \F{\alpha}{\Delta}
  must now be covariant in $\oftp{\psi}{\alpha}{r}$ for the $n$-ary
  $\Delta$ to match $\psi$.} in the mode theory: we have a mode \dsd{c}
with a cartesian monoid $(\times,\top)$ and a context descriptor
\oftp{x:\dsd{c}}{\dsd{r}(x)}{\dsd{c}} such that $\dsd{r}(\dsd{r}(x))
\deq \dsd{r}(x)$ and there is a directed transformation $\dsd{r}(x) \spr
x$.  Then we define $\flat A := \F{\dsd{r}}{A}$ and $\sharp A :=
\Uempty{\dsd{r}}{A}$. These are adjoint, and the transformation gives
the counit $\F{\dsd{r}}{A} \vdash A$ and the unit $A \vdash
\Uempty{\dsd{r}}{A}$.  Now that we have a multi-assumptioned logic, we
can model the fact that $\flat{A}$ preserves products by the equational
axiom $\dsd{r}(x \times y) \deq \dsd{r}(x) \times \dsd{r}(y)$.  Overall,
we encode a simply-typed spatial type theory judgement $x_1 :
\crispj{A_1},\ldots;y_1:\cohesivej{B_1},\ldots \vdash \cohesivej{C}$ as
$\seq{x_1:A_1,\ldots,y_1:B_1,\ldots}{\dsd{r}(x_1)\times\ldots\times
  y_1\times\ldots}{C}$.  As a sequent calculus, the rules
from~\cite{shulman15realcohesion} are
\begin{small}
\[
\begin{array}{c}
\infer{\Delta;\Gamma \vdash C}
      {A \in \Delta &
       \Delta;\Gamma,A \vdash C}
\quad
\infer{\Delta; \Gamma \vdash {\Flat A}}
      {\Delta; \cdot \vdash {A}}
\quad
\infer{\Delta; \Gamma,\Flat{A} \vdash C}
      {\Delta,A; \Gamma \vdash C}
\quad
\infer{\Delta;\Gamma \vdash {\Sharp C}}
      {\Delta,\Gamma; \cdot \vdash C}
\quad
\infer{\Delta;\Gamma \vdash C}
      {\Sharp A \in \Delta &
        \Delta;\Gamma,A \vdash {C}}
\quad
\end{array}
\]
\end{small}
In order, these correspond to (1) the action of the contraction and
$\dsd{r}(x) \spr x$ transformations; (2) \FR\/ with weakening, using
monoidalness of \dsd{r} in one direction; (3) \FL; (4) \UR, using
monoidalness of \dsd{r} in the other direction and idempotence; (5) \UL,
with contraction.  This provides a satisfying explanation for the
unusual features of these rules, such as promoting all cohesive
variables to crisp in \Sharp{}-right, and eliminating a crisp \Sharp{}
in \Sharp{}-left, and illustrates how our framework can be used in
investigating extensions of homotopy type theory.

%% ----------------------------------------------------------------------

\section{Equational Theory on Derivations}
\label{sec:equational}

In this section we give an equational theory describing $\beta\eta$-equality of
derivations.  We use this equational theory in the categorical semantics
below, and to reason about terms in encoded languages (for example, to
prove that a pair of entailments is an isomorphism, we show that the
maps compose to the identity up to these equations).

First, we need a notation for derivations of the $\alpha \spr \beta$
judgement in Figure~\ref{fig:2multicategory}.  We assume names for
constants are given in the signature $\Sigma$, and write $1_\alpha$ for
reflexivity, $s_1;s_2$ for transitivity (in diagrammatic order), and
$s_1[s_2/x]$ for congruence.  We extend the signature $\Sigma$ to allow
axioms for equality of transformations $s_1 \deq s_2$ (for two
derivations of the same judgement $s_1,s_2 ::
\wfsp{\psi}{\alpha}{\beta}{p}$), and define equality to be the least
congruence closed under those axioms and some associativity, unit, and
interchange laws, which are the 2-category axioms extended to the
multicategorical case (see the extended version for details).  As with
equality of context descriptors, we think of all definitions as being
parametrized by \deq-equivalence-classes of transformations, not raw
syntax.

To simplify the axiomatic description of equality, we use a notation for
derivations where the admissible transformation, identity, and cut rules
are internalized as explicit rules---so the calculus has the flavor of
an explicit substitution one.  We write proof terms for these plus the 4
\Usymb/\Fsymb\, rules (the hypothesis rule for atoms is derivable from
these) as follows:
\begin{small}
\[
\begin{array}{c}
\infer{{\Gamma,x:A} \vdash_{x} x : {A}}{}
\qquad
\infer{{\Gamma} \vdash_{\alpha} \Trd{s}{d} : {A}}
      {s :: \alpha \spr \beta &
        {\Gamma} \vdash_{\beta} d : {A}}
\qquad
\infer{{\Gamma} \vdash_{\subst{\beta}{\alpha}{x}} \Cut{e}{d}{x} : {B}}
      {{\Gamma,x:A} \vdash_{\beta} e : {B} &
        {\Gamma} \vdash_{\alpha} d : {A}}
\\\\ 
\infer{{\Gamma,x:\F{\alpha}{\Delta},\Gamma'} \vdash_{\beta} (\FLd{x}{\Delta}{d}) : {C}}
      {{\Gamma,\Gamma',\Delta} \vdash_{\subst \beta {\alpha}{x}} d : {C}}
\quad
\infer{{\Gamma} \vdash_{\beta} \FRd{}{s}{\vec{d_i/x_i}} : {\F{\alpha}{\Delta}}}
      {%% \modeof{\Gamma} \vdash \gamma : \modeof{\Delta} & 
        s :: \beta \spr \tsubst{\alpha}{\gamma} &
        {\Gamma} \vdash_{\gamma} \vec{d_i/x_i} : {\Delta} 
      }
\\\\
\infer{{\Gamma} \vdash_{\beta} \ULd{x}{}{s}{\vec{d_i/x_i}}{z}{d} : {C}}
      {
        x:\U{x.\alpha}{\Delta}{A} \in \Gamma &
        s :: \beta \spr \subst{\beta'}{\tsubst{\alpha}{\gamma}}{z} &
        {\Gamma} \vdash_{\gamma} {\vec{d_i/x_i}} : {\Delta} &
        {\Gamma,\tptm{z}{A}} \vdash_{\beta'} d' : {C}
      }
\quad
\infer{{\Gamma} \vdash_{\beta} \URd{\Delta}{d} : {\U{x.\alpha}{\Delta}{A}}}
      {{\Gamma,\Delta} \vdash_{\subst{\alpha}{\beta}{x}} d : {A}}
\end{array}
\]
\end{small}

The equational theory of derivations is the least congruence containing
the following equations.  
\begin{small}
\[
\begin{array}{rcll} 
\Cut{\D}{\Ident{x}}{x} & \deq & \D \\
\Cut{\Ident{x}}{\D}{x} & \deq & \D \\
\Cut{\D_1}{\D_2}{x} & \deq & \D_1 \text{ if $x \# \D_1$}\\
\Cut{(\Cut{\D_1}{\D_2}{x})}{\D_3}{y} & \deq & \Cut{(\Cut{\D_1}{\D_3}{y})}{\Cut{\D_2}{\D_3}{y}}{x}\\
\end{array}
\qquad
\begin{array}{rcll}
\Trd{1}{\D} & \deq & \D\\
\Trd{(s_1;s_2)}{\D} & \deq & \Trd{{s_1}}{\Trd{{s_2}}{\D}} \\
\Trd{(\subst{s_2}{s_1}{x})}{\Cut{\D_2}{\D_1}{x}} & \deq & \Cut{\Trd{{s_2}}{\D_2}}{\Trd{{s_1}}{\D_1}}{x} \\
\end{array}
\]
\[
\begin{array}{rcll}
\Cut{(\FLd{x_0}{\Delta}{\D})}{\FRd{}{s}{\vec{\D_i/x_i}}}{x_0} & \deq & \Trd{(1_\beta[s/x_0])}{\D[\vec{\D_i/x_i}]} & \dsd{F\beta} \\
\Cut{(\ULd{x_0}{}{s}{\vec{\D_i}/x_i}{z}{\D'})}{\URd{\Delta}{\D}}{x_0} & \deq & \Trd{(s[1_{\alpha}/{x_0}])}{\Cut{\D'}{(\D[{\vec{d_i}/x_i}])}{z}} & \dsd{U\beta} \\
\D :: \seq{\Gamma,x:\F{\alpha}{\Delta},\Gamma'}{\beta}{C} & \deq &
\FLd{x}{\Delta}{\Cut{\D}{\FRd{}{1}{\Delta/\Delta}}{x}} & \dsd{F\eta}\\
\D :: \seq{\Gamma}{\beta}{\U{x.\alpha}{\Delta}{A}} & \deq & \URd{\Delta}{\Cut{(\ULd{x}{}{1}{\Delta/\Delta}{z}{z})}{\D}{x}} & \dsd{U\eta}\\
\end{array}
\]
\end{small}

In the top-left, the first two equations say that identity is a unit for
cut.  The third says that non-occurrence of a variable is a projection.
The fourth is functoriality of cut.  In the top-right, the first two
rules say that the action of a transformation is functorial, and the
third says that it commutes with cut.  The typing in the third rule is
$\D_1 :: \seq{\Gamma}{\alpha'}{A}$ and $\D_2 ::
\seq{\Gamma,x:A}{\beta'}{C}$ and $s_1 :: \alpha \spr \alpha'$ and $s_2
:: \beta \spr \beta'$, so both sides are derivations of as derivations
of \seq{\Gamma}{\subst{\beta}{\alpha}{x}}{C}.  Finally, we have the
$\beta\eta$-laws for \dsd{F} and \dsd{U}.  The $\beta$ laws are the
principal cut cases from our cut admissibility proof.  The $\eta$ laws
witness left-invertibility of \Fsymb\, and right-invertibility of
\Usymb.

%% ----------------------------------------------------------------------

\newcommand\cD{\ensuremath{\mathcal{D}}}
\newcommand\IndF[3]{\ensuremath{{#1}^\Fsymb_{{#2},{#3}}}}
\newcommand\IndU[4]{\ensuremath{{#1}^\Usymb_{{#2},{#3},{#4}}}}

\section{Categorical Semantics}
\label{sec:semantics}

In this section, we give a category-theoretic structure corresponding to
the above syntax.  First, we define a cartesian 2-multicategory as a
semantic analogue of the syntax in Figure~\ref{fig:2multicategory}. 

\begin{definition}
  A \textbf{(strict) cartesian 2-multicategory} consists of
  \begin{enumerate}
  \item A set $\M_0$ of \emph{objects}.
  \item For every object $B$ and every finite list of objects $(A_1,\dots,A_n)$, a category $\M(A_1,\dots,A_n;B)$.
    The objects of this category are \emph{1-morphisms} and its morphisms are \emph{2-morphisms}; we write composition of 2-morphisms as $\compv{s_1}{s_2}$.
  \item For each object $A$, an identity arrow $1_A\in\M(A;A)$.
  \item For any object $C$ and lists of objects $(B_1,\dots,B_m)$ and
    $(A_{i1},\dots,A_{in_i})$ for $1\le i\le m$, a composition functor
    $(g,(f_1,\dots,f_m)) \mapsto g\circ (f_1,\dots,f_m) : 
    \M(B_1,\dots,B_m;C) \times \prod_{i=1}^m \M(A_{i1},\dots,A_{in_i};B_i) \longrightarrow \M(A_{11},\dots,A_{mn_m};C)$.
    We write the action of this functor on 2-cells as $\comph{d}{(e_1,\dots,e_m)}$.
  \item For any function $\sigma : \{1,\dots,m\} \to \{1,\dots,n\}$ and
    objects $A_1,\dots,A_n,B$, a \emph{renaming} functor $f \mapsto
    f\sigma^* : \M(A_{\sigma 1},\dots,A_{\sigma m}; B) \to \M(A_1,\dots,A_n;B)$
  \item satisfying some equalities (see the extended version)
  \end{enumerate}
\end{definition}

The next three definitions will be used to describe the
\seq{\Gamma}{\alpha}{A} judgement.  

\begin{definition}
  A \textbf{functor of cartesian 2-multicategories} $\pi:\cD\to\M$
  consists of a function $\pi_0 : \cD_0 \to \M_0$ and functors
  $\cD(A_1,\ldots,A_n;B) \to \M(\pi_0(A_1),\ldots,\pi_0(A_n);\pi_0(B))$ such
  that the chosen identities, compositions, and renamings are preserved
  (strictly).  Given a functor $\pi$, we write
  $\cD_\alpha(A_1,\dots,A_n;B)$ for the fiber over 
  $\alpha \in \M(\pi A_1,\dots,\pi A_n;\pi B)$.
\end{definition}

\begin{definition}
  A functor of cartesian 2-multicategories $\pi:\cD\to\M$ is a
  \textbf{local discrete fibration} if each induced functor of ordinary
  categories $\cD(A_1,\dots,A_n;B)\to\M(\pi A_1,\dots,\pi A_n;\pi B)$ is
  a discrete fibration.  When $\pi$ is a local discrete fibration, each
  fiber is a discrete set.
\end{definition}

\begin{definition}
  If $\pi:\cD\to\M$ is a local discrete fibration, then a morphism
  $\xi\in\cD(A_1,\dots,A_n;B)$ is \textbf{opcartesian} if all diagrams
  of the left-hand form are pullbacks of categories, and a morphism
  $\xi\in\cD(\vec C,B,\vec D;E)$ is \textbf{cartesian at $B$} if all
  diagrams of the right-hand form are pullbacks of categories: 
  \[ \xymatrix{
    \cD(\vec C,B,\vec D;E) \ar[r]^-{(-)\circ_B \xi} \ar[d]_\pi &
    \cD(\vec C,\vec A,\vec D;E) \ar[d]^\pi \\
    \M(\pi\vec C,\pi B, \pi\vec D; \pi E) \ar[r]_-{(-)\circ_{\pi B} \pi\xi} &
    \M(\pi\vec C,\pi\vec A,\pi\vec D;\pi E)
  }
  \qquad
  \xymatrix{
    \cD(\vec A;B) \ar[r]^-{\xi\circ_B (-)} \ar[d]_\pi &
    \cD(\vec C,\vec A,\vec D;E) \ar[d]^\pi \\
    \M(\pi\vec A;\pi B) \ar[r]_-{\pi\xi\circ_{\pi B} (-)} &
    \cD(\pi\vec C,\pi\vec A,\pi\vec D;\pi E)}
  \]
  Given $\mu:(p_1,\dots,p_n) \to q$ in $\M$, we say that $\pi$ \textbf{has $\mu$-products} if for any $A_i$ with $\pi A_i = p_i$, there exists a $B$ with $\pi B = q$ and an opcartesian morphism in $\cD_\mu(A_1,\dots,A_n;B)$.
  Dually, we say $\pi$ \textbf{has $\mu$-homs} if for any $i$, any $B$ with $\pi B = q$, and any $A_j$ with $\pi A_j = p_j$ for $j\neq i$, there exists an $A_i$ with $\pi A_i = p_i$ and a cartesian morphism in $\cD_\mu(A_1,\dots,A_n;B)$.
  We say that $\pi$ is an \textbf{opfibration} if it has $\mu$-products for all $\mu$, a \textbf{fibration} if it has $\mu$-homs for all $\mu$, and a \textbf{bifibration} if it is both an opfibration and a fibration.
\end{definition}

The proofs of the following soundness and completeness results are in
the extended version:

\begin{theorem}[Mode theory presents a multicategory]
\label{thm:completeness-mode-theory}
A mode theory $\Sigma$ presents a cartesian 2-multicategory $\M$, where
$\M_0$ is the set of modes, and an object of $\M(p_1,\ldots,p_n;q)$ is a
term $\oftp{x_1:p_1,\ldots,x_n:p_n}{\alpha}{q}$ and a morphism of $\M(p_1,\ldots,p_n;q)$ is a structural transformation
$s :: \wfsp{\psi}{\alpha}{\beta}{q}$, both considered modulo $\deq$.
\end{theorem}

\begin{theorem}[Completeness/Syntactic Bifibration] \label{thm:completeness}
For a fixed mode theory $\M$, the syntax presents a bifibration $\pi : \cD \to \M$, where:
\begin{itemize}
\item Objects of $\cD$ are pairs $(p, \wftype{A}{p})$;
\item 1-morphisms $\Gamma \to B$, i.e., objects of $\cD(\Gamma; B)$, are pairs $(\alpha, d :: \seq{\Gamma}{\alpha}{B})$ (up to \deq); 
\item 2-morphisms $(\alpha, d) \to (\alpha', d')$ are structural
  transformations $s :: \alpha \spr \alpha'$ such that $\Trd{s}{d'} \deq d$;
\item the $\mu$-products are \Fsymb-types, and the $\mu$-homs are \Usymb-types.
\end{itemize}
The functor $\pi : \cD \to \M$ is given by first projection on objects and 1-morphisms, and sends 2-morphisms to the underlying structural transformations.
\end{theorem}

\begin{theorem}[Soundness/Interpretation in any bifibration] \label{thm:soundness}
Fix a bifibration $\pi : \cD \to \M$.  Then there is a function $\llb -
\rrb$ from types \wftype{A}{p} to $\llb A \rrb \in \cD_0$ with $\pi(\llb
A \rrb) = p$ and from $\deq$-classes of derivations $\seq{x:A_1, \ldots,
  x_n:A_n}{\alpha}{C}$ to morphisms $d \in \cD(\llb A_1 \rrb, \dots, \llb
A_n \rrb; \llb C \rrb)$, such that $\pi(d) = \alpha$.
\end{theorem}

%% ----------------------------------------------------------------------

\section{Related and Future Work}

We have described a sequent calculus that can express a variety of
substructural and modal logics through a suitable choice of mode theory.
Our framework builds on many approaches to substructural and modal logic
in the literature.  Logical rules that act at a leaf of a
tree-structured context go back to the Lambek
calculus~\cite{lambek58calculus}.  A rich collection of context
structures that correspond to type constructors plays a central role in
display logic~\cite{belnap82display}.  The $\lambda$-calculus for
resource separation~\cite{atkey04separation} is similar to mode theories
with one mode, where there is at most one 2-cell between a given pair of
1-cells; at the logical level, our calculus is a unification of this
with multimodal adjoint logic~\cite{reed09adjoint}.  Algebras of
resources play a central role in semantics of substructural and modal
logics~\cite{restall00introduction} and in their encodings in first-order
logic~\cite{reed09constructiveresource}, and resources on variables are
used to track modalities in Agda's implementation~\cite{abel15modal} and
in linear dependent types~\cite{mcbride16nuttin}.  LF representations of
modal or substructural logics work by restricting the use of cartesian
variables~\cite{crary10substructural}.  Relative to all of these
approaches, we believe that the analysis of the context
structures/resources as a \emph{term} in a base type theory, and the
fibrational structure of the derivations over them, is a new and useful
observation.  For example, rather than needing extra-logical conditions
on proof rules to ensure cut admissibility, as in display logic, the
conditions are encoded in the language of context descriptors and the
definition of types from them.  Moreover, none of these existing
approaches allow for proof-relevant 2-cells/structural rules, and their
presence (and the equational theory we give for them) is important for
our applications to homotopy type theory.

A point of contrast with substructural logical
frameworks~\cite{cervesatopfenning02llf,watkins+03clf-tr,reed09thesis}
is that logics are ``embedded'' in our calculus (giving a type
translation such that provability in the object logic corresponds to
provability in ours), rather than ``encoding'' the structure of
derivations.  This way, we obtain cut elimination for object languages
as a corollary of framework cut elimination.

Bifibrations have also been used recently to model refinement types and
logical
relations~\cite{zeilberger14functors,zeilbergermelies17presheaf,johann15bifibrational}.
Superficially, this seems to be a different use of the same semantic
structure, because the base and domain categories of the bifibration
play a different role.  Here, the base mode theory describes different
categories of types and functors (type constructors) between between
them, and the domain represents types and terms; whereas in refinement
types/logical relations, the base category represents types and terms,
and the domain represents a further notion of predicate/specification.
It would be interesting to investigate deeper connections and combine
the two notions.

One direction for future work is to continue a preliminary investigation
of equational adequacy that is discussed in the extended version,
investigating whether the logical adequacy proofs are an isomorphism on
$\beta\eta$-classes of derivations---or, phrased semantically, that a
bifibration over the mode theory is the usual notion of categorical
model for a particular logic (e.g. a bifibration over the ordered logic
mode theory presents the free monoidal category).  It is generally easy
to show that object-language equations are true in the framework.  We
conjecture that the converse is true for the mode theories we have
described here, which says that the ``extra'' types and judgements
available in the framework do not add to the equations between terms in
the image of encoded sequents.  Proving this is challenging because the
equational theory of Section~\ref{sec:equational} does not itself
obviously have the subformula property.  We have sketched a proof of
equational adequacy for a simple case (ordered logic products), assuming
a lemma that the equational theory from Section~\ref{sec:equational} can
be characterized by permuting conversions on cut-free derivations.  A
related avenue for improvement is that our adequacy proofs require
reasoning about the 1- and 2-cells in the mode theory, which we have
currently done entirely na\"ively, but could possibly benefit from
higher-dimensional rewriting techniques.

Additionally, we plan to apply our framework to investigate more
extensions of homotopy type theory, such as an internal language for
parametrized spectra, and an extension of spatial type theory to
differential cohesion~\cite{schreiber13differential}.  We also plan to
consider encodings of programming-focused type theories, such as
specialized effect calculi.

A final direction for future work is to extend our framework with
first-order quantifiers, structured conclusions (as in classical or
display logic), and dependent types, which all seem possible but not
obvious.  Scaling to dependent types will require more worked examples
to understand the patterns of substructural and modal type dependency
that a framework should capture.  

\bibliographystyle{plainurl}
\bibliography{../drl-common/cs}

\appendix

\section{Technical Appendix: Categorical Semantics}

To illustrate the connection between the syntax and the semantics, we
sketch the proofs of soundness and completeness; full details are
available in the extended version.

\begin{proof}[Proof of Theorem~\ref{thm:completeness}]
Our goal is to construct a bifibration $\pi : \cD \to \M$ from the
syntax.

First, a mode theory $\Sigma$ presents a cartesian 2-multicategory $\M$,
where $\M_0$ is the set of modes, and an object of
$\M(p_1,\ldots,p_n;q)$ is a term
$\oftp{x_1:p_1,\ldots,x_n:p_n}{\alpha}{q}$ and a morphism of
$\M(p_1,\ldots,p_n;q)$ is a structural transformation $s ::
\wfsp{\psi}{\alpha}{\beta}{q}$, both considered modulo $\deq$.

Next, we construct the domain category $\cD$ as indicated in the theorem
statement: an object is a pair $(p,A)$ where $A$ is a type of mode $p$;
a 1-cell $(\psi,\Gamma) \to (p,A)$ is a pair $(\alpha,d)$ where $\psi
\vdash \alpha : p$ and $d$ is a derivation of \seq{\Gamma}{\alpha}{A};
and a 2-cell $(\alpha, d) \Rightarrow (\alpha', d')$ is a structural
transformation $s :: \alpha \spr \alpha'$ such that $\Trd{s}{d'} \deq
d$.  We write just $A$ and $d$ and $s$ for objects and 1-cells and
2-cells, leaving the underlying modes and mode morphisms implicit, and
we also omit variable names from sequents.

Composition of 1-morphisms is defined by iterating cut: given 
$g :: (\seq{x_1 : B_1, \dots, x_m : B_m}{\alpha}{C})$ and 
$f_i :: (\seq{A_{i1}, \dots, A_{in_i}}{\beta_i}{B_i})$ 
we set \[g \circ (f_1, \dots, f_n) := (\alpha[\beta_1/x_1, \dots, \beta_m], g[f_1/x_1, \dots, f_m/x_m])\]
That the latter derivation lies over $\alpha[\beta_1/x_1, \dots,
  \beta_m]$ follows from the cut and weakening principles.  

For the action of these composition functors on 2-morphisms, suppose we are given 1-morphisms 
\begin{align*}
d &:: \seq{x_1 : B_1, \dots, x_m : B_m}{\alpha}{C} \\
d' &:: \seq{x_1 : B_1, \dots, x_m : B_m}{\alpha'}{C} \\
e_i &:: \seq{A_{i1}, \dots, A_{in_i}}{\beta_i}{B_i} \\
e'_i &:: \seq{A_{i1}, \dots, A_{in_i}}{\beta'_i}{B_i} 
\end{align*}
and 2-morphisms $S : (\alpha, d) \spr (\alpha', d')$ and $T_i :
(\beta_i, e_i) \to (\beta'_i, e'_i)$ such that $S$ has underlying
transformation $s :: \alpha \spr \alpha'$ and the $T_i$ have underlying
transformations $t_i :: \beta_i \spr \beta'_i$ respectively. This means
that $d \deq s_*(d')$ and $e_i \deq (t_i)_*(e'_i)$ for all $i$. The
composite $\comph{S}{(T_1, \dots, T_m)}$ is the 2-morphism given by the
underlying transformation $s[t_1/x_1, \dots, t_m/x_m]$. This is a valid
2-morphism $d[e_1/x_1, \dots, e_m/x_m] \spr d'[e'_1/x_1, \dots, e'_m/x_m]$ because
\begin{align*}
& (s[t_1/x_1, \dots, t_m/x_m])_*(d'[e'_1/x_1, \dots, e'_m/x_m]) \\
\deq & (s[t_1/x_1, \dots, t_{m-1}/x_{m-1}])_*(d'[e'_1/x_1, \dots, e'_{m-1}m/x_{m-1}])[(t_m)_*(e'_m)/x_m] \\
& \vdots \\
\deq & s_*(d')[(t_1)_*(e'_1)/x_1, \dots, (t_m)_*(e'_m)/x_m] \\
\deq & d[e_1/x_1, \dots, e_m/x_m]
\end{align*}
as required, where we have repeatedly applied the equation $\Trd{(\subst{s_2}{s_1}{x})}{\Cut{\D_2}{\D_1}{x}} \deq \Cut{\Trd{{s_2}}{\D_2}}{\Trd{{s_1}}{\D_1}}{x}$.

The unit and associativity laws for 1-morphisms in $\cD$ follow from the
first set of equations for derivations, and from the definition of
multi-variable substitution as iterated cut.  For 2-morphisms, they
follow as composition of 2-morphisms is simply composition of the
underlying transformations in the mode theory.

The cartesian structure in $\cD$ is given by the admissible rules for
weakening-over-weakening, exchange-over-exchange and
contraction-over-contraction, from which all renamings can be
made. These rules also all preserve the underlying mode morphisms in the
correct way to make $\pi$ functorial.

The next step is to show that $\pi$ is a local discrete
fibration. Suppose we have a context $\Gamma$ and object $B$. We must
show that the functor $\pi : \cD(\Gamma; B) \to \M(\pi \Gamma; \pi B)$
is a discrete fibration. Let $\alpha, \alpha' \in \M(\pi \Gamma; \pi B)$
be mode morphisms and suppose we have a transformation $s :: \alpha \spr
\alpha'$ between them. Any 2-morphism in $\cD(\Gamma; B)$ lying over $s$
must clearly have $s$ as the underlying transformation. Given a lift $d'
:: \seq{\Gamma}{\alpha'}{B}$ of $\alpha'$, then we can consider $s$ as a
2-morphism $(\alpha, s_*(d')) \spr (\alpha',d')$ over $s$, whose domain
is the action of $s$ on $d'$, $s_*(d')$, as expected.  The equational
condition $s_*(d) \deq s_*(d)$ is trivially satisfied, and in fact
forces $s_*(d)$ as the only possible choice of domain, so the lift is
unique. So $\pi$ is a local discrete fibration.

We now show that $\pi$ is an opfibration, i.e., has $\alpha$-homs for
all mode morphisms $\oftp{\psi}{\alpha}{q}$. Suppose we have lifts for
the modes in $\psi$, i.e., a context $\Delta$ with $\pi\Delta =
\psi$. We define the opcartesian lift of $\alpha$ to be $\FR^* ::
\seq{\Delta}{\alpha}{\F{\alpha}{\Delta}}$, the generating map from
$\Delta$ to ${\F{\alpha}{\Delta}}$ that lives over $\alpha$, which is an
instance of the \dsd{F} right rule where both premises are the
identity. To verify that this is an opcartesian morphism, we must show
that all squares of the form
\[ \xymatrix{
    \cD(\Gamma,\F{\alpha}{\Delta},\Gamma';C) \ar[r]^-{\Cut{-}{\FR^*}{x_0}} \ar[d]_\pi &
    \cD(\Gamma,\Delta,\Gamma';C) \ar[d]^\pi \\
    \M(\pi \Gamma, q, \pi \Gamma'; \pi C) \ar[r]_-{\Cut{-}{\alpha}{x_0}} &
    \M(\pi\Gamma,\psi,\pi\Gamma'; \pi C)
}\]
are pullbacks of categories. For this we will use the following characterisation:
a diagram of categories
\[ \xymatrix{
    \mathcal{A} \ar[r]^-{H} \ar[d]_K & \mathcal{B} \ar[d]^F \\
    \mathcal{C} \ar[r]_-{G} & \mathcal{D}
  }\]
is a pullback diagram iff
\begin{itemize}
\item For every pair of objects $b \in \mathcal{B}$ and $c \in
  \mathcal{C}$ with $Fb = Gc$, there is a unique object $a \in
  \mathcal{A}$ such that $Ha = b$ and $Ka = c$; and,

\item For every pair of morphisms $f \in \mathcal{B}(b,b')$ and $g \in
  \mathcal{C}(c,c')$ with $F b = G c$ and $F b' = G c'$ and $Ff= Gg$,
  there is a unique morphism $\theta \in \mathcal{A}$ such that $H\theta
  = f$ and $K\theta = g$. The domain and codomain of $\theta$ are fixed
  by the previous property.
\end{itemize}

First, we show the property for objects. Suppose we have an object $d
\in \cD(\Gamma,\Delta,\Gamma';C)$ and $\beta \in \M(\pi \Gamma, q,\pi
\Gamma'; \pi C)$ such that $\pi(d) = \Cut{\beta}{\alpha}{x_0}$. This
simply states that $d$ is of the form $d :: \seq{\Gamma, \Delta,
  \Gamma'}{\Cut{\beta}{\alpha}{x_0}}{C}$. We must produce a unique
object $e \in \cD(\Gamma,\F{\alpha}{\Delta},\Gamma';C)$ such that
$\pi(e) = \beta$ and $\Cut{e}{\FR^*}{x_0} \deq d$.
We take as our $e$ the derivation $\FLd{x_0}{\Delta}{\D}$. This lies over $\beta$, and we calculate
\begin{align*}
\Cut{e}{\FR^*}{x_0} &= \Cut{(\FLd{x_0}{\Delta}{\D})}{\FRd{\vec{x/x}}{{1_\alpha}}{\Ident{x}/x}}{x_0} \\
&\deq \Trd{(1_\beta[1_\alpha/x_0])}{\D[\vec{\Ident{x}/x}]}\\
&\deq \Trd{(1_{\beta[\alpha/x_0]})}{\D}\\
&\deq \D
\end{align*}
by the $\beta$-law for \dsd{F} and unit laws. 
It remains to show uniqueness. Suppose we have some derivation $e'$ such
that $\pi(e') = \beta$ and $\Cut{e'}{\FR^*}{x_0} \deq d$. By the
$\eta$-law for \dsd{F}, we have
\begin{align*}
e' \deq \FLd{x_0}{\Delta}{\Cut{e'}{\FRs}{x_0}} \deq \FLd{x_0}{\Delta}{d} = e
\end{align*}
as required.

We now turn to the pullback property for morphisms. Let $\beta, \beta'
\in \M(\pi\Gamma, q, \pi\Gamma'; \pi C)$ and let $s :: \beta \spr
\beta'$ be a morphism. Further suppose that we have derivations $d ::
\seq{\Gamma, \Delta, \Gamma'}{\Cut{\beta}{\alpha}{x_0}}{C}$ and $d' ::
\seq{\Gamma, \Delta, \Gamma'}{\Cut{\beta'}{\alpha}{x_0}}{C}$ such that
$(\Cut{s}{1_\alpha}{x_0})_*(d') \deq d$. This describes a morphism $T :
d \spr d'$ in $\cD(\Gamma,\F{\alpha}{\Delta},\Gamma';C)$ that lies over
$\Cut{s}{1_\alpha}{x_0}$. This latter transformation is the result of
applying the functor $\Cut{-}{\alpha}{x_0}$ to $s$.

We now must find a morphism $S$ in $\cD(\Gamma,\Delta,\Gamma';C)$ that
lies over $s$, and such that the functor $\Cut{-}{\FR^*}{x_0}$ applied
to the morphism $S$ yields $T$. We know that for $S$ to lie over $s$,
its underlying structural transformation must be $s$. The action of
$\Cut{-}{\FR^*}{x_0}$ on $S$ then takes $s$ to $\Cut{s}{1_\alpha}{x_0}$
as expected.

By the previous argument for objects, we know that $S$ must have domain
$\FLd{x_0}{\Delta}{\D}$ and codomain $\FLd{x_0}{\Delta}{\D'}$. We can
verify that choosing the underlying transformation $s$ gives a
well-defined 2-morphism $S : (\FLd{x_0}{\Delta}{\D}) \spr
(\FLd{x_0}{\Delta}{\D'})$:
\begin{align*}
s_*( \FLd{x_0}{\Delta}{\D'}) &\deq \FLd{x_0}{\Delta}{\Cut{s_*( \FLd{x_0}{\Delta}{\D'})}{\FRs}{x_0}}\\
&\deq \FLd{x_0}{\Delta}{\Cut{s_*( \FLd{x_0}{\Delta}{\D'})}{(1_\alpha)_*(\FRs)}{x_0}}\\
&\deq \FLd{x_0}{\Delta}{(\Cut{s}{1_\alpha}{x_0})_*(\Cut{\FLd{x_0}{\Delta}{\D'}}{\FRs}{x_0})}\\
&\deq \FLd{x_0}{\Delta}{(\Cut{s}{1_\alpha}{x_0})_*(\D') }\\
&\deq \FLd{x_0}{\Delta}{\D}
\end{align*}
where we have used the $\eta$-law followed by the $\beta$-law. 

We conclude that all squares of the given form are pullback squares, and
so every $\alpha$ has an opcartesian lift. Therefore $\pi$ is an
opfibration.  The proof that  $\pi$ is also a fibration is very similar,
using \dsd{U} types instead of \dsd{F} types. 
\end{proof}

\begin{proof}[Proof of Theorem~\ref{thm:soundness}]

Conversely, we show some cases of the interpretation of the syntax in
any bifibration $\pi$ over the 2-multicategory $\M$ determined by the
mode theory.

Since $\pi$ is a local discrete fibration, the 2-cells of $\M$ act on
the fibers. Suppose $\psi \vdash \alpha, \beta : p$ and $s : \alpha \spr
\beta$. We re-use the notation $s_*$ for the induced function (of sets)
$\cD_\beta(\Gamma; A) \to \cD_\alpha(\Gamma; A)$ that sends an object $d
\in \cD_\alpha(\Gamma; A)$ to the domain of the unique lift of $s$ with
codomain $d$.

The definition of an opfibration of 2-multicategories guarantees that,
given a morphism in the mode category $\oftp{\psi}{\alpha}{q}$ and a set
of objects $\Delta$ that lies over $\psi$, there is an opcartesian
morphism over $\alpha$ with domain $\Delta$. For each $\alpha$ we choose
one such lift and take the codomain of this morphism as our
interpretation of $\F{\alpha}{\Delta}$. Let us name this opcartesian
lift $\zeta_{\alpha, \Delta} : \Delta \to \F{\alpha}{\Delta}$. $\zeta$
corresponds to the axiomatic $\FR^*$.

We assume a given interpretation of each atomic proposition $\llb
\wftype{P}{p} \rrb$ as an object of $\cD$ that lies over $p$.  The
sequent calculus rules are interpreted as follows (we elide semantic
brackets on objects):

\begin{itemize}
\item The identity derivation of a sequent $x :: \seq{\Gamma}{x}{A}$ is
  defined to be $\llb x \rrb = 1_A$.

\item Given a derivation $d :: \seq{\Gamma}{\beta}{A}$ and
  transformation $s :: \beta' \spr \beta$, the
  respect-for-transformations derivation is interpreted as $\llb s_*(d)
  \rrb = s_*(\llb d \rrb)$.

\item For $d_1 :: \seq{\Gamma, x : A, \Gamma'}{\alpha}{B}$
  and $d_2 :: \seq{\Gamma, \Gamma'}{\beta}{A}$, cut is interpreted as
  $\llb d_1[d_2/x] \rrb = \llb d_1 \rrb \circ_A \llb d_1 \rrb$ (writing
  $e \circ_A f$ for a one-place composition derived from the $n$-place
  multicategory composition).

\item For $\FL$
\[
\infer[\FL]{\seq{\Gamma,x:\F{\alpha}{\Delta},\Gamma'}{\beta}{M : C}}
      {\seq{\Gamma,\Gamma',\Delta}{\subst \beta {\alpha}{x}}{C}}
\]
the inductive hypothesis (after an exchange, which preserves the size of
the derivations) gives a morphism $\llb \D \rrb \in \cD_{\subst \beta
  {\alpha}{x}}(\Gamma, \Delta, \Gamma'; C)$ and we must produce a morphism
$\cD_{\beta}(\Gamma, \F{\alpha}{\Delta},\Gamma'; C)$. By the
opcartesian-ness of $\zeta_{\alpha, \Delta}$, the following square is a
pullback:
\[ \xymatrix{
    \cD(\Gamma,\F{\alpha}{\Delta},\Gamma';C)
    \ar[r]^-{(-)\circ\zeta_{\alpha, \Delta}} \ar[d]_\pi &
    \cD(\Gamma,\Delta,\Gamma';C) \ar[d]^\pi \\ \M(\pi\Gamma,\pi
    \F{\alpha}{\Delta}, \pi\Gamma'; \pi C) \ar[r]_-{(-)\circ \alpha} &
    \M(\pi\Gamma,\pi\Delta,\pi\Gamma'; \pi C) }\] We are given an object
of the bottom left ($\beta$) and the top right ($\llb \D \rrb$), with
$\pi\llb \D \rrb = \beta \circ_{\pi \F{\alpha}{\Delta}}\, \alpha$. By
the above characterization of pullbacks of categories, there is a unique
object $\IndF{\llb \D \rrb}{\alpha}{\Delta} \in
\cD(\Gamma,\F{\alpha}{\Delta},\Gamma';C)$ so that $\pi(\IndF{\llb \D
  \rrb}{\alpha}{\Delta}) = \beta$. We take this object to be our
interpretation.

\item For $\FR$
\[
\infer[\FR]{\seq{\Gamma}{\beta}{\F{\alpha}{\Delta}}}
      {
        s : \beta \spr \tsubst{\alpha}{\gamma} &
        \seq{\Gamma}{\gamma}{M : \Delta} 
      }
\]
where $\gamma = (\alpha_1, \dots, \alpha_n)$ and $\Delta = (C_1, \dots,
C_n)$, the first premise is a 2-cell $s : \beta \spr
{\alpha} \circ {(\alpha_1,\ldots,\alpha_n)}$, and the second is interpreted as a set of morphisms $\llb
d_i \rrb \in \cD_{\alpha_i}(\Gamma; C_i)$. We take the interpretation of
the conclusion to be $s_*(\zeta_{\alpha, \Delta} \circ (\llb d_1 \rrb,
\dots, \llb d_n \rrb))$
\end{itemize}

\noindent What remains is to check that the above interpretation
function respects the equational theory on derivations (see the extended
version).
\end{proof}

\end{document}

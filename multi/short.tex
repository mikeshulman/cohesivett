\documentclass[letterpaper,USenglish,numberwithinsect]{lipics-v2016}
%% \usepackage{microtype}%if unwanted, comment out or use option "draft"

%%SHORT \documentclass[conference,compsoconf]{../drl-common/IEEEtran}
%%SHORT \IEEEoverridecommandlockouts

\usepackage{microtype}

\usepackage{times}
\usepackage{authblk}
%% \usepackage{fullpage}

\usepackage[all]{xy}
\usepackage{multicol}
\usepackage{mathptmx}
\usepackage{color}
%% \usepackage[cmex10]{amsmath}
\usepackage{amsthm}
\usepackage{amssymb}
\usepackage{stmaryrd}
\usepackage{../drl-common/proof}
\usepackage{../drl-common/typesit}
\usepackage{../drl-common/typescommon}
%% \usepackage{../drl-common/theorem-envs}
\usepackage[sort]{natbib}
%% \usepackage{arydshln}
\usepackage{graphics}
\usepackage{natbib}
\usepackage{url}
\usepackage{relsize}
\usepackage{tipa}

\usepackage{tikz}
\usetikzlibrary{decorations.pathmorphing}

\usepackage{fancyvrb}
\newcommand{\ttt}[1]{\texttt{#1}}


\newcommand\Bx[2]{\ensuremath{\Box_{#1} \, {#2}}}
\newcommand\Crc[2]{\ensuremath{\bigcirc_{#1} \, {#2}}}
\newcommand\Dia[2]{\ensuremath{\Diamond_{#1} \, {#2}}}
\newcommand\Flat[1]{\ensuremath{\flat \, {#1}}}
\newcommand\Sharp[1]{\ensuremath{\sharp \, {#1}}}
\newcommand{\sh}{\text{\textesh}}

\newcommand\D{\ensuremath{d}} %% originally was mathcal{D} but switched notation for derivations
\newcommand\E{\ensuremath{e}}

\newcommand\magicwand{\mathrel{-\mkern-6mu*}}
\newcommand\mor[3]{\ensuremath{#2} \longrightarrow_#1 #3}
\newcommand\C{\ensuremath{\mathcal{C}}}
\newcommand\deq{\ensuremath{\equiv}}
\newcommand\spr{\ensuremath{\Rightarrow}} %% structural property/2-cell
\newcommand\seq[3]{\ensuremath{#1 \vdash_{#2} #3}}
\newcommand\seql[3]{\ensuremath{#1 \vdash^{\dsd{#2}} #3}}
\newcommand\F[2]{\ensuremath{\dsd{F}_{#1}(#2)}}
\newcommand\U[3]{\ensuremath{\dsd{U}_{#1}(#2 \mid #3)}}
\newcommand\Uempty[2]{\ensuremath{\dsd{U}_{#1}(#2)}}
\newcommand\Fsymb[0]{\dsd{F}}
\newcommand\Usymb[0]{\dsd{U}}
\newcommand\tsubst[2]{\ensuremath{#1[#2]}}
\renewcommand\subst[3]{\ensuremath{#1[#2/#3]}}
\newcommand\wftype[2]{\ensuremath{#1 \,\, \dsd{type}_{#2}}}
\renewcommand\wfctx[2]{\ensuremath{#1 \,\, \dsd{ctx}_{#2}}}
\newcommand\modeof[1]{\ensuremath{\hat{#1}}}
\newcommand\many[1]{\ensuremath{\overline{#1}}}
\renewcommand{\oftp}[3]{\ensuremath{#1 \, \vdash #2 \, \dcd{:} \, #3}}
\newcommand\FL{\dsd{FL}}
\newcommand\FR{\dsd{FR}}
\newcommand\UL{\dsd{UL}}
\newcommand\UR{\dsd{UR}}
\newcommand\lolli\multimap
\newcommand\la\dashv

\def\M{\mathcal{M}}
\def\toiso{\xrightarrow{\sim}}
\let\To\Rightarrow
%\newcommand\compo[2]{\ensuremath{#1 \circ #2}}
\newcommand\compv[2]{\ensuremath{#1 \cdot #2}}
\newcommand\comph[2]{\ensuremath{#1 \mathbin{\circ_2} #2}}

\def\llb{\llbracket}
\def\rrb{\rrbracket}

\newcommand\seqa[2]{\seql{#1}{a}{#2}}
\newcommand\seqc[2]{\seql{#1}{c}{#2}}
\newcommand\splits{\rightrightarrows}
\newcommand\vars[1]{\ensuremath{\overline{#1}}}

\newcommand{\ignore}[1]{}

\newcommand\FLd[3]{\ensuremath{\dsd{split} \, #2 \, = \, {#1} \, \dsd{in} \, #3}}
\newcommand\FRd[3]{\ensuremath{\Trd{#2}{#3}}}
\newcommand\ULd[6]{\ensuremath{\Trd{#3}{\dsd{let} \, #5 \, = \, #1(#4) \, \dsd{in} \, #6 }}}
\newcommand\URd[2]{\ensuremath{\lambda #1.#2}}
\newcommand\Trd[2]{\ensuremath{#1_*(#2)}}
\newcommand\Ident[1]{\ensuremath{{#1}}}
\newcommand\Cut[3]{\ensuremath{#1[#2/#3]}}
\newcommand\Cuta[3]{\ensuremath{#1\{#2/#3\}}}
\newcommand\Cutta[2]{\ensuremath{#1\{#2\}}}
\newcommand\Trda[2]{\ensuremath{#1_*\{#2\}}}
\newcommand\Identa[1]{\ensuremath{{\dsd{id}\{#1\}}}}
\newcommand\FRs{\ensuremath{\FR^*}}
\newcommand\ULs[1]{\ensuremath{\UL^*_{#1}}}

\newcommand\elim[1]{#1\mathord{\downarrow}}
\newcommand\deqp{\deq_{\dsd p}}
\newcommand\Linv[3]{\ensuremath{\dsd{linv}(#1,#2,#3)}}
\newcommand\Rinv[2]{\ensuremath{\dsd{rinv}(#1,#2)}}

\usepackage{ifthen}
\newboolean{short}
\setboolean{short}{true}

\begin{document}

\title{A Fibrational Framework for \hspace{4in} Substructural and Modal Logics}
%% \titlerunning{A Sample LIPIcs Article}

\author[1]{{Daniel R. Licata}}
\author[2]{{Michael Shulman}}
\author[1]{{Mitchell Riley}}
\affil[1]{Wesleyan University\thanks{This material is based on research
    sponsored by The United States Air Force Research Laboratory under
    agreement number FA9550-15-1-0053 and FA9550-16-1-0292.  The U.S. Government is authorized to reproduce and distribute reprints for Governmental purposes notwithstanding any copyright notation thereon.  The views and conclusions contained herein are those of the authors and should not be interpreted as necessarily representing the official policies or endorsements, either expressed or implied, of the United States Air Force Research Laboratory, the U.S. Government, or Carnegie Mellon University.}}
\affil[2]{University of San Diego$^*$}

\Copyright{Daniel R. Licata, Michael Shulman, Mitchell Riley}

\authorrunning{D. Licata and M. Shulman and M. Riley} 

%% \subjclass{Dummy classification -- please refer to \url{http://www.acm.org/about/class/ccs98-html}}% mandatory: Please choose ACM 1998 classifications from http://www.acm.org/about/class/ccs98-html . E.g., cite as "F.1.1 Models of Computation". 
%% \keywords{Dummy keyword -- please provide 1--5 keywords}% mandatory: Please provide 1-5 keywords

\maketitle

\begin{abstract}
We define a general framework that abstracts the common features of many
intuitionistic substructural and modal logics.  The framework is a
sequent calculus / normal-form type theory parametrized by a \emph{mode
  theory}, which is used to describe the structure of contexts and the
structural properties they obey.  The framework makes use of resource
annotations, where we pair the context itself, which obeys standard
structural properties, with a term, drawn from the mode theory, that
constrains how the context can be used.  
%% Product types, implications,
%% and modalities are defined as instances of two general connectives, one
%% positive and one negative, that manipulate these resource annotations.
%% We show that specific mode theories can express non-associative,
%% ordered, linear, affine, relevant, and cartesian products and
%% implications; monoidal and non-monoidal comonads and adjunctions; strong
%% and non-strong monads; n-linear variables; bunched implications; and the
%% adjunctions that arose in our work on homotopy type theory.  
We prove cut (and identity) admissibility independently of the mode
theory, obtaining it for many different logics at once. Further, we
give a general equational theory on derivations / terms that, in
addition to the usual beta-eta rules, characterizes when two derivations
differ only by the placement of structural rules.  Finally, we give an
equivalent semantic presentation of these ideas, in which a mode theory
corresponds to a 2-dimensional cartesian multicategory, the
framework corresponds to another such multicategory with a functor to
the mode theory, and the logical connectives make this into a bifibration.
%% The logical connectives have universal properties
%% relative to this functor, making it into a bifibration.  The sequent
%% calculus rules and the equational theory on derivations are sound and
%% complete for this.  The resulting framework can be used both to
%% understand existing logics / type theories and to design new ones.
\end{abstract}


\section{Introduction}

In ordinary intuitionistic logic or $\lambda$-calculus, assumptions or
variables can go unused (weakening), be used in any order (exchange), be
used more than once (contraction), and be used in any position in a
term.  \emph{Substructural} logics, such as linear logic, ordered logic,
relevant logic, and affine logic, drop some of these structural
properties of weakening, exchange, and contraction, while \emph{modal
  logics} place restrictions on where variables may be used---e.g. a
formula $\Bx{} C$ can only be proved using assumptions of $\Bx{} A$,
while an assumption of $\Dia{}{A}$ can only be used when the conclusion
is $\Dia{}{C}$.  Substructural and modal logics have had many
applications to both functional and logic programming (modeling concepts
such state, staged computation, distribution, and concurrency, to name
just a few).

Substructural and modal logics can also be used as \emph{internal
  languages} of categories, where one uses an appropriate logical
language to do constructions ``inside'' a particular mathematical
setting, which often leads to shorter statements than working
``externally''.  For example, to define a function when working
``externally'' in domains, one must first define the underlying
set-theoretic function, and then prove that it is continuous.  But when
using untyped $\lambda$-calculus as an internal language of domains,
there is no need to prove that a function described by a $\lambda$-term
is continuous, because all terms are shown to denote continous functions
once and for all.  Substructural logics extend this idea to various
forms of monoidal categories, while modal logics describe monads and
comonads.  Recently,
\citet{schreibershulman12cohesive,shulman15realcohesion} proposed using
modal operators to add a notion of \emph{cohesion} to homotopy type
theory/univalent foundations~\citep{,voevodsky06homotopy,uf13hott-book}.
Without going into the precise details of this application, the idea is
to add a triple $\sh{} \la \Flat{} \la \Sharp{}$ of type operators,
where for example $\Sharp A$ is a monad (like a modal possibility
$\Diamond$ or $\bigcirc$), $\Flat A$ is a comonad (like a modal
necessity $\Box$), and there is an adjunction structure between them
(e.g. $\flat{A} \to B$ is the same as $A \to \Sharp{B}$).  This raised
the question of how to best add modalities with these properties to type
theory.

Because other similar applications would have different monads and
comonads with different properties, we would like general tools for
going from a semantic situation of interest to a ``nice'' type
theory/logic for it, e.g. one with cut and identity admissibility and
the subformula property. In previous work~\citep{ls16adjoint}, we
considered the special case of a single-assumption logic, building most
directly on the adjoint logics of
\citet{benton94mixed,bentonwadler96adjoint,reed09adjoint}.  Here we
extend this previous work to the multi-assumption case.  The resulting
framework is quite general and covers many existing intuitionistic
substructural and modal connectives: cartesian, linear, affine,
relevant, ordered, bunched~\citep{ohearnpym99bunched} and
non-associative products and implications; $n$-linear
variables~\citep{reed08namessubstructural}; the comonadic $\Box$ and
linear exponential $!$ and
subexponentials~\citep{nigammiller09subexponentials,danos+93subexponentials};
monadic $\Diamond$ and $\bigcirc$ modalities; and adjoint logic $F$ and
$G$~\citep{benton94mixed,bentonwadler96adjoint,reed09adjoint}, including
the single-assumption 2-categorical version from our previous
work~\citep{ls16adjoint}.  It also supports variations on these, such as
non-monoidal comonads and non-strong monads.  We show that a single,
simple structural~\citep{pfenning94cut} proof of cut (and identity)
admissibility applies to all of these logics, as well as any new logics
that can be described in the framework.
%% While it is not too surprising
%% that this is possible, given that cut proofs for these logics all follow
%% a similar template, it is nonetheless satisfying to codify this pattern
%% as an abstraction.

At a high level, the framework expresses the idea that all of the above
logics are a restriction on how variables can be used in ordinary
structural/cartesian proofs.  We express these restrictions using a
first layer of the logic, which is a simple type theory for what we will
call \emph{modes} and \emph{context descriptors}.  The modes are just a
collection of base types, which we write as $p,q,r$, while a context
descriptor is a term built from variables and constants.  The next layer
is the main logic.  Each proposition of the logic is assigned a mode,
and the basic sequent is \seq{x_1 : A_1, \ldots, x_n : A_n}{\alpha}{C},
where if $A_i$ has mode $p_i$, and $C$ has mode $q$, then $\oftp{x_1 :
  p_1,\ldots, x_n : p_n}{\alpha}{q}$.  
%% In a sequent
%% \seq{\Gamma}{\alpha}{A}, the idea is that $\Gamma$ binds some variables
%% for use both in $\alpha$ and in the derivation.  
$\Gamma$ itself behaves like an ordinary structural/cartesian context,
while the substructural and modal aspects are enforced by the
\emph{term} $\alpha$, which describes how the resources from $\Gamma$
are allowed to be used.  For example, in linear logic/ordered logic/BI,
the context is usually taken to be a multiset/list/tree (respectively).
We represent the multiset or list or tree using a pair of an ordinary
structural context $\Gamma$, together with a term $\alpha$ that
describes the multiset or list or tree structure, labeled with variables
from the ordinary context at the leaves.  We pronounce a sequent
\seq{\Gamma}{\alpha}{A} as ``$\Gamma$ proves $A$ {along,over} $\alpha$''
or ``$\Gamma$ structured according to $\alpha$ proves $A$''.

For example, suppose we have one mode $\dsd{n}$, together with a context
descriptor constant
\[
x : \dsd{n}, y:\dsd{n} \vdash x \odot y : \dsd{n}
\]
Then an example sequent
\[
\seq{x:A, y:B, z:C, w:D}{(y \odot x) \odot z}{E}
\]
should be read as saying that we must prove $E$ using the resources $y$
and $x$ and $z$ (but not $w$) according to the particular tree structure
${(y \odot x) \odot z}$.  If we say nothing else, the framework will
treat $\odot$ as describing a non-associative, linear, ordered context:
if we have a product-like type $A \odot B$ internalizing this context
operation,\footnote{We overload binary operations to refer both to
  context descriptors and propositional connectives, because it is clear
  from whether it is applied to variables $x,y,z$ or propositions
  $A,B,C$ which we mean.}  then we will \emph{not} be able to prove
associativity ($(A \odot B) \odot C \dashv\vdash A \odot (B \odot C)$)
or contraction ($A \vdash A \odot A$) or exchange ($A \odot B \vdash B
\odot A$) etc.

To get from this basic structure to linear or affine or relevant or
cartesian logic, we need to add some structural properties to the
context descriptor term $\alpha$.  We analyze structural properties as
\emph{equations}, or more generally \emph{directed transformations}, on
such terms.  For example, to specify linear logic, we will add a unit
element $1 : \dsd{n}$ together with equations making $(\odot,1)$ into a
commutative monoid:
\[
\begin{array}{c}
x \odot (y \odot z) = (x \odot y) \odot z\\
x \odot 1 = x = 1 \odot x\\
x \odot y = y \odot x
\end{array}
\]
so that the context descriptors ignore associativity and order.  To get
BI, we add an additional commutative monoid $(\times,\top)$ (with
weakening and contraction, as discussed below), so that a BI context
tree $(x:A,y:B);(z:C,w:D)$ can be represented by the ordinary context
$x:A,y:B,z:C,w:D$ with the term $(x \odot y) \times (z \odot w)$
describing the tree.  Because the context descriptors are themselves
ordinary structural/cartesian terms, the same variable can occur more
than once or not at all.  A descriptor such as $x \odot x$ captures the
idea that we can use the \emph{same} variable $x$ twice, expressing
$n$-linear types~\citep{reed08namessubstructural}.  Thus, we can express
contraction for a particular context descriptor $\odot$ as an equation
$x = x \odot x$ (one use of $x$ is the same as two, or $\odot$ is an
idempotent binary operation).  However, weakening cannot be represented
as an equation between context descriptors: an equation $x = 1$ would
trivialize the logic to ordinary intuitionistic logic.  Instead, to
express weakening, we use a directed transformation $x \spr 1$, which is
oriented to allow throwing away an allowed use of $x$, but not creating
an allowed use from nothing.  We refer to these as \emph{structural
  transformations}, to evoke their use in representing the structural
properties of object logics that are embedded in our framework.
Structural transformations are also used to describe relationships
between adjunctions~\citep{ls16adjoint}.

In summary, to specify a particular substructural or modal logic, one
gives constants generating context descriptors $\alpha$, with equations
$\alpha = \beta$ and transformations $\alpha \spr \beta$ expressing
structural properties.  The main sequent $\seq{\Gamma}{\alpha}{A}$
respects the specified structural properties in the sense that when
$\alpha = \beta$, we regard $\seq{\Gamma}{\alpha}{A}$ and
$\seq{\Gamma}{\beta}{A}$ as the same sequent, while when $\alpha \spr
\beta$, there will be an operation that takes a derivation of
\seq{\Gamma}{\beta}{A} to a derivation of \seq{\Gamma}{\alpha}{A}.

A guiding principle of the framework is a meta-level notion of
\emph{structurality over structurality}.  For example, we always have
\emph{weakening over weakening}: if \seq{\Gamma}{\alpha}{A} then
\seq{\Gamma,y:B}{\alpha}{A}, where $\alpha$ itself is weakened with $y$.
This does not prevent encodings of e.g. linear logic: it is permissible
to weaken a derivation of \seq{\Gamma}{x_1 \odot \ldots \odot x_n}{A}
(``use $x_1$ through $x_n$'') to a derivation of \seq{\Gamma,y:B}{x_1
  \odot \ldots \odot x_n}{A} because the (weakened) context descriptor
still disallows the use of $y$.  Similarly, we always have exchange over
exchange and contraction over contraction.  The identity and and cut
principles are analogous:
\[
\infer{\seq{\Gamma,x:A}{x}{A}}{}
\qquad
\infer{\seq{\Gamma}{\subst{\beta}{\alpha}{x}}{B}}
    {\seq{\Gamma,x:A}{\beta}{B} &
     \seq{\Gamma}{\alpha}{A}}
\]
The identity-over-identity principle says that we should be able to
prove $A$ using exactly an assumption $x:A$.  The cut principle says
that the context descriptor for the result of the cut is the
substitution of the context descriptor used to prove $A$ into the one
used to prove $B$.  For example, together with weakening-over-weakening,
this captures the usual cut principle of linear logic, which says that
cutting $\Gamma,x:A \vdash B$ and $\Delta \vdash A$ yields
$\Gamma,\Delta \vdash B$.  If $\Gamma$ binds $x_1,\ldots,x_n$ and
$\Delta$ binds $y_1,\ldots,y_n$, then we will represent the two
derivations to be cut together by sequents with
\[
\begin{array}{l}
\beta = x_1 \odot \ldots \odot x_n \odot x\\
\alpha = y_1 \odot \ldots \odot y_n
\end{array}
\]
so
\[
\beta[\alpha/x] = x_1 \odot \ldots \odot x_n \odot y_1 \odot \ldots \odot y_n
\]
correctly deletes $x$ and replaces it with the variables from $\Delta$.
Moreover, in more subtle situations such as BI, the substitution will
insert the resources used to prove the cut formula in the correct place
in the tree.

The framework has two main logical connectives.  The first,
\F{\alpha}{\Delta}, generalizes the \dsd{F} of adjoint
logic~\citep{bentonwadler96adjoint,reed09adjoint} and the tensor
($\otimes$) of linear logic.  The second, \U{x.\alpha}{\Delta}{A},
generalizes the $\dsd{G}/\dsd{U}$ of adjoint logic and the implication
$A \lolli B$ of linear logic.  Here $\Delta$ is a context of assumptions
$x_i:A_i$, and trivializing the context descriptors (i.e. adding an
equation $\alpha = \beta$ for all $\alpha$ and $\beta$) degenerates
$\F{\alpha}{\Delta}$ into the ordinary intuititionistic product $A_1
\times \ldots \times A_n$, while \U{x.\alpha}{\Delta}{A} becomes $A_1
\to \ldots \to A_n \to A$.  Though we do not give a full
polarized/focused proof theory in this paper, we do prove that \dsd{F}
is left-invertible and \dsd{U} is right-invertible, and we conjecture
that focusing works with the polarization that one would expect based on
these degeneracies ($\F{\alpha}{\Delta^{\mathord{+}}}^{\mathord{+}}$ and
$\U{x.\alpha}{\Delta^{\mathord{+}}}{A^{\mathord{-}}}^{\mathord{-}}$).
In linear logic terms, our \dsd{F} and \dsd{U} cover both the
multiplicatives and exponentials; additives can be added separately by
essentially the usual rules.

Being a very general theory, our framework treats the structural
properties in a general but na\"ive way, allowing an arbitrary
structural transformation to be applied at the non-invertible rules for
$\dsd{F}$ and $\dsd{U}$ and at the leaves of a derivation.  For specific
embedded logics, there will often be a more refined discipline that
suffices---e.g. for cartesian logic, always contract all assumptions at
in all premises, rather than choosing which assumptions to contract.  We
view our framework as a tool for bridging the gap between an intended
semantic situation such as the cohesion example mentioned above (``a
comonad and a monad which are themselves adjoint'') and a proof theory:
the framework gives \emph{some} proof theory for the semantics, and the
placement of structural rules can then be optimized purely in syntax.
To support this mode of use, we give an equational theory on sequent
derivations that identifies different placements of the same structural
rules.  This equational theory is used to prove correctness of such
optimizations not just at the level of provability, but also identity of
derivations---which matters for our intended applications to internal
languages.

Semantically, the logic corresponds to a functor between
\emph{2-dimensional cartesian multicategories} which is a fibration in
various senses.  Multicategories are a generalization of categories
which allow more than one object in the domain, and cartesianness means
that the multiple domain objects are treated structurally.  The
2-dimensionality supplies a notion of morphism between (multi)morphisms,
which correspond to the structural transformations.  The functor
specifies the mode of each proposition and the context descriptor of a
sequent.  The fibration conditions (similar to \citep{hermida,hormann})
specify respect for the structural transformations and the presence of
\dsd{F} and \dsd{U} types.

The remainder of this paper is organized as follows.  FIXME

FIXME: comparison with display logic, L/CLF, what else?  

 

In Section~\ref{sec:syntax}, we present the syntax of the framework.  In
Section~\ref{sec:exampleencodings}, we discuss how a number of logics
are represented.  In Section~\ref{sec:equational}, we give the
$\beta\eta$-equational theory on derivations.  In
Section~\ref{sec:semantics}, we discuss the framework's categorical
semantics.  Proof and additional examples are presented in an extended
version of this paper, available from
\url{dlicata.web.wesleyan.edu/pubs.html}.  

\newcommand\wfsp[4]{\ensuremath{#1 \vdash #2 \spr_{#4} #3}}

\section{Sequent Calculus}
\label{sec:syntax}

\newcommand\wfsig[1]{\ensuremath{#1 \, \dsd{sig}}}
\newcommand\deqtms[5]{\ensuremath{#1 \vdash_{#2} #3 \deq #4 : #5}}
\newcommand\wfsps[5]{\ensuremath{#1 \vdash_{#2} #3 \spr_{#5} #4}}

\subsection{Mode Theories}

\begin{figure}
\begin{small}
\[
\begin{array}{l}
\framebox{Signatures \wfsig{\Sigma}}
\qquad
\infer{\wfsig{\cdot}}
      {}
\qquad
\infer{\wfsig{(\Sigma,p \, \dsd{mode})}}
      {\wfsig{\Sigma}}
\qquad
\infer{\wfsig{(\Sigma,c : \, p_1,\ldots,p_n \to q)}}
      {\wfsig{\Sigma} &
        (p_1 \, \dsd{mode},\ldots,p_n \, \dsd{mode},q \, \dsd{mode}) \in \Sigma
      }
\\ \\
\infer{\wfsig{(\Sigma, (\alpha \deq \alpha' : \psi \to p))}}
      {\wfsig{\Sigma} &
        \psi \dsd{ctx}_\Sigma & 
        p \, \dsd{mode} \in \Sigma &
        \oftps{\psi}{\Sigma}{\alpha}{p} & 
        \oftps{\psi}{\Sigma}{\alpha'}{p} 
      }
\qquad
\infer{\wfsig{(\Sigma, (\alpha \spr \alpha' : \psi \to p))}}
      {\wfsig{\Sigma} &
        \psi \dsd{ctx}_\Sigma & 
        p \, \dsd{mode} \in \Sigma &
        \oftps{\psi}{\Sigma}{\alpha}{p} & 
        \oftps{\psi}{\Sigma}{\alpha'}{p} 
      }
\\\\
\ifthenelse{\boolean{short}}{}
{\framebox{Mode contexts $\psi \, \dsd{ctx}_{\Sigma}$}
\qquad
\infer{\cdot \, \dsd{ctx}_{\Sigma}}{}
\qquad
\infer{(\psi, x:p) \, \dsd{ctx}_{\Sigma}}
      {\psi \, \dsd{ctx}_{\Sigma} & 
        p \, \dsd{mode} \in \Sigma
      }
\\\\
}
\framebox{Context descriptors \oftps{\psi}{\Sigma}{\alpha}{p},
  where $\psi \, \dsd{ctx}_\Sigma$ and $p \, \dsd{mode} \in \Sigma$}
\qquad
\infer{\oftps{\psi}{\Sigma}{x}{p}}
      {x:p \in \psi}
\quad
\infer{\oftps{\psi}{\Sigma}{\dsd{c}(\alpha_1,\ldots,\alpha_n)}{q}}
      {(\dsd{c} : p_1,\ldots,p_n \to q) \in \Sigma &
       \oftps{\psi}{\Sigma}{\alpha_i}{p_i}
      }
\\\\
\framebox{Mode Substitutions \oftps{\psi}{\Sigma}{\gamma}{\psi'}, where
  $\psi \, \dsd{ctx}_\Sigma$ and $\psi' \, \dsd{ctx}_\Sigma$ }
\qquad
\infer{\oftps{\psi}{\Sigma}{\cdot}{\cdot}}{}
\qquad
\infer{\oftps{\psi}{\Sigma}{\gamma,\alpha/x}{\psi',x:p}}
      {\oftps{\psi}{\Sigma}{\gamma}{\psi'} &
        \oftps{\psi}{\Sigma}{\alpha}{p}}

\ifthenelse{\boolean{short}}{}
{\framebox{Equalities of mode morphisms
  \deqtms{\psi}{\Sigma}{\alpha}{\alpha'}{p},
where $\psi \, \dsd{ctx}_\Sigma$ and $p \, \dsd{mode} \in \Sigma$
and \oftps{\psi}{\Sigma}{\alpha}{p}
and \oftps{\psi}{\Sigma}{\alpha'}{p}
}
\qquad
\infer{\deqtms{\psi}{\Sigma}{\alpha}{\alpha}{p}}{}
\qquad
\infer{\deqtms{\psi}{\Sigma}{\alpha_1}{\alpha_2}{p}}
      {\deqtms{\psi}{\Sigma}{\alpha_2}{\alpha_1}{p}}
\qquad
\infer{\deqtms{\psi}{\Sigma}{\alpha_1}{\alpha_3}{p}}
      {\deqtms{\psi}{\Sigma}{\alpha_1}{\alpha_2}{p} &
        \deqtms{\psi}{\Sigma}{\alpha_2}{\alpha_3}{p} &
      }
\\ \\
\infer{\deqtms{\psi,\psi'}{\Sigma}{\subst{\beta}{\alpha}{x}}{\subst{\beta'}{\alpha'}{x}}{q}}
      {\deqtms{\psi,x:p,\psi'}{\Sigma}{\beta}{\beta'}{q} &
        \deqtms{\psi,\psi'}{\Sigma}{\alpha}{\alpha'}{p}}
\qquad
\infer{\deqtms{\psi}{\Sigma}{\alpha}{\alpha'}{p}}
      {(\alpha \deq \alpha' : \psi \to p) \in \Sigma}
}
\\\\
\framebox{Structural transformations \wfsps{\psi}{\Sigma}{\alpha}{\alpha'}{p},
where \oftps{\psi}{\Sigma}{\alpha}{p}
and \oftps{\psi}{\Sigma}{\alpha'}{p}
}
\qquad
\infer{\wfsps{\psi}{\Sigma}{\alpha}{\alpha}{p}}{}
\\\\
\infer{\wfsps{\psi}{\Sigma}{\alpha_1}{\alpha_3}{p}}
      {\wfsps{\psi}{\Sigma}{\alpha_1}{\alpha_2}{p} &
       \wfsps{\psi}{\Sigma}{\alpha_2}{\alpha_3}{p} &
      }
\qquad
\infer{\wfsps{\psi,\psi'}{\Sigma}{\subst{\beta}{\alpha}{x}}{\subst{\beta'}{\alpha'}{x}}{q}}
      {\wfsps{\psi,x:p,\psi'}{\Sigma}{\beta}{\beta'}{q} &
       \wfsps{\psi,\psi'}{\Sigma}{\alpha}{\alpha'}{p}}
\qquad
\infer{\wfsps{\psi}{\Sigma}{\alpha}{\alpha'}{p}}
      {(\alpha \spr \alpha' : \psi \to p) \in \Sigma}
\end{array}
\]
\end{small}
\caption{Syntax for mode theories}
\label{fig:2multicategory}
\end{figure}

The first layer of our framework is a type theory whose types we will call
\emph{modes}, and whose terms we will call \emph{context descriptors} or
\emph{mode morphisms}.  The only modes are atomic/base types $p$.  A
term is either a variable (bound in a context $\psi$) or a typed $n$-ary
constant (function symbol) \dsd{c} applied to terms of the appropriate
types.

This is formalized in the notion of signature, or \emph{mode theory},
defined in Figure~\ref{fig:2multicategory}.  The judgement $\wfsig
\Sigma$ means that $\Sigma$ is a well-formed signature.  The top line
says that a signature is either empty, or a signature extended with a
new mode declaration, or a signature extended with a typed
constant/function symbol, all of whose modes are declared previously in
the signature.  The notation $p_1,\ldots,p_n \to q$ is not itself a
mode, but notation for declaring a function symbol in the signature (it
cannot occur on the right-hand side of a typing judgement).  For
example, the type and term constructors for a monoid $(\odot,1)$ are
represented by a signature $\dsd{p} \, \dsd{mode}, \dsd{\odot} :
(\dsd{p},\dsd{p} \to \dsd{p}), 1 : (\to \dsd{p})$.

\ifthenelse{\boolean{short}}{
We elide the rules for 
the judgement $\psi \, \dsd{ctx}_\Sigma$, which simply says that each
mode used in the 
context of variable declarations $\psi$ is declared in $\Sigma$.  
}
{
The judgement $\psi \, \dsd{ctx}_\Sigma$ defines well-formedness of a
context of variable declarations relative to a signature $\Sigma$: each
mode in the context must be declared in the signature.}  The judgement
$\oftps{\psi}{\Sigma}{\alpha}{p}$ defines well-typedness of context
descriptor terms, which are either a variable declared in the context,
or a constant declared in the signature applied to arguments of the
correct types.  The judgement $\oftps{\psi}{\Sigma}{\gamma}{\psi'}$
defines a substitution as a tuple of terms in the standard way.  The
context $\psi$ in these judgements enjoys the cartesian structural
properties (associativity, unit, weakening, exchange, contraction).
Simultaneous substitution into terms and substitutions is defined as
usual (e.g.  $x[\gamma,\alpha/x] := \alpha$ and
$\dsd{c}(\vec{\alpha_i})[\gamma] := \dsd{c}(\alpha_i[\gamma])$).

Returning to the top of the figure, the final two rules of the judgement
$\wfsig{\Sigma}$ permit two additional forms of signature declaration.
The first of these extends a signature with an equational axiom between
two terms $\alpha$ and $\alpha'$ that have the same mode $p$, in the
same context $\psi$, relative to the prior signature $\Sigma$.  These
equational axioms will be used to encode reversible object language
structural properties, such as associativity, commutativity, and unit
laws.  For example, to specify the right unit law for the above monoid
$(\odot,1)$, we add an axiom $(x \odot 1 \deq x : (x : \dsd{p}) \to
\dsd{p})$ to the signature, which can be read as ``$x \odot 1$ is equal
to $x$ as a morphism from $(x : \dsd{p})$ to \dsd{p}''.  The judgement
\deqtms{\psi}{\Sigma}{\alpha}{\alpha'}{p} (omitted from the figure; the
rules are the same as for $\spr$ plus symmetry) is the least congruence
closed under these axioms.

The second of these extends a signature with a directed structural
transformation axiom between two terms $\alpha$ and $\alpha'$ that have
the same mode $p$, in the same context $\psi$, relative to the prior
signature $\Sigma$.  As discussed above, these structural
transformations will be used to represent object language structural
properties such as weakening and contraction that are not invertible.
The judgement \wfsps{\psi}{\Sigma}{\alpha}{\alpha'}{p} defines these
transformations: it is the least precongruence (preorder compatible with
the term formers) closed under the axioms specified in the signature
$\Sigma$.  For example, to say that the above monoid $(\odot,1)$ is
affine, we add in $\Sigma$ a transformation axiom $(x \spr 1 : (x:\dsd{p}) \to
{\dsd{p}})$.
%% Then, using the rules in the figure, we can for example derive a
%% transformation $(x \odot y) \spr (1 \odot y) \spr y$ that, when
%% applied (contravariantly) to a sequent, will allow weakening $y$ to
%% $x \odot y$.
\ifthenelse{\boolean{short}}{}
{An alternative to including the judgement $\alpha \deq \alpha'$ would be
to present a desired equation $\alpha \deq \alpha'$ as an isomorphism,
with transformation axioms $s : \alpha \spr \alpha'$ and $s' : \alpha'
\spr \alpha$.  While this is conceptually and technically sufficient, we
have found it helpful in examples to use ``strict'' equality of context
descriptors.  This simplifies the description of some situations, though
the difference is important mainly at the level of identity of
derivations rather than provability---for example, we can make a binary
operation $\odot$ into a strict monoid, rather than adding associator
and unitor isomorphisms.
}

Because context descriptors
$\alpha$ and their equality $\alpha_1 \deq \alpha_2$ are defined prior
to the subsequent judgements, we suppress this equality by using
$\alpha$ to refer to a term-modulo-\deq---that is, we assume a
metatheory with quotient sets/types, and use meta-level equality for
object-level equality, as recently advocated by
\citet{altenkirchkaposi16qit}.  For example, because the judgement
\wfsp{\psi}{\alpha}{\beta}{p} is indexed by equivalence classes of
context descriptions, the reflexivity rule above implicitly means
$\alpha \deq \beta$ implies $\alpha \spr \beta$.
\ifthenelse{\boolean{short}}{}
{
As discussed in Section~\ref{sec:equational}, we will eventually need an
equational theory between two structural property derivations $s \deq s'
:: \wfsp{\psi}{\alpha}{\alpha'}{q}$.  Because this equational theory
does not influence provability in the sequent calculus, only identity of
proofs, we defer the details to that section.  

}
In examples, we will notate a signature declaration introducing a term
constant/function symbol by showing the function symbol applied to
variables, rather than writing the formal $\dsd{c} : p_1,\ldots,p_n \to
q$. For example, we write $x : \dsd{p}, y : \dsd{p} \vdash x \odot y :
\dsd{p}$ for $\odot : \dsd{p},\dsd{p} \to \dsd{p}$.  We also suppress
the signature $\Sigma$.

\subsection{Sequent Calculus Rules}

\begin{figure}
\begin{small}
\[
\begin{array}{l}
%% \begin{array}{llll}
%% \text{Types} & A & ::= & P \mid \F{\alpha}{\Delta} \mid \U{\alpha}{\Delta}{A} \\
%% \end{array}
%% \\ \\
\framebox{Types $A,B,C$ \quad \wftype{A}{p}}
\qquad
\infer{\wftype{P}{p}}{}
\qquad
\infer{\wftype{\F{\alpha}{\Delta}}{q}}
      {\oftp{\psi}{\alpha}{q} &
        \wfctx{\Delta}{\psi}}
\qquad
\infer{\wftype{\U{x.\alpha}{\Delta}{A}}{q}}
      {\oftp{\psi,x:q}{\alpha}{p} &
        \wfctx{\Delta}{\psi} &
        \wftype{A}{p}
      }
\\ \\
\framebox{Contexts $\Gamma,\Delta$ \quad \wfctx{\Gamma}{\psi}}
\qquad
\infer{\wfctx{\cdot}{\cdot}}{}
\qquad
\infer{\wfctx{\Gamma,x:A}{\psi,x:p}}
      {\wfctx{\Gamma}{\psi} &
        \wftype{A}{p}}
\\ \\
\framebox{\seq{\Gamma}{\alpha}{A} where $\wfctx{\Gamma}{\psi}$ and $\wftype{A}{q}$ and  $\oftp{\psi}{\alpha}{q}$}
\quad
\infer[\FL]{\seq{\Gamma,x:\F{\alpha}{\Delta},\Gamma'}{\beta}{C}}
      {\seq{\Gamma,\Gamma',\Delta}{\subst \beta {\alpha}{x}}{C}}
\quad
\infer[\FR]{\seq{\Gamma}{\beta}{\F{\alpha}{\Delta}}}
      {%% \modeof{\Gamma} \vdash \gamma : \modeof{\Delta} & 
        \beta \spr \tsubst{\alpha}{\gamma} &
        \seq{\Gamma}{\gamma}{\Delta} 
      }
%% \infer{\seq{\Gamma}{\beta}{C}}
%%       {{x}:{\F{\alpha}{\Delta}} \in \Gamma & 
%%         \oftp{\modeof{\Gamma},{x'} : {\modeof{\F{\alpha}{\Delta}}}}{\beta'}{\modeof{C}} &
%%         \beta \deq \tsubst{\beta'}{x/x'} &
%%         \seq{\Gamma,\Delta}{\subst {\beta'} {\alpha}{x'}}{C}}
\\ \\
\infer[\UL]{\seq{\Gamma}{\beta}{C}}
      {\begin{array}{llll}
          x:\U{x.\alpha}{\Delta}{A} \in \Gamma &
          \beta \spr \subst{\beta'}{\tsubst{\alpha}{\gamma}}{z} &
          \seq{\Gamma}{\gamma}{\Delta} &
          \seq{\Gamma,\tptm{z}{A}}{\beta'}{C}
       \end{array}
      }
\quad
\infer[\UR]{\seq{\Gamma}{\beta}{\U{x.\alpha}{\Delta}{A}}}
      {\seq{\Gamma,\Delta}{\subst{\alpha}{\beta}{x}}{A}}
\quad
\infer[\dsd{v}]{\seq{\Gamma}{\beta}{P}}
      {x:P \in \Gamma & \beta \spr x}
\\ \\
\framebox{\seq{\Gamma}{\gamma}{\Delta} where $\wfctx{\Gamma}{\psi}$ and $\wfctx{\Delta}{\psi'}$ and  $\oftp{\psi}{\gamma}{\psi'}$}
\qquad
\infer[\cdot]{\seq{\Gamma}{\cdot}{\cdot}}
      {}
\qquad
\infer[\_,\_]{\seq{\Gamma}{\gamma,\alpha/x}{\Delta,x:A}}
      {\seq{\Gamma}{\gamma}{\Delta} &
       \seq{\Gamma}{\alpha}{A}
      }
\end{array}
\]    
\caption{Sequent Calculus}
\label{fig:sequent}
\hrule
\end{small}
\end{figure}

For a fixed mode theory $\Sigma$, we define a second layer of judgements
in Figure~\ref{fig:sequent}.  The first judgement assigns each
proposition/type $A$ a mode $p$.  Encodings of non-modal logics will
generally only make use of one mode, while modal logics use different
modes to represent different notions of truth, such as the linear and
cartesian categories in the adjoint decomposition of linear
logic~\citep{benton94mixed,bentonwadler96adjoint} and the true/valid/lax
judgements in modal logic~\citep{pfenningdavies}.  The next judgement
assigns each context $\Gamma$ a mode context $\psi$.  Formally, we think
of contexts as ordered: we do not regard $x:A,y:B$ and $y:B,x:A$ as the
same context, though we will have an admissible exchange rule that
passes between derivations in one and the other.

The sequent judgement \seq{\Gamma}{\alpha}{A} relates a context
$\wfctx{\Gamma}{\psi}$ and a type $\wftype{A}{p}$ and context descriptor
\oftp{\psi}{\alpha}{p}, while the substitution judgement \seq{\Gamma}{\gamma}{\Delta} relates
$\wfctx{\Gamma}{\psi}$ and $\wfctx{\Delta}{\psi'}$ and
$\oftp{\psi}{\gamma}{\psi'}$. Because $\wfctx{\Gamma}{\psi}$ means that
each variable in $\Gamma$ is in $\psi$, where $x : A_i \in \Gamma$
implies $x : p_i$ in $\psi$ with \wftype{A_i}{p_i}, we think of $\Gamma$
as binding variable names both in $\alpha$ and for use in the
derivation.

\ifthenelse{\boolean{short}}{}{
As discussed in the introduction, a guiding principle is to make the
following rules admissible (see Section~\ref{sec:synprop-long} for
details), which express respect for structural transformations and
structurality-over-structurality:
\[
\begin{array}{c}
\infer[Lem~\ref{lem:respectspr}]{\seq{\Gamma}{\alpha}{A}}
      {\alpha \spr \beta &
       \seq{\Gamma}{\beta}{A}}
\qquad
\infer[Thm~\ref{thm:identity}]{\seq{\Gamma,x:A}{x}{A}}{}
\qquad
\infer[Thm~\ref{thm:cut}]{\seq{\Gamma}{\subst{\beta}{\alpha}{x}}{B}}
    {\seq{\Gamma,x:A}{\beta}{B} &
     \seq{\Gamma}{\alpha}{A}}
\\ \\
\infer[Lem~\ref{lem:weakening}]{\seq{\Gamma,y:A}{\alpha}{C}}
      {\seq{\Gamma}{\alpha}{C}}
\quad
\infer[Lem~\ref{lem:exchange}]{\seq{\Gamma,y:B,x:A}{\alpha}{C}}
      {\seq{\Gamma,x:A,y:B}{\alpha}{C}}
\qquad
\infer[Cor~\ref{cor:contraction}]{\seq{\Gamma,x:A}{\subst \alpha x y}{C}}
      {\seq{\Gamma,x:A,y:A}{\alpha}{C}}
\end{array}
\]
}

We now explain the rules for the sequent calculus; the reader may wish
to refer to the examples in Section~\ref{sec:exampleencodings} in
parallel with this abstract description.  We assume atomic propositions
$P$ are given a specified mode $p$, and state identity as a primitive
rule only for them with the \dsd{v} rule.  This says that
\seq{\Gamma,x:P}{x}{P}, and additionally composes with a structural
transformation $\beta \spr x$.  Using a structural property at a leaf of
a derivation is common in e.g. affine logic, where the derivation of
$\beta \spr x$ would use weakening to forget any additional resources
besides $x$.

Next, we consider the \F{\alpha}{\Delta} type, which ``internalizes''
the context operation $\alpha$ as a type/proposition.  Syntactically, we
view the context $\Delta = x_1:A_1,\ldots,x_n:A_n$ where
\wftype{A_i}{p_i} as binding the variables $x_i:p_i$ in $\alpha$, so for
example \F{\alpha}{x:A,y:B} and \F{\alpha[x \leftrightarrow
    x']}{x':A,y:B} are $\alpha$-equivalent types (in de Bruijn form we
would write \F{\alpha}{A_1,\ldots,A_n} and use indices in $\alpha$).
The type formation rule says that \dsd{F} moves covariantly along a mode
morphism $\alpha$, representing a ``product'' (in a loose sense) of the
types in $\Delta$ structured according to the context descriptor
$\alpha$. A typical binary instance of \dsd{F} is a multiplicative
product ($A \otimes B$ in linear logic), which, given a binary context
descriptor $\odot$ as in the introduction, is written \F{x \odot
  y}{x:A,y:B}.  A typical nullary instance is a unit (1 in linear
logic), written \F{1}{}.  A typical unary instance is the \dsd{F}
connective of adjoint logic, which for a unary context descriptor
constant $\dsd{f} : \dsd{p} \to \dsd{q}$ is written \F{\dsd{f}(x)}{x:A}.
We sometimes write \F{\dsd{f}}{A} in this case, eliding the variable
name, and similarly for a unary \dsd{U}.

The rules for our \dsd{F} connective capture a pattern common to all of
these examples.  The left $\FL$ rule says that \F{\alpha}{\Delta}
``decays'' into $\Delta$, but \emph{structuring the uses of resources in
  $\Delta$ with $\alpha$ by the substitution \subst{\beta}{\alpha}{x}}.
We assume that $\Delta$ is $\alpha$-renamed to avoid collision with
$\Gamma$ (the proof term here is a ``\dsd{split}'' that binds
variables for each position in $\Delta$).  The placement of $\Delta$ at
the right of the context is arbitrary (because we have
exchange-over-exchange), but we follow the convention that new variables
go on the right to emphasize that $\Gamma$ behaves mostly as in ordinary
cartesian logic.  The right \FR\/ rule says that you must rewrite (using
structural transformations) the context descriptor to have an $\alpha$
at the outside, with a mode substitution $\gamma$ that divides the
existing resources up between the positions in $\Delta$, and then prove
each formula in $\Delta$ using the specified resources.  We leave the
typing of $\gamma$ implicit, though there is officially a requirement
$\oftp{\psi}{\gamma}{\psi'}$ where $\wfctx{\Gamma}{\psi}$ and
$\wfctx{\Delta}{\psi'}$, as required for the second premise to be a
well-formed sequent.  Another way to understand this rule is to begin
with the ``axiomatic \FR'' instance 
$\FR^* :: {\seq{\Delta}{\alpha}{\F{\alpha}{\Delta}}}{}$
which says that there is a map from $\Delta$ to \F{\alpha}{\Delta} along
$\alpha$.  Then, in the same way that a typical right rule for
coproducts builds a precomposition into an ``axiomatic injection'' such
as $\dsd{inl} :: A \vdash A + B$, the \FR\/ rule builds a precomposition
with $\seq{\Gamma}{\gamma}{\Delta}$ and then an application of a
structural rule $\beta \spr \alpha[\gamma]$ into the ``axiomatic''
version, in order to make cut and respect for transformations
admissible.

Next, we turn to $\U{x.\alpha}{\Delta}{A}$.  As a first approximation,
if we ignore the context descriptors and structural properties,
\U{-}{\Delta}{A} behaves like $\Delta \to A$, and the \UL\/ and \UR\/
rules are an annotation of the usual structural/cartesian rules for
implication.  In a formula \U{x.\alpha}{\Delta}{A}, the context
descriptor $\alpha$ has access to the variables from $\Delta$ as well as
an extra variable $x$, whose mode is the same as the \emph{overall mode
  of \U{x.\alpha}{\Delta}{A}}, while the mode of $A$ itself is the mode
of the conclusion of $\alpha$---in terms of typing, \dsd{U} is
contravariant where \dsd{F} is covariant.  It is helpful to think of $x$
as standing for the context that will be used to prove
\U{x.\alpha}{\Delta}{A}.  For example, a typical function type $A \lolli
B$ is represented by \U{x.x \otimes y}{y:A}{B}, which says to extend the
``current context'' $x$ with a resource $y$.  In \UR, the context
descriptor $\beta$ being used to prove the \dsd{U} is substituted
\emph{for $x$} in $\alpha$ (dual to \FL, which substituted $\alpha$ into
$\beta$).  The ``axiomatic'' \UL\/ instance
$\UL^* :: {\seq{\Delta,x:\U{x.\alpha}{\Delta}{A}}{\alpha}{A}}$
says that \U{x.\alpha}{\Delta}{A} together with $\Delta$ has a map to
$A$ along $\alpha$.  (The bound $x$ in $x.\alpha$ subscript is tacitly
renamed to match the name of the assumption in the context, in the same
way that the typing rule for $\lambda x.e : \Pi x:A.B$ requires
coordination between two variables in different scopes).  The full rule
builds in precomposition with \seq{\Gamma}{\gamma}{\Delta},
postcomposition with \seq{\Gamma,z:A}{\beta'}{C}, and precomposition
with $\beta \spr \beta'[\alpha[\gamma]/z]$.

Finally, the rules for substitutions are pointwise.  In examples, we
will write the components of a substitution directly as multiple
premises of \FR\/ and \UL\/, rather than packaging them with 
$\_,\_$ and $\cdot$.

\ifthenelse{\boolean{short}}{}{
One subtle point about the $\FL$ rule is that there are two competing
principles: making the rules ``obviously'' structural-over-structural,
and reducing inessential non-determinism.  Here, we choose the later,
and treat the assumption of \F{\alpha}{\Delta} affinely, removing it
from the context when it is used.  It will turn out that the judgement
nonetheless enjoys contraction-over-contraction
(Corollary~\ref{cor:contraction}), because contraction
for negatives is built into the \UL-rule, and contraction for positives
follows from this and the fact that we can always reconstruct a positive
from what it decays to on the left (c.f. how purely positive formulas
have contraction in linear logic).
}

Additives can be added to this sequent calculus; e.g. a mode $p$ has
sums $\wftype{A_p + B_p}{p}$ if
\[
\begin{array}{c}
\infer{\seq{\Gamma}{\alpha}{A + B}}
      {\seq{\Gamma}{\alpha}{A}}
\quad
\infer{\seq{\Gamma}{\alpha}{A + B}}
      {\seq{\Gamma}{\alpha}{B}}
\quad
\infer{\seq{\Gamma,x:A+B,\Gamma'}{\beta}{C}}
      {\seq{\Gamma,\Gamma',y:A}{\subst \beta y x}{C} &
       \seq{\Gamma,\Gamma',z:B}{\subst \beta z x}{C} 
      }
%% \infer{\wftype{A \& B}{p}}
%%       {\wftype{A}{p} &
%%        \wftype{B}{p}}
%% \qquad
%% \infer{\seq{\Gamma,x:A \& B}{\alpha}{C}}
%%       {\seq{\Gamma,y:A}{\alpha[y/x]}{C}}
%% \quad
%% \infer{\seq{\Gamma}{\alpha}{A + B}}
%%       {\seq{\Gamma}{\alpha}{B}}
\end{array}
\]


\begin{theorem}[Admissibility of cut, identity,
    structurality-over-structurality, and respect for 2-cells]
The following rules are admissible:
\[
\begin{array}{c}
\infer{\seq{\Gamma}{\subst{\beta}{\alpha}{x}}{B}}
    {\seq{\Gamma,x:A}{\beta}{B} &
     \seq{\Gamma}{\alpha}{A}}
\quad
\infer{\seq{\Gamma,x:A}{x}{A}}{}
\quad
\infer{\seq{\Gamma,y:A}{\alpha}{C}}
      {\seq{\Gamma}{\alpha}{C}}
\quad
\infer{\seq{\Gamma,y:B,x:A}{\alpha}{C}}
      {\seq{\Gamma,x:A,y:B}{\alpha}{C}}
\quad
\infer{\seq{\Gamma}{\alpha}{A}}
      {\alpha \spr \beta &
       \seq{\Gamma}{\beta}{A}}
%% \quad
%% \infer{\seq{\Gamma,x:A}{\subst \alpha x y}{C}}
%%       {\seq{\Gamma,x:A,y:A}{\alpha}{C}}
\end{array}
\]
\end{theorem}
The complete proofs are in the extended version.  

The following general constructions can be helpful for understanding how
the types behave.  The first three ``fusion'' rules (which are
additionally type isomorphisms, not just interprovabilities) relate
$\Fsymb$ and $\Usymb$.  Special cases include: $A \times (B \times C)$
is isomorphic to a primitive triple product $\{x:A,y:B,z:C\}$; currying;
and associativity of $n$-ary functions ($A_1,\ldots,A_n \to
(B_1,\ldots,B_m \to C)$ is isomorphic to $A_1,\ldots,A_n,B_1,\ldots,B_m
\to C$).  Second, the types respect a transformation covariantly for
\Fsymb\/ and contravariantly for \Usymb\/.
\[
\begin{array}{rcl}
\F{\alpha}{\Delta,x:\F{\beta}{\Delta'},\Delta''} & \dashv \vdash & \F{\subst{\alpha}{\beta}{x}}{\Delta,\Delta',\Delta''}\\
\U{x.\alpha}{\Delta,y:\F{\beta}{\Delta'},\Delta''}{A} & \dashv \vdash & \U{x.\subst{\alpha}{\beta}{y}}{\Delta,\Delta',\Delta''}{A}\\
\U{x.\alpha}{\Delta}{\U{y.\beta}{\Delta'}{A}} & \dashv \vdash & \U{x.\subst{\beta}{\alpha}{y}}{\Delta,\Delta'}{A}\\
\F{\alpha}{\Delta} & \vdash & \F{\beta}{\Delta} \text { if } \alpha \spr \beta \\
\U{x.\beta}{\Delta}{A} & \vdash & \U{x.\alpha}{\Delta}{A} \text { if } \alpha \spr \beta \\
\end{array}
\]

%% FIXME: discussion of how the theories of context descriptors for the
%% various systems related to any semantic structure in existing models
%% of those systems.

\newcommand\truej[1]{#1 \,\, \dsd{true}}
\newcommand\possj[1]{#1 \,\, \dsd{poss}}
\newcommand\validj[1]{#1 \,\, \dsd{valid}}
\newcommand\crispj[1]{#1 \,\, \dsd{crisp}}
\newcommand\cohesivej[1]{#1 \,\, \dsd{coh}}

\section{Examples}
\label{sec:exampleencodings}

\subsection{Products and Implications}

First, we show how to encode substructural products and implications
with various structural properties.  A mode theory with one mode \dsd{m}
and a constant \oftp{x : \dsd{m}, y : \dsd{m}}{x \odot y}{\dsd{m}}
specifies a completely astructural context (no weakening, exchange,
contraction, associativity), as in non-associative Lambek
calculus~\citep{lambek58calculus}.  To pass to \emph{ordered logic}
(associativity and unit laws but none of exchange, weakening, and
contraction), we add a constant $1 : \dsd{m}$ and equational axioms $x
\odot (y \odot z) \deq (x \odot y) \odot z$ and $x \odot 1 \deq x \deq 1
\odot x$---i.e. $(\odot,1)$ is a monoid.  To get linear logic, we
additionally add commutativity $x \odot y \deq y \odot x$.  As a first
example of using the sequent calculus, we show how commutativity of
$\odot$ in the mode theory for linear logic generates commutativity of
the corresponding $A \otimes B$ type, which is represented by $\F{x
  \odot y}{x:A,y:B}$:
\begin{small}
\[
\infer[\FL]
      {\seq{q:\F{x\odot y}{x:A,y:B}}{q}{\F{z\odot w}{z:B,w:A}}}
      {\infer[\FR]{\seq{x:A,y:B}{x \odot y}{\F{z\odot w}{z:B,w:A}}}
        {
            x \odot y \spr (z \odot w) [y/z,x/w] &
            \seq{x:A,y:B}{y}{B} &
            \seq{x:A,y:B}{x}{A} 
      }}
\]
\end{small}%
First, we use \FL\/ to split the product type on the left up, obtaining
permission to use its pieces by substituting $(x \odot y)$ for the
variable $q$ we began with.  Next, to use \FR\/, we must transform the
current context descriptor $x \odot y$ into a substitution instance of
the one from the type $z \odot w$---dividing our resources in the form
dictated by the type.  We take $y/z,x/w$, which requires a
transformation $x \odot y \spr y \odot x$, which is given by reflexivity
because of the commutativity axiom in the mode theory.  Then we can
prove each of $A$ and $B$ by identity, because we have the correct
resources in each branch.  In the mode theory for ordered logic, without
commutativity, the only possible division is $x/z,y/w$, and with
permission only to use $x$ the first premise and $y$ in the second, the
derivation fails.

Returning to the mode theory of a non-symmetric $\odot$, we show how the
two implications of ordered logic are modeled by \Usymb-types; the
expected rules are
\begin{small}
\[
\begin{array}{l}
\infer{\seql{\Gamma}{o}{ A \rightharpoonup B}}
      {\seql{\Gamma,A}{o}B}
\qquad
\infer{\seql{\Gamma,A \rightharpoonup B,\Delta,\Gamma'}{o}{C}}
      {\seql{\Delta}{o}{A} &
       \seql{\Gamma,B,\Gamma'}{o}{C}
      }
\qquad
\infer{\seql{\Gamma}{o}{A \leftharpoonup B}}
      {\seql{A,\Gamma}{o}{B}}
\qquad
\infer{\seql{\Gamma,\Delta,A \leftharpoonup B,\Gamma'}{o}{C}}
      {\seql{\Delta}{o}{A} &
        \seql{\Gamma,B,\Gamma'}{o}{C}
      }
\end{array}
\]
\end{small}%
We represent these by the \Usymb-types $A \rightharpoonup B := \U{c.c
  \odot x}{x:A}{B}$ and $A \leftharpoonup B := \U{c.x \odot c}{x:A}{B}$.
The \UL\/ and \UR\/ rules specialize as follows:
\begin{small}
\[
\infer{\seq{\Gamma}{\beta}{\U{c.c \odot x}{x:A}{B}}}
      {\seq{\Gamma,x:A}{\beta \odot x}{B}}
\qquad
\infer{\seq{\Gamma} {\beta} {C}}
      {\begin{array}{l}
          c:\U{c.c \odot x}{x:A}{B} \in \Gamma \\
          \beta \spr \beta'[c \odot \alpha/z] \\
          \seq{\Gamma}{\alpha}{A} \\
          \seq{\Gamma,z:A}{\beta'}{C}
        \end{array}
      }
\qquad
\infer{\seq{\Gamma}{\beta}{\U{c.x \odot c}{x:A}{B}}}
      {\seq{\Gamma,x:A}{x \odot \beta}{B}}
\qquad
\infer{\seq{\Gamma} {\beta} {C}}
      {\begin{array}{l}
          c:\U{c.x \odot c}{x:A}{B} \in \Gamma \\
          \beta \spr \beta'[\alpha \odot c/z] \\
          \seq{\Gamma}{\alpha}{A} \\
          \seq{\Gamma,z:A}{\beta'}{C}
       \end{array}
      }
\]
\end{small}%
The \UR\, instances put $x$ on the left or right of the current context
descriptor $\beta$, by the substitution $\beta/c$ in \UR.  Consider the
left rule for $\rightharpoonup$/\U{c.c \odot x}{x:A}{B}, and suppose
that the $\beta$ in the conclusion is of the form $x_1 \odot \ldots c
\ldots \odot x_n$ for distinct variables $x_i$.  Because the only
structural transformations are the associativity and unit equations, the
transformation must reassociate $\beta$ as $\beta_1 \odot (c \odot
\alpha) \odot \beta_2$, with $\beta' = \beta_1 \odot z \odot \beta_2$,
for some $\beta_1$ and $\beta_2$.  Here $\alpha$ plays the role of
$\Delta$ in the ordered logic rule---the resources used to prove $A$,
which occur to the right of the implication being eliminated.  Reading
the substitution backwards, the resources $\beta'$ used for the
continuation are ``$\beta$ with $c \odot \alpha$ replaced by the result
of the implication,'' as desired.  While $c$ and any variables used in
$\alpha$ are still in $\Gamma$, permission to use them has been removed
from $\beta'$---and there is no way to restore such permissions in this
mode theory.  The rule for $\leftharpoonup$ is the same, but with
$\alpha$ on the opposite side of $c$.  For the linear logic mode theory,
\U{c.c \odot x}{x:A}{B} and \U{c.x \odot c}{x:A}{B} are equal types
(because commutativity is an equation, and types are parametrized by
equivalence-classes of context descriptors), and both represent $A
\lolli B$.

Weakening (affine logic) is modeled by adding a directed structural
transformation $\dsd{w} :: x \spr 1$, while contraction (relevant logic)
is modeled by $\dsd{c} :: x \spr x \odot x$.  These transformations in
the mode theory induce sequents $A \odot B \vdash A$ and $A \vdash A \odot A$:
\begin{small}
\[
\infer[\FL]{\seq{z : \F{x \odot y}{x:A,y:B}}{z}{A}}
           {
             \infer{\seq{x:A,y:B}{x \odot y}{A}}
             {\infer{x \odot y \spr x \odot 1 \deq x}
                    {w :: y \spr 1}
               &
               \infer{\seq{x:A,y:B}{x}{A}}{}
           }}
\qquad
\infer[\FR]{\seq{z : A}{z}{\F{x \odot y}{x:A,y:A}}}
           {c :: z \spr (x \odot y)[z/x,z/y] &
            \infer{\seq{z:A}{z}{A}}{}
           }
\]
\end{small}%
If we have both $\dsd{w} :: x \spr 1$ and $\dsd{c} :: x \spr x \odot x$
(with some equations relating them), then $x \odot y$ is a cartesian
product in the mode theory, and consequently $A \odot B$ will behave
like a cartesian product type, and $\U{c.c \otimes x}{x:A}{B}$ like the
usual structural $A \to B$.  We refer to this mode theory as an
\emph{cartesian monoid} and write $(\times,\top)$ for it.

These encodings are adequate in the following sense:
\begin{theorem}[Logical Adequacy for Products and Implications]
Write $A^*$ for the encoding of a type as above and extend this
pointwise to contexts $\Gamma^*$.  Further, define
$\vars{x_1:A_1,\ldots,x_n:A_n} = x_1 \odot \ldots \odot x_n$.  Then
$\seql{\Gamma}{}{A}$ in the standard sequent calculus iff
$\seq{\Gamma^*}{\vars{\Gamma}}{A^*}$.
\end{theorem}
\begin{proof}
Proofs for ordered logic (products), affine logic, and cartesian logic
are in the extended version. Encoding an object-language derivation is
straightforward, because the mode theory is chosen to make each rule
derivable.  The back-translation from the framework relies on
cut-freeness (so that we only need to back-translate normal forms), and
a lemma that, for these mode theories, left-rules on variables that are
in the framework context $\Gamma$ but do not occur in the context
descriptor $\alpha$ can be strengthened away.
\end{proof}

This approach extends to contexts with more than one type of tree node,
as in bunched implication~\citep{ohearnpym99bunched}, which has two
context-forming operations $\Gamma,\Gamma'$ and $\Gamma;\Gamma'$, along
with corresponding products and implications.  Both are associative,
unital, and commutative, but $;$ has weakening and contraction while $,$
does not.  A context is represented by a tree such as $(x:A, y:B);(z :
C, w : D)$ (considered modulo the laws), and the notation
$\Gamma[\Delta]$ is used to refer to a tree with a hole $\Gamma[-]$ that
has $\Delta$ as a subtree at the hole.  In sequent calculus style, the
rules for the product and implication corresponding to $,$ are
\begin{small}
\[
\begin{array}{l}
\infer{\Gamma[A * B] \vdash C}
      {\Gamma[A , B] \vdash C}
\quad
\infer{\Gamma,\Delta \vdash A * B}
      {\Gamma \vdash A &
       \Delta \vdash B}
\quad
\infer{\Gamma \vdash A \magicwand B}
      {\Gamma, A \vdash B}
\quad
\infer{\Gamma[A \magicwand B, \Delta] \vdash C}
      {\Delta \vdash A &
       \Gamma[B] \vdash C}
\end{array}
\]
\end{small}%
We model BI by a mode \dsd{m} with both a commutative monoid $(*,I)$ and
a cartesian monoid $(\times,\top)$.  We define the BI products and
implications using the monoids as above: $A * B := \F{x * y}{x : A, y :
  B}$ and $A \times B := \F{x \times y}{x:A,y:B}$ and $A \magicwand B :=
\U{c.c * x}{x : A}{B}$ and $A \to B := \U{c.c \times x}{x : A}{B}$.  A
context descriptor such as $(x \times y) * (z \times w)$ captures the
``bunched'' structure of a BI context, and substitution for a variable
models the hole-filling operation $\Gamma[\Delta]$.  The derived left
rules for $*$ and $\magicwand$ are
\begin{small}
\[
\infer{\seq{\Gamma,z:A*B,\Gamma'}{\beta}{C}}
      {\seq{\Gamma,\Gamma',x:A,y:B}{\subst{\beta}{x * y}{z}}{C}}
\qquad
\infer{\seq{\Gamma}{\beta}{C}}
      {
        c : A \magicwand B \in \Gamma &
        \beta \spr \beta'[ c * \alpha / z] & 
        \seq{\Gamma}{\alpha}{A} &
        \seq{\Gamma,z:B}{\beta'}{C} 
      }
\]
\end{small}%
The rule for $*$ (and similarly $\times$) acts on a leaf $z$ and replaces
the leaf where $z$ occurs in the tree $\beta$ with the correct bunch
$x*y$. The left rule for $\magicwand$ (and similarly for $\to$) isolates
a subtree containing the implication $c$ and resources $*$'ed with it,
uses those resources to prove $A$, and then replaces the subtree with
the variable $z$ standing for the result of the implication.

%% We assume the BI sequent is given as a judgement $\Gamma \vdash A$ where
%% $\Gamma$ is a tree and there are explicit equality premises for the
%% algebraic laws on bunches.  Then we define $\Gamma^*$ as an in-order
%% flattening of the tree into one of our contexts (e.g.  $(x:A)^* = x:A^*$ and
%% $(\Gamma,\Delta)^* = (\Gamma;\Delta)^*=\Gamma^*,\Delta^*$), while we
%% define $\vars{\Gamma}$ as a context descriptor that preserves the tree
%% structure (e.g. $\vars{x:A} = x$ and $\vars{(\Gamma,\Delta)} =
%% \vars{\Gamma}*\vars{\Delta}$ and
%% $\vars{\Gamma;\Delta}=\vars{\Gamma}\times\vars{\Delta}$).  Then we have
%% the usual adequacy statement $\Gamma \vdash A$ iff
%% \seq{\Gamma^*}{\vars{\Gamma}}{A^*}.

\subsection{Multi-use variables}
\label{sec:ex:nlinear}

An $n$-use
variable~\citep{reed08namessubstructural,abel15modal,mcbride16nuttin} is
a variable that is used ``exactly $n$ times'' (modulo additives), as
expressed by the following sequent calculus rules for $n$-use functions
\begin{small}
\[
\infer{{0\cdot \Gamma,x:^1 P} \vdash {P}}
      {}
\qquad
\infer{\Gamma \vdash A \to^n B}
      {{\Gamma, x :^n A} \vdash {B}}
\qquad
\infer{\Gamma + f:^k A \to^n B + (nk \cdot \Delta) \vdash C}
      {\Delta \vdash A &
       {\Gamma, z :^k B} \vdash {C}}
\]
\end{small}%
where $\Gamma + \Delta$ acts pointwise by $x :^{n} A + x :^{m}
A = x :^{n+m} A$ and $n \cdot \Delta$ acts pointwise by $n \cdot x^{m} A
= x :^{nm} A$.  In the left rule, $\Gamma$ and $\Delta$ have the same
underlying variables and types (but potentially different counts), and
$f:^kA \to^n B$ abbreviates a context with the same variables and types
but $0$'s for all counts besides $f$'s.  The left rule says that if you
spend $k$ ``uses'' of a function that takes $n$ uses of an
argument, then you need $nk$ uses of whatever you use to
construct the argument, in order to get $k$ uses of the result.  

We can model this in the mode theory of a commutative monoid by using
context descriptors that are themselves non-linear: we define $A \to^n B
:= \U{c.c \odot (x^n)}{x:A}{B}$ where $x^n := x \odot x \odot \ldots
\odot x$ ($n$ times).  This has the following instances of \UL{}{} and
\UR{}:
\begin{small}
\[
\infer{\seq{\Gamma}{\beta}{A \to^n B}}
      {\seq{\Gamma, x:A}{\beta \odot x^n}{B}}
\qquad
\infer{\seq{\Gamma}{\beta}{C}}
      {f : \U{f.f \odot x^n}{x : A}{B} \in \Gamma &
        \beta \spr \beta'[f \odot (\alpha)^n/z] &
        \seq{\Gamma}{\alpha}{A} &
        \seq{\Gamma, z:B}{\beta'}{C} 
      }
\]
\end{small}%
In the left rule, $\beta'$ must be equal to some term $\beta'' \odot
z^k$ for some $k$ and $\beta''$ not mentioning $z$ (for this mode
theory, any term is a polynomial of variables), and the only structural
transformations are the commutative monoid equations, so the premise is
$\beta \deq (\beta'' \odot z^k) [f \odot (\alpha)^n/z] \deq \beta''
\odot f^k \odot (\alpha)^{nk}$.  Here $\beta''$ corresponds to the
$\Gamma$ in the above left rule (the resources of the continuation,
besides $z^k$) and $\alpha$ corresponds to $\Delta$.  The full proof of
adequacy is in the extended version:
\begin{theorem}[Logical adequacy for $n$-use variables]
$x_1:^{k_1} A_1,\ldots,x_n :^{k_n} A_n \vdash C$ iff
  \seq{x_1:A_1^*,\ldots,x_n:A_n^*}{x_1^{k_1} \odot \ldots \odot
    x_n^{k_n}}{C^*}, where $A^*$ translates $A \to^n B$ to 
$\U{c.c \odot (x^n)}{x:A^*}{B^*}$
\end{theorem}

\subsection{Comonads}  
\label{sec:example:bang}

Following \citet{benton94mixed,bentonwadler96adjoint}, we decompose the
$!$ exponential of intuitionistic linear logic as the comonad of an
adjunction between ``linear'' and ``cartesian'' categories.  We start
with two modes \dsd{l} (linear) and \dsd{c} (cartesian), along with a
commutative monoid $(\otimes,1)$ on \dsd{l} and a cartesian monoid
$(\times,\top)$ on \dsd{c}.  Next, we add a context descriptor from
\dsd{c} to \dsd{l} ($x : \dsd{c} \vdash \dsd{f}(x) : \dsd{l}$) that we
think of as including a cartesian context in a linear context.  This
generates types \wftype {\F{\dsd{f}(x)}{x : A_{\dsd{c}}}}{\dsd{l}} and
\wftype {\U{x.\dsd{f}(x)}{\cdot}{A_{\dsd{l}}}}{\dsd{c}} which are
adjoint $\F{\dsd{f}(x)}{x:-} \la {\U{x.\dsd{f}(x)}{\cdot}{-}}$.  The
bijection on hom-sets is defined using \FL\/ and \UR\/ and their
invertibility.  The comonad of the adjunction
\F{\dsd{f}(x)}{x:\U{c.\dsd{f}(c)}{\cdot}{A}} is the linear logic $!A$.

In LNL~\citep{benton94mixed}, $F(A \times B) \cong F(A) \otimes F(B)$
and $F(\top) \cong 1$ (these properties of $F$ are necessary to
prove that $!  A$ has weakening and contraction with respect to
$\otimes$, for example), which we can add to the mode theory by
equations $\dsd{f}(x \times y) \deq \dsd{f}(x) \otimes \dsd{f}(y)$ and
$\dsd{f}(\top) \deq 1$. By Theorem~\ref{lem:fusion-respect}, these
equations induce type isomorphisms because all of $F,\otimes,\times$ are
represented by \Fsymb-types in our framework.  For example, $F(A \times
B) \vdash F(A) \otimes F(B)$ is derived as follows:
\begin{small}
\[
\infer[\FL]{\seq{q:\F{\dsd{\dsd{f}(x)}}{x:\F{y \times z}{y:A,z:B}}}{q}{\F{z \otimes w}{z:\F{\dsd{f}(x)}{x:A},w:\F{\dsd{f}(x)}{x:B}}}}
      {\infer[\FL]{\seq{x:{\F{y \times z}{y:A,z:B}}}{\dsd{f}{(x)}}{\F{z \otimes w}{z:\F{\dsd{f}(x)}{x:A},w:\F{\dsd{f}(x)}{x:B}}}}
        {\infer[\FR]{\seq{y:A,z:B}{\dsd{f}{(y \times z)}}{\F{z \otimes w}{z:\F{\dsd{f}(x)}{x:A},w:\F{\dsd{f}(x)}{x:B}}}}
          {\dsd{f}{(y \times z)} \deq \dsd{f}(y) \otimes \dsd{f}(z) &
            \infer[\FR^*]{\seq{y:A,z:B}{\dsd{f}{(y)}}{\F{x.\dsd{f}(x)}{x:A}}}{} & 
            \infer[\FR^*]{\seq{y:A,z:B}{\dsd{f}{(z)}}{\F{x.\dsd{f}(x)}{x:B}}}{} & 
          }}}
\]
\end{small}%
Omitting these equations allows us to describe non-monoidal (or lax
monoidal, if we add only one direction) left adjoints: in the extended
version, we consider S4 $\Box$, and prove adequacy for it.

%% \begin{theorem}[Logical adequacy for Adjoint $!$]
%% Translate $F(A)^* = \F{\dsd{f}(x)}{x:A^*}$ and $G(A)^* =
%% \U{x.\dsd{f}(x)}{\cdot}{A}$ and products and functions as usual.  Then
%% $x_1:C_1,\ldots,x_n:C_n \vdash C$ in the cartesian category iff
%% \seq{x_1:C_1^*,\ldots,x_n:C_n^*}{x_1 \times \ldots \times x_n}{C^*}, and
%% a mixed sequent with cartesian and linear assumptions and a linear
%% conclusion $x_1:C_1,\ldots,x_n:C_n;y_1:A_1,\ldots,y_m:A_m \vdash A$
%% holds iff
%% \seq{x_1:C_1^*,\ldots,y_1:A_1^*,\ldots}{\dsd{f}(x_1)
%%   \otimes\ldots\otimes \dsd{f}(x_n) \otimes y_1 \otimes \ldots \otimes
%%   y_n}{A^*}.
%% \end{theorem}

\subsection{Monads}
\label{sec:example:monad}

We model a \Dia{}{A} modality with rules in the style of
\citet{pfenningdavies}
%% \[
%% \infer{\Gamma \vdash \possj{A}}
%%       {\Gamma \vdash \truej{A}}
%% \qquad
%% \infer{\Gamma \vdash \truej{\Dia{}{A}}}
%%       {\Gamma \vdash \possj{A}}
%% \qquad
%% \infer{\Gamma,\truej{\Dia{}{A}} \vdash \possj{C}}
%%       {\truej{A} \vdash \possj{C}}
%% \]
by a mode theory with two modes \dsd{t} and \dsd{p}
and context descriptor \oftp{x:\dsd{t}}{\dsd{g}(x)}{\dsd{p}}; we define
$\Dia{}{A} := \U{c.\dsd{g}(c)}{\cdot}{\F{\dsd{g}(x)}{x:A}}$.
This is always a monad, but it does not automatically have a tensorial
strength.  For example, if we have a monoid $(\otimes,1)$ on mode
\dsd{t} and try to derive strength
\[
\infer[\UR]
      {\seq{x : A, y : \Dia{\dsd{g}}{B}}{x \otimes y}{\Dia{\dsd{g}}{(A \otimes B)}}}
      {\infer[\UL]
        {\seq{x : A, y : \Dia{\dsd{g}}{B}}{\dsd{g}(x \otimes y)}{\F{\dsd{g}}{A \otimes B}}}
        {\dsd{g}(x \otimes y) \spr \subst{\beta'}{\dsd{g}(y)}{z} &
          \seq{x:A,y : \Dia{\dsd{g}}{B},z:\F{\dsd{g}}{B}}{\beta'}{\F{\dsd{g}}{A \otimes B}}
        }}
\]
\noindent we are stuck, because there is no way to rewrite $\dsd{g}(x
\otimes y)$ as a term containing $\dsd{g}(y)$.  If $\otimes$ is affine,
then we can weaken away $x$ and take $\beta' = z$---corresponding to the
context-clearing in the left rule for $\Dia{}{A}$ in
\citet{pfenningdavies}---but then in the right-hand premise we will only
have access to $z$, not $x$, so $\Diamond$ correctly represents a
non-strong monad in this setting.  In the extended version, we prove
adequacy for this and extend the mode theory to express strong monads.

\begin{theorem}[Logical adequacy for a monad]
We translate all types at mode \dsd{t}, representing
\Dia{}{A} as above. Then $\truej{A_1}, \ldots,
\truej{A_1} \vdash \truej{C}$ iff
\seq{x_1:A_1^*,\ldots,x_1:A_n^*}{x_1\otimes\ldots\otimes x_n}{C^*}, and 
$\truej{A_1}, \ldots, \truej{A_n} \vdash \possj{C}$ iff 
\seq{x_1:A_1^*,\ldots,x_1:A_n^*}{\dsd{g}(x_1\otimes\ldots\otimes
  x_n)}{\F{\dsd{g}}{C^*}}.  
%% The three ``native'' rules above are
%% \FR, \UR, and a composite of \UL\/ followed by \FL, respectively.
\end{theorem}

\subsection{Spatial Type Theory}

The spatial type theory for cohesion~\citep{shulman15realcohesion}
which motivated this work has an adjoint pair $\flat \la \sharp$,
where $\flat$ is a comonad and $\sharp$ is a monad, with some additional
properties.  In the one-variable case~\citep{ls16adjoint}, we analyzed
this as arising from an idempotent comonad\footnote{There it was an
  idempotent monad; the variance of \dsd{F} and \dsd{U} has been flipped
  in paper.} in the mode theory: we have a mode \dsd{c} with a cartesian
monoid $(\times,\top)$ and a context descriptor
\oftp{x:\dsd{c}}{\dsd{r}(x)}{\dsd{c}} such that $\dsd{r}(\dsd{r}(x))
\deq \dsd{r}(x)$ and there is a directed transformation $\dsd{r}(x) \spr
x$.  Then we define $\flat A := \F{\dsd{r}}{A}$ and $\sharp A :=
\Uempty{\dsd{r}}{A}$. These are adjoint, and the transformation gives
the counit $\F{\dsd{r}}{A} \vdash A$ and the unit $A \vdash
\Uempty{\dsd{r}}{A}$.  Now that we have a multi-assumptioned logic, we
can model the fact that $\flat{A}$ preserves products by the equational
axiom $\dsd{r}(x \times y) \deq \dsd{r}(x) \times \dsd{r}(y)$.  Overall,
we encode a simply-typed spatial type theory judgement $x_1 :
\crispj{A_1},\ldots;y_1:\cohesivej{B_1} \vdash \cohesivej{C}$ as
$\seq{x_1:A_1,\ldots,y_1:B_1,\ldots}{\dsd{r}(x_1)\times\ldots\times
  y_1\times\ldots}{C}$.  As a sequent calculus, the rules
from~\citep{shulman15realcohesion} are
\begin{small}
\[
\begin{array}{c}
\infer{\Delta;\Gamma \vdash C}
      {A \in \Delta &
       \Delta;\Gamma,A \vdash C}
\quad
\infer{\Delta; \Gamma \vdash {\Flat A}}
      {\Delta; \cdot \vdash {A}}
\quad
\infer{\Delta; \Gamma,\Flat{A} \vdash C}
      {\Delta,A; \Gamma \vdash C}
\quad
\infer{\Delta;\Gamma \vdash {\Sharp C}}
      {\Delta,\Gamma; \cdot \vdash C}
\quad
\infer{\Delta;\Gamma \vdash C}
      {\Sharp A \in \Delta &
        \Delta;\Gamma,A \vdash {C}}
\quad
\end{array}
\]
\end{small}
In order, these correspond to (1) the action of the contraction and
$\dsd{r}(x) \spr x$ transformations; (2) \FR\/ with weakening, using
monoidalness of \dsd{r} in one direction; (3) \FL; (4) \UR, using
monoidalness of \dsd{r} in the other direction and idempotence; (5) \UL,
with contraction.  This provides a satisfying explanation for the
unusual features of these rules, such as promoting all cohesive
variables to crisp in \Sharp{}-right, and eliminating a crisp \Sharp{}
in \Sharp{}-left, and illustrates how our framework can be used in
investigating extensions of homotopy type theory.

%% \subsection{Non-adjoints}

%% TODO E.g. the graded effects stuff, modalities in Lambek calculus  

\section{Equational Theory on Derivations}
\label{sec:equational}

In this section we give an equational theory describing $\beta\eta$-equality of
derivations.  We use this equational theory in the categorical semantics
below, and to reason about terms in encoded languages (for example, to
prove that a pair of entailments is an isomorphism, we show that the
maps compose to the identity up to these equations).

First, we need a notation for derivations of the $\alpha \spr \beta$
judgement in Figure~\ref{fig:2multicategory}.  We assume names for
constants are given in the signature $\Sigma$, and write $1_\alpha$ for
reflexivity, $s_1;s_2$ for transitivity (in diagramatic order), and
$s_1[s_2/x]$ for congruence.  We extend the signature $\Sigma$ to allow
axioms for equality of transformations $s_1 \deq s_2$ (for two
derivations of the same judgement $s_1,s_2 ::
\wfsp{\psi}{\alpha}{\beta}{p}$), and define equality to be the least
congruence closed under those axioms and some associativity, unit, and
interchange laws, which are the 2-category axioms extended to the
multicategorical case (see the extended version for details).  As with
equality of context descriptors, we think of all definitions as being
parametrized by \deq-equivalence-classes of transformations, not raw
syntax.

To simplify the axiomatic description of equality, we use a notation for
derivations where the admissible transformation, identity, and cut rules
are internalized as explicit rules---so the calculus has the flavor of
an explicit substitution one.
%% \[
%% \begin{array}{rcl}
%% \D & ::= & \Ident{x} \mid \Trd{s}{\D} \mid \Cut{\D_1}{\D_2}{x} \mid
%%  \FLd{x}{\Delta}{\D} \mid \FRd{\gamma}{s}{\vec{\D_i/x_i}} \mid \ULd{x}{}{s}{\vec{\D_i/x_i}}{z}{\D} \mid \URd{\Delta}{\D} 
%% \end{array}
%% \]
We write proof terms for these plus the 4 \Usymb/\Fsymb\, rules (the
hypothesis rule for atoms is derivable from these) as follows.
\begin{small}
\[
\begin{array}{c}
\infer{{\Gamma,x:A} \vdash_{x} x : {A}}{}
\qquad
\infer{{\Gamma} \vdash_{\alpha} \Trd{s}{d} : {A}}
      {s :: \alpha \spr \beta &
        {\Gamma} \vdash_{\beta} d : {A}}
\qquad
\infer{{\Gamma} \vdash_{\subst{\beta}{\alpha}{x}} \Cut{e}{d}{x} : {B}}
      {{\Gamma,x:A} \vdash_{\beta} e : {B} &
        {\Gamma} \vdash_{\alpha} d : {A}}
\\\\ 
\infer{{\Gamma,x:\F{\alpha}{\Delta},\Gamma'} \vdash_{\beta} (\FLd{x}{\Delta}{d}) : {C}}
      {{\Gamma,\Gamma',\Delta} \vdash_{\subst \beta {\alpha}{x}} d : {C}}
\quad
\infer{{\Gamma} \vdash_{\beta} \FRd{}{s}{\vec{d_i/x_i}} : {\F{\alpha}{\Delta}}}
      {%% \modeof{\Gamma} \vdash \gamma : \modeof{\Delta} & 
        s :: \beta \spr \tsubst{\alpha}{\gamma} &
        {\Gamma} \vdash_{\gamma} \vec{d_i/x_i} : {\Delta} 
      }
\\\\
\infer{{\Gamma} \vdash_{\beta} \ULd{x}{}{s}{\vec{d_i/x_i}}{z}{d} : {C}}
      {
        x:\U{x.\alpha}{\Delta}{A} \in \Gamma &
        s :: \beta \spr \subst{\beta'}{\tsubst{\alpha}{\gamma}}{z} &
        {\Gamma} \vdash_{\gamma} {\vec{d_i/x_i}} : {\Delta} &
        {\Gamma,\tptm{z}{A}} \vdash_{\beta'} d' : {C}
      }
\quad
\infer{{\Gamma} \vdash_{\beta} \URd{\Delta}{d} : {\U{x.\alpha}{\Delta}{A}}}
      {{\Gamma,\Delta} \vdash_{\subst{\alpha}{\beta}{x}} d : {A}}
\end{array}
\]
\end{small}

%% We write \FRs\/ for $\FRd{\vec{x/x}}{1_\alpha}{\Ident{x}/x} ::
%% \seq{\Gamma}{\alpha}{\F{\alpha}{\Delta}}$ when $\Delta \subseteq \Gamma$
%% and we write and \ULs{x} for $\ULd{x}{\vec{x/x}}{1_\alpha}{\Ident{x}/x}{z.z} ::
%% \seq{\Gamma}{\alpha}{A}$ when $x:\U{x.\alpha}{\Delta}{A} \in \Gamma$ and
%% $\Delta \subseteq \Gamma$.  

The equational theory of derivations is the least congruence containing
the following equations.  
\begin{small}
\[
\begin{array}{rcll} 
\Cut{\D}{\Ident{x}}{x} & \deq & \D \\
\Cut{\Ident{x}}{\D}{x} & \deq & \D \\
\Cut{\D_1}{\D_2}{x} & \deq & \D_1 \text{ if $x \# \D_1$}\\
\Cut{(\Cut{\D_1}{\D_2}{x})}{\D_3}{y} & \deq & \Cut{(\Cut{\D_1}{\D_3}{y})}{\Cut{\D_2}{\D_3}{y}}{x}\\
\end{array}
\qquad
\begin{array}{rcll}
\Trd{1}{\D} & \deq & \D\\
\Trd{(s_1;s_2)}{\D} & \deq & \Trd{{s_1}}{\Trd{{s_2}}{\D}} \\
\Trd{(\subst{s_2}{s_1}{x})}{\Cut{\D_2}{\D_1}{x}} & \deq & \Cut{\Trd{{s_2}}{\D_2}}{\Trd{{s_1}}{\D_1}}{x} \\
\end{array}
\]
\[
\begin{array}{rcll}
\Cut{(\FLd{x_0}{\Delta}{\D})}{\FRd{}{s}{\vec{\D_i/x_i}}}{x_0} & \deq & \Trd{(1_\beta[s/x_0])}{\D[\vec{\D_i/x_i}]} & \dsd{F\beta} \\
\Cut{(\ULd{x_0}{}{s}{\vec{\D_i}/x_i}{z}{\D'})}{\URd{\Delta}{\D}}{x_0} & \deq & \Trd{(s[1_{\alpha}/{x_0}])}{\Cut{\D'}{(\D[{\vec{d_i}/x_i}])}{z}} & \dsd{U\beta} \\
\D :: \seq{\Gamma,x:\F{\alpha}{\Delta},\Gamma'}{\beta}{C} & \deq &
\FLd{x}{\Delta}{\Cut{\D}{\FRd{}{1}{\Delta/\Delta}}{x}} & \dsd{F\eta}\\
\D :: \seq{\Gamma}{\beta}{\U{x.\alpha}{\Delta}{A}} & \deq & \URd{\Delta}{\Cut{(\ULd{x}{}{1}{\Delta/\Delta}{z}{z})}{\D}{x}} & \dsd{U\eta}\\
\end{array}
\]
\end{small}

In the top-left, the first two equations say that identity is a unit for
cut.  The third says that non-occurence of a variable is a projection.
The fourth is functoriality of cut.  In the top-right, the first two
rules say that the action of a transformation is functorial, and the
third says that it commutes with cut.  The typing in the third rule is
$\D_1 :: \seq{\Gamma}{\alpha'}{A}$ and $\D_2 ::
\seq{\Gamma,x:A}{\beta'}{C}$ and $s_1 :: \alpha \spr \alpha'$ and $s_2
:: \beta \spr \beta'$, so both sides are derivations of as derivations
of \seq{\Gamma}{\subst{\beta}{\alpha}{x}}{C}.  Finally, we have the
$\beta\eta$-laws for \dsd{F} and \dsd{U}.  The $\beta$ laws are the
principal cut cases from our cut elimination proof.  The $\eta$ laws
witness left-invertibility of \Fsymb\, and right-invertibility of
\Usymb.



\newcommand\cD{\ensuremath{\mathcal{D}}}
\newcommand\IndF[3]{\ensuremath{{#1}^\Fsymb_{{#2},{#3}}}}
\newcommand\IndU[4]{\ensuremath{{#1}^\Usymb_{{#2},{#3},{#4}}}}

\section{Categorical Semantics}
\label{sec:semantics}

In this section, we give a category-theoretic structure corresponding to
the above syntax.  First, we define a cartesian 2-multicategory as a
semantic analogue of the syntax in Figure~\ref{fig:2multicategory}. 

%The
%semantics uses total substitutions (for the entire context at once)
%instead of single-variable substitutions, and explicit weakening and
%exchange instead of named variables.

\begin{definition}
  A \textbf{(strict) cartesian 2-multicategory} consists of
  \begin{enumerate}
  \item A set $\M_0$ of \emph{objects}.
  \item For every object $B$ and every finite list of objects $(A_1,\dots,A_n)$, a category $\M(A_1,\dots,A_n;B)$.
    The objects of this category are \emph{1-morphisms} and its morphisms are \emph{2-morphisms}; we write composition of 2-morphisms as $\compv{s_1}{s_2}$.
  \item For each object $A$, an identity arrow $1_A\in\M(A;A)$.
  \item For any object $C$ and lists of objects $(B_1,\dots,B_m)$ and
    $(A_{i1},\dots,A_{in_i})$ for $1\le i\le m$, a composition functor
    $(g,(f_1,\dots,f_m)) \mapsto g\circ (f_1,\dots,f_m) : 
    \M(B_1,\dots,B_m;C) \times \prod_{i=1}^m \M(A_{i1},\dots,A_{in_i};B_i) \longrightarrow \M(A_{11},\dots,A_{mn_m};C)$.
    We write the action of this functor on 2-cells as $\comph{d}{(e_1,\dots,e_m)}$.
  \item For any function $\sigma : \{1,\dots,m\} \to \{1,\dots,n\}$ and
    objects $A_1,\dots,A_n,B$, a \emph{renaming} functor $f \mapsto
    f\sigma^* : \M(A_{\sigma 1},\dots,A_{\sigma m}; B) \to \M(A_1,\dots,A_n;B)$
  \item satisfying some equalities (see the extended version)
  \end{enumerate}
% satisfying some equations that we elide here.  
\end{definition}

The next three definitions will be used to describe the
\seq{\Gamma}{\alpha}{A} judgement.  

\begin{definition}
  A \textbf{functor of cartesian 2-multicategories} $F:\cD\to\M$ consists
  of a function $F_0 : \cD_0 \to \M_0$ and functors $\cD(A_1,\ldots,A_n;B)
  \to \M(F_0(A_1),\ldots,F_0(A_n);F_0(B))$ such that the chosen
  identities, compositions, and renamings are preserved (strictly).
\end{definition}

\begin{definition}
  A functor of cartesian 2-multicategories $\pi:\cD\to\M$ is a \textbf{local discrete fibration} if each induced functor of ordinary categories
  $\cD(A_1,\dots,A_n;B)\to\M(\pi A_1,\dots,\pi A_n;\pi B)$
  is a discrete fibration.
\end{definition}

We write $\cD_\alpha(A_1,\dots,A_n;B)$ for the fiber of such a functor over
$\alpha \in \M(\pi A_1,\dots,\pi A_n;\pi B)$; when $\pi$ is a local discrete
fibration, this fiber is a discrete set.

\begin{definition}
  If $\pi:\cD\to\M$ is a local discrete fibration, then a morphism
  $\xi\in\cD(A_1,\dots,A_n;B)$ is \textbf{opcartesian} if all diagrams
  of the lefthand form are pullbacks of categories, and a morphism
  $\xi\in\cD(\vec C,B,\vec D;E)$ is \textbf{cartesian at $B$} if all
  diagrams of the right-hand form are pullbacks of categories: 
  \[ \xymatrix{
    \cD(\vec C,B,\vec D;E) \ar[r]^-{(-)\circ_B \xi} \ar[d]_\pi &
    \cD(\vec C,\vec A,\vec D;E) \ar[d]^\pi \\
    \M(\pi\vec C,\pi B, \pi\vec D; \pi E) \ar[r]_-{(-)\circ_{\pi B} \pi\xi} &
    \M(\pi\vec C,\pi\vec A,\pi\vec D;\pi E)
  }
  \qquad
  \xymatrix{
    \cD(\vec A;B) \ar[r]^-{\xi\circ_B (-)} \ar[d]_\pi &
    \cD(\vec C,\vec A,\vec D;E) \ar[d]^\pi \\
    \M(\pi\vec A;\pi B) \ar[r]_-{\pi\xi\circ_{\pi B} (-)} &
    \cD(\pi\vec C,\pi\vec A,\pi\vec D;\pi E)}
  \]
  Given $\mu:(p_1,\dots,p_n) \to q$ in $\M$, we say that $\pi$ \textbf{has $\mu$-products} if for any $A_i$ with $\pi A_i = p_i$, there exists a $B$ with $\pi B = q$ and an opcartesian morphism in $\cD_\mu(A_1,\dots,A_n;B)$.
  Dually, we say $\pi$ \textbf{has $\mu$-homs} if for any $i$, any $B$ with $\pi B = q$, and any $A_j$ with $\pi A_j = p_j$ for $j\neq i$, there exists an $A_i$ with $\pi A_i = p_i$ and a cartesian morphism in $\cD_\mu(A_1,\dots,A_n;B)$.
  We say that $\pi$ is an \textbf{opfibration} if it has $\mu$-products for all $\mu$, a \textbf{fibration} if it has $\mu$-homs for all $\mu$, and a \textbf{bifibration} if it is both an opfibration and a fibration.
\end{definition}

The proofs of the following soundness and completeness results are in
the extended version:

\begin{theorem}[Mode theory presents a mutilcategory]
\label{thm:completeness-mode-theory}
A mode theory $\Sigma$ presents a cartesian 2-multicategory $\M$, where
$\M_0$ is the set of modes, and an object of $\M(p_1,\ldots,p_n;q)$ is a
term $\oftp{x_1:p_1,\ldots,x_n:p_n}{\alpha}{q}$ and a morphism of $\M(p_1,\ldots,p_n;q)$ is a structural transformation
$s :: \wfsp{\psi}{\alpha}{\beta}{q}$, both considered modulo $\deq$.
\end{theorem}

\begin{theorem}[Completeness/Syntactic Bifibration]
For a fixed mode theory $\M$, the syntax presents a bifibration $\pi : \cD \to \M$, where:
\begin{itemize}
\item Objects of $\cD$ are pairs $(p, \wftype{A}{p})$;
\item 1-morphisms $\Gamma \to B$, i.e., objects of $\cD(\Gamma; B)$, are pairs $(\alpha, d :: \seq{\Gamma}{\alpha}{B})$ (up to \deq); 
\item 2-morphisms $(\alpha, d) \to (\alpha', d')$ are structural
  transformations $s :: \alpha \spr \alpha'$ such that $\Trd{s}{d'} \deq d$;
\item the $\mu$-products are \Fsymb-types, and the $\mu$-homs are \Usymb-types.
\end{itemize}
The functor $\pi : \cD \to \M$ is given by first projection on objects and 1-morphisms, and sends 2-morphisms to the underlying structural transformations.
\end{theorem}

\begin{theorem}[Soundness/Interpretation in any bifibration]
Fix a bifibration $\pi : \cD \to \M$.  Then there is a function $\llb -
\rrb$ from types \wftype{A}{p} to $\llb A \rrb \in \cD_0$ with $\pi(\llb
A \rrb) = p$ and from $\deq$-classes of derivations $\seq{x:A_1, \ldots,
  x_n:A_n}{\alpha}{C}$ to morphisms $d \in \cD(\llb A_1 \rrb, \dots, \llb
A_n \rrb) \to \llb C \rrb$, such that $\pi(d) = \alpha$.
\end{theorem}


%% 
% non-associative products NO
% ordered products YES
%         implications NO
% linear products     NO
%        implications NO
% multi-use functions YES
%           products NO
% affine products YES
%        implications YES
% relevant NO
% cartesian products     YES
%           implications YES
% BI NO
% Adjoint ! NO
% Adjoint Box YES
% Subexponentials NO
% Monads nonstrong YES
%        strong NO
%        box-strong NO
% spatial NO

\section{Logical Adequacy}
\label{sec:adequacy-long}

Suppose we are representing some object logic, like the examples from
Section~\ref{sec:exampleencodings}, in our framework.  In general, for a
specific mode theory, the framework will have more types than the object
logic.  For example, if we represent a logic with a binary product by a
product in the mode theory, then the framework will have not only \F{x
  \otimes y}{x:A,y:B}, but also a primitive triple product \F{x \otimes
  y \otimes z}{x:A,y:B,z:C}, and so on.  If we represent a modal logic
with a monad by its adjoint decomposition, then the framework will have
not only the $\Usymb \Fsymb$ composite, but also the \Usymb\/ and
\Fsymb\/ types separately.  Thus, in general we will define a
translation from object-logic sequents $J$ to framework sequents $J^*$,
in such a way that $J$ is provable in the object logic iff $J^*$ is
provable in the framework.  No claims are made about framework
derivations of sequents that are not in the image of the translation.
We call this \emph{logical adequacy}, because it says that entailment in
the object logic is soundly and completely represented by entailment in
the framework.  We plan to consider stronger adequacy theorems, which
extend this logical correspondence to an isomorphism on
equivalence-classes of proofs.

We will often use the following lemma:  

\begin{lemma}[0-use Strengthing] \label{lem:0-use-strengthening}
We say that a formula $\F{\alpha}{\Delta}$ and \U{c.\alpha}{\Delta}{A}
is relevant if every variable from $\Delta$ (and $c$ for \Usymb) occurs
at least once in $\alpha$.

Suppose the mode theory has the property that for all $x$, $\alpha$,
$\beta$, if $\alpha \spr \beta$ and $x \# \alpha$ then $x \# \beta$ (in
particular, equations must have the same variables on both sides).
Suppose additionally a sequent \seq{\Gamma}{\alpha}{A} such that every
\Fsymb/\Usymb\/ subformula of $\Gamma,A$ is relevant.

Then if $\D :: \seq{\Gamma}{\alpha}{A}$ and $\vec{x}$ are variables such
that $\vec{x} \# \alpha$ then there is a $\D' ::
\seq{\Gamma-\vec{x}}{\alpha}{A}$ and $size(\D') \le size(\D)$.
\end{lemma}

\begin{proof}
The proof is by induction on $\D$.  In all cases, the assumption that
every formula is relevant is preserved for all premises of a rule by the
subformula property.

\begin{itemize}
\item In the case for a variable $x:P$ with transformation $\beta \spr
  x$, we need to show that the variable $x$ being used is not one of the
  ones being strengthened away.  But if $x$ were in $\vec{x}$, then we
  would have $x \# \beta$, and therefore by the assumption, $x \# x$, a
  contradiction.  Therefore $x \in {\Gamma-\vec{x}}$, and we can reapply
  the variable rule, which has the same size.

\item In the case for \FR, we have $\vec{x} \# \beta$, so by the
  reduction condition, $\vec{x} \# \alpha[\gamma]$.  By the relevance
  condition, all variables that $\gamma$ substitutes occur in $\alpha$,
  which means each component of $\gamma$ occurs in $\alpha[\gamma]$.
  Therefore $\vec{x} \# \gamma$.  We can use the inductive hypothesis to
  obtain a no-bigger derivation of \seq{\Gamma-\vec{x}}{\gamma}{\Delta},
  and then reapply \FR.

\item In the case for \FL, we distinguish cases on whether $x \in
  \vec{x}$ or not.

  If it is, then this is an elimination on a 0-use variable that we
  would like to drop.  Because $x \# \beta$, $\beta[\alpha/x] = \beta$,
  and note that $\Delta \# \beta$ because it occurs only in $\alpha$.
  Thus, if we appeal to the inductive hypothesis on the premise with
  $\vec{x}-x,\Delta$, we get
  \seq{(\Gamma,\Gamma',\Delta)-(\vec{x}-x,\Delta)}{\beta}{C}, 
  i.e.
  \seq{\Gamma,x:\F{\alpha}{\Delta},\Gamma',-\vec{x}}{\beta}{C}
  as desired.  That is, we recursively drop all variables that came from
  the elimination, in addition to any others that we were trying to drop
  besides $\vec{x}$.  

  If it is not, then $\vec{x} \# \beta$ and $\vec{x} \# \alpha$ (by
  scoping) implies $\vec{x} \# \beta[\alpha/x]$, so by the inductive
  hypothesis we get a no-bigger derivation of
  \seq{\Gamma,\Gamma'-\vec{x},\Delta}{\beta[\alpha/x]}{C}, and we can
  reapply \FL (because the principal variable $x$ of the left rule is
  not removed).

\item In the case for \UR, we have $\vec{x}$ a collection of variables
  bound in $\Gamma$, so $\vec{x} \# \alpha$ (since the domain of
  $\alpha$ is not $\Gamma$) in addition to $\vec{x} \# \beta$.  Thus
  $\vec{x} \# \subst\alpha{\beta}{x}$, so the inductive hypothesis gives a
  no-bigger derivation of
  \seq{\Gamma-\vec{x},\Delta}{\alpha[\beta/x]}{A}, and we can reapply
  the rule.

\item In the case for \UL, we distinguish cases on whether $z$ occurs in
  $\beta'$.  

  If $z \# \beta'$, then this \UL\/ is generating a 0-use assumption $z$, so
  we can remove it and the \UL\/ along with $\vec{x}$.  That is, we appeal
  to the inductive hypothesis on the continuation with $\vec{x},z$,
  which gives \seq{\Gamma,z:A-(\vec{x},z)}{\beta'}{C}, i.e.
  \seq{\Gamma-\vec{x}}{\beta'}{C}.  We also have $\beta \spr \beta'$
  because the $[\alpha[\gamma]/z]$ substitution cancels.  So we have
  \seq{\Gamma-\vec{x}}{\beta}{C} by Lemma~\ref{lem:respectspr}.
  
  If $z$ occurs in $\beta'$, we further distinguish cases on whether $x
  \in \vec{x}$ or not.  

  If it is not, then we know $\vec{x} \# \beta$, so pushing this along
  the transformation gives $\vec{x} \# \beta'[\alpha[\gamma]/z]$.  Thus
  $\vec{x} \# \beta'$ (note that $z$ cannot be in $\vec{x}$ because it
  is bound only in the continuation), and because $z$ occurs in
  $\beta'$, $\alpha[\gamma/z]$ occurs in the substitution, so $\vec{x}
  \# \alpha[\gamma]$.  By the relevance assumption, each term in
  $\gamma$ also occurs after the substitution, so $\vec{x} \# \gamma$ as
  well.  Thus, by the inductive hypotheses we get no-bigger derivations
  of \seq{\Gamma-\vec{x}}{\gamma}{\Delta} and
  \seq{\Gamma-\vec{x},z:A}{\beta'}{C}, and the principal $x$ survives in
  $\Gamma-\vec{x}$, so we can reapply \UL.

  Finally, if $x \in \vec{x}$ and $z \in \beta'$, then we have $x \#
  \beta$, so $x \# \beta'[\alpha[\gamma]/z]$ by moving along the
  transformation, and then $x \# \alpha[\gamma]$ by the fact that $z$
  occurs.  However, this contradicts the relevance assumption on
  $\U{x.\alpha}{\Delta}{A}$, which says that $x$ occurs in $\alpha$.  
\end{itemize}

\end{proof}

\begin{lemma} \label{lem:spr-doesnt-introduce}
Suppose each axiom $c : \alpha \spr \beta$ has the property that $x \#
\alpha$ implies $x \# \beta$.  Then for any derivation of $\alpha \spr
\beta$, $x \# \alpha$ implies $x \# \beta$.
\end{lemma}

\begin{proof}
The cases for reflexivity is immediate, and the case for axioms is
assumed.  In the case for transitivity $\alpha \spr \beta_1 \spr
\beta_2$, we get $\vec{x} \# \beta_1$ by the first inductive hypothesis
and then $\vec{x} \# \beta_2$ by the second.  In the case for
congruence, we have $\vec{x} \# \alpha[\beta/y]$.  This means that
either $\vec{x} \# \alpha$ and $\vec{x} \# \beta$, or $\vec{x} \#
\alpha$ and $y \# \alpha$ (in which case $\vec{x}$ might occur in
$\beta$).  In the first case, we get $\vec{x} \# \alpha'$ and $\vec{x}
\# \beta'$ by the inductive hypotheses, so $\vec{x} \#
\alpha'[\beta'/y]$.  In the second, we get $\vec{x},y \# \alpha'$ by the
inductive hypothesis, so $\alpha'[\beta'/y] = \alpha'$, and $\vec{x} \#
\alpha'$.
\end{proof}

\subsection{Ordered Logic (Product Only)}

\newcommand\dotLd[2]{\ensuremath{\mathord{\odot}}\dsd{L}^{#1}(#2)}
\newcommand\dotRd[2]{\ensuremath{\mathord{\odot}}\dsd{R}(#1,#2)}

As a first example of an adequacy proof, we consider ordered logic with only $A \odot B$:
\[
\infer{\seql{A}{o}{A}}{}
\quad
\infer{\seql{\Gamma,\Delta,\Gamma'}{o}{C}}
      {\seql{\Gamma,A,\Gamma'}{o}{C} &
        \seql{\Delta}{o}{A}}
\quad
\infer{\seql{\Gamma,A \odot B,\Gamma'}{o}{C}}
      {\seql{\Gamma,A,B,\Gamma'}{o}{C}}
\quad
\infer{\seql{\Gamma,\Delta}{o}{A \odot B}}
      {\seql{\Gamma}{o}{A} &
        \seql{\Delta}{o}{B}}
\]

We use a mode theory with a monoid $(\odot,1)$, so the only
transformation axioms are equality axioms for associativity and unit.  

The type translation is given by $P^* := P$ and $(A \odot B)^* := \F{x
  \odot y}{x:A^*,y:B^*}$.  A context $(x_1:A_1,\ldots,x_n:A_n)^* :=
x_1:A_1^*,\ldots,x_n:A_n^*$.  Writing $\vars{x_1:A_1,\ldots,x_n:A_n} :=
x_1 \odot \ldots \odot x_n$, a sequent $\seql{\Gamma}{o}{A}$ is
translated to \seq{\Gamma^*}{\vars{\Gamma}}{A^*}.

We use the following properties of the mode theory:
\begin{itemize}
\item If ${\vars{\Gamma^*}} \deq {x}$ then $\Gamma$ is $x:Q$ for some
  $Q$.  
\item If $\vars{\Gamma} \deq \alpha_1 \odot \alpha_2$, then there exist
  $\Gamma_1,\Gamma_2$ such that $\Gamma = \Gamma_1,\Gamma_2$ and
  $\vars{\Gamma_1} \deq \alpha_1$ and $\vars{\Gamma_2} \deq \alpha_2$.
\item $A^*$ and $\Gamma^*$ are relevant propositions, and the monoid
  axioms preserve variables, so by Lemma~\ref{lem:0-use-strengthening} we can
  strengthen away any variables that are not in the context descriptor.  
\end{itemize}

Using these defintions, we have

\begin{theorem}[Logical adequacy for $\seql{}{o}{}$] 
$\seql{\Gamma}{o}{A}$ iff $\seq{\Gamma^*}{\vars{\Gamma}}{A^*}$
\end{theorem}

\begin{proof}
The forward direction is by induction on \seql{\Gamma}{o}{A}, where 
the inference rules for $\odot$ are derived as follows:

\[
\infer[\FL]{\seq{\Gamma^*,z:\F{x \otimes y}{x:A^*,y:B^*},{\Gamma'}^*}{\vars{\Gamma}\odot z \odot \vars{\Gamma'}}{C}}
      {\infer[Lemma~\ref{lem:exchange}]
        {\seq{\Gamma^*,{\Gamma'}^*,x:A,y:B}{\vars{\Gamma}\odot x \odot y \odot \vars{\Gamma'}}{C}}
        {\seq{\Gamma^*,x:A,y:B,{\Gamma'}^*}{\vars{\Gamma}\odot x \odot y \odot \vars{\Gamma'}}{C}}}
\]

\[
\infer{\seq{\Gamma^*,\Delta^*}{\vars{\Gamma} \odot \vars{\Delta}}{\F{x \odot y}{x:A,y:B}}}
      {{\vars{\Gamma} \odot \vars{\Delta}} \spr (x \odot y)[\vars{\Gamma}/x,\vars{\Delta}/y]
        \infer[Lemma~\ref{lem:weakening}]
              {\seql{\Gamma^*,\Delta^*}{\vars{\Gamma}}{A}}
              {\seql{\Gamma^*}{\vars{\Gamma}}{A}} &
        \infer[Lemma~\ref{lem:weakening}]
              {\seql{\Gamma^*,\Delta^*}{\vars{\Gamma}}{A}}
              {\seql{\Delta^*}{\vars{\Delta}}{B}}}
\]

The backward direction is also by induction on the given derivation:
\begin{itemize}
\item For identity
\[
\infer{\seq{\Gamma^*}{\vars{\Gamma}}{P}}
      {{\vars{\Gamma^*}} \spr {x} &
        x : P \in \Gamma^*}
\]
Because the only structural transformation axioms are equalities for
associativity and unit, we have ${\vars{\Gamma^*}} \deq {x}$, which in
turn implies that $\Gamma$ is $x:Q$ for some $Q$ (because if $\Gamma$ is
empty, does not contain $x$, or contains anything else, \vars{\Gamma}
will not equal $x$).  By definition, this implies $Q = P$, so $\Gamma$
is $x:P$.  Therefore the identity rule applies.

\item For \FR, because the only type that encodes to \Fsymb is $\odot$,
  we have
\[
\infer{\seq{\Gamma^*}{\vars{\Gamma}}{\F{x \otimes y}{x:A_1^*,y:A_2^*}}}
      {\vars{\Gamma} \deq \alpha_1 \odot \alpha_2 &
       \seq{\Gamma^*}{\alpha_1}{A_1^*} &
       \seq{\Gamma^*}{\alpha_2}{A_2^*}
      }
\]
By properties of the mode theory, $\Gamma = \Gamma_1,\Gamma_2$ with
$\vars{\Gamma_i} \deq \alpha_i$, so we have derivations of
\seq{\Gamma^*}{\vars{\Gamma_i}}{A_i^*}.  Because 0-use strengthening
applies, we can strengthen these to
\seq{\Gamma_i^*}{\vars{\Gamma_i}}{A_i^*}.  Then the inductive hypothesis
gives \seql{\Gamma_i}{A_i}, so applying the $\odot$ right rule gives the
result.

\item For \FL, because the only type encoding to $\Fsymb$ is $A \odot
  B$, we have
\[
\infer{\seq{\Gamma^*,z:\F{x \odot y}{x:A^*,y:B^*},{\Gamma'}^*}{\vars{\Gamma} \otimes z \otimes \vars{\Gamma'}}{C^*}}
      {\seq{\Gamma^*,{\Gamma'}^*,x:A^*,y:B^*}{\vars{\Gamma} \otimes (x \otimes y) \otimes \vars{\Gamma'}}{C^*}}
\]
By exchange (Lemma~\ref{lem:exchange}), we have a no-bigger derivation
of
{\seq{\Gamma^*,x:A^*,y:B^*,{\Gamma'}^*}{\vars{\Gamma} \otimes (x \otimes y) \otimes \vars{\Gamma'}}{C^*}} 
so applying the IH gives 
\seql{{\Gamma,x:A,y:B,{\Gamma'}}}{o}{C}, and then $\odot$-left gives the result.
\end{itemize}
\end{proof}

\subsection{Affine Logic}

Consider the following rules for affine logic, where the context is
represented by a list of assumptions labeled with variables, and $\Gamma
\splits \Delta_1,\Delta_2$ means interleaving $\Delta_1$ and $\Delta_2$
in some order equals $\Gamma$.

\begin{small}
\[
\begin{array}{c}
\infer{\seqa{\Gamma}{P}}{P \in \Gamma}
\qquad
\infer{\seqa{\Gamma,z:A\otimes B,\Gamma'}{C}}
      {\seqa{\Gamma,\Gamma',x:A,y:B}{C}}
\qquad
\infer{\seqa{\Gamma}{A \otimes B}}
      {\Gamma \splits \Delta_1;\Delta_2 &
        \seqa{\Delta_1}{A} &
        \seqa{\Delta_2}{B}}
\\\\
\infer{\seqa{\Gamma}{A \lolli B}}
      {\seqa{\Gamma,x:A}{B}}
\qquad
\infer{\seqa{\Gamma}{C}}
      {\Gamma \splits \Delta_1;\Delta_2;(f:A \lolli B) &
        \seqa{\Delta_1}{A} &
        \seqa{\Delta_2,z:B}{C}
      }
\end{array}
\]
\end{small}

\noindent Weakening and exchange are admissible for these rules.  

Using the mode theory from Section~\ref{sec:ex:affine}, we translate the
propositions and contexts of adjoint logic as follows:
\[
\begin{array}{rcl}
P^* & = & P \\
(A\otimes B)^* & = & \F{x \otimes y}{x:A^*,y:B^*} \\
(A\lolli B)^* & = & \U{c.c \otimes x}{x:A^*}{B^*} \\
\\
\cdot^* & = & \cdot\\
(\Gamma,x:A)^* & = & \Gamma^*,x:A^*\\
\end{array}
\]
We also define a function that collects the variables from $\Gamma$ as a
context descriptor:
\[
\begin{array}{rcl}
\vars{\cdot} & = & 1\\
\vars{(\Gamma,x:A)} & = & \vars{\Gamma} \otimes x\\
\end{array}
\]

Overall, we have
\begin{theorem}[Logical adequacy for $\seqa{}{}$] ~\\
$\seqa{\Gamma}{A}$ iff $\seq{\Gamma^*}{\vars{\Gamma}}{A^*}$
\end{theorem}

\begin{proof}

The forward direction is by induction on $\seqa{\Gamma}{A}$:
\begin{itemize}
\item For the hypothesis rule, we need to show
  \seq{\Gamma^*}{\vars{\Gamma}}{P}.  Because $x$ is in $\Gamma$, we can
  prove by induction on $\Gamma$ that $x:P$ is in $\Gamma^*$ and that
  $\vars{\Gamma} \deq \alpha \otimes x$.  Thus, the weakening
  transformation gives $\alpha \otimes x \spr 1 \otimes x \deq
  x$. Therefore we can derive
\[
\infer[\dsd{v}]
      {\seq{\Gamma^*}{\vars{\Gamma}}{P}}
      {x:P \in \Gamma & 
        {\vars{\Gamma}} \spr x}
\]

\item For $\otimes$-left, the inductive hypothesis gives
\seq{\Gamma^*,\Gamma'^*,x:A^*,y:B^*}{\vars{\Gamma} \otimes \vars{\Gamma'} \otimes x \otimes y}{C^*}
and we want 
\seq{\Gamma^*,z:\F{x\otimes y}{x:A^*,y:B^*},\Gamma'^*}{\vars{\Gamma} \otimes z \otimes \vars{\Gamma'}}{C^*}.
This is \FL\/ on the inductive hypothesis, with using associativity and commutativity of
$\otimes$ from the mode theory to move $x \otimes y$ to the end.  

\item 
For $\otimes$-right, we 
have \seq{\Delta_1^*}{\vars{\Delta_1}}{A^*}
and \seq{\Delta_2^*}{\vars{\Delta_2}}{B^*} by the inductive hypotheses,
which we can weaken to
 \seq{\Delta_1^*,\Delta_2^*}{\vars{\Delta_1}}{A^*}
and 
 \seq{\Delta_1^*,\Delta_2^*}{\vars{\Delta_2}}{B^*}, 
and then exchange to 
 \seq{\Gamma^*}{\vars{\Delta_1}}{A^*}
and 
 \seq{\Gamma^*}{\vars{\Delta_2}}{B^*} (using a lemma that when $\Gamma \splits \Delta_1,\Delta_2$,
$\Gamma$ and $(\Delta_1,\Delta_2)$ differ only in order, and that
 reordering is preserved by the mapped application of $*$).  
To apply \FR\/ to derive \seq{\Gamma^*}{\vars{\Gamma}}{\F{x \otimes y}{x:A,y:B}}, it thus suffices to show that 
$\vars{\Gamma} \spr \vars{\Delta_1}\otimes\vars{\Delta_2}$.  In fact
they are $\deq$,  which we can
prove by induction on $\Gamma \splits \Delta_1,\Delta_2$ using the
commutative monoid laws.  

\item For $\lolli$-right, we have
  \seq{\Gamma^*,x:A^*}{\vars{\Gamma}\otimes x}{B^*}
by the inductive hypothesis, which is exactly the premise of using \UR\/
to prove
  \seq{\Gamma^*}{\vars{\Gamma}}{\U{c.c\otimes x}{x:A^*}{B^*}}.  

\item For $\lolli$-left, we have \seq{\Delta_1^*}{\vars{\Delta_1}}{A}
  and \seq{\Delta_2^*,z:B^*}{\vars{\Delta_2}\otimes z}{C^*} by the
  inductive hypothesis, and by similar reasoning to the $\otimes R$
  case, we can weaken and exchange to
  \seq{\Gamma^*,\Gamma'^*}{\vars{\Delta_1}}{A} and
  \seq{\Gamma^*,\Gamma'^*,z:B^*}{\vars{\Delta_2}\otimes z}{C^*} and then
  finally weaken to \seq{\Gamma^*,f:(A\lolli
    B)^*,\Gamma'^*}{\vars{\Delta_1}}{A} and \seq{\Gamma^*,f:(A\lolli
    B)^*,\Gamma'^*,z:B^*}{\vars{\Delta_2}\otimes z}{C^*}.
Thus, we only need to show the transformation premise of
\[
\infer[\UL]{\seq{\Gamma^*,f:(A\lolli B)^*,\Gamma'^*}{\vars{\Gamma}\otimes f \otimes \vars{\Gamma'}}{C^*}}
      {  
        \begin{array}{l}
        {\vars{\Gamma}\otimes f \otimes \vars{\Gamma'}} \spr
        (\vars{\Delta_2} \otimes z)[f \otimes \vars{\Delta_1} /z] \\
        \seq{\Gamma^*,f:(A\lolli B)^*,\Gamma'^*}{\vars{\Delta_1}}{A} \\
        \seq{\Gamma^*,f:(A\lolli B)^*,\Gamma'^*,z:B^*}{\vars{\Delta_2}\otimes z}{C^*}
        \end{array}
      }
\]
In fact 
${\vars{\Gamma}\otimes f \otimes \vars{\Gamma'}} \deq (\vars{\Delta_2} \otimes f \otimes \vars{\Delta_1})$,
which again follows from $\Gamma \splits \Delta_1,\Delta_2$, using associativity and commutativity
of $\otimes$.  
\end{itemize}
In terms of structural property placement, observe that the above proof
uses only identity transformations on \FR\/ and \UL, and uses the
\dsd{w} axiom only at the leaves.  

We need the following facts about the mode theory.  

\begin{lemma} \label{lem:affine-mode-1}
If $\alpha \spr \beta$ then $\alpha \deq \beta \otimes \beta'$
for some $\beta'$.
\end{lemma}
\begin{proof}

In the case for weakening $\alpha \spr 1$ (the only axiom), take
$\beta' = \beta$.  In the case for reflexivity take $\beta' =
1$.  In the case for transitivity, we have $\alpha \spr \beta_1
\spr \beta_2$.  By the second inductive hypothesis, we have
$\beta_1 \deq \beta_2 \otimes \beta_2'$, and by the first we have
$\alpha \deq \beta_1 \otimes \beta_1'$, so 
$\alpha \deq \beta_2 \otimes (\beta_2' \otimes \beta_1')$.  

In the case for congruence, we have
$\alpha_1[\alpha_2/x] \spr \beta_1[\beta_2/x]$
and the inductive hypotheses give
    $\alpha_1 \deq \beta_1 \otimes \beta_1'$
and $\alpha_2 \deq \beta_2 \otimes \beta_2'$.  
Thus, 
$\alpha_1[\alpha_2/x] \deq \beta_1[\beta_2 \otimes \beta_2'/x] \otimes \beta_1'[\beta_2 \otimes \beta_2'/x]$,
and then the right-hand side equals 
$\beta_1[\beta_2/x] \otimes \beta_2^n \otimes \beta_1'[\beta_2 \otimes \beta_2'/x]$
for some $n$.  
This is because, 
for this mode theory, we can rewrite any $\alpha$ as $\alpha' \otimes (x
\otimes \ldots \otimes x)$ where $\alpha'$ does not contain $x$ (because
any context descriptor is equal to a ``polynomial'' giving the
multiplicity of each variable), so in general we have $\alpha[\beta_1
  \otimes \beta_2/x] \deq \alpha' \otimes (\beta_1 \otimes \beta_2)^n
\deq \alpha' \otimes \beta_1^n \otimes \beta_2^n \deq \alpha[\beta_1/x]
\otimes \beta_2^n$.
\end{proof}

\begin{lemma} \label{lem:affine-mode-5}
If $\vars{\Gamma} \deq \alpha_1 \otimes \alpha_2$, 
then $\Gamma \splits \Gamma_1,\Gamma_2$ with 
$\vars{\Gamma_1} \deq \alpha_1$ 
and $\vars{\Gamma_2} \deq \alpha_2$.  
\end{lemma}
\begin{proof}
We define the splitting $\Gamma \splits \Gamma_1,\Gamma_2$ adding each
variable in $\Gamma$ to $\Gamma_1$ if it occurs in $\alpha_1$, or
$\Gamma_2$ if it occurs in $\alpha_2$ and not $\alpha_1$ (occurrence
respects \deq).  Because $\deq$ is associativity, commutativity, and
unit, $\alpha_1$ and $\alpha_2$ have no duplicates and every variable
from \vars{\Gamma} occurs exactly once in one or the other, which gives
$\vars{\Gamma_1} \deq \alpha_1$ and $\vars{\Gamma_2} \deq \alpha_1$.
%% FIXME: last sentence is a little sketchy
\end{proof}

\begin{lemma} \label{lem:affine-mode-4}
If $\vars{\Gamma} \spr \alpha$, then there is a $\Gamma'$ such that
$\Gamma \ge \Gamma'$ and $\alpha \deq \vars{\Gamma'}$.
\end{lemma}

\begin{proof}
By Lemma~\ref{lem:affine-mode-1}, $\vars{\Gamma} \deq \alpha \otimes
\beta$ for some $\beta$.  By Lemma~\ref{lem:affine-mode-5}, this means
$\Gamma \splits \Gamma_1,\Gamma_2$ with $\vars{\Gamma_1} \deq \alpha$.
Then the fact that $\Gamma \splits \Gamma_1,\Gamma_2$ implies $\Gamma
\ge \Gamma_1$ gives the result.
\end{proof}

\begin{lemma} \label{lem:strengthening-affine}
If $\seq{{\Gamma_0}^*}{\alpha}{A^*}$ and $\vec{x} \# \alpha$ then there
is a no larger derivation of $\seq{{\Gamma}^*-\vec{x}}{\alpha}{A^*}$
\end{lemma}

\begin{proof}
We will use Lemma~\ref{lem:0-use-strengthening}.  First, no variables
are free in the range of weakening, so
Lemma~\ref{lem:spr-doesnt-introduce} gives that $\alpha \spr \beta$ and
$x \# \alpha$ imply $x \# \beta$.  Second, we prove by induction that
every subformula of $\Gamma^*$ and $A^*$ is relevant, because the only
context descriptors used are \F{x \otimes y}{x:A,y:B} and \U{c.c \otimes x}{x:A}{B}.
\end{proof}

We now prove that if $\seq{\Gamma^*}{\vars{\Gamma}}{A^*}$ then
$\seqa{\Gamma}{A}$.  The proof is by induction on the size of the
assumption, because we will sometimes use
Lemma~\ref{lem:strengthening-affine} before appealing to the inductive
hypothesis.
\begin{itemize}
\item In the case for the assumption rule, we have
\[
\infer{\seq{\Gamma^*}{\vars{\Gamma}}{P}}
      {x:P \in \Gamma^* &
       \vars{\Gamma} \spr x}
\]
Since $x:P \in \Gamma^*$, $x:P \in \Gamma$, and we can apply the affine
logic rule.  

\item In the case where \UR\/ was used to derive
  \seq{\Gamma^*}{\vars{\Gamma}}{A^*}, $A$ must be $A_1 \lolli A_2$
  (because no other types encode to \Usymb), and the premise is
  \seq{\Gamma^*,x:A_1^*}{\vars{\Gamma}\otimes x}{A_2^*}.  The inductive
  hypothesis gives $\seqa{\Gamma,x:A_1}{A_2}$, and we can apply
  $\lolli$-right.

\item In the case where \FR\/ was used, the conclusion must be $(A_1
  \otimes A_2)$, and we have $\vars{\Gamma} \spr (\alpha_1 \otimes
  \alpha_2)$ with \seq{\Gamma^*}{\alpha_1}{A_1^*} and
  \seq{\Gamma^*}{\alpha_2}{A_1^*}.  By Lemma~\ref{lem:affine-mode-4},
  this means there is a $\Gamma \ge \Gamma'$ with $\vars{\Gamma'} \deq
  (\alpha_1 \otimes \alpha_2)$.  By Lemma~\ref{lem:affine-mode-5}, we
  have $\Gamma' \splits \Gamma_1,\Gamma_2$ with $\vars{\Gamma_1} \deq
  \alpha_1$ and $\vars{\Gamma_2} \deq \alpha_2$. So the premises are
  \seq{\Gamma^*}{\vars{\Gamma_1}}{A_1^*} and
  \seq{\Gamma^*}{\vars{\Gamma_2}}{A_2^*}.  By
  Lemma~\ref{lem:strengthening-affine}, we can modify the premises to
  no-bigger derivations of \seq{\Gamma_1^*}{\vars{\Gamma_1}}{A_1^*} and
  \seq{\Gamma_2^*}{\vars{\Gamma_2}}{A_2^*}.  Thus, by the inductive
  hypotheses we get $\seqa{\Gamma_1}{A_1}$ and $\seqa{\Gamma_2}{A_2}$,
  so $\seqa{\Gamma'}{A_1 \otimes A_2}$.  Then weakening and exchange on
  $\Gamma \ge \Gamma'$ gives the result.

\item 
  In the case where \FL\/ was used, the formula under elimination must
  be the translation of $z:(A_1 \otimes A_2) \in \Gamma$. The premise is
  \seq{\Gamma^*-z,x:A_1^*,y:A_2^*}{\vars{\Gamma}[(x \otimes y)/z]}{C^*},
  and we want \seqa{\Gamma}{C}.  Since \vars{\Gamma} has exactly one
  occurrence of $z$, $\vars{\Gamma}[(x \otimes y)/z] \deq
  (\vars{\Gamma}-z)\otimes x \otimes y$, so by the inductive hypothesis
  we get \seqa{\Gamma-z,x:A_1,y:A_2}{C} by the inductive hypothesis, and
  can apply $\otimes$-left.

\item In the case for \UL, the assumption $f$ that is eliminated must be
  the translation of $f:(A_1 \lolli A_2) \in \Gamma$ , so the premises
  are $\vars{\Gamma} \spr \beta'[f \otimes \alpha/z]$ with
  $\seq{\Gamma^*}{\alpha}{A_1^*}$ and
  $\seq{\Gamma^*,z:A_2^*}{\beta'}{C}$. 

  By Lemma~\ref{lem:affine-mode-4}, there is a $\Gamma'$ with $\Gamma
  \ge \Gamma'$ and $\vars{\Gamma'} \deq \beta'[f \otimes \alpha/z]$.
  Since $\vars{\Gamma'}$ has no duplicates, $z$ occurs at most once in
  $\beta'$ (or else $f$ would occur more than once in the substitution).
  
  If $z$ occurs once in $\beta'$, then because all context descriptors
  are products of variables, we can commute it to the end, writing
  $\beta' \deq \beta'' \otimes z$, so $\vars{\Gamma'} \deq \beta''
  \otimes f \otimes \alpha$.  Since $f$ is in \vars{\Gamma'}, $f$ must
  be declared with some type in $\Gamma'$, and since $\Gamma \ge
  \Gamma'$ and $f:A_1 \lolli A_2 \in \Gamma$, we must have $f:A_1 \lolli
  A_2 \in \Gamma'$.  So by Lemma~\ref{lem:affine-mode-5}, we can split
  $\Gamma' \splits \Delta_1;\Delta_2;f:A_1 \lolli A_2$ where
  $\vars{\Delta_2} \deq \beta''$ and $\vars{\Delta_1} \deq \alpha$.
  Using these equalities, the premises derive
  $\seq{\Gamma^*}{\vars{\Delta_1}}{A_1^*}$ and
  $\seq{\Gamma^*,z:A_2^*}{\vars{\Delta_2} \otimes z}{C}$, so by
  Lemma~\ref{lem:strengthening-affine} we can strengthen to no bigger
  derivations of
  $\seq{\Delta_1^*}{\vars{\Delta_1}}{A_1^*}$ (removing $\Gamma-\Delta_1$) and
  $\seq{\Delta_2*,z:A_2^*}{\vars{\Delta_2} \otimes z}{C}$ (removing $\Gamma-\Delta_2$).
  Then the inductive hypotheses give \seq{\Delta_1}{A_1} and
  \seq{\Delta_2,z:A_2}{C}, so we have the premises to use $\lolli$-left
  to conclude \seq{\Gamma'}{C}.  Finally, we weaken/exchange with
  $\Gamma \ge \Gamma'$.
  Since the splitting implies that $\Gamma' \ge \Gamma_1$ and $\Gamma'
  \ge \Gamma_2$, we have $\Gamma_0,z:A_2 \ge \Gamma_1,z:A_2$ and
  $\Gamma_0 \ge \Gamma_2$.  Moreover, we have $\beta' \deq \vec{y}
  \otimes z \deq \vars{\Gamma_1} \otimes z$ and $\alpha \deq
  \vars{\Gamma_2}$, so the premises are
  $\seq{\Gamma_0^*}{\vars{\Gamma_2}}{A_1^*}$ and
  $\seq{\Gamma_0^*,z:A_2^*}{\vars{\Gamma_1,z:A_2}}{C^*}$.  Thus, the
  inductive hypotheses give \seqa{\Gamma_2}{A_1} and
  \seqa{\Gamma_1,z:A_2}{C}, which combined with the splitting gives
  \seqa{\Gamma'}{C} by $\lolli$-left.  Finally, we have $\Gamma \ge
  \Gamma'$, so we can weaken/exchange to get \seqa{\Gamma}{C}.

  If $z$ occurs 0 times in $\beta'$ (that is, we did a \UL\/ that
  introduced a 0-use variable in the continuation), then we have
  premises \seq{\Gamma^*,z:A_2^*}{\beta'}{C} and $\vars{\Gamma} \spr
  \beta'$ (the substitution cancels).  By Lemma~\ref{lem:affine-mode-4},
  we get $\Gamma \ge \Gamma'$ with $\vars{\Gamma'} \deq \beta'$.  By
  Lemma~\ref{lem:strengthening-affine}, we can remove $z$ and anything
  in $\Gamma$ but not in $\Gamma'$ to get a no-bigger derivation of
  \seq{\Gamma'^*}{\vars{\Gamma'}}{C}.  Then the inductive hypothesis on
  this premise gives \seqa{\Gamma'}{C}, and weakening/exchanging with
  $\Gamma \ge \Gamma'$ gives the result.
\end{itemize}

Inspecting this proof, we can see that the translation from a ``native''
sequent proof in affine logic to our framework and back is the identity
on cut-free derivations.  The other round-trip is not the identity,
because the framework allows two things that the native sequent calculus
does not.  First, the framework allows weakening at the non-invertible
rules, rather than pushing it to the leaves.  For example, we have
the following two derivations of $P,Q,R \vdash P \otimes R$.

\[
\infer[\FR]
      {\seq{x:P,y:Q,z:R}{x \otimes y \otimes z}{\F{x' \otimes z'}{x':P,z':R}}}
      {x \otimes y \otimes z \spr ((x \otimes y) \otimes z) &
        \infer[\dsd{v}]
              {\seq{x:A,y:B,z:C}{x \otimes y}{C}}
              {(x \otimes y) \spr x} &
        \infer[\dsd{v}]
              {\seq{x:A,y:B,z:C}{z}{C}}
              {z \spr z}
      }
\]
\[
\infer[\FR]
      {\seq{x:P,y:Q,z:R}{x \otimes y \otimes z}{\F{x' \otimes z'}{x':P,z':R}}}
      {x \otimes y \otimes z \spr (x \otimes z) &
        \infer[\dsd{v}]
              {\seq{x:A,y:B,z:C}{x}{C}}
              {x \spr x} &
        \infer[\dsd{v}]
              {\seq{x:A,y:B,z:C}{z}{C}}
              {z \spr z}
      }
\]

\noindent The second is that a derivation may perform a left rule on a
$0$-linear (in the sense of the previous section) variable, i.e. one
that does not occur in the context descriptor.  Such variables arise
because \UL\/ ``removes a variable from the context'' by marking it as
0-use, not by actually removing it.  For this mode theory (and the other
ones we consider, besides the previous section), these left rules
produce only other 0-use variables, which ultimately cannot be used, and
can be strengthened away (see Lemma~\ref{lem:0-use-strengthening}).

The equational theory of derivations (see Section~\ref{sec:equational})
handles both of these issues, so we expect that the
framework-native-framework composite of adequacy produces a derivation
that is equal in this equational theory.  

\end{proof}

%% FIXME: move to adequacy section
%% \paragraph{Example}

%% Consider an affine product (commutative monoid $(\otimes,1)$ with
%% $\dsd{w} :: x \spr 1$) and the following two derivations of $P,Q,R
%% \vdash P \otimes R$:

%% \begin{footnotesize}
%% \[
%% \infer[\FR]
%%       {\seq{x:P,y:Q,z:R}{x \otimes y \otimes z}{\F{x' \otimes z'}{x':P,z':R}}}
%%       {x \otimes y \otimes z \spr ((x \otimes y) \otimes z) &
%%         \infer[\dsd{v}]
%%               {\seq{x:A,y:B,z:C}{x \otimes y}{C}}
%%               {(x \otimes y) \spr x} &
%%         \infer[\dsd{v}]
%%               {\seq{x:A,y:B,z:C}{z}{C}}
%%               {z \spr z}
%%       }
%% \]
%% \end{footnotesize}
%% \begin{footnotesize}
%% \[
%% \infer[\FR]
%%       {\seq{x:P,y:Q,z:R}{x \otimes y \otimes z}{\F{x' \otimes z'}{x':P,z':R}}}
%%       {x \otimes y \otimes z \spr (x \otimes z) &
%%         \infer[\dsd{v}]
%%               {\seq{x:A,y:B,z:C}{x}{C}}
%%               {x \spr x} &
%%         \infer[\dsd{v}]
%%               {\seq{x:A,y:B,z:C}{z}{C}}
%%               {z \spr z}
%%       }
%% \]
%% \end{footnotesize}

%% These differ by the placement of weakening: our rules allow for
%% weakening away $y$ either at a leaf (there is a third possible
%% derivation that weakens in the $z$ leaf instead), as is typical in
%% affine logic, or at the context division in $\FR$.  However, these two
%% derivations are equal in the above equational theory:
%% \[
%% \FRd{1}{\Trd{(1_{x \otimes y}[\dsd{w}/y])}{x},\Trd{1}{z}}
%% \deq
%% \FRd{1_{x \otimes y \otimes z}[\dsd{w}/y]}{\Trd{1}{x},\Trd{1}{z}}
%% \]
%% By associativity, unit and projection laws, we have that a general
%% instance of \FR\/ is equal to an iterated cut (in any order) and then a
%% structural transformation on the ``axiomatic'' $\FR^* ::
%% \seq{\Gamma,\Delta}{\alpha}{\F{\alpha}{\Delta}}$
%% \[
%% \FRd{\alpha}{\vec{\D_i/x_i}} \deq \Trd{\alpha}{\FR^*[\vec{\D_i/x_i}]}
%% \]
%% So applying this to both sides it suffices to show
%% \[
%% \begin{array}{rcl}
%% & &  \Trd{{(1_{x \otimes y \otimes z})}}{\FR^*[\Trd{1}{z}/z'][\Trd{(1_{x \otimes y}[\dsd{w}/y])}{x}/x']}\\
%% & \deq & \Trd{{(1_{x \otimes y \otimes z}[\dsd{w}/y])}}{\FR^*[\Trd{1}{z}/z'][\Trd{1}{x}/x']} 
%% \end{array}
%% \]
%% Here we have
%% \[
%% \begin{array}{rcl}
%% \FR^*[\Trd{1}{z}/z'] & :: & \seq{x:P,y:Q,z:R,x':P}{x' \otimes z}{P \otimes R}\\
%% \Trd{1}{x} & :: & \seq{x:P,y:Q,z:R}{x}{P}\\
%% {(1_{x \otimes y}[\dsd{w}/y])} & :: & x \otimes y \spr y\\
%% 1_{x'\otimes z} & :: & x' \otimes z
%% \end{array}
%% \]
%% so the transformation-on-cut equation gives
%% \[
%% \begin{array}{rl}
%%      & \Cut{(\Trd{({1_{x'\otimes z}})}{\FR^*[\Trd{1}{z}/z']})}{\Trd{{(1_{x \otimes y}[\dsd{w}/y])}}{\Trd{1}{x}}}{x'}\\
%%  \deq & \Trd{ (\subst{({1_{x'\otimes z}})}{(1_{x \otimes y}[\dsd{w}/y])}{x'})  }{\Cut{\FR^*[\Trd{1}{z}/z']}{\Trd{1}{x}}{x'}}
%% \end{array}
%% \]
%% The left-hand side is equal to the left-hand side of the goal by
%% functoriality of \Trd{-}{-}, and the right-hand side is equal to the
%% right-hand side of the goal by associativity of horizontal composition
%% and $1_{x' \otimes z}[1_{x \otimes y/x'}] \deq 1_{x \otimes y \otimes
%%   z}$.  

%% %% \begin{array}{rcll}
%% %% & &  \FRd{1}{\Trd{(1_{x \otimes y}[\dsd{w}/y])}{x},\Trd{1}{z}} \\
%% %% & \deq & \Trd{{(1_{x \otimes y \otimes z})}}{\FR^*[\Trd{1}{z}/z'][\Trd{(1_{x \otimes y}[\dsd{w}/y])}{x}/x']}\\
%% %% & \deq & ? \\
%% %% & \deq & \Trd{{(1_{x \otimes y \otimes z}[\dsd{w}/y])}}{\FR^*[\Trd{1}{z}/z'][\Trd{1}{x}/x']} \\
%% %% & \deq & \FRd{1_{x \otimes y \otimes z}[\dsd{w}/y]}{\Trd{1}{x},\Trd{1}{z}} \\
%% %% \end{array}
%% %% \]

\subsection{$n$-use Variables}

%% FIXME: this is the main idea; could prove the mode theory properties
%% more carefully

Consider the rules and mode theory from Section~\ref{sec:ex:nlinear}.
We use the following normal form theorem for the linear logic mode
(commutative monoid) mode theory, which says that any mode morphism can
be written as a ``polynomial'' of its variables:

\begin{lemma} \label{lem:monoid-normal} 
If $x_1 : \dsd{l},\ldots,x_n : \dsd{l} \vdash \alpha : \dsd{l}$ then
there exist unique ${k_1,\ldots,k_n}$ such that $\alpha \deq x_1^{k_1}
\otimes \ldots \otimes x_n^{k_n}$.
\end{lemma}

\begin{theorem}[Logical adequacy for $n$-use functions] ~\\
$x_1:^{k_1} A_1,\ldots,x_n :^{k_n} A_n \vdash C$ iff
  \seq{x_1:A_1^*,\ldots,x_n:A_n^*}{x_1^{k_1} \otimes \ldots \otimes
    x_n^{k_n}}{C^*}
\end{theorem}

\begin{proof}
When $\Gamma$ is $x_1:^{k_1} A_1,\ldots,x_n :^{k_n} A_n$, we write
\vars{\Gamma} for ${x_1^{k_1} \otimes \ldots \otimes x_n^{k_n}}$.  

The native inference rules are derivable as follows:
\begin{itemize}
\item For the identity rule, we use the fact that \vars{0 \cdot \Gamma}
  is equal to 1 by the unit laws for the monoid:
\[
\infer{{0\cdot \Gamma + x:^1 P} \vdash {P}}
      {}
\qquad
\infer{\seq{\Gamma^*,x:P}{\vars{0 \cdot \Gamma}\otimes{x}}{P}}
      {{\vars{0 \cdot \Gamma} \otimes x} \spr x}
\]

\item 
\[
\infer{\Gamma \vdash A \to^n B}
      {{\Gamma, x :^n A} \vdash {B}}
\qquad
\infer{\seq{\Gamma^*}{\vars{\Gamma}}{\U{c.c\otimes x^n}{x:A^*}{B^*}}}
      {\seq{\Gamma^*, x:A^*}{\vars{\Gamma} \otimes x^n}{B^*}}
\]

\item Note that $\Gamma + \Delta$ is only defined on contexts that have
  the same variables and types, so $(\Gamma + \Delta)^* = \Gamma^* =
  \Delta^*$.  Additionally, $\vars{\Gamma + \Delta} \deq \vars{\Gamma}
  \otimes \vars{\Delta}$, and $\vars{n \cdot \Gamma} \deq
  \vars{\Gamma}^n$.
\[
\infer{\Gamma + f:^k A \to^n B + (nk \cdot \Delta) \vdash C}
      {\Delta \vdash A &
        {\Gamma, z :^k B} \vdash {C}}
\qquad
\infer{\seq{\Gamma^*}{\vars{\Gamma} \otimes f^k \otimes \vars{\Delta}^{nk}}{C}}
      {\begin{array}{l}
          f : \U{f.f \otimes x^n}{x : A}{B} \in \Gamma^* \\
          {\vars{\Gamma} \otimes f^k \otimes \vars{\Delta}^{nk}} \deq (\vars{\Gamma} \otimes z^k)[f \otimes (\alpha)^n/z] \\
          \seq{\Delta^* = \Gamma^*}{\vars{\Delta}}{A} \\
          \seq{\Gamma^*, z:B^*}{\vars{\Gamma} \otimes z^k}{C^*} 
       \end{array}
      }
\]
Here we use associativity and commutativity to show
$(f \otimes (\alpha)^n)^k \deq f^k \otimes (\alpha)^{nk}$.  

\end{itemize}

Conversely, suppose we have \seq{\Gamma^*}{\vars{\Gamma}}{A^*}.  
\begin{itemize}
\item Case for 
\[
\infer{\seq{\Gamma^*}{\vars{\Gamma}}{P}}
      {x:P \in {\Gamma}^* & 
       \vars{\Gamma} \spr x}
\]
Let $\Gamma$ be $x_1 :^{k_1} A_1,\ldots,x:^{k}A,\ldots,x_n:^{k_n}{A_n}$.
Since the only type that encodes to an atom is that atom, we have $A =
P$.  For the linear logic mode theory (commutative monoid), we have only
equations, so $\vars{\Gamma} \spr x$ implies $\vars{\Gamma} \deq x$,
which in turn implies that $k_i = 0$ and $k = 1$ (anything else would
encode to a monoid term with a non-zero coefficient for some variable
besides $x$, or with a non-one coefficient for $x$).  Thus 
\[
\Gamma = (x_1 :^{0} A_1,\ldots,x:^{1}P,\ldots,x_n:^{0}{A_n}) = 0 \cdot (x_1 :^{0} A_1,\ldots,x:^{0}P,\ldots,x_n:^{0}{A_n}) + x:^{1}P
\]
so the hypothesis rule applies:
\[
\infer{0 \cdot (x_1 :^{0} A_1,\ldots,x:^{0}A,\ldots,x_n:^{0}{A_n}) + x:^{1}P \vdash P}{}
\]

\item Since the only type that encodes to a \U{c.\alpha}{\Delta}{B} is
  $A \to^n B$, if the derivation was by \UR, we have
\[
\infer{\seq{\Gamma^*}{\vars{\Gamma}}{\U{c.c\otimes x^n}{x:A^*}{B^*}}}
      {\seq{\Gamma^*, x:A^*}{\vars{\Gamma} \otimes x^n}{B^*}}
\]
Noting that the context of the premise is ${(\Gamma,x:^{n}A)}^*$ and the
context descriptor of the premise is $\vars{\Gamma,x:^{n}A}$, the
inductive hypothesis gives a derivation of ${{\Gamma, x :^n A} \vdash
  {B}}$, so we can derive
\[
\infer{\Gamma \vdash A \to^n B}
      {{\Gamma, x :^n A} \vdash {B}}
\]

\item Since the only type that encodes to a \U{c.\alpha}{\Delta}{B} is
  $A \to^n B$, if the derivation was by \UL, we have
\[
\infer{\seq{\Gamma^*}{\vars{\Gamma}}{C}}
      {\begin{array}{l}
          f : \U{f.f \otimes x^n}{x : A^*}{B^*} \in \Gamma^* \\
          \vars{\Gamma} \deq \beta'[f \otimes (\alpha)^n/z] \\
          \seq{\Gamma^*}{\alpha}{A^*} \\
          \seq{\Gamma^*, z:B^*}{\beta'^*}{C^*} 
       \end{array}
      }
\]
Suppose $\Gamma = x_1 :^{k_1} A_1, \ldots, x_n :^{k_n} A_n,$.  By
Lemma~\ref{lem:monoid-normal}, we have $\alpha \deq x_1^{a_1} \otimes
\ldots \otimes x_n^{a_n}$ and $\beta' \deq x_1^{b_1} \otimes \ldots
\otimes x_n^{b_n} \otimes z^{k}$.  The fact that
\[
x_1 ^{k_1} \otimes \ldots \otimes x_n^{k_n} \deq \beta'[f \otimes (\alpha)^n/z]
\]
implies that $k_i = b_i + kn a_i$ if $x_i \neq f$ and $k_i = b_i + kn a_i
+ k$ if $x_i = f$, so $\Gamma = \Gamma' + f:^{k} (A \to^n B) + nk\Delta$.
Writing 
\[
\begin{array}{l}
\Delta  = x_1 :^{a_1} A_1,\ldots,x_n :^{a_n} A_n
\Gamma' = x_1 :^{b_1} A_1,\ldots,x_n :^{b_n} A_n
\end{array}
\]
We have $\Delta^* = \Gamma^*$ and $\Gamma'^* = \Gamma^*$ and
$\vars{\Delta} = \alpha$ and $\vars{\Gamma',z:^{k}B} = \beta'$, so the
inductive hypotheses give $\Delta \vdash A$ and $\Gamma',z:^{k}B \vdash
C$, and we can apply the rule to get
\[
\infer{\Gamma = (\Gamma' + f:^k A \to^n B + (nk \cdot \Delta)) \vdash C}
      {\Delta \vdash A &
        {\Gamma, z :^k B} \vdash {C}}
\]
\end{itemize}
\end{proof}



\subsection{Cartesian Logic}

We compare the cartesian monoid mode theory from
Section~\ref{sec:ex:relevant-cartesian} with the following rules:
\[
\begin{array}{c}
\infer{\seqc{\Gamma}{P}}{x:P \in \Gamma}
\quad
\infer{\seqc{\Gamma}{A \times B}}
      {\seqc{\Gamma}{A} &
        \seqc{\Gamma}{B}}
\quad
\infer{\seqc{\Gamma}{C}}
      {p:A\times B \in \Gamma & 
        \seqc{\Gamma,x:A,y:B}{C}}
\\\\
\infer{\seqc{\Gamma}{A \to B}}
      {\seqc{\Gamma,x:A}{B}}
\quad
\infer{\seqc{\Gamma}{C}}
      {f: A \to B \in \Gamma &
        \seqc{\Gamma}{A} &
        \seqc{\Gamma,z:B}{C}
      }
\end{array}
\]
In this case, neither round-trip will be the identity on raw derivations
(as opposed to equivalence classes), though the one starting at these
native rules will be the identity up to a (positive/left) $\eta$ law for
$\times$.  The difference is that the above rules allow contraction for
$A \times B$, whereas the framework reduces this to contraction at $A$
and $B$ separately.  We could instead compare against a native sequent
calculus that does not allow contraction for positives, but the above is
more standard.

Overall, we have
\begin{theorem}[Logical adequacy for cartesian logic]
$\seqc{\Gamma}{A}$ iff \seq{\Gamma^*}{\vars{\Gamma}}{A^*}
\end{theorem}

\begin{proof}
The proof of the forward direction is by induction on the given
derivation, using the derivations of each rule:

\[
\infer[\dsd{v}]{\seq{\Gamma^*}{P^*}}
      {x:P^* \in \Gamma^* &
        1_{(\vars{\Gamma-x})\times x}[\dsd{w}/x] :: \vars{\Gamma-x} \times x \spr x}
\]

\[
\infer[\FR]{\seq{\Gamma^*}{\vars{\Gamma}}{\F{x\times y}{x:A^*,y:B^*}}}
      {\dsd{c} :: \vars{\Gamma} \spr \vars{\Gamma} \times \vars{\Gamma} &
       \seq{\Gamma^*}{\vars{\Gamma}}{A^*} & 
       \seq{\Gamma^*}{\vars{\Gamma}}{B^*}}
\]

\[
\infer[Lemma~\ref{lem:respectspr}]
      {\seq{\Gamma^*}{\vars{\Gamma}}{C^*}}
      { %% 1_{\vars{\Gamma-p}\times p}[\dsd{c}/p] :: 
        \vars{\Gamma} \spr \vars{\Gamma} \times p &
        \infer[Cor.~\ref{cor:contraction}]
              {\seq{\Gamma^*}{\vars{\Gamma}\times p}{C^*}}
              {p : (A\times B)^* \in \Gamma^* &
                \infer[\FL]
                      {\seq{\Gamma^*,q:(A\times B)^*}{\vars{\Gamma}\times q}{C^*}}
                      {\seq{\Gamma^*,x:A^*,y:B^*}{\vars{\Gamma}\times x \times y}{C^*}}}
      }
\]

\[
\infer[\UR]
      {\seq{\Gamma^*}{\vars{\Gamma}}{\U{f.f\times x}{x:A^*}{B^*}}}
      {\seq{\Gamma^*,x:A^*}{\vars{\Gamma}\times x}{B^*}}
\]

\begin{small}
\[
\infer[\UL]
      {\seq{\Gamma^*}{\vars{\Gamma}}{C}}
      {f:(A\to B)^* \in \Gamma^* &
        \vars{\Gamma} \spr \vars{\Gamma} \times (f \times \vars{\Gamma}) &
        \seq{\Gamma^*}{\vars{\Gamma}}{A^*} &
        \seq{\Gamma^*,z:B^*}{\vars{\Gamma}\times z}{B^*}
      }
\]
\end{small}

This shows that the above sequent calculus uses structural rules in the
following places: The hypothesis rule weakens away all other variables.
The $\times$ right rule contracts the entire context.  The $\times$ left
rule uses contraction for the mode theory ($p \spr p \times p$) and
contraction-over-contraction to duplicate $p$ to $q$---if we did not
have a contraction here in the native rule, then this would just be \FL,
as the $\to$ right rule is just \UR.  The $\to$ left rule contracts
everything in $\Gamma$ for use in both the argument and the
continuation, and contracts the function an additional time for use
here.

Conversely, we show \seq{\Gamma^*}{\vars{\Gamma}}{A^*} implies
$\seqc{\Gamma}{A}$ The proof is by induction on the size of the given
derivation, to allow uses of Lemma~\ref{lem:respectspr} before applying
the inductive hypothesis.

\begin{itemize}

\item The hypothesis rule is immediate because $x:P^* \in \Gamma^*$
  implies $x:P \in \Gamma$.  

\item For a general use of \FR, the conclusion must be $(A \times B)^*$
  because this is the only type that encodes to an $\Fsymb$:
\[
\infer{\seq{\Gamma^*}{\vars{\Gamma}}{\F{x\times y}{x:A,y:B}}}
      {\vars{\Gamma} \spr \alpha \times \beta &
        \seq{\Gamma^*}{\alpha}{A} &
        \seq{\Gamma^*}{\beta}{B}}
\]
Because we have projections, we can compose $\vars{\Gamma} \spr \alpha
\times \beta \spr \alpha$ and $\vars{\Gamma} \spr \alpha \times \beta
\spr \beta$, and apply these to the premises by
Lemma~\ref{lem:respectspr} to get no-bigger derivations of
$\seq{\Gamma^*}{\vars{\Gamma}}{A}$ and $\seq{\Gamma^*}{\vars{\Gamma}}{B}$,
and then the inductive hypotheses and $\times$-right give the result.

This corresponds to treating this derivation as if it were 
\[
\infer{\seq{\Gamma^*}{\vars{\Gamma}}{\F{x\times y}{x:A,y:B}}}
      {\vars{\Gamma} \spr \vars{\Gamma} \times \vars{\Gamma} &
        \infer{\seq{\Gamma^*}{\vars{\Gamma}}{A}}
              {\vars{\Gamma} \spr \alpha &
                \seq{\Gamma^*}{\alpha}{A}} &
        \infer{\seq{\Gamma^*}{\vars{\Gamma}}{B}}
              {\vars{\Gamma} \spr \beta &
                \seq{\Gamma^*}{\beta}{B}}
      }
\]
where we contract all variables and weaken the premises with any that
did not already occur in $\alpha/\beta$.  In our equational theory on
derivations these two are indeed equal, assuming equations on
transformations giving the universal property of a cartesian product in
the mode theory.

\item For a general use of \FL, which must be on the encoding of a $p: A
  \times B \in \Gamma$, we have
\[
\infer{\seq{\Gamma^*}{\vars{\Gamma}}{C}}
      {\seq{(\Gamma-p)^*,x:A^*,y:B^*}{\vars{\Gamma}[x \times y/p]}{C}}
\]
Because $\Gamma \deq (\Gamma-p)\times p$, the inductive hypothesis gives
a derivation of \seqc{\Gamma-p,x:A,y:B}{C}.  Using the admissible
weakening for cartesian logic, we have \seqc{\Gamma,x:A,y:B}{C}, so
$\times$-left gives the result.  That is, our given derivation does not
contract $p$, so we weaken with the extra occurence of $p$.

\item For \UR, the inductive hypothesis applied to the premise gives
  exactly the premise of $\to$-right.  

\item For \UL\/ on $f:(A\to B)^*$, we have
\[
\infer{\seq{\Gamma^*}{\vars{\Gamma}}{C^*}}
      {
        {\vars{\Gamma}} \spr \beta'[(f \times \alpha)/z] &
        \seq{\Gamma^*}{\alpha}{A^*} &
        \seq{\Gamma^*,z:B^*}{\beta'}{B^*}
      }
\]
Because all context descriptors are products of variables, we can
rewrite $\beta' \deq \beta'' \times z^k$ for some $k$ and $\beta''$ not
containing $z$.  Thus, we have 
$\vars{\Gamma} \spr \beta'' \times (f \times \alpha)^k$
so using
projections we have
$\vars{\Gamma} \spr \beta''$
and 
$\vars{\Gamma} \spr \alpha$.  
Using contraction, we have 
$\vars{\Gamma} \times z \spr \beta'' \times z^k$.
Applying these to the premises with Lemma~\ref{lem:respectspr}
gives derivations of 
\seq{\Gamma^*}{\vars{\Gamma}}{A^*} 
and 
\seq{\Gamma^*,z:B^*}{\vars{\Gamma}\times z}{B^*}.
Thus, the inductive hypotheses give the premises of $\to$-left.  

Equationally, the only thing suspicious about this is projecting
\emph{one} of the $\alpha$'s from $(f \times \alpha)^k$.   However,
\vars{\Gamma} has no duplicate variables, and for this mode theory any
map $x \spr x^k$ is the $k$-fold contraction of $x$.  Therefore, all
projections are the same, and contracting-projecting-recontracting is
the same as the original contraction.  

\end{itemize}

\end{proof}

\subsection{Constructive S4 \Bx{}{}}

The native rules (writing $\seql{\Delta;\Gamma}{}{A}$ for 
$\validj{\Delta};\truej{\Gamma} \vdash \truej{A}$) are
\[
\infer{\seql{\Delta;\Gamma}{}{P}}
      {P \in \Gamma}
\quad
\infer{\seql{\Delta;\Gamma}{}{C}}
      {A \in \Delta & 
       \seql{\Delta;\Gamma,A}{}{C}
      }
\quad
\infer{\seql{\Delta;\Gamma,\Bx{}{A},\Gamma'}{}{C}}
      {\seql{\Delta,A;\Gamma,\Gamma'}{}{C}}
\quad
\infer{\seql{\Delta;\Gamma}{}{\Bx{}{A}}}
      {\seql{\Delta;\cdot}{}{A}}
\]

For the mode theory in Section~\ref{sec:example:box}, we have
\begin{theorem}[Logical adequacy for a comonad]
\[
\begin{array}{c}
 x_1:\validj{A_1},\ldots;y_1:\truej{B_1},\ldots \vdash \truej{C}\\
\text{iff}\\
\seq{x_1:\Uempty{\dsd{f}}{A_1^*},\ldots;y_1:B_1^*,\ldots}
    {\dsd{f}(x_1) \times\ldots\times \dsd{f}(x_n) \times y_1 \times \ldots \times y_n}{C^*}
\end{array}
\]
\end{theorem}

\begin{proof}
We write $\Delta^*$ for $\vec{x_i : \Uempty{\dsd{f}}{A_i^*}}$ for each
assumption in $x : A_i \in \Delta$ and $\Gamma^*$ as usual.  
We write \vars{\Delta} for $\dsd{f}(x_1) \times \ldots \dsd{f}(x_n)$ for
the variables in $\Delta$ 
and \vars{\Gamma} as usual.  

First we show that \seql{\Delta;\Gamma}{}{C} implies
\seq{\Delta^*,\Gamma^*}{\vars{\Delta}\times\vars{\Gamma}}{C^*}
by induction, using the following encodings:

For the hyp rule:
\[
\infer{\seq{\Delta^*,\Gamma^*}{\vars{\Delta}\times\vars{\Gamma}}{P}}
      { x:P \in \Gamma^* &
        {\vars{\Delta}\times\vars{\Gamma}} \spr x
      }
\]
where the transformation weakens everything else.  

For the copy rule:
\[
\infer[\UL]{\seq{\Delta^*,\Gamma^*}{\vars{\Delta}\times\vars{\Gamma}}{C^*}}
      {
        \begin{array}{l}
          x:\Uempty{\dsd{f}}{A^*} \in \Delta^* \\
          {\vars{\Delta}\times\vars{\Gamma}} \spr
          (\vars{\Delta}\times\vars{\Gamma}\times z)[\dsd{f}(x)/z] \\
        \seq{\Delta^*,\Gamma^*,z:A^*}{\vars{\Delta}\times\vars{\Gamma}\times z}{C^*}
        \end{array}
      }
\]
where the transformation contracts the \dsd{f}(x) that must be in $\vars{\Delta}$.

For \Bx{}{}-left:
\[
\infer[\FL]{\seq{\Delta^*,\Gamma^*}{\vars{\Delta}\times\vars{\Gamma}}{C^*}}
      {x:\F{\dsd{f}}{\Uempty{\dsd{f}}{A^*}} \in \Gamma^* &
        \infer[\ref{lem:exchange}]
              {\seq{\Delta^*,\Gamma^*-x,z:{\Uempty{\dsd{f}}{A^*}}}
                   {\vars{\Delta}\times\vars{\Gamma-x}\times z}{C^*}}
              {\seq{\Delta^*,z:{\Uempty{\dsd{f}}{A^*}},\Gamma^*-x}{\vars{\Delta}\times\vars{\Gamma-x}\times z}{C^*}}
      }
\]
We took the liberty of making \Bx{}{}-left remove the \Bx{}{}-assumption
(which as usual for positives is a choice), or else we could do a
contraction here to match it.  We use commutativity of $\times$ and
exchange to make the order match the native rule.  

For \Bx{}{}-right:
\begin{footnotesize}
\[
\infer[\FR]
      {\seq{\Delta^*,\Gamma^*}{\vars{\Delta}\times\vars{\Gamma}}{\F{\dsd{f}}{\Uempty{\dsd{f}}{A^*}}}}
      { {\vars{\Delta}\times\vars{\Gamma}} \spr 
        \dsd{f}(x_1\times_v \ldots) &
        \infer[\UR]
              {\seq{\Delta^*,\Gamma^*}{(x_1\times_v \ldots)}{{\Uempty{\dsd{f}}{A^*}}}}
              {\infer[\ref{lem:respectspr}]
                {\seq{\Delta^*,\Gamma^*}{\dsd{f}(x_1\times_v \ldots)}{A^*}}
                {{\dsd{f}(x_1\ldots x_n)} \spr {\dsd{f}(x_1) \times \ldots} &
                  \infer[\ref{lem:weakening}]
                        {\seq{\Delta^*,\Gamma^*}{\dsd{f}(x_1) \times \ldots }{A^*}}
                        {\seq{\Delta^*}{\vars{\Delta}}{A^*}}
                }
              }}
\]
\end{footnotesize}
We write $x_1 \ldots$ for the variables from $\Delta$.  The first
transformation weakens away $\Gamma$ and uses the monoidalness
transformation axioms for \dsd{f} to pull \dsd{f} outside the product.
After the \UR, we uses the converse $\dsd{f}(\top_v) \spr \top$ and
$\dsd{f}(x \times_v y) \spr \dsd{f}(x) \times \dsd{f}(y)$ that follow
from the intro forms for the cartesian $(\times,\top)$ and congruence of
\dsd{f}\/ on the projections.  Finally, we weaken-over-weaken the
encoding of the premise with $\Gamma^*$

Conversely, suppose we have
\seq{\Delta^*,\Gamma^*}{\vars{\Delta}\times\vars{\Gamma}}{C^*}.
\begin{itemize}
\item For the axiom rule, we know $x:P$ is in $\Gamma^*$ not $\Delta^*$
  because all $\Delta$-formulas are prefixed with a \Usymb.  

\item FIXME TODO

\end{itemize}
\end{proof}

\subsection{Non-strong Monad (\Dia{}{})}

We compare the mode theory for the non-strong \Dia{}{} (modes \dsd{t}
and \dsd{p} with an affine (semicartesian) commutative monoid
$(\otimes,1,\dsd{w} :: x \spr 1)$ on \dsd{t} and
\oftp{x:\dsd{t}}{\dsd{g}(x)}{\dsd{p}}) against the rules at the
beginning of Section~\ref{sec:example:monad}.

Recall from above that $\Dia{}{A}^* =
\Uempty{\dsd{g}}{\F{\dsd{g}}{A^*}}$ and that the correspondence is:
\begin{theorem}[Logical adequacy for $\Dia{}{}$]
\[
\begin{array}{c}
\truej{A_1}, \ldots, \truej{A_1} \vdash \truej{C}\\
\text{iff}\\
\seq{x_1:A_1^*,\ldots,x_1:A_n^*}{x_1\otimes\ldots\otimes x_n}{C^*}
\end{array}
\]
and 
\[
\begin{array}{c}
\truej{A_1}, \ldots, \truej{A_n} \vdash \possj{C}\\
\text{iff}\\
\seq{x_1:A_1^*,\ldots,x_1:A_n^*}{\dsd{g}(x_1\otimes\ldots\otimes x_n)}{\F{\dsd{g}}{C^*}}.
\end{array}
\]
\end{theorem}

\begin{proof}
We write $\Gamma^*$ as usual and \vars{\Gamma} for
$x_1\otimes\ldots\otimes x_n$.  

The three rules are represented by
\[
\infer[\FR]
      {\seq{\Gamma^*}{\dsd{g}(\vars{\Gamma})}{\F{\dsd{g}}{C^*}}}
      {\dsd{g}(\vars{\Gamma}) \spr \dsd{g}(\vars{\Gamma}) &
        \seq{\Gamma^*}{\vars{\Gamma}}{A^*}}
\]

\[
\infer[\UR]
      {\seq{\Gamma^*}{\vars{\Gamma}}{\Uempty{\dsd{g}}{\F{\dsd{g}}{C^*}}}}
      {\seq{\Gamma^*}{\dsd{g}(\vars{\Gamma})}{\F{\dsd{g}}{C^*}}}
\]

\[
\infer{\seq{\Gamma^*}{\dsd{g}(\vars{\Gamma})}{\F{\dsd{g}}{C^*}}}
      {x : \Uempty{\dsd{g}}{\F{\dsd{g}}{A^*}} \in \Gamma^* &
        \dsd{g}(\vars{\Gamma}) \spr \dsd{g}(x) &
        \infer[\FL]
              {\seq{y:\F{\dsd{g}}{A}^*}{\dsd{y}}{\F{\dsd{g}}{C^*}}}
              {\seq{z:A^*}{\dsd{g}(z)}{\F{\dsd{g}}{C^*}}}
      }
\]
(and an identity rule $\Gamma,\truej{P} \vdash \truej{P}$ would be
translated as usual).

Conversley, suppose we have a derivation of
\seq{\Gamma^*}{\vars{\Gamma}}{C^*} or
\seq{\Gamma^*}{\dsd{g}(\vars{\Gamma})}{\F{\dsd{g}}{C^*}}.
Lemma~\ref{lem:0-use-strengthening} can be used on such derivations: the
equational axioms preserve variables and \dsd{w} removes but does not
add, so we use Lemma~\ref{lem:spr-doesnt-introduce}; and \F{\dsd{g}}{A}
and \Uempty{\dsd{g}}{A} are both relevant, so by induction the encoding
of any sequent is.

Suppose we have \seq{\Gamma^*}{\vars{\Gamma}}{C^*}.  
The hypothesis rule is translated back to itself as usual.  Since there
are no types that encode to \Fsymb, any other final rule must be \UR\/ or \UL,
and the type must be \Uempty{\dsd{g}}{\F{\dsd{g}}{A^*}}.
\begin{itemize}
\item 
If we have
\[
\infer[\UR]
      {\seq{\Gamma^*}{\vars{\Gamma}}{\Uempty{\dsd{g}}{\F{\dsd{g}}{A^*}}}}
      {\seq{\Gamma^*}{\dsd{g}(\vars{\Gamma})}{\F{\dsd{g}}{A^*}}}
\]
then in the inductive hypothesis gives $\Gamma \vdash \possj{A}$, so we
have $\Gamma \vdash \truej{\Dia{}{A}}$ by rule.  

\item 
Suppose we have 
\[
\infer[\UL]
      {\seq{\Gamma^*}{\vars{\Gamma}}{C^*}}
      {x:{\Uempty{\dsd{g}}{\F{\dsd{g}}{A^*}}} \in \Gamma &
        \vars{\Gamma} \spr \beta'[\dsd{g}(x)/z] &
        \seq{\Gamma^*,z:{\F{\dsd{g}}{A^*}}}{\beta'}{C^*}}
\]
For this mode theory, 
the constants do not allow embedding a \dsd{p}-mode term in a
\dsd{t}-mode term.  Therefore, the subterm $\dsd{g}(x)$ cannot occur in
a ``reduct'' of \vars{\Gamma}, which has mode \dsd{t}.  Thus, $z$ does
not occur in $\beta'$, and $\vars{\Gamma} \spr \beta'$.  Applying
Lemma~\ref{lem:respectspr} to the premise gives a no-bigger derivation
of \seq{\Gamma^*,z:{\F{\dsd{g}}{A^*}}}{\vars{\Gamma}}{C^*}, and applying
Lemma~\ref{lem:0-use-strengthening} to strengthen away $z$ gives a
no-bigger derivation of \seq{\Gamma^*}{\vars{\Gamma}}{C^*}.
Then the inductive hypothesis gives $\Gamma \vdash \truej{C}$ as
desired.  
\end{itemize}

Suppose we have \seq{\Gamma^*}{\dsd{g}(\vars{\Gamma})}{\F{\dsd{g}}{C^*}}
and want $\Gamma \vdash \possj{C}$.  Since there are no \Fsymb's in the
context, the only possibilities are \UL\/ and \FR:
\begin{itemize}

\item Suppose we have
\[
\infer[\FR]
      {\seq{\Gamma^*}{\dsd{g}(\vars{\Gamma})}{\F{\dsd{g}}{C^*}}}
      {{\dsd{g}(\vars{\Gamma})} \spr \dsd{g}(\alpha) &
        {\seq{\Gamma^*}{\alpha}{C^*}}}
\]
For this mode theory, there are no there are no equalities or
transformations between terms of the form $\dsd{g}(\alpha)$ and any
other \dsd{p}-mode term besides congruence on $\alpha \spr \alpha'$, so
we can extract a transformation $\vars{\Gamma} \spr \alpha$ (such that
the given one is equal to congruence with \dsd{g} on it).  So we get
$\vars{\Gamma} \spr \alpha$ and can use Lemma~\ref{lem:respectspr} to
get a no-bigger derivation of {\seq{\Gamma^*}{\vars{\Gamma}}{C^*}}.  By
the inductive hyptothesis on this premise we get $\Gamma \vdash
\truej{A}$, which gives $\Gamma \vdash \possj{A}$ as desired.  
%% : reflexivity
%% goes to reflexivity, merge composites, and rewrite any congruence 
%% $1_\dsd{g}(\alpha)$ as congruence with $\dsd{g}$

\item Suppose we have 
\[
\infer[\UL]
      {\seq{\Gamma^*}{\dsd{g}(\vars{\Gamma})}{\F{\dsd{g}}{C^*}}}
      {\begin{array}{l}
          x:{\Uempty{\dsd{g}}{\F{\dsd{g}}{A^*}}} \in \Gamma \\
          \dsd{g}(\vars{\Gamma}) \spr \beta'[\dsd{g}(x)/z] \\
          \seq{\Gamma^*,z:{\F{\dsd{g}}{A^*}}}{\beta'}{\F{\dsd{g}}{C^*}}
        \end{array}
      }
\]
We have \oftp{x_i:\dsd{t},z:\dsd{p}}{\beta'}{\dsd{p}}, so by inversion
the only possibilities are $z$ or $\dsd{g}(-)$, and in the latter case
$z$ does not occur, as argued above.  

If $\beta'$ is \dsd{z}, then we have 
\seq{\Gamma^*,z:{\F{\dsd{g}}{A^*}}}{z}{\F{\dsd{g}}{C^*}}.  By
Lemma~\ref{lem:0-use-strengthening} (strengthening away $\Gamma^*$) we
have a no-bigger derivation of
\seq{z:{\F{\dsd{g}}{A^*}}}{z}{\F{\dsd{g}}{C^*}}.
By Lemma~\ref{lem:Finv}, we can left-invert to get a no-bigger
derivation of 
\seq{z':{A^*}}{\dsd{g}{(z')}}{\F{\dsd{g}}{C^*}}.  
By the inductive hypothesis, this translates to a derivation of 
$\truej{A} \vdash \possj{C}$, so 
applying \Dia{}{}-left gives the result.

If $z$ does not occur in $\beta'$ we have $\dsd{g}(\vars{\Gamma}) \spr
\beta'$ so pushing this into the premise gives by
Lemma~\ref{lem:respectspr} gives a no-bigger derivation of
\seq{\Gamma^*,z:{\F{\dsd{g}}{A^*}}}{\dsd{g}(\vars{\Gamma})}{\F{\dsd{g}}{C^*}}.
Then strengthening $z$ by Lemma~\ref{lem:0-use-strengthening} gives a
no-bigger derivation of
\seq{\Gamma^*}{\dsd{g}(\vars{\Gamma})}{\F{\dsd{g}}{C^*}}.  so the
inductive hypothesis gives the result.  That is, we did an elimination
to produce a 0-use variable, which we can strengthen away.  
\end{itemize}
\end{proof}

\subsection{Strong Monad (\Crc{}{A})}

For the linear logic (commutative monoid)  mode theory, we compare the rules
\[
\infer{\Gamma \vdash \possj{A}}
      {\Gamma \vdash \truej{A}}
\qquad
\infer{\Gamma \vdash \truej{\Crc{}{A}}}
      {\Gamma \vdash \possj{A}}
\qquad
\infer{\Gamma,\truej{\Crc{}{A}},\Gamma' \vdash \possj{C}}
      {\Gamma,\Gamma',\truej{A} \vdash \possj{C}}
\]
with the mode theory consisting of a commutative monoid $(\otimes,1)$ on
\dsd{p}, a functor \dsd{g} from \dsd{t} to \dsd{p}, and 
\[
\begin{array}{ll}
\oftp{x : \dsd{t}, y : \dsd{p}}{\ttp x y}{\dsd{p}}
& \dsd{g}(x \otimes y) \deq \ttp x {\dsd{g}(y)}\\
\end{array}
\]
(We elide the equations $\ttp {(x \otimes y)} z \deq \ttp x {(\ttp y
  z)}$ and $\ttp {\dsd{1}} z \deq z$ discussed above because they
provable when $z$ is $\dsd{g}(x)$, which are the only terms that come up
in this encoding.)
%% FIXME: do proof with them, since they seem useful for the general
%% case where we consider more connectives.  

Translating \Crc{}{A} by \Uempty{\dsd{g}}{\F{\dsd{g}}{A^*}}, we have
the same adequacy statement as above:
\begin{theorem}[Logical adequacy for $\Crc{}{}$]
\[
\begin{array}{c}
\truej{A_1}, \ldots, \truej{A_1} \vdash \truej{C}\\
\text{iff}\\
\seq{x_1:A_1^*,\ldots,x_1:A_n^*}{x_1\otimes\ldots\otimes x_n}{C^*}
\end{array}
\]
and 
\[
\begin{array}{c}
\truej{A_1}, \ldots, \truej{A_n} \vdash \possj{C}\\
\text{iff}\\
\seq{x_1:A_1^*,\ldots,x_1:A_n^*}{\dsd{g}(x_1\otimes\ldots\otimes x_n)}{\F{\dsd{g}}{C^*}}.
\end{array}
\]
\end{theorem}

\begin{proof}
When $\Gamma$ is $x_1:\truej{A_1}, \ldots, x_n:\truej{A_n}$, we write 
\vars{\Gamma} for $x_1 \otimes \ldots \otimes x_n$.  

The three rules are represented by
\[
\infer[\FR]
      {\seq{\Gamma^*}{\dsd{g}(\vars{\Gamma})}{\F{\dsd{g}}{C^*}}}
      {\dsd{g}(\vars{\Gamma}) \spr \dsd{g}(\vars{\Gamma}) &
        \seq{\Gamma^*}{\vars{\Gamma}}{A^*}}
\]

\[
\infer[\UR]
      {\seq{\Gamma^*}{\vars{\Gamma}}{\Uempty{\dsd{g}}{\F{\dsd{g}}{C^*}}}}
      {\seq{\Gamma^*}{\dsd{g}(\vars{\Gamma})}{\F{\dsd{g}}{C^*}}}
\]

\[
\infer{\seq{(\Gamma^* = \Gamma_1^*,x:\Uempty{\dsd{g}}{\F{\dsd{g}}{A^*}},\Gamma_2^*)}{\dsd{g}(\vars{\Gamma})}{\F{\dsd{g}}{C^*}}}
      {x : \Uempty{\dsd{g}}{\F{\dsd{g}}{A^*}} \in \Gamma^* &
        \dsd{g}(\vars{\Gamma}) \spr \ttp {(\vars{\Gamma_1} \otimes \vars{\Gamma_2})} {\dsd{g}(x)} &
        \infer[\FL]
              {\seq{\Gamma^*,z:\F{\dsd{g}}{A^*}}{\vars{\Gamma_1} \otimes \vars{\Gamma_2} \otimes z}{\F{\dsd{g}}{C^*}}}
              {\infer[Lemma~\ref{lem:weakening}]
                     {\seq{\Gamma^*,y:{A^*}}{(\vars{\Gamma_1} \otimes \vars{\Gamma_2} \otimes \dsd{g}(y)) \deq \dsd{g}(\vars{\Gamma_1} \otimes y \otimes \vars{\Gamma_2})}{\F{\dsd{g}}{C^*}}}
                     {\seq{\Gamma_1*,\Gamma_2^*,y:{A^*}}{\dsd{g}(\vars{\Gamma_1}\otimes\vars{\Gamma_2}\otimes y)}{\F{\dsd{g}}{C^*}}}}
      }
\]
The transformation $\dsd{g}(\vars{\Gamma}) \spr \ttp {(\vars{\Gamma_1}
  \otimes \vars{\Gamma_2})} {\dsd{g}(x)}$ is an equality given by
associating and commuting $\vars{\Gamma} = (\vars{\Gamma_1} \otimes x
\otimes \vars{\Gamma_2})$ to $(\vars{\Gamma_1} \otimes \vars{\Gamma_2})
\otimes x$ and then using 
\[
\dsd{g}((\vars{\Gamma_1} \otimes
\vars{\Gamma_2}) \otimes x) \deq \ttp {(\vars{\Gamma_1} \otimes
  \vars{\Gamma_2})} {\dsd{g}(x)}
\]

An identity rule $\truej{P} \vdash \truej{P}$ is translated as usual.

Conversely, suppose we have a derivation of
\seq{\Gamma^*}{\vars{\Gamma}}{C^*} or
\seq{\Gamma^*}{\dsd{g}(\vars{\Gamma})}{\F{\dsd{g}}{C^*}}.
Lemma~\ref{lem:0-use-strengthening} can be used on such derivations: the
equational axioms preserve variables and there are no transformations,
so we use Lemma~\ref{lem:spr-doesnt-introduce}; and \F{\dsd{g}}{A} and
\Uempty{\dsd{g}}{A} are both relevant, so by induction the encoding of
any sequent is.  We use the following properties of the mode theory:
\begin{itemize}
\item If $\alpha \spr \beta$ then $\alpha \deq \beta$, because there are
  no structural transformation axioms.

\item If $\oftp{\tptm{x_1}{\dsd{t}}, \ldots, \tptm{x_n}{\dsd{t}},
  \tptm{z}{\dsd{p}}}{\beta}{\dsd{p}}$ then $\beta$ is of the form
  $\ttp{\alpha_1}{\ttp{\alpha_2}{\ldots z}}$, or
  $\ttp{\alpha_1}{\ttp{\alpha_2}{\ldots \dsd{g}(\alpha_{n})}}$ (in which
  case $z$ does not occur).  Moreover, if $\oftp{\tptm{x_1}{\dsd{t}},
    \ldots, \tptm{x_n}{\dsd{t}}}{\beta}{\dsd{p}}$ then $\beta$ is of the
  form $\ttp{\alpha_1}{\ttp{\alpha_2}{\ldots \dsd{g}(\alpha_{n})}}$.
  This is because the only constants of mode \dsd{p} are $\ttp{}{}$,
  which has only 1 \dsd{p}-mode subposition, and \dsd{g}, which has 0,
  so any term of mode \dsd{p} must be an iterated application of
  $\ttp{}{}$ ending in either a variable (if there is one in the
  context), or \dsd{g}.

\item If $\vec{x_i : \dsd{t}} \vdash \dsd{g}(\alpha) \deq
  \dsd{g}(\beta)$ then $\alpha \deq \beta$.  We generalize and define a
  meta-operation $\oftp{\psi}{t(\alpha)}{\dsd{t}}$ when
  $\oftp{\psi}{\alpha}{\dsd{p}}$ that picks out the \dsd{t} parts of
  $\alpha$: 
  \[
  \begin{array}{l}
    t(z) = 1
    t(\ttp \alpha \beta) = \alpha \otimes t(\beta)\\
    t(\dsd{g}(\alpha)) = \alpha
  \end{array}
  \]
  It now suffices to show that $\alpha \deq \beta$ implies $t(\alpha)
  \deq t(\beta)$.  The cases for reflexivity, symmetry, and transitivity
  all follow from the inductive hypotheses and the corresponding rules.
  For the axiom $\dsd{g}(x \otimes y) \deq \ttp{x} \otimes \dsd{g}(y)$,
  we get $x \otimes y$ on both sides.  For congruence, suppose $\psi,x:q
  \vdash \alpha' \deq \beta' : \dsd{p}$ and $\psi \vdash \alpha \deq
  \beta : q$.  By the inductive hypothesis, $t(\alpha) \deq t(\beta)$.
  We distinguish cases on whether $q$ is \dsd{p} or $\dsd{t}$.  

  If the congruence variable $x$ has mode \dsd{t}, then the result
  follows from the fact that $x : \dsd{t} \vdash \alpha : \dsd{p}$ implies
  $t(\alpha[\beta/x]) \deq t(\alpha)[\beta/x]$, because we get
  $t(\alpha)[\alpha'/x] \deq t(\beta)[\beta'/x]$ by the congruence rule.
  To prove this, the case for $z$ is immediate, because both sides are
  $1$.  In the case for $\dsd{g}(\alpha)$, both sides are
  $\alpha[\beta/x]$.  In the case for $\ttp{\alpha_1}{\alpha_2}$, by the
  inductive hypothesis we get $t(\alpha_2[\beta/x]) \deq
  t(\alpha_2)[\beta/x]$ by the inductive hypothesis, and we need to show
  that $\alpha_1[\beta/x] \otimes t(\alpha_2[\beta/x]) \deq (\alpha_1
  \otimes t(\alpha_2))[\beta/x]$, which by definition of substitution.
  
  If the congruence variable $x$ has mode \dsd{p}, then because all
  equational axioms have the same variables on the left and right, $x$
  occurs either on both sides or on neither.  If it occurs on neither,
  then the inductive hypothesis $t(\alpha) \deq t(\beta)$ is enough,
  because $t(\alpha[\beta/x]) = t(\alpha) \deq t(\beta) =
  t(\alpha'[\beta'/x])$.  If it occurs on both, then the result follows
  from the fact that $t(\alpha[\beta/x]) \deq t(\alpha) \otimes
  t(\beta)$, because $t(\alpha) \deq t(\beta)$ and $t(\alpha') \deq
  t(\beta')$ by the inductive hypotheses on the two subderivations, both
  of which have mode \dsd{p}.  We prove the lemma that $x : \dsd{p}
  \vdash \alpha : \dsd{p}$ (where $x$ occurs in $\alpha$) and $\beta :
  \dsd{p}$ imply $t(\alpha[\beta/x]) \deq t(\alpha) \otimes t(\beta)$ by
  induction on $\alpha$.  If it is $x$, then we have $t(\beta) \deq 1
  \otimes t(\beta)$.  It cannot be some other $y$ or $\dsd{g}(\alpha)$
  because then $x$ has no place to occur.  If it is
  $\ttp{\alpha_1}{\alpha_2}$, then by the inductive hypothesis,
  $t(\alpha_2[\beta/x]) \deq t(\alpha_2) \otimes t(\beta)$, and we have
  to show that $\alpha_1[\beta/x] \otimes t(\alpha_2[\beta/x]) \deq
  \alpha_1 \otimes t(\alpha_2) \otimes t(\beta)$.  This follows because
  $x$, which has mode \dsd{p}, cannot occur in $\alpha_1$, which has
  mode \dsd{t}.
\end{itemize}

Suppose we have \seq{\Gamma^*}{\vars{\Gamma}}{C^*}.  
The hypothesis rule is translated back to itself as usual.  Since there
are no types that encode to \Fsymb, any other final rule must be \UR\/ or \UL,
and the type must be \Uempty{\dsd{g}}{\F{\dsd{g}}{A^*}}.
\begin{itemize}
\item 
If we have
\[
\infer[\UR]
      {\seq{\Gamma^*}{\vars{\Gamma}}{\Uempty{\dsd{g}}{\F{\dsd{g}}{A^*}}}}
      {\seq{\Gamma^*}{\dsd{g}(\vars{\Gamma})}{\F{\dsd{g}}{A^*}}}
\]
then in the inductive hypothesis gives $\Gamma \vdash \possj{A}$, so we
have $\Gamma \vdash \truej{\Crc{}{A}}$ by rule.

\item 
Suppose we have 
\[
\infer[\UL]
      {\seq{\Gamma^*}{\vars{\Gamma}}{C^*}}
      {x:{\Uempty{\dsd{g}}{\F{\dsd{g}}{A^*}}} \in \Gamma &
        \vars{\Gamma} \spr \beta'[\dsd{g}(x)/z] &
        \seq{\Gamma^*,z:{\F{\dsd{g}}{A^*}}}{\beta'}{C^*}}
\]
For this mode theory, there are no transformations besides identities,
so $\vars{\Gamma} \deq \beta'[\dsd{g}(x)/z]$.  The constants do not
allow embedding a \dsd{p}-mode term in a \dsd{t}-mode term.  Therefore,
the subterm $\dsd{g}(x)$ cannot occur in a anything equal to
\vars{\Gamma}, which has mode \dsd{t}.  Thus, $z$ does not occur in
$\beta'$, and $\vars{\Gamma} \deq \beta'$.  Applying
Lemma~\ref{lem:0-use-strengthening} to strengthen away $z$ gives a
no-bigger derivation of \seq{\Gamma^*}{\vars{\Gamma}}{C^*}.  Then the
inductive hypothesis gives $\Gamma \vdash \truej{C}$ as desired.
\end{itemize}

Suppose we have \seq{\Gamma^*}{\dsd{g}(\vars{\Gamma})}{\F{\dsd{g}}{C^*}}
and want $\Gamma \vdash \possj{C}$.  Since there are no \Fsymb's in the
context, the only possibilities are \UL\/ and \FR:
\begin{itemize}

\item Suppose we have
\[
\infer[\FR]
      {\seq{\Gamma^*}{\dsd{g}(\vars{\Gamma})}{\F{\dsd{g}}{C^*}}}
      {{\dsd{g}(\vars{\Gamma})} \spr \dsd{g}(\alpha) &
        {\seq{\Gamma^*}{\alpha}{C^*}}}
\]

We have ${\dsd{g}(\vars{\Gamma})} \deq \dsd{g}(\alpha)$, which 
implies $\vars{\Gamma} \deq \alpha$.  By the inductive hyptothesis on
the premise we get $\Gamma \vdash \truej{A}$, which gives $\Gamma \vdash
\possj{A}$ as desired.

\item Suppose we have 
\[
\infer[\UL]
      {\seq{\Gamma^*}{\dsd{g}(\vars{\Gamma})}{\F{\dsd{g}}{C^*}}}
      {\begin{array}{l}
          x:{\Uempty{\dsd{g}}{\F{\dsd{g}}{A^*}}} \in \Gamma \\
          \dsd{g}(\vars{\Gamma}) \deq \beta'[\dsd{g}(x)/z] \\
          \seq{\Gamma^*,z:{\F{\dsd{g}}{A^*}}}{\beta'}{\F{\dsd{g}}{C^*}}
        \end{array}
      }
\]
We have \oftp{x_1:\dsd{t},\ldots,x_n:\dsd{t},z:\dsd{p}}{\beta'}{\dsd{p}}, so 
$\beta'$ is either $\ttp {\alpha_1}{\ttp \ldots z}$ 
or $\ttp {\alpha_1}{\ttp \ldots \dsd{g}{(\alpha)}}$, 
and in this case \dsd{z} does not occur.  

If $\beta'$ is $\ttp {\alpha_1}{\ttp \ldots z}$, then we have
\seq{\Gamma^*,z:{\F{\dsd{g}}{A^*}}}{\ttp {\alpha_1}{\ttp \ldots \ttp
    {\alpha_n} z}}{\F{\dsd{g}}{C^*}} and $\dsd{g}(\vars{\Gamma}) \deq
\ttp {\alpha_1}{\ttp \ldots \dsd{g}(x)}$.  This entails ${\vars{\Gamma}}
\deq \alpha_1 \otimes \ldots \otimes \alpha_n \otimes x$, and because
the only equations are the commutative monoid laws, $\vars{\Gamma-x}
\deq \alpha_1 \otimes \ldots \otimes \alpha_n$.  By
Lemma~\ref{lem:Finv}, we have a no-bigger derivation of
\seq{\Gamma^*,y:A^*}{\ttp {\alpha_1}{\ttp \ldots
    \dsd{g}(y)}}{\F{\dsd{g}}{C^*}}.  The subscript ${\ttp
  {\alpha_1}{\ttp \ldots \dsd{g}(y)}}$ is equal to $\dsd{g}(\alpha_1
\otimes \ldots \otimes \alpha_n \otimes y)$ and therefore
$\dsd{g}(\vars{\Gamma-x} \otimes y)$, so we have 
\seq{\Gamma^*,y:A^*}{\vars {\Gamma-x,y:A}}{\F{\dsd{g}}{C^*}}.
By Lemma~\ref{lem:0-use-strengthening}, we can strengthen away $x$,
giving a no-bigger derivation of 
\seq{(\Gamma^*-x),y:A^*}{\vars {\Gamma-x,y:A}}{\F{\dsd{g}}{C^*}}.
By the inductive hypothesis, this translates to a
derivation of $\Gamma-x,\truej{A} \vdash \possj{C}$, so applying
\Crc{}{}-left gives the result.

If $z$ does not occur in $\beta'$ we have $\dsd{g}(\vars{\Gamma}) \deq
\beta'$, so the premise is
\seq{\Gamma^*,z:{\F{\dsd{g}}{A^*}}}{\dsd{g}(\vars{\Gamma})}{\F{\dsd{g}}{C^*}}.
Then strengthening $z$ by Lemma~\ref{lem:0-use-strengthening} gives a
no-bigger derivation of
\seq{\Gamma^*}{\dsd{g}(\vars{\Gamma})}{\F{\dsd{g}}{C^*}}.  So the
inductive hypothesis gives the result.  That is, we did an elimination
to produce a 0-use variable, which we can strengthen away.
\end{itemize}
\end{proof}



%% \section{Equational Adequacy}
\label{sec:adequacy-equational}

\subsection{Template}

In addition to the logical adequacy results above, we expect that the
translation from an object logic into the framework extends to something
like a full and faithful functor from the object logic to the framework.
Unpacking this, the object part of the functor means we want a
translation $A^*$ from object language types to framework types---and an
extension translating object-language sequents $J$ to framework sequents
$J^*$.  The morphism part of the functor maps each object-logic
derivation $d : J$ to a derivation $d^* : J^*$.  Functoriality means
that the translation takes identities to identities and cuts to cuts.
Together, full and faithfullness say that for each sequent $J$, the
object language derivations of $J$ are bijective with framework
derivations of $J^*$.  In particular, fullness says that for any sequent
$J$, the translation on derivations of that sequent is surjective: for
every derivation $e$ of $J^*$, there (merely) exists an object language
derivation $d : J$ such that $d^* = e$.  In terms of provability, this
says that no more sequents can be proved in the framework, and in terms
of proof identity, it says that every derivation could have been written
in the object language. Faithfullness says that the translation on
derivations is injective---$d_1^* = d_2^*$ implies $d_1 = d_2$---so no
more equalities can be proved in the framework.  The fact that a
function is a bijection iff it is surjective and injective gives the
overall result.

In the above discussion, we would like equality of derivations to
correspond to the categorical universal properties for the connectives,
which generally equate more morphisms than syntactic equality of
cut-free proofs (unless one uses more sophisticated sequent calculi than
we consider here, e.g. focusing/multifocusing).  On the framework side,
the equational theory of Section~\ref{sec:equational} already accounts
for this.  On the source side, will define a logic by the usual sequent
calculus rules that make cut and identity admissible, along with
primitive cut and identity rules, and an equality judgement analogous to
Section~\ref{sec:equational}, which is a concise description of
$\beta\eta$ rules.  Cut elimination for the source will be a corollary
of the adequacy theorem (we could simplify the source syntax by removing
the built-in cuts in the non-invertible rules, using the general cut
rule in their place, but including them is convenient for stating the
cut elimination corollary).  Thus, we refine the discussion above by
taking equality of derivations to be \deq-classes.

\newcommand\backtrf[1]{\ensuremath{#1^{\leftarrow}}}
\newcommand\backtr[1]{\ensuremath{#1^{\Leftarrow}}}
\newcommand\str[2]{\ensuremath{\dsd{str}_{#1}(#2)}}

We will generally focus on the following aspects of constructing such a
full and faithful functor:
\begin{definition}[The interesting part of an adequacy proof] ~
\begin{enumerate}
\item The translation from types to types ($A^*$) and sequents to
  sequents ($J^*$).

\item For each source inference rule for each connective, a
  \emph{derivation} $d^*$ from the translated premises to the translated
  conclusion (not just an admissibility: each rule will be defined by a
  composition of framework inference rules).

\item A proof that equality axioms are preserved: for each
  connective-specific equality axiom (typically $\beta\eta$) $d_1 \deq
  d_2$, $d_1^* \deq d_2^*$.

\item A function \backtrf{-} from normal derivations $e : J^*$ to source
  derivations of $J$.  If the output does not use the cut rule, or
  identity at non-base-types, this cut and identity elimination for the
  source as a corollary.

\item A proof that ${\backtrf{e}}^* \deq e$.  

\item A proof that when $\Identa{x} : J^*$, $\backtrf{\Identa{x}} \deq
  x$.

\item A proof that for normal $e$ and $e'$ in the image of ``cutable''
  sequents $J^*$ and $J'^*$, $\backtrf{(\Cuta{e}{e'}{x})} \deq
  \backtrf{e}[\backtrf{e'}/x]$ (Note: this can be stated for $d^*$ and
  $d'^*$ if that is more convenient).

\item A proof that $\backtrf{(\elim{d^*})} \deq d$.  The cases for
  identity and cut will use the previous two bullets.

\item A proof that for normal $e,e' : J^*$, $e \deqp e$ imples
  $\backtrf{e} \deq \backtrf{e'}$.  (This can be stated for
  $\elim{d^*}$ if that is convenient.)
\end{enumerate}
\end{definition}

From this, the full construction is as follows:
\begin{remark}[The routine part of an adequacy proof] ~
\begin{enumerate}
\item For the construction of the functor:
\begin{enumerate}
\item The translation of types and sequents was given in part 1 above.

\item The cases of the translation of derivations $d^*$ given above are
  extended by sending identity to identity and cut to cut (possibly with
  some weakening-over-weakening and exchange-over-exchange), to
  determine a function from cutfull source derivations to cutfull
  framework derivations.  So functoriality is true by definition.  

\item We extend the above function $d^*$ on derivations to
  \deq-equivalence classes by proving $d_1 \deq d_2$ implies $d_1^* \deq
  d_2^*$.  The type-specific cases are given by part 3 above.
  Reflexivity, symmetry, and transitivity are sent to reflexivity,
  symmetry, transitivity rules in the framework. The congruence rule for
  each source derivation constructor is sent to a composition of
  framework congruence rules, which works because because each inference
  rule is shown derivable (not just admissible) in part 2 above. The
  unit and associativity laws for cut will be modeled by the
  corresponding laws in the framework.  
  %% application of a directed HIT?!
\end{enumerate}

\item For fullness, every $e$ is equal (by
  Theorem~\ref{thm:permutative-soundess}) to a
  cut/identity/transformation-free derivation $\elim{e}$, and the proof
  for cut-free derivations is given by \backtrf{-} (parts 4 and 5
  above).  Even though we are constructing a bijection between
  derivations modulo \deq, we do not need to show that this function
  respects the quotient: because of the ``mere
  existence''/$-1$-truncation in the definition of surjective, the
  function on representives automatically extends to the quotient.
  %% Even constructively, this suffices to define a (untruncated) bijection
  %% betweeb quotiented derivations---the fact that the framework-to-source
  %% direction respects $\deq$ follows from injectivity.

  If $\backtrf{-}$ does not use the cut rule in the source (or identity
  at non-base-type), then the composite $\backtrf{\elim{d^*}}$ 
  witnesses cut/identity elimination for the source.  

\item For faithfulness, we need to show that $d_1^* \deq d_2^*$ implies
  $d_1 \deq d_2$.  By part 8 above, it suffices to show 
  $\backtrf{\elim{d_1^*}} \deq \backtrf{\elim{d_2^*}}$.
  By completeness of permuatitive equality
  (Theorem~\ref{thm:permutative-completeness}), 
  $d_1^* \deq d_2^*$ implies $\elim{d_1^*} \deqp \elim{d_2^*}$,
  so part 9 above gives the result.
\end{enumerate}
\end{remark}

We do not abstract this ``template'' as a lemma because the class of
``native sequent calculi'' taken as input is not precisely
defined.  

\begin{lemma}[Equational 0-use Strengthing] \label{lem:0-use-strengthening-eq}
Under the conditions of Lemma~\ref{lem:0-use-strengthening}, the input
derivation $\D$ is $\deq$ the strengthened derivation $\D'$.
\end{lemma}

%% We say that a formula $\F{\alpha}{\Delta}$ and \U{c.\alpha}{\Delta}{A}
%% is relevant if every variable from $\Delta$ (and $c$ for \Usymb) occurs
%% at least once in $\alpha$.

%% Suppose the mode theory has the property that for all $x$, $\alpha$,
%% $\beta$, if $\alpha \spr \beta$ and $x \# \alpha$ then $x \# \beta$ (in
%% particular, equations must have the same variables on both sides).
%% Suppose additionally a sequent \seq{\Gamma}{\alpha}{A} such that every
%% \Fsymb/\Usymb\/ subformula of $\Gamma,A$ is relevant.

%% Then if $\D :: \seq{\Gamma}{\alpha}{A}$ and $\vec{x}$ are variables such
%% that $\vec{x} \# \alpha$ then there is a $\D' ::
%% \seq{\Gamma-\vec{x}}{\alpha}{A}$ and $size(\D') \le size(\D)$ 
%% $\D' \deq \D$ (when $\D'$ is weakened to reintroduce $\vec{x}$).  

\subsection{Ordered Logic (Product Only)}

As a first example of an adequacy proof, we consider the following mode
theory for ordered logic with only $A \odot B$:
\[
\infer{\seql{A}{o}{A}}{}
\quad
\infer{\seql{\Gamma,\Delta,\Gamma'}{o}{C}}
      {\seql{\Gamma,A,\Gamma'}{o}{C} &
        \seql{\Delta}{o}{A}}
\quad
\infer{\seql{\Gamma,A \odot B,\Gamma'}{o}{C}}
      {\seql{\Gamma,A,B,\Gamma'}{o}{C}}
\quad
\infer{\seql{\Gamma,\Delta}{o}{A \odot B}}
      {\seql{\Gamma}{o}{A} &
        \seql{\Delta}{o}{B}}
\]
\[
\begin{array}{c}
\Cut{\dotLd{z}{x,y.d}}{\dotRd{d_1}{d_2}}{z} \deq \Cut{\Cut{d}{d_1}{x}}{d_2}{y}\\
d : \seql{\Gamma,z:A \odot B,\Gamma'}{o}{C} \deq \dotLd{z}{x,y.\Cut{d}{\dotRd{x}{y}}{z}}\\
\end{array}
\]

We use a mode theory with a monoid $(\odot,1)$, so the only
transformation axioms are equality axioms for associativity and unit.  

The interesting parts of the adequacy proof are:
\begin{enumerate}
\item The type translation is given by $P^* := P$ and $(A \odot B)^* :=
  \F{x \odot y}{x:A^*,y:B^*}$.  A context $(x_1:A_1,\ldots,x_n:A_n)^* :=
  x_1:A_1^*,\ldots,x_n:A_n^*$.  Writing $\vars{x_1:A_1,\ldots,x_n:A_n}
  := x_1 \odot \ldots \odot x_n$, a sequent $\seql{\Gamma}{o}{A}$ is
  translated to \seq{\Gamma^*}{\vars{\Gamma}}{A^*}.

We use the following properties of the mode theory:
\begin{itemize}
\item If ${\vars{\Gamma^*}} \deq {x}$ then $\Gamma$ is $x:Q$ for some
  $Q$.  
\item If $\vars{\Gamma} \deq \alpha_1 \odot \alpha_2$, then there exist
  $\Gamma_1,\Gamma_2$ such that $\Gamma = \Gamma_1,\Gamma_2$ and
  $\vars{\Gamma_1} \deq \alpha_1$ and $\vars{\Gamma_2} \deq \alpha_2$.
\item $A^*$ and $\Gamma^*$ are relevant propositions, and the monoid
  axioms preserve variables, so by Lemma~\ref{lem:0-use-strengthening} we can
  strengthen away any variables that are not in the context descriptor.  
\end{itemize}

\item As discussed in Section~\ref{sec:adequacy:ordered-logical}, the
  inference rules for $\odot$ are derived as follows:

\[
\infer[\FL]{\seq{\Gamma^*,z:\F{x \odot y}{x:A^*,y:B^*},{\Gamma'}^*}{\vars{\Gamma}\odot z \odot \vars{\Gamma'}}{C}}
      {\infer[Lemma~\ref{lem:exchange}]
        {\seq{\Gamma^*,{\Gamma'}^*,x:A,y:B}{\vars{\Gamma}\odot x \odot y \odot \vars{\Gamma'}}{C}}
        {\seq{\Gamma^*,x:A,y:B,{\Gamma'}^*}{\vars{\Gamma}\odot x \odot y \odot \vars{\Gamma'}}{C}}}
\]

\[
\infer{\seq{\Gamma^*,\Delta^*}{\vars{\Gamma} \odot \vars{\Delta}}{\F{x \odot y}{x:A,y:B}}}
      {{\vars{\Gamma} \odot \vars{\Delta}} \spr (x \odot y)[\vars{\Gamma}/x,\vars{\Delta}/y]
        \infer[Lemma~\ref{lem:weakening}]
              {\seql{\Gamma^*,\Delta^*}{\vars{\Gamma}}{A}}
              {\seql{\Gamma^*}{\vars{\Gamma}}{A}} &
        \infer[Lemma~\ref{lem:weakening}]
              {\seql{\Gamma^*,\Delta^*}{\vars{\Gamma}}{A}}
              {\seql{\Delta^*}{\vars{\Delta}}{B}}}
\]

Identity and cut are

\[
\infer[Thm~\ref{thm:identity}]
      {\seq{x:A^*}{x}{A}}
      {}
\qquad
\infer[Thm~\ref{thm:cut}]
      {\seq{{\Gamma}^*,{\Delta}^*,{\Gamma'}^*}{\vars{\Gamma}\odot \vars{\Delta} \odot \vars{\Gamma'}}{C}}
      {\infer[Lem~\ref{lem:weakening}]
        {\seq{\Gamma^*,\Delta^*,x:A^*,{\Gamma'}^*}{\vars{\Gamma}\odot x \odot \vars{\Gamma'}}{C}}
        {\seq{\Gamma^*,x:A^*,{\Gamma'}^*}{\vars{\Gamma}\odot x \odot \vars{\Gamma'}}{C}} &
        \infer[Lem~\ref{lem:weakening}]{\seq{\Gamma^*,\Delta^*,x:A^*,{\Gamma'}^*}{\vars{\Delta}}{A^*}}
             {{\seq{\Delta^*}{\vars{\Delta}}{A^*}}}}
\]

Since we do not notate weakening and exchange, we can summarize these
as:
\[
\begin{array}{rcl}
(\dotLd{z}{x,y.d})^* & := & \FLd{z}{x,y.d^*}\\
(\dotRd{d_1}{d_2})^* & := & \FRd{}{1}{(d_1^*/x,d_2^*/y)}\\
x^* & := & x\\
(\Cut{e}{d}{x})^* & := & \Cut{e^*}{d^*}{x}
\end{array}
\]

\item The $\beta\eta$ axioms for $\odot$ translate almost exactly to the
  corresponding axioms for \F{x \odot y}{x:A^*,y:B^*}: for $\beta$, we
  also use the fact that \Trd{1}{-} is the identity.  

\item For the back-translation on normal derivations, suppose we have a
  normal derivation of \seq{\Gamma^*}{\vars{\Gamma}}{A^*}.  Because
  there are no $\Usymb$-formulas in the context, the only possible rules
  are hypothesis and the \Fsymb-rules.

\begin{itemize}
\item For identity
\[
\infer{\seq{\Gamma^*}{\vars{\Gamma}}{P}}
      {{\vars{\Gamma^*}} \spr {x} &
        x : P \in \Gamma^*}
\]
Because the only structural transformation axioms are equalities for
associativity and unit, we have ${\vars{\Gamma^*}} \deq {x}$, which in
turn implies that $\Gamma$ is $x:Q$ for some $Q$ (because if $\Gamma$ is
empty, does not contain $x$, or contains anything else, \vars{\Gamma}
will not equal $x$).  By definition, this implies $Q = P$, so $\Gamma$
is $x:P$.  Therefore the identity rule applies.

\item For \FR, because the only type that encodes to \Fsymb is $\odot$,
  we have
\[
\infer{\seq{\Gamma^*}{\vars{\Gamma}}{\F{x \otimes y}{x:A_1^*,y:A_2^*}}}
      {\vars{\Gamma} \deq \alpha_1 \odot \alpha_2 &
       \seq{\Gamma^*}{\alpha_1}{A_1^*} &
       \seq{\Gamma^*}{\alpha_2}{A_2^*}
      }
\]
By properties of the mode theory, $\Gamma = \Gamma_1,\Gamma_2$ with
$\vars{\Gamma_i} \deq \alpha_i$, so we have derivations of
\seq{\Gamma^*}{\vars{\Gamma_i}}{A_i^*}.  Because 0-use strengthening
applies, we can strengthen these to
\seq{\Gamma_i^*}{\vars{\Gamma_i}}{A_i^*}.  Then the inductive hypothesis
gives \seql{\Gamma_i}{A_i}, so applying the $\odot$ right rule gives the
result.

\item For \FL, because the only type encoding to $\Fsymb$ is $A \odot
  B$, we have
\[
\infer{\seq{\Gamma^*,z:\F{x \odot y}{x:A^*,y:B^*},{\Gamma'}^*}{\vars{\Gamma} \otimes z \otimes \vars{\Gamma'}}{C^*}}
      {\seq{\Gamma^*,{\Gamma'}^*,x:A^*,y:B^*}{\vars{\Gamma} \otimes (x \otimes y) \otimes \vars{\Gamma'}}{C^*}}
\]
By exchange (Lemma~\ref{lem:exchange}), we have a no-bigger derivation
of
{\seq{\Gamma^*,x:A^*,y:B^*,{\Gamma'}^*}{\vars{\Gamma} \otimes (x \otimes y) \otimes \vars{\Gamma'}}{C^*}} 
so applying the IH gives 
\seql{{\Gamma,x:A,y:B,{\Gamma'}}}{o}{C}, and then $\odot$-left gives the result.

\end{itemize}

That is,
\[
\begin{array}{rcl}
\backtrf{\Trd{1}{x}} & := & x\\
\backtrf{\FRd{}{1}{e_1/x,e_2/y}} & := & \dotRd{\backtrf{\str{}{e_1}}}{\backtrf{\str{}{e_2}}}\\
\backtrf{\FLd{z}{x,y.e}} & := & \dotLd{z}{x,y.\backtrf{e}}\\
\end{array}
\]
where $\str{x}{e_i}$ is the result of Lemma~\ref{lem:0-use-strengthening}.  

\item Next, we show that for normal $e :
  \seq{\Gamma^*}{\vars{\Gamma}}{A^*}$, ${\backtrf{e}}^* \deq e$.

In the case for the hypothesis rule for atoms, we have
\[
{\backtrf{\Trd{1}{x}}}^* = x^* = x \deq \Trd{1}{x}
\]

In the case for \FL, we have 
\[
(\backtrf{\FLd{z}{x,y.e}})^* = (\dotLd{z}{x,y.\backtrf{e}})^* =
{\FLd{z}{x,y.{\backtrf{e}}^*}}
\]
so the result follows from the inductive hypothesis.  

In the case for \FR, we have
\[
(\backtrf{\FRd{}{1}{e_1/x,e_2/y}})^* = (\dotRd{\backtrf{\str{}{e_1}}}{\backtrf{\str{}{e_2}}})^* =
{\FRd{}{1}{({\backtrf{{\str{}{e_1}}}}^*/x,{\backtrf{\str{}{e_2}}}^*/y)}}
\]
By the inductive hypothesis, we have 
${\backtrf{\str{}{e_i}}}^* \deq \str{}{e_i}$, but we have $\str{}{e_i} \deq e_i$ by
Lemma~\ref{lem:0-use-strengthening}.  

\item If $\Identa{x} : \seq{\Gamma^*}{x}{A^*}$, then
  $\backtrf{\Identa{x}} \deq x$.

\begin{itemize}
\item Case for $A = P$: We have $\backtrf{(\Trd{1}{x})} = x$ as required.

\item Case for $A = \F{x_1 \otimes x_2}{x_1:A_1^*,x_2:A_2^*}$.  By definition,
$\Identa{x}$ is 
$\FLd{x}{x_1,x_2.\FRd{}{1}{\Identa{x_1}/x_1,\Identa{x_2}/x_2}}$, 
so 
\backtrf{\Identa{x}} is 
\[
\dotLd{x}{x_1,x_2.\dotRd{\backtrf{(\str{x_2}{\Identa{x_1}})}/x_1}{\backtrf{(\str{x_1}{\Identa{x_2}})}/x_2}}
\]
Since $x_2$ doesn't occur in \Identa{x_1}, \str{x_2}{\Identa{x_1}}
is literally the same term as {\Identa{x_1}} (interpreted in a
bigger context), without rewriting by any definitional equalities.  
%% FIXME: state lemma 
Therefore by the inductive hypothesis for $A_1^*$ and $A_2^*$ gives
\[
\dotLd{x}{x_1,x_2.\dotRd{{x_1}/x_1}{{x_2}/x_2}}
\]
which is equal to $x$ by the $\eta$ law for $A \odot B$.

\end{itemize}

\item cut

\item Next, we show that $\backtrf{{\elim{d^*}}} \deq d$.  

The proof is by induction on $d$.  

\begin{itemize}
\item In the case for $\dotLd{}{}$, expanding definitions, we have
\[
\backtrf{\elim{(\dotLd{z}{x,y.d})^*}} \deq
\backtrf{\elim{\FLd{z}{x,y.d^*}}} \deq 
\backtrf{\FLd{z}{x,y.\elim{d^*}}} \deq
{\dotLd{z}{x,y.\backtrf{\elim{d^*}}}}
\]
so the inductive hypothesis gives the result.  

\item In the case for $\dotRd{}{}$, we have
\[
\begin{array}{ll}
& \backtrf{\elim{(\dotRd{d_1}{d_2})^*}} \\
\deq & \backtrf{\elim{\FRd{}{1}{(d_1^*/x,d_2^*/y)}}} \\
\deq & \backtrf{\FRd{}{1}{(\elim{d_1^*}/x,\elim{d_2^*}/y)}} \\
\deq & \dotRd{\backtrf{\str{}{\elim{d_1^*}}}/x}{\backtrf{\str{}{\elim{d_2^*}}}/y}
\end{array}
\]
Note that the forward translation weakens $d_i^*$ when it constructs 
${\FRd{}{1}{(d_1^*/x,d_2^*/y)}}$, and cut elimination does not introduce
left rules on variables that are not case-analyzed somewhere in the
proof.  
%% FIXME: state lemma
So in this case the 0-use strengthening $\str{}{\elim{d_1}^*}$ will
simply undo the weakening done in constructing the term, and
\backtrf{\str{}{\elim{d_1^*}}} equals \backtrf{{\elim{d_1^*}}}.  (Note
that it doesn't type check to say $\backtrf{\str{}{e}} = e$ in general,
because $e$ is not always in the image of the translation when
$\str{}{e}$ is, but in this case it does.)  Therefore the inductive
hypotheses give the result.  

\item In the case for a variable, we have
\[
\backtrf{\elim{x^*}} \deq \backtrf{\elim{x}} \deq \backtrf{\Identa{x}}
\]
so the above part gives the result.  

\item In the case for cut, we have
\[
\backtrf{\elim{(\Cut{e}{d}{x})^*}} \deq \backtrf{\elim{\Cut{e^*}{d^*}{x}}}
\deq \backtrf{(\Cuta{\elim{e^*}}{\elim{d^*}}{x})}
\]
By the above part, this is 
\Cut{\backtrf{\elim{e^*}}}{\backtrf{\elim{d^*}}}{x}, 
so the inductive hypothesis gives the result.  

\end{itemize}

\item Permutative

\end{enumerate}

\section{Related and Future Work}

We have described a sequent calculus that can express a variety of
substructural and modal logics through a suitable choice of mode theory.
Our framework builds on many approaches to substructural and modal logic
in the literature.  Logical rules that act at a leaf of a
tree-structured context go back to the Lambek
calculus~\citep{lambek58calculus}.  A rich collection of context
structures that correspond to type constructors plays a central role in
display logic~\citep{belnap82display}.  \citet{atkey04separation}'s
$\lambda$-calculus for resource separation is similar to mode theories
with one mode, where there is at most one 2-cell between a given pair of
1-cells; at the logical level, our calculus is a unification of this
with the multimode adjoint logic of
\citet{reed09adjoint}.  Algebraic resource annotations on variables are
used to track modalities in Agda's implementation~\citep{abel15modal}
and in \citet{mcbride16nuttin}'s approach to linear dependent types.  LF
representations of modal or substructural logics work by restricting the
use of cartesian variables~\citep{crary10substructural}.  Relative to
all of these approaches, we believe that the analysis of the context
structures/resources as a \emph{term} in a base type theory, and the
fibrational structure of the derivations over them, is a new and useful
observation.  For example, rather than needing extra-logical conditions
on proof rules to ensure cut admissibility, as in display logic, the
conditions are encoded in the language of context descriptors and the
definition of types from them.  Moreover, none of these existing
approaches allow for proof-relevant 2-cells/structural rules, and their
presence (and the equational theory we give for them) is important for
our applications to extensions of homotopy type theory.  A point of
contrast with substructural logical
frameworks~\citep{cervesatopfenning02llf,watkins+03clf-tr,reed09thesis}
is that logics are ``embedded'' in our calculus (giving a type
translation such that provability in the object logic corresponds to
provability in ours), rather than ``encoding'' the structure of
derivations.  This way, we obtain cut elimination for object languages
as a corollary of framework cut elimination.


In future work, we plan to continue a preliminary investigation of
equational adequacy that is discussed in the extended version, showing
that the logical adequacy results extend to an isomorphism on
$\beta\eta$-classes of derivations.  It is generally easy to show that
object-language equations are true in the framework.  We conjecture that
the converse is true for the mode theories we have described here, which
says that the ``extra'' types and judgements available in the framework
do not add to the equations between terms in the image of encoded
sequents.  Proving this is challenging because the equational theory of
Section~\ref{sec:equational} does not itself obviously have the
subformula property---we can in principle prove equations by introducing
and then eliminating cuts.  We have sketched a proof of equational
adequacy for the simplest case (ordered logic with products only),
assuming a lemma that the equational theory from
Section~\ref{sec:equational} can be characterized as some ``permuting
conversions'' on cut-free derivations (i.e. that one can first
$\beta$-reduce and then rearrange the cut-free term)---we proved the
analogue of this lemma for the single-variable case~\citep{ls16adjoint}.

Addtionally, we plan to apply our framework to investigate more
extensions of homotopy type theory like the spatial type theory
considered here; in current work with Eric Finster, we are designing a
variant of cohesion for talking synthetically about spectra.  We also
plan to consider encodings of programming-focused type theories, such as
specialized effect calculi~\citep{gaboardi16coeffect}.  Finally, our
adequacy proofs require reasoning about the 1- and 2-cells in the mode
theory, which we have currently done entirely na\"ively; we would like
to investigate using techniques from higher-dimensional rewriting to
simplify these proofs.





\setlength{\bibsep}{-2pt} %% dirty trick: make this negative
{ \small
\linespread{0.70}
\bibliographystyle{abbrvnat}
\bibliography{../drl-common/cs}
}

\end{document}

\section{Equational Adequacy}
\label{sec:adequacy-equational}

\subsection{Template}

In addition to the logical adequacy results above, we expect that the
translation from an object logic into the framework extends to something
like a full and faithful functor from the object logic to the framework.
Unpacking this, the object part of the functor means we want a
translation $A^*$ from object language types to framework types---and an
extension translating object-language sequents $J$ to framework sequents
$J^*$.  The morphism part of the functor maps each object-logic
derivation $d : J$ to a derivation $d^* : J^*$.  Functoriality means
that the translation takes identities to identities and cuts to cuts.
Together, full and faithfullness say that for each sequent $J$, the
object language derivations of $J$ are bijective with framework
derivations of $J^*$.  In particular, fullness says that for any sequent
$J$, the translation on derivations of that sequent is surjective: for
every derivation $e$ of $J^*$, there (merely) exists an object language
derivation $d : J$ such that $d^* = e$.  In terms of provability, this
says that no more sequents can be proved in the framework, and in terms
of proof identity, it says that every derivation could have been written
in the object language. Faithfullness says that the translation on
derivations is injective---$d_1^* = d_2^*$ implies $d_1 = d_2$---so no
more equalities can be proved in the framework.  The fact that a
function is a bijection iff it is surjective and injective gives the
overall result.

In the above discussion, we would like equality of derivations to
correspond to the categorical universal properties for the connectives,
which generally equate more morphisms than syntactic equality of
cut-free proofs (unless one uses more sophisticated sequent calculi than
we consider here, e.g. focusing/multifocusing).  On the framework side,
the equational theory of Section~\ref{sec:equational} already accounts
for this.  On the source side, will define a logic by the usual sequent
calculus rules that make cut and identity admissible, along with
primitive cut and identity rules, and an equality judgement analogous to
Section~\ref{sec:equational}, which is a concise description of
$\beta\eta$ rules.  Cut elimination for the source will be a corollary
of the adequacy theorem (we could simplify the source syntax by removing
the built-in cuts in the non-invertible rules, using the general cut
rule in their place, but including them is convenient for stating the
cut elimination corollary).  Thus, we refine the discussion above by
taking equality of derivations to be \deq-classes.

\newcommand\backtrf[1]{\ensuremath{#1^{\leftarrow}}}
\newcommand\backtr[1]{\ensuremath{#1^{\Leftarrow}}}
\newcommand\str[2]{\ensuremath{\dsd{str}_{#1}(#2)}}

We will generally focus on the following aspects of constructing such a
full and faithful functor:
\begin{definition}[The interesting part of an adequacy proof] ~
\begin{enumerate}
\item The translation from types to types ($A^*$) and sequents to
  sequents ($J^*$).

\item For each source inference rule for each connective, a
  \emph{derivation} $d^*$ from the translated premises to the translated
  conclusion (not just an admissibility: each rule will be defined by a
  composition of framework inference rules).

\item A proof that equality axioms are preserved: for each
  connective-specific equality axiom (typically $\beta\eta$) $d_1 \deq
  d_2$, $d_1^* \deq d_2^*$.

\item A function \backtrf{-} from normal derivations $e : J^*$ to source
  derivations of $J$.  If the output does not use the cut rule, or
  identity at non-base-types, this gives cut and identity elimination
  for the source as a corollary.

\item A proof that ${\backtrf{e}}^* \deq e$.  

\item A proof that the meta-operations are preserved by
  back-translation.  For example, for identity $\Identa{x} : J^*$, we
  should have $\backtrf{\Identa{x}} \deq x$.  For normal $e$ and $e'$ in
  the image of ``cutable'' sequents $J^*$ and $J'^*$,
  $\backtrf{(\Cuta{e}{e'}{x})} \deq \backtrf{e}[\backtrf{e'}/x]$ (Note:
  this can be stated for $\elim{d^*}$ and $\elim{d'^*}$ if that is more
  convenient).

\item A proof that $\backtrf{(\elim{d^*})} \deq d$.  The cases for
  identity and cut will use the previous bullet.

\item A proof that for normal $e,e' : J^*$, $e \deqp e$ imples
  $\backtrf{e} \deq \backtrf{e'}$.  (This can be stated for
  $\elim{d^*}$ if that is convenient.)
\end{enumerate}
\end{definition}

From this, the full construction is as follows:
\begin{remark}[The routine part of an adequacy proof] ~
\begin{enumerate}
\item For the construction of the functor:
\begin{enumerate}
\item The translation of types and sequents was given in part 1 above.

\item The cases of the translation of derivations $d^*$ given above are
  extended by sending identity to identity and cut to cut (possibly with
  some weakening-over-weakening and exchange-over-exchange), to
  determine a function from cutfull source derivations to cutfull
  framework derivations.  So functoriality is true by definition.  

\item We extend the above function $d^*$ on derivations to
  \deq-equivalence classes by proving $d_1 \deq d_2$ implies $d_1^* \deq
  d_2^*$.  The type-specific cases are given by part 3 above.
  Reflexivity, symmetry, and transitivity are sent to reflexivity,
  symmetry, transitivity rules in the framework. The congruence rule for
  each source derivation constructor is sent to a composition of
  framework congruence rules, which works because because each inference
  rule is shown derivable (not just admissible) in part 2 above. The
  unit and associativity laws for cut will be modeled by the
  corresponding laws in the framework.  
  %% application of a directed HIT?!
\end{enumerate}

\item For fullness, every $e$ is equal (by
  Theorem~\ref{thm:permutative-soundess}) to a
  cut/identity/transformation-free derivation $\elim{e}$, and the proof
  for cut-free derivations is given by \backtrf{-} (parts 4 and 5
  above).  Even though we are constructing a bijection between
  derivations modulo \deq, we do not need to show that this function
  respects the quotient: because of the ``mere
  existence''/$-1$-truncation in the definition of surjective, the
  function on representives automatically extends to the quotient.
  %% Even constructively, this suffices to define a (untruncated) bijection
  %% betweeb quotiented derivations---the fact that the framework-to-source
  %% direction respects $\deq$ follows from injectivity.

  If $\backtrf{-}$ does not use the cut rule in the source (or identity
  at non-base-type), then the composite $\backtrf{\elim{d^*}}$ 
  witnesses cut/identity elimination for the source.  

\item For faithfulness, we need to show that $d_1^* \deq d_2^*$ implies
  $d_1 \deq d_2$.  By part 8 above, it suffices to show 
  $\backtrf{\elim{d_1^*}} \deq \backtrf{\elim{d_2^*}}$.
  By completeness of permuatitive equality
  (Theorem~\ref{thm:permutative-completeness}), 
  $d_1^* \deq d_2^*$ implies $\elim{d_1^*} \deqp \elim{d_2^*}$,
  so part 9 above gives the result.
\end{enumerate}
\end{remark}

We do not abstract this ``template'' as a lemma because the class of
``native sequent calculi'' taken as input is not precisely
defined.  

\begin{lemma}[Equational 0-use Strengthing] \label{lem:0-use-strengthening-eq}
Under the conditions of Lemma~\ref{lem:0-use-strengthening}, write
\str{\vec{x}}{\D} for the derivation produced by the lemma.  Then
$\str{\vec{x}}{\D}$ (weakened with $\vec{x}$) $\deq \D$.  
%% Moreover, if $\vec{x}$ do not occur in the
%% derivation $d$ (i.e. at an identity rule or as the principal argument of
%% a left rule), then $\str{\vec{x}}{d}$ is syntactically equal to $d$.
\end{lemma}
\begin{proof}
Inspecting the proof, we have the following reductions:
\[
\begin{array}{rcll}
\str{\vec{x}}{\Trd{\beta}{y}} & := & {\Trd{\beta}{y}}\\
\str{\vec{x}}{\FRd{}{s}{\vec{d}}} & := & \FRd{}{s}{\str{\vec{x}}{\vec{d}}}\\
\str{\vec{x}}{\FLd{x}{\Delta.d}} & := & \str{\vec{x},x,\Delta}{d} & \text{ if } x \in \vec{x}\\
\str{\vec{x}}{\FLd{x}{\Delta.d}} & := & \FLd{x}{\Delta.\str{\vec{x}}{d}} & \text{ if } x \not\in \vec{x}\\
\str{\vec{x}}{\URd{\Delta.d}} & := & \URd{\Delta.\str{\vec{x}}{d}}\\
\str{\vec{x}}{\ULd{x}{s}{\vec{d}}{z.d'}} & := & \Trd{s}{\str{\vec{x,z}}{d'}} & \text{ if } z \# \beta'\\
\str{\vec{x}}{\ULd{x}{}{s}{\vec{d}}{z.d'}} & := & \ULd{x}{}{s}{\str{\vec{x}}{d}}{z.\str{\vec{x}}{d'}} & \text{ if } z \in \beta', x \notin \vec{x}\\
\end{array}
\]
%% If $\vec{x}$ do not occur in the derivation, then no rules are deleted,
%% so we can prove that the term is unchanged by the inductive hypothesis
%% in each case.  
Most cases follow from the inductive hypothesis; the interesting ones
are where a rule is deleted.  

\begin{itemize}

\item Suppose we began with $\FLd{x}{\Delta.d} :
  \seq{\Gamma,x:\F{\alpha}{\Delta}}{\beta}{B}$ with $x \in \vec{x}$ and
  produced $\str{\vec{x},x,\Delta}{d} : \seq{\Gamma-\vec{x}}{\beta}{B}$
  (because neither $x$ nor $\Delta$ occur in $\beta$).  Then weakening
  with $x,\Delta$ gives $\str{\vec{x},x,\Delta}{d} :
  \seq{\Gamma,x:\F{\alpha}{\Delta},\Delta}{\beta}{B}$, and the inductive
  hypothesis gives that this is equal to $d$.  

  Thus, it suffices to show that for any $e :
  \seq{\Gamma,x:\F{\alpha}{\Delta}}{\beta}{B}$ where $x$ doesn't occur
  in $\beta$ or $e$, then $e$ is equal to \FLd{x}{\Delta.e}---i.e. we
  can introduce a ``dead branch'' on $x$.  But by the $\eta$ rule we
  have $e \deq \FLd{x}{\Cut{e}{\FR}{x}}$, and the cut cancels
  because $x$ does not occur.
  
\item Suppose we began with $\ULd{x}{}{s}{\vec{d}}{z.d'} :
  \seq{\Gamma,x:\U{\alpha}{\Delta}{A}}{\beta}{B}$ and $z$ does not occur
  in $\beta'$, the resources of $d'$, and we produce
  \Trd{s}{\str{\vec{x},z}{d'}}, and want to show this is equal to
  $\ULd{x}{}{s}{\vec{d}}{z.d'}$.  By the IH, weakening
  $\str{\vec{x},z}{d'}$ is equal to $d'$.

  Thus, it suffices to show that $\ULd{x}{}{s}{\vec{d}}{z.d'} \deq
  \Trd{s}{d'}$ when $z$ does not occur in the \emph{derivation} $d'$.
  This is immediate from the semantic expansion of \UL{} 
  \[
  \ULd{x}{}{s}{\vec{d/y}}{z.d'} \deq \Trd{s}{(d'[\ULs{x}[\vec{d/y}]/z])}
  \]
  because in this the substitution for $z$, which does not occur,
  cancels.

\end{itemize}

\end{proof}
%% We say that a formula $\F{\alpha}{\Delta}$ and \U{c.\alpha}{\Delta}{A}
%% is relevant if every variable from $\Delta$ (and $c$ for \Usymb) occurs
%% at least once in $\alpha$.

%% Suppose the mode theory has the property that for all $x$, $\alpha$,
%% $\beta$, if $\alpha \spr \beta$ and $x \# \alpha$ then $x \# \beta$ (in
%% particular, equations must have the same variables on both sides).
%% Suppose additionally a sequent \seq{\Gamma}{\alpha}{A} such that every
%% \Fsymb/\Usymb\/ subformula of $\Gamma,A$ is relevant.

%% Then if $\D :: \seq{\Gamma}{\alpha}{A}$ and $\vec{x}$ are variables such
%% that $\vec{x} \# \alpha$ then there is a $\D' ::
%% \seq{\Gamma-\vec{x}}{\alpha}{A}$ and $size(\D') \le size(\D)$ 
%% $\D' \deq \D$ (when $\D'$ is weakened to reintroduce $\vec{x}$).  

\begin{lemma}[Properties of Strengthening] \label{lem:strengthening-eq-properties} ~ 
\begin{itemize}
\item $\str{\Gamma}{\str{\Delta}{e}} = \str{\Gamma,\Delta}{e}$
\item $\str{\Delta,x}{d} = \str{\Delta,x,\Delta'}{\Linv{d}{\Delta'}{x}}$
\item $\str{\Delta}{\Linv{d}{\Delta'}{x}} = \Linv{\str{\Delta}{d}}{\Delta'}{x}$ if $x \# \Delta$
\item $\str{\vec{x}}{\Ident{y}} = \Ident{y}$ if $y \not \in x$.  
\item $\str{\vec{x}}{d} = d$ if $\vec{x} \# d$ and for every $\UL$ in $d$,
  $z$ occurs in $\beta'$.    
\end{itemize}
\end{lemma}

\begin{proof}
For the first part, deleting rules in two passes is the same one, as
long as all of the same variables are deleted overall.  

For the second part, deleting all uses of $x$ is the same as first
inverting $x$ (which deletes all left rules on it) and then deleting all
uses of the variables that are produced by the inversion (which are
deleted when strengthening hits $\FL^{x}$).

For the third, left inversion commutes with deleting variables other
than the ones being deleted.  

For the fourth, the $\eta$-expanded identity on $y$ contains no variables
beside $y$ and those introduced by rules in it, so $\str{\vec{x}}$
deletes nothing.  

For the fifth, strengthening only deletes (1) left rules on $x$ and
subsidiaries, and (2) unneeded occurences of $\UL$, and the premises say
that there are none of these.  
\end{proof}

\begin{lemma}[Cut does not introduce left rules] \label{lem:cut-doesnt-intro}
If $\FL^{x}$ or $\UL^{x}$ occus in $\elim{d}$, then it occurs in $d$
\end{lemma}

\begin{proof}
By induction on the execution trace of cut elimination: for each rule in
Figure~\ref{fig:admissible-rule-equations}, $\FL^{x}$ or $\UL^{x}$ 
occurs on the right only if it occurs on the left.  
\end{proof}

\subsection{Ordered Logic (Product Only)}

As a first example of an adequacy proof, we consider the following mode
theory for ordered logic with only $A \odot B$:
\[
\infer{\seql{A}{o}{A}}{}
\quad
\infer{\seql{\Gamma,\Delta,\Gamma'}{o}{C}}
      {\seql{\Gamma,A,\Gamma'}{o}{C} &
        \seql{\Delta}{o}{A}}
\quad
\infer{\seql{\Gamma,A \odot B,\Gamma'}{o}{C}}
      {\seql{\Gamma,A,B,\Gamma'}{o}{C}}
\quad
\infer{\seql{\Gamma,\Delta}{o}{A \odot B}}
      {\seql{\Gamma}{o}{A} &
        \seql{\Delta}{o}{B}}
\]
\[
\begin{array}{c}
\Cut{\dotLd{z}{x,y.d}}{\dotRd{d_1}{d_2}}{z} \deq \Cut{\Cut{d}{d_1}{x}}{d_2}{y}\\
d : \seql{\Gamma,z:A \odot B,\Gamma'}{o}{C} \deq \dotLd{z}{x,y.\Cut{d}{\dotRd{x}{y}}{z}}\\
\end{array}
\]

We use a mode theory with a monoid $(\odot,1)$, so the only
transformation axioms are equality axioms for associativity and unit.  

The interesting parts of the adequacy proof are:
\begin{enumerate}
\item The type translation is given by $P^* := P$ and $(A \odot B)^* :=
  \F{x \odot y}{x:A^*,y:B^*}$.  A context $(x_1:A_1,\ldots,x_n:A_n)^* :=
  x_1:A_1^*,\ldots,x_n:A_n^*$.  Writing $\vars{x_1:A_1,\ldots,x_n:A_n}
  := x_1 \odot \ldots \odot x_n$, a sequent $\seql{\Gamma}{o}{A}$ is
  translated to \seq{\Gamma^*}{\vars{\Gamma}}{A^*}.

We use the following properties of the mode theory:
\begin{itemize}
\item If ${\vars{\Gamma^*}} \deq {x}$ then $\Gamma$ is $x:Q$ for some
  $Q$.  
\item If $\vars{\Gamma} \deq \alpha_1 \odot \alpha_2$, then there exist
  $\Gamma_1,\Gamma_2$ such that $\Gamma = \Gamma_1,\Gamma_2$ and
  $\vars{\Gamma_1} \deq \alpha_1$ and $\vars{\Gamma_2} \deq \alpha_2$.
\item $A^*$ and $\Gamma^*$ are relevant propositions, and the monoid
  axioms preserve variables, so by Lemma~\ref{lem:0-use-strengthening} we can
  strengthen away any variables that are not in the context descriptor.  
\end{itemize}

\item As discussed in Section~\ref{sec:adequacy:ordered-logical}, the
  inference rules for $\odot$ are derived as follows:

\[
\infer[\FL]{\seq{\Gamma^*,z:\F{x \odot y}{x:A^*,y:B^*},{\Gamma'}^*}{\vars{\Gamma}\odot z \odot \vars{\Gamma'}}{C}}
      {\infer[Lemma~\ref{lem:exchange}]
        {\seq{\Gamma^*,{\Gamma'}^*,x:A,y:B}{\vars{\Gamma}\odot x \odot y \odot \vars{\Gamma'}}{C}}
        {\seq{\Gamma^*,x:A,y:B,{\Gamma'}^*}{\vars{\Gamma}\odot x \odot y \odot \vars{\Gamma'}}{C}}}
\]

\[
\infer{\seq{\Gamma^*,\Delta^*}{\vars{\Gamma} \odot \vars{\Delta}}{\F{x \odot y}{x:A,y:B}}}
      {{\vars{\Gamma} \odot \vars{\Delta}} \spr (x \odot y)[\vars{\Gamma}/x,\vars{\Delta}/y]
        \infer[Lemma~\ref{lem:weakening}]
              {\seql{\Gamma^*,\Delta^*}{\vars{\Gamma}}{A}}
              {\seql{\Gamma^*}{\vars{\Gamma}}{A}} &
        \infer[Lemma~\ref{lem:weakening}]
              {\seql{\Gamma^*,\Delta^*}{\vars{\Gamma}}{A}}
              {\seql{\Delta^*}{\vars{\Delta}}{B}}}
\]

Identity and cut are

\[
\infer[Thm~\ref{thm:identity}]
      {\seq{x:A^*}{x}{A}}
      {}
\qquad
\infer[Thm~\ref{thm:cut}]
      {\seq{{\Gamma}^*,{\Delta}^*,{\Gamma'}^*}{\vars{\Gamma}\odot \vars{\Delta} \odot \vars{\Gamma'}}{C}}
      {\infer[Lem~\ref{lem:weakening}]
        {\seq{\Gamma^*,\Delta^*,x:A^*,{\Gamma'}^*}{\vars{\Gamma}\odot x \odot \vars{\Gamma'}}{C}}
        {\seq{\Gamma^*,x:A^*,{\Gamma'}^*}{\vars{\Gamma}\odot x \odot \vars{\Gamma'}}{C}} &
        \infer[Lem~\ref{lem:weakening}]{\seq{\Gamma^*,\Delta^*,x:A^*,{\Gamma'}^*}{\vars{\Delta}}{A^*}}
             {{\seq{\Delta^*}{\vars{\Delta}}{A^*}}}}
\]

Since we do not notate weakening and exchange, we can summarize these
as:
\[
\begin{array}{rcl}
(\dotLd{z}{x,y.d})^* & := & \FLd{z}{x,y.d^*}\\
(\dotRd{d_1}{d_2})^* & := & \FRd{}{1}{(d_1^*/x,d_2^*/y)}\\
x^* & := & x\\
(\Cut{e}{d}{x})^* & := & \Cut{e^*}{d^*}{x}
\end{array}
\]

\item The $\beta\eta$ axioms for $\odot$ translate almost exactly to the
  corresponding axioms for \F{x \odot y}{x:A^*,y:B^*}: for $\beta$, we
  also use the fact that \Trd{1}{-} is the identity.  

\item For the back-translation on normal derivations, suppose we have a
  normal derivation of \seq{\Gamma^*}{\vars{\Gamma}}{A^*}.  Because
  there are no $\Usymb$-formulas in the context, the only possible rules
  are hypothesis and the \Fsymb-rules.

\begin{itemize}
\item For identity
\[
\infer{\seq{\Gamma^*}{\vars{\Gamma}}{P}}
      {{\vars{\Gamma^*}} \spr {x} &
        x : P \in \Gamma^*}
\]
Because the only structural transformation axioms are equalities for
associativity and unit, we have ${\vars{\Gamma^*}} \deq {x}$, which in
turn implies that $\Gamma$ is $x:Q$ for some $Q$ (because if $\Gamma$ is
empty, does not contain $x$, or contains anything else, \vars{\Gamma}
will not equal $x$).  By definition, this implies $Q = P$, so $\Gamma$
is $x:P$.  Therefore the identity rule applies.

\item For \FR, because the only type that encodes to \Fsymb is $\odot$,
  we have
\[
\infer{\seq{\Gamma^*}{\vars{\Gamma}}{\F{x \otimes y}{x:A_1^*,y:A_2^*}}}
      {\vars{\Gamma} \deq \alpha_1 \odot \alpha_2 &
       \seq{\Gamma^*}{\alpha_1}{A_1^*} &
       \seq{\Gamma^*}{\alpha_2}{A_2^*}
      }
\]
By properties of the mode theory, $\Gamma = \Gamma_1,\Gamma_2$ with
$\vars{\Gamma_i} \deq \alpha_i$, so we have derivations of
\seq{\Gamma^*}{\vars{\Gamma_i}}{A_i^*}.  Because 0-use strengthening
applies, we can strengthen these to
\seq{\Gamma_i^*}{\vars{\Gamma_i}}{A_i^*}.  Then the inductive hypothesis
gives \seql{\Gamma_i}{A_i}, so applying the $\odot$ right rule gives the
result.

\item For \FL, because the only type encoding to $\Fsymb$ is $A \odot
  B$, we have
\[
\infer{\seq{\Gamma^*,z:\F{x \odot y}{x:A^*,y:B^*},{\Gamma'}^*}{\vars{\Gamma} \otimes z \otimes \vars{\Gamma'}}{C^*}}
      {\seq{\Gamma^*,{\Gamma'}^*,x:A^*,y:B^*}{\vars{\Gamma} \otimes (x \otimes y) \otimes \vars{\Gamma'}}{C^*}}
\]
By exchange (Lemma~\ref{lem:exchange}), we have a no-bigger derivation
of
{\seq{\Gamma^*,x:A^*,y:B^*,{\Gamma'}^*}{\vars{\Gamma} \otimes (x \otimes y) \otimes \vars{\Gamma'}}{C^*}} 
so applying the IH gives 
\seql{{\Gamma,x:A,y:B,{\Gamma'}}}{o}{C}, and then $\odot$-left gives the result.

\end{itemize}

That is,
\[
\begin{array}{rcl}
\backtrf{\Trd{1}{x}} & := & x\\
\backtrf{\FRd{}{1}{e_1/x,e_2/y}} & := & \dotRd{\backtrf{\str{}{e_1}}}{\backtrf{\str{}{e_2}}}\\
\backtrf{\FLd{z}{x,y.e}} & := & \dotLd{z}{x,y.\backtrf{e}}\\
\end{array}
\]
where $\str{x}{e_i}$ is the result of Lemma~\ref{lem:0-use-strengthening}.  

\item Next, we show that for normal $e :
  \seq{\Gamma^*}{\vars{\Gamma}}{A^*}$, ${\backtrf{e}}^* \deq e$.

In the case for the hypothesis rule for atoms, we have
\[
{\backtrf{\Trd{1}{x}}}^* = x^* = x \deq \Trd{1}{x}
\]

In the case for \FL, we have 
\[
(\backtrf{\FLd{z}{x,y.e}})^* = (\dotLd{z}{x,y.\backtrf{e}})^* =
{\FLd{z}{x,y.{\backtrf{e}}^*}}
\]
so the result follows from the inductive hypothesis.  

In the case for \FR, we have
\[
(\backtrf{\FRd{}{1}{e_1/x,e_2/y}})^* = (\dotRd{\backtrf{\str{}{e_1}}}{\backtrf{\str{}{e_2}}})^* =
{\FRd{}{1}{({\backtrf{{\str{}{e_1}}}}^*/x,{\backtrf{\str{}{e_2}}}^*/y)}}
\]
By the inductive hypothesis, we have 
${\backtrf{\str{}{e_i}}}^* \deq \str{}{e_i}$, but we have $\str{}{e_i} \deq e_i$ by
Lemma~\ref{lem:0-use-strengthening}.  

\item Meta-operations are preserved by back-translation:
%% FIXME: check this in more detail

\begin{itemize}
\item If $\Identa{x} : \seq{\Gamma^*}{x}{A^*}$, then
  $\backtrf{\Identa{x}} \deq x$.

\begin{itemize}
\item Case for $A = P$: We have $\backtrf{(\Trd{1}{x})} = x$ as required.

\item Case for $A = \F{x_1 \otimes x_2}{x_1:A_1^*,x_2:A_2^*}$.  By definition,
$\Identa{x}$ is 
$\FLd{x}{x_1,x_2.\FRd{}{1}{\Identa{x_1}/x_1,\Identa{x_2}/x_2}}$, 
so 
\backtrf{\Identa{x}} is 
\[
\dotLd{x}{x_1,x_2.\dotRd{\backtrf{(\str{x_2}{\Identa{x_1}})}/x_1}{\backtrf{(\str{x_1}{\Identa{x_2}})}/x_2}}
\]
Since $x_2$ doesn't occur in \Identa{x_1}, \str{x_2}{\Identa{x_1}}
is literally the same term as {\Identa{x_1}} (interpreted in a
bigger context), without rewriting by any definitional equalities.  
Therefore by the inductive hypothesis for $A_1^*$ and $A_2^*$ gives
\[
\dotLd{x}{x_1,x_2.\dotRd{{x_1}/x_1}{{x_2}/x_2}}
\]
which is equal to $x$ by the $\eta$ law for $A \odot B$.

\end{itemize}

\item Left-inversion: if $y,z \# d$, then 
$\backtrf{\Linv{d}{(y,z)}{x}} \deq \Cut{\backtrf{d}}{\dotRd{y}{z}}{x}$.

If $d$ is \FLd{x}{y,z.d'} then we get \backtrf{d'} on the left, and 
$\Cut{\dotLd{x}{y,z.d'}}{\dotRd{y}{z}}{x}$ on the right, so they are
equal by $\beta$ in the source.  Otherwise the result follows from the
inductive hypotheses.  

%% \Linv{\Trd{s}{x}}{\vec{x}}{x_0} & := & {\Trd{s}{x}}\\
%% \Linv{\FLd{y}{\Delta.\D}}{\vec{x}}{x_0} & := & \D[\Delta \leftrightarrow \vec{x}]\\
%% \Linv{\FLd{x}{\Delta.\D}}{\vec{x}}{x_0} & := & \FLd{x}{\Delta.\Linv{\D}{\vec{x}}{x_0}} & (x \# x_0)\\
%% \Linv{\FRd{}{s}{\vec{\D_i/x_i}}}{\vec{x}}{x_0} & := & \FRd{}{{s[1_{\alpha_0}/x_0]}}{{{\Linv{\D_i}{\vec{x}}{y}}/x_i}}\\

\item Cut: If 
$e : \seq{\Gamma'^*,\Gamma^*,\Delta^*}{\vars{\Gamma}}{B^*}$
and 
$d : \seq{\Gamma'^*,\Gamma^*-x_0,\Delta^*}{\vars{\Delta}}{A^*}$ (where
$x_0 : A \in \Gamma$), then
$\backtrf{\str{\Gamma'}{\Cuta{e}{d}{x_0}}} \deq \Cut{\backtrf{\str{\Gamma',\Delta}{e}}}{\backtrf{\str{\Gamma-x,\Gamma'}{d}}}{x_0}$.

%% If 
%% $e : \seq{\Gamma'^*,\Gamma^*,\Delta^*}{\vars{\Gamma}}{B^*}$
%% and 
%% $d : \seq{\Gamma'^*,\Gamma^*-x_0,\Delta^*}{\vars{\Delta}}{A^*}$ (where
%% $x_0 : A \in \Gamma$), then
%% $\backtrf{\str{\Gamma'}{\Cuta{e}{d}{x_0}}} \deq \Cut{\backtrf{\str{\Gamma',\Delta}{e}}}{\backtrf{\str{\Gamma-x,\Gamma'}{d}}}{x_0}$.

Here, $\Gamma'$ is some extra variables that occur in neither side of
the cut, which is necessary to get the induction to go through.

Since the only transformations are the identity, $\Trda{s}{d} = d$ for
any $s$ and $d$.  

The proof is to go through each reduction of cut and check that it is
valid in the source.  We check a few cases of the definition of cut:

\begin{itemize}
\item 
$\Cuta{x_0}{d}{x_0} = d$.  In this case, $\Gamma,\Gamma'$ are empty
  besides $x_0$, and the cut in the source reduces to $\backtrf{\str{}{d}}$.

%% \item 
%% $\Cuta{y}{d}{x_0} = y$.  

\item 
$\Cuta{\FLd{x_0}{x_1,x_2.\E}}{\FRd{}{1}{\vec{\D_i/x_i}}}{x_0} = {\Cutta \E {\vec{\D_i/x_i}}}$

We need to show
\[
\backtrf{\str{\Gamma'}{{\Cutta \E {\vec{\D_i/x_i}}}}} = 
\Cut{\backtrf{\str{}{{\FLd{x_0}{x_1,x_2.\E}}}}}{\backtrf{\str{}{\FRd{}{1}{\vec{\D_i/x_i}}}}}{x_0}
\]
The right-hand side is equal to
\Cut{\dotLd{x_0}{x_1,x_2.\backtrf{\str{}\E}}}{\dotRd{\backtrf{\str{}{\D_1}}}{\backtrf{\str{}{\D_2}}}}{x_0}, 
so reducing the cut in the source gives 
$\backtrf{\str{}\E}[{\backtrf{\str{}{\D_1}}}/x_1,{\backtrf{\str{}{\D_2}}}/x_2]$
and the inductive hypothesis gives the result.  

%% \item 
%% $\Cuta{\FRd{}{1}{\vec{\E}}}{\D}{x_0} = {\FRd{}{1}{\Cuta{\vec{\E}}{\D}{x_0}}}$

%% \item 
%% $\Cuta{\FLd{x}{y,x.\E}}{\D}{x_0} = \FLd{x}{y,z.\Cuta{\E}{\Linv{\D}{\Delta}{x}}{x_0}}$

%% \item 
%% $\Cuta{\E}{\FLd{x}{y,z.\D}}{x_0} = \FLd{x}{y,z.\Cuta{\Linv{\E}{y,z}{x}}{\D}{x_0}}$

%% \item The commutative cases follow from the inductive hypothesis, merging
%% strengthenings, the left inversion lemma in the previous part, and the
%% commutative cut rules in the source.  

\end{itemize}


\end{itemize}

\item Next, we show that $\backtrf{{\elim{d^*}}} \deq d$.

The proof is by induction on $d$.  

\begin{itemize}
\item In the case for $\dotLd{}{}$, expanding definitions, we have
\[
\backtrf{\elim{(\dotLd{z}{x,y.d})^*}} \deq
\backtrf{\elim{\FLd{z}{x,y.d^*}}} \deq 
\backtrf{\FLd{z}{x,y.\elim{d^*}}} \deq
{\dotLd{z}{x,y.\backtrf{\elim{d^*}}}}
\]
so the inductive hypothesis gives the result.  

\item In the case for $\dotRd{}{}$, we have
\[
\begin{array}{ll}
& \backtrf{\elim{(\dotRd{d_1}{d_2})^*}} \\
\deq & \backtrf{\elim{\FRd{}{1}{(d_1^*/x,d_2^*/y)}}} \\
\deq & \backtrf{\FRd{}{1}{(\elim{d_1^*}/x,\elim{d_2^*}/y)}} \\
\deq & \dotRd{\backtrf{\str{}{\elim{d_1^*}}}/x}{\backtrf{\str{}{\elim{d_2^*}}}/y}
\end{array}
\]
Note that the forward translation weakens $d_i^*$ when it constructs 
${\FRd{}{1}{(d_1^*/x,d_2^*/y)}}$, and cut elimination does not introduce
left rules on variables that are not case-analyzed somewhere in the
proof by Lemma~\ref{lem:cut-doesnt-intro}.  
So by Lemma~\ref{lem:strengthening-eq-properties}, strengthening
$\str{}{\elim{d_1}^*}$ will simply undo the weakening done in
constructing the term, and \backtrf{\str{}{\elim{d_1^*}}} equals
\backtrf{{\elim{d_1^*}}}.  Therefore the inductive hypotheses give the
result.

\item In the case for a variable, we have
\[
\backtrf{\elim{x^*}} \deq \backtrf{\elim{x}} \deq \backtrf{\Identa{x}}
\]
so the above part gives the result.  

\item In the case for cut, we have
\[
\backtrf{\elim{(\Cut{e}{d}{x})^*}} \deq \backtrf{\elim{\Cut{e^*}{d^*}{x}}}
\deq \backtrf{(\Cuta{\elim{e^*}}{\elim{d^*}}{x})}
\]
By the previous part, this is
\Cut{\backtrf{\elim{e^*}}}{\backtrf{\elim{d^*}}}{x} (the strengthening
only undoes the weakening that the forward translation puts in), so the
inductive hypothesis gives the result.
\end{itemize}

\item For normal derivations $e,e' :
  \seq{\Gamma^*}{\vars{\Gamma}}{A^*}$, if $e \deqp e'$, then
  $\backtrf{e} \deq \backtrf{e'}$.

We generalize slightly and prove that for 
$e,e' : \seq{\Gamma^*,\Delta^*}{\vars{\Gamma}}{A^*}$, 
if $e \deqp e'$, then $\backtrf{\str{\Delta}{e}} \deq \backtrf{\str{\Delta}{e'}}$.

\begin{itemize}

\item The cases for reflexivity, symmetry, and transitivity follow by
  induction, using the the correponding rules of source-language
  equalty.

\item The cases for compatibility all have a similar structure.  For
  example, suppose we have $\FRd{}{1}{e_1,e_2} \deqp
  \FRd{}{1}{e_1',e_2}$ because $e_1 \deqp e_1'$.
  Then by the inductive
  hypothesis we get $\backtrf{\str{\Delta,\Gamma_2}{e_1}} \deq
  \backtrf{\str{\Delta,\Gamma_2}{e_1'}}$.
  We need to show that $\backtrf{\str{\Delta}{\FRd{}{1}{e_1,e_2}}} \deq 
  \backtrf{\str{\Delta}{\FRd{}{1}{e_1,e_2}}}$.  By expanding definitions
  on both sides, it suffices to show
  \[
    \backtrf{\FRd{}{1}{\str{\Delta}{e_1},\str{\Delta}{e_2}}}
    \deq 
    \backtrf{\FRd{}{1}{\str{\Delta}{e_1'},\str{\Delta}{e_2}}}
  \]
  and so
  \[
    \dotRd{\backtrf{\str{\Gamma_2}{\str{\Delta}{e_1}}}}{\backtrf{\str{\Gamma_1}{\str{\Delta}{e_2}}}}
    \deq 
    \dotRd{\backtrf{\str{\Gamma_2}{\str{\Delta}{e_1'}}}}{\backtrf{\str{\Gamma_1}{\str{\Delta}{e_2}}}}
  \]
  By Lemma~\ref{lem:strengthening-eq-properties}, ${\str{\Gamma_2}{\str{\Delta}{e_1'}}} =
  {\str{\Gamma_2,\Delta}{e_1'}}$, so the inductive hypothesis gives the
  result.

\item For the axiom

\[
\FRd{}{s}{\D_1/x_1,\ldots,\Trd{{s_i}}{\D_i}/x_i,\ldots} \deqp \FRd{}{s;(1_{\alpha[\alpha_1/x_1,\ldots,\alpha_{i-1}/x_{i-1},\ldots]}[s_i/x_i])}{\vec{\D_j/x_j}}\\
\]
since there are only identity structural transformations in this mode
theory, $e$ and $e'$ must be syntactically identical terms.

\item For 
\[
\D \deqp \FLd{x}{y,z.\Linv{\D}{(y,z)}{x}}
\]
we need to show that
\[
\backtrf{\str{\Delta}{d}}
\deq 
\backtrf{\str{\Delta}{\FLd{x}{y,z.\Linv{\D}{(y,z)}{x}}}}
\]

We distinguish cases on whether $x \in \Delta$ (so the strengthening
deletes the \FL) or not.

If it does, we need to show that 
\[
\backtrf{\str{\Delta}{d}}
\deq 
\backtrf{\str{\Delta,x,y,z}{\Linv{\D}{(y,z)}{x}}}
\]
This follows from Lemma~\ref{lem:strengthening-eq-properties}.  

If it is not, we need to show that 
\[
\backtrf{\str{\Delta}{d}}
\deq 
\dotLd{x}{y,z.\backtrf{\str{\Delta}{\Linv{\D}{(y,z)}{x}}}}
\]
and by $\eta$-expanding the left-hand side gives
$\dotLd{x}{y,z.\Cut{\backtrf{\str{\Delta}{d}}}{\dotRd{y}{z}}{x}}$.
But in this case 
$\str{\Delta}{\Linv{\D}{(y,z)}{x}} = \Linv{\str{\Delta}{d}}{(y,z)}{x}$,
so the fact that back-translation preserves left inversion gives the
result.  

\end{itemize}

\end{enumerate}

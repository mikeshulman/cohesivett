\documentclass{amsart}
\usepackage{mathpartir,mathtools,amssymb}
\usepackage[all]{xy}
\def\flet#1:=^#2#3in{\mathsf{let}\;#1 \coloneqq^{#2} #3\;\mathsf{in}\;}
\def\type{\mathsf{type}}
\def\lax{\sim\hspace{-5pt}:\hspace{3pt}}
\def\inr{\mathsf{inr}\;}
\def\inl{\mathsf{inl}\;}
\def\case{\mathsf{case}}
\def\refl{\mathsf{refl}}
\def\J{\mathsf{J}}
\def\K{\ensuremath{\mathcal{K}}}
\def\O{\ensuremath{\mathbb{O}}}
\theoremstyle{definition}
\newtheorem{defn}{Definition}
\title{Thoughts on adjoint logic indexed by a 2-category}
\begin{document}
\maketitle

\section{Orientals}
\label{sec:orientals}

Let \K\ be the 2-category of modes.
I will write $\alpha^* \dashv \alpha_*$ for the adjunction $F_\alpha \dashv U_\alpha$, since I'm thinking of it as a geometric morphism.

Let's think about what data in \K should parametrize a judgment with multiple hypotheses and dependent types.
The simple case is when $\alpha:y\to x$ and we have only one hypothesis:
\[ x:A_x \vdash_{\alpha} b : B_y \]
Semantically, this is a map $\alpha^*A_x \to B_y$, or equivalently $A_x \to \alpha_* B_y$.
But the same rule should apply to dependent types, since they are maps into the universe:
\[ x:A_x \vdash_{\alpha} B_y : \type_y \]
Semantically, a judgment like this means that $B_y$ is a fibration over $\alpha^* A_x$.

Now we should be able to add a variable of this type to the context:
\[ x:A_x ,_\alpha y:B_y \vdash \]
which means that in a context we must keep track of morphisms relating the modes at which the variables live.
A term in this context
\[ x:A_x ,_\alpha y:B_y \vdash c : C_z\]
ought to involve morphisms $\beta : z \to y$ and $\gamma: z\to x$, which are related in some way to $\alpha$.
It seems to me that the correct relationship is a 2-cell $\mu : \gamma \to \alpha \beta$.
This allows us to say semantically that this is a map to $C_z$ from the following pullback:
\begin{equation*}
  \vcenter{\xymatrix{
      \bullet \ar[r]\ar[d] &
      \beta^* B_y\ar[d]\\
      \gamma^* A_x \ar[r] &
      \beta^* \alpha^* A_x
      }}
\end{equation*}
Similarly, a dependent type
\[ x:A_x ,_\alpha y:B_y \vdash C_z :\type_z\]
will be a fibration over this pullback, and so when we add it to the context, we have to record not only $\alpha$, $\beta$, and $\gamma$, but also $\mu$.
We can proceed in this way, leading to the following definition.

\begin{defn}
  An \textbf{$n$-oriental} in a 2-category \K consists of
  \begin{itemize}
  \item $n$ objects $p_1,\dots, p_n$
  \item Whenever $1\le i<j\le n$, a morphism $\alpha_{ji} : p_j \to p_i$
  \item Whenever $1\le i<j<k\le n$, a 2-cell $\mu_{kji} : \alpha_{ki} \to \alpha_{ji}\alpha_{ki}$
  \item Whenever $1\le i<j<k<\ell\le n$, we have
    \begin{equation*}
      \vcenter{\xymatrix@+3pc{
          p_j \ar[d]_{\alpha_{ji}}
          \ar@{}[dr]|(.3){\Uparrow\mu_{kji}}
          \ar@{}[dr]|(.7){\Uparrow\mu_{\ell ki}}
          &
          p_k\ar[l]^{\alpha_{kj}} \ar[dl]|{\alpha_{ki}}\\
          p_i &
          p_\ell \ar[l]^{\alpha_{\ell i}} \ar[u]_{\alpha_{\ell k}}
        }}
      = 
      \vcenter{\xymatrix@+3pc{
          p_j \ar[d]_{\alpha_{ji}} &
          p_k\ar[l]^{\alpha_{kj}} \\
          p_i
          \ar@{}[ur]|(.3){\Uparrow\mu_{\ell ji}}
          \ar@{}[ur]|(.7){\Uparrow\mu_{\ell kj}}
          &
          p_\ell \ar[l]^{\alpha_{\ell i}} \ar[u]_{\alpha_{\ell k}} \ar[ul]|{\alpha_{\ell j}}
        }}\end{equation*}
  \end{itemize}
\end{defn}

(It is possible to define orientals in an $\omega$-category involving cells up to dimension $n-1$, but in a 2-category all the higher cells trivialize.
I suppose one might imagine an $(\infty,2)$-adjoint logic in which we need higher cells too, but not today.)

The claim now is that every judgment
\[ u_1 : A_1, \cdots, u_{n-1} : A_{n-1} \vdash_{\O} a_{n} : A_{n} \]
in 2-adjoint logic should be parametrized by an $n$-oriental $\O$ in \K, where each type $A_i$ has mode $p_i$.
It should be possible to define inductively the intended semantic meaning of this, with iterated applications of left adjoints and pullbacks as above.
In the non-dependent case, it's easy to say what we mean: such a judgment should correspond to a map
\[ \alpha_{n1}^* A_1 \times \cdots \times \alpha_{n,n-1}^* A_{n-1} \longrightarrow A_n \]
in mode $n$.

\section{The adjoint connectives}
\label{sec:adjoint-connectives}

For now, let's think about the rules for the adjoint connectives $\alpha^*$ and $\alpha_*$.
The easiest one is UR:
\[ \inferrule{\Gamma \vdash_{\O \circ \beta} M : A}{\Gamma \vdash_{\O} M_\beta : \beta_* B} \]
Here $\O$ is an oriental ending at a mode $p_{n}$, and $\beta$ is a morphism $y \to p_{n}$, so that we have an oriental $\O \circ \beta$ obtained by composing each morphism $\alpha_{ni}$ with $\beta$.
To see that this is sensible, think about the non-dependent semantics: the hypothesis is a map
\[ (\alpha_{n1}\beta)^* A_1 \times \cdots \times (\alpha_{n,n-1}\beta)^* A_{n-1} \longrightarrow B \]
which, assuming $\beta^*$ preserves products, is equivalent to a map
\[ \beta^*\Big(\alpha_{n1}^* A_1 \times \cdots \times \alpha_{n,n-1}^* A_{n-1}\Big) \longrightarrow B \]
and hence to a map
\[ \alpha_{n1}^* A_1 \times \cdots \times \alpha_{n,n-1}^* A_{n-1} \longrightarrow \beta_* B \]
which is the conclusion.

For UL, note that any $n$-oriental $\O$ contains $n$ $(n-1)$-orientals $\partial_i\O$ obtained by omitting the mode $p_i$.
Now let $\O$ be an $(n+1)$-oriental such that $\alpha_{n,n-1} = \beta$; then the rule is
\[ \inferrule{\Gamma,u:A \vdash_{\partial_{n-1} \O} M:B}{\Gamma,x:\beta_*A \vdash_{\partial_n \O} (\flet u_\beta :=^\O x in M):B} \]
Note that the oriental $\O$ can be considered extra data chosen on $\partial_{n}\O$ and $\beta$.
So we can read this rule backwards as: given an arbitrary $n$-oriental ``$\partial_{n}\O$'' and a morphism $\beta$ ending at $p_{n-1}$, we can construct an element of $B$ judged along $\partial_{n}\O$ with a last hypothesis in $\beta_*A$ by choosing an $(n+1)$-oriental extending $\partial_{n}\O$ and $\beta$ and giving an element of $B$ judged along $\partial_{n-1}\O$ whose last hypothesis is instead in $A$.
Semantically (and non-dependently), the hypothesis is a map
\[ \alpha_{n+1,\bullet}^*\Gamma \times \alpha_{n+1,n}^* A \longrightarrow B \]
and since $\beta = \alpha_{n,n-1}$, we have a natural transformation
\[ \alpha_{n+1,n-1}^* \beta_* \to \alpha_{n+1,n}^* \beta^* \beta_*  \to \alpha_{n+1,n}^* \]
using $\mu_{n+1,n,n-1}$ and the counit of the adjunction $\beta^* \dashv \beta_*$.
Thus, we can compose with this transformation at $A$ to obtain the conclusion:
\[ \alpha_{n+1,\bullet}^*\Gamma \times \alpha_{n+1,n-1}^* \beta_* A \longrightarrow B. \]

The rule FL is similar, but with an extra condition.
Let $\O$ be an $(n+1)$-oriental such that $\alpha_{n,n-1} = \beta$, and moreover $\mu_{n+1,n,n-1}$ is an identity (or an isomorphism).
Then the rule is
\[ \inferrule{\Gamma,u:A \vdash_{\partial_{n} \O} M:B}{\Gamma,x:\beta^* A \vdash_{\partial_{n-1} \O} (\flet u^\beta :=^\O x in M):B} \]
Again, the rule can be read backwards by considering $\O$ as extra data chosen on $\partial_{n-1}\O$ and $\beta$.
Semantically and non-dependently, the hypothesis is a map
\[ \alpha_{n+1,\bullet}^*\Gamma \times \alpha_{n+1,n-1}^* A \longrightarrow B \]
But since $\alpha_{n+1,n-1} \cong \beta \alpha_{n+1,n}$ (the isomorphism $\mu_{n+1,n,n-1}$), this is equivalent to a map
\[ \alpha_{n+1,\bullet}^*\Gamma \times \alpha_{n+1,n}^* \beta^* A \longrightarrow B \]
which is the conclusion.

(Is the fact that the hypotheses of UR and FL are \emph{equivalent} to their conclusions what Reed meeds by saying that U is invertible on the right and F is invertible on the left?)

I would like to also have ``Frobenius-ified'' versions of UL and FL that apply to a hypothesis anywhere in the context instead of just at the end.
I'm not sure whether that is easy or not.

Finally, FR is the trickiest.
Here's the best I've been able to come up with; I'm not positive it's correct.
Let $\O$ be an $(n+1)$-oriental and $1\le k\le n$, with $\alpha_{n+1,k} = \beta$.
Then in addition to the $n$-oriental $\partial_k \O$, we have a $k$-oriental $\partial_{\le k} \O$ obtained by discarding all modes after $p_k$.
Now the rule is
\[ \inferrule{\Gamma\!\downharpoonright_k\; \vdash_{\partial_{\le k}\O} M : B }{\Gamma \vdash_{\partial_k \O} M^\beta : \beta^* B} \]
Here $\Gamma\!\downharpoonright_k$ denotes $\Gamma$ restricted to $\partial_{< k}\O$.
Semantically and non-dependently, the hypothesis is a map
\[ \alpha_{k1}^* A_1 \times \cdots \times \alpha_{k,k-1}^* A_{k-1} \longrightarrow B \]
so we can obtain the conclusion as the composite
\begin{align*}
  (\alpha_{n+1,1})^* A_1 \times \overbrace{\cdots}^{\widehat{k}} \times (\alpha_{n+1,n})^* A_{n}
  &\to (\alpha_{n+1,1})^* A_1 \times {\cdots} \times (\alpha_{n+1,k-1})^* A_{k-1}\\
  % &\to (\alpha_{k1}\alpha_{n+1,k})^* A_1 \times {\cdots} \times (\alpha_{k,k-1}\alpha_{n+1,k})^* A_{k-1}\\
  &\to (\alpha_{k1}\beta)^* A_1 \times \cdots \times (\alpha_{k,k-1}\beta)^* A_{k-1}\\
  &\cong \beta^*\alpha_{k1}^* A_1 \times \cdots \times \beta^*\alpha_{k,k-1}^* A_{k-1}\\
  &\cong \beta^*\Big(\alpha_{k1}^* A_1 \times \cdots \times \alpha_{k,k-1}^* A_{k-1} \Big)\\
  &\to \beta^* B.
\end{align*}
where the noninvertible maps are respectively a projection, induced by the 2-cells $\mu_{n+1,k,i}$, and the hypothesis.

There should be computation rules that involve information about composition in \K.


\section{Flat and sharp}
\label{sec:flat-sharp}

Now let's specialize to the case when $\K$ is the walking coreflection.
Thus, it has two modes $v$ and $t$, two generating morphisms $\delta:t\to v$ and $\gamma:v\to t$, such that $\delta\gamma = 1_v$ (hence $\gamma\delta$ is idempotent), and a 2-cell $\epsilon : \gamma\delta \to 1_t$ such that $\delta*_l\epsilon$ and $\epsilon*_r\gamma$ are identities.

An $n$-oriental in this \K\ is determined by the following:
\begin{itemize}
\item A list $p_1,\dots,p_n$ of modes (i.e.\ each either $v$ or $t$).
\item Whenever $i<j$ and $p_i = p_j = t$, a choice of either $1_t$ or $\gamma\delta$ to be $\alpha_{ji}$.
  (All other hom-sets are uniquely inhabited.)
\item Whenever $i<j<k$ and $p_i = p_k = t$ and $\alpha_{ki} = 1_t$, it must be the case that $p_j = t$ and $\alpha_{kj} = \alpha_{ji} = 1_t$ as well.
  (Any 2-cell other than $1_t$ is initial in its hom-category, so there is a unique choice of $\mu$.)
\end{itemize}
In particular, when making a judgment whose conclusion has mode $t$, the context splits up naturally as $\Delta;\Gamma$, where $\Gamma$ consists of those hypotheses of mode $t$ for which $\alpha_{nk} = 1_t$.

The hypotheses in $\Delta$ can be of either mode, as can those in a judgment with conclusion of mode $v$.
I'd like to say that in these cases we can go back and forth between the two kinds of hypotheses easily, but that seems to require Frobenius-ified versions of FL and UL.


\end{document}

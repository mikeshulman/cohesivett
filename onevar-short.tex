
\documentclass{drl-common/llncs}

\usepackage{drl-common/diagrams}
\usepackage{multicol}
\usepackage{mathptmx}
\usepackage{color}
\usepackage[cmex10]{amsmath}
\usepackage{amssymb}
\usepackage{stmaryrd}
\usepackage{drl-common/proof}
\usepackage{drl-common/typesit}
\usepackage{drl-common/typescommon}
\usepackage[square,numbers,sort]{natbib}
%% \usepackage{arydshln}
\usepackage{graphics}
\usepackage{url}
\usepackage{relsize}
\usepackage{fancyvrb}
\usepackage{tikz}
\usetikzlibrary{decorations.pathmorphing}
\usepackage{tipa}

\usepackage{drl-common/code}
\DefineVerbatimEnvironment{code}{Verbatim}{fontsize=\small,fontfamily=tt}

%% small tightcode, with space around it
\newenvironment{stcode}
{\smallskip
\begin{small}
\begin{tightcode}}
{\end{tightcode}
\end{small}
\smallskip}

\renewcommand{\sem}[1]{\ensuremath{ \llbracket #1 \rrbracket}}

\newcommand{\inv}[1]{\ensuremath{{#1}^{-1}}}

\newcommand{\C}{\ensuremath{\mathcal{C}}}
\newcommand{\D}{\ensuremath{\mathcal{D}}}
\newcommand{\M}{\ensuremath{\mathcal{M}}}
\newcommand{\la}{\ensuremath{\dashv}}
\newcommand{\arrow}[3]{\ensuremath{#2 \longrightarrow_{#1} #3}}
\newcommand{\tc}[2]{\ensuremath{#1 \Rightarrow #2}}
\newcommand{\sh}{\text{\textesh}}
\newcommand{\Adj}{\textbf{Adj}}

\newcommand\compo[2]{\ensuremath{#1 \circ #2}}
\newcommand\compv[2]{\ensuremath{#1 \cdot #2}}
\newcommand\comph[2]{\ensuremath{#1 \mathbin{\circ_2} #2}}

\renewcommand\wftp[2]{\ensuremath{#1 \,\,\, \dsd{type}_{#2}}}
\newcommand\F[2]{\ensuremath{F_{#1} \,\, #2}}
\newcommand\U[2]{\ensuremath{U_{#1} \,\, #2}}
\newcommand\coprd[2]{\ensuremath{#1 + #2}}
\newcommand\seq[3]{\ensuremath{#1 \, [ #2 ] \, \vdash \, #3}}
\renewcommand\irl[1]{\dsd{#1}}

\newcommand\tr[2]{\ensuremath{{{#1}_{*}(#2)}}}
\newcommand\ident[1]{\ensuremath{\dsd{ident}_{#1}}}
\newcommand\cutsym{\ensuremath{\dsd{cut}}}
\newcommand\cut[2]{\ensuremath{{\cutsym \,\, #1 \,\, #2}}}
\newcommand\cuti{\ensuremath{\bullet}}

\newcommand\hyp[1]{\ensuremath{\dsd{hyp} \, {#1}}}
\newcommand\Inl[1]{\ensuremath{\dsd{Inl}(#1)}}
\newcommand\Inr[1]{\ensuremath{\dsd{Inr}(#1)}}
\newcommand\Case[2]{\ensuremath{\dsd{Case}(#1,#2)}}
\newcommand\UL[3]{\ensuremath{\dsd{UL}^{#1}_{#2}(#3)}}
\newcommand\FR[3]{\ensuremath{\dsd{FR}^{#1}_{#2}(#3)}}
\newcommand\FL[1]{\ensuremath{\dsd{FL}(#1)}}
\newcommand\UR[1]{\ensuremath{\dsd{UR}(#1)}}

\newcommand\ap[2]{\ensuremath{#1 \approx #2}}

\newcommand\Bx[2]{\ensuremath{\Box_{#1} \, {#2}}}
\newcommand\Crc[2]{\ensuremath{\bigcirc_{#1} \, {#2}}}

\newcommand\Flat[1]{\ensuremath{\flat \, {#1}}}
\newcommand\Sharp[1]{\ensuremath{\sharp \, {#1}}}

\newcommand\iso{\cong}
\newcommand\ltor[2]{\ensuremath{{#1^{\vartriangleright_{#2}}}}}
\newcommand\rtol[2]{\ensuremath{{#1^{\vartriangleleft_{#2}}}}}

%% \newtheorem{\example}[theorem]{Example}

\title{Adjoint Logic with a 2-Category of Modes}

\author{Daniel R. Licata\inst{1} \and Michael Shulman\inst{2}
\thanks{
This material is based on research sponsored by The United States Air
Force Research Laboratory under agreement number FA9550-15-1-0053. The
U.S. Government is authorized to reproduce and distribute reprints for
Governmental purposes notwithstanding any copyright notation thereon.
The views and conclusions contained herein are those of the authors and
should not be interpreted as necessarily representing the official
policies or endorsements, either expressed or implied, of the United
States Air Force Research Laboratory, the U.S. Government, or Carnegie
Mellon University.
}}

\institute{Wesleyan University \and University of San Diego}

\begin{document}
\maketitle

\begin{abstract}
We generalize the adjoint logics of Benton and Wadler (1996) and Reed
(2009) to allow multiple different adjunctions between the same
categories.  This provides insight into the structural proof theory of
cohesive homotopy type theory, which integrates the synthetic homotopy
theory of homotopy type theory with the synthetic topology of Lawvere's
axiomatic cohesion.
%% Whereas Reed's calculus is
%% parametrizd by a preorder of modes, we generalize this to a 2-category
%% of modes, where each mode represents a category, each mode morphism
%% represents an adjunction, and each mode 2-morphism represents a morphism
%% of adjunctions.  We give a sequent calculus, show that identity and cut
%% are admissible, and define an equational theory on proofs.  We show that
%% this syntax is sound and complete for pseudofunctors from the mode
%% 2-category to the 2-category of categories and adjunctions.  Finally, we
%% investigate the consequences of some specific mode 2-categories, such as
%% a theory giving rise to an adjoint triple, and a theory corresponding to
%% the rules for cohesive homotopy type theory in Shulman (2015).
\end{abstract}

\section{Introduction}

An adjunction $F \la U$ between categories \C and \D\/ consists of a
pair of functors $F : \C \to \D$ and $U : \D \to \C$ such that maps
\arrow{\D}{F C}{D} correspond naturally to maps \arrow{\C}{C}{U D}.  A
prototypical adjunction, which provides a mnemonic for the notation, is
where $U$ takes the underlying set of some algebraic structure such as a
group, and $F$ is the free structure on a set---the adjunction property
says that a structure-preserving map from $F C$ to $D$ corresponds to a
a map of sets from $C$ to $U D$ (because the action on the structure is
determined by being a homomorphism).  Adjunctions are important to the
proof theories and $\lambda$-calculi of modal logics, because the
composite $FU$ is a comonad on \D, while $UF$ is a monad on $\C$.
\citet{bentonwadler96adjoint} describe an adjoint $\lambda$-calculus for
mixing linear logic and structural/cartesian logic, with functors $U$
from linear to cartesian and $F$ from cartesian to linear; the $! A$
modality of linear logic arises as the comonad $FU$, while the monad of
Moggi's metalanguage~\citep{moggi91monad} arises as $UF$.
\citet{reed09adjoint} describes a generalization of this idea to
situations involving more than one category: the logic is parametrized
by a preorder of \emph{modes}, where every mode $p$ determines a
category, and there is an adjunction $F \la U$ between categories $p$
and $q$ (with $F : q \to p$) exactly when $q \ge p$.  For example, the
intuitionistic modal logics of \citet{pfenningdavies} can be encoded as
follows: the necessitation modality $\Box$ is the comonad $FU$ for an
adjunction between ``truth'' and ``validity'' categories, the lax
modality $\bigcirc$ is the monad $UF$ of an adjunction between ``truth''
and ``lax truth'' categories, while the possibility modality $\diamond$
requires a more complicated encoding involving four adjunctions between
four categories.  While specific adjunctions such as $(- \times A) \la
(A \to -)$ arise in many logics, adjoint logic provides a formalism for
abstract/uninterpreted adjunctions.

In Reed's logic, modes are specified by a preorder, which allows at most
one adjunction between any two categories (more precisely, there can be
two isomorphic adjunctions if both $p \ge q$ and $q \ge p$).  However,
it is sometimes useful to consider multiple different adjunctions
between the same two categories.  A motivating example is Lawvere's
axiomatic cohesion~\citep{lawvere07cohesion}, a general categorical
interface that describes \emph{cohesive spaces}, such as topological
spaces, or manifolds with differentiable or smooth structures.  The
interface consists of two categories \C and $\mathcal{S}$, and a
quadruple of adjoint functors $\Pi_0,\Gamma : \mathcal{C} \to
\mathcal{S}$ and $\Delta,\nabla : \mathcal{S} \to \mathcal{C}$ where
$\Pi_0 \la \Delta \la \Gamma \la \nabla$.  The idea is that
$\mathcal{S}$ is some category of ``sets'' that provides a notion of
``point'', and \C\/ is some category of cohesive spaces built out of
these sets, where points may be stuck together in some way (e.g. via
topology).  $\Gamma$ takes the underlying set of points a cohesive
space, forgetting the cohesive structure.  This forgetful functor's
right adjoint $\Gamma \la \nabla$ equips a set with codiscrete cohesion,
where all points are stuck together; the adjunction says that a map
\emph{into} a codiscrete space is the same as a map of sets.  The
forgetful functor's left adjoint $\Delta \la \Gamma$ equips a set with
\emph{discrete cohesion}, where no points are stuck together; the
adjunction says that a map \emph{from} a discrete space is the same as a
map of sets.  The further left adjoint $\Pi_0 \la \Delta$, gives the set
of connected components---i.e. each element of $\Pi_0 C$ is an
equivalence class of points of $C$ that are stuck together.  $\Pi_0$ is
important because it translates some of the cohesive information about a
space into a setting where we no longer need to care about the cohesion.
These functors must satisfy some additional laws, such as $\Delta$ and
$\nabla$ being fully faithful (maps between discrete or codiscrete
cohesive spaces should be the same as maps of sets).

A variation on axiomatic cohesion called \emph{cohesive homotopy type
  theory}~\citep{schreiber13dcct,schreibershulman12cohesive,shulman15realcohesion} is
currently being explored in the setting of homotopy type theory and
univalent foundations~\citep{voevodsky06homotopy,uf13hott-book}.
Homotopy type theory uses Martin-L\"of's intensional type theory as a
logic of \emph{homotopy spaces}: the identity type provides an
$\infty$-groupoid structure on each type, and spaces such as the spheres
can be defined by their universal properties using higher inductive
types~\citep{lumsdaine+13hits,shulman11hitsblog,lumsdaine11hitsblog}.
Theorems from homotopy theory can be proved \emph{synthetically} in this
logic~\citep{ls13pi1s1,lb13pinsn,lf14emspace,lb15cubical,favonia14covering,cavallo14mayervietoris},
and these proofs can be interpreted in a variety of
models~\citep{shulman13inversediag,voevodsky+12simpluniv,coquand+13cubical}.
However, an important but subtle distinction is that there is no
\emph{topology} in synthetic homotopy theory: the ``homotopical circle''
is defined as a higher inductive type, essentially ``the free
$\infty$-groupoid on a point and a loop,'' which a priori has nothing to
do with the ``topological circle,'' $\{ (x,y) \in \mathbb{R}^2 \mid x^2
+ y^2 = 1\}$, where $\mathbb{R}^2$ has the usual topology.  This is both
a blessing and a curse: on the one hand, proofs are not encumbered by
topological details; but on the other, internally to homotopy type
theory, we cannot use synthetic theorems to prove facts about
topological spaces.

Cohesive homotopy type theory combines the synthetic homotopy theory of
homotopy type theory with the synthetic topology of axiomatic cohesion,
using an adjoint quadruple $\sh \la \Delta \la \Gamma \la \nabla$.  In
this higher categorical generalization, $\mathcal{S}$ is an
$(\infty,1)$-category of homotopy spaces (e.g. $\infty$-groupoids), and
$\C$ is an $(\infty,1)$-category of cohesive homotopy spaces, which are
additionally equipped with a topological or other cohesive structure at
each level.  The rules of type theory are now interpreted in \C, so that
each type has an $\infty$-groupoid structure (given by the identity
type) \emph{as well as} a separate cohesive structure on its objects,
morphisms, morphisms between morphisms, etc.  For example, types have
both morphisms, given by the identity type, and topological paths, given
by maps that are continuous in the sense of the cohesion.  As in the
1-categorical case, $\Gamma$ forgets the cohesive structure, yielding
the underlying homotopy space, while $\Delta$ and $\nabla$ equip a
homotopy space with the discrete and codiscrete cohesion.  But in the
$\infty$-categorical case, $\Delta$'s left adjoint $\sh A$ (pronounced
``shape of $A$'') generalizes from the connected components to the
\emph{fundamental homotopy space} functor, which makes a homotopy space
from the topological/cohesive paths, paths between paths, etc. of $A$.
This captures the process by which homotopy spaces arise from cohesive
spaces; for example, one can prove (using some additional axioms) that
the shape of the topological circle is the homotopy
circle~\citep{shulman15realcohesion}.  This allows synthetic
homotopy theory to be used in proofs about topological spaces, and opens
up possibilities for using synthetic homotopy theory as a tool in other
areas of mathematics and theoretical physics.

This paper begins an investigation into the structural proof theory of
cohesive homotopy type theory, as a special case of generalizing Reed's
adjoint logic to allow multiple adjunctions between the same categories.
As one might expect, the first step is to generalize the mode preorder
to a mode category, so that we can have multiple different morphisms
$\alpha, \beta : p\ge q$.  This allows the logic to talk about different
but unrelated adjunctions between two categories.  However, in order to
describe an adjoint triple such as $\Delta \la \Gamma \la \nabla$, we
need to know that the same functor $\Gamma$ is both a left and right
adjoint.  To describe such a situation, we generalize to a 2-category of
modes, where each mode $p$ determines a category, each morphism $\alpha
: p \ge q$ determines adjoint functors $F_\alpha : p \to q$ and
$U_\alpha : q \to p$ where $F_\alpha \la U_\alpha$, and each 2-cell $e :
\tc \alpha  \beta : q \ge p$ determines a morphism of adjunctions between 
$F_\beta \la U_\beta$ and $F_\alpha \la U_\alpha$.  Using this logic, an
adjoint triple is specified by the mode 2-category with
\begin{itemize}
\item objects $c$ and $s$
\item 1-cells $\dsd{d} : s \ge c$ and $\dsd{n} : c \ge s$
\item 2-cells $\tc {1_c} {\compo{\dsd{n}}{\dsd{d}}}$ 
and $\tc {\compo{\dsd{d}}{\dsd{n}}} {1_s}$ satisfying 
some equations
\end{itemize}
The 1-cells generate $F_{\dsd d} \la U_{\dsd d}$ and $F_{\dsd n} \la
U_{\dsd n}$, while the 2-cells are sufficient to prove that $U_{\dsd d}$ is
naturally isomorphic to $F_{\dsd n}$, so we can define $\Delta :=
F_{\dsd n}$, $\nabla := U_{\dsd n}$, and $\Gamma := U_{\dsd d} \cong F_{\dsd n}$
and have the desired adjoint triple.  Indeed, you may recognize this
2-category as the ``walking adjunction'' with $\dsd d \la \dsd n$---that
is, we give an adjoint triple by saying that the mode morphism generating
the adjunction $\Delta\la \Gamma$ is itself left adjoint to the mode morphism generating the
adjunction $\Gamma\la\nabla$.

The main judgement of the logic is a ``mixed-category'' entailment
judgement \seq{A}{\alpha}{C} where $A$ has mode $q$ and $C$ has mode $p$
and $\alpha : q \ge p$.  Semantically, this judgement means a morphism
from $A$ to $C$ ``along'' the adjunction determined by $\alpha$---i.e. a
map $\arrow{}{\F \alpha A}{C}$ or $\arrow{}{A}{\U \alpha C}$.
% \footnote{We could instead
%   use a structure that includes a basic notion of ``morphisms along
%   $\alpha$,' such as a Grothendieck bifibration over the 2-category of
%   modes, or a pseudofunctor to the bicategory of profunctors that are
%   both representable and corepresentable; these are equivalent to the
%   structures used here.}
However, taking the mixed-mode judgement as
primitive makes for a nicer sequent calculus: $U$ and $F$ can be
specified independently from each other, by left and right rules, in
such a way that identity and cut (composition) are admissible, and the
subformula property holds.  While we do not consider
focusing~\citep{andreoli92focus}, we conjecture that the connectives can
be given the same focusing behavior as in \citep{reed09adjoint}: $F$ is
positive and $U$ is negative (which, because limits are negative and
colimits are positive, and like-polarity connectives compose together
well, matches what left and right adjoints should preserve).

The resulting logic has a good definition-to-theorem ratio: from 
simple sequent calculus rules for $F$ and $U$, we can prove a variety of
general facts that are true for any mode 2-category ($F_\alpha$ and
$U_\alpha$ are functors; $F_\alpha U_\alpha$ is a comonad and $U_\alpha
F_\alpha$ is a monad; $F_\alpha$ preserves colimits and $U_\alpha$
preserves limits), as well as facts specific to a particular theory
(e.g. for the adjoint triple above, $\Gamma$ preserves both colimits
and limits, because it is an equivalent $U_\Delta$ and $F_\nabla$; the
comonad $\flat := \Delta\Gamma$ and monad $\sharp := \nabla\Gamma$ are
themselves adjoint).  Moreover, we can use different mode 2-categories
to add additional structure; for example, moving from the walking
adjunction to the walking reflection (taking $\Delta \nabla = 1$)
additionally gives that $\Delta$ and $\nabla$ are full and faithful and
that $\flat$ and $\sharp$ are idempotent, which are some of the
additional conditions for axiomatic cohesion.

We make a few simplifying restrictions for this paper. First, we
consider only single-hypothesis, single-conclusion sequents, deferring
an investigation of products and exponentials to future work.
Second, on the semantic side, we consider only 1-categorical semantics
of the derivations of the logic, rather than the $\infty$-groupoid
semantics that we are ultimately interested in.  More precisely, for any
2-category \M\/ of modes, we can interpret the logic using a
pseudofunctor $S : \M \to \Adj$, where $\Adj$ is the 2-category of
categories, adjunctions, and morphisms of adjunctions (conjugate pairs
of natural transformations).  By $S$, each mode determines a 1-category,
and derivations in the logic are interpreted as morphisms in these
categories.  The action of $S$ on 1- and 2-cells is used to interpret
$F$ and $U$.  We show that the syntax forms such a pseudofunctor, and
conjecture that the syntax is initial in some category or 2-category of
pseudofunctors, but have not yet tried to make this precise.

Third, we consider only a logic of simple-types, rather than a dependent
type theory.  Consequently, we do not have an identity type available
for proving equalities of proof terms.  However, we need an equational
theory to make many of the statements we would like to make (e.g. ``$UF$ is a monad''
require proving some equational laws), and the definitional equalities
arising from admissibility of cut and identity are not sufficient.
Thus, in addition to the sequent calculus itself, we give an equality
judgement on sequent calculus derivations.  This judgement is
interpreted by actual equality of morphisms in the semantics above, but
we intend some of these rules to be propositional equalities in an eventual
adjoint type theory.

In Section~\ref{sec:rules}, we define the rules of the logic, prove
admissibility of identity and cut, and define an equational theory on
derivations.  In Section~\ref{sec:semantics}, discuss the semantics of
the logic in pseudofunctors $\M \to \Adj$.  Finally, in
Section~\ref{sec:triple}, we examine some specific mode specifications
for adjoint triples, which are related to the rules for spatial type
theory used in \citep{shulman15realcohesion}.  While the main results
are summarized here, the extended version of this
paper\footnote{Available from
  \url{http://dlicata.web.wesleyan.edu/pubs.html}}  contains much more
discussion of the examples, definitions, and proofs.  All of the
syntactic metatheory of the logic and the examples have been formalized
in Agda~\citep{norell07thesis}.\footnote{See
  \url{github.com/dlicata335/hott-agda/tree/master/metatheory/adjointlogic}}

\section{Sequent Calculus and Equational Theory}
\label{sec:rules}

\subsection{Sequent Calculus}

The logic is parametrized by a strict 2-category of modes.  We write
$p,q$ for the 0-cells (modes), $\alpha,\beta,\gamma,\delta : p \ge q$
for the 1-cells, and $e : \tc \alpha \beta$ for the 2-cells.  We write
\compo{\beta}{\alpha} for 1-cell composition in function composition
order (i.e. if $\beta : r \ge q$ and $\alpha : q \ge p$ then
$\compo{\beta}{\alpha} : r \ge p$), \compv{e_1}{e_2} for vertical
composition of 2-cells in diagrammatic order, and \comph{e_1}{e_2} for
horizontal composition of 2-cells in ``congruence of \compo{}{}'' order
(if $e_1 : \tc \alpha {\alpha'}$ and $e_2 : \tc \beta \beta'$ then
$\comph{e_1}{e_2} :
\tc{\compo{\alpha}{\beta}}{\compo{\alpha'}{\beta'}}$).  The equations
for 2-cells say that \compv{}{} is associative with unit $1_\alpha$ for
any $\alpha$, that \comph{}{} is associative with unit $1_1$, and that
the interchange law $\comph{(\compv{e_1}{e_2})}{(\compv{e_3}{e_4})} =
\compv{(\comph{e_1}{e_3})}{(\comph{e_2}{e_4})}$ holds.  We think of the
mode category as being fixed at the outset, and the syntax and
judgements of the logic as being indexed by the 0/1/2-cells of this
category.
%% ; therefore,
%% equal morphisms in the mode 2-category automatically determine equal
%% propositions, judgements, and derivations.
%% An alternative would be to give a syntax and explicit equality
%% judgement for the mode category, which would be helpful if we needed
%% a mode theory where equality of morphisms or 2-morphisms were
%% undecidable.

Each object $p$ of the mode category determines a category, and types
represent objects of that category; the syntactic judgement \wftp{A}{p}
will mean that $A$ is an object of the category $p$.  A morphism $\alpha
: q \ge p$ in the mode category determines an adjunction between
categories $p$ and $q$, with $\F \alpha {} : q \to p$ and $\U \alpha {}
: p \to q$; syntactically, the action on objects is given by $\wftp{\F
  \alpha A}{p}$ when $\wftp{A}{q}$ and $\wftp{\U \alpha A}{q}$ when
$\wftp{A}{p}$.  We write $P$ for atomic propositions, each of which has
a designated mode.  To add additional structure to a category or to all
categories, we can add rules for additional connectives; for example, a
rule \wftp{\coprd{A}{B}}{p} if \wftp{A}{p} and \wftp{B}{p} (parametric
in $p$) says that any category $p$ has a coproduct type constructor.

The sequent calculus judgement has the form \seq A \alpha C where
\wftp{A}{q} and \wftp{C}{p} and $\alpha : q \ge p$.  The judgement
represents a map from an object of some category $q$ to an object of
another category $p$ along some adjunction $\F \alpha {} \la \U \alpha
{}$.  Semantically, this mixed-category map can be interpreted
equivalently as an arrow \arrow{p}{\F \alpha A}{C} or \arrow{q}{A}{\U
  \alpha C}.  In the rules, we write $A_p$ to indicate an elided premise
\wftp{A}{p}.  The rules for atomic propositions and for $U$ and $F$ are
as follows:
\[
\begin{array}{c}
\infer[\irl{hyp}]
      {\seq P \alpha P}
      {\tc 1 \alpha}
\quad
\infer[\irl{FL}]
      {\seq {\F {\alpha : r \ge q} A_r} {\beta : q \ge p}{C_p}}
      {\seq {A_r} {\compo{\alpha}{\beta}} {C_p}
      }
\quad
\infer[\irl{FR}]
      {\seq {C_r} {\beta : r \ge p} {\F {\alpha : q \ge p} A_q}}
      { \gamma : r \ge q & \tc{\compo{\gamma}{\alpha}}{\beta} &
        \seq {C_r} \gamma {A_q}}
\\ \\
\infer[\irl{UL}]
      {\seq {\U {\alpha : r \ge q} A_q} {\beta : r \ge p} {C_p}}
      { \gamma : q \ge p &
        \tc{\compo{\alpha}{\gamma}} {\beta} &
        \seq{A_q}{\gamma}{C_p}}
\quad
\infer[\irl{UR}]
      {\seq {C_r} {\beta : r \ge q} {\U {\alpha : q \ge p} A_p}}
      {\seq {C_r} {\compo{\beta}{\alpha}} {A_p}}
\end{array}
\]
The rules for other types do not change $\alpha$---e.g., see the rules
for coproducts in Figure~\ref{fig:coprod}.

\begin{figure}[t]
\[
\infer[\irl{Inl}]
      {\seq {C_q} {\alpha} {\coprd{A_p}{B_p}}}
      {\seq {C_q} {\alpha} {A_p}}
\quad
\infer[\irl{Inr}]
      {\seq {C_q} {\alpha} {\coprd{A_p}{B_p}}}
      {\seq {C_q} {\alpha} {B_p}}
\quad
\infer[\irl{Case}]
      {\seq {\coprd{A_q}{B_q}} {\alpha} {C_p}}
      {\seq {A_q} {\alpha} {C_p} & 
       \seq {B_q} {\alpha} {C_p} 
      }
\]

%% Definitions of \tr{-}{-} and \ident{-} and \cut{-}{-}:

%% \[
%% \begin{array}{rcl}
%% \tr{e}{\Inl{D}} & := &  \Inl{\tr e D}\\
%% \tr{e}{\Inr{D}} & := &  \Inr{\tr e D}\\
%% \tr{e}{\Case{D_1}{D_2}} & := & \Case{\tr e {D_1}}{\tr e {D_2}}\\\\

%% \ident{\coprd{A}{B}} & := & \Case {\Inl {\ident{A}}} {\Inr {\ident{B}}}\\\\
                                                                           
%% \cut{(\Inl{D})}{(\Case{E_1}{E_2})} & := & \cut{D}{E_1}\\
%% \cut{(\Inr{D})}{(\Case{E_1}{E_2})} & := & \cut{D}{E_2}\\
%% \cut{D}{(\Inl{E})} & := & \Inl{\cut{D}{E}}\\
%% \cut{D}{(\Inr{E})} & := & \Inr{\cut{D}{E}}\\
%% \cut{(\Case{D_1}{D_2})}{E} & := & \Case{\cut{D_1}{E}}{\cut{D_2}{E}} \text{if $E$ is not a right rule}\\
%% \end{array}
%% \]

%% Equational theory:
%% \[
%% \begin{array}{l}
%% \infer{\ap D {\Case{\cut{(\Inl{\ident{A}})}{D}}{\cut{(\Inr{\ident{B}})}{D}}}}
%%       {D : \seq{\coprd{A}{B}}{\alpha}{C}}
%% \\ \\ 
%% \infer{\ap{\Inl{\UL{\gamma}{e}{D}}}{\UL{\gamma}{e}{\Inl D}}}{}
%% \qquad
%% \infer{\ap{\Inr{\UL{\gamma}{e}{D}}}{\UL{\gamma}{e}{\Inr D}}}{}
%% \end{array}
%% \]

\caption{Rules for coproducts; see extended version for identity, cut,
  and equational theory}
\label{fig:coprod}
\end{figure}

To understand the rules for $F$ and $U$, it is helpful to begin with
\irl{FL} and \irl{UR}.  In the special case where $\beta$ is 1, these
rules pass from \seq{\F {\alpha}{A}}{1}{C} and \seq{A}{1}{\U{\alpha}{C}}
to \seq{A}{\alpha}{C}, which makes sense because the judgement
\seq{A}{\alpha}{C} is intended to mean either/both of these.  When
$\beta$ is not 1, these rules also express how
$\F{\compo{\alpha}{\beta}}{}$ and $\U{\compo{\beta}{\alpha}}$ distribute
over compositions (see below).  While we do not formally give a focused
sequent calculus, we conjecture that these two rules are
\emph{invertible}: whenever you have \F{\alpha}{A} on the left or
\U{\alpha}{A} on the right, you can immediately apply the rule, no
matter what is on the other side of the sequent.

On the other hand, the other two rules \irl{UL} and \irl{FR} cannot be
applied at any time, because they involve some constraints that may not
be satisfied.  Consider \irl{FR}: we have $\alpha : q \ge p$ and $\beta
: r \ge p$ and want to reduce proving \F{\alpha}{A} from $C$ to proving
$A$ from $C$.  However, there is not necessarily any relationship
between $C$'s mode $r$ and $A$'s mode $q$, because all we know is that
both of these are bigger than $p$.  Thus, to form a premise sequent, we
need to choose a $\gamma : r \ge q$.  To make adjunctions generated by
different morphisms be different, it is important that we choose not
just any $\gamma$, but one where the triangle that it forms with
$\alpha$ and $\beta$ is filled by a 2-cell, which is the second premise
of the rule.  The direction of the 2-cell, and the fact that it may not
be invertible, are chosen to match the intended semantics.  The case for
\irl{UR} is dual.

Because we are interested not only in provability, but also in the
equational theory of proofs in this logic, one might think the next step
would be to annotate the sequent judgement with a proof term, writing
e.g. $x : A [ \alpha ] \vdash M : B$.  However, the proof terms $M$
would have exactly the same structure as the derivations of this typing
judgement, so we instead use the derivations themselves as the proof
terms.  This corresponds to an ``intrinsic representation'' in Agda.  We
sometimes write $D : \seq{A}{\alpha}{B}$ to indicate ``typing'' in the
metalanguage; i.e. this should be read ``$D$ is a derivation tree of the
judgement \seq{A}{\alpha}{B}.''

We give some examples in Figure~\ref{fig:examples}; these examples and
many more like them are in the companion Agda code.  First, we give the
functorial action of $F$ on maps ($U$ is similar); note that because
\M\/ is a strict 2-category, the identity 2-cell has type
\tc{\compo{1_q} \alpha}{\compo {\alpha} {1_p}}.  Next, the composites
$FU$ and $UF$ should be a comonad and a monad respectively; define $\Bx
\alpha A := \F \alpha (\U \alpha A)$ and $\Crc{\alpha} :=
\F{\alpha}{(\U{\alpha}{A})}$ for any $\alpha : q \ge p$.  As an example
of the (co)monad structure, the comonad comultiplication is defined in
the figure.  An advantage of using a cut-free sequent calculus is that
we can observe some non-provabilities.  For example, there is not in
general a map \seq{P}{1_p}{\Bx{\alpha}{P}} (unit for the comonad): by
inversion, a derivation must begin with \irl{FR}, but to apply this
rule, we need a $\gamma : p \ge q$ and a 2-cell
$\tc{\compo{\gamma}{\alpha}}{1}$, which may not exist.  Next, we give
one half of the isomorphism showing that $F$ preserves coproducts.  The
key idea is that we can apply the left rule for $F$ first, and then
case-analyze the $\coprd{P}{Q}$, before choosing between \irl{Inl} and
\irl{Inr} on the right; this direction of map doesn't exist for
\U{\alpha}{(\coprd{P}{Q})} because the left rule for $U$ cannot be
applied until after applying \irl{UR}.  Next, for identity and
composition of 1-cells, an obvious question is the relationship between
$\F 1 A$ and $\U 1 A$ and $A$, between $\F {\compo{\beta}{\alpha}} A$
and $\F \alpha {(\F \beta A)}$, and between $\U {\compo{\beta}{\alpha}}
A$ and $\U \beta {(\U \alpha A)}$.  We do not have definitional
equalities of types, but the types in each group are isomorphic.  For
example, the maps for $F$ on compositions are in the figure.  As a final
example, for every 2-cell $e : \tc \alpha \beta$, we have derivations of
\seq{\F \beta P}{1}{\F \alpha P} and \seq{\U \alpha P}{1}{\U \beta P}
that are an adjunction morphism between $F_\beta \la U_\beta$ and
$F_\alpha \la U_\alpha$.

\begin{figure}
Functoriality, given $\alpha : q \ge p$ and $D : \seq{A}{1_q}{B}$:
\[
\infer[\irl{FL}]
      {\seq{\F \alpha A}{1_p}{\F \alpha B}}
      {\infer[\irl{FR}]
        {\seq{A}{\compo{\alpha} {1_p}}{\F \alpha B}}
        {\infer{1_q : q \ge q}{} & \infer{1 : \tc{\compo{1_q} \alpha}{\compo {\alpha} {1_p}}}{} & D : \seq{A}{1_q}{B}}}
\]

Comonad comultiplication:
\[
\infer[\irl{FL}]{\seq{\Bx{\alpha}{P}}{1}{\Bx{\alpha}{\Bx{\alpha}{P}}}}
      {\infer[\irl{FR}]{\seq{\U{\alpha}{P}}{\alpha}{\Bx{\alpha}{\Bx{\alpha}{P}}}}
        {\infer{1 : q \ge q}{} & 
          \infer{1 : \tc {\alpha} {\alpha}}{} & 
          \infer[\irl{UR}]
                {\seq{\U{\alpha}{P}}{1}{\U {\alpha} {\Bx \alpha P}}}
                {\infer[\irl{FR}]{\seq{\U{\alpha}{P}}{\alpha}{\Bx \alpha P}}
                  {\infer{1 : q \ge q}{} &
                   \infer{1 : \tc \alpha \alpha}{} &
                   \infer[\irl{UR}]{\seq{\U{\alpha}{P}}{1}{\U \alpha P}}
                         {\infer[\irl{UL}]{\seq{\U \alpha P}{\alpha}{P}}
                           {\infer{1 : p \ge p}{} & \infer{1 : \tc {\alpha} {\alpha}}{} &
                             \infer[\irl{hyp}]{\seq{P}{1}{P}}{\infer{\tc{1}{1}}{}}
                         }}}}}}
\]

$F$ preserves coproducts (one half of a natural isomorphism):
\[
\infer[\irl{FL}]{\seq{\F \alpha {(\coprd{P}{Q})}}{1}{\coprd{\F \alpha P}{\F \alpha Q}}}
      {\infer[\irl{Case}]
        {\seq{\coprd{P}{Q}}{\alpha}{\coprd{\F \alpha P}{\F \alpha Q}}}
        {\infer[\irl{Inl}]
          {\seq{P}{\alpha}{\coprd{\F \alpha P}{\F \alpha Q}}}
          {\infer[\irl{FR}]
                 {\seq{P}{\alpha}{\F \alpha P}}
                 {\infer{1 : q \ge q}{} & \infer{1 : \tc{\alpha}{\alpha}}{} & \infer[\irl{\hyp 1}]{\seq{P}{1}{P}}{}}}
          &
         \infer[\irl{Inr}]
          {\seq{Q}{\alpha}{\coprd{\F \alpha P}{\F \alpha Q}}}
          {\infer[\irl{FR}]
                 {\seq{Q}{\alpha}{\F \alpha Q}}
                 {\infer{1 : q \ge q}{} & \infer{1 : \tc{\alpha}{\alpha}}{} & \infer[\irl{\hyp 1}]{\seq{Q}{1}{Q}}{}}}
         }}
\]


$F$ on $\compo{\beta}{\alpha}$:
\[
\infer[\irl{FL}]
      {\seq{\F {\compo{\beta}{\alpha}} P}{1}{\F \alpha {(\F \beta P)}}}
      {\infer[\irl{FR}]
             {\seq{P}{\compo{(\compo{\beta}{\alpha})}{1}}{\F \alpha {(\F \beta P)}}}
             {{\beta : r \ge p} & 
              \infer{1 : \tc{\compo{(\compo{\beta}{\alpha})}{1}}{\compo \beta \alpha}}{} & 
              \infer{\seq{P}{\beta}{\F \beta P}}
                    {\infer{1 : r \ge r}{} &
                      \infer{1 : \tc{\compo{1}{\alpha}}{\alpha}}{} &
                      \infer[\irl{\hyp 1}]{\seq{P}{1}{P}}{}
                    }
             }}
\]
\[
\infer[\irl{FL}]
      {\seq{\F \alpha {(\F \beta P)}}{1}{\F {\compo{\beta}{\alpha}} P}}
      {\infer[\irl{FL}]
             {\seq{\F \beta P}{\compo{\alpha}{1}}{\F {\compo{\beta}{\alpha}} P}}
             {\infer[\irl{FR}]
               {\seq{P}{\compo{\beta}{(\compo{\alpha}{1})}}{\F {\compo{\beta}{\alpha}} P}}
               {\infer{1 : r \ge r}{} & 
                 \infer{1 : \tc {\compo{1}{(\compo{\beta}{\alpha})}} {\compo{\beta}{(\compo{\alpha}{1})}}}{} &
                 \infer[\irl{\hyp 1}]{\seq{P}{1}{P}}{}
               }}}
\]

$F/U$ on 2-cells:
\[
\infer[\irl{FL}]
      {\seq{\F \beta P}{1}{\F \alpha P}}
      {\infer[\irl{FR}]{\seq{P}{\beta}{\F \alpha P}}
                       {\infer{1 : q \ge q}{} & {e : \tc{\compo 1 \alpha}{\beta}} & \infer[\hyp{1}]{\seq{P}{1}{P}}{} }{}}
\qquad
\infer[\irl{UR}]
      {\seq{\U \alpha P}{1}{\U \beta P}}
      {\infer[\irl{UL}]
             {\seq{\U \alpha P}{\beta}{P}}
             { \infer{1 : p \ge p}{} & e : \tc{\compo{\alpha}{1}}{\beta} & \infer[\hyp 1]{\seq{P}{1}{P}}{}}}
\]

\caption{Some examples}
\label{fig:examples}
\end{figure}

\subsection{Admissible Rules}

\paragraph{Adjunction morphisms}

The following rule provides the action of a 2-cell $e : \tc \alpha
\beta$ on a derivation $D : \seq{A}{\alpha}{B}$, yielding a new
derivation which we write as $\tr{e}{D} : \seq{A}{\beta}{B}$.
Semantically, $D$ is interpreted as a map \arrow{}{\F \alpha A}{B}
(say), so to get a map \arrow{}{\F \beta A}{B} we can precompose with
the $F$ part of the morphism of adjunctions determined by $e$.

\[
\infer[\irl{\tr{-}{-}}]
      {\seq A {\beta} C}
      {\tc \alpha \beta &
       \seq A {\alpha} {C}}
\]
This rule is defined as a transformation on derivations as follows:
\[
\begin{array}{rcl}
  \tr {e}{\hyp e'} & := & \hyp {(\compv{e'}{e})}\\
  \tr {e}{\FR \gamma {e'} D} & := & \FR \gamma {\compv{e'}{e}} D \\
  \tr {e}{\FL D} & := &\FL {\tr{(\comph{1}{e})} D}\\
  \tr {e}{\UL \gamma {e'} D} & := & \UL \gamma {\compv{e'}{e}} D\\
  \tr {e}{\UR D} & := & \UR {\tr {(\comph{e}{1})} D} \\
\end{array}
\]
%
The hypothesis rule and \irl{FR} and \irl{UL} build in some movement
along a 2-cell, so in those cases we compose the $e$ with the 2-cells
that are already present.  For \irl{FL} and \irl{UR} (and for the rules
for coproducts), the operation commutes with the rule.

\paragraph{Identity}

The identity rule is admissible:
\[
\infer[\irl{ident}]
      {\seq {A_p} {1} {A_p}}
      {}
\]
The general strategy is ``apply the invertible rule and then the focus
rule and then the inductive hypothesis.'' 
%% For example, for $\F \alpha
%% A$, the following reduces the problem to identity on $A$:
%% \[
%% \infer[\irl{FL}]
%%       {\seq{\F {\alpha : q \ge p} A}{1}{\F \alpha A}}
%%       {\infer[\irl{FR}]
%%              {\seq{A}{\alpha}{\F \alpha A}}
%%              {1_q : q \ge q & 1 : \tc{\compo{1}{\alpha}}{\alpha} &
%%                \seq{A}{1}{A}}}
%% \]
As a function from types to derivations, we have
\[
\begin{array}{rcl}
  \ident{P} & := & \hyp 1\\
  \ident{\U \alpha A} & := & \UR {\UL 1 1 {\ident A}}\\
  \ident{\F \alpha A} & := & \FL {\FR 1 1 {\ident A}}\\
\end{array}
\]

\paragraph{Cut}

The following cut rule is admissible:
\[
\infer[\irl{cut}]
      {\seq {A_r} {\compo{\beta}{\alpha}} {C_p}}
      {\seq {A_r} {\beta} {B_q} &
       \seq {B_q} {\alpha} {C_p}}
\]
For example, consider the principal cut for $F$:
\[
\infer[\irl{cut}]
      {\seq {A} {\compo{\beta}{\alpha}} {C}}
      {\infer[\irl{FR}]
             {\seq {A} {\beta} {\F {\alpha_1} B}}
             {e : \tc{\compo{\gamma}{\alpha_1}}{\beta} & 
              D : \seq {A} {\gamma} {B}} &
       \infer[\irl{FL}]
             {\seq {\F {\alpha_1} B} {\alpha} {C}}
             {E : {\seq{B}{\compo{\alpha_1}{\alpha}}{C}}}}
\]
In this case the cut reduces to
\[
\infer[\irl{\tr{-}{-}}]
      {\seq{A}{\compo{\beta}{\alpha}}{C}}
      {\comph{e}{1} : \tc {\compo{(\compo{\gamma}{\alpha_1})}{\alpha}} {\compo{\beta}{\alpha}} &
        \infer[\irl{cut}]
              {\seq{A}{\compo{\compo{\gamma}{\alpha_1}}{\alpha}}{C}}
              {D : \seq{A}{\gamma}{B} &
                E : \seq{B}{\compo{\alpha_1}{\alpha}}{C}}}
\]

As a transformation on derivations, we have
\[
\begin{array}{rcll}
  \cut {(\hyp e)} {(\hyp {e'})} & := & \hyp {(\comph{e}{e'})}\\
  \cut {(\FR \gamma e D)} {(\FL E)} & := & \tr {(\comph{e}{1})} {\cut D E}\\
  \cut {(\UR D)} {(\UL \gamma e E)} & := & \tr {(\comph{1}{e})} {\cut D E}\\
  \cut D {(\FR \gamma e E)} & := & \FR {\compo{\beta}{\gamma}} {\comph{1}{e}} {\cut D E}\\
  \cut D {(\UR E)} & := & \UR {\cut D E} \\
  \cut {(\FL D)} E & := & \FL {\cut D E} & \text{if $E$ is not a right rule} \\
  \cut {(\UL \gamma e D)} E & := & \UL {\compo{\gamma}{\alpha}} {\comph{e}{1}} {\cut D E} & \text{if $E$ is not a right rule}
\end{array}
\]
The first case is for atomic propositions.  The next two cases are the
principal cuts, when a right rule meets a left rule; these correspond to
$\beta$-reduction in natural deduction.  The next two cases are
right-commutative cuts, which push any $D$ inside a right rule for $E$.
The final two cases are left commutative cuts, which push any $E$ inside
a left rule for $D$.  The left-commutative and right-commutative cuts
overlap when $D$ is a left rule and $E$ is a right rule; we give
precedence to right-commutative cuts definitionally, but using the
equational theory below, we will be able to prove the general
left-commutative rules.

As an example using identity and cut, we give one of the maps from the
bijection-on-hom-sets adjunction for $F$ and $U$: given $\alpha : q \ge
p$ we can transform $D : \seq { \F \alpha A}{1}{B}$ into {\seq{A}{1}{\U
    \alpha B}}:
\[
\infer[\irl{UR}]{\seq{A}{1}{\U \alpha B}}
      {\infer[\irl{cut}]
             {\seq{A}{\alpha}{B}}
             {\infer[\irl{FR}]
                    {\seq{A}{\alpha}{\F \alpha A}}
                    {\infer{1 : q \ge q}{} & \infer{1 : \tc{\alpha}{\alpha}}{} & \infer[\irl{ident}]{\seq{A}{1}{A}}{} } & 
               D : \seq { \F \alpha A}{1}{B} }}
\]


\subsection{Equations}
\label{sec:rules:equations}

When we construct proofs using the admissible rules \tr{e}{D} and
\ident{A} and \cut{D}{E}, there is a natural notion of definitional
equality induced by the above definitions of these operations (e.g.
$\ident{\U \alpha A}$ is definitionally equal to $\UR {\UL 1 1 {\ident
    A}}$)---the cut- and identity-free proofs are normal forms, and a
proof using cut or identity is equal to its normal form.  However, to
prove the desired equations in the examples below, we will need some
additional ``propositional'' equations, which, because we are using
derivations as proof terms, we represent by a judgement \ap{D}{D'} on
two derivations $D,D' : \seq{A}{\alpha}{C}$.  This judgement is the
least congruence closed under the following rules.  First, we have
uniqueness/$\eta$ rules.  The rule for $F$ says that any map from
\F{\alpha}{A} is equal to a derivation that begins with an application
of the left rule and then cuts the original derivation with the right
rule; the rule for $U$ is dual.

\[
\begin{array}{l}
\infer[\irl{F\eta}]
      {\ap{D}{\FL {\cut{(\FR 1 1 {\ident{A}})}{D}} }}
      {D : \seq{\F \alpha A}{\beta}{C}}
\quad
\infer[\irl{U\eta}]
      {\ap{D}{\UR {\cut{D}{(\UL 1 1 {\ident{A}})}}}}
      {D : \seq{C}{\beta}{\U \alpha A}}
\end{array}
\]

Second, we have rules arising from the 2-cell structure.  For example,
suppose we construct a derivation by $\FR{\gamma}{e}{D}$ for some
$\gamma : r \ge q$ and $e : \tc {\compo{\gamma}{\alpha}}{\beta}$, but
there is another morphism $\gamma' : r \ge q$ such that there is a
2-cell between $\gamma$ and $\gamma'$.  The following says that we can
equally well pick $\gamma'$ and suitably transformed $e$ and $D$, 
using composition and \tr{e_2}{-} to make the types match up.  
\[
\infer{\ap{\FR{\gamma}{e}{\tr{e_2}{D'}}}{\FR{\gamma'}{(\compv{(\comph{e_2}{1})}{e})}{D'}}}
      {e : \tc{\compo{\gamma}{\alpha}}{\beta} & 
       D : \seq{C}{\gamma'}{A} &
       e_2 : \tc{\gamma'}{\gamma} & }
\quad
\infer{\ap{\UL{\gamma}{e}{\tr{e_2}{D'}}}{\UL{\gamma'}{(\compv{(\comph{1}{e_2})}{e})}{D'}}}
      {e : \tc{\compo{\gamma}{\alpha}}{\beta} & 
       D : \seq{C}{\gamma'}{A} &
       e_2 : \tc{\gamma'}{\gamma} & }
\]
Semantically, these rules will be justified by some of the pseudofunctor
laws.

The final rules say that left rules of negatives and right rules of
positives commute. These are needed to prove the left-commutative cut
equations in the case where $E$ is a right rule, which seem necessary
for showing that cut is unital and associative.  For $U$ and $F$, we
have
\[
\infer{\ap{\UL {}{e_2} {\FR {} {e_1} {D}}}{\FR {} {e_4} {\UL {} {e_3} {D}}}}
      {\compv{(\comph{1}{e_1})}{e_2} = {\compv{(\comph{e_3}{1})}{e_4}}}
\]

\subsubsection{Admissible rules}

The following equality rules are admissible for logic containing the
$U/F$ rules described above and the coproduct rules in
Figure~\ref{fig:coprod}.  The proofs have a lot of cases (about 800
lines of Agda total) but are not difficult, except for somewhat subtle
staging.  The rules in each of the following lines (except the first)
are proved by mutual induction, and use the preceding lines:

\[
\begin{array}{c}
\infer{\tr{1}{D} = D}{}
\qquad
\infer{\tr{(\compv{e_1}{e_2})}{D} = \tr{e_2}{\tr{e_1}{D}}}
      {}
\\ \\
\infer{\ap{\tr{e}{D}}{\tr{e}{D'}}}
      {\ap{D}{D'}}
\qquad
\infer{\ap{\tr{(\comph{e}{e'})}{\cut{D}{D'}}}{\cut{(\tr{e}{D})}{(\tr{e'}{D'})}}}
      {e : \tc{\alpha}{\alpha'} &
       e' : \tc{\beta}{\beta'} &
       D : \seq{A}{\alpha}{B} &
       D' : \seq{B}{\beta}{C}}
\\ \\
\infer{\ap{\cut{D_1}{(\cut{D_2}{D_3})}}{\cut{(\cut{D_1}{D_2})}{D_3}}}
      {}
\\ \\
\infer{\ap{\cut{D}{\ident{}}}{D}}
      {}
\quad
\infer{\ap{\cut{\ident{}}{D}}{D}}
      {}
\quad
\infer{\ap{\cut{D}{E}}{\cut{D'}{E}}}
      {\ap{D}{D'}}
\quad
\infer{\ap{\cut{D}{E}}{\cut{D}{E'}}}
      {\ap{E}{E'}}
\\ \\
\infer{\ap{\cut {(\FL D)} E} {\FL {\cut D E}}}{}
\quad 
\infer{\ap{\cut {(\UL \gamma e D)} E} {\UL {\compo{\gamma}{\alpha}} {\comph{e}{1}} {\cut D E}}}{}
\end{array}
\]

\section{Semantics}
\label{sec:semantics}

In the extended version of this paper, we give a detailed account of the
following result:

\begin{theorem}[Soundness and Completeness]
For any mode theory \M, the rules of adjoint logic can be interpreted in
any pseudofunctor $\M \to \Adj$, where $\Adj$ is the 2-category of
categories, adjunctions, and conjugate natural transformations.
Moreover, there is a pseudofunctor $\M \to \Adj$ determined by the
syntax.
\end{theorem}
A pseudofunctor is a map between 2-categories that preserves identity
and composition of 1-cells up to coherent isomorphism, rather than on
the nose. In this case, we have a pseudofunctor because $\F 1 A \iso A
\iso \U 1 A$ and $\F{\compo{\beta}{\alpha}}{A} \iso
\F{\alpha}{\F{\beta}{A}}$ and $\U{\compo{\beta}{\alpha}}{A} \iso
\U{\beta}{\U{\alpha}{A}}$, but these are not equalities of types.  Here
we expand the statement of completeness.

%% 
%% Some standard category-theoretic terminology
%% applied to categories of this form unpacks as follows:
%% \begin{enumerate}
%% \item For $\wftp{A,B}{p}$, an isomorphism $A \iso B$ consists of a pair $D :
%% \seq{A}{1}{B}$ and $E : \seq{B}{1}{A}$ such that
%% $\ap{{D} \cuti {E}}{\ident{A}}$ and $\ap{{E} \cuti {D}}{\ident{B}}$.

%% \item For modes $p$ and $q$, a functor from $p$ to $q$ consists of 
%% a function $G_0$ from types with mode $p$ to types with mode $q$ 
%% and a function $G_1$ from derivations \seq{A}{1}{B} to 
%% derivations \seq{G_0 \, A}{1}{G_0 \, B}, such that
%% $\ap {G_1(\ident{A})}  {\ident{G_0 A}}$
%% and
%% $\ap {G_1(D \cuti E)}  {G_1(D) \cuti G_1(E)}$.

%% \item For two functors $G,H : p \to q$, a natural transformation $t : G
%%   \to H$ consists of a family of derivations $D_A :
%%   \seq{G_0(A)}{1}{H_0(A)}$ for each \wftp{A}{p}, such that for any $D :
%%   \seq{A}{1_p}{B}$, \ap{\cut{D_A}{(H_1(D))}}{\cut{(G_1(D))}{D_B}}.  A
%%   natural isomorphism consists of a natural transformation along with
%%   inverses demonstrating that each $D_A$ is an isomorphism $G_0(A) \iso
%%   H_0(A)$.  

%% \item For two functors $L,R : p \to q$, an adjunction $L \la R$ (using
%%   the natural-bijection-of-hom-sets definition) consists of functions
%%   $\ltor{-}{} : (\seq{L_0(A)}{1}{B}) \, \to \, (\seq{A}{1}{R_0(B)})$ and
%%   $\rtol{-}{} : (\seq{A}{1}{R_0(B)}) \, \to \, (\seq{L_0(A)}{1}{B})$ which are
%%   mutually inverse and such that \ltor{-}{} is natural in $A$ and $B$: for
%%   $D_1 : \seq{A'}{1}{A}$ and $D_3 : \seq{B}{1}{B'}$ and $D_2 :
%%   \seq{L_0(A)}{1}{B}$, \ap{\ltor{(L_1(D_1) \cuti D_2 \cuti D_3)}{}}{D_1
%%     \cuti \ltor{D_2}{} \cuti {R_1(D_3)}} (it follows that \rtol{-}{} is
%%   natural as well).  

%% \item A morphism of adjunctions (or ``adjunction morphism'') from $L^1
%%   \la_1 R^1$ to $L^2 \la_2 R^2$ consists of two natural transformations
%%   $t^L : L^1 \to L^2$ and $t^R : R^2 \to R^1$ between the corresponding
%%   functors that are ``conjugate'' under the adjunction
%%   structure~\citep[\S IV.7]{maclane98working}.
%%   This means that for any $D : \seq{L^2(A)}{1}{B}$ we have
%%   \[
%%   \ltor{(t^L \cuti D)}{1} = \ltor{D}{2} \cuti t^R.
%%   \]
%%   An adjunction isomorphism consists of an adjunction morphism together
%%   with inverses showing that $t^L$ and $t^R$ are each natural
%%   isomorphisms.\footnote{This definition is equivalent to ``two inverse
%%     adjunction morphisms,'' similarly to how isomorphisms that are
%%     natural are the same as iso-natural transformations---we can recover
%%     the conjugation condition for one direction from the other.}

%% \item Because we treat equality of morphisms as
%%   propositional/proof-irrelevant, two adjunction morphisms $(t^L,t^R)$
%%   and $({u^L},{u^R}')$ between the same two adjunctions are equal iff
%%   $\ap{{t^L}_A}{{u^L}_A}$ and $\ap{{t^L}_A}{{u^L}_A}$
%%   for all $A$.
%% \end{enumerate}
%% %
%% While these definitions are ``external'' (meta-theoretic), we are
%% hopeful that it would be possible to internalize them in a dependent
%% type theory based on adjoint logic.  For example, although the above
%% definition allows a functor to be given by arbitrary meta-theoretic
%% functions, in all of the examples we consider, the action on objects is
%% in fact given by a syntactic type with a ``placeholder'', and the action
%% on morphisms is given by taking an assumed derivation $D$ and applying
%% rules to it.  Similarly, all of the equalities are proved by chaining
%% together the equality rules (including the admissible ones, such as
%% associativity and identity of cut) from the previous section.

\begin{theorem}[Syntax Determines a Pseudofunctor] \label{thm:syntacticpseudofunctor}
The syntax of adjoint logic determines a pseudofunctor $\M \to \Adj$:
\begin{enumerate}
\item An object $p$ of \M is sent to the category whose objects are
  \wftp{A}{p} and morphisms are morphisms are derivations of
  \seq{A}{1_p}{B} quotiented by $\ap{}{}$, with identities given by
  \ident{} and composition given by \cutsym.

\item For each $q,p$, there is a functor from the category of morphisms
  $q \ge p$ to the category of adjoint functors between $q$ and $p$.
  \begin{itemize}
  \item 
  Each $\alpha : q \ge p$ is sent to $F_\alpha \la U_\alpha$ in
  \Adj---$F_\alpha$ and $U_\alpha$ are functors and they are adjoint.

  \item Each 2-cell $e : \tc{\alpha}{\beta}$ is sent to an adjunction
    morphism $(F(e),U(e)) : (F_\beta \la U_\beta) \to (F_\alpha \la
    U_\alpha)$, and this preserves $1$ and $\compv{e_1}{e_2}$.
  \end{itemize}

\item $\F 1 A \iso A$ and $\U 1 A \iso A$ naturally in $A$, and these
  are conjugate, so there is an adjunction isomorphism $P^1$ between $\F 1 {}
  \la \U 1 {}$ and the identity adjunction.

\item $\F {\compo{\beta}{\alpha}} A \iso \F \alpha {(\F \beta A)}$ and
  $\U {\compo{\beta}{\alpha}} A \iso \U \beta {(\U \alpha A)}$ naturally
  in $A$, and these are conjugate, so there is an adjunction isomorphism
  $P^{\circ}(\alpha,\beta)$ between $\F {\compo{\beta}{\alpha}} \la \U
  {\compo{\beta}{\alpha}}$ and the composition of the adjunctions $\F
  {\alpha} \la \U {\alpha}$ and $\F {\beta} \la \U {\beta}$.  Moreover,
  this family of adjunction isomorphisms is natural in $\alpha$ and
  $\beta$.

\item Three coherence conditions between these identity and composition
  isomorphisms are satisfied.
\end{enumerate}
\end{theorem}

\begin{proof}
We have given a flavor for some of the maps in the examples above; the
complete construction is about 500 lines of Agda.  There are many
equations to verify---inverses, naturality, conjugation, and
coherence---but they are all true for \ap{}{}.
\end{proof}

Next, we summarize some constructions on $F_\alpha \la U_\alpha$ that
can be made in the logic. We write $D \cuti E$ as an infix notation for
\cut{D}{E} (composition in diagrammatic notation).

\begin{lemma}[Some constructions on adjunctions] \label{lem:constructionsonadjunctions}
Let $\alpha : q \ge p$.  Then:
\begin{enumerate}
\item The composite functor $\Bx{\alpha}{A} := \F{\alpha}{\U{\alpha}{A}}$ is a comonad:
\begin{itemize}
\item[] $\dsd{counit} : \seq{\Bx \alpha A}{1}{A}$ naturally in $A$
\item[] $\dsd{comult} : \seq{\Bx \alpha A}{1}{\Bx \alpha {\Bx \alpha A}}$
  naturally in $A$
\item[] 
 \ap{\dsd{comult} \cuti (\Bx{}{\dsd{comult}})}{\dsd{comult} \cuti
   \dsd{comult}}
and \ap{\dsd{comult} \cuti \dsd{counit}}{\ident{}} \\
and \ap{\dsd{comult} \cuti (\Bx{}{\dsd{counit}})}{\ident{}}.  
\end{itemize}

\item The composite functor $\Crc{\alpha}{A} := \U{\alpha}{\F{\alpha}{A}}$ is a monad:
\begin{itemize}
\item[] $\dsd{unit} : \seq{A}{1}{\Crc \alpha A}$ naturally in $A$
\item[] $\dsd{mult} : \seq{\Crc \alpha {\Crc \alpha A}}{1}{\Crc \alpha A}$
  naturally in $A$
\item[] 
 \ap{(\Crc{}{\dsd{mult}}) \cuti \dsd{mult} }{\dsd{mult} \cuti
   \dsd{mult}}
and \ap{\dsd{unit} \cuti \dsd{mult}}{\ident{}} \\
and \ap{(\Crc{}{\dsd{unit}}) \cuti \dsd{mult}}{\ident{}}.  
\end{itemize}

\item $F$ preserves coproducts: $\F \alpha (\coprd A B) \iso \coprd {\F
  \alpha A} {\F \alpha B}$ naturally in $A$ and $B$.
\end{enumerate}
\end{lemma}

\begin{proof}
We showed some of the maps above; the (co)monad laws, naturality
conditions, and inverse laws are all true for \ap{}{}; the construction
is about 150 lines of Agda.
\end{proof}

%% \section{Semantics}
%% \label{sec:semantics}

%% \newcommand\semF[2]{\ensuremath{\mathcal{F}_{#1} \,\, #2}}
%% \newcommand\semU[2]{\ensuremath{\mathcal{U}_{#1} \,\, #2}}
%% \newcommand\semFo[1]{\ensuremath{{\mathcal{F}_{#1}}}}
%% \newcommand\semUo[1]{\ensuremath{\mathcal{U}_{#1}}}
%% \newcommand\semFone{\ensuremath{{\mathcal{F}^1}}}
%% \newcommand\semUone{\ensuremath{{\mathcal{U}^1}}}
%% \newcommand\semUcomp{\ensuremath{\mathcal{U}^\circ}}
%% \newcommand\semFcomp{\ensuremath{\mathcal{F}^\circ}}
%% \newcommand\semltor[2]{\ensuremath{#1^{\vartriangleright_{#2}}}}
%% \newcommand\semrtol[2]{\ensuremath{#1^{\vartriangleleft_{#2}}}}

%% \newcommand\seminl[0]{\ensuremath{\mathit{inl}}}
%% \newcommand\seminr[0]{\ensuremath{\mathit{inr}}}
%% \newcommand\semcase[2]{\ensuremath{[#1,#2]}}

%% Next, we show that we can interpret the rules of adjoint logic in any
%% pseudofunctor $S : \M \to \Adj$.  This shows that the syntax is sound
%% for these models.  On the semantic side, we unpack the definition of a
%% pseudofunctor as follows:
%% \begin{itemize}
%% \item We write $\C_p$ for $S(p)$.  We write ``$;$'' for composition of
%%   morphisms in $\C_p$ in diagrammatic order.

%% \item We write $\semFo{\alpha}{} \la \semUo{\alpha}$ for $S(\alpha)$,
%%   and \semltor{-}{\alpha} and \semrtol{-}{\alpha} for the two maps of
%%   hom-sets of the adjunction.  Naturality of the maps of hom-sets says
%%   that $\semltor{(\semF{\alpha}{m_1};m_2;m_3)}{\alpha} =
%%   m_1;\semltor{m_2}{\alpha};\semU{\alpha}{m_3}$ and
%%   $\semrtol{(m_1;m_2;\semU{\alpha}{m_3})}{\alpha} =
%%   \semF{\alpha}{m_1};{\semrtol{m_2}{\alpha}};{m_3}$.

%% \item We write $\semFo{e} : \arrow{}{\semFo \beta A}{\semFo \alpha A}$
%%   and $\semUo{e} : \arrow{}{\semUo \alpha A}{\semUo \beta A}$ when {e :
%%     \tc{\alpha}{\beta}} for the components of the two natural
%%   transformations in the adjunction morphism $S(e) :
%%   \arrow{}{\semFo{\beta}{} \la \semUo{\beta}} {\semFo{\alpha}{} \la
%%     \semUo{\alpha}}$.  Functoriality gives that $\semFo{\compv{e}{e'}} =
%%   \semFo{e'};\semFo{e}$ and $\semFo{1} = 1$ and $\semUo{\compv{e}{e'}} =
%%   \semUo{e};\semUo{e'}$ and $\semUo{1} = 1$.  The conjugation property 
%%   specifies that for any $m :
%%   \arrow{}{\semF{\alpha}{A}}{B}$, we have $\semltor{m}{\alpha};\semUo{e}
%%   =_{\arrow{}{A}{\semU{\beta}{B}}} \semltor{(\semFo{e};m)}{\beta}$, or
%%   equivalently that for any $m : \arrow{}{{A}}{\semU{\alpha} B}$, we
%%   have $\semFo{e};\semrtol{m}{\alpha} =_{\arrow{}{\semF \beta A}{B}}
%%   \semrtol{(m;\semUo{e})}{\beta}$.

%% \item We write $\semFone : \semF{1}{A} \iso A$ and $\semUone :
%%   \semU{1}{A} \iso A$ for the components of the two natural isomorphisms
%%   in the adjunction isomorphism $S^1$ between $\semFo{1}{} \la
%%   \semUo{1}$ and the identity adjunction.  The conjugation property
%%   specifies that for any $m : \arrow{}{A}{B}$, $\ltor{(\semFone_A;m)}{1}
%%   = m;\inv{\semUone_B}$, and for any $m : \arrow{}{A}{B}$,
%%   $\semrtol{(m;\inv{\semUone})}{1} = \semFone;m$.  In particular, taking
%%   $m = 1$ in the former, $\ltor{(\semFone_A)}{{1}} =
%%   \inv{(\semUone_A)}$.

%% \item We write $\semFcomp(\beta,\alpha) : \semF{\compo{\beta}{\alpha}}{A} \iso
%%   \semF{\alpha}{(\semF \beta A)}$ and $\semUcomp(\beta,\alpha) :
%%   \semU{\compo{\beta}{\alpha}}{A} \iso \semU{\beta}{(\semU{\alpha}{A})}$
%%   for the components of the two natural isomorphisms in the 
%%   natural adjunction isomorphism $S^\circ$ between $\semFo{\compo{\beta}{\alpha}}
%%   \la \semUo{\compo{\beta}{\alpha}}$ and the composition of the
%%   adjunctions $\semFo{\alpha} \la \semUo{\alpha}$ and
%%   $\semFo{\beta} \la \semUo{\beta}$.  
%%   Naturality in $\alpha,\beta$ means that 
%%   for any 
%%   $e_1 : \tc{\beta}{\beta'}$ and $e_2 : \tc{\alpha}{\alpha'}$, 
%%   we have
%%   $ \semFo{\comph{e_1}{e_2}};\semFcomp(\beta,\alpha) =_{\arrow{}{\semF{\compo{\beta'}{\alpha'}}{A}}{\semF{\alpha}{\semF{\beta}{A}}}}
%%   \semFcomp(\beta',\alpha');\semF{\alpha'}{\semFo{e_1}};{\semFo{e_2}}$,
%%   or equivalently 
%%   $\semFcomp(\beta',\alpha');{\semFo{e_2}};\semF{\alpha}{\semFo{e_1}}$,
%%   and similarly for $\semU{}{}$.
%%   The conjugation property specifies that 
%%   $\semrtol{(m;\inv{\semUcomp(\beta,\alpha)})}{\compo{\beta}{\alpha}} = \semFcomp(\beta,\alpha);\semrtol{{\semrtol{m}{\beta}}}{\alpha}$
%%   and similarly for \semltor{-}{}.  


%% \item Using the fact that both $\M$ and $\Adj$ are strict 2-categories,
%%   and unpacking the definition of horizontal composition of natural
%%   transformations, the three coherence laws relating $S^\circ$ and $S^1$
%%   specify the following: 
%%   \begin{itemize}
%%   \item 
%%     $\inv{\semFcomp(1,\alpha)_A}
%%     =_{\arrow{}{\semF{\alpha}{\semF{1}{A}}}{\semF{\alpha}{A}}}
%%     \semF{\alpha}{(\semFone_A)}$
%%   \item $\inv{\semFcomp(\alpha,1)_{A}} =_{\arrow{}{\semF{1}{\semF{\alpha}{A}}}{\semF{\alpha}{A}}} \semFone_{(\semF{\alpha}{A})}$
%%   \item $\inv{\semUcomp(\alpha,1)_A}
%%     =_{\arrow{}{\semU{\alpha}{\semU{1}{A}}}{\semU{\alpha}{A}}}
%%     \semU{\alpha}{(\semUone_A)}$
%%   \item $\inv{\semUcomp(1,\alpha)_A} =_{\arrow{}{\semU{1}{\semU{\alpha}{A}}}{\semU{\alpha}{A}}} {(\semUone_{\semU{\alpha}{A}})}$
%%   \item
%%     $\semFcomp(\gamma,\compo{\beta}{\alpha});\semFcomp(\beta,\alpha)
%%   =_{\arrow{}{\semF{\compo{\gamma}{\compo{\beta}{\alpha}}}{A}}{\semF{\alpha}{\semF{\beta}{\semF{\gamma}{A}}}}}
%%   \semFcomp(\compo{\gamma}{\beta},\alpha);\semF{\alpha}{\semFcomp(\gamma,\beta)}$
%%   and similarly for \semU{}{}.
%%   \end{itemize}

%% \end{itemize}

%% \sem{\wftp{A}{p}} is an object of $\C_p$; we assume an interpretation is
%% given for each atomic proposition, and the basic rules of adjoint logic
%% require only that we can interpret \F{\alpha}{A} and \U{\alpha}{A},
%% which are interpreted as \semF{\alpha}{\sem{A}} and
%% \semU{\alpha}{\sem{A}}.

%% We can interpret the judgement \seq{A}{\alpha}{B} as either a morphism
%% $\arrow{}{\semF \alpha {\sem A}}{\sem B}$ or a morphism $\arrow{}{\sem
%%   A}{\semU \alpha {\sem B}}$.  We choose $\arrow{}{\semF \alpha A}{B}$
%% because it seems like it will generalize better to a multiple-hypothesis
%% sequent.  This means that the interpretations of the rules for $F$ do
%% not use the adjunction structure, while the interpretations of the rules
%% for $U$ do.  We now consider the interpretation of the sequent calculus
%% rules:

%% \begin{theorem}[Soundness of the sequent calculus]  \label{thm:semsequent}
%% There is a function \sem{-} from derivations $D : \seq{A}{\alpha}{B}$ 
%% to morphisms \arrow{}{\semF \alpha {\sem A}}{\sem B}.
%% \end{theorem}

%% \begin{proof}
%% \begin{itemize}
%% \item 
%% For the hypothesis rule
%% \[
%% \infer[\irl{hyp}]
%%       {\seq P \alpha P}
%%       {e : \tc 1 \alpha}
%% \]
%% we need a morphism \arrow{}{\semF \alpha {\sem P}}{\sem P}, which we take to be the
%% composite 
%% \begin{diagram}
%% {\semF \alpha {\sem P}}  & \rTo^{\semFo{e}} & {\semF{1}{\sem P}} & \rTo^{\semFone} & {\sem P}
%% \end{diagram}

%% \item 
%% For \irl{FL}
%% \[
%% \infer[\irl{FL}]
%%       {\seq {\F {\alpha} A} {\beta}{C}}
%%       {D : \seq {A} {\compo{\alpha}{\beta}} {C}
%%       }
%% \]
%% the premise is interpreted as $\sem{D} : \arrow{}{\semF{\compo{\alpha}{\beta}}{\sem A}}{\sem C}$
%% and we want 
%% \arrow{}{\semF{\beta}{\semF{\alpha}{\sem A}}}{\sem C}, so we take the
%% interpretation to be
%% \begin{diagram}
%% {\semF{\beta}{\semF{\alpha}{\sem A}}} & \rTo^{\inv{\semFcomp(\alpha,\beta)}} & {\semF{\compo{\alpha}{\beta}}{\sem A}} & \rTo^ {\sem{D}} & {\sem C}
%% \end{diagram}

%% \item 
%% For \irl{FR}
%% \[
%% \infer[\irl{FR}]
%%       {\seq {C} {\beta} {\F {\alpha} A}}
%%       { \gamma : r \ge q & e : \tc{\compo{\gamma}{\alpha}}{\beta} &
%%          D : \seq {C} \gamma {A}}
%% \]
%% the premise $\sem{D}$ is \arrow{}{\semF \gamma {\sem C}}{\sem A}, and we
%% want \arrow{}{\semF \beta {\sem C}}{\semF \alpha {\sem A}}.  Using
%% functoriality, we have $\semF{\alpha}{\sem D} : \arrow{}{\semF{\alpha}{\semF
%%     \gamma {\sem C}}}{\semF \alpha {\sem A}}$, so 
%% \begin{diagram}
%% {\semF{\beta}{\sem{C}}} & 
%% \rTo^{\semFo{e}} & {\semF{\compo{\gamma}{\alpha}}{\sem{C}}} &
%% \rTo^{\semFcomp(\gamma,\alpha)} & \semF{\alpha}{\semF \gamma {\sem C}} & \rTo^{\semF{\alpha}{\sem D}} & {\semF \alpha {\sem A}}
%% \end{diagram}

%% \item For \irl{UL}
%% \[
%% \infer[\irl{UL}]
%%       {\seq {\U {\alpha} A} {\beta} {C}}
%%       { \gamma : q \ge p &
%%         e : \tc{\compo{\alpha}{\gamma}} {\beta} &
%%         \seq{A}{\gamma}{C}}
%% \]
%% The premise gives $\sem{D} : \arrow{}{\semF \gamma {\sem A}}{\sem C}$, 
%% and we want \arrow{}{\semF {\beta} {\semU {\alpha} {\sem A}}}{\sem C}.  
%% Using the adjunction, the premise gives
%% \semltor{\sem{D}}{\gamma} : \arrow{}{A}{\semU \gamma C} and it suffices to give
%% \arrow{}{{\semU {\alpha} {\sem A}}}{\semU {\beta} \sem C}.  So we form
%% the composite
%% \begin{diagram}
%% {{\semU {\alpha} {\sem A}}} & \rTo^{\semU{\alpha}{(\semltor{\sem{D}}{\gamma})}} &
%% \semU{\alpha}{\semU \gamma C} & \rTo^{\inv{\semUcomp(\alpha,\gamma)}} & 
%% {\semU{\compo{\alpha}{\gamma}}{\sem{C}}} & \rTo^{\semUo{e}} &
%% {\semU{\beta}{\sem{C}}}
%% \end{diagram}
%% and then move it along the adjunction.  

%% \item For \irl{UR}
%% \[
%% \infer[\irl{UR}]
%%       {\seq {C} {\beta} {\U {\alpha} A}}
%%       {\seq {C} {\compo{\beta}{\alpha}} {A}}
%% \]
%% The premise gives \arrow{}{\semF{\compo{\beta}{\alpha}}{\sem C}}{\sem A}
%% and we want 
%% \arrow{}{\semF{{\beta}}{\sem C}}{\semU {\alpha} A}.  We have

%% \begin{diagram}
%% \semF{\alpha}{\semF{\beta}{\sem{C}}} & \rTo^{\inv{\semFcomp(\beta,\alpha)}} & {\semF{\compo{\beta}{\alpha}}{\sem C}} & \rTo^{\sem{D}} & {\sem A}
%% \end{diagram}

%% so using the adjunction gives the result.  

%% \end{itemize}

%% In summary, we have
%% \[
%% \begin{array}{rcl}
%% \sem{\hyp e} & := & \semFo{e};\semFone\\
%% \sem{\FL {D}} & := & \inv{\semFcomp(\alpha,\beta)} ; \sem{D}\\
%% \sem{\FR {\gamma}{e}{D}} & := & \semFo{e}  ; \semFcomp(\gamma,\alpha) ; \semF{\alpha}{\sem D}\\
%% \sem{\UR {D}} & := & \semltor{(\inv{\semFcomp(\beta,\alpha)};{\sem{D}})}{\alpha} \\
%% \sem{\UL {\gamma}{e}{D}} & := & \semrtol{({\semU{\alpha}{(\semltor{\sem{D}}{\gamma})}} ; {\inv{\semUcomp(\alpha,\gamma)}} ; {\semUo{e}})}{\beta}
%% \end{array}
%% \]
%% \end{proof}

%% In general, an admissible inference rule need not hold in all models.
%% However, in this case, we are considering a class of models
%% (pseudofunctors into \Adj) in which the admissible rules (e.g. cut and
%% identity) are true.

%% \begin{lemma}
%% The admissible sequent calculus rules \tr{e}{D} and \ident{A} and
%% \cut{D}{E} are sound.
%% \end{lemma}

%% \begin{proof}
%% \begin{itemize}
%% \item For \tr{e}{D}, we have
%% \[
%% \infer[\irl{\tr{e}{D}}]
%%       {\seq A {\beta} C}
%%       {e : \tc \alpha \beta &
%%        D : \seq A {\alpha} {C}}
%% \]

%% The premise gives \arrow{}{\semF \alpha {\sem A}}{\sem C}, and we want
%% \arrow{}{\semF \beta {\sem A}}{\sem C} so we have
%% \begin{diagram}
%% {\semF \beta {\sem A}} & \rTo^{\semFo{e}} & {\semF \alpha {\sem A}} & \rTo^{\sem{D}} & {\sem C}
%% \end{diagram}

%% \item For identity
%% \[
%% \infer[\irl{ident}]
%%       {\seq {A} {1} {A}}
%%       {}
%% \]
%% we want \arrow{}{\semF{1}{\sem A}}{\sem A}, so use $\semFone$.  

%% \item For cut

%% \[
%% \infer[\irl{cut}]
%%       {\seq {A} {\compo{\beta}{\alpha}} {C}}
%%       {D : \seq {A} {\beta} {B} &
%%        E : \seq {B} {\alpha} {C}}
%% \]
%% the interpretations of the premises gives
%% \arrow{}{\semF \beta {\sem A}}{\sem B}
%% and 
%% \arrow{}{\semF \alpha {\sem B}}{\sem C}.  
%% To get 
%% \arrow{}{\semF {\compo{\beta}{\alpha}} {\sem A}}{\sem C}, we compose as
%% follows:
%% \begin{diagram}
%% {\semF {\compo{\beta}{\alpha}} {\sem A}} & \rTo^{\semFcomp(\beta,\alpha)} &
%% {\semF {\alpha} {\semF \beta {\sem A}}}  & \rTo^{\semF{\alpha}{\sem{D}}} &
%% {\semF {\alpha} {\sem B}} & \rTo^{\sem{E}} &
%% {\sem{C}}
%% \end{diagram}

%% \end{itemize}

%% \end{proof}

%% We now have two possible interpretations for the admissible rules:
%% first, the one given by expanding the definition in each instance, and
%% second, the compositional definition given above.  In the next few
%% lemmas, we show that these agree:
%% \[
%% \begin{array}{rcl}
%% \sem{\tr{e}{D}} & = & \semFo{e};{\sem{D}} \\
%% \sem{\ident A} & = & \semFone\\
%% \sem{\cut D E} & = & {\semFcomp(\beta,\alpha)};{\semF{\alpha}{\sem{D}}};{\sem{E}}
%% \end{array}
%% \]

%% \begin{lemma} \label{lem:semtr}
%% For all $e : \tc{\beta}{\beta'}$ and derivations $D : \seq{A}{\beta}{B}$, $\sem{\tr{e}{D}} = \semFo{e};{\sem{D}}$.  
%% \end{lemma}
%% \begin{proof}

%% The proof is by induction on $D$, and in each case we can unfold the
%% definition of \tr{e}{D}, so we have to show:

%% \begin{itemize}
%% \item \semFo{e};{\sem{{\hyp e'}}} = \sem {\hyp {(\compv{e'}{e})}}
%% After unfolding definitions, it suffices to use functoriality of
%% $\semFo{e}$ to show $\semFo{\compv{e'}{e}} = \semFo{e};\semFo{e'}$.  

%% \item \semFo{e};{\sem{{\FR \gamma {e'} D}}} = \sem{\FR \gamma {\compv{e'}{e}} D}
%% Again, $\semFo{\compv{e'}{e}} = \semFo{e};\semFo{e'}$ suffices.  

%% \item \semFo{e};{\sem{{\FL D}}} = \sem{\FL {\tr{(\comph{1}{e})} D}}
%% After unfolding definitions and applying the inductive hypothesis, we need to know that
%% $\semFo{e}_{(\semF{\alpha} {A})}; \inv{\semFcomp(\alpha,\beta)} = \inv{\semFcomp(\alpha,\beta')}; \semFo{(\comph{1_\alpha}{e})}_{A}$
%% as arrows \arrow{}{\semF{\beta'}{\semF{\alpha}{A}}}{\semF{\compo \alpha {\beta'}}{A}},
%% which is true by naturality of $\semFcomp(-,-)$ and $\semFo{1} = 1$ and $\semF{\beta}{1} = 1$.


%% \item \semFo{e};{\sem{\UL \gamma {e'} D}} = \sem{ \UL \gamma {\compv{e'}{e}} D}
%% After unfolding the definitions, it suffices to use functoriality
%% $\semUo{\compv{e'}{e}} = \semUo{e'};\semUo{e}$ and the conjugation
%% property for \rtol{-}{}.  

%% \item $\semFo{e};{\sem{\UR D}} = \sem{\UR {\tr {(\comph{e}{1})} D}}$
%% After unfolding definitions and applying the IH, we need to know that
%% $\semFo{e};\ltor{(\inv{\semFcomp(\beta,\alpha)};\sem{D})}{} = 
%% \ltor{(\inv{\semFcomp(\beta',\alpha)};\semFo{\comph{e}{1_\alpha}};\sem{D})}{}$
%% By naturality of the adjunction, the former is equal to
%% \ltor{(\semF{\alpha}{(\semFo{e})};\inv{\semFcomp(\beta,\alpha)};\sem{D})}{}
%% and then naturality of $\semFcomp(-,-)$ gives the result.  

%% \end{itemize}

%% \end{proof}

%% \begin{lemma}\label{lem:semident}
%% For all types $A$, $\sem{\ident A} = \semFone_{\sem{A}}$.    
%% \end{lemma}

%% \begin{proof}  The proof is by
%% induction on $A$.  In each case, we can unfold the definition of
%% \ident{A}, so we need to show:

%% \begin{itemize}

%% \item Case for $P$: $\semFone_P = \sem{\hyp 1}$.  Works because
%%   $\semFo{1} = 1$.  

%% \item Case for $\F \alpha A$: $\semFone_{\semF \alpha {\sem A}} = \sem{\FL {\FR 1 1 {\ident A}}}$
%% After unfolding the definitions and using the IH, it suffices to show
%% that the composite
%% \begin{diagram}
%% \semF{1}{\semF{\alpha}{\sem{A}}} & \rTo^{\inv{\semFcomp(\alpha,1)}} &
%% \semF{\alpha}{\sem A} & \rTo^{\semFcomp(1,\alpha)} &
%% \semF{\alpha}{\semF{1}{\sem A}} & \rTo^{\semF{\alpha}{\semFone_{\sem A}}} & \semF{\alpha}{\sem A}
%% \end{diagram}
%% is $\semF{1}_{\semF{\alpha}{\sem A}}$, which is true by the
%% $\semFone/\semFcomp$ coherence laws.  


%% \item Case for $\U \alpha A$: $\semFone_{\semU \alpha {\sem A}} = \sem {\UR {\UL 1 1 {\ident A}}}$

%% Expanding the definitions and using the IH and using $\semU{1} = 1$, the
%% right-hand side is equal to
%% \[
%% \semltor{(\inv{\semFcomp(1,\alpha)};
%%           \semrtol{(\semU{\alpha}{(\semltor{(\semFone_{\sem A})}{1})}; 
%%           \inv{\semUcomp(\alpha,1)})}{\alpha}
%%           )}{\alpha}
%% \]
%% By coherence $\inv{\semFcomp(1,\alpha)} = \semF{\alpha}{(\semFone_{\semU {\alpha} {\sem A}})}$, 
%% so by naturality of \semltor{-}{\alpha}, 
%% it's equal to
%% \[
%% \semFone;\semltor{(
%%           \semrtol{(\semU{\alpha}{(\semltor{(\semFone_{\sem A})}{1})}; 
%%           \inv{\semUcomp(\alpha,1)})}{\alpha}
%%           )}{\alpha}
%% \]
%% and canceling \semltor{{\semrtol{-}{\alpha}}}{\alpha} gives 
%% \[
%% \semFone;\semU{\alpha}{(\semltor{(\semFone_{\sem A})}{1})}; \inv{\semUcomp(\alpha,1)}
%% \]
%% so it suffices to show
%% \[
%% \semU{\alpha}{(\semltor{(\semFone_{\sem A})}{1})}; \inv{\semUcomp(\alpha,1)} = 1
%% \]
%% But by conjugation $(\semltor{(\semFone)}{1}) = \inv{\semUone}$ and by
%% coherence $\inv{\semUcomp(\alpha,1)} =_{\arrow{}{\semU{\alpha}{\sem{A}}}{\semU{\alpha}{\semU{1}{\sem{A}}}}} \semU{\alpha}{\semUone}$
%% so this is true.

%% \end{itemize}

%% \end{proof}

%% To prove the cut lemma, it will helpful to use the following equivalent
%% definition of \sem{\UL{\gamma}{e}{D}}, which uses only \semF{}{}
%% operations, except for the counit
%% $\semrtol{(1_{\semU{\alpha}{\sem{A}}})}{\alpha} :
%% \arrow{}{\semF{\alpha}{\semU{\alpha}{A}}}{A}$

%% \begin{lemma}\label{lem:semulalt}
%% For any $\gamma,e,D$, 
%% $\sem{\UL \gamma e D}  = \semF{e};\semFcomp(\alpha,\gamma);\semF{\gamma}{(\semrtol{1_{\semU{\alpha}{\sem{A}}}}{\alpha})};\sem{D}$
%% \end{lemma}

%% \begin{proof}
%% By conjugation for $\semU{e}$ and $\semUcomp$, $\sem{\UL \gamma e D} = 
%% \semF{e};\semFcomp(\alpha,\gamma);\semrtol{(\semrtol{(\semU{\alpha}{(\semltor{\sem{D}}{\gamma})})}{\alpha})}{\gamma}$
%% By naturality of \semrtol{-}{\alpha},
%% \[
%% {\semrtol{(\semU{\alpha}{(\semltor{\sem{D}}{\gamma})})}{\alpha}} = 
%% \semrtol{1}{\alpha};{{(\semltor{\sem{D}}{\gamma})}} 
%% \]
%% and by naturality of \semrtol{-}{\gamma},
%% \[
%% \semrtol{(\semrtol{1}{\alpha};{{(\semltor{\sem{D}}{\gamma})}})}{\gamma}
%% = \semF{\gamma}{(\semrtol{1}{\alpha})};\semrtol{{{(\semltor{\sem{D}}{\gamma})}}}{\gamma}
%% \]
%% so collapsing inverses gives the result.  
%% \end{proof}

%% \begin{lemma}\label{lem:semcut}
%% For all derivations $D : \seq{A}{\beta}{B}$ and $E : \seq{B}{\beta'}{C}$, $\sem{\cut D E} = {\semFcomp(\beta,\beta')};{\semF{\beta'}{\sem{D}}};{\sem{E}}$
%% \end{lemma}

%% \newcommand{\semofcut}[2]{{\semFcomp(\beta,\beta')};{\semF{\beta'}{\sem{#1}}};{\sem{#2}}}

%% \begin{proof}

%% The proof is by the same induction on $A,D,E$ that defines cut, and in
%% each case the cut reduces, so we need to show:

%% \begin{itemize}
%% \item $\semofcut {(\hyp e)} {(\hyp {e'})} = \sem{\hyp {(\comph{e}{e'})}}$

%% After expanding definitions, we need to show that
%% \[
%% \semFcomp(\beta,\beta');\semF{\beta'}{(\semF{e};\semFone)};\semF{e'};\semFone = \semFo{\comph{e}{e'}};\semFone
%% \]
%% By functoriality, $\semF{\beta'}{(\semF{e};\semFone)} =
%% \semF{\beta'}{\semF{e}};\semF{\beta'}{\semFone}$, and by naturality of
%% \semFo{e'}, $\semF{e'};{\semF{1}{\semFone}} =
%% \semF{\beta'}{\semFone};\semFo{e'}$,
%% so the LHS equals
%% \[
%% (\semFcomp(\beta,\beta');\semF{\beta'}{\semFo{e}};\semF{e'});{\semF{1}{\semFone}};\semFone
%% \]
%% By naturality of \semFcomp in $\alpha,\beta$, this equals
%% \[
%% \semFo{\comph{e}{e'}};\semFcomp(1,1);{\semF{1}{\semFone}};\semFone
%% \]
%% Coherence implies that $\inv{\semFcomp(1,1)} = \semF{1}{\semFone}$, so
%% collapsing inverses gives the result.  

%% \item $\semofcut {(\FR \gamma e D)} {(\FL E)} = \sem{ \tr {(\comph{e}{1})} {\cut D E} }$

%% After unfolding the definitions on the left-hand side, and using
%% Lemma~\ref{lem:semtr} and the IH on the right-hand side, 
%% the calculation consists of using naturality of $\inv{\semFcomp_A}$ in $A$ to
%% show that $\semF{\beta'}{\semF{\alpha}{\sem
%%     D}};\inv{\semFcomp(\alpha,\beta')} =
%% \inv{\semFcomp(\alpha,\beta')};\semF{\compo \alpha \beta'}{\sem{D}}$,
%% using naturality of $\semFcomp(\alpha,\beta)$ in $\alpha,\beta$ to
%% show 
%% $\semFcomp(\beta,\beta');\semF{\beta'}{\semF{e}} =
%% \semF{\comph{e}{1_{\beta'}}};\semFcomp(\compo{\gamma}{\alpha},\beta')$, 
%% and using the associativity coherence to show
%% $\semFcomp(\compo{\gamma}{\alpha},\beta');\semF{\beta'}{(\semFcomp(\gamma,\alpha))}
%% = \semFcomp(\gamma,\compo{\alpha}{\beta'});\semFcomp(\alpha,\beta')$.

%% \item $\semofcut {(\UR D)} {(\UL \gamma e E)} = \sem{\tr {(\comph{1}{e})} {\cut D E}}$

%% Unfolding the definitions, using Lemma~\ref{lem:semtr} and the IH, we
%% need to show $LHS = RHS$, where
%% \[
%% \begin{array}{l}
%% LHS := \semFcomp(\beta,\beta');\semF{\beta}{(\semltor{(\inv{\semFcomp(\beta,\alpha)};\sem{D})}{\alpha}  )};
%% \semrtol{(\semU{\alpha}{(\semltor{\sem{E}}{\gamma})};\inv{\semUcomp(\alpha,\gamma)};\semU{e})}{\beta'}\\
%% RHS := \semF{\comph{1}{e}};\semFcomp(\compo{\beta}{\alpha},\gamma);\semF{\gamma}{\sem{D}};\sem{E}
%% \end{array}
%% \]

%% Using conjugation to move \semU{e} and
%% \inv{\semUcomp(\alpha,\gamma)} outside of the \semrtol{-}{\beta'}, we have
%% \[
%% \semrtol{(\semU{\alpha}{(\semltor{\sem{E}}{\gamma})};\inv{\semUcomp(\alpha,\gamma)};\semU{e})}{\beta'}
%% = 
%% \semF{e};\semFcomp(\alpha,\gamma);\semrtol{{\semrtol{(\semU{\alpha}{\semltor{\sem E}{\gamma}})}{\alpha}}}{\gamma}
%% \]
%% By naturality of \semrtol{-}{\alpha}, the right-hand side of that is equal to
%% \[
%% \semF{e};\semFcomp(\alpha,\gamma);\semrtol{{\semrtol{(\semrtol{1}{\alpha};\semltor{\sem{E}}{\gamma})}{\alpha}}}{\gamma}
%% \]
%% and then by naturality of \semrtol{-}{\gamma}, that is equal to
%% \[
%% \semF{e};\semFcomp(\alpha,\gamma);\semF{\gamma}{(\semrtol{1}{\alpha})};\semrtol{{{(\semltor{\sem{E}}{\gamma})}}}{\gamma}
%% \]
%% so collapsing inverses, we have overall that
%% \[
%% \semltor{(\semU{\alpha}{(\semltor{\sem{E}}{\gamma})};\inv{\semUcomp(\alpha,\gamma)};\semU{e})}{\beta'}
%% =
%% \semF{e};\semFcomp(\alpha,\gamma);\semF{\gamma}{(\semrtol{1}{\alpha})};\sem{E}
%% \]

%% Therefore 
%% \[
%% LHS = \semFcomp(\beta,\beta');\semF{\beta}{(\semltor{(\inv{\semFcomp(\beta,\alpha)};\sem{D})}{\alpha}  )};
%% \semF{e};\semFcomp(\alpha,\gamma);\semF{\gamma}{(\semrtol{1}{\alpha})};\sem{E}
%% \]
%% Moving \semFo{e} to the left using its naturality, this is equal to
%% \[
%% \semFcomp(\beta,\beta');\semF{e};\semF{\compo{\alpha}{\gamma}}{(\semltor{(\inv{\semFcomp(\beta,\alpha)};\sem{D})}{\alpha}  )} ;\semFcomp(\alpha,\gamma);\semF{\gamma}{(\semrtol{1}{\alpha})};\sem{E}
%% \]
%% and then moving it to the left again using naturality of
%% $\semFcomp(\alpha,\gamma)$ in $\alpha,\beta$ gives
%% \[
%% \semF{(\comph{1}{e})};\semFcomp(\beta,\compo{\alpha}{\gamma});\semF{\compo{\alpha}{\gamma}}{(\semltor{(\inv{\semFcomp(\beta,\alpha)};\sem{D})}{\alpha}  )};\semFcomp(\alpha,\gamma);\semF{\gamma}{(\semrtol{1}{\alpha})};\sem{E}
%% \]
%% and moving $\semFcomp_A(\alpha,\beta)$ to the left using naturality in $A$ gives
%% \[
%% \semF{(\comph{1}{e})};\semFcomp(\beta,\compo{\alpha}{\gamma});\semFcomp(\alpha,\gamma);\semF{\gamma}{\semF{\alpha}{(\semltor{(\inv{\semFcomp(\beta,\alpha)};\sem{D})}{\alpha}  )}};\semF{\gamma}{(\semrtol{1}{\alpha})};\sem{E}
%% \]
%% By the associativity coherence, this is equal to
%% \[
%% \semF{(\comph{1}{e})};\semFcomp(\compo{\beta}{\alpha},{\gamma});\semF{\gamma}{\semFcomp(\beta,\alpha)};\semF{\gamma}{\semF{\alpha}{(\semltor{(\inv{\semFcomp(\beta,\alpha)};\sem{D})}{\alpha}  )}};\semF{\gamma}{(\semrtol{1}{\alpha})};\sem{E}
%% \]
%% so collecting the three terms that are under \semF{\gamma}, to show that $LHS
%% = RHS$, it suffices to show that
%% \[
%% {\semFcomp(\beta,\alpha)};{\semF{\alpha}{(\semltor{(\inv{\semFcomp(\beta,\alpha)};\sem{D})}{\alpha})}};{(\semrtol{1}{\alpha})} = \sem D
%% \]
%% By naturality of the adjunction,
%% \[
%% \semF{\alpha}{(\semltor{(\inv{\semFcomp(\beta,\alpha)};\sem{D})}{\alpha})};(\semrtol{1}{\alpha})
%% = 
%% \semrtol{({(\semltor{(\inv{\semFcomp(\beta,\alpha)};\sem{D})}{\alpha})};1)}{\alpha}
%% =
%% {(\inv{\semFcomp(\beta,\alpha)};\sem{D})}
%% \]
%% so collapsing inverses gives the result.  

%% \item $\semofcut D {(\FR \gamma e E)} = \sem{\FR {\compo{\beta}{\gamma}} {\comph{1}{e}} {\cut D E}}$

%% After unfolding the definitions and using the
%% IH on the right-hand side, 
%% the proof uses naturality of $\semF{e}_A$ in $A$ to show
%% $\semF{\beta'}{\sem D}; \semF{e} =
%% \semF{e};\semF{\compo{\gamma}{\alpha}}{\sem D}$, 
%% naturality of $\semFcomp_A$ in $A$ to show
%% $\semF{\compo{\gamma}{\alpha}}{\sem D};\semFcomp(\gamma,\alpha) = 
%% \semFcomp(\gamma,\alpha);\semF{\alpha}{\semF{\gamma}{\sem D}}$,
%% naturality of $\semFcomp(\alpha,\beta)$ in $\alpha,\beta$ to equate
%% $\semFcomp(\beta,\beta');\semF{e} =
%% \semF{\comph{1}{e}};\semFcomp(\beta,\compo{\gamma}{\alpha})$
%% and the associativity coherence to equate 
%% $\semFcomp(\beta,\compo{\gamma}{\alpha});\semFcomp(\gamma,\alpha)
%% = \semFcomp(\compo{\beta}{\gamma},\alpha);\semF{\alpha}(\semFcomp(\beta,\gamma))$.

%% \item $\semofcut D {(\UR E)} = \sem{\UR {\cut D E}}$

%% After expanding definitions and using the IH, we need to show
%% \[
%% \semFcomp(\beta,\beta');\semF{\beta'}{\sem{D}};\semltor{(\inv{\semFcomp(\beta',\alpha)};\sem{E})}{\alpha}
%% =
%% \semltor{(\inv{\semFcomp(\compo{\beta}{\beta'},\alpha)};\semFcomp(\beta,\compo{\beta'}{\alpha});\semF{\compo{\beta'}{\alpha}}{\sem D};\sem{E})}{\alpha}
%% \]
%% By the associativity coherence, 
%% \[
%% \inv{\semFcomp(\compo{\beta}{\beta'},\alpha)} = 
%% \semF{\alpha}{(\semFcomp(\beta,\beta'))};\inv{\semFcomp(\beta',\alpha)};\inv{\semFcomp(\beta,\compo{\beta'}{\alpha})}
%% \]
%% and plugging this in to the RHS and then collapsing inverses gives 
%% \[
%% \semltor{(\semF{\alpha}{\semFcomp(\beta,\beta')};\inv{\semFcomp(\beta',\alpha)};\semF{\compo{\beta'}{\alpha}}{\sem D};\sem{E})}{\alpha}
%% \]
%% By naturality of \semltor{}{\alpha}, this is the same as
%% \[
%% {\semFcomp(\beta,\beta')};\semltor{(\inv{\semFcomp(\beta',\alpha)};\semF{\compo{\beta'}{\alpha}}{\sem D};\sem{E})}{\alpha}
%% \]
%% By naturality of \inv{\semFcomp(\beta',\alpha)}, this is the same as 
%% \[
%% {\semFcomp(\beta,\beta')};\semltor{(\semF{\alpha}{\semF{\beta'}{\sem D}};\inv{\semFcomp(\beta',\alpha)};\sem{E})}{\alpha}
%% \]
%% so using naturality of \semltor{}{\alpha} again gives the result.  

%% \item $\semofcut {(\FL D)} E = \sem{\FL {\cut D E}}$ (note: we could
%%   assume that $E$ is not a right rule, but this assumption is not
%%   necessary).  After expanding the definitions and using the IH, the
%%   main step is to use the associativity coherence to show
%%   $\semFcomp(\alpha,\compo{\beta}{\beta'});\semFcomp(\beta,\beta') =
%%   \semFcomp(\compo{\alpha}{\beta},\beta');\semF{\beta'}(\semFcomp(\alpha,\beta))$.

%% \item $\semofcut {(\UL \gamma e D)} E = \sem{\UL {\compo{\gamma}{\beta'}}
%%   {\comph{e}{1}} {\cut D E}}$ (note: we could assume that $E$ is not a
%%   right rule, but this assumption is not necessary).  

%% Applying Lemma~\ref{lem:semulalt}, to the left and the right sides, and
%% using the IH, we need to show $LHS = RHS$, where
%% \[
%% \begin{array}{l}
%% LHS :=
%% \semFcomp(\beta,\beta');\semF{\beta'}{\semFo{e}};\semF{\beta'}{\semFcomp(\alpha,\gamma)};\semF{\beta'}{\semF{\gamma}{(\semrtol
%%     1 \alpha)}};\semF{\beta'}{\sem D};\sem{E}\\
%% RHS :=
%% \semFo{\comph{e}{1}};\semFcomp(\alpha,\compo{\gamma}{\beta'});\semF{\compo{\gamma}{\beta'}}{(\semrtol{1}{\alpha})};\semFcomp(\gamma,\beta');\semF{\beta'}{\sem D};\sem{E}
%% \end{array}
%% \]

%% By naturality of $\semFcomp(\alpha,\beta)$ in $\alpha,\beta$
%% \[
%% \semFcomp(\beta,\beta');\semF{\beta'}{\semFo{e}} = \semFo{\comph{e}{1}};\semFcomp(\compo{\alpha}{\gamma},\beta')
%% \]
%% and by the associativity coherence, 
%% \[
%% \semFcomp(\compo{\alpha}{\gamma},\beta');\semF{\beta'}{\semFcomp(\alpha,\gamma)}
%% = 
%% \semFcomp(\alpha,\compo{\gamma}{\beta'});\semFcomp(\gamma,\beta')
%% \]
%% so 
%% \[
%% LHS = 
%% \semFo{\comph{e}{1}};
%% \semFcomp(\alpha,\compo{\gamma}{\beta'});\semFcomp(\gamma,\beta')
%% ;\semF{\beta'}{\semF{\gamma}{(\semrtol
%%     1 \alpha)}};\semF{\beta'}{\sem D};\sem{E}
%% \]
%% Therefore using naturality of $\semFcomp(\gamma,\beta')_A$ in $A$ to
%% move it to the right gives the result. 
%% \end{itemize}
%% \end{proof}

%% Next, we validate the rules for \ap{}{}.  

%% \begin{theorem}[Soundess of the equational theory.] \label{thm:semeq}
%% If \ap{D}{D'} then $\sem{D} = \sem{D'}$.  
%% \end{theorem}

%% \begin{proof}
%% Because the goal is equality of morphisms, the congruence (equivalence
%% relation, compatibility for each derivation constructor) rules are all
%% true.  It remains to validate the axioms:

%% \begin{itemize}
%% \item $\sem{D} = \sem{\FL {\cut{(\FR 1 1 {\ident{A}})}{D}} }$
%% when ${D : \seq{\F \alpha A}{\beta}{C}}$

%% After expanding the definitions and using
%% Lemmas~\ref{lem:semident},~\ref{lem:semcut}, it suffices to show
%% \[
%% \inv{\semFcomp(\alpha,\beta)};\semFcomp(\alpha,\beta);\semF{\beta}{(\semFo{1};\semFcomp(1,\alpha);\semF{\alpha}{\semFone})};\sem{D}
%% = \sem{D}
%% \]
%% This is true because $\semFo{1} = 1$ and because
%% $\inv{\semFcomp(1,\alpha)} = \semF{\alpha}{\semFone}$, so canceling
%% identities and inverses gives the result.  

%% \item $\sem{D} = \sem{\UR {\cut{D}{(\UL 1 1 {\ident{A}})}}}$
%% when ${D : \seq{C}{\beta}{\U \alpha A}}$

%% After expanding the definitions and using
%% Lemmas~\ref{lem:semident},~\ref{lem:semcut}, it suffices to show
%% \[
%% \semltor{(\inv{\semFcomp(\beta,\alpha)};\semFcomp(\beta,\alpha);\semF{\alpha}{\sem D};\semrtol{(\semU{\alpha}{(\semltor{\semFone}{1})};\inv{\semUcomp(\alpha,1)};\semUo{1})}{\alpha})}{\alpha}
%% = \sem{D}
%% \]
%% Canceling the $\semFcomp(\beta,\alpha)$ and using naturality of
%% \semltor{-}{\alpha}, this is equal to
%% \[
%% \sem{D};\semltor{(\semrtol{(\semU{\alpha}{(\semltor{\semFone}{1})};\inv{\semUcomp(\alpha,1)};\semUo{1})}{\alpha})}{\alpha}
%% \]
%% so it suffices to show that the later is the identity.  Canceling the
%% adjunction round-trip and $\semUo{1}$, it is equal to
%% \[
%% \semU{\alpha}{(\semltor{\semFone}{1})};\inv{\semUcomp(\alpha,1)}
%% \]
%% By conjugation for \semFone, $\semrtol{\semFone}{1} = \semUone$, and by
%% the associativity coherence, $\inv{\semUcomp(\alpha,1)} =
%% \semU{\alpha}{\semUone}$, so canceling inverses gives the result.  

%% \item $\sem{{\FR{\gamma}{e}{\tr{e_2}{D}}}} = \sem{\FR{\gamma'}{(\compv{(\comph{e_2}{1})}{e})}{D}}$
%% when $e : \tc{\compo{\gamma}{\alpha}}{\beta}$
%% and $D : \seq{C}{\gamma'}{A}$
%% and $e_2 : \tc{\gamma'}{\gamma}$.

%% After expanding the definitions and using Lemma~\ref{lem:semtr}, we need
%% to show
%% \[
%% \semFo{e};\semFcomp(\gamma,\alpha);\semF{\alpha}{\semFo{e_2}};\semF{\alpha}{\sem{D}}
%% = \semFo{\compv{(\comph{e_2}{1})}{e}};\semFcomp(\gamma',\alpha);\semF{\alpha}{\sem{D}}
%% \]
%% This is true using functoriality to show $\semFo{\compv{(\comph{e_2}{1})}{e}} =
%% \semFo{e};\semFo{{(\comph{e_2}{1})}}$,
%% and naturality of $\semFcomp(\alpha,\beta)$ in $\alpha,\beta$ to show
%% $\semFo{{(\comph{e_2}{1})}};\semFcomp(\gamma',\alpha) =
%% \semFcomp(\gamma,\alpha);\semF{\alpha}{\semFo{e_2}}$.  
  

%% \item $\sem{\UL{\gamma}{e}{\tr{e_2}{D}}} = \sem{\UL{\gamma'}{(\compv{(\comph{1}{e_2})}{e})}{D}}$
%% when $e : \tc{\compo{\gamma}{\alpha}}{\beta}$
%% and $D : \seq{C}{\gamma'}{A}$
%% and $e_2 : \tc{\gamma'}{\gamma}$.

%% Using Lemmas~\ref{lem:semulalt} and \ref{lem:semtr}, we need to show
%% \[
%% \semFo{e};\semFcomp(\alpha,\gamma);\semF{\gamma}{(\semltor{1}{\alpha})};\semFo{e_2};\sem{D}
%% =
%% \semFo{ \compv{(\comph{1}{e_2})}{e}   };\semFcomp(\alpha,\gamma');\semF{\gamma'}{(\semltor{1}{\alpha})};\sem{D}
%% \]
%% By functoriality, 
%% $\semFo{ \compv{(\comph{1}{e_2})}{e}   } = \semF{e};\semF{\comph{1}{e_2}}$, 
%% and by naturality of $\semFcomp(\alpha,\beta)$ in $\alpha,\beta$,
%% $\semFo{ {(\comph{1}{e_2})} };\semFcomp(\alpha,\gamma')
%% = \semFcomp(\alpha,\gamma);\semFo{e_2}$, so
%% the right-hand side is equal to 
%% \[
%% \semFo{e};\semFcomp(\alpha,\gamma);\semFo{e_2};\semF{\gamma'}{(\semltor{1}{\alpha})};\sem{D}
%% \]
%% so using naturality of \semFo{e_2} gives the result.  

%% \item $\sem{\UL {\beta}{e_2} {\FR {\gamma} {e_1} {D}}} = \sem{\FR {\delta_3} {e_4} {\UL {\gamma} {e_3} {D}}}$
%% when ${\compv{(\comph{1}{e_1})}{e_2} = {\compv{(\comph{e_3}{1})}{e_4}}}$,
%% where 
%% $e_1 : \tc{(\compo {\gamma} {\alpha})}{\beta}$  and
%% $e_2 : \tc{(\compo {\delta_1} {\beta})}{\delta_2}$ and
%% $e_3 : \tc{(\compo {\delta_1} {\gamma})}{\delta_3}$ and
%% $e_4 : \tc{(\compo {\delta_3} {\alpha})}{\delta_2}$ and

%% Expanding the definitions and using Lemma~\ref{lem:semulalt}, we need to show
%% \[
%% \begin{array}{ll}
%%   & \semFo{e_2};\semFcomp(\delta_1,\beta);\semF{\beta}{(\semrtol{1}{\delta_1})};\semFo{e_1}  ; \semFcomp(\gamma,\alpha) ; \semF{\alpha}{\sem D}\\
%% = & \semFo{e_4};\semFcomp(\delta_3,\alpha);\semF{\alpha}{\semFo{e_3}};\semF{\alpha}{\semFcomp(\delta_1,\gamma)};\semF{\alpha}{\semF{\gamma}{(\semltor{1}{\delta_1})}};\semF{\alpha}{\sem{D}}
%% \end{array}
%% \]
%% Using naturality of \semFo{e_1} and $\semFcomp(\gamma,\alpha)$, the
%% left-hand side is equal to
%% \[
%% \semFo{e_2};\semFcomp(\delta_1,\beta);\semFo{e_1};\semFcomp(\gamma,\alpha);\semF{\alpha}{\semF{\gamma}{(\semrtol{1}{\delta_1})}};\semF{\alpha}{\sem D}
%% \]
%% so it suffices to show
%% \[
%% \semFo{e_2};\semFcomp(\delta_1,\beta);\semFo{e_1};\semFcomp(\gamma,\alpha)
%% = \semFo{e_4};\semFcomp(\delta_3,\alpha);\semF{\alpha}{\semFo{e_3}};\semF{\alpha}{\semFcomp(\delta_1,\gamma)}
%% \]
%% Using naturality of $\semFcomp(\alpha,\beta)$ in $\alpha,\beta$ the LHS equals
%% \[
%% \semFo{e_2};\semFo{\comph{1}{e_1}};\semFcomp(\delta_1,\compo{\gamma}{\alpha});\semFcomp(\gamma,\alpha)
%% \]
%% and the RHS equals
%% \[
%% \semFo{e_4};\semFo{\comph{e_3}{1}};\semFcomp(\compo{\delta_1}{\gamma},\alpha);\semF{\alpha}{\semFcomp(\delta_1,\gamma)}
%% \]
%% But $\semFcomp(\delta_1,\compo{\gamma}{\alpha});\semFcomp(\gamma,\alpha) = \semFcomp(\compo{\delta_1}{\gamma},\alpha);\semF{\alpha}{\semFcomp(\delta_1,\gamma)}$
%% by the associativity coherence, and 
%% $\semFo{e_2};\semFo{\comph{1}{e_1}} =
%% \semFo{e_4};\semFo{\comph{e_3}{1}}$
%% by functoriality using the assumption that 
%% ${\compv{(\comph{1}{e_1})}{e_2} = {\compv{(\comph{e_3}{1})}{e_4}}}$.  

%% \end{itemize}
%% \end{proof}

%% Just as the class of models we are considering supported interpretations
%% of \tr{e}{D} and \ident{A} and \cut{D}{E} in general, the admissible
%% rules for \ap{}{} hold in general:

%% \begin{theorem}[Soundness of admissible equational rules] 
%% The admissible rules for \ap{D}{D'} in Section~\ref{sec:rules:equations}
%% are true in the semantics.  
%% \end{theorem}

%% \begin{proof}

%% \begin{itemize}

%% \item For ${\tr{1}{D} = D}$
%% and ${\tr{(\compv{e_1}{e_2})}{D} = \tr{e_2}{\tr{e_1}{D}}}$,
%% by Lemma~\ref{lem:semtr}, we need to show $\semFo{e};\sem{D} = \sem{D}$
%% and $\semFo{\compv{e_1}{e_2}};{\sem D} =
%% \semFo{e_2};\semFo{e_1};\sem{D}$, which are true by functoriality.  

%% \item For the congruence rules:
%% \[
%% \infer{\ap{\tr{e}{D}}{\tr{e}{D'}}}
%%       {\ap{D}{D'}}
%% \quad
%% \infer{\ap{\cut{D}{E}}{\cut{D'}{E}}}
%%       {\ap{D}{D'}}
%% \quad
%% \infer{\ap{\cut{D}{E}}{\cut{D}{E'}}}
%%       {\ap{E}{E'}}
%% \]
%% By assumption, $\sem{D} = \sem{D'}$ or $\sem{E} = \sem{E'}$.  
%% By Lemma~\ref{lem:semtr} and~\ref{lem:semcut}, 
%% \sem{\tr{e}{D}} and \sem{\cut D E} are compositional in \sem{D} and
%% \sem{E}, so the conclusions are equal as well.  

%% \item For
%%   {\ap{\tr{(\comph{e}{e'})}{\cut{D}{D'}}}{\cut{(\tr{e}{D})}{(\tr{e'}{D'})}}}
%% where $e : \tc \alpha {\beta}$ and $e' : \tc {\alpha'}{\beta'}$, 
%% by Lemmas~\ref{lem:semtr} and \ref{lem:semcut}, we need to show
%% \[
%% \semFo{(\comph{e}{e'})};\semFcomp(\alpha,\alpha');\semF{\alpha'}{(\sem{D})};\sem{D'}
%% =
%% \semFcomp(\beta,\beta');\semF{\beta'}{\semFo{e}};\semF{\beta'}{\sem{D}};\semFo{e'};\sem{D'}
%% \]
%% Using naturality for \semFo{e'}, the right-hand side equals
%% \[
%% \semFcomp(\beta,\beta');\semF{\beta'}{\semFo{e}};\semFo{e'};\semF{\alpha'}{\sem{D}};\sem{D'}
%% \]
%% so naturality of $\semFcomp(\alpha,\beta)$ in $\alpha,\beta$ gives the
%% result.  

%% \item 
%% For {\ap{\cut{D_1}{(\cut{D_2}{D_3})}}{\cut{(\cut{D_1}{D_2})}{D_3}}},
%% by Lemma~\ref{lem:semcut}, we need to show
%% \[
%% \begin{array}{ll}
%%   & \semFcomp(\beta_1,\compo{\beta_2}{\beta_3});\semF{\compo{\beta_2}{\beta_3}}{\sem{D_1}};\semFcomp(\beta_2,\beta_3);\semF{\beta_3}{\sem{D_2}};\sem{D_3}\\
%% = & \semFcomp(\compo{\beta_1}{\beta_2},\beta_3);\semF{\beta_3}{\semFcomp(\beta_1,\beta_2)};\semF{\beta_3}{\semF{\beta_2}{\sem {D_1}}};\semF{\beta_3}{\sem{D_2}};\sem{D_3}
%% \end{array}
%% \]
%% By naturality of $\semFcomp(\beta_2,\beta_3)_A$ in $A$, the LHS equals 
%% \[
%% \semFcomp(\beta_1,\compo{\beta_2}{\beta_3});\semFcomp(\beta_2,\beta_3);\semF{\beta_2}{\semF{\beta_3}{\sem{D_1}}};\semF{\beta_3}{\sem{D_2}};\sem{D_3}
%% \]
%% so the associativity coherence gives the result.

%% \item For  
%% {\ap{\cut{D}{\ident{}}}{D}},
%% by Lemmas~\ref{lem:semcut} and \ref{lem:semident}, 
%% it suffices to show 
%% \[
%% \semFcomp(\beta,1);\semF{1}{\sem{D}};\semFone = \sem{D}
%% \]
%% By naturality of \semFone, the left-hand side equals
%% $\semFcomp(\beta,1);\semFone;\sem{D}$, so coherence (and a unit law for
%% $\C_p$---interpreting a unit law for the syntax involves a unit law for
%% the semantics) gives the result.

%% \item For {\ap{\cut{\ident{}}{D}}{D}}, by 
%% by Lemmas~\ref{lem:semcut} and \ref{lem:semident}, 
%% it suffices to show 
%% \[
%% \semFcomp(1,\beta);\semF{\beta}{\semFone};{\sem{D}} = \sem{D}
%% \]
%% which is true by coherence (and a unit law for $\C_p$).

%% \item The unrestricted left-commutative rules ${\ap{\cut {(\FL D)} E} {\FL {\cut D E}}}$
%% and\\ ${\ap{\cut {(\UL \gamma e D)} E} {\UL {\compo{\gamma}{\alpha}}
%%     {\comph{e}{1}} {\cut D E}}}$ were checked as part of the proof of
%% Lemma~\ref{lem:semcut} above.   
%% \end{itemize}
%% \end{proof}

%% \begin{lemma}[Interpretation of Coproducts]
%% If each $\C_p$ has coproducts, then
%% Theorem~\ref{thm:semsequent} and Lemmas~\ref{lem:semtr} and
%% \ref{lem:semident} and \ref{lem:semcut} and Theorem \ref{thm:semeq} are
%% true when the rules for coproducts in Figure~\ref{fig:coprod} are added
%% to the logic.  
%% \end{lemma}

%% \begin{proof}  Write $\seminl : \arrow{}{A}{\coprd{A}{B}}$
%% and $\seminr : \arrow{}{B}{\coprd{A}{B}}$
%% and \semcase{m_1}{m_2} for the coproduct maps.  

%% \begin{itemize}

%% \item First, we show how to interpret the sequent calculus rules.
%%  For \Inl{D} and \Inr{D}, define
%% \[
%% \begin{array}{rcl}
%% \sem{\Inl{D}} & := & \sem{D};\seminl\\
%% \sem{\Inr{D}} & := & \sem{D};\seminr\\
%% \end{array}
%% \]
%% The left case makes sense because \sem{D} : \arrow{}{\sem C}{\sem A}, so
%% post-composing with \seminl\/ has the right codomain; the other case is
%% analogous.  

%% For \sem{\Case{D_1}{D_2}}, we essentially need to do the proof that left adjoints
%% preserve coproducts: we have
%% $\sem{D_1} : \arrow{}{\semF{\alpha}{A}}{C}$
%% and 
%% $\sem{D_2} : \arrow{}{\semF{\alpha}{B}}{C}$
%% and we want a map
%% $\arrow{}{\semF{\alpha}{(\coprd{A}{B})}}{C}$, which we define as
%% follows:
%% \[
%% \begin{array}{rcl}
%% \sem{\Case{D_1}{D_2}} & := & \semrtol{\semcase{\semltor{\sem{D_1}}{\alpha}}{\semltor{\sem{D_2}}{\alpha}}}{\alpha}
%% \end{array}
%% \]

%% \item Next, we give the new cases of Lemma~\ref{lem:semtr}, where $e :
%%   \tc{\alpha}{\beta}$ and the given derivation has mode $\alpha$.
%% For \Inl{D}, we need to show
%% \[
%% \sem{\tr e D};\seminl = \semFo{e};\sem{D};\seminl
%% \]
%% which is immediate by the IH (and associativity).  The \Inr{D} case is
%% analogous.  For \Case{D_1}{D_2}, after expanding the definitions and
%% using the IH, we need to show
%% \[
%% \semrtol{\semcase{\semltor{(\semFo{e};\sem{D_1})}{\beta}}{\semltor{(\semFo{e};\sem{D_2})}{\beta}}}{\beta} = \semFo{e};\semrtol{(\semcase{\semltor{\sem{D_1}}{\alpha}}{\semltor{\sem{D_2}}{\alpha}})}{\alpha}
%% \]
%% By conjugation, the right-hand side equals
%% \[
%% \semrtol{(\semcase{\semltor{\sem{D_1}}{\alpha}}{\semltor{\sem{D_2}}{\alpha}};\semUo{e})}{\beta}
%% \]
%% and the left-hand side equals 
%% \[
%% \semrtol{\semcase{\semltor{\sem{D_1}}{\beta};\semUo{e}}{\semltor{\sem{D_2}}{\beta};\semUo{e}}}{\beta}
%% \]
%% and these are equal by the uniqueness part of the universal property for coproducts.

%% \item 
%% Next, we give the new case of Lemma~\ref{lem:semident}:
%% $\sem{\Case {\Inl {\ident{A}}} {\Inr {\ident{B}}}} = \semFone$.
%% After expanding the definitions and using the IH, we need to show
%% \[
%% \semrtol{\semcase{\semltor{(\semFone;\seminl)}{1}}{\semltor{(\semFone;\seminr)}{1}}}{1}
%% \]
%% By conjugation for \semFone, this is equal to
%% \[
%% \semrtol{\semcase{\seminl;\inv{\semUone}}{\seminr;\inv{\semUone}}}{1}
%% \]
%% By uniqueness for coproducts, this is equal to
%% \[
%% \semrtol{(\semcase{\seminl}{\seminr};{\inv{\semUone}})}{1}
%% \]
%% By conjugation for \semUone, this is equal to 
%% \[
%% \semFone;{\semcase{\seminl}{\seminr}}
%% \]
%% and by uniqueness for coproducts, $\semcase{\seminl}{\seminr} = 1_{\coprd{A}{B}}$.  

%% \item Next, we give the new cases of Lemma~\ref{lem:semcut}.  There are
%%   5 reductions; we show the \Inl{-} cases of the principal and
%%   right-commutative cuts, and the left-commutative cut case; the \Inr{-}
%%   cases are analogous.

%% \begin{itemize}
%% \item For $\cut{(\Inl{D})}{(\Case{E_1}{E_2})} := \cut{D}{E_1}$,
%% by the IH we need to show that 
%% \[
%% \semofcut{D}{E_1} = \semFcomp(\beta,\beta');\semF{\beta'}{\sem{D}};\semF{\beta'}{\seminl};{\semrtol{\semcase{\semltor{\sem{E_1}}{\beta'}}{\semltor{\sem{E_2}}{\beta'}}}{\beta'}}
%% \]
%% By naturality of \semrtol{\beta'}, 
%% $\semF{\beta'}{\seminl};{\semrtol{\semcase{\semltor{\sem{E_1}}{\beta'}}{\semltor{\sem{E_2}}{\beta'}}}{\beta'}}
%% =
%% \semrtol{(\seminl;{\semcase{\semltor{\sem{E_1}}{\beta'}}{\semltor{\sem{E_2}}{\beta'}}})}{\beta'}$,
%% which by the universal property for coproducts equals 
%% \semrtol{({{\semltor{\sem{E_1}}{\beta'}}})}{\beta'},
%% which equals {\sem{E_1}} by collapsing inverses.  

%% \item For $\cut{D}{(\Inl{E})} := \Inl{\cut{D}{E}}$,
%% the result is immediate by the IH.
%% %% \cut{D}{(\Inr{E})} & := & \Inr{\cut{D}{E}}

%% \item For $\cut{(\Case{D_1}{D_2})}{E} := \Case{\cut{D_1}{E}}{\cut{D_2}{E}}$,
%% by the IH we need to show that
%% \[
%% \begin{array}{ll}
%% & {\semrtol{\semcase{\semltor{(\semofcut{D_1}{E})}{\compo{\beta}{\beta'}}}{\semltor{(\semofcut{D_2}{E})}{\compo{\beta}{\beta'}}}}{\compo{\beta}{\beta'}}}\\
%% = &\semFcomp(\beta,\beta');\semF{\beta'}{{\semrtol{\semcase{\semltor{\sem{D_1}}{\beta}}{\semltor{\sem{D_2}}{\beta}}}{\beta}}};\sem{E}
%% \end{array}
%% \]
%% Conjugating the $\semFcomp(\beta,\beta')$ outside the
%% $\semrtol{-}{\compo{\beta}{\beta'}}$, and then conjugating the resulting 
%% $\semUcomp(\beta,\beta')$ outside the $\semltor{-}{\compo{\beta}{\beta'}}$,
%% the left-hand side is equal to
%% \[
%% \semFcomp(\beta,\beta');
%% \semrtol{(\semrtol{\semcase{\semltor{(\semltor{(\semF{\beta'}{\sem{D_1}};\sem{E})}{\beta'})}{\beta}}{\semltor{(\semltor{(\semF{\beta'}{\sem{D_2}};\sem{E})}{\beta'})}{\beta}}}{\beta})}{\beta'}
%% \]
%% By naturality of the adjunction, this is the same as 
%% \[
%% \semFcomp(\beta,\beta');
%% \semrtol{(\semrtol{\semcase{\semltor{({\sem{D_1}};\semltor{\sem{E}}{\beta'})}{\beta}}{\semltor{({\sem{D_2}};\semltor{\sem{E}}{\beta'})}{\beta}}}{\beta})}{\beta'}
%% \]
%% and then
%% \[
%% \semFcomp(\beta,\beta');
%% \semrtol{(\semrtol{\semcase{\semltor{{\sem{D_1}}}{\beta};\semU{\beta}{(\semltor{\sem{E}}{\beta'})}}
%%                            {\semltor{{\sem{D_2}}}{\beta};\semU{\beta}{(\semltor{\sem{E}}{\beta'})}}}
%%   {\beta})}{\beta'}
%% \]
%% By uniqueness for coproducts, this is
%% \[
%% \semFcomp(\beta,\beta');
%% \semrtol{(\semrtol{(\semcase{\semltor{{\sem{D_1}}}{\beta}}
%%                            {\semltor{{\sem{D_2}}}{\beta}};\semU{\beta}{(\semltor{\sem{E}}{\beta'})})}
%%   {\beta})}{\beta'}
%% \]
%% By naturality of the adjunction, that is
%% \[
%% \semFcomp(\beta,\beta');
%% \semrtol{(\semrtol{(\semcase{\semltor{{\sem{D_1}}}{\beta}}
%%                            {\semltor{{\sem{D_2}}}{\beta}})}
%%   {\beta};{(\semltor{\sem{E}}{\beta'})})}{\beta'}
%% \]
%% and then
%% \[
%% \semFcomp(\beta,\beta');
%% \semF{\beta'}{
%% (\semrtol{(\semcase{\semltor{{\sem{D_1}}}{\beta}}
%%                            {\semltor{{\sem{D_2}}}{\beta}})}
%%   {\beta})};
%% \semrtol{{(\semltor{\sem{E}}{\beta'})}}{\beta'}
%% \]
%% so collapsing inverses gives the result.  

%% \end{itemize}

%% \item For 
%% {\ap D {\Case{\cut{(\Inl{\ident{A}})}{D}}{\cut{(\Inr{\ident{B}})}{D}}}}
%% (where {D : \seq{\coprd{A}{B}}{\alpha}{C}}),
%% we need to show that
%% \[
%% \semrtol{\semcase{\semltor{(\semFcomp(1,\alpha);\semF{\alpha}{\semFone};\semF{\alpha}{\seminl};\sem{D})}{\alpha}}{\semltor{(\semFcomp(1,\alpha);\semF{\alpha}{\semFone};\semF{\alpha}{\seminr};\sem{D})}{\alpha}}}{\alpha}
%% \]
%% By coherence, $\semFcomp(1,\alpha) = \inv{\semF{\alpha}{\semFone}}$, so
%%   this is
%% \[
%% \semrtol{\semcase{\semltor{(\semF{\alpha}{\seminl};\sem{D})}{\alpha}}{\semltor{(\semF{\alpha}{\seminr};\sem{D})}{\alpha}}}{\alpha}.
%% \]
%% By naturality of the adjunction, this is 
%% \[
%% \semrtol{\semcase{{\seminl};\semltor{\sem{D}}{\alpha}}{{\seminr};\semltor{\sem{D}}{\alpha}}}{\alpha}.
%% \]
%% which by uniqueness for coproducts is 
%% \[
%% \semrtol{(\semltor{\sem{D}}{\alpha})}{\alpha}.
%% \]
%% so collapsing inverses gives the result.  

%% \item The rule \ap{\Inl{\UL{\gamma}{e}{D}}}{\UL{\gamma}{e}{\Inl D}} (and
%%   the analogous rule for \Inr{D}) is immediate by
%%   Lemma~\ref{lem:semulalt}, because \sem{\UL{\gamma}{e}{D}} precomposes
%%   \sem{D} with something, and \sem{\Inl{D}} postcomposes \sem{D} with
%%   \seminl.

%% \end{itemize}
%% \end{proof}

\section{Example Mode Theories}
\label{sec:triple}

\subsection{Adjoint Triple}

Consider the walking adjunction $\dsd d \la \dsd n$, which has

\begin{itemize}
\item objects $\dsd{c}$ and $\dsd s$
\item 1-cells $\dsd{d} : \dsd s \ge \dsd c$ and $\dsd{n} : \dsd c \ge
  \dsd s$
\item 2-cells $\dsd{unit} : \tc {1_{\dsd c}} {\compo{\dsd{n}} {\dsd{d}}}$ 
and $\dsd{counit} : \tc {\compo{\dsd{d}} {\dsd{n}}} {1_{\dsd s}}$ satisfying \\
$\compv{(\comph{1_{\dsd{d}}}{\dsd{unit}})}{(\comph{\dsd{counit}}{1_{\dsd d}})} = 1$
and 
$\compv{(\comph{\dsd{unit}}{1_{\dsd{n}}})}{(\comph{1_{\dsd n}}{\dsd{counit}})} = 1$.
\end{itemize}
\noindent
The 1-cells specify two adjunctions $\F{\dsd d}{} \la \U{\dsd d}{}$ and
$\F{\dsd n}{} \la \U{\dsd n}{}$.  However, the functoriality of $\F{}{}$
and $\U{}{}$ on 2-cells also yields adjunctions $\F{\dsd d}{} \la
\F{\dsd n}{}$ and $\U{\dsd d}{} \la \U{\dsd n}{}$.  Since a right or
left adjoint of a given functor is unique up to isomorphism, it follows
that the two functors $\U{\dsd d}{},\F{\dsd n}{} : \dsd c \to \dsd s$
are equal, resulting in an adjoint triple $\F{\dsd d}{} \la (\U{\dsd
  d}{} \iso \F{\dsd{n}}{}) \la \U{\dsd{n}}{}$.  However, rather than
proving $\F{\dsd d}{} \la \F{\dsd n}{}$ or $\U{\dsd d}{} \la \U{\dsd
  n}{}$ and then concluding $\U{\dsd d}{} \iso \F{\dsd n}{}$ from
uniqueness of adjoints, we can construct the isomorphism directly:

\begin{lemma} \label{lem:mergeFU}
$\U{\dsd d}{A} \iso \F{\dsd n}{A}$ naturally in $A$.
\end{lemma}

\begin{proof}
%% One way to define the maps is to use the constructions of
%% Theorem~\ref{thm:syntacticpseudofunctor} and
%% Lemma~\ref{lem:constructionsonadjunctions} (the adjunction, the isomorphisms for $F/U$
%% on $1$ and $\compo{}{}$, and the action of $F/U$ on 2-cells, the comonad
%% structure):
%% \begin{diagram}
%% \U{\dsd d}{A} & \rTo & \F{1}{\U{\dsd d}{A}} & \rTo^{\F{\dsd{counit}}{}} & \F{\compo{\dsd d}{\dsd n}}{\U{\dsd d} A} & \rTo & \F{\dsd n}{(\F{\dsd d}{\U{\dsd d}{A}})} & \rTo & \F{\dsd n}{(A)}\\
%% \end{diagram}
%% \begin{diagram}
%% \text{For $\F{\dsd n}{A} \to \U{\dsd d}{A}$, transpose } &
%% A & \rTo & \U{1}{A} & \rTo^{\U{\dsd{unit}}{}} & \U{\compo{\dsd n}{\dsd d}}{A} & \rTo & \U{\dsd n}{\U{\dsd d}{A}}
%% \end{diagram}
We can write the maps as follows:
\[
\begin{array}{c}
\infer[\dsd{FR}]
      {\seq{\U{\dsd d}{A}}{1}{\F{\dsd n}{A}}}
      {\dsd{d} : \dsd s \ge \dsd c &
            \dsd{counit} : \tc{\compo{\dsd{d}}{\dsd{n}}}{1} &
            \infer[\irl{UL}]
                  {\seq{\U{\dsd d}{A}}{\dsd d}{A}}
                  {1 : \dsd c \ge \dsd c &
                   1 : \tc{\dsd{d}}{\dsd d} &
                   \infer[\dsd{ident}]
                         {\seq{A}{1}{A}}
                         {}
                  }
      }
\\ \\
\infer[\dsd{FL}]
      {\seq{\F{\dsd n}{A}}{1}{\U{\dsd d}{A}}}
      {\infer[\dsd{UR}]
        {\seq{A}{\dsd n}{\U{\dsd d}{A}}}
        {\infer[\tr{-}{-}]{\seq{A}{\compo{\dsd n}{\dsd d}}{A}}
               {
                 \dsd{unit} : \tc 1 {\compo{\dsd{n}}{\dsd{d}}} &
                 \infer[\dsd{ident}]
                       {\seq{A}{1}{A}}{}}}}
\end{array}
\]
In the Agda code, we verify that these are inverse and natural.  
\end{proof}

We can develop some of the expected properties of an adjoint triple $L
\la M \la R$, such as the fact that the ``left'' comonad $LM$ is itself
left adjoint to the ``right'' monad $RM$, and consequently, $LM$
preserves colimits.  In this case, we have $L = \F{\dsd{d}}{}$, $M =
\U{\dsd{d}}{} \cong \F{\dsd{n}}{}$, and $R = \U{\dsd{n}}{}$, and we write
$\Bx{\dsd{d}}{A} := \F{\dsd{d}}{\U{\dsd{d}}{A}}$
and 
$\Crc{\dsd{n}}{A} := \U{\dsd{n}}{\F{\dsd{n}}{A}}$.  


\begin{theorem}[Properties of an adjoint triple] \label{thm:adjointtriple}
\begin{enumerate}
\item $\Bx{\dsd{d}}{} \la \Crc{\dsd{n}}{}$ 
\item $\Bx{\dsd{d}}{(\coprd A B)} \iso \coprd{\Bx{\dsd{d}}{A}}{\Bx{\dsd{d}}{A}}$
\end{enumerate}
\end{theorem}

\begin{proof}
%% Using the fact that functors preserve natural isomorphisms,
%% Lemma~\ref{lem:mergeFU}, and properties of $U$ and $F$ from
%% Theorem~\ref{thm:syntacticpseudofunctor}
We can prove that \Bx{\dsd d}{A} and \Crc{\dsd n}{A} are isomorphic to a
single $F$ and $U$, respectively:
\[
\begin{array}{l}
\Bx{\dsd d}{A} = \F{\dsd{d}}{\U{\dsd d}{A}} \cong \F{\dsd d}{\F{\dsd n}{A}} \cong \F{\compo{\dsd{n}}{\dsd{d}}}{A}\\
\Crc{\dsd{n}}{A} = \U{\dsd{n}}{\F{\dsd n}{A}} \cong \U{\dsd n}{\U{\dsd d}{A}} \cong \U{\compo{\dsd{n}}{\dsd d}}{A}
\end{array}
\]
This implies the above properties because
$\F{\compo{\dsd{n}}{\dsd{d}}}{} \la \U{\compo{\dsd{n}}{\dsd d}}{}$ and
$\F{\compo{\dsd{n}}{\dsd{d}}}$ preserves coproducts and these facts
respect natural isomorphism.
\end{proof}

From a polarity point of view, it is unusual for a comonad
$\F{}{\U{}{A}}$ to preserve positives, because the negative connective
$\U{}{}$ interrupts focus/inversion phases.  Here, this behavior is
explained by the fact that $\F{\dsd{d}}{\U{\dsd{d}}{A}}$ is isomorphic
to a single positive connective $\F{\compo{\dsd n}{\dsd d}}{A}$.  The
ambipolar middle connective in an adjoint triple thus \emph{emerges}
from the presence of two isomorphic connectives, one positive and one
negative.

\subsection{Reflection}

In our motivating example of axiomatic cohesion, the adjoint triple
$\Delta \la \Gamma \la \nabla$ has some additional properties.  We now
write $\Flat{}$ for the comonad $\Delta\Gamma$ and $\Sharp{}$ for the
monad $\nabla\Gamma$. $\Flat{}$ takes a cohesive space and
``retopologizes'' it with the discrete cohesion, while $\Sharp{}$ takes
a cohesive space and retopoligizes it with the codiscrete cohesion.
Intuitively, retopologizing twice should be the same as retopologizing
once, because each retopologization forgets the existing cohesive
structure; that is, we want $\Flat{\Flat{A}} \iso \Flat A$ and
$\Sharp{\Sharp{A}} \iso \Sharp A$ and $\Flat{\Sharp{A}} \iso \Flat A$ and
$\Sharp{\Flat{A}} \iso \Sharp A$.  Moreover, $\Delta$ and $\nabla$ should
be full and faithful, because a map between discrete or codiscrete
spaces is exactly a map of sets.

We can capture these properties by considering a different mode
2-category, the ``walking reflection''.  This has the same objects and
morphisms as above, but we now take $\compo{\dsd d}{\dsd n} = 1$, with
the counit being just the identity 2-cell, and the equations for
$\dsd{unit} : \tc {1} {\compo{\dsd n}{\dsd d}}$ simplify to
$\comph{\dsd{unit}}{1_{\dsd n}} = 1$ and $\comph{1_{\dsd d}}{\dsd{unit}}
= 1$.  Note that the only non-identity morphisms of this mode category
are \dsd{d}, \dsd{n}, and \compo{\dsd{n}}{\dsd{d}}.  

We write $\Delta := \F{\dsd{d}}{}$, $\Gamma := (\U{\dsd d}{} \iso
\F{\dsd n}{})$, and $\nabla := \U{\dsd{n}}{}$, so $\Flat{} =
\Bx{\dsd{d}}{}$ and $\Sharp{} = \Crc{\dsd{n}}{}$.  Since in particular
we still have an adjunction, this mode theory inherits all the theorems
from the previous section; but it also has the following additional
properties:

\begin{theorem}[Properties of the walking reflection]
\begin{enumerate}
\item $\Flat{\Flat A} \iso \Flat A$ and $\Sharp{\Sharp A} \iso \Sharp A$
  naturally in $A$.
\item $\Sharp{\Flat A} \iso \Sharp{A}$ 
and $\Flat{\Sharp A} \iso \Flat{A}$ naturally in $A$.
\item \F{\dsd{d}}{} and \U{\dsd{n}}{} are full and faithful.
\end{enumerate}
\end{theorem}

\begin{proof}

We discuss the first two parts. Using
Theorem~\ref{thm:syntacticpseudofunctor}, the equality of morphisms
$\compo{\dsd d}{\dsd n} = 1$ implies that
\[
\begin{array}{l}
\F{\dsd{n}}{\F{\dsd{d}}{A}} \iso \F{\compo{\dsd d}{\dsd n}}{A} = \F{1}{A} \iso A\\
\U{\dsd{d}}{\U{\dsd{n}}{A}} \iso \U{\compo{\dsd d}{\dsd n}}{A} = \U{1}{A} \iso A\\
\end{array}
\]
Consequently, by Lemma~\ref{lem:mergeFU},
the other (co)monads besides $\flat$ and $\sharp$ are
trivial: 
\[
\begin{array}{l}
\Crc{\dsd{d}}{A} = \U{\dsd{d}}{\F{\dsd d}{A}} \iso \F{\dsd n}{\F{\dsd d}{A}} \iso A\\
\Bx{\dsd{n}}{A} = \F{\dsd{n}}{\U{\dsd n}{A}} \iso \U{\dsd d}{\U{\dsd n}{A}} \iso A
\end{array}
\]

This gives idempotence and absorption:
\[
\begin{array}{l}
\Flat{\Flat A} = \F{\dsd d}{(\U {\dsd d} {\F{\dsd d}{(\U {\dsd d} A)}})} \cong \F{\dsd d}{\U {\dsd d} A} = \Flat A\\
\Sharp{\Sharp A} = \U{\dsd n}{(\F {\dsd n} {\U{\dsd n}{(\F {\dsd n} A)}})} \cong \U{\dsd n}{\F {\dsd n} A} = \Sharp A\\
\Flat{\Sharp A} = \F{\dsd d}{(\U{\dsd d}{\U{\dsd n}{(\F{\dsd n}{A})}})} 
\cong \F{\dsd d}{\F{\dsd n}{A}}
\cong \F{\compo{\dsd n}{\dsd d}}{A} \cong \Flat A \\

\Sharp{\Flat A} = \U{\dsd n}{(\F{\dsd n}{\F{\dsd d}{(\U{\dsd d}{A})}})} 
\cong \U{\dsd n}{\U{\dsd d}{A}}
\cong \U{\compo{\dsd n}{\dsd d}}{A} \cong \Sharp A \\
\end{array}
\]

%% Finally, we check that \F{\dsd d}{} and \U{\dsd n}{} are full and
%% faithful.  This follows by general category-theoretic arguments from
%% the triviality of $\Crc{\dsd d}{}$ and $\Bx{\dsd n}{}$ (see~\cite[\S
%% IV.3]{maclane98working}), but to avoid needing to prove the Yoneda
%% lemma in our formalization, we give an explicit argument instead.
%% Consider \F{\dsd d}{}; an analogous argument works for \U{\dsd
%%   n}{}.  We want to show that the functoriality on derivations $\F{\dsd
%%   d}{(D : \seq{A}{1}{B})} : \seq{\F{\dsd d}{A}}{1}{\F{\dsd d}{A}}$ is a
%% bijection.  Above, we showed that \F{\dsd{n}}{} is post-inverse to
%% \F{\dsd d}{}---we have a natural isomorphism $i : \F{\dsd n}{\F{\dsd d}
%%   A} \iso A$.  Therefore we can map a derivation $D : \seq{\F{\dsd
%%     d}{A}}{1}{\F{\dsd d}{B}}$ to a derivation of \seq{A}{1}{B} by
%% \begin{diagram}
%% A & \rTo^{{\inv i}} & \F{\dsd n}{\F{\dsd d} A} & \rTo^{\F{\dsd n}{D}} & \F{\dsd n}{\F{\dsd d} B} & \rTo^{i} B
%% \end{diagram}
%% By naturality of $i$, this function is post-inverse to \F{\dsd d}{}
%% (i.e. $\ap{(\inv{i} \cuti (\F{\dsd n}{\F{\dsd d} D}) \cuti i)}{D}$),
%% which shows that \F{\dsd d}{} is faithful.  To show that it is full, we
%% need to check the other composite, which simplifies to checking that 
%% for $D' : \seq{\F{\dsd d}A}{1}{\F{\dsd d}B}$,
%% \[
%% \ap {(\F{\dsd d}{\F{\dsd n} D'}) \cuti \F{\dsd d}{i}} { \F{\dsd d}{i} \cuti D'}
%% \]
%% (i.e. that $\F{\dsd d}{i}$ is a natural isomorphism between the
%% inclusion functor from the full subcategory whose objects are of the
%% form \F{\dsd d}{A}, and the functor \F{\dsd d}{\F{\dsd n}{}} restricted
%% to this subcategory).  To show this, we prove that \F{\dsd d}{i} is
%% equal to the $\F{\dsd d}{A}$ component of the following natural
%% transformation $j$, which uses $\dsd{unit} :
%% \tc{1}{\compo{\dsd{n}}{\dsd{d}}}$
%% \begin{diagram}
%% \F{\dsd d}{\F{\dsd n}{A}} & \rTo & 
%% \F{\compo{\dsd n}{\dsd d}}{A} & \rTo^{\F{\dsd{unit}}{}} &
%% \F{1}{A} & \rTo & 
%% A
%% \end{diagram}
%% The fact that $\ap{\F{\dsd{d}}{i}}{j_{\F{\dsd d}{A}}}$ follows from the
%% pseudofunctor associativity/unit coherences (used to show $\ap{\F{\dsd
%%     d}{i}}{ (F^\circ(\dsd n,\dsd d) \cuti F^\circ(\dsd{d},\compo{\dsd
%%     n}{\dsd d}))}$ and $F^\circ(\dsd d,1) = F^1_{\F{\dsd{d}}{A}}$) and
%% naturality of the composition isomorphism (used to show
%% $\ap{\F{\dsd{unit}}{} \cuti F^\circ(\dsd
%%   d,1)}{F^\circ(\dsd{d},\compo{\dsd n}{\dsd d})}$). Then $\ap {(\F{\dsd
%%     d}{\F{\dsd n} D'}) \cuti j} { j \cuti D'}$ is exactly the naturality
%% square for $j$.

\end{proof}

\subsection{Spatial Type Theory}
\label{sec:idempotent}

The above mode theory allows us to work with cohesive types (which have
mode \dsd{c}) and non-cohesive types (which have mode \dsd{s}).
However, because $\Delta$ and $\nabla$ are full and faithful, it is not
strictly necessary to ever work in \dsd{s} itself---we could
equivalently work in the image of $\Delta$ or $\nabla$ in \dsd{c}.  If
we wish to restrict ourselves to constructions in \dsd{c}, we can
simplify the mode theory to the (strictly) idempotent monad:
\begin{itemize}
\item object \dsd{c}
\item 1-cell $\dsd{r} : \dsd{c} \ge \dsd{c}$ such that
  $\compo{\dsd{r}}{\dsd{r}} = 1$, so the only 1-cells are $1$ and \dsd{r}
\item 2-cell $\dsd{unit} : \tc{1}{\dsd{r}}$ satisfying $\comph{\dsd
  t}{\dsd{unit}} = 1$ and $\comph{\dsd{unit}}{\dsd{r}} = 1$, so 
  the only 2-cells are $1_1$, $1_{\dsd{r}}$, and  $\dsd{unit}$. 
\end{itemize}
This mode theory embeds in the walking reflection, with $\dsd{r} :=
\compo{\dsd{n}}{\dsd{d}}$, so we could equivalently work in the
\dsd{c}-types above.

For this mode theory, we define $\Flat{} := \F{\dsd{r}}{}$ and $\Sharp{}
:= \U{\dsd{r}}{}$.  In the walking reflection, we defined $\Flat{} :=
\Bx{\dsd{d}}{}$ and $\Sharp{} := \Crc{\dsd{n}}{}$ and then proved (in
the proof of Theorem~\ref{thm:adjointtriple}) that $\Flat{} \iso
\F{\compo{\dsd n}{\dsd d}}$ and $\Sharp{} \iso \U{\compo{\dsd n}{\dsd
    d}}$. Here, we take the other side of this isomorphism as the
definition, so we immediately have $\Flat{} \la \Sharp{}$ and $\Flat{}$
preserves coproducts, but we must prove that they are (co)monads.  A
simple route to this is to prove absorption, because $\Flat{\Sharp A} =
\F{\dsd{r}}{\U{\dsd{r}}{A}}$ is a comonad, and dually for $\Sharp{\Flat
  A}$.  

\begin{theorem}[Idempotence and Absorption]
 $\Flat{\Flat A} \iso \Flat A$ and $\Sharp{\Sharp A} \iso \Sharp A$
and $\Sharp{\Flat A} \iso \Sharp{A}$ 
and $\Flat{\Sharp A} \iso \Flat{A}$ naturally in $A$.
\end{theorem}
\begin{proof}
Because $\compo{\dsd r}{\dsd r} = {\dsd r}$, idempotence is just the composition
isomorphisms $F^\circ$ and $U^\circ$ from
Theorem~\ref{thm:syntacticpseudofunctor}.  The absorption isomorphisms
are constructed directly; see our Agda formalization.
\end{proof}

In the extended version of this paper, we connect this mode theory to
the rules for spatial type theory used in
~\citet{shulman15realcohesion}.


%% \subsubsection{Restricted Sequent Calculus I}

%% The general sequent calculus rules allow choices of $\gamma$ and $e$ in
%% \irl{FR} and \irl{UL}.  However, for this specific mode theory, no
%% choices are necessary: we can without loss of generality use the
%% following instances of the sequent calculus rules:
%% \[
%% \begin{array}{c}
%% \infer[\hyp{1_{1}}]
%%       {\seq{P}{\dsd{1}}{P}}
%%       {}
%% \quad
%% \infer[\hyp{\dsd{unit}}]
%%       {\seq{P}{\dsd{r}}{P}}
%%       {}
%% \\ \\
%% \infer[\UR{-}]
%%       {\seq{A}{\_}{\Sharp{B}}}
%%       {\seq{A}{\dsd{r}}{B}}
%% \quad
%% \infer[\UL{\dsd{r}}{1}{-}]
%%       {\seq{\Sharp A}{\dsd{r}}{B}}
%%       {\seq{A}{\dsd{r}}{{B}}}
%% \quad
%% \infer[\FR{\dsd{r}}{1}{-}]
%%       {\seq{A}{\dsd{r}}{\Flat B}}
%%       {\seq{A}{\dsd{r}}{B}}
%% \quad
%% \infer[\FL{-}]
%%       {\seq{\Flat A}{\dsd{\_}}{B}}
%%       {\seq{A}{\dsd{r}}{B}}
%% \\ \\
%% \infer[\UR{-}]
%%       {\seq{A}{\alpha}{\U{1}{B}}}
%%       {\seq{A}{\alpha}{B}}
%% \quad
%% \infer[\UL{\alpha}{1}{-}]
%%       {\seq{\U{1} A}{\dsd{\alpha}}{B}}
%%       {\seq{A}{\dsd{\alpha}}{{B}}}
%% \quad
%% \infer[\FR{\alpha}{1}{-}]
%%       {\seq{A}{\alpha}{\F{1} B}}
%%       {\seq{A}{\alpha}{B}}
%% \quad
%% \infer[\FL{-}]
%%       {\seq{\F{1} A}{\alpha}{B}}
%%       {\seq{A}{\alpha}{B}}
%% \end{array}
%% \]
%% The intuitive reading of the restricted rules is as
%% follows.  We have two kinds of assumptions, cohesive ($A[1_\dsd{c}]$)
%% and \emph{crisp} ($A[\dsd r]$, \dsd{r} for cRisp).  A crisp assumption
%% is the judgemental analogue of \Flat{A}---i.e. it means we know $A$
%% retopologized with the discrete cohesion.  The \Flat{} left rule says
%% that if we know \Flat{A} either crisply or cohesively, then we know $A$
%% crisply.  The \Flat{} right rule says that we can map into \Flat{B} by
%% mapping into $B$, as long as the assumption is already crisp---this
%% restriction prevents using a cohesive variable to map into a discrete
%% type.  The \Sharp{} right rule says that if we are mapping into the
%% codiscretization of $B$, then we can make the assumption crisp.  The
%% \Sharp{} left rule says that knowing the codiscretization crisply is the
%% same as knowing $A$ crisply, because in either case we know $A$ under
%% the discrete retopologization.

%% \begin{theorem}
%% For all $D : \seq{A}{\alpha}{B}$, there is a derivation $D' :
%% \seq{A}{\alpha}{B}$ that uses only the above rules and satisfies $\ap D {D'}$.  
%% \end{theorem}

%% \begin{proof}
%% For this mode theory, the only mode is \dsd{c}, the only 1-cells are 
%% 1 and \dsd{r}, and the only 2-cells are $1_1$ and $1_{\dsd r}$ and
%% \dsd{unit}.  The case for \hyp{e} is immediate, since the only 2-cells
%% from $1$ are $1_1$ and \dsd{unit}, and we have included both of those.  

%% We have included rules for $F$ and $U$ for both 1-cells $1$ and \dsd{r};
%% we discuss why the rules for $\Sharp{}$ are complete ($\Flat{}$ and
%% $\F{1}{}$ and $\U{1}{}$ are analogous).  For \irl{UR}, a general
%% instance of the rule will have some $\beta$ in the conclusion, and pass
%% to $\compo{\dsd r}{\beta}$ in the premise.  But $\beta$ must be $1$ or
%% \dsd{r}, and in either case the premise is \dsd{r}, so the IH gives the
%% result.  For \irl{UL}, a general instance will have the form
%% \[
%% \infer[\irl{UL}]
%%       {\seq {\U {\dsd{r}} A} {\beta} {C}}
%%       { \tc{\compo{\dsd r}{\gamma}} {\beta} &
%%         \seq{A}{\gamma}{C}}
%% \]
%% Here $\gamma$ can be $1$ or \dsd{r}, but in either case $\beta$ must
%% have been \dsd{r}, because for either value of $\gamma$ we have
%% $\compo{\dsd r}{\gamma} = \dsd{r}$ and there is no 2-cell \tc{\dsd
%%   r}{1}.  Moreover, the only 2-cell \tc{\dsd{r}}{\dsd{r}} is 1, so the
%% derivation must be of the form
%% \[
%% \infer[\UL{\gamma}{1}{-}]
%%       {\seq {\U {\dsd{r}} A} {\dsd{r}} {C}}
%%       {\seq{A}{\gamma}{C}}
%% \]
%% By the IH we have an equivalent derivation $D' : \seq{A}{\gamma}{C}$
%% that uses only the specialized rules.  If $\gamma$ is \dsd{r}, this
%% gives the result.  If $\gamma$ is 1, then we can make a derivation 
%% \[
%% \infer[\UL{\dsd{r}}{1}{-}]
%%       {\seq {\U {\dsd{r}} A} {\dsd{r}} {C}}
%%       {\infer[\tr{\dsd{unit}}{-}]
%%              {\seq{A}{\dsd{r}}{C}}
%%              {
%%               D' : \seq{A}{1}{C}
%%              }
%%       }
%% \]
%% and show using the equational rules that
%% \ap {\UL{\gamma}{1}{D'}} {\UL{\dsd r}{1}{\tr{\dsd{unit}}{D'}}}.
%% It remains to show a lemma that \tr{\dsd{unit}}{D} is closed under the
%% restricted rules, which can be proved by induction on $D$.  The complete
%% proof of this theorem is about 250 lines of Agda.  
%% \end{proof}

%% \newcommand\addcrisp[1]{\ensuremath{#1\mathord{\uparrow}^{\dsd{crisp}}}}
%% \newcommand\dropcoh[1]{\ensuremath{#1\mathord{\downarrow}_{\dsd{coh}}}}

%% For the restricted rules, there is a natural generalization to a
%% multiple assumption sequent, where each of the assumptions is either
%% cohesive or crisp.  We write $\Gamma$ for a context consisting of
%% assumptions $A[1]$ or $A[\dsd r]$. We write \addcrisp{\Gamma} for a
%% context where each cohesive assumption $A[1]$ has been promoted into a
%% crisp assumption $A[\dsd r]$, and each $A[\dsd{r}]$ has been kept the
%% same. We write \dropcoh{\Gamma} for a context where each cohesive
%% assumption $A[1]$ has been dropped, and each crisp assumption $A[r]$ has
%% been kept the same.  Then the multi-assumptioned rules are as follows:
%% \[
%% \infer{{\Gamma} \vdash {P}}
%%       {P[1] \in \Gamma}
%% \quad
%% \infer{{\Gamma} \vdash {P}}
%%       {P[\dsd{r}] \in \Gamma}
%% \quad
%% \infer{\Gamma \vdash {\Sharp{B}}}
%%       {\addcrisp {\Gamma} \vdash {B}}
%% \quad
%% \infer{\Gamma, {\Sharp A}[\dsd{r}] \vdash {B}}
%%       {\Gamma, A[\dsd{r}] \vdash {{B}}}
%% \quad
%% \infer{\Gamma \vdash {\Flat B}}
%%       {\dropcoh \Gamma \vdash {B}}
%% \quad
%% \infer{\Gamma, {\Flat A}[\dsd{\_}] \vdash {B}}
%%       {\Gamma, {A}[\dsd{r}] \vdash {B}}
%% \]
%% The left rules are just the single-assumption left rules in context.
%% The single-assumption \Sharp{} right rule promotes the one assumption to
%% crisp; here, we promote all assumptions.  The single assumption \Flat{}
%% right rule insists that the one assumption be crisp; here, we drop all
%% non-crisp assumptions.  The rules for \Flat{} are reminiscent of the
%% $\Bx{}{A}$ modality in \citet{pfenningdavies}, with crisp corresponding
%% to the valid judgement, except here we can apply a left rule to a
%% valid/crisp assumption of $\Bx{}{A}$, because \Flat{} is an $F$ which is
%% a positive-to-positive connective.  The right rule for \Sharp{} is
%% reminiscent of the right rule for the proof irrelevance/erasability
%% modality $[A]$ in \citet{pfenning01proofirrel,pfenning08proofirrel},
%% though the left rule is different.\footnote{The $[A]$ type in that work
%%   is a product-preserving functor with no additional properties, so it
%%   cannot be mapped into adjoint logic in a straightforward way, because
%%   all of our functors have more properties.}  Like in
%% Figure~\ref{fig:coprod}, the left rule for coproducts would allow
%% elimination on a coproduct in either mode:
%% \[
%% \infer{\Gamma \vdash {\coprd{A}{B}}}
%%       {\Gamma \vdash {A}}
%% \quad
%% \infer{\Gamma \vdash {\coprd{A}{B}}}
%%       {\Gamma \vdash {B}}
%% \quad
%% \infer{\Gamma, {(\coprd{A}{B})} [\alpha] \vdash {C}}
%%       {\Gamma, {(\coprd{A}{B})} [\alpha], A [\alpha] \vdash {C} & 
%%        \Gamma, {(\coprd{A}{B})} [\alpha], B [\alpha] \vdash {C} & 
%%       }
%% \]

%% \subsubsection{Restricted Sequent Calculus II}

%% All of the sequent calculi that we have defined so far, including the
%% above restricted sequent calculus for this mode theory, are well-behaved
%% from a proof-theoretic point view, in that they have the subformula
%% property and the identity and cut rules are admissible.  The cut
%% admissibility algorithm takes certain left-commutative cuts as
%% definitional equalities; e.g. we have
%% \[
%% \cut{(\Case{D_1}{D_2})}{E} =\Case{\cut{D_1}{E}}{\cut{D_2}{E}} \text{ if $E$ is not a right rule}\\
%% \]
%% However, these left-commutative cuts are not typically taken as
%% definitional equalities in intensional dependent type theories, and
%% adding them may actually be undesirable from a semantic point view,
%% where we are interested in interpreting the type theory in as wide a
%% class of models as possible.  From a sequent calculus point of view, a
%% positive type in intensional type theory is treated more like a base
%% type/atomic proposition that is equipped with an elimination constant of
%% function type.  For example, for coproducts, the elimination constant is
%% \[
%% \dsd{case} : (A \to C) \to (B \to C) \to ((\coprd{A}{B}) \to C)
%% \]  
%% and this is often used via (iterated) implication-left:
%% \[
%% \infer{\Gamma \vdash D}
%%       {\Gamma \vdash A \to C & 
%%        \Gamma \vdash B \to C & 
%%        \Gamma \vdash \coprd{A}{B} &
%%        \Gamma, C \vdash D
%%       }
%% \]
%% This builds in a cut from the result type of the case $C$ to the overall
%% goal $D$, so the left-commutative cuts do not need to be commuted into
%% the case branches.  However, because the elimination constant is
%% polymorphic in $C$, the sequent calculus no longer has the subformula
%% property, and identity and cut are no longer admissible rules.  

%% \newcommand\rseq[3]{\ensuremath{#1 \, [ #2 ] \, \Vdash \, #3}}


%% We can give a sequent calculus in this style for the current mode theory:
%% \[
%% \begin{array}{l}
%% \infer{\rseq{A}{\dsd r}{A}}{}
%% \quad
%% \infer{\rseq{A}{1}{A}}{}
%% \quad
%% \infer{\rseq A { \compo {\beta} {\alpha}} C}
%%       {\rseq A {\beta} B &
%%        \rseq B {\alpha} C}
%% \\ \\
%% \infer{\rseq{A}{\alpha}{\Sharp B}}
%%       {\rseq{A}{\dsd r} B}
%% \quad
%% \infer{\rseq{\sharp A}{\dsd r} B}
%%       {\rseq{A}{\dsd r}{B}}
%% \quad
%% \infer{\rseq{A}{\dsd r}{\Flat B}}
%%       {\rseq{A}{\dsd r}{B}}
%% \quad
%% \infer{\rseq{\Flat A}{1}{C}}
%%       {\rseq{A}{\dsd r}{C}}
%% \\ \\
%% \infer{\rseq{C}{\alpha}{\coprd{A}{B}}}
%%       {\rseq{C}{\alpha}{A}} 
%% \quad
%% \infer{\rseq{C}{\alpha}{\coprd{A}{B}}}
%%       {\rseq{C}{\alpha}{B}} 
%% \quad
%% \infer{\rseq{(\coprd{A}{B})}{1}{C}}
%%       {\rseq{A}{1}{C} &
%%         \rseq{B}{1}{C}}
%% \end{array}
%% \]
%% On the first line, we have identity (for both crisp and cohesive
%% variables) and cut as derivable rules; the identity rules could be
%% restricted to positives and atoms and made admissible for negatives in
%% order to force $\eta$-expansion.  The rules for \Sharp{} and the right
%% rules for \Flat{} and \coprd{}{} are the same as in the previous sequent
%% calculus.  The left rules for positives are restricted to cohesive
%% variables, and the left rule for coproducts binds cohesive variables in
%% the branches.  

%% These restrictions correspond to having an elimination constant of
%% function type, and in the presence of cut and identity rules, suffice to
%% derive the left rules for crisp assumptions.  For \Flat{A}, this is just
%% precomposition with using a crisp variable:
%% \[
%% \infer{\rseq{\Flat{A}}{\dsd r}{C}}
%%       {\rseq{\Flat{A}}{\dsd r}{\Flat A} &
%%        \infer{\rseq{\Flat{A}}{1}{C}}
%%              {\rseq{A}{\dsd r}{C}}}
%% \] 
%% The fact that this works is quite specific to this mode theory, where
%% forgetting the {\dsd r}-ness of the assumption and then reintroducing it
%% arrives at the same place, because of idempotence.  For \coprd{A}{B}, we
%% need to use, \Sharp{} on the right to derive the more general rule:
%% \[
%% \infer{\rseq{\coprd{A}{B}}{\dsd r}{C}}
%%       {\infer{\rseq{\coprd{A}{B}}{1}{\Sharp C}}
%%              {\infer{\rseq{A}{1}{\Sharp{C}}}
%%                     {\rseq{A}{\dsd r}{C}} &
%%               \infer{\rseq{B}{1}{\Sharp{C}}}
%%                     {\rseq{B}{\dsd r}{C}} &
%%              } & 
%%        \infer{\rseq{\Sharp C}{\dsd r}{C}}
%%              {\infer{\rseq{C}{\dsd r}{C}}{}}
%%       }
%% \] 
%% Because they are derivable, we could of course add these as additional
%% rules, but this shows that it is not necessary to do so.  A motivation
%% for not including them as primitives is that the analogous crisp
%% elimination rules do not hold for all higher inductive types in cohesive
%% homotopy type theory, so it is more uniform never to include them.

%% This sequent calculus is sound and complete for the original one, and
%% every \ap{}{}-equivalence-class of sequent derivations in our original
%% calculus can be described using these rules:\footnote{ignoring or adding
%%   rules for \F{1}{} and \U{1}{}}
%% \begin{theorem}
%% There are functions from \seq{A}{\alpha}{B} to \rseq{A}{\alpha}{B}
%% and back, and the composite from \seq{A}{\alpha}{B} to itself is the
%% identity up to \ap{}{}.  
%% \end{theorem}

%% \begin{proof}
%% The proof is about 150 lines of Agda.  
%% \end{proof}

%% \newcommand\nd[3]{\ensuremath{#1 \, [ #2 ] \, \vdash^{\dsd{nd}} \, #3}}

%% Next, we convert this sequent calculus to a single-variable natural
%% deduction system:
%% \[
%% \begin{array}{c}
%% \infer{\nd{A}{\dsd r}{A}}{}
%% \quad
%% \infer{\nd{A}{1}{A}}{}
%% \\ \\
%% \infer{\nd{A}{\alpha}{\Sharp B}}
%%       {\nd{A}{\dsd r} B}
%% \quad
%% \infer{\nd{C}{\dsd r}{A}}
%%       {\nd{C}{\dsd r}{\Sharp A}}
%% \quad
%% \infer{\nd{C}{\dsd r}{\Flat A}}
%%       {\nd{C}{\dsd r}{A}}
%% \quad
%% \infer{\nd{C}{\alpha}{B}}
%%       {\nd{C}{\alpha}{\Flat A} &
%%         \nd{A}{r}{B}}
%% \\ \\
%% \infer{\nd{C}{\alpha}{\coprd{A}{B}}}
%%       {\nd{C}{\alpha}{A}} 
%% \quad
%% \infer{\nd{C}{\alpha}{\coprd{A}{B}}}
%%       {\nd{C}{\alpha}{B}} 
%% \quad
%% \infer{\nd{C}{\alpha}{D}}
%%       {\nd{C}{\alpha}{\coprd{A}{B}} &
%%        \nd{A}{1}{D} &
%%        \nd{A}{1}{D}}
%% \end{array}
%% \]

%% The cut/substitution rule
%% \[
%% \infer{\rseq A { \compo {\beta} {\alpha}} C}
%%       {\rseq A {\beta} B &
%%        \rseq B {\alpha} C}
%% \]
%% is not included, but it is admissible, because each rule allows for
%% precomposition.  

%% The natural deduction system is also sound and complete for the original
%% one, and the the proof factors through the above sequent calculus:
%% \begin{theorem}
%% There are functions 
%% $(\seq{A}{\alpha}{B}) \leftrightarrow (\rseq{A}{\alpha}{B}) \leftrightarrow (\nd{A}{\alpha}{B})$
%% and back, and the composite from \seq{A}{\alpha}{B} to itself is the
%% identity up to \ap{}{}.  
%% \end{theorem}

%% \begin{proof}
%% The proof is about 150 lines of Agda.  
%% \end{proof}

%% We can generalize these natural deduction rules to multiple assumptions
%% as follows, writing $\Delta$ for the crisp assumptions and $\Gamma$
%% for the cohesive assumptions:
%% \[
%% \begin{array}{c}
%% \infer{\Delta;\Gamma \vdash A}
%%       {A \in \Gamma}
%% \quad
%% \infer{\Delta;\Gamma \vdash A}
%%       {A \in \Delta}
%% \\ \\
%% \infer{\Delta;\Gamma \vdash {\Sharp B}}
%%       {\Delta,\Gamma; \cdot \vdash B}
%% \quad
%% \infer{\Delta;\Gamma \vdash A}
%%       {\Delta;\cdot \vdash {\Sharp A}}
%% \quad
%% \infer{\Delta; \Gamma \vdash {\Flat A}}
%%       {\Delta; \cdot \vdash {A}}
%% \quad
%% \infer{\Delta; \Gamma \vdash B}
%%       {\Delta; \Gamma \vdash {\Flat A} &
%%         \Delta,A; \Gamma \vdash B}
%% \\ \\
%% \infer{\Delta;\Gamma \vdash {\coprd{A}{B}}}
%%       {\Delta;\Gamma \vdash {A}} 
%% \quad
%% \infer{\Delta;\Gamma \vdash {\coprd{A}{B}}}
%%       {\Delta;\Gamma \vdash {B}} 
%% \quad
%% \infer{\Delta;\Gamma \vdash {C}}
%%       {\Delta;\Gamma \vdash {\coprd{A}{B}} &
%%        \Delta;\Gamma,A \vdash {C} &
%%        \Delta;\Gamma,B \vdash {C}}
%% \end{array}
%% \]
%% These rules, generalized to dependent types, are exactly the rules used
%% in an investigation of cohesive type theory in
%% \citet{shulman15realcohesion}.  Here, we have given a proof theoretic
%% explanation for them, by connecting them to a particular mode theory in
%% adjoint logic.  The sequent calculus and natural deduction systems
%% defined in this section allow arbitrary cuts/non-normal forms; a next
%% step is to investigate normal forms for the equational theory given by
%% only the $\beta$ rules for positives---the translation to the general
%% adjoint logic sequent calculus normalizes too much, because it also
%% performs left-commutative cuts.

\section{Conclusion}

In this paper, we have defined an adjoint logic that allows multiple
different adjunctions between the same categories, shown soundness and
completeness of the logic in pseudofunctors $\M \to \Adj$, and used some
specific mode theories to model adjoint triples and the $\Flat{}$ and
$\Sharp{}$ modalities of axiomatic cohesion.
%% While we have considered
%% only a single-hypothesis sequent through most of the paper, we discussed
%% a generalization to multiple hypotheses for the specific mode theory in
%% Section~\ref{sec:idempotent}, and the generalization of these rules to
%% dependent types is discussed in \citet{shulman15realcohesion}.  
One area for future work is to extend this adjoint logic with multiple
assumptions and dependent types (we discuss some special cases in the
extended version).  This would provide a context for investigating the
shape modality $\sh{} \la \Flat{}$.  We could certainly give a mode
theory with one mode and $\sh{} \la \Flat{} \la \Sharp{}$, or with two
modes and $\sh \la \Delta \la \Gamma \la \nabla$, but it remains to be
investigated whether this can provide the right properties for \sh{}
beyond adjointness.  On the one hand, too much might be true: \sh{} does
not preserve identity types, and the general dependently typed rules for
$F$ might force it to. On the other, too little might be true: for
applications such as relating the shape of the topological circle to the
homotopical circle, extra properties are needed, such as
$\sh{\mathbb{R}} \cong 1$.  Both of these issues can be addressed as in
\citep{shulman15realcohesion} by treating \sh{} not as an abstract
adjoint, of the kind we can represent using the mode 2-category, but as
a defined type (specifically, a higher inductive), which among other
things has the property that it is adjoint to $\Flat{}$ (adjoint logic /
is still essential for representing $\Flat{}$ and $\Sharp{}$
themselves).  Another area for future work is to consider
$\infty$-category semantics, rather than the 1-categorical semantics of
derivations that we have considered here.  A final area for future work
is to look for applications of other mode theories in our generalized
adjoint logic beyond the motivating example of cohesive homotopy type
theory.

\setlength{\bibsep}{-1.75pt} %% dirty trick: make this negative
{ \small
\linespread{0.50}
\bibliographystyle{abbrvnat}
\bibliography{drl-common/cs}
}


\end{document}

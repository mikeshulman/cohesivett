
\documentclass[letter,11pt] {article}

\usepackage{fullpage}

\usepackage{pslatex} %Times font
%\usepackage{apacite} %apa citation style
%\bibliographystyle{apacite}
%\usepackage[pdfborder={0 0 0}]{hyperref}%for hyperlinks without ugly boxes
%\usepackage{graphicx} %for figures
%\usepackage{enumerate} %for lists
%% \fancyhf{}
%% %% \fancyhead[l,lo]{\textit{Licata FY 2016 YIP Project Narrative}} %left top header
%% \fancyhead[r,ro]{\thepage} %right top header
\usepackage{pdfpages}
\usepackage{url}
\usepackage{tipa}
\usepackage{xspace}
\newcommand{\shape}{\ensuremath{\mathop{\text{\textesh}}}\xspace}
\let\mysharp\sharp
\renewcommand{\sharp}{\ensuremath{\mathop{\mysharp}}\xspace}
\let\myflat\flat
\renewcommand{\flat}{\ensuremath{\mathop{\myflat}}\xspace}
\usepackage{latexsym,amsmath,amsthm,amssymb,amscd}
\usepackage{tikz}
\usepackage{../drl-common/typesit}
\usepackage[sort,round]{natbib}
\usepackage{../drl-common/lagdatotex}
\usepackage{../drl-common/proof}

\newcommand{\transport}[3]{\ensuremath{\mathsf{transport}_{#1}(#2,#3)}}
\newcommand{\ua}[1]{\mathsf{ua}(#1)}

\renewcommand{\to}{\rightarrow}
%
\newcommand{\UU}{\ensuremath{\mathcal{U}}} 
\newcommand{\MM}{\ensuremath{\mathcal{M}}} 
\renewcommand{\L}{\ensuremath{\mathsf{L}}} 
\newcommand{\R}{\ensuremath{\mathsf{R}}} 
\newcommand{\W}{\ensuremath{\mathsf{W}}} 
\newcommand{\C}{\ensuremath{\mathcal{C}}} 
\newcommand{\sfC}{\ensuremath{\mathsf{C}}} 
\newcommand{\T}{\ensuremath{\mathbb{T}}} 
\newcommand{\NN}{\ensuremath{\mathbb{N}}} 
\newcommand{\Co}{\ensuremath{\mathcal{C}_0}} 
\newcommand{\Cs}{\ensuremath{\C^\sharp}}% C^sharp
\newcommand{\Sm}{\ensuremath{\mathcal{S}}}% smallness
\newcommand{\St}{\ensuremath{\mathsf{S}}}% set predicate
\newcommand{\SQ}{\ensuremath{\S}}% set quantifier
\newcommand{\SC}{\ensuremath{\mathcal{S}_\mathcal{C}}}% small objects
\newcommand{\E}{\ensuremath{\mc{E}}}
\newcommand{\Sets}{\ensuremath{\mathrm{Sets}}}
\newcommand{\cat}{\ensuremath{\mathrm{Cat}}}
\newcommand{\presh}[1]{\ensuremath{\Sets^{\op{#1}}}}
%
\newcommand{\down}[1]{\;\downarrow\!{(#1)}}
\newcommand{\bb}[1]{\mathbb{#1}}
\newcommand{\mc}[1]{\ensuremath{\mathcal{#1}}}
\newcommand{\pr}[1]{\ensuremath{P_{r}(#1)}}
\newcommand{\inc}{\ensuremath{\subseteq}} 
\newcommand{\incl}{\ensuremath{\hookrightarrow}} 
\newcommand{\mono}{\rightarrowtail}
%\newcommand{\epi}{\twoheadrightarrow}
%\newcommand{\iso}{\cong}
\newcommand{\aar}{\rightrightarrows}
\newcommand{\ccup}{\ensuremath{\cup}}
%\newcommand{\Y}{\ensuremath{\mc{Y}}}
\newcommand{\pow}{\ensuremath{\mc{P}}}
\newcommand{\elem}{\ensuremath{\in}}
\newcommand{\imp}{\ensuremath{\Rightarrow}}
\newcommand{\To}{\ensuremath{\Rightarrow}}
\newcommand{\iiff}{\quad\text{iff}\quad}
\newcommand{\colim}{\varinjlim}
%
% scott brackets
\newcommand{\sem}[1]{\ensuremath{[\![{#1}]\!]}}
\newcommand{\csem}[2]{\ensuremath{[\![{#1}\ |\ {#2}]\!]}}% with
%                                                         % context
%
% Operators
%% \DeclareMathOperator{\im}{im}
\newcommand{\sh}{\text{\textesh}}
%% \DeclareMathOperator{\sub}{Sub}
%% \DeclareMathOperator{\ssub}{SSub}
%% \DeclareMathOperator{\srel}{SRel}
\newcommand{\Hom}[3]{\ensuremath{\dsd{Hom}_{#1}(#2,#3)}}


\newcommand\co{\ensuremath{{\text{\tiny +}}}}
\newcommand\con{\ensuremath{{\text{\small -}}}}
\newcommand\inv{\ensuremath{{\text{\tiny $\pm$}}}}
\newcommand\flip[1]{\ensuremath{\overline{#1}}}

\newcommand\Ob[1]{\ensuremath{Ob(#1)}}
\newcommand\morphismtp[3]{\ensuremath{#2 \longrightarrow_{#1} #3}}

\newcommand\optm[1]{\ensuremath{\dsd{op}(#1)}}
\newcommand\letop[4]{\ensuremath{\dsd{letop}_{#1}^{#2}(#3,#4)}}

\newcommand\coretm[1]{\ensuremath{\dsd{core}(#1)}}
\newcommand\letc[3]{\ensuremath{\dsd{letc}_{#1}(#2,#3)}}

\newcommand{\sset}{\dsd{set}}
\newcommand{\set}{\dsd{set}}
\newcommand{\lst}[1]{\ensuremath{\dsd{list}(#1)}}
\newcommand{\comp}[2]{\ensuremath{#1 \circ #2}}

\newcommand{\refls}[0]{\ensuremath{\dsd{id}}}                                             


\newcommand\Id[3]{\ensuremath{\dsd{Id}_{#1}(#2,#3)}}
\newcommand\op[1]{\ensuremath{#1 ^{\dsd{op}}}}
\newcommand\invt[1]{\ensuremath{#1 ^{\dsd{core}}}}

\newcommand{\la}{\ensuremath{\dashv}}


\renewcommand{\sem}[1]{\ensuremath{ \llbracket #1 \rrbracket}}

\newcommand{\D}{\ensuremath{\mathcal{D}}}
\newcommand{\arrow}[3]{\ensuremath{#2 \longrightarrow_{#1} #3}}
\newcommand{\tc}[2]{\ensuremath{#1 \Rightarrow #2}}
\newcommand{\Adj}{\textbf{Adj}}

\newcommand\compo[2]{\ensuremath{#1 \cdot #2}}
\newcommand\compv[2]{\ensuremath{#1 \cdot #2}}
\newcommand\comph[2]{\ensuremath{#1 \mathbin{\circ_2} #2}}

%% \newcommand\wftp[2]{\ensuremath{#1 \,\,\, \dsd{type}_{#2}}}
\newcommand\F[2]{\ensuremath{\dsd{F}_{#1} \,\, #2}}
\newcommand\U[2]{\ensuremath{U_{#1} \,\, #2}}
\newcommand\coprd[2]{\ensuremath{#1 + #2}}
\newcommand\seq[3]{\ensuremath{#1 \, [ #2 ] \, \vdash \, #3}}
\newcommand\irl[1]{\dsd{#1}}

\newcommand\tr[2]{\ensuremath{{{#1}_{*}(#2)}}}
\newcommand\ident[1]{\ensuremath{\dsd{ident}_{#1}}}
\newcommand\cutsym{\ensuremath{\dsd{cut}}}
\newcommand\cut[2]{\ensuremath{{\cutsym \,\, #1 \,\, #2}}}
\newcommand\cuti{\ensuremath{\bullet}}

\newcommand\hyp[1]{\ensuremath{\dsd{hyp} \, {#1}}}
\newcommand\Inl[1]{\ensuremath{\dsd{Inl}(#1)}}
\newcommand\Inr[1]{\ensuremath{\dsd{Inr}(#1)}}
\newcommand\Case[2]{\ensuremath{\dsd{Case}(#1,#2)}}
\newcommand\UL[3]{\ensuremath{\dsd{UL}^{#1}_{#2}(#3)}}
\newcommand\FR[3]{\ensuremath{\dsd{FR}^{#1}_{#2}(#3)}}
\newcommand\FL[1]{\ensuremath{\dsd{FL}(#1)}}
\newcommand\UR[1]{\ensuremath{\dsd{UR}(#1)}}

\newcommand\ap[2]{\ensuremath{#1 \approx #2}}

\newcommand\Bx[2]{\ensuremath{\Box_{#1} \, {#2}}}
\newcommand\Crc[2]{\ensuremath{\bigcirc_{#1} \, {#2}}}

\newcommand\Flat[1]{\ensuremath{\flat \, {#1}}}
\newcommand\Sharp[1]{\ensuremath{\sharp \, {#1}}}

\newcommand\rseq[3]{\ensuremath{#1 \, [ #2 ] \, \Vdash \, #3}}

\newcommand\gm{\dsd{s}}
\newcommand\cm{\dsd{d}}
\newcommand\corem{\dsd{c}}
\newcommand\opm{\dsd{op}}
\newcommand\Forget[1]{\ensuremath{\dsd{Forget} \: #1}}
\newcommand\Core[1]{\ensuremath{\dsd{Core} \: #1}}
\newcommand\seqm[2]{\ensuremath{#1 \vdash #2}}

\newcommand\uncorep[0]{\ensuremath{\dsd{uncore}^\co}}
\newcommand\uncoren[0]{\ensuremath{\dsd{uncore}^\con}}

\newcommand\ectx[5]{\ensuremath{#1 , #2 \,^{#3}{:} \, #5}} %%supress opfibs
\newcommand\vtptm[4]{\ensuremath{\dcd{#1} \,^{#2}{:} \, \dcd{#4}}} %%supress opfibs
\newcommand\vvoftp[5]{\ensuremath{#1 \vdash #2 \, {\,^{#3}{:}} \, \dcd{#5}}} %%supress opfibs
\newcommand\wfsub[3]{\ensuremath{#1 \vdash #2 : #3}}
\newcommand\tsubst[2]{\ensuremath{#1 [ #2 ]}}

\newcommand{\forw}[1]{\ensuremath{\stackrel{\rightarrow}{#1}}}
\newcommand{\backw}[1]{\ensuremath{\stackrel{\leftarrow}{#1}}}
\newcommand\letfunc[4]{\ensuremath{\dsd{letfunc}^{#1}_{#2}(#3,#4)}}

\newcommand\inF[2]{\ensuremath{\dsd{in}_{#1}(#2)}}
\newcommand\outF[2]{\ensuremath{\dsd{out}_{#1}(#2)}}

\newcommand\idm[1]{\ensuremath{\dsd{id}_{#1}}}
\newcommand\morind[3]{\ensuremath{\dsd{hind}_{#1}(#2,#3)}}


%
% theorem styles
%\newtheorem{theorem}{Theorem}
%\newtheorem*{theorem*}{Theorem}
%\newtheorem{proposition}[theorem]{Proposition} 
%\newtheorem{lemma}[theorem]{Lemma}
%\newtheorem{corollary}[theorem]{Corollary} 
%\theoremstyle{remark}
%\newtheorem{remark}[theorem]{Remark} 
%\newtheorem{definition}[theorem]{Definition}

\newtheorem*{conjecture*}{Conjecture}
\newtheorem*{lemma}{Lemma}
%
%




%%----------------------------------------------------------------------
%% spacing hacks:

%%\setlength{\parskip}{0pt}
%% \setlength{\parsep}{0pt}
%% \setlength{\headsep}{8pt}
%% \setlength{\topskip}{0pt}
%% \setlength{\topmargin}{0pt}
%% \setlength{\topsep}{0pt}
%% \setlength{\partopsep}{0pt}

%% eliminate some of the whitespace around section titles
\usepackage[compact]{titlesec}  
%\linespread{0.97}


\newcommand{\noagda}[0]{}
\usepackage{ucs}
\usepackage[utf8x]{inputenc}
\usepackage[T1]{fontenc}
\DeclareUnicodeCharacter{8594}{$\shortrightarrow$}
\DeclareUnicodeCharacter{9001}{$\langle$}
\DeclareUnicodeCharacter{9002}{$\rangle$}
\DeclareUnicodeCharacter{12314}{$\llbracket$}
\DeclareUnicodeCharacter{12315}{$\rrbracket$}
\DeclareUnicodeCharacter{8872}{$\vDash$}
\DeclareUnicodeCharacter{9711}{\ensuremath{\bigcirc}}
\DeclareUnicodeCharacter{9675}{$\circ$}
\DeclareUnicodeCharacter{9671}{$\Diamond$}
\DeclareUnicodeCharacter{9657}{$\triangleright$}
\DeclareUnicodeCharacter{8640}{$\rightharpoonup$}
\DeclareUnicodeCharacter{8659}{$\Downarrow$}
\DeclareUnicodeCharacter{8984}{$\Shamrock$}
\DeclareUnicodeCharacter{8596}{$\leftrightarrow$}
\DeclareUnicodeCharacter{7522}{${}_i$}
\DeclareUnicodeCharacter{11388}{$\!{}_j$}
% Bullet and box, the heights are adjusted to make it look nice
\DeclareUnicodeCharacter{8226}{\raisebox{-1.6pt}{$\bullet$}}
\DeclareUnicodeCharacter{9726}{$\rule[-1pt]{3pt}{3pt}$}

\newcommand\textPsi{$\Psi$}
\newcommand\textpsi{$\psi$}
\newcommand\textXi{$\Xi$}
\newcommand\textxi{$\xi$}
\newcommand\textDelta{$\Delta$}
\newcommand\textGamma{$\Gamma$}
\newcommand\textsigma{$\sigma$}
\newcommand\textSigma{$\Sigma$}
\newcommand\textPi{$\Pi$}
\newcommand\textrho{$\rho$}
\renewcommand\textphi{$\varphi$}
\newcommand\texttau{$\tau$}
\newcommand\texteta{$\eta$}
\newcommand\textalpha{$\alpha$}
\renewcommand\textbeta{$\beta$}
\renewcommand\textepsilon{$\epsilon$}
\newcommand\textkappa{$\kappa$}
\newcommand\textOmega{$\Omega$}
\newcommand\textmho{$\mho$}
\renewcommand\textgamma{$\gamma$}
\renewcommand\textlambda{$\lambda$}
\newcommand\textpi{$\pi$}
\renewcommand\texttheta{$\theta$}
\newcommand\textdelta{$\delta$}

\usepackage{fancyvrb}

\usepackage{../drl-common/code}
\DefineVerbatimEnvironment{code}{Verbatim}{fontsize=\small,fontfamily=tt}

\newcommand{\ignore}[1]{}

%% small tightcode, with space around it
\newenvironment{stcode}
{\smallskip
\begin{small}
\begin{tightcode}}
{\end{tightcode}
\end{small}
\smallskip}

\title{Functor Logic}
\author{Dan Licata}

\begin{document}

\maketitle

\section{1-variable Sequent Calculus}

Like in adjoint logic, we assume a mode 2-category, with objects $p,q$,
1-cells $\alpha : p \le q$, and 2-cells \tc{\alpha}{\beta}.  For
reference, here is the restriction of the one-variable adjoint logic
sequent calculus that gives a pseudofunctor into Cat instead of Adj.
Note that I flipped the variance relative to adjoint logic; here \F{}{}
is covariant on both 1-cells and 2-cells.

\[
\begin{array}{c}
\infer[\irl{hyp}]
      {\seq P \alpha P}
      {\tc \alpha 1}
\quad
\infer[\irl{FL}]
      {\seq {\F {\alpha : p \le q} A_p} {\beta : q \le r}{C_r}}
      {\seq {A_p} {\compo{\alpha}{\beta}} {C_r}
      }
\quad
\infer[\irl{FR}]
      {\seq {C_p} {\beta : p \le r} {\F {\alpha : q \le r} A_q}}
      { \gamma : p \le q & \tc{\beta}{\compo{\gamma}{\alpha}} &
        \seq {C_p} \gamma {A_q}}
\end{array}
\]

The following rules should be admissible:

\[
\infer[\irl{identity}]
      {\seq {A} {1} {A}}{}
\qquad
\infer[\irl{cut}]
      {\seq{A} {\compo{\alpha}{\beta}} {C}}
      {\seq {A} {\alpha} {B} &
        \seq{B} {\beta} {C}
      }
\qquad
\infer[\irl{2cell}]
      {\seq{A} {\alpha} {B}}
      { \tc \alpha \beta &
        \seq {A} {\beta} {B}
      }
\]

\section{Natural Deduction with Modal Hypotheses Only}

\newcommand\ndh[2]{\ensuremath{#1 \vdash #2}}

This version is what you get if you pass the above sequent calculus
through the usual translation to natural deduction, and add multiple
assumptions (with full structural rules; for simplicity we consider
contexts to be unordered, and will arrange for weakening and contraction
to be admissible), with \F{\alpha}{} preserving products.  In this
version, the judgement has the form \ndh{\Gamma}{A} where $\Gamma$
consists of assumptions $A_1[\alpha_1],\ldots,A_n[\alpha_n]$.

\[
\begin{array}{c}
\infer[\irl{hyp}]
      {\ndh \Gamma A}
      {A[\alpha] \in \Gamma & 
       \tc \alpha 1
      }
\quad
\infer[\irl{FE}]
      {\ndh{\Gamma,\overline{A_i[\beta_i]}}{C}}
      {\ndh{\overline{A_i[\alpha_i]}}{\F{\alpha}{A}} &
        \ndh{\Gamma,\overline{A_i[\beta_i]},A[\compo{\alpha}{\beta}]}{C} &
        \overline{\tc{\beta_i}{\compo{\alpha_i}{\beta}}}}
\quad
\infer[\irl{FI}]
      {\ndh{\Gamma,\overline{A_i[\beta_i]}}{\F{\alpha}{A}}}
      {\overline{\tc {\beta_i} {\compo{\gamma_i}{\alpha}}} &
        \ndh{\overline{A_i[\gamma_i]}}{A}
      }
\end{array}
\]

The following rules should be admissible:

\[
\infer[\irl{wkn}]
      { \ndh{\Gamma,A[\alpha]}{C}}
      { \ndh{\Gamma}{C}
      }
\qquad
%% \infer[\irl{contr}]
%%       { \ndh{\Gamma,A[\alpha]}{C}}
%%       { \ndh{\Gamma,A[\alpha],A[\alpha]}{C}
%%       }
%% \qquad
\infer[\irl{2cell}]
      { \ndh{\Gamma,A[\alpha]}{C}}
      { \tc{\alpha}{\beta} &
        \ndh{\Gamma,A[\beta]}{C}
      }
\qquad
\infer[\irl{subst}]
      {\ndh{\Gamma,(\compo{\Delta}{\alpha})}{C}}
      {\ndh{\Delta}{A} & 
        \ndh{\Gamma,A[\alpha]}{C}
      }
\]

where
\[
\compo{(A_1[\alpha_1],\ldots,A_n[\alpha_n])}{\alpha} := 
{(A_1[\compo{\alpha_1}{\alpha}],\ldots,A_n[\compo{\alpha_n}{\alpha}])}
\]
Note that $\compo{\Delta}{1} = \Delta$

\section{Natural Deduction with Modal Hypotheses and Conclusions}

\newcommand\nd[3]{\ensuremath{#1 \vdash #2 [ #3 ]}}
\newcommand\nds[2]{\ensuremath{#1 \vdash #2}}

In this version, we have a judgemental notion of $A[\alpha]$ on the
right.  The pseudofunctor operations show up more individually: the
\F{\alpha}{} intro/elim rules are the composition isomorphisms, and the
action of 2-cells and functoriality are each a separate rule acting on
the judgements.  

\[
\begin{array}{c}
\infer{\nd{\Gamma}{A}{\alpha}}
      {A[\alpha] \in \Gamma}
\quad
\infer{\nd{\Gamma}{A}{\alpha'}}
      {\nd{\Gamma}{A}{\alpha} & \tc{\alpha}{\alpha'}}
\quad
\infer{\nd{\Gamma}{\F{\alpha} A}{\beta}}
      {\nd{\Gamma}{A}{\compo{\alpha}{\beta}}}
\quad
\infer{\nd{\Gamma}{A}{\compo{\alpha}{\beta}}}
      {\nd{\Gamma}{\F{\alpha} A}{\beta}}
\quad
\infer{\nd{\Gamma}{C}{\compo{\gamma}{\alpha}}}
      {\Gamma \vdash (\compo{\Gamma'}{\alpha}) &
       \nd{\Gamma'}{C}{\gamma}}
\\ \\
\infer{\nds{\Gamma}{\Delta}}
      {\forall (A[\alpha] \in \Delta): \nd{\Gamma}{A}{\alpha}  }
\end{array}
\]

Substitution should be admissible:

\[
\infer{\nd{\Gamma}{C}{\gamma}}
      {\nds{\Gamma}{\Delta} &
       \nd{\Delta}{C}{\gamma}}
\]

However, this relies on the explicit substitution in the functoriality
rule.  To see why, suppose we had only the rule

\[
\infer{\nd{(\compo{\Gamma'}{\alpha})}{C}{\compo{\gamma}{\alpha}}}
      {\nd{\Gamma'}{C}{\gamma}}
\]
and consider the special case
\[
\infer{\nd{A[\alpha]}{C}{{\alpha}}}
      {\nd{A[1]}{C}{1}}
\]
Then a substitution
\[
\infer{{\nd{\Gamma}{C}{{\alpha}}}}
      { \infer{\nd{\Gamma}{A}{\alpha}}
              {\deduce{\nd{\Gamma}{\F{\alpha}{A}}{1}}{D}}
        &
        \infer{\nd{A[\alpha]}{C}{{\alpha}}}
              {\deduce{\nd{A[1]}{C}{1}}{E}}}
\]
cannot necessarily be reduced because there is not necessarily a
derivation whose conclusion is $A[1]$ in $D$ to compose with $E$.
Semantically, a judgement \nd{\Gamma}{A}{\alpha} means an arbitrary map
into \F{\alpha}{A}, not necessarily one given by functoriality of
$\alpha$, and its only when the left-hand derivation is given by
functoriality that the overall composition is an instance of
functoriality.

%% \addcontentsline{toc}{section}{References}
%% \bibliographystyle{abbrvnat}
%% \bibliography{drl-common/cs}

\end{document}

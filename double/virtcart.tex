\documentclass{article}
\usepackage[utf8]{inputenc}
\usepackage{mathpartir,tikz-cd,cleveref,amsmath,latexsym}
\newcommand{\mgph}{\mathbf{MGph}}
\let\types\vdash
\title{Virtual double cartesian multicategories}
\author{DL, MR, PS, MS}
\date{June 2017}

\begin{document}

\maketitle

\section{Introduction}

A presentation of a multicategory is a natural way to describe a multi-sorted algebraic theory: the objects are the sorts, the generating morphisms are the operations, and the relations on composition are the axioms.
A multicategory, in turn, can be defined as a ``generalized multicategory'' with respect to the (algebraic) theory of monoids, which is a sort of ``terminal algebraic theory'' with exactly one operation of each arity.
This suggests that a natural way to describe a more general kind of ``algebraic theory'' is using a generalized multicategory with respect to the ``terminal'' such theory.
In particular, \emph{type theories} should be defined using generalized multicategories with respect to the ``terminal type theory'', which (at least for intuitionistic simple type theories) is ordinary cartesian type theory.

\section{Mode theories}

Let $\mgph$ denote the category of \textbf{multigraphs}, or equivalently ``multi-sorted function signatures''.
A multigraph is a set of ``objects'' together with a family of ``morphism'' sets $M(\Delta,p)$ indexed by finite lists of objects $\Delta$ and single objects $p$.
A cartesian multicategory can be regarded as a multigraph with algebraic structure, and so there is a monad $T$ on $\mgph$ whose algebras are cartesian multicategories.

In fact, $T$ is a \textbf{cartesian} monad, meaning that it preserves pullbacks and the naturality squares for its unit and multiplication are pullbacks.
This roughly means that $T$ is described by a theory of ``shapes'' that are the same for all possible inputs.
In fact $T M$ has a very concrete description: it has the same objects as $M$, and its morphisms are terms in simple intuitionistic type theory with generating function symbols drawn from the morphisms of $M$.
Cartesianness follows from the fact that the ``shape'' of such terms is insensitive to what the symbols actually are.

Now for any cartesian monad there is a notion of \emph{$T$-multicategory} due to Burroni and Leinster.
A $T$-multicategory consists of an object $M_0$ of the category on which $T$ acts (here, $M_0$ is a multigraph) together with a span
\[ T M_0 \leftarrow M_1 \to M_0 \]
with a ``multiplication'' and ``unit`` structure
\[ T M_1 \times_{T M_0} M_1 \to M_1 \qquad M_0 \to M_1 \]
lying over the multiplication and unit of $T$, and appropriately associative and unital.

Following the terminology of Cruttwell and Shulman, if $T$-algebras are called widgets, then $T$-multicategories can be called ``virtual double widgets''.
The ``double'' is because in addition to the multi-ness there is also a categorification going on in this definition.
For instance, when $T$-algebras are monoids, then $T$-multicategories are multicategories, which are a multi-fied version of monoidal categories, which in turn are categorified monoids.
And when $T$-algebras are categories, then $T$-multicategories are ``virtual double categories'', which are a multi-fied version of double categories, which in turn are categorified categories.

Thus, when $T$ is the monad on $\mgph$ whose algebras are cartesian multicategories, we will call a $T$-multicategory a \textbf{virtual double cartesian multicategory}.
Explicitly, it has the following structure.
\begin{enumerate}
    \item A set of \textbf{objects}.
    \item For each pair of objects, a set of \textbf{vertical arrows}, with composition and identities forming a category.
    \item For each list of objects $\Delta$ and object $p$, a set of \textbf{horizontal arrows} from $\Delta$ to $p$.
    \item A set of \textbf{2-cells}, whose boundary consists of:
    \begin{enumerate}
        \item A horizontal arrow $\Delta \to p$.
        \item A vertical arrow $p'\to p$.
        \item A list of vertical arrows $\Delta'\to\Delta$ (so that in particular $\Delta'$ and $\Delta$ have the same length).
        \item An arrow $\Delta'\to p$ in the free cartesian multicategory generated by the horizontal arrows (i.e.\ a term in cartesian simple type theory with the horizontal arrows as generating function symbols).
    \end{enumerate}
    \item For each horizontal arrow, an identity 2-cell whose vertical-arrow boundaries are identities.
    \item A composition operation on 2-cells, which substitutes terms for individual symbols in the domains, which is appropriately associative and unital.
\end{enumerate}

Inspired by type-theoretic syntax, we write horizontal arrows with a turnstile.
For instance, with objects $p,q,r$ we could have horizontal arrows $p,q \types f:r$, or more explicitly $x:p,y:q\types f(x,y):r$.
If we also have say $q,r\types g:s$ and $s\types h:r$, then a possible 2-cell would be
\[
\begin{tikzcd}[column sep=huge]
x:p,y:q \ar[r,"{h(g(y,f(x,y)))}"] \ar[d,equals] \ar[dr,phantom,"\Downarrow"] & p \ar[d,equals]\\
x:p,y:q \ar[r,"{f(x,y)}"'] & r
\end{tikzcd}
\]
which we could write more type-theoretically as
\[
\inferrule{ x:p,y:q \types h(g(y,f(x,y))) : p}{x:p,y:q \types f(x,y):p}
\]

We define a \textbf{mode theory} to be a virtual double cartesian multicategory with only identity vertical arrows.
We call the objects \textbf{modes}, the horizontal arrows \textbf{mode morphisms}, and the 2-cells \textbf{structural rules}.
(I will come back to the vertical arrows later.)

\section{Type theories over a mode theory}

Now, again for any cartesian monad $T$ and any $T$-multicategory $M$, there is a notion of an \textbf{$M$-algebra}: an object $D$ with a map $D\to M_0$ and an action map
\[ T D \times_{T M_0} M_1 \to D \]
lying over $M_0$ and that is appropriately associative and unital.
In our case, an algebra for a virtual double cartesian multicategory $M$ consists of:
\begin{enumerate}
    \item For each mode, a set of \textbf{types} with that mode.
    \item A functorial action of vertical arrows on types.
    \item For each mode morphism $\alpha$, a set of \textbf{term judgments} parametrized by $\alpha$, with a domain that is a list of types whose modes are those in the domain of $\alpha$, and a codomain that is a type whose mode is the codomain of $\alpha$.
    As in previous work, we pun the variables in the mode morphism and the term judgment.
    Thus for instance if $x:p,y:q\types \alpha(x,y):r$ is a mode morphism, and $A,B,C$ are types with modes $p,q,r$ respectively, we write
    \[ x:A, y:B \types_{\alpha(x,y)} f(x,y) : C \]
    for a term judgment.
    \item An action of the structural rules on the term judgments, which can be expressed conveniently as follows.
    We first generate $T D$ by using ordinary cartesian simple type theory applied to the term judgments, with the parametrizing term of mode morphisms (the image in $T M_0$) constructed in parallel fashion (i.e.\ with cut-over-cut and structurality-over-structurality).
    But unlike in previous work, the terms in this type theory are just ``auxiliary'' or \emph{potential}.
    Instead, for any such ``potential term'', and any structural rule (i.e.\ mode 2-cell) whose domain is its parametrizing mode term, we have a specified \emph{actual} term judgment parametrized by the codomain of that structural rule.
    
    For example, given types $A,B,C,C',D$ of modes $p,q,r,r,s$ respectively, plus term judgments
    \begin{mathpar}
      x:A,y:B \types_{f(x,y)} \phi(x,y):C\and
      y:B,z:C \types_{g(y,z)} \psi(y,z): D \and
      w:D \types_{h(w)} \chi(w):C'
    \end{mathpar}
    there is a potential term judgment
    \[x:A,y:B \types_{h(g(y,f(x,y)))} \chi(\psi(y,\phi(x,y))) : C'.\]
    If we apply the 2-cell structural rule pictured above, then we obtain an induced actual term judgment
    \[x:A,y:B \types_{f(x,y)} \chi(\psi(y,\phi(x,y))) : C'.\]
    Note that ``actual'' term judgments are always parametrized by a \emph{single} mode morphism, rather than a ``mode term``.
    We can write the action of this 2-cell in more type-theoretic style as:
    \[\inferrule{x:A,y:B \types_{f(x,y)} \phi(x,y):C \\ y:B,z:C \types_{g(y,z)} \psi(y,z): D \\ w:D \types_{h(w)} \chi(w):C'
    }{x:A,y:B \types_{f(x,y)} \chi(\psi(y,\phi(x,y))) : C'}
    \]
    This rule combines the action of two cuts with a contraction on the antecedent.
    Usually we will generate the mode theory from structural rules that do only ``one thing'', with axioms on their composites that explain how the rules commute with each other; but the theory is expressive enough to permit such ``combined rules'' as well.
\end{enumerate}

\section{Examples}

\subsection{Ordered logic}
\label{sec:ordered-logic}

There is one mode $p$, one mode morphism $\overbrace{p,p,\dots,p}^n \types \alpha_n : p$ for each natural number $n$, and identity and cut structural rules
\begin{mathpar}
  \inferrule*[right=id]{x:p\types x:p}{x:p \types \alpha_1(x) :p}
  \and
  \inferrule*[right=cut]{x_1:p,\dots, x_{m+n-1}:p \types \alpha_{m}(y_1,\dots, y_{i-1},\alpha_n(x_1,\dots,x_n),y_{i+1},\dots,y_m) :p}
  {x_1:p,\dots, x_{m+n-1}:p \types \alpha_{m+n-1}(y_1,\dots, y_{i-1},x_1,\dots,x_n,y_{i+1},\dots,y_m) :p}
\end{mathpar}
satisfying the identity and associativity axioms in the ``$\circ_i$'' definition of an ordinary multicategory.
For instance, the following composite is equal to the identity 2-cell of $\alpha_n$:
\begin{mathpar}
  \inferrule*[right=cut]{
    \inferrule*[Right=id($1_{\alpha_n}$)]{x_1:p,\dots, x_{n}:p \types \alpha_{n}(x_1,\dots,x_n) :p}
    {x_1:p,\dots, x_{n}:p \types \alpha_{1}(\alpha_n(x_1,\dots,x_n)) :p}}
  {x_1:p,\dots, x_{n}:p \types \alpha_{n}(x_1,\dots,x_n) :p}
\end{mathpar}
where \textsc{id}$(1_{\alpha_n})$ indicates the ``auxiliary 2-cell'' obtained by applying \textsc{id} to the identity $1_{\alpha_n}$.
In general, when composing rules, what we can plug in on top is a ``horizontal composition'' of other rules built using ordinary cartesian type theory, of the same ``shape'' as the top of the rule we are composing with below.

We could also present the same virtual double cartesian multicategory using a basic multi-composition 2-cell, but it would look less like type theory as it is usually presented.

\subsection{Linear logic}
\label{sec:linear-logic}

For this we add exchange rules
\begin{mathpar}
  \inferrule*[right=exch]{x_1:p,\dots,x_n:p \types \alpha_n(x_1,\dots,x_{i-1},x_{i+1},x_i,x_{i+2},\dots,x_n):p}
  {x_1:p,\dots,x_n:p \types \alpha_n(x_1,\dots,x_n):p}
\end{mathpar}
satisfying the relations in the usual presentation of the symmetric group using transpositions, plus commutativity with cut as in the usual notion of symmetric multicategory.
For instance, the following composite rules are equal:
\begin{mathpar}
  \inferrule*[right=exch]{
    \inferrule*[Right=cut]{x_1:p,x_2:p,x_3:p \types \alpha_2(x_1,\alpha_2(x_3,x_2)) :p}
    {x_1:p,x_2:p,x_3:p \types \alpha_3(x_1,x_3,x_2):p}}
  {x_1:p,x_2:p,x_3:p \types \alpha_3(x_1,x_2,x_3):p}
  \and
  \inferrule*[right=cut]{\inferrule*[Right={1($x_1$,exch)}]
  {x_1:p,x_2:p,x_3:p \types \alpha_2(x_1,\alpha_2(x_3,x_2)) :p}
  {x_1:p,x_2:p,x_3:p \types \alpha_2(x_1,\alpha_2(x_2,x_3)) :p}}
  {x_1:p,x_2:p,x_3:p \types \alpha_3(x_1,x_2,x_3):p}
\end{mathpar}

\subsection{Cartesian logic}
\label{sec:cartesian-logic}

Now we add contraction and weakening rules:
\begin{mathpar}
  \inferrule*[right=contract]{x_1:p,\dots,x_n:p \types
    \alpha_{n+1}(x_1,\dots,x_i,x_i,\dots,x_n):p}
  {x_1:p,\dots,x_n:p \types
    \alpha_{n}(x_1,\dots,x_n):p}
  \and
  \inferrule*[right=weak]{x_1:p,\dots,x_n:p \types
    \alpha_{n-1}(x_1,\dots,x_{i-1},x_{i+1},\dots,x_n):p}
  {x_1:p,\dots,x_n:p \types
    \alpha_{n}(x_1,\dots,x_n):p}
\end{mathpar}
satisfying the usual relations in a cartesian multicategory.

\subsection{Linear/cartesian logic}
\label{sec:linear-cartesian}

Now suppose we have two modes $c,l$, with morphisms $c^n \types \alpha_n:c$ and $\mu^n \types \beta : l$, where each of the $n$ occurrences of $\mu$ can be either $c$ or $l$.
In particular, we have $c \types \gamma : l$, and $l^n \types \beta_n : l$.
We include arbitrary exchange rules for the $\alpha$'s and the $\beta$'s, contraction and weakening rules that apply only to $c$ variables, and the obvious sort of cut.

\subsection{Spatial type theory}
\label{sec:spatial-type-theory}

We have one mode $p$, with morphisms $p^n \types \alpha_{\varepsilon_1,\dots,\varepsilon_n} :p$ where each $\varepsilon_i$ is either $1$ (cohesive) or $c$ (crisp).
Exchange, contraction, and weakening act on cohesive and crisp variables separately, plus we have
\begin{mathpar}
  \inferrule{p^n \types \alpha_{\varepsilon_1,\dots,\varepsilon_n} :p}
  {p^n \types \alpha_{\varepsilon'_1,\dots,\varepsilon'_n} :p}
\end{mathpar}
where $\varepsilon'_i =c$ if $\varepsilon_i =c$, but if $\varepsilon_i =1$ then $\varepsilon'_i$ could be either $c$ or $1$.
There are appropriate axioms.

\subsection{Cartesian 2-multicategories}
\label{sec:cart-2-multi}

From any cartesian 2-multicategory we can construct a virtual double cartesian multicategory with the same modes and the same mode morphisms, and where the 2-cells are determined by actually composing up the term in the domain according to the cartesian multicategory structure, and then considering 2-cells from that composite to the codomain.
In this way we recover the previous work.

However, from any virtual double cartesian multicategory we get another one by restricting to any subset of the mode morphisms.
Thus, we can start with a cartesian 2-multicategory and declare that we are only interested in term judgments of a particular shape, and retain all the (possibly composite) structural rules that relate these judgments only to each other.

For instance, the cartesian 2-multicategory containing a commutative monoid has mode morphisms that we care about like $x:p,y:p \types x\otimes y:p$, but also ``nonlinear junk'' like $x:p,y:p \types x\otimes x:p$.
If we restrict to the former sort of mode morphism only, we obtain a virtual double cartesian multicategory, which should in fact be the one of \cref{sec:linear-logic}.
Similarly, we should obtain \cref{sec:linear-cartesian,sec:spatial-type-theory} and so on from the analogous 2-multicategories we considered before, by restricting to a subset of the ``interesting'' mode morphisms.

\section{Generalizations and thoughts}
\label{sec:generalizations}

To deal with first-order logic and substitution, we could use the vertical arrows in a virtual double cartesian multicategory, since they act on the types in an algebra.
However, for that purpose I think it would be better to instead consider generalized multicategories relative to a monad on the category of ``first-order signatures'' whose algebras are hyperdoctrines.

What are the vertical arrows for then?
I think they are for type forming rules.
In the unary case, with two modes $p,q$ and a morphism $p\types \alpha:q$, we can have a vertical arrow $F_\alpha : p\to q$ whose action on types implements the like-named type former.
The left and right rules for $F_\alpha$ are 2-cells:
\[
  \begin{tikzcd}
    p \ar[r,"\alpha"] \ar[d,"F_\alpha"'] \ar[dr,phantom,"\Downarrow"] & q \ar[d,equals] \\
    q \ar[r,equals] & q
  \end{tikzcd}
  \qquad
  \begin{tikzcd}
    p \ar[r,equals] \ar[d,equals] \ar[dr,phantom,"\Downarrow"] & p \ar[d,"F_\alpha"]\\
    p \ar[r,"\alpha"'] & q
  \end{tikzcd}
\]
giving rules in the type theory
\begin{mathpar}
  \inferrule{A_p \types_{\alpha} B_q}{F_\alpha(A_p) \types_1 B_q}
  \and
  \inferrule{A_p \types_1 B_p}{A_p \types_\alpha F_\alpha(B_p)}
\end{mathpar}
The cut reduction
\begin{equation*}
  \inferrule{\inferrule*{A_p \types_1 B_p}{A_p \types_\alpha F_\alpha(B_p)}\\
    \inferrule*{B_p \types_{\alpha} C_q}{F_\alpha(B_p) \types_1 C_q}}
  {A_p \types_\alpha C_q}
  \quad\leadsto\quad
  \inferrule{A_p \types_1 B_p\\ B_p \types_{\alpha} C_q}{A_p \types_\alpha C_q}
\end{equation*}
corresponds to the equation on 2-cell composites
\[
  \begin{tikzcd}
    p \ar[r,equals] \ar[d,equals] \ar[dr,phantom,"\Downarrow"] & 
    p \ar[d,"F_\alpha" description]  \ar[r,"\alpha"] \ar[dr,phantom,"\Downarrow"] &
    q \ar[d,equals] \\
    p \ar[r,"\alpha"'] \ar[d,equals] & q \ar[r,equals] & q \ar[d,equals] \\
    p \ar[rr,"\alpha"'] && q
  \end{tikzcd}
  \qquad=\qquad
  \begin{tikzcd}
    p \ar[r,equals] \ar[d,equals] & 
    p \ar[r,"\alpha"] &
    q \ar[d,equals] \\
    p \ar[rr,"\alpha"'] && q
  \end{tikzcd}
\]
Similarly, the identity expansion
\begin{equation*}
  \inferrule{ }{F_\alpha(A_p) \types_1 F_\alpha(A_p)}
  \quad\leadsto\quad
  \inferrule{\inferrule*{\inferrule*{ }{A_p \types_1 A_p}}
  {A_p \types_\alpha F_\alpha(A_P)}}{F_\alpha(A_p) \types_1 F_\alpha(A_p)}
\end{equation*}
corresponds to the equation on 2-cell composites:
\begin{equation*}
  \begin{tikzcd}
    & p \ar[d,"F_\alpha"] \\
    & q \ar[dl,equals] \ar[dr,equals]\\
    q \ar[rr,equals] && q
  \end{tikzcd}
  \qquad=\qquad
  \begin{tikzcd}
    & p \ar[dl,equals] \ar[dr,equals]\\
    p \ar[rr,equals] \ar[d,equals] \ar[drr,phantom,"\Downarrow"] && p \ar[d,"F_\alpha"]\\
    p \ar[rr,"\alpha" description] \ar[d,"F_\alpha"'] \ar[drr,phantom,"\Downarrow"] && q \ar[d,equals] \\
    q \ar[rr,equals] && q
  \end{tikzcd}
\end{equation*}
Together, these 2-cell equations say that $F_\alpha$ is a \textbf{vertical companion} of $\alpha$ in the double-category sense.
(Well, technically that would only be true if the cut and identity rules were actual ``composites'' with a universal property in our virtual double category, which they may not be.
But the similarity is clear.)
Similarly, the left and right rules for $U_\alpha$ say that it is a \textbf{vertical conjoint} of $\alpha$.

I think this is very pretty.
Unfortunately it doesn't work for higher-ary $F$ and $U$ functors, because our vertical arrows have only unary domains.
If instead of the monad for cartesian multicategories, we used the monad for cartesian monoidal categories, then we would have vertical arrows with higher-ary domains and I think we could treat higher-ary $F$-functors similarly.
However, to deal with higher-ary $U$-functors it seems we would need a monad that builds in some kind of duality, and I'm not sure yet how to do that.

Finally, to deal with multiple-conclusioned ``classical'' logics, we would need a monad that looks like some sort of bicartesian prop.
We saw before that bicartesian props are problematic, since for instance bicartesianness of a monoidal category trivializes duality.
However, I have some hopes that the switch to \emph{virtual} algebras may improve things: since the morphisms in $T M$ only have to describe the ``input shapes'', and we are free to impose rules via 2-cells, we may be able to just leave out some of the ``natural'' axioms of a bicartesian prop that create the problems.

\end{document}

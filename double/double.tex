%%SHORT \documentclass[conference,compsoconf]{../drl-common/IEEEtran}
%%SHORT \IEEEoverridecommandlockouts

\documentclass{article}
\usepackage{times}
\usepackage{authblk}
%% \usepackage{fullpage}

\usepackage[all]{xy}
\usepackage{multicol}
\usepackage{mathptmx}
\usepackage{color}
\usepackage[cmex10]{amsmath}
\usepackage{amsthm}
\usepackage{amssymb}
\usepackage{stmaryrd}
\usepackage{../drl-common/proof}
\usepackage{../drl-common/typesit}
\usepackage{../drl-common/typescommon}
\usepackage{../drl-common/theorem-envs}
\usepackage[sort]{natbib}
%% \usepackage{arydshln}
\usepackage{graphics}
\usepackage{natbib}
\usepackage{url}
\usepackage{relsize}
\usepackage{tipa}

\usepackage{tikz}
\usetikzlibrary{decorations.pathmorphing}

\usepackage{fancyvrb}
\newcommand{\ttt}[1]{\texttt{#1}}


\newcommand\Bx[2]{\ensuremath{\Box_{#1} \, {#2}}}
\newcommand\Crc[2]{\ensuremath{\bigcirc_{#1} \, {#2}}}
\newcommand\Dia[2]{\ensuremath{\Diamond_{#1} \, {#2}}}
\newcommand\Flat[1]{\ensuremath{\flat \, {#1}}}
\newcommand\Sharp[1]{\ensuremath{\sharp \, {#1}}}
\newcommand{\sh}{\text{\textesh}}

\newcommand\D{\ensuremath{d}} %% originally was mathcal{D} but switched notation for derivations
\newcommand\E{\ensuremath{e}}

\newcommand\magicwand{\mathrel{-\mkern-6mu*}}
\newcommand\mor[3]{\ensuremath{#2} \longrightarrow_#1 #3}
\newcommand\C{\ensuremath{\mathcal{C}}}
\newcommand\deq{\ensuremath{\equiv}}
\newcommand\spr{\ensuremath{\Rightarrow}} %% structural property/2-cell
\newcommand\seq[3]{\ensuremath{#1 \vdash_{#2} #3}}
\newcommand\seql[3]{\ensuremath{#1 \vdash^{\dsd{#2}} #3}}
\newcommand\F[2]{\ensuremath{\dsd{F}_{#1}(#2)}}
\newcommand\U[3]{\ensuremath{\dsd{U}_{#1}(#2 \mid #3)}}
\newcommand\Uempty[2]{\ensuremath{\dsd{U}_{#1}(#2)}}
\newcommand\Fsymb[0]{\dsd{F}}
\newcommand\Usymb[0]{\dsd{U}}
\newcommand\tsubst[2]{\ensuremath{#1[#2]}}
\renewcommand\subst[3]{\ensuremath{#1[#2/#3]}}
\newcommand\wftype[2]{\ensuremath{#1 \vdash #2 \,\, \dsd{type}}}
\renewcommand\wfctx[2]{\ensuremath{#1 \vdash #2 \,\, \dsd{ctx}}}
\newcommand\modeof[1]{\ensuremath{\hat{#1}}}
\newcommand\many[1]{\ensuremath{\overline{#1}}}
\renewcommand{\oftp}[3]{\ensuremath{#1 \, \vdash #2 \, \dcd{:} \, #3}}
\newcommand\FL{\dsd{FL}}
\newcommand\FR{\dsd{FR}}
\newcommand\UL{\dsd{UL}}
\newcommand\UR{\dsd{UR}}
\newcommand\lolli\multimap
\newcommand\la\dashv

\def\M{\mathcal{M}}
\def\toiso{\xrightarrow{\sim}}
\let\To\Rightarrow
%\newcommand\compo[2]{\ensuremath{#1 \circ #2}}
\newcommand\compv[2]{\ensuremath{#1 \cdot #2}}
\newcommand\comph[2]{\ensuremath{#1 \mathbin{\circ_2} #2}}

\def\llb{\llbracket}
\def\rrb{\rrbracket}

\newcommand\seqa[2]{\seql{#1}{a}{#2}}
\newcommand\seqc[2]{\seql{#1}{c}{#2}}
\newcommand\splits{\rightrightarrows}
\newcommand\vars[1]{\ensuremath{\overline{#1}}}

\newcommand{\ignore}[1]{}

\newcommand\FLd[2]{\ensuremath{\FL^{#1}(#2)}}
\newcommand\FRd[3]{\ensuremath{\FR_{#1}(#2,#3)}}
\newcommand\ULd[5]{\ensuremath{\UL^{#1}_{#2}(#3,#4,#5)}}
\newcommand\URd[1]{\ensuremath{\UR(#1)}}
\newcommand\Trd[2]{\ensuremath{#1_*(#2)}}
\newcommand\Ident[1]{\ensuremath{{#1}}}
\newcommand\Cut[3]{\ensuremath{#1[#2/#3]}}
\newcommand\Cuta[3]{\ensuremath{#1\{#2/#3\}}}
\newcommand\Cutta[2]{\ensuremath{#1\{#2\}}}
\newcommand\Trda[2]{\ensuremath{#1_*\{#2\}}}
\newcommand\Identa[1]{\ensuremath{{\dsd{id}\{#1\}}}}
\newcommand\FRs{\ensuremath{\FR^*}}
\newcommand\ULs[1]{\ensuremath{\UL^*_{#1}}}

\newcommand\elim[1]{#1\mathord{\downarrow}}
\newcommand\deqp{\deq_{\dsd p}}
\newcommand\Linv[3]{\ensuremath{\dsd{linv}(#1,#2,#3)}}
\newcommand\Rinv[2]{\ensuremath{\dsd{rinv}(#1,#2)}}

\newcommand\pto{\mathrel{\ooalign{\hfil$\mapstochar$\hfil\cr$\to$\cr}}}

\begin{document}

\title{Double-Categorical Adjoint Logic}
\author{DL, PS, MS, MR}

%% SHORT
% author names and affiliations
% use a multiple column layout for up to three different
% affiliations
%% \author{\IEEEauthorblockN{Daniel R. Licata}
%% \IEEEauthorblockA{Wesleyan University\\
%% \url{dlicata@wesleyan.edu}}
%% \and
%% \IEEEauthorblockN{Michael Shulman}
%% \IEEEauthorblockA{University of San Diego \\
%%   \url{shulman@sandiego.edu}}
%% \and
%% \IEEEauthorblockN{Mitchell Riley}
%% \IEEEauthorblockA{Wesleyan University \\
%%   \url{mvriley@wesleyan.edu}}

%% %% \thanks{
%% %%   ?
%% %% }

%% \author[1]{{Daniel R. Licata}}
%% \author[2]{{Michael Shulman}}
%% \author[1]{{Mitchell Riley}}
%% \affil[1]{Wesleyan University}
%% \affil[2]{University of San Diego}

\maketitle

\section{Modes}

Write $p$ for modes and $\psi := x_1:p_1,\ldots,x_n:p_n$ or $\phi$ for
mode contexts.  We stipulate that contexts $\psi$ that differ only by
the order of variables are equal (i.e. we have exchange).  Write
\oftp{\psi}{a}{p} and \oftp{\psi}{\theta}{\psi'} for vertical morphisms,
which are linear:
\[
\infer{\oftp{x:p}{x}{p}}{}
\qquad
\infer{\oftp{\psi_1,\psi_2}{\theta,a/x}{\psi',x:p}}
      {\oftp{\psi_1}{\theta}{\psi} &
        \oftp{\psi_2}{a}{p}}
\]

\newcommand\hmor[3]{\ensuremath{#1 \mid #2 \vdash^{\dsd h} #3}}
\newcommand\sq[4]{\ensuremath{#1 \Rightarrow #2 \: [#3 \mid #4]}}
\newcommand\hid[1]{1_{#1}}

We write a horizontal morphism $\psi \pto \psi'$ profunctor-like, as a
term $\hmor{\psi}{\phi}{\alpha}$ with both $\psi$ and $\phi$ as free
variables.  These have identity and composition:

\[
\infer{\hmor{\psi}{\psi}{\hid{\psi}}}{}
\qquad
\infer{\hmor{\psi_1,\psi_1'}{\psi_3}{[\alpha/\psi_2]\beta}}
      {\hmor{\psi_1}{\psi_2}{\alpha} &
       \hmor{\psi_1',\psi_2}{\psi_3}{\beta}}
\]

We also need functoriality of the monoidal product:
\[
\infer{\hmor{\psi_1,\psi_2}{\psi_2,\psi_2'}{\alpha,\alpha'}}
      {\hmor{\psi_1}{\psi_2}{\alpha} &
       \hmor{\psi_1'}{\psi_2'}{\alpha'}}
\]

Exisitng adjoint logic terms $\psi \vdash \alpha : p$ can be encoded in
``destination-passing-style:'' E.g.  instead of $x:p,y:p \vdash x
\otimes y : p$, we'd have $x:p,y:p \mid d:p \vdash x \otimes y > d$, and
$(x \otimes y) \otimes z$ is the formal composite
\[
[((x \otimes y) > d_1)/d_1](d_1 \otimes z > d)
\]

2-cells are squares \sq{\alpha}{\alpha'}{\theta}{\rho} where
$\oftp{\psi}{\theta}{\psi'}$ and 
$\oftp{\phi}{\rho}{\phi'}$ and
$\hmor{\psi}{\phi}{\alpha}$ an
$\hmor{\psi'}{\phi'}{\alpha'}$.  

We assume that horizontal morphisms can be composed with vertical
morphism on either side via an operation, probably with a filler square
relating them too?
\[
\infer{\hmor{\psi'}{\phi}{\alpha[\theta/\psi]}}
      {\hmor{\psi}{\phi}{\alpha} &
        \oftp{\psi'}{\theta}{\psi}}
\qquad
\infer{\hmor{\psi}{\phi'}{\alpha[\theta/\phi]}}
      {\hmor{\psi}{\phi}{\alpha} &
        \oftp{\psi}{\theta}{\phi}}
\]

When the vertical 1-cells $\oftp{\Psi}{\theta}{\Psi'}$ are cartesian
variable-for-variable substitutions, and the codomain contexts always
have 1 variable, this should be clubs.  

\section{Judgements}

\newcommand{\seqh}[5]{\ensuremath{#1 \mid #2 \vdash_{#3} #4 \mid #5}}

Types and contexts live over mode contexts, so we have judgements
\wftype{\psi}{A} and \wfctx{\psi}{\Gamma}.  

We require admissible rules pulling back along vertical morphisms:
\[
\infer{\wftype{\psi_1,\psi_2'}{A[\theta]}}
      {\wftype{\psi_1,\psi_2}{A} & 
        \oftp{\psi_2'}{\theta}{\psi_2}}
\qquad
\infer{\wfctx{\psi_1,\psi_2'}{\Gamma[\theta]}}
      {\wfctx{\psi_1,\psi_2}{\Gamma} & 
        \oftp{\psi_2'}{\theta}{\psi_2}}
\]

Contexts are defined by
\[
\infer{\wfctx{\cdot}{\cdot}}{}
\qquad
\infer{\wfctx{\psi,\psi'}{\Gamma,A}}
      {\wfctx{\psi}{\Gamma} & \wfctx{\psi'}{A}}
\]

TODO: define $n$-ary partial-context substitution; it's a bit of a pain because of linearity.  

We should have judgements $\seqh{\psi}{\Gamma}{\theta}{\psi'}{A}$ where
$\oftp{\psi}{\theta}{\psi'}$ (vertical morphism over vertical morphism)
and $\seqh{\psi}{\Gamma}{\alpha}{\phi}{A}$ where $\wfctx{\psi}{\Gamma}$
and $\wftype{\phi}{A}$ and 
$\hmor{\psi}{\phi}{\alpha}$ (horizontal morphism over horizontal
morphism).  However, because we're thinking of the vertical arrows as
terms that the types depend on, and insisting on pullbacks along them,
we can shove everything into the fiber rather than having a primitive
``over'' judgement. So a general vertical morphism
\seqh{\psi}{\Gamma}{\theta}{\psi'}{A} can be represented by a
homogeneous vertical morphism
\seqh{\psi}{\Gamma}{x_i/x_i}{\psi}{A[\theta]}.  But a degenerate
vertical morphism should be (???) the same as a degenerate horizontal
morphism, so it suffices to have horizontal morphisms up top.

Structural rules: identity over identity, cut over cut, transporting
along a square:
\[
\infer{\seqh{\psi}{x:A}{\hid{\psi}}{\psi}{A}}{}
\qquad
\infer{\seqh{\psi_1,\psi_2}{\Gamma_1,\Gamma_2}{[\alpha/\psi]\beta}{\phi}{B}}
      {\seqh{\psi_2}{\Gamma_2}{\alpha}{\psi}{A} &
       \seqh{\psi_1,\psi}{\Gamma_1,x:A}{\beta}{\phi}{B}}
\]

\[
\infer{\seqh{\psi}{\Gamma[\theta]}{\alpha}{\phi}{A[\rho]}}
      {\seqh{\psi'}{\Gamma}{\beta}{\phi'}{A} &
        \sq{\alpha}{\beta}{\theta}{\rho}}
\]

Atomic propositions:
\[
\infer{\wftype{\psi}{P(\theta)}}
      {\psi' \vdash P &
        \oftp{\psi}{\theta}{\psi'}
      }
\qquad
\infer{\seqh{\psi}{P(\theta)}{\beta}{\phi}{P(\rho)}}
      {\psi' \vdash P &
        \sq{\beta}{\hid{\psi'}}{\theta}{\rho}}
\]
\[
P(\theta)[\theta'] := P(\theta[\theta'])
\]

Pushforward along a horizontal morphism:

%% (FIXME: should some of the
%% dependency leak through to the conclusion? this seems weird):
%% \[
%% \infer{\wftype{\phi}{\F{\alpha}{\Delta}}}
%%       {\wfctx{\psi}{\Delta} &
%%         \hmor{\psi}{\phi}{\alpha}
%%       }
%% \qquad
%% \F{\alpha}{\Delta}[\theta] := \F{\alpha[\theta/\psi]}{\Delta}
%% \]
%% Axiomatic FR:
%% \[
%% \infer{\seqh{\psi}{\Delta}{\alpha}{\phi}{\F{\alpha}{\Delta}}}{}
%% \]

\[
\infer{\wftype{\psi_0,\phi}{\F{\alpha}{\Delta}}}
      {\wfctx{\psi_0,\psi}{\Delta} &
        \hmor{\psi}{\phi}{\alpha}
      }
\qquad
\F{\alpha}{\Delta}[\theta] := \F{\alpha[\theta\mid_\psi]}{\Delta[\theta\mid_{\psi_0}]}
\]

\[
\infer[\FR^*]{\seqh{\psi_0,\psi}{\Delta}{1_{\psi_0},\alpha}{\psi_0,\phi}{\F{\alpha}{\Delta}}}{}
\]

\[
\infer[\FL]
      {\seqh{\psi_1,(\psi_0,\phi)}{\Gamma,x:\F{\alpha}{\Delta}}{\beta}{\phi_1}{C}}
      {\seqh{\psi_1,(\psi_0,\psi)}{\Gamma,\Delta}{[\alpha/\psi]\beta}{\phi_1}{C}}
\]

\end{document}


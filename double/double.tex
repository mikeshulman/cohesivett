%%SHORT \documentclass[conference,compsoconf]{../drl-common/IEEEtran}
%%SHORT \IEEEoverridecommandlockouts

\documentclass{article}
\usepackage{times}
\usepackage{authblk}
\usepackage{fullpage}

\usepackage[all]{xy}
\usepackage{multicol}
\usepackage{mathptmx}
\usepackage{color}
\usepackage[cmex10]{amsmath}
\usepackage{amsthm}
\usepackage{amssymb}
\usepackage{stmaryrd}
\usepackage{../drl-common/proof}
\usepackage{../drl-common/typesit}
\usepackage{../drl-common/typescommon}
\usepackage{../drl-common/theorem-envs}
\usepackage[sort]{natbib}
%% \usepackage{arydshln}
\usepackage{graphics}
\usepackage{natbib}
\usepackage{url}
\usepackage{relsize}
\usepackage{tipa}

\usepackage{tikz}
\usetikzlibrary{decorations.pathmorphing}

\usepackage{fancyvrb}
\newcommand{\ttt}[1]{\texttt{#1}}


\newcommand\Bx[2]{\ensuremath{\Box_{#1} \, {#2}}}
\newcommand\Crc[2]{\ensuremath{\bigcirc_{#1} \, {#2}}}
\newcommand\Dia[2]{\ensuremath{\Diamond_{#1} \, {#2}}}
\newcommand\Flat[1]{\ensuremath{\flat \, {#1}}}
\newcommand\Sharp[1]{\ensuremath{\sharp \, {#1}}}
\newcommand{\sh}{\text{\textesh}}

\newcommand\D{\ensuremath{d}} %% originally was mathcal{D} but switched notation for derivations
\newcommand\E{\ensuremath{e}}

\newcommand\magicwand{\mathrel{-\mkern-6mu*}}
\newcommand\mor[3]{\ensuremath{#2} \longrightarrow_#1 #3}
\newcommand\C{\ensuremath{\mathcal{C}}}
\newcommand\deq{\ensuremath{\equiv}}
\newcommand\spr{\ensuremath{\Rightarrow}} %% structural property/2-cell
\newcommand\seq[3]{\ensuremath{#1 \vdash_{#2} #3}}
\newcommand\seql[3]{\ensuremath{#1 \vdash^{\dsd{#2}} #3}}
\newcommand\F[2]{\ensuremath{\dsd{F}_{#1}(#2)}}
\newcommand\U[3]{\ensuremath{\dsd{U}_{#1}(#2 \mid #3)}}
\newcommand\Uempty[2]{\ensuremath{\dsd{U}_{#1}(#2)}}
\newcommand\Fsymb[0]{\dsd{F}}
\newcommand\Usymb[0]{\dsd{U}}
\newcommand\tsubst[2]{\ensuremath{#1[#2]}}
\renewcommand\subst[3]{\ensuremath{#1[#2/#3]}}
\newcommand\wftype[2]{\ensuremath{#1 \vdash #2 \,\, \dsd{type}}}
\renewcommand\wfctx[2]{\ensuremath{#1 \vdash #2 \,\, \dsd{ctx}}}
\newcommand\modeof[1]{\ensuremath{\hat{#1}}}
\newcommand\many[1]{\ensuremath{\overline{#1}}}
\renewcommand{\oftp}[3]{\ensuremath{#1 \, \vdash #2 \, \dcd{:} \, #3}}
\newcommand\FL{\dsd{FL}}
\newcommand\FR{\dsd{FR}}
\newcommand\UL{\dsd{UL}}
\newcommand\UR{\dsd{UR}}
\newcommand\lolli\multimap
\newcommand\la\dashv

\def\M{\mathcal{M}}
\def\toiso{\xrightarrow{\sim}}
\let\To\Rightarrow
%\newcommand\compo[2]{\ensuremath{#1 \circ #2}}
\newcommand\compv[2]{\ensuremath{#1 \cdot #2}}
\newcommand\comph[2]{\ensuremath{#1 \mathbin{\circ_2} #2}}

\def\llb{\llbracket}
\def\rrb{\rrbracket}

\newcommand\seqa[2]{\seql{#1}{a}{#2}}
\newcommand\seqc[2]{\seql{#1}{c}{#2}}
\newcommand\splits{\rightrightarrows}
\newcommand\vars[1]{\ensuremath{\overline{#1}}}

\newcommand{\ignore}[1]{}

\newcommand\FLd[2]{\ensuremath{\FL^{#1}(#2)}}
\newcommand\FRd[3]{\ensuremath{\FR_{#1}(#2,#3)}}
\newcommand\ULd[5]{\ensuremath{\UL^{#1}_{#2}(#3,#4,#5)}}
\newcommand\URd[1]{\ensuremath{\UR(#1)}}
\newcommand\Trd[2]{\ensuremath{#1_*(#2)}}
\newcommand\Ident[1]{\ensuremath{{#1}}}
\newcommand\Cut[3]{\ensuremath{#1[#2/#3]}}
\newcommand\Cuta[3]{\ensuremath{#1\{#2/#3\}}}
\newcommand\Cutta[2]{\ensuremath{#1\{#2\}}}
\newcommand\Trda[2]{\ensuremath{#1_*\{#2\}}}
\newcommand\Identa[1]{\ensuremath{{\dsd{id}\{#1\}}}}
\newcommand\FRs{\ensuremath{\FR^*}}
\newcommand\ULs[1]{\ensuremath{\UL^*_{#1}}}

\newcommand\elim[1]{#1\mathord{\downarrow}}
\newcommand\deqp{\deq_{\dsd p}}
\newcommand\Linv[3]{\ensuremath{\dsd{linv}(#1,#2,#3)}}
\newcommand\Rinv[2]{\ensuremath{\dsd{rinv}(#1,#2)}}

\newcommand\pto{\mathrel{\ooalign{\hfil$\mapstochar$\hfil\cr$\to$\cr}}}

\newcommand{\seqh}[5]{\ensuremath{#1 \mid #2 \vdash_{#3} #4 \mid #5}}
\newcommand{\ct}[2]{\ensuremath{{{#2}^{#1}}}}

\begin{document}

\title{Double-Categorical Adjoint Logic}
\author{DL, PS, MS, MR}

%% SHORT
% author names and affiliations
% use a multiple column layout for up to three different
% affiliations
%% \author{\IEEEauthorblockN{Daniel R. Licata}
%% \IEEEauthorblockA{Wesleyan University\\
%% \url{dlicata@wesleyan.edu}}
%% \and
%% \IEEEauthorblockN{Michael Shulman}
%% \IEEEauthorblockA{University of San Diego \\
%%   \url{shulman@sandiego.edu}}
%% \and
%% \IEEEauthorblockN{Mitchell Riley}
%% \IEEEauthorblockA{Wesleyan University \\
%%   \url{mvriley@wesleyan.edu}}

%% %% \thanks{
%% %%   ?
%% %% }

%% \author[1]{{Daniel R. Licata}}
%% \author[2]{{Michael Shulman}}
%% \author[1]{{Mitchell Riley}}
%% \affil[1]{Wesleyan University}
%% \affil[2]{University of San Diego}

\maketitle

\section{PROPs as Processes}

\subsection{Symmetric Monoidal PROPs}

\newcommand\const[3]{\ensuremath{\dsd{#1}(#2 \mid #3)}}
\newcommand\cpy[2]{\ensuremath{\dsd{copy}(#1 \mid #2)}}
\newcommand\cut[2]{\ensuremath{\nu #1. \, #2}}
\newcommand\lst[1]{\ensuremath{\overline{#1}}}
\newcommand\bfunc[2]{\ensuremath{#1 \bullet #2}}
\newcommand\idp[1]{\ensuremath{1_{#1}}}

\[
\begin{array}{l}
\psi,\phi ::= \cdot \mid \psi, x:p\\
b ::= \cpy{x}{y} \mid \const{f}{\lst{x}}{\lst{y}}\\
e ::= \cut{\psi}{\lst{b}}
\end{array}
\]

Contexts are treated as \emph{lists}: we do not regard $x:p,y:q$ and
$y:q,x:p$ as literally the same context (it might simplify things to do
so, but the correspondence with the semantics would be less clear).  

A basic process $b$ is either an identity \cpy{x}{y} or a generator
\dsd{f} applied to arguments.  We sometimes write \lst{x_i} for a list
of variables $x_1,x_2,\ldots,x_n$.  A process $e$ binds some
``private'' names $\psi$ and then consists of a list of basic processes.

Here are the rules for basic processes, and then general processes.  We
write $\psi \splits \psi_1;\ldots;\psi_n$ to mean that $\psi_1$ through
$\psi_n$ is a shuffle of $\psi$, i.e.  an interleaving of $\psi_1$
through $\psi_n$ is equal to $\psi$.  When $\psi_j : 1 \le j \le n$ is a
sequence of contexts, we write \lst{\psi_{<j}} for the concatenation of
all contexts $\psi_1,\psi_2,\ldots,\psi_{j-1}$.  

\newcommand\tseq[3]{\ensuremath{#1 \vdash #2 : #3}}

\[
\begin{array}{c}
\infer{\tseq{\lst{x_i:p_i}}
            {\const{f}{\lst{x_i}}{\lst{y_j}}}
            {\lst{y_j:q_j}}}
      {  \dsd{f} \in G({\lst{p_i}};{\lst{q_j}})   }
\qquad
\infer{\tseq{x:p}{\cpy{x}{y}}{y:p}}{}
\qquad
\infer{\tseq{\psi}{\cut{\psi_0}{(b_1,\ldots,b_n)}}{\phi}}
      { 
       \begin{array}{l}
        \psi \splits \psi_1;\ldots;\psi_n \\
        \phi \splits \phi_1;\ldots;\phi_n \\
        \psi_0 \splits \psi_1^{o};\ldots;\psi_{n-1}^o;\cdot \\
        \psi_0 \splits \cdot;\psi_2^i,\ldots,\psi_n^{i} \\
        \forall 1 \le j \le n: \psi_j^{i} \subseteq \lst{\psi_{<j}^o} \\
        \forall 1 \le j \le n: \tseq{\psi_j,\psi_j^i}{b_j}{\phi_j,\psi_j^o}\\
       \end{array}
      }
\end{array}
\]
We refer to the four shuffles as the source, target, output, and input
respectively.  

Here are some special cases of the third rule:
\begin{itemize}
\item In the case where $\psi_0$ is empty, e.g. \cut{()}{b_1,b_2}, we
  have functoriality of $,$ (``parallel composition of processes''):
\[
\infer{\tseq{\psi}{\cut{()}{(b_1,b_2)}}{\phi}}
      { 
        \psi \splits \psi_1;\psi_2 & 
        \phi \splits \phi_1;\phi_2 &
        \tseq{\psi_1}{b_1}{\phi_1} &
        \tseq{\psi_2}{b_2}{\phi_2}
      }
\]

\item In the case where $\psi_i$ is empty for $i > 1$ and $\phi_i$ is
  empty for $i<n$, and $\psi^{i}_j \deq \psi^o_{j-1}$, we have an
  interated sequence of compostions, e.g.
\[
\begin{array}{l}
\infer{\tseq{\psi}{\cut{(\psi_0)}{(b_1,b_2)}}{\phi}}
      { 
        \psi \splits \psi;\cdot & 
        \phi \splits \cdot;\phi &
        \psi_0 \splits \psi_0;\cdot &
        \psi_0 \splits \cdot;\psi_0 &
        \tseq{\psi}{b_1}{\psi_0} &
        \tseq{\psi_0}{b_2}{\phi}
      }
\\ \\
\infer{\tseq{\psi}{\cut{(\psi_0)}{(b_1,b_2,b_3)}}{\phi}}
      { 
        \psi \splits \psi;\cdot;\cdot & 
        \phi \splits \cdot;\cdot;\phi &
        \psi_0 \splits \psi_1;\psi_2;\cdot &
        \psi_0 \splits \cdot;\psi_1;\psi_2 &
        \tseq{\psi}{b_1}{\psi_1} &
        \tseq{\psi_1}{b_2}{\psi_2} &
        \tseq{\psi_2}{b_3}{\phi}
      }
\end{array}
\]

\item When $\phi$ and $\psi^o$ are both non-trivial in some position,
  the term can both send outputs into subsequent cuts and into the
  overall output, e.g.
\[
\begin{array}{l}
\infer{\tseq{\psi}{\cut{(\psi_0)}{(b_1,b_2)}}{\phi}}
      { 
        \psi \splits \psi;\cdot & 
        \phi \splits \phi_1;\phi_2 &
        \psi_0 \splits \psi_0;\cdot &
        \psi_0 \splits \cdot;\psi_0 &
        \tseq{\psi}{b_1}{\phi_1,\psi_0} &
        \tseq{\psi_0}{b_2}{\phi_2}
      }
\end{array}
\]
This implicitly uses functoriality of $,$ to route the $\phi_1$ to the
overall output.  

\item When $\psi^i$ is not the previous $\psi^o$, there is some
  implicit tensoring with the identity to route outputs down the line,
  e.g. 
\[
\infer{\tseq{\psi}{\cut{(\psi_0)}{(b_1,b_2,b_3)}}{\phi}}
      { 
        \psi \splits \psi;\cdot;\cdot & 
        \phi \splits \cdot;\phi_1;\phi_2 &
        \psi_0 \splits \psi_1,\psi_2;\cdot;\cdot &
        \psi_0 \splits \cdot;\psi_1;\psi_2 &
        \tseq{\psi}{b_1}{\psi_1,\psi_2} &
        \tseq{\psi_1}{b_2}{\phi_1} &
        \tseq{\psi_2}{b_3}{\phi_2}
      }
\]

\item The condition that each $\psi^i$ is a subset of the previous
  outputs prevents feedback loops / something from being used before it
  is computed.
\end{itemize}

\subsubsection{Admissible Rules}

Admissible rules of identity, functoriality, composition, and scope extrusion:
\[
\begin{array}{c}
\infer{\tseq{\psi}{\idp{\psi}}{\psi}}
      {%% \psi \splits x_1;x_2;\ldots;x_n &
       %%  \tseq{x_i:p_i}{\cpy{x_i}{x_i}}{x_i:p_i}
      }
\qquad
\infer{\tseq{\psi,\psi'}{\bfunc{e}{e'}}{\phi,\phi'}}
      {
        \tseq{\psi}{e}{\phi} &
        \tseq{\psi'}{e'}{\phi'} &
      }
\qquad
\infer{\tseq{\psi_1}{e_1;e_2}{\psi_3}}
      {\tseq{\psi_1}{e_1}{\psi_2} &
       \tseq{\psi_2}{e_2}{\psi_3}}
\qquad
\infer{\tseq{\psi}{\cut{\psi_0}{b_1,\ldots,e_k,\ldots,b_n}}{\phi}}
      {
        \begin{array}{l}
        \psi \splits \psi_1;\ldots;\psi_n \\
        \phi \splits \phi_1;\ldots;\phi_n \\
        \psi_0 \splits \psi_1^{o};\ldots;\psi_{n-1}^o;\cdot \\
        \psi_0 \splits \cdot;\psi_2^i,\ldots,\psi_n^{i} \\
        \forall 1 \le j \le n: \psi_j^{i} \subseteq \lst{\psi_{<j}^o} \\
        \forall j \ne k: \tseq{\psi_j,\psi_j^i}{b_j}{\phi_j,\psi_j^o}\\
        \tseq{\psi_k,\psi_k^i}{e_k}{\phi_k,\psi_k^o}
        \end{array}
      }
\end{array}
\]

\begin{itemize}
\item Identity:
\[
\infer{\tseq{\psi}{\idp{\psi} := \cut{()}{\cpy{x_1}{x_1},\ldots,\cpy{x_n}{x_n}}}{\psi}}
      {\psi \splits x_1;x_2;\ldots;x_n &
       \tseq{x_i:p_i}{\cpy{x_i}{x_i}}{x_i:p_i}
      }
\]

\item 
Functoriality: Given 
\[
\infer{\tseq{\psi}{\cut{\psi_0}{(b_1,\ldots,b_n)}}{\phi}}
      { 
       \begin{array}{l}
        \psi \splits \psi_1;\ldots;\psi_n \\
        \phi \splits \phi_1;\ldots;\phi_n \\
        \psi_0 \splits \psi_1^{o};\ldots;\psi_{n-1}^o;\cdot \\
        \psi_0 \splits \cdot;\psi_2^i,\ldots,\psi_n^{i} \\
        \psi_j^{i} \subseteq \lst{\psi_{<j}^o} \\
        \tseq{\psi_j,\psi_j^i}{b_j}{\phi_j,\psi_j^o}\\
       \end{array}
      }
\qquad
\infer{\tseq{\psi'}{\cut{\psi_0'}{(b_1',\ldots,b_m')}}{\phi'}}
      { 
       \begin{array}{l}
        \psi' \splits \psi_1';\ldots;\psi_m' \\
        \phi' \splits \phi_1';\ldots;\phi_m' \\
        \psi_0' \splits {\psi_1^{o}}';\ldots;{\psi_{m-1}^o}';\cdot \\
        \psi_0' \splits \cdot;{\psi_2^i}',\ldots,{\psi_m^{i}}' \\
        {\psi_j^{i}}' \subseteq \lst{{\psi_{<j}^o}'} \\
        \tseq{\psi_j',{\psi_j^i}'}{b_j'}{\phi_j',{\psi_j^o}'}\\
       \end{array}
      }
\]
we make (by just concatenating all the contexts and shuffles)
\[
\infer{\tseq{\psi,\psi'}{\bfunc {e}{e'} := \cut{(\psi_0,\psi_0')}{(\lst{b_j},\lst{b_j'})}}{\phi,\phi'}}
      { 
       \begin{array}{l}
        \psi,\psi' \splits \psi_1;\ldots;\psi_n;\psi_1';\ldots;\psi_m' \\
        \phi,\phi' \splits \phi_1;\ldots;\phi_n;\phi_1';\ldots;\phi_m' \\
        \psi_0,\psi_0' \splits \psi_1^{o};\ldots;\psi_{n-1}^o;\cdot;{\psi_1^{o}}';\ldots;{\psi_{m-1}^o}';\cdot \\
        \psi_0,\psi_0' \splits \cdot;\psi_2^i,\ldots,\psi_n^{i};\cdot;{\psi_2^i}',\ldots,{\psi_m^{i}}' \\
        \psi_j^{i} \subseteq \lst{\psi_{<j}^o} \quad {\psi_j^{i}}' \subseteq \psi_0,\lst{{\psi_{<j}^o}'} \\
        \tseq{\psi_j,\psi_j^i}{b_j}{\phi_j,\psi_j^o} \quad \tseq{\psi_j',{\psi_j^i}'}{b_j'}{\phi_j',{\psi_j^o}'}\\
       \end{array}
      }
\]

\item 
Composition: Given 
\[
\infer{\tseq{\psi}{\cut{\psi_0}{(b_1,\ldots,b_m)}}{\psi'}}
      { 
       \begin{array}{l}
        \psi \splits \psi_1;\ldots;\psi_n \\
        \psi' \splits \phi_1;\ldots;\phi_n \\
        \psi_0 \splits \psi_1^{o};\ldots;\psi_{n-1}^o;\cdot \\
        \psi_0 \splits \cdot;\psi_2^i,\ldots,\psi_n^{i} \\
        \psi_j^{i} \subseteq \lst{\psi_{<j}^o} \\
        \tseq{\psi_j,\psi_j^i}{b_j}{\phi_j,\psi_j^o}\\
       \end{array}
      }
\qquad
\infer{\tseq{\psi'}{\cut{\psi_0'}{(b_1',\ldots,b_n')}}{\phi'}}
      { 
       \begin{array}{l}
        \psi' \splits \psi_1';\ldots;\psi_m' \\
        \phi' \splits \phi_1';\ldots;\phi_m' \\
        \psi_0' \splits {\psi_1^{o}}';\ldots;{\psi_{m-1}^o}';\cdot \\
        \psi_0' \splits \cdot;{\psi_2^i}',\ldots,{\psi_m^{i}}' \\
        {\psi_j^{i}}' \subseteq \lst{{\psi_{<j}^o}'} \\
        \tseq{\psi_j',{\psi_j^i}'}{b_j'}{\phi_j',{\psi_j^o}'}\\
       \end{array}
      }
\]
we make 
\[
\infer{\tseq{\psi}{\cut{(\psi_0,\psi_0',\psi')}{(\lst{b_j},\lst{b_j'})}}{\phi'}}
      { 
       \begin{array}{l}
        \psi \splits \psi_1;\ldots;\psi_n;\cdot;\ldots;\cdot \\
        \phi' \splits \cdot;\ldots;\cdot;\phi_1;\ldots;\phi_m \\
        \psi_0,\psi_0',\psi' \splits (\phi_1,\psi_1^{o});\ldots;(\phi_{n-1},\psi_{n-1}^o);(\phi_n,\cdot);{\psi_1^{o}}';\ldots;{\psi_{m-1}^o}';\cdot \\
        \psi_0,\psi_0',\psi' \splits \cdot;\psi_2^i,\ldots,\psi_n^{i};(\psi_1',\cdot);(\psi_2',{\psi_2^i}'),\ldots,(\psi_m',{\psi_m^{i}}') \\
        \psi_j^{i} \subseteq \lst{(\phi_k,\psi_{k}^o)_{k<j}} \quad
            {(\psi_j',{\psi_j^{i}}')} \subseteq \psi',\lst{{\psi_{<j}^o}'} \\ 
        \tseq{\psi_j,\psi_j^i}{b_j}{\phi_j,\psi_j^o} \quad 
        \tseq{\psi_j',{\psi_j^i}'}{b_j'}{\phi_j',{\psi_j^o}'}\\
       \end{array}
      }
\]
We concatenate the $n$ processes of the first morphism (we will call
this the first ``half'') with the $m$ of the second (and the second
``half'', even though $n$ is not necessarily equal to $m$).  The source
shuffle is the source shuffle from the first half and then constant on
the second half.  The target shuffle is contant on the first half and
then the target shuffle from the second half.  The output shuffles are
concatenated, and then the target shuffle of the first half is added to
the output shuffle of the first half.  The input shuffles are
concatenated, and the source shuffle from the second half is added to
the input shuffle for the second half.

\item More generally, if we have general process with a general process
  where one of the basic processes should go
\[
\infer{\tseq{\psi}{\cut{\psi_0}{b_1,\ldots,e_k,\ldots,b_n}}{\phi}}
      {
        \begin{array}{l}
        \psi \splits \psi_1;\ldots;\psi_n \\
        \phi \splits \phi_1;\ldots;\phi_n \\
        \psi_0 \splits \psi_1^{o};\ldots;\psi_{n-1}^o;\cdot \\
        \psi_0 \splits \cdot;\psi_2^i,\ldots,\psi_n^{i} \\
        \forall 1 \le j \le n: \psi_j^{i} \subseteq \lst{\psi_{<j}^o} \\
        \forall j \ne k: \tseq{\psi_j,\psi_j^i}{b_j}{\phi_j,\psi_j^o}\\
        \tseq{\psi_k,\psi_k^i}{e_k}{\phi_k,\psi_k^o}
        \end{array}
      }
\qquad
\infer{\tseq{\psi_k,\psi_k^i}{e_k := \cut{\psi_0'}{c_1,\ldots,c_m}}{\phi_k,\psi_k^o}}
      {
        \begin{array}{l}
        {\psi_k,\psi_k^i} \splits \psi_1';\ldots;\psi_m' \\
        {\phi_k,\psi_k^o} \splits \phi_1';\ldots;\phi_m' \\
        \psi_0' \splits {\psi_1^{o}}';\ldots;{\psi_{m-1}^o}';\cdot \\
        \psi_0' \splits \cdot;{\psi_2^i}',\ldots,{\psi_m^{i}}' \\
        {\psi_j^{i}}' \subseteq \lst{{\psi_{<j}^o}'} \\
        \tseq{\psi_j',{\psi_j^i}'}{c_j}{{\phi_j}',{\psi_j^o}'}\\
        \end{array}
      }
\]
then we can merge them
\[
\infer{\tseq{\psi}{\cut{\psi_0,\psi_0'}{b_1,\ldots,c_1,\ldots,c_m,\ldots,b_n}}{\phi}}
      {
        \begin{array}{l}
        \psi \splits \psi_1;\ldots;({\psi_1'}^s;\ldots;{\psi_m'}^s);\ldots;\psi_n \\
        \phi \splits \phi_1;\ldots;({\phi_1'}^t;\ldots;{\phi_m'}^t);\ldots;\phi_n \\
        \psi_0,\psi_0' \splits \psi_1^{o};\ldots;(({\psi_1^{o}}',{\phi_1'}^o);\ldots;({\psi_{m-1}^o}',{\phi_{m-1}'}^o);(\cdot,{\phi_{m}'}^o));\ldots;\psi_{n-1}^o;\cdot \\
        \psi_0,\psi_0' \splits \cdot;\psi_2^i,\ldots,((\cdot,{\psi_1'}^i);({\psi_2'}^i,{\psi_2^i}'),\ldots,{\psi_m^{i}}',{\psi_m'}^i),\ldots,\psi_n^{i} \\
        \text{(constraints are preserved)}\\
        \tseq{\psi_j,\psi_j^i}{b_j}{\phi_j,\psi_j^o}\\
        \tseq{\psi_j',{\psi_j^i}'}{c_j}{{\phi_j}',{\psi_j^o}'}\\
        \end{array}
      }
\]
We write $\psi_j'^s$ for the part that came from $\psi_k$ and 
${\psi_j'}^i$ for the part that came from $\psi_k^i$.  
Similarly we write 
${\phi_j'}^t$ for the part that came from $\phi_k$ and 
${\phi_j'}^o$ for the part that came from $\psi_k^o$.  

I haven't checked, but it seems like it should be the case that a term
\cut{\psi_0}{e_1,\ldots,e_n} expands to the same thing (at least up to
the equational theory below) no matter what order you expand things in,
so is unambiguous.
\end{itemize}

\subsubsection{Equational Theory}

First, we have an interchange rule, which says that the order of
\emph{independent} processes doesn't matter:
\[
\dsd{Interchange} \qquad
\cut{\psi_0}{(\ldots,b_j,b_k,\ldots)} \deq \cut{\psi_0}{(\ldots,b_k,b_j,\ldots)} 
\text{ if } \psi^i_k \# \psi^o_j
\]
Officially, the equality rule has premises saying that the left-hand
side is well-typed.  The side condition says that the outputs of $b_j$
are not inputs to $b_k$, so the swapped version is also well-typed.  The
shuffles in the derivation need to be adjusted in the corresponding
way.  

Second, we have some unit laws.  Suppose some process
\tseq{\psi_j,\psi_j^i}{b_j}{\phi_j,\psi_j^o} is \cpy{x}{y}, then we have
the following definitional equalities:
\[
\begin{array}{crcll}
\dsd{Id1} & \cut{\psi_0}{(\lst{b},\cpy{x}{y},\lst{b'})} & \deq & \cut{(\psi_0-y)}{(\lst{b},\lst{b'}[x/y])} & \text{ if } \psi_j = x, \psi_j^i = \cdot, \phi_j = \cdot, \psi_j^o = y\\
\dsd{Id2} & \cut{\psi_0}{(\lst{b},\cpy{x}{y},\lst{b'})} & \deq & \cut{(\psi_0-x)}{(\lst{b}[y/x],\lst{b'})} & \text{ if } \psi_j = \cdot, \psi_j^i = x, \phi_j = y, \psi_j^o = \cdot\\
\dsd{Id3} & \cut{\psi_0}{(\lst{b},\cpy{x}{y},\lst{b'})} & \deq & \cut{(\psi_0-y)}{(\lst{b},\lst{b'}[x/y])} & \text{ if } \psi_j = \cdot, \psi_j^i = x, \phi_j = \cdot, \psi_j^o = y\\
%% \cut {(\psi_0,x,\psi_0')}{\cpy{x'}{x},c} & \deq &
%% \cut{(\psi_0,\psi_0')}{c[x'/x]} \text{ if $x \neq x'$ }\\
%% \cut {(\psi_0,x,\psi_0')}{\cpy{x}{x'},c} & \deq &
%% \cut{(\psi_0,\psi_0')}{c[x'/x]} \text{ if $x \neq x'$ }
\end{array}
\]
There is no reduction if $x \in \psi_j,y \in \phi_j$ because this copy
is visible to the output.  

Finally, the order of variables in $\psi_0$ doesn't matter (adjusting
the shuffles accordingly in the derivation): 
\[
\begin{array}{crcll}
\dsd{Swap} & \cut{(\psi_0,x,y,\psi_0')}{\lst{b}} & \deq & \cut{(\psi_0,y,x,\psi_0')}{\lst{b}}
\end{array}
\]
From this and the third identity law and some $\alpha$-renaming, we
should be able to get the symmetric version of the third identity law:
\[
\begin{array}{rcll}
\cut{\psi_0}{(\lst{b},\cpy{x}{y},\lst{b'})} & \deq & \cut{(\psi_0-x)}{(\lst{b}[y/x],\lst{b'})} & \text{ if } \psi_j = \cdot, \psi_j^i = x, \phi_j = \cdot, \psi_j^o = y\\
\end{array}
\]

Definitional equality consists of the axioms \dsd{Interchange},
\dsd{Swap}, \dsd{Id\{1,2,3\}} plus the following rules (we elide the
typing premises necessary to make both sides well-typed, but officially
they're there):
\[
\begin{array}{c}
\infer{\tseq{\psi}{e \deq e'}{\phi}}
      {{e \deq e'} \in G}
\qquad
\infer{\tseq{\psi}{e \deq e}{\psi}}
      {}
\qquad
\infer{\tseq{\psi}{e_1 \deq e_2}{\psi}}
      {\tseq{\psi}{e_2 \deq e_1}{\psi}}
\qquad
\infer{\tseq{\psi}{e_1 \deq e_3}{\psi}}
      {\tseq{\psi}{e_1 \deq e_2}{\psi} &
        \tseq{\psi}{e_2 \deq e_3}{\psi} 
      }
\\ \\
\infer{\tseq{\psi}{\cut{\psi_0}{b_1,\ldots,e,\ldots,b_n} \deq \cut{\psi_0}{b_1,\ldots,e',\ldots,b_n}}{\phi}}
      {\tseq{\psi_j,\psi^i_j}{e \deq e'}{\phi_j,\psi^o_j} 
      }
\end{array}
\]

\subsubsection{Example: exchange}

For example, we have the exchange term
\[
\infer{\tseq{x:p,y:q}{\cut{()}{\cpy{x}{x'},\cpy{y}{y'}}}{y':q,x':p}}
      {x:p,y:q \splits x:p;y:q &
       y':q,x':p \splits x':p;y':q &
       \tseq{x:p}{\cpy{x}{x'}}{x':p} &
       \tseq{y:q}{\cpy{y}{y'}}{y':q}
      }
\]
Composing
\[
(\cut{()}{\cpy{x}{x'},\cpy{y}{y'}});(\cut{()}{\cpy{x'}{x''},\cpy{y'}{y''}})
\]
is  
\[
\cut{(x',y')}{\cpy{x}{x'},\cpy{y}{y'},\cpy{x'}{x''},\cpy{y'}{y''}}
\]
which is equal to 
\[
\cut{(x',y')}{\cpy{x}{x''},\cpy{y}{y''}}
\]

\subsubsection{Example: Duals and Traces}

Axioms:
\[
\begin{array}{l}
\tseq{\cdot}{\const{\eta_A}{}{x,y}}{x:A,y:A^*}\\
\tseq{y:A^*,x:A}{\const{\epsilon_A}{y,x}{}}{\cdot}\\
\tseq{x:A}{\cut{y:A^*}{\const{\eta_A}{}{x',y},{\const{\epsilon_A}{y,x}{}} \deq \cpy{x}{x'} }}{x':A} \\
\tseq{y:A^*}{\cut{x:A}{\const{\eta_A}{}{x,y'},{\const{\epsilon_A}{y,x}{}} \deq \cpy{y}{y'} }}{y':A^*} \\
\end{array}
\]
We refer to the first equation as \emph{duality} (and won't use the second).  

Assume a basic process \tseq{u:A}{\const{h}{u}{v}}{v:A}.  The trace of $h$ is written
\[
\tseq{\cdot}{\cut{(x:A,y:A^*,x':A)}{\const{\eta_A}{}{x,y},\const{h}{x}{x'},\const{\epsilon_A}{y,x'}{}}}{\cdot}
\]

Assume both $A$ and $B$ have duals and we have basic processes
\tseq{x:A}{\const{f}{x}{u}}{u:B} and \tseq{u:B}{\const{g}{u}{x}}{x:A}.
Then
\[
\begin{array}{l}
tr(f;g) := \tseq{\cdot}{\cut{(x:A,y:A^*,u:B,x':A)}{\const{\eta_A}{}{x,y},\const{f}{x}{u},\const{g}{u}{x'},\const{\epsilon_A}{y,x'}{}}}{\cdot}\\
tr(g;f) := \tseq{\cdot}{\cut{(u:B,v:B^*,x:A,u':B)}{\const{\eta_B}{}{u,v},\const{g}{u}{x},\const{f}{x}{u'},\const{\epsilon_A}{v,u'}{}}}{\cdot}
\end{array}
\]

These are equal as follows:
\[
\begin{array}{llll}
     & \nu {(x:A,y:A^*,u:B,x':A)} & {\const{\eta_A}{}{x,y},\const{f}{x}{u},\const{g}{u}{x'},\const{\epsilon_A}{y,x'}{}} \\
\deq & \nu {(x:A,y:A^*,u:B,u':B,x':A)} & {\const{\eta_A}{}{x,y},\const{f}{x}{u'},\cpy{u'}{u},\const{g}{u}{x'},\const{\epsilon_A}{y,x'}{}} & identity \\
\deq & \nu {(x:A,y:A^*,u:B,u':B,x':A,v:B^*)} & {\const{\eta_A}{}{x,y},\const{f}{x}{u'}, \const{\eta_B}{}{u,v},{\const{\epsilon_A}{v,u'}{}} ,\const{g}{u}{x'},\const{\epsilon_A}{y,x'}{}} & duality_A\\
\deq & \nu {(u:B,v:B^*,x':A,x:A,u':B,y:A^*)} & \const{\eta_B}{}{u,v},\const{g}{u}{x'},\const{\eta_A}{}{x,y},{\const{\epsilon_A}{y,x'}{}},\const{f}{x}{u'},\const{\epsilon_B}{v,u'}{} & exchange \\
\deq & \nu {(u:B,v:B^*,x':A,x:A,u':B)} & \const{\eta_B}{}{u,v},\const{g}{u}{x'},\cpy{x'}{x},\const{f}{x}{u'},\const{\epsilon_B}{v,u'}{} & duality_B \\
\deq & \nu {(u:B,v:B^*,x:A,u':B)} & {\const{\eta_B}{}{u,v},\const{g}{u}{x},\const{f}{x}{u'},\const{\epsilon_B}{v,u'}{}} & identity
\end{array}
\]
On the duality lines, we have used the congruence rule for scope
extrusion, which by definition floats the cut variables of each term out
to the outside.

\subsubsection{Exmaple: Mode Theory for Classical Linear Logic}

\newcommand\lm{\dsd{l}}
\newcommand\mo[1]{\ensuremath{1_{#1}}}
\newcommand\mten[3]{#1 \otimes #2 \rhd #3}
\newcommand\mz[1]{\dsd{0}_{#1}}
\newcommand\mpar[3]{#1 \lhd #2 \bindnasrepma #3}
\newcommand\pars[0]{\bindnasrepma}

We have one mode $\lm$ with a commutative monoid
\begin{itemize}
\item \tseq{\cdot}{\mo x}{x : \lm}
\item \tseq{x : \lm, y : \lm}{(\mten x y z)}{z : \lm}
\item \tseq{y : \lm}{(\cut{x}{\mo x, \mten{x}{y}{z}}) \deq \cpy{y}{z}}{z : \lm}
\item \tseq{x : \lm}{(\cut{y}{\mo y, \mten{x}{y}{z}}) \deq \cpy{x}{z}}{z : \lm}
\item \tseq{x : \lm,y:\lm,z:\lm}{(\cut{w}{\mten{x}{y}{w},\mten{w}{z}{u}}) \deq (\cut{w}{\mten{y}{z}{w},\mten{x}{w}{u}})}{u : \lm}
\item \tseq{x : \lm,y:\lm}{(\cut{()}{\mten{x}{y}{u}}) \deq (\cut{()}{\mten{y}{x}{u}}) }{u : \lm}
\end{itemize}
and a commutative comonoid
\begin{itemize}
\item \tseq{x : \lm}{\mz x}{\cdot}
\item \tseq{z : \lm}{(\mpar z x y)}{x : \lm, y : \lm}
\item \tseq{z : \lm}{(\cut{x}{\mpar{z}{x}{y},\mz x}) \deq \cpy{z}{y}}{y : \lm}
\item \tseq{z : \lm}{(\cut{y}{\mpar{z}{x}{y},\mz y}) \deq \cpy{z}{x}}{x : \lm}
\item \tseq{u : \lm}{(\cut{w}{\mpar{u}{w}{z},\mpar{w}{x}{y}}) \deq (\cut{w}{\mpar{u}{x}{w},\mpar{w}{y}{z}})}{x : \lm,y:\lm,z:\lm}
\item \tseq{u : \lm}{(\cut{()}{\mpar{u}{x}{y}}) \deq (\cut{()}{\mpar{u}{y}{x}}) }{x : \lm,y:\lm}
\end{itemize}
and the Frobenius axiom
\begin{itemize}
\item \tseq{(x:\lm,y:\lm)}{(\cut{w}{\mpar{x}{u}{w}, \mten w y v}) \deq (\cut{w}{\mten x y w, \mpar{w}{u}{v}})}{(u:\lm,v:\lm)}
(These are also equal to (\cut{w}{\mpar{y}{w}{v}, \mten x w u}) 
by commutativity.)
\end{itemize}

Write $x_1 \otimes \ldots \otimes x_n \rhd w_n$ for 
\[
\nu (w_0,w_1,\ldots,w_n). \mo{w_0}, \mten {w_0} {x_1} {w_1}, \mten {w_1} {x_2} {w_2}, \ldots, \mten {w_{n-1}} {x_n} {w_n}
\]
and dually write  $u_n \lhd y_1 \pars \ldots \pars y_n$ for 
\[
\nu (u_0,u_1,\ldots,u_n). \mpar {u_0} {y_1} {u_1}, \ldots, \mpar {y_{n-2}} {y_{n-1}} {u_{n-1}}, \mpar {u_{n-1}} {y_n} {u_n}, \mz {u_n}
\]
Then a linear logic sequent $x_1 : A_1, \ldots x_n : A_n \vdash y_1 : B_1, \ldots y_m : B_m$
is over $x_1 \otimes \ldots \otimes x_n \rhd w, w \lhd y_1 \pars \ldots \pars y_n$.  

The Frobenius axiom is used in cut, e.g. in cutting $a:A,b:B \vdash
c:C,d:D$ with $d:D,e:E \vdash f:F$ to get $a:A,b:B,e:E \vdash c:C,f:F$,
the mode morphisms of the premises of the cut are
\[
\nu w. \mten a b w, \mpar w c d  \qquad \mten d e f
\]
Composing the mode morphisms along $d$ gives
\[
\nu w,d. \mten a b w, \mpar w c d, \mten d e f
\]
which is by Frobenius
\[
\nu w,d. \mten a b w, \mten w e d, \mpar d c f
\]

\subsubsection{Example: Mode Theory for Directed Type Theory}

\newcommand{\dualm}[1]{\ensuremath{#1^{\mathord{\circ}}}}

We write modes as $A$ rather than $p$, and assume each mode has an
associated other mode $\dualm A$ (we don't treat this as a type
constructor because then we'd have to think about $\dualm {{\dualm A}}$).
We have the dualization axioms as above, which we write as \dsd{s} and
\dsd{t} for whether the variables are paired in the source or the
target.  
\[
\begin{array}{l}
\tseq{\cdot}{\const{t}{}{x,y}}{x:A,y:\dualm A}\\
\tseq{y:\dualm A,x:A}{\const{s}{y,x}{}}{\cdot}\\
\tseq{x:A}{\cut{y:\dualm A}{\const{t}{}{x',y},{\const{s}{y,x}{}} \deq \cpy{x}{x'} }}{x':A} \\
\tseq{y:\dualm A}{\cut{x:A}{\const{t}{}{x,y'},{\const{s}{y,x}{}} \deq \cpy{y}{y'} }}{y':\dualm A} \\
\end{array}
\]
Then a sequent 
\[
x:A,y:B,z:C \mid M(x^+,x^-,y^+) \vdash N(z^+,z^-,y^+)
\]
would be represented by 
\[
x:A, x':\dualm{A}, y:B \mid M(x,x',y) \vdash_{\const{s}{x',x}{},\const{t}{}{z,z'},\cpy{y}{y'}} z:C, z':\dualm{C}, y':B \mid N(z,z',y')
\]

\subsection{Logic over a 2-PROP}

Write $\alpha$ instead of $e$ for the morphisms of a PROP, and suppose
there is 2-cell judgement $\alpha \spr \alpha'$ given by rules like
equality above but without symmetry.

Types have a mode context
\[
\wftype{\psi}{A}
\]
where what was previously $A_p$ would now be $\wftype{x:p}{A}$.  

The sequent judgement is
\[
\seq{(\ct {\psi_1}{A_1}),\ldots,(\ct {\psi_n}{A_n})}{\alpha}{(\ct {\phi_1}{B_1}),\ldots,(\ct {\phi_m}{B_m})}
\text { where } 
\wftype{\psi_i}{A_i}, \wftype{\phi_i}{B_i}, 
\tseq{\psi_1,\ldots,\psi_n}{\alpha}{\phi_1,\ldots,\phi_m}.
\]
We write $\Gamma$, $\Delta$ to abbreviate contexts.    

Local fibration (should be admissible):
\[
\infer{\seq{\Gamma}{\beta}{\Delta}}
      {\beta \spr \alpha &
       \seq{\Gamma}{\alpha}{\Delta}}
\]

Identity (should be admissible):
\[
\infer{\seq{\ct \psi A}{\idp{\psi}}{\ct \psi A}}
      {}
\]

Composition (should be admissible):
\[
\infer{\seq{\Gamma}{\alpha;\beta}{\Gamma''}}
      {\seq{\Gamma}{\alpha}{\Gamma'} &
       \seq{\Gamma'}{\beta}{\Gamma''}}
\]
%% \[
%% \infer{\seq{\Gamma_1,\Gamma_2}{\cut{\psi}{\alpha,\beta}}{\Delta_1,\Delta_2}}
%%       {\seq{\Gamma_1}{\alpha}{\Delta_1,(\ct \psi A)} &
%%         \seq{\Gamma_2,(\ct \psi A)}{\beta}{\Delta_2}}
%% \]

Functoriality/mix (should be admissible):
\[
\infer{\seq{\Gamma,\Gamma'}{\bfunc {\alpha} {\alpha'}}{\Delta,\Delta'}}
      {\seq{\Gamma}{\alpha}{\Delta} &
       \seq{\Gamma'}{\alpha'}{\Delta'}}
\]

Atomic hypothesis rule:
\[
\infer[\text{TODO: build in functoriality}]
      {\seq{\ct \psi P}{\beta}{\ct \psi P}}
      {\beta \spr \idp{\psi} &
        \wftype \psi P}
\]

\Fsymb-types:

\[
\infer{\wftype{\phi}{\F{\alpha}{\Gamma \mid \Delta}}}
      {\Gamma := \ct{\psi_1}{A_1},\ldots,\ct{\psi_n}{A_n} &
       \Delta := \ct{\phi_1}{B_1},\ldots,\ct{\phi_n}{B_n} &
       \wftype{\psi_i}{A_i} &
       \wftype{\phi_i}{B_i} &
       \tseq{\lst{\psi_i}}{\alpha}{\lst{\phi_i},\phi} &
      }
\]

%% \[
%% \infer[\text{TODO: build in cuts}]{\seq{\Gamma}{\alpha}{\Delta,(\ct \phi {\F{\alpha}{\Gamma \mid \Delta}})}}{}
%% \]

\[
\infer{\seq{\Gamma'}{\beta}{(\ct \phi {\F{\alpha}{\Gamma \mid \Delta}}),\Delta'}}
      {\seq{\Gamma'}{\gamma}{\Gamma} &
       \seq{\Delta}{\delta}{\Delta'} &
       \beta \spr \cut{(\lst{\psi_i},\lst{\phi_i})}{(\gamma,\alpha,\delta)}
      }
\]

\[
\infer{\seq{\Gamma,(\ct \phi {\F{\alpha}{\Gamma_0 \mid \Delta_0}}),\Gamma'}{\beta}{\Delta}}
      {\seq{\Gamma,\Gamma_0,\Gamma'}{\cut{\phi}{\alpha,\beta}}{\Delta,\Delta_0}}
\]

\Usymb-types:

\[
\infer{\wftype{\psi}{\U{\alpha}{\Gamma}{\Delta}}}
      {\Gamma := \ct{\psi_1}{A_1},\ldots,\ct{\psi_n}{A_n} &
       \Delta := \ct{\phi_1}{B_1},\ldots,\ct{\phi_n}{B_n} &
       \wftype{\psi_i}{A_i} &
       \wftype{\phi_i}{B_i} &
       \tseq{\lst{\psi_i},\psi}{\alpha}{\lst{\phi_i}} &
      }
\]

\[
\infer[\text{TODO: build in cuts}]{\seq{\Gamma,(\ct \psi {\U{\alpha}{\Gamma}{\Delta}})}{\alpha}{\Delta}}{}
\]

\[
\infer{\seq{\Gamma}{\beta}{\Delta,(\ct \psi {\U{\alpha}{\Gamma_0}{\Delta_0}}),\Delta'}}
      {\seq{\Gamma,\Gamma_0}{\cut{\psi}{\beta,\alpha}}{\Delta,\Delta_0,\Delta'}}
\]

\subsection{Logic over a 2-PROP (Single-conclusion)}

\emph{TODO: Writing this out because it's easier, but is it the case that for
mode theories that don't make use of multiple conclusions, the above
restricts to this?}

Write $\alpha$ instead of $e$ for the morphisms of a PROP, and suppose
there is 2-cell judgement $\alpha \spr \alpha'$ given by rules like
equality above but without symmetry.

Types have a mode context
\[
\wftype{\psi}{A}
\]
where what was previously $A_p$ would now be $\wftype{x:p}{A}$.  

The sequent judgement is
\[
\seq{(\ct {\psi_1}{A_1}),\ldots,(\ct {\psi_n}{A_n})}{\alpha}{(\ct {\phi}{B})}
\text { where } 
\wftype{\psi_i}{A_i}, \wftype{\phi}{B}, 
\tseq{\psi_1,\ldots,\psi_n}{\alpha}{\phi}.
\]
We write $\Gamma$ to abbreviate contexts.    

Local fibration (should be admissible):
\[
\infer{\seq{\Gamma}{\beta}{\ct \phi B}}
      {\beta \spr \alpha &
       \seq{\Gamma}{\alpha}{\ct \phi B}}
\]

Identity (should be admissible):
\[
\infer{\seq{\ct \psi A}{\idp{\psi}}{\ct \psi A}}
      {}
\]

Composition (should be admissible):
\[
\infer{\seq{\Gamma,\Gamma'}{\cut{\psi}{\alpha,\beta}}{\ct \phi B}}
      {\seq{\Gamma}{\alpha}{\ct \psi A} &
       \seq{\Gamma',\ct \psi A}{\beta}{\ct \phi B}}
\]
Note that the well-formedness of the premises implies well-formedness of
the conclusion, unlike usual, where we work bottom-up.  
%% \[
%% \infer{\seq{\Gamma_1,\Gamma_2}{\cut{\psi}{\alpha,\beta}}{\Delta_1,\Delta_2}}
%%       {\seq{\Gamma_1}{\alpha}{\Delta_1,(\ct \psi A)} &
%%         \seq{\Gamma_2,(\ct \psi A)}{\beta}{\Delta_2}}
%% \]

Substitutions:   
\[
\infer{\seq{\Gamma}{\bfunc{\bfunc{\alpha_1}{\ldots}}{\alpha_n}}{(\ct {\phi_1} {B_1}, \ldots, \ct {\phi_n} {B_n})}}
      {\Gamma \splits \Gamma_1; \ldots ; \Gamma_n &
        \seq{\Gamma_i}{\alpha_i}{\ct {\phi_i} {B_i}}
      }
\]

Atomic hypothesis rule:
\[
\infer{\seq{\ct \psi P}{\beta}{\ct \psi P}}
      {\beta \spr \idp{\psi} &
        \wftype \psi P}
\]

\Fsymb-types:

\[
\infer{\wftype{\phi}{\F{\alpha}{\Gamma}}}
      {\Gamma := \ct{\psi_1}{A_1},\ldots,\ct{\psi_n}{A_n} &
       \wftype{\psi_i}{A_i} &
       \tseq{\lst{\psi_i}}{\alpha}{\phi} &
      }
\qquad
\infer{\seq{\Gamma}{\beta}{\ct \phi {\F{\alpha}{\Delta}}}}
      {
        \beta \spr \gamma;\alpha &
        \seq{\Gamma}{\gamma}{\Delta} &
      }
%% \infer{\seq{\Gamma_\Psi}{\beta}{\ct \phi {\F{\alpha}{\Delta^\Phi}}}}
%%       {
%%         \tseq{\lst{\Psi}}{\beta}{\phi} &
%%         \tseq{\lst{\Psi}}{\gamma}{\lst{\Phi}} &
%%         \tseq{\lst{\Phi}}{\alpha}{\phi} &
%%         \beta \spr \gamma;\alpha &
%%         \seq{\Gamma}{\gamma}{\Delta} &
%%       }
\qquad
\infer{\seq{\Gamma,(\ct \phi {\F{\alpha}{\Gamma_0}}),\Gamma'}{\beta}{\ct \phi B}}
      {\seq{\Gamma,\Gamma_0,\Gamma'}{\cut{\phi}{\alpha,\beta}}{\ct \phi B}}
\]

\Usymb-types:

\[
\infer{\wftype{\psi}{\U{\alpha}{\Gamma}{\ct \phi B}}}
      {\Gamma := \ct{\psi_1}{A_1},\ldots,\ct{\psi_n}{A_n} &
       \wftype{\psi_i}{A_i} &
       \wftype{\phi}{B} &
       \tseq{\lst{\psi_i},\psi}{\alpha}{\phi} &
      }
\]

\[
\infer{\seq{\Gamma,(\ct \psi {\U{\alpha}{\Delta}{(\ct \phi B)}})}
           {\alpha}
           {\ct {\phi'} C}}
      {
        \Gamma \splits {\Gamma_1;\Gamma_2} & 
        \seq{\Gamma_1}{\gamma}{\ct{\Phi}{\Delta}} & 
        \seq{\Gamma_2,\ct \phi B}{\beta'}{\ct {\phi'} C} &
        \beta \spr \cut{\phi}{(\cut{\Phi}{\gamma,\alpha}),\beta'}
      }
\qquad
\infer{\seq{\Gamma}{\beta}{\ct \psi {\U{\alpha}{\Delta}{(\ct \phi B)}}}}
      {\seq{\Gamma,\Delta}{\cut{\psi}{\beta,\alpha}}{\ct \phi B}}
\]

\subsection{Example: Directed Type Theory}

Suppose we have modes $A$ with duals $\dualm A$ a separate mode $\lm$
with a commutative monoid.  We encode a module $\Psi \vdash M$ as a type
as $\wftype{\Psi,d : \lm}{M}$, where $d$ is the ``destination'' of the
type.

\begin{itemize}
\item $\Psi \vdash \int^{x:\dualm A,y:A} M^{\psi,x,y} := \F{\cut{()}{\const{s}{x,y}},\idp{\Psi,d}}{\ct {\Psi,d,x,y} M}$
\item $\Psi \vdash \int_{x:\dualm A,y:A} M^{\psi,x,y} := \U{\cut{()}{\const{t}{}{x,y}},\idp{\Psi,d}}{\cdot}{\ct {\Psi,d,x,y} M}$
\item $x : \dualm A, y : A \vdash \dsd{hom}_A(x,y) :=
  \F{\cut{()}{\const{t}{}{x,y},\cdot_d}}{()}$
\item $\Psi_1,\Psi_2 \vdash \ct {\Psi_1} {M_1} \otimes \ct {\Psi_2} {M_2} := \F{\cut{()}{\mten x y d, \idp{\psi_1,\psi_2}}}{M_1[x/d],M_2[y/d]}$
\item $\dualm{\Psi_1},\Psi_2 \vdash \ct {\Psi_1} {M_1} \multimap \ct {\Psi_2} {M_2} := \U{\alpha}{M_1[x/d]}{M_2[y/d]}$
where\\ $\tseq{\Psi_1,x,(\dualm{\Psi_1},\Psi_2,d)}{\alpha := \cut {()} {\const{s}{\dualm{\Psi_1},\Psi_1}{},\idp{\Psi_2},\mten d x y} }{\Psi_2,y}$
\end{itemize}

\section{Bicartesian PROPs}

Bicartesian props are easier, because the shuffles mostly go away:
\[
\begin{array}{c}
\infer{\tseq{\psi}
            {\const{f}{\lst{x_i}}{\lst{y_i}}}
            {\phi}}
      {  \dsd{f} \in G({\lst{p_i}};{\lst{q_i}}) &
        {\lst{x_i:p_i}} \in \psi &
        {\lst{y_i:q_i}} \in \phi
      }
\qquad
\infer{\tseq{\psi}{\cpy{x}{y}}{\phi}}
      {x:p \in \psi &
       y:p \in \phi
      }
\\ \\
\infer{\tseq{\psi}{\cut{\psi_0}{(b_1,\ldots,b_n)}}{\phi}}
      { 
        \psi_0 \splits \psi_1;\psi_2;\ldots;\psi_{n-1};\psi_n &
        \lst{\tseq{\psi,\lst{\psi_{<j}}}{b_j}{\phi,\psi_{j}}}\\
      }
\end{array}
\]

FIXME: think about equational theory between diagonals and codiagonals


%% \emph{FIXME: not updated below here}

%% \section{Modes}

%% Write $p$ for modes and $\psi := x_1:p_1,\ldots,x_n:p_n$ or $\phi$ for
%% mode contexts.  We stipulate that contexts $\psi$ that differ only by
%% the order of variables are equal (i.e. we have exchange).  Write
%% \oftp{\psi}{a}{p} and \oftp{\psi}{\theta}{\psi'} for vertical morphisms,
%% which are linear:
%% \[
%% \infer{\oftp{x:p}{x}{p}}{}
%% \qquad
%% \infer{\oftp{\psi_1,\psi_2}{\theta,a/x}{\psi',x:p}}
%%       {\oftp{\psi_1}{\theta}{\psi} &
%%         \oftp{\psi_2}{a}{p}}
%% \]

%% \newcommand\hmor[3]{\ensuremath{#1 \mid #2 \vdash^{\dsd h} #3}}
%% \newcommand\sq[4]{\ensuremath{#1 \Rightarrow #2 \: [#3 \mid #4]}}
%% \newcommand\hid[1]{1_{#1}}

%% We write a horizontal morphism $\psi \pto \psi'$ profunctor-like, as a
%% term $\hmor{\psi}{\phi}{\alpha}$ with both $\psi$ and $\phi$ as free
%% variables.  These have identity and composition:

%% \[
%% \infer{\hmor{\psi}{\psi}{\hid{\psi}}}{}
%% \qquad
%% \infer{\hmor{\psi_1,\psi_1'}{\psi_3}{[\alpha/\psi_2]\beta}}
%%       {\hmor{\psi_1}{\psi_2}{\alpha} &
%%        \hmor{\psi_1',\psi_2}{\psi_3}{\beta}}
%% \]

%% We also need functoriality of the monoidal product:
%% \[
%% \infer{\hmor{\psi_1,\psi_2}{\psi_2,\psi_2'}{\alpha,\alpha'}}
%%       {\hmor{\psi_1}{\psi_2}{\alpha} &
%%        \hmor{\psi_1'}{\psi_2'}{\alpha'}}
%% \]

%% Exisitng adjoint logic terms $\psi \vdash \alpha : p$ can be encoded in
%% ``destination-passing-style:'' E.g.  instead of $x:p,y:p \vdash x
%% \otimes y : p$, we'd have $x:p,y:p \mid d:p \vdash x \otimes y > d$, and
%% $(x \otimes y) \otimes z$ is the formal composite
%% \[
%% [((x \otimes y) > d_1)/d_1](d_1 \otimes z > d)
%% \]

%% 2-cells are squares \sq{\alpha}{\alpha'}{\theta}{\rho} where
%% $\oftp{\psi}{\theta}{\psi'}$ and 
%% $\oftp{\phi}{\rho}{\phi'}$ and
%% $\hmor{\psi}{\phi}{\alpha}$ an
%% $\hmor{\psi'}{\phi'}{\alpha'}$.  

%% We assume that horizontal morphisms can be composed with vertical
%% morphism on either side via an operation, probably with a filler square
%% relating them too?
%% \[
%% \infer{\hmor{\psi'}{\phi}{\alpha[\theta/\psi]}}
%%       {\hmor{\psi}{\phi}{\alpha} &
%%         \oftp{\psi'}{\theta}{\psi}}
%% \qquad
%% \infer{\hmor{\psi}{\phi'}{\alpha[\theta/\phi]}}
%%       {\hmor{\psi}{\phi}{\alpha} &
%%         \oftp{\psi}{\theta}{\phi}}
%% \]

%% When the vertical 1-cells $\oftp{\Psi}{\theta}{\Psi'}$ are cartesian
%% variable-for-variable substitutions, and the codomain contexts always
%% have 1 variable, this should be clubs.  

%% \section{Judgements}


%% Types and contexts live over mode contexts, so we have judgements
%% \wftype{\psi}{A} and \wfctx{\psi}{\Gamma}.  

%% We require admissible rules pulling back along vertical morphisms:
%% \[
%% \infer{\wftype{\psi_1,\psi_2'}{A[\theta]}}
%%       {\wftype{\psi_1,\psi_2}{A} & 
%%         \oftp{\psi_2'}{\theta}{\psi_2}}
%% \qquad
%% \infer{\wfctx{\psi_1,\psi_2'}{\Gamma[\theta]}}
%%       {\wfctx{\psi_1,\psi_2}{\Gamma} & 
%%         \oftp{\psi_2'}{\theta}{\psi_2}}
%% \]

%% Contexts are defined by
%% \[
%% \infer{\wfctx{\cdot}{\cdot}}{}
%% \qquad
%% \infer{\wfctx{\psi,\psi'}{\Gamma,A}}
%%       {\wfctx{\psi}{\Gamma} & \wfctx{\psi'}{A}}
%% \]

%% TODO: define $n$-ary partial-context substitution; it's a bit of a pain because of linearity.  

%% We should have judgements $\seqh{\psi}{\Gamma}{\theta}{\psi'}{A}$ where
%% $\oftp{\psi}{\theta}{\psi'}$ (vertical morphism over vertical morphism)
%% and $\seqh{\psi}{\Gamma}{\alpha}{\phi}{A}$ where $\wfctx{\psi}{\Gamma}$
%% and $\wftype{\phi}{A}$ and 
%% $\hmor{\psi}{\phi}{\alpha}$ (horizontal morphism over horizontal
%% morphism).  However, because we're thinking of the vertical arrows as
%% terms that the types depend on, and insisting on pullbacks along them,
%% we can shove everything into the fiber rather than having a primitive
%% ``over'' judgement. So a general vertical morphism
%% \seqh{\psi}{\Gamma}{\theta}{\psi'}{A} can be represented by a
%% homogeneous vertical morphism
%% \seqh{\psi}{\Gamma}{x_i/x_i}{\psi}{A[\theta]}.  But a degenerate
%% vertical morphism should be (???) the same as a degenerate horizontal
%% morphism, so it suffices to have horizontal morphisms up top.

%% Structural rules: identity over identity, cut over cut, transporting
%% along a square:
%% \[
%% \infer{\seqh{\psi}{x:A}{\hid{\psi}}{\psi}{A}}{}
%% \qquad
%% \infer{\seqh{\psi_1,\psi_2}{\Gamma_1,\Gamma_2}{[\alpha/\psi]\beta}{\phi}{B}}
%%       {\seqh{\psi_2}{\Gamma_2}{\alpha}{\psi}{A} &
%%        \seqh{\psi_1,\psi}{\Gamma_1,x:A}{\beta}{\phi}{B}}
%% \]

%% \[
%% \infer{\seqh{\psi}{\Gamma[\theta]}{\alpha}{\phi}{A[\rho]}}
%%       {\seqh{\psi'}{\Gamma}{\beta}{\phi'}{A} &
%%         \sq{\alpha}{\beta}{\theta}{\rho}}
%% \]

%% Atomic propositions:
%% \[
%% \infer{\wftype{\psi}{P(\theta)}}
%%       {\psi' \vdash P &
%%         \oftp{\psi}{\theta}{\psi'}
%%       }
%% \qquad
%% \infer{\seqh{\psi}{P(\theta)}{\beta}{\phi}{P(\rho)}}
%%       {\psi' \vdash P &
%%         \sq{\beta}{\hid{\psi'}}{\theta}{\rho}}
%% \]
%% \[
%% P(\theta)[\theta'] := P(\theta[\theta'])
%% \]

%% Pushforward along a horizontal morphism:

%% %% (FIXME: should some of the
%% %% dependency leak through to the conclusion? this seems weird):
%% %% \[
%% %% \infer{\wftype{\phi}{\F{\alpha}{\Delta}}}
%% %%       {\wfctx{\psi}{\Delta} &
%% %%         \hmor{\psi}{\phi}{\alpha}
%% %%       }
%% %% \qquad
%% %% \F{\alpha}{\Delta}[\theta] := \F{\alpha[\theta/\psi]}{\Delta}
%% %% \]
%% %% Axiomatic FR:
%% %% \[
%% %% \infer{\seqh{\psi}{\Delta}{\alpha}{\phi}{\F{\alpha}{\Delta}}}{}
%% %% \]

%% \[
%% \infer{\wftype{\psi_0,\phi}{\F{\alpha}{\Delta}}}
%%       {\wfctx{\psi_0,\psi}{\Delta} &
%%         \hmor{\psi}{\phi}{\alpha}
%%       }
%% \qquad
%% \F{\alpha}{\Delta}[\theta] := \F{\alpha[\theta\mid_\psi]}{\Delta[\theta\mid_{\psi_0}]}
%% \]

%% \[
%% \infer[\FR^*]{\seqh{\psi_0,\psi}{\Delta}{1_{\psi_0},\alpha}{\psi_0,\phi}{\F{\alpha}{\Delta}}}{}
%% \]

%% \[
%% \infer[\FL]
%%       {\seqh{\psi_1,(\psi_0,\phi)}{\Gamma,x:\F{\alpha}{\Delta}}{\beta}{\phi_1}{C}}
%%       {\seqh{\psi_1,(\psi_0,\psi)}{\Gamma,\Delta}{[\alpha/\psi]\beta}{\phi_1}{C}}
%% \]

\end{document}


\documentclass{article}
\usepackage{amsthm,hyperref,mathtools,mathpartir,cleveref,mathrsfs,amssymb,url,paralist}
\newtheorem{thm}{Theorem}%[section]
\newtheorem{lem}[thm]{Lemma}
\newtheorem{conj}[thm]{Conjecture}
\theoremstyle{definition}
\newtheorem{defn}[thm]{Definition}
\newtheorem{eg}[thm]{Example}
\theoremstyle{remark}
\newtheorem{rmk}[thm]{Remark}
\usepackage{tikz}
\usepackage{tikz-cd}
\let\sto\looparrowright
\def\M{\mathcal{M}}
\def\N{\mathcal{N}}
\def\C{\mathcal{C}}
\def\K{\mathcal{K}}
\def\Q{\mathcal{Q}}
\def\D{\mathcal{D}}
\def\Cat{\mathcal{C}\mathit{at}}
\def\Fib{\mathcal{F}\mathit{ib}}
\def\Prof{\mathcal{P}\mathit{rof}}
\def\dom{\mathrm{dom}}
\def\cod{\mathrm{cod}}
\def\id{\mathrm{id}}
\def\side#1{{\scriptstyle(#1)}}
\def\sid{\side{\id}}
\def\twocell#1#2#3#4{\inferrule*[Left={$\side{#1}$},Right={$\side{#4}$}]{#2}{#3}}
\def\twocelll#1#2#3#4{\inferrule*[left={$\side{#1}$},Right={$\side{#4}$}]{#2}{#3}}
\def\twocellr#1#2#3#4{\inferrule*[Left={$\side{#1}$},right={$\side{#4}$}]{#2}{#3}}
\def\twocelllr#1#2#3#4{\inferrule*[left={$\side{#1}$},right={$\side{#4}$}]{#2}{#3}}
\let\ot\leftarrow
\let\xto\xrightarrow
\let\xot\xleftarrow
\let\tot\leftrightarrow
\def\o{^{\circ}}
\def\type{\;\mathsf{type}}
\let\types\vdash
\tikzset{horiz/.style={"\mathclap\bullet" description}}
\title{Simple type theory over virtual double categories}
\begin{document}
\maketitle

A \textbf{virtual double category}~\cite{cs:multicats} (also called an ``fc-multicategory''~\cite{leinster:higher-opds,leinster:fc-multicategories}) is a Leinster generalized multicategory for the free-category monad on directed graphs.
It has a category of objects and vertical arrows, a directed graph of horizontal arrows with the same objects, and a collection of 2-cells that look like this:
\[
\begin{tikzcd}
  A_0 \ar[r,horiz] \ar[d] \ar[drrr,phantom,"\Downarrow"] & A_1 \ar[r,horiz] & \cdots \ar[r,horiz] & A_n \ar[d] \\
  B_0 \ar[rrr,horiz] &&& B_n
\end{tikzcd}
\]
The 2-cells can be composed in unsurprising ways:
\[\hspace{-2cm}
\begin{tikzcd}
  \cdot \ar[r,horiz] \ar[d] \ar[drrr,phantom,"\Downarrow"] & \cdot \ar[r,horiz] & \cdots \ar[r,horiz] &
  \cdot \ar[r,horiz] \ar[d] \ar[drrr,phantom,"\Downarrow"] & \cdot \ar[r,horiz] & \cdots \ar[r,horiz] &
  \cdot \ar[r,phantom,"\cdots"] \ar[d] &
  \cdot \ar[r,horiz] \ar[d] \ar[drrr,phantom,"\Downarrow"] & \cdot \ar[r,horiz] & \cdots \ar[r,horiz] &
  \cdot \ar[d]\\
  \cdot \ar[rrr,horiz] \ar[d] \ar[drrrrrrrrrr,phantom,"\Downarrow"] &&& \cdot \ar[rrr,horiz] &&& \cdot \ar[r,phantom,"\cdots"] & \cdot \ar[rrr,horiz] &&& \cdot \ar[d]\\
  \cdot \ar[rrrrrrrrrr,horiz] &&&&&&&&&& \cdot
\end{tikzcd}\hspace{-2cm}
\]
The vertical arrows on the left and right get composed in the category of vertical arrows.
This is associative in a straightforward sense, and unital with respect to identity 2-cells of horizontal arrows, which have identity vertical arrows on their boundary:
\[
\begin{tikzcd}
  A \ar[r,horiz,"m"] \ar[d,equals] \ar[dr,phantom,"="] & B \ar[d,equals] \\ A \ar[r,horiz,"m"'] & B
\end{tikzcd}
\]
These are ``unary'' 2-cells where $n=1$; note that there also exist ``nullary'' 2-cells where $n=0$:
\[
\begin{tikzcd}[column sep=small]
  & A \ar[dl] \ar[dr] \ar[d,phantom,"\Downarrow"] \\ B \ar[rr,horiz] & ~ & C
\end{tikzcd}
\]
If $\M$ is a virtual double category, we write $\M_0$ for the category of objects and vertical arrows, and $\M_1$ for the category whose objects are horizontal arrows and whose morphisms are unary 2-cells.

A \textbf{functor} between virtual double categories is the only possible thing: it takes objects, both kinds of arrow, and 2-cells to objects, arrows, and 2-cells, preserving all identities and compositions.
We similarly have \textbf{transformations}, with a vertical component for every object and a unary 2-cell component for every horizontal arrow, forming a 2-category of virtual double categories.

\begin{defn}
  A \textbf{strict cocartesian virtual double category} is a strict cocartesian monoid in the 2-category of virtual double categories.
  That is, it has left adjoints to the functors $\M \to \M\times\M$ and $\M\to 1$ such that the induced pseudo-monoid structure on $\M$ is in fact strict.
  A \textbf{strict cocartesian functor} between such virtual double categories is one that strictly respects these adjunctions.
\end{defn}

Concretely, a virtual double category is strict cocartesian if:
\begin{itemize}
\item The vertical category $\M_0$ is strict cocartesian, i.e. has specified finite coproducts inducing a strict monoidal structure,
\item The category $\M_1$ is also strict cocartesian,
\item The source and target functors $\M_1 \rightrightarrows \M_0$ are strict cocartesian, and
\item The codiagonals and counits of horizontal arrows commute with arbitrary 2-cells.
\end{itemize}
We write monoidal products as lists; thus the third condition above implies that the product of horizontal arrows $A\to B$ and $C\to D$ has the form $(A,C)\to (B,D)$, and so on.
The cocartesian structure gives us a specified vertical arrow $(A_{\sigma 1},\dots, A_{\sigma m}) \to (A_1,\dots, A_n)$ for any function $\sigma:\mathbf{m}\to \mathbf{n}$ and any objects $(A_1,\dots, A_n)$, and similarly specified unary 2-cells for lists of horizontal arrows.
We will refer to these as \textbf{generalized codiagonals}.

\begin{defn}\label{defn:difib}
  Let $\M$ be a strict cocartesian virtual double category, with a specified class of 2-cells that we call \textbf{structural}.
  A strict cocartesian functor $\pi:\D\to\M$ of virtual double categories is a \textbf{cocartesian local discrete difibration} if:
  \begin{enumerate}
  \item The induced functor $\D_0 \to \M_0$ on vertical categories is a discrete fibration.
    That is, for any object $\Gamma\in \D$ and vertical arrow $\sigma:\Delta\to \pi A$ in $\M$, there exists a unique vertical arrow $\overline\sigma_A : \sigma^*\Gamma \to \Gamma$ with target $\Gamma$ such that $\pi(\overline\sigma_A)=\sigma$.\label{item:difib1}
  \item The pullback of $\D_1\to\M_1$ to the subcategory of generalized codiagonals in $\M_1$ is a discrete fibration.\label{item:difib2}
  \item For any structural 2-cell in $\M$:
    \[
    \begin{tikzcd}
      \Delta_0 \ar[r,horiz,"\alpha_1"] \ar[d,"\sigma_0"'] \ar[drrr,phantom,"\Downarrow e"] & \Delta_1 \ar[r,horiz,"\alpha_2"] & \cdots \ar[r,horiz,"\alpha_n"] & \Delta_n \ar[d,"\sigma_n"] \\
      \Psi_0 \ar[rrr,horiz,"\beta"'] &&& \Psi_n
    \end{tikzcd}
    \]
    and objects and arrows in $\D$ lying over all of its boundary except the horizontal arrow target:
    \[
    \begin{tikzcd}
      \sigma_0^*\Gamma_0 \ar[r,horiz,"f_1"] \ar[d,"\overline\sigma_0"'] & \Gamma_1 \ar[r,horiz,"f_2"] & \cdots \ar[r,horiz,"f_n"] & \sigma_n^*\Gamma_n \ar[d,"\overline\sigma_n"] \\
      \Gamma_0  &&& \Gamma_n
    \end{tikzcd}
    \]
    (that is, $\pi(\overline\sigma_i)=\sigma_i$ and $\pi(f_i)=\alpha_i$) there exists a unique 2-cell
    \[
    \begin{tikzcd}
      \sigma_0^*\Gamma_0 \ar[r,horiz,"f_1"] \ar[d,"\overline\sigma_0"'] \ar[drrr,phantom,"\Downarrow \overline e"] & \Gamma_1 \ar[r,horiz,"f_2"] & \cdots \ar[r,horiz,"f_n"] & \sigma_n^*\Gamma_n \ar[d,"\overline\sigma_n"] \\
      \Gamma_0 \ar[rrr,horiz,"{e_!(f_1,\dots,f_n)}"'] &&& \Gamma_n
    \end{tikzcd}
    \]
    with $\pi(\overline e)=e$ (hence also $\pi(e_!(f_1,\dots,f_n))=\beta$).\label{item:difib3}
  \end{enumerate}
\end{defn}

I'm not sure exactly what to call these; ``difibration'' is intended to suggest ``both a fibration and an opfibration'' but in a different way from a ``bifibration''.

\begin{defn}
  A \textbf{semantic mode theory} is a strict cocartesian virtual double category $\M$ with a specified class of structural 2-cells.
  A \textbf{semantic type theory} over such an $\M$ is a cocartesian local discrete difibration $\pi:\D\to\M$.
\end{defn}

When we start adding $F$ and $U$ functors, we may want to also equip $\M$ with a specified class of horizontal arrows paired with an ``input or output slot''. 

In general, the vertical category of the semantic mode theory will be the free cocartesian strict monoidal category on the set of modes.
That means its objects will be finite lists of modes $(p_1,\dots,p_k)$ and its morphisms $(p_1,\dots,p_k) \to (q_1,\dots,q_m)$ are functions $\sigma:\mathbf{k}\to \mathbf{m}$ such that $q_{\sigma(i)} = p_i$ for all $i\in\mathbf{k}$.
In other words, $\sigma$ exhibits its codomain as obtained from its domain by exchange, weakening, and contraction.

The full syntactic mode theory will be specified relative to some generating modes, some generating horizontal arrows (mode morphisms), some generating structural 2-cells, and some axioms on the composites of those 2-cells.
The corresponding semantic mode theory should be the free semantic mode theory presented by these data.
In particular, note that we do not include any ``generating vertical arrows''.

Syntactically, each structural mode 2-cell can be thought of as a ``rule scheme''
\[\twocell{\sigma_0:\Delta_0 \to \Psi_0}
{\Delta_0 \types \alpha_1:\Delta_1 \\ \Delta_1 \types \alpha_2:\Delta_2 \\ \cdots \\ \Delta_{n-1}\types \alpha_n :\Delta_n}
{\Psi_0 \types \beta:\Psi_1}
{\sigma_n:\Delta_n\to\Psi_n}. \]
We have to be careful with this notation because the term $\beta$ in the conclusion is \emph{not} a term describing how this rule produces $\Psi_0\types \Psi_n$ from the $\alpha_i$'s.
Instead $\beta$ is a horizontal arrow that's given just like the $\alpha_i$'s.
If we want to add a term that describes the rule, it would be better to put it outside somehow like
\begin{equation}
  \twocell{\sigma_0:\Delta_0 \to \Psi_0}
  {\Delta_0 \types \alpha_1:\Delta_1 \\ \Delta_1 \types \alpha_2:\Delta_2 \\ \cdots \\ \Delta_{n-1}\types \alpha_n :\Delta_n}
  {e:(\Psi_0 \types \beta:\Psi_1)}
  {\sigma_n:\Delta_n\to\Psi_n}.\label{eq:mode-2cell}
\end{equation}
The composites of 2-cells can then be written as derivations in a usual way, although we have to be careful that the vertical arrow parts match up (see e.g.~\eqref{eq:sym-mult-ax} below).

We may also want a separate syntactic system for generating the horizontal arrows in terms of basic ones.
This is a little weird because in a general virtual double category there are no ``operations'' on the horizontal arrows at all.
In our situation we do have a monoidal product of horizontal arrows, but we might want to be able to generate horizontal arrows from less than this.
For instance, in the examples below I put in $n$-ary mode morphisms for all $n$ manually, whereas in the multicategorical case we were able to generate these from the binary and nullary version with associativity axioms.

The syntactic type theory will be specified relative to some generating types (each with an assigned mode), some generating judgments (each with an assigned mode morphism), and some generating 2-cells (each with an assigned mode 2-cell).
The corresponding semantic type theory will be the free one generated by these data.

Its vertical category will be the free cocartesian strict monoidal category on the types.
The fibration condition on vertical categories in $\pi:\D\to\M$ will then hold automatically; it says that given a context of types $\Gamma$ with corresponding mode context $\Psi$, and a function $\sigma:\Delta\to\Psi$ exhibiting $\Psi$ as obtained by exchange, contraction, and weakening from $\Delta$, there is a unique context $\sigma^*\Gamma$ of types with mode context $\Delta$ such that $\Gamma$ is obtained from $\sigma^*\Gamma$ by the same structural rules described by $\sigma$.
For instance, if $\sigma : (p,q,q) \to (p,q,r)$ is the only possible morphism, and $\Gamma = (A_p,B_q,C_r)$, then $\sigma^*\Gamma = (A_p,B_q,B_q)$.

Note that this operation on type \emph{contexts} exists regardless of whether the theory ``has structural rules''.
The existence or nonexistence of mode 2-cells with $\sigma$ on the boundary is what specifies whether judgments with domain or codomain $\sigma^*\Gamma$ can be exchanged/contracted/weakened to have $\Gamma$ instead.

As usual, the mode morphism parametrizing a judgment indicates how the types in the domain and codomain can be used.
When the mode morphism is a monoidal product, then we think of it as parametrizing formal lists of judgments.
Condition~\ref{item:difib2} in \cref{defn:difib} says that we can also duplicate and omit elements of such lists.

Condition~\ref{item:difib3} in \cref{defn:difib} says that each structural mode 2-cell induces a structural rule on judgments.
That is, given a structural mode 2-cell $e$ as in~\eqref{eq:mode-2cell}, we have an induced rule in the type theory:
\[
\inferrule{\sigma_0^*\Gamma_0 \types_{\alpha_1} \Gamma_1 \\ \Gamma_1\types_{\alpha_2} \Gamma_2 \\ \cdots \\\Gamma_{n-1} \types_{\alpha_n} \sigma_n^*\Gamma_n}{\Gamma_0 \types_\beta \Gamma_n}
\]
When $n=2$, and the $\sigma$'s are identities, this specializes to a cut-like rule:
\[ \inferrule{\Gamma_0 \types_{\alpha_1} \Gamma_1 \\ \Gamma_1 \types_{\alpha_2} \Gamma_2}{\Gamma_0\types_\beta \Gamma_2}. \]
When $n=1$, it looks like
\[\inferrule{\sigma_0^*\Gamma_0 \types_{\alpha} \sigma_1^*\Gamma_1}{\Gamma_0 \types_\beta \Gamma_1}
\]
which will include structural operations such as exchange, contraction, weakening for appropriately chosen $\sigma$'s.
When the $\sigma$'s are identities here, we find things like the action of the units and counits of mode adjunctions.
(We restrict these rules to the subclass of ``structural'' mode 2-cells so that we can exclude the generalized codiagonals from producing rules in this way, which wouldn't make sense.)
Finally, the case $n=0$ looks like
\[ \inferrule{\sigma_0^*\Gamma_0 = \sigma_1^*\Gamma_1}{\Gamma_0 \types_\beta \Gamma_1}\]
which is an identity-like operation: it says that we get $\Gamma_0 \types_\beta \Gamma_1$ if we can identify some specified part of $\Gamma_0 $ with some specified part of $\Gamma_1$ (for instance, if they share a type in common).
The other operation on judgments in the type theory is the strict monoidal structure, which in the binary and nullary cases looks like
\begin{mathpar}
\inferrule{\Gamma_0 \types_{\alpha} \Gamma_1 \\ \Delta_0 \types_{\beta} \Delta_1}
{(\Gamma_0,\Delta_0) \types_{(\alpha,\beta)} (\Gamma_1,\Delta_1)}
\and
\inferrule{ }{() \types_{()} ()}
\end{mathpar}
This isn't usually present in the type theories we are trying to model, but we should be able to make it disappear in the adequacy theorems.
Hopefully that won't be too bad; I think it's the price to pay for having a single framework that can handle both single-conclusion and multiple-conclusion sequents.

\begin{eg}
  Let the mode theory have one mode $p$, one generating $n$-ary morphism $p^n \types \alpha_n : p$ for each $n$, and the following generating 2-cells:
  \begin{mathpar}
    \twocell{\id}{ }{p \types \alpha_1 : p}{\id}
    \and
    \twocell{\id}{p^{k_1+\cdots+k_n} \types (\alpha_{k_1},\dots,\alpha_{k_n}) : p^n  \\
      p^n \types \alpha_n :p }
    {p^{k_1+\cdots+k_n} \types \alpha_{k_1+\cdots+k_n} : p}{\id}
  \end{mathpar}
  plus some associativity and unit axioms for the composites of these 2-cells.
  The resulting type theory should be ordered logic.
  The nullary 2-cell gives an identity rule:
  \[ \inferrule{ }{A \types_{\alpha_1} A} \]
  and the binary one, when combined with the monoidal structure, gives a multi-cut rule:
  \[ \inferrule{
    \inferrule*{\Gamma_1\types_{\alpha_{k_1}} A_1 \\ \cdots \\ \Gamma_n \types_{\alpha_{k_n}} A_n}
    {(\Gamma_1,\dots,\Gamma_n) \types_{(\alpha_{k_1},\dots,\alpha_{k_n})} (A_1,\dots,A_n)} \\
    (A_1,\dots,A_n) \types_{\alpha_n} \Psi}
  {(\Gamma_1,\dots,\Gamma_n) \types_{\alpha_{k_1+\cdots+k_n}} \Psi}
  \]
  A semantic type theory for this example is not exactly a multicategory.
  It has an underlying multicategory whose $n$-ary morphisms are those lying above $\alpha_n$, and whose composition is defined by going through the monoidal structure as above.
  But in general there might be morphisms with non-unary domain that aren't monoidal products of $n$ morphisms with unary domain.
  However, I believe that the forgetful functor from these objects to multicategories is a coreflection, i.e.\ it has a fully faithful left adjoint, and that therefore the \emph{free} structures (generated by the syntactic type theory) will in fact be just multicategories.
  This is of course a semantic version of the adequacy theorem.
\end{eg}

\begin{eg}
  In the previous example, instead of the 2-cell generating multi-cut, we can have a 2-cell generating a one-formula cut:
  \[\inferrule{p^{m+k+n} \types (\alpha_1^m,\alpha_k,\alpha_1^n) : p^{m+1+n} \\ p^{m+1+n} \types \alpha_{m+1+n} : p}
  {p^{m+k+n} \types \alpha_{m+k+n} : p}
  \]
  with appropriate axioms (essentially, the axioms of a multicategory presented using $\circ_i$-operations rather than multi-composition).
  The resulting rule in the type theory can be combined with identities and the monoidal structure to get the usual one-formula cut rule:
  \[
  \inferrule{
    \inferrule*{
      \inferrule*{ }{\Gamma\types_{\alpha_1^m} \Gamma}\\
      \Psi \types_{\alpha_k} A \\
      \inferrule*{ }{\Delta\types_{\alpha_1^n} \Delta}
    }{(\Gamma,\Psi,\Delta) \types_{(\alpha_1^m,\alpha_k,\alpha_1^n)} (\Gamma,A,\Delta)}\\
    (\Gamma,A,\Delta) \types_{\alpha_{m+1+n}} B}
  {(\Gamma,\Psi,\Delta) \types_{\alpha_{m+k+n}} B}
  \]
\end{eg}

\begin{eg}
  Now suppose we add a structural 2-cell for any permutation $\sigma$:
  \[\twocell{\sigma}{p^n \types \alpha_n : p}{p^n \types \alpha_n : p}{\id}
  \]
  with axioms making them functorial under composition of permutations.
  This gives rise to a general exchange rule:
  \[ \inferrule{\sigma^*\Gamma \types_{\alpha_n} A}{\Gamma \types_{\alpha_n} A} \]
  Alternatively, we could add only the binary exchange 2-cells
  \[\twocell{\id^m,\tau,\id^n}{p^{m+2+n} \types \alpha_{m+2+n} :p}{p^{m+2+n} \types \alpha_{m+2+n} :p}{\id}
  \]
  where $\tau : (p,p) \to (p,p)$ is the transposition, with axioms coming from the presentation of symmetric groups using transpositions.
  This would yield two-formula exchange rules
  \[ \inferrule{(\Gamma,A,B,\Delta) \types_{\alpha_{m+2+n}} C}{(\Gamma,B,A,\Delta) \types_{\alpha_{m+2+n}} C} \]
  We should also impose equivariance axioms with respect to the cut 2-cells, mimicking the axioms of a symmetric multicategory.
  For instance, if we permute and then postcompose, that should be the same as postcomposing and then permuting.
  That is, the composite
  \[
  \twocell{\sigma_1\oplus\cdots\oplus\sigma_n}
  {\twocell{\id}{p^{k_1+\cdots+k_n} \types (\alpha_{k_1},\dots,\alpha_{k_n}) : p^n  \\
      p^n \types \alpha_n :p }
    {p^{k_1+\cdots+k_n} \types \alpha_{k_1+\cdots+k_n} : p}{\id}}
  {p^{k_1+\cdots+k_n} \types \alpha_{k_1+\cdots+k_n} : p}{\id}
  \]
  should equal
  \begin{equation}
    \twocell{\id}{
      \twocell{\sigma_1\oplus\cdots\oplus\sigma_n}
      {p^{k_1+\cdots+k_n} \types (\alpha_{k_1},\dots,\alpha_{k_n}) : p^n}
      {p^{k_1+\cdots+k_n} \types (\alpha_{k_1},\dots,\alpha_{k_n}) : p^n}{\id}  \\
      p^n \types \alpha_n :p }
    {p^{k_1+\cdots+k_n} \types \alpha_{k_1+\cdots+k_n} : p}{\id}\label{eq:sym-mult-ax-0}
  \end{equation}
  And if we permute and then precompose with permuted inputs, that should be the same as precomposing and then permuting.
  That is,
  \begin{equation}
    {\twocell{\id}{
        \twocellr{\sigma(k_1,\dots,k_n)}
        {p^{k_1+\cdots+k_n} \types (\alpha_{k_1},\dots,\alpha_{k_n}) : p^n}
        {p^{k_1+\cdots+k_n} \types (\alpha_{k_1},\dots,\alpha_{k_n}) : p^n}{\sigma}
        \\
        \twocelll{\sigma}{p^n \types \alpha_n :p}{p^n \types \alpha_n :p}{\id} }
      {p^{k_1+\cdots+k_n} \types \alpha_{k_1+\cdots+k_n} : p}{\id}}\label{eq:sym-mult-ax}
  \end{equation}
  should equal
  \begin{equation}
    \twocell{\sigma(k_1,\dots,k_n)}{\twocell{\id}{p^{k_1+\cdots+k_n} \types (\alpha_{k_1},\dots,\alpha_{k_n}) : p^n  \\
        p^n \types \alpha_n :p }
      {p^{k_1+\cdots+k_n} \types \alpha_{k_1+\cdots+k_n} : p}{\id}}{p^{k_1+\cdots+k_n} \types \alpha_{k_1+\cdots+k_n} : p}{\id}\label{eq:sym-mult-ax-2}
  \end{equation}
  The composite in~\eqref{eq:sym-mult-ax} illustrates something to be aware of when using derivation syntax for 2-cell composites, namely that the vertical arrows have to match up in the middle: in this case the two $\sigma$'s.
  (Similarly, in~\eqref{eq:sym-mult-ax-0} we ought formally to include an identity 2-cell in the upper right.)
  Also, the upper-left 2-cell in~\eqref{eq:sym-mult-ax} is \emph{not} one of the structural exchange 2-cells, but rather a generalized codiagonal.
  With these axioms imposed, the semantic type theories should coreflect into symmetric multicategories.
\end{eg}

\begin{eg}
  If we generalize this to arbitrary functions $\sigma:\mathbf{m}\to \mathbf{n}$, we obtain contraction and weakening rules, and cartesian multicategories.
  We can do this unbiasedly:
  \[\twocell{\sigma}{p^m \types \alpha_n : p}{p^n \types \alpha_n : p}{\id}
  \]
  or biasedly, with only binary and nullary operations:
  \begin{mathpar}
    \twocelll{\id^m,\mu,\id^n}{p^{m+2+n} \types \alpha_{m+2+n} :p}{p^{m+1+n} \types \alpha_{m+1+n} :p}{\id}
    \and
    \twocelll{\id^m,\eta,\id^n}{p^{m+n} \types \alpha_{m+n} :p}{p^{m+1+n} \types \alpha_{m+1+n} :p}{\id}
  \end{mathpar}
  The axioms are essentially the same.
  Note in particular that the analogue of~\eqref{eq:sym-mult-ax} is
  \begin{equation*}
    {\twocell{\id}{
        \twocellr{\sigma(k_1,\dots,k_n)}
        {p^{k_{\sigma 1}+\cdots+k_{\sigma m}} \types (\alpha_{k_{\sigma 1}},\dots,\alpha_{k_{\sigma m}}) : p^m}
        {p^{k_1+\cdots+k_n} \types (\alpha_{k_1},\dots,\alpha_{k_n}) : p^n}{\sigma}
        \\
        \twocelll{\sigma}{p^m \types \alpha_n :p}{p^n \types \alpha_n :p}{\id} }
      {p^{k_1+\cdots+k_n} \types \alpha_{k_1+\cdots+k_n} : p}{\id}}
  \end{equation*}
  In the resulting rule it is important that we transport \emph{backwards} along generalized codiagonals but \emph{forwards} along structural 2-cells, so that in the corresponding analogue of~\eqref{eq:sym-mult-ax-2}
  \begin{equation*}
    \twocell{\sigma(k_1,\dots,k_n)}{\twocell{\id}{p^{k_{\sigma 1}+\cdots+k_{\sigma m}} \types (\alpha_{k_{\sigma 1}},\dots,\alpha_{k_{\sigma m}}) : p^m  \\
        p^m \types \alpha_n :p }
      {p^{k_{\sigma 1}+\cdots+k_{\sigma m}} \types \alpha_{k_{\sigma 1}+\cdots+k_{\sigma m}} : p}{\id}}{p^{k_1+\cdots+k_n} \types \alpha_{k_1+\cdots+k_n} : p}{\id}
  \end{equation*}
  the first composition is with a list of morphisms that contains duplicates and omissions, i.e.\ is contraction/weaken-able.
\end{eg}

\begin{eg}
  More generally, any (non-cartesian, non-symmetric) 2-multicategory $\M$ can be regarded as a semantic mode theory.
  Its vertical category is the free cocartesian strict monoidal category on the objects of $\M$, its horizontal arrows are finite lists of 1-morphisms of $\M$, and its 2-cells are lists of 2-cells in $\M$ whose domain is decomposed as a composite.
  We take the vertical-arrow source and target of such 2-cells to always be identities.
  In particular, any composable string of horizontal arrows in such a mode theory is the domain of at least one 2-cell, namely the identity in $\M$.

  Thus the semantic type theories have multicategorical structure over that of $\M$, together with an action by the 2-cells of $\M$; so they should coreflect into local discrete opfibrations of 2-multicategories.
  So we obtain examples like two ordered logics related by a functor or an adjunction, ordered cohesive logic, and so on.
  Then we can add additional 2-cells for the desired structural rules at each mode, with nonidentity vertical arrow boundary, and axioms commuting them with the other 2-cells.
\end{eg}

\begin{eg}
  Now suppose we have one mode $p$, and take the mode morphisms $p^m \types p^n$ to be cospans $\mathbf{m} \to \mathbf{k}\ot \mathbf{n}$ of order-preserving maps, with monoidal structure given by disjoint union.
  There is always a unique such morphism with $k=1$, which parametrizes the usual sort of judgment $\Gamma\types\Delta$ of classical linear logic; let us write this morphism as $\alpha_{m,n}$.
  The mode morphisms for $k\neq 1$ are formal monoidal products (lists) of these.
  The unbiased exchange rule
  \[\inferrule{\sigma^* \Gamma \types_{\alpha_{m,n}} \tau^* \Delta}{\Gamma\types_{\alpha_{m,n}}\Delta} \]
  comes from a family of 2-cells
  \[ \twocell{\sigma}{p^m \types \alpha_{m,n}: p^n}{p^m \types \alpha_{m,n}: p^n}{\tau} \]
  for any permutations $\sigma$ and $\tau$ of $\mathbf{m}$ and $\mathbf{n}$ respectively.
  The linear logic one-formula cut rule can be generated by 2-cells of the following form:
  \[ \twocell{\id}
  {p^{k+m} \types (\alpha_{1,1}^k,\alpha_{m,1+q}) : p^{k+1+q} \\ p^{k+1+q} \types (\alpha_{k+1,n},\alpha_{1,1}^q) : p^{n+q}}
  {p^{k+m} \types \alpha_{k+m,n+q} : p^{n+q}}
  {\id}
  \]
  % where the cospans are
  % \begin{mathpar}
  %   \mathbf{k+m} \xto{\alpha} \mathbf{k+1} \xot{\beta} \mathbf{k+1+q}
  %   \and
  %   \mathbf{k+1+q} \xto{\gamma} \mathbf{1+q} \xot{\delta} \mathbf{n+q}
  % \end{mathpar}
  % with $\alpha$ and $\beta$ being the identity on $\mathbf{k}$, and $\gamma$ and $\delta$ the identity on $\mathbf{q}$.
  % 
  % \[ \twocell{\id}{p^m \types (\alpha,\beta) : p^q \\ p^q \types (\gamma,\delta) : p^n}{p^m \types (\phi,\psi) : p^n}{\id} \]
  % where in the cospans $\mathbf{m} \xto{\alpha} \mathbf{k}\xot{\beta} \mathbf{q}$ and $\mathbf{q} \xto{\gamma} \mathbf{l}\xot{\delta} \mathbf{n}$ there are specified elements $i_0\in \mathbf{k}$ and $j_0\in\mathbf{l}$ such that
  % \begin{itemize}
  % \item If $i\neq i_0$ then $\alpha^{-1}(i)$ and $\beta^{-1}(i)$ are singletons,
  % \item If $j\neq j_0$ then $\gamma^{-1}(j)$ and $\delta^{-1}(j)$ are singletons, and
  % \item $\beta^{-1}(i_0) \cap \gamma^{-1}(j_0)$ is a singleton.
  % \end{itemize}
  % The conclusion cospan is then the pushout
  % \[
  % \begin{tikzcd}[row sep=small,column sep=small]
  %   \mathbf{k+m} \ar[dr,"\alpha"] \ar[ddrr,bend right,"\phi"'] &&
  %   \mathbf{k+1+q} \ar[dl,"\beta"'] \ar[dr,"\gamma"] &&
  %   \mathbf{n+q} \ar[dl,"\delta"'] \ar[ddll,bend left,"\psi"] \\
  %   & \mathbf{k+1} \ar[dr] && \mathbf{1+q} \ar[dl] \\
  %   && \bullet
  % \end{tikzcd}
  % \]
  The resulting rule can be combined with identities and the monoidal structure to give the usual linear logic cut rule:
  \[\inferrule*{
    \inferrule*{\inferrule*{ }{\Psi\types_{\alpha_{1,1}^k} \Psi} \\ \Gamma\types_{\alpha_{m,1+q}} A,\Phi}
    {\Psi,\Gamma \types_{\alpha_{1,1}^k,\alpha_{m,1+q}} \Psi,A,\Phi} \\
    \inferrule*{\Psi,A\types_{\alpha_{k+1,n}} \Delta \\ \inferrule*{ }{\Phi \types_{\alpha_{1,1}^q} \Phi}}
    {\Psi,A,\Phi \types_{\alpha_{k+1,n},\alpha_{1,1}^q} \Delta,\Phi}}
  {\Psi,\Gamma \types_{\alpha_{k+m,n+q}} \Delta,\Phi}
  \]
  We should impose on these 2-cells associativity and equivariance axioms corresponding to those of a polycategory.
  The semantic type theories should then coreflect into polycategories.

  If we add also contraction and weakening, then it's not clear what axioms should be imposed, but at least we should get a type theory for derivability in classical nonlinear logic.
\end{eg}

\begin{eg}
  If we keep the same mode morphisms and exchange rules as in the previous example, but replace the cut 2-cell by
  \[ \twocell{\id}
  {p^{m} \types \alpha_{m,k} : p^{k} \\ p^{k} \types \alpha_{k,n} : p^{n}}
  {p^{k} \types \alpha_{k,n} : p^{n}}
  {\id}
  \]
  and also add structural 2-cells that ``internalize the formal monoidal structure''
  \begin{mathpar}
    \twocellr{\id}{p^{m+q} \types (\alpha_{m,k},\alpha_{q,n}) : p^{k+n}}{p^{m+q} \types \alpha_{m+q,k+n} : p^{k+n}}{\id}
    \and
    \twocelll{\id}{p^0 \types () : p^0}{p^0 \types \alpha_{0,0} : p^0}{\id}
  \end{mathpar}
  with appropriate axioms, then we should get a type theory for props.
\end{eg}


\bibliography{../all.bib}
\bibliographystyle{alpha}

\end{document}
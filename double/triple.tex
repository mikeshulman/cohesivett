\documentclass{article}
\usepackage{amsthm,mathtools,mathpartir,cleveref}
\theoremstyle{definition}
\newtheorem{defn}{Definition}
\newtheorem{eg}{Example}
\usepackage{tikz}
\usepackage{tikz-cd}
\def\M{\mathcal{M}}
\def\K{\mathcal{K}}
\def\Q{\mathcal{Q}}
\def\SQ{\mathcal{S}\mathit{q}}
\def\Cat{\mathcal{C}\mathit{at}}
\def\D{\mathcal{D}}
\def\DD#1#2{\mathcal{D}^{#1}_{#2}}
\let\ot\leftarrow
\let\xto\xrightarrow
\let\xot\xleftarrow
\let\tot\leftrightarrow
\def\o{^{\circ}}
\def\twocell#1#2#3{\ar[from=#1,to=#2,phantom,""{name=1,near start},""{name=2,near end}]\ar[Rightarrow,from=1,to=2,"#3"]}
\def\drtwocell{\twocell{r}{d}}
\def\drrtwocell{\twocell{rr}{d}}
\title{Type theory over cartesian double and triple multicategories}
\author{DL, PS, MS, MR}
\begin{document}
\maketitle

\section{Definitions}
\label{sec:definitions}

In my head, the basic object parametrizing an intuitionistic type theory is a polynomial monad on $\mathbf{Cat}$, determining a notion of generalized multicategory.
I think of a 2-multicategory as a simpler approximation to this.
Similarly, the basic object parametrizing a ``classical'' type theory is a ``polynomial loose distributive law'' between two such polynomial monads, determining a notion of generalized polycategory.
By looking for a similar simpler approximation to that, I arrived at the following.

\begin{defn}
  A \textbf{(cartesian) multi-double-category} consists of:
  \begin{itemize}
  \item Two (cartesian) multicategories with the same set of objects, whose arrows are called vertical and horizontal.
  \item For horizontal arrows $\beta:(r_1,\dots,r_m)\to s$ and $\alpha_i:(p_{i1},\dots,p_{im_i})\to q_i$ and vertical arrows $\delta:(q_1,\dots,q_n) \to s$ and $\gamma_j:(p'_{j1},\dots,p'_{j n_j})\to r_j$ and permutations $(p_{11},\dots,p_{nm_n})\cong (p'_{11},\dots,p'_{mn_m})$, a set of squares
    \[\begin{tikzcd}
      \vec{p}\cong\vec{p'} \ar[r,"\vec{\alpha}"] \ar[d,swap,"\vec{\gamma}"]\drtwocell{e}
      & \vec{q} \ar[d,"\delta"]
      \\ \vec{r} \ar[r,swap,"\beta"] & s
    \end{tikzcd}\]
  \item (In the cartesian case) cartesian actions on the squares, in both directions, that respect the cartesian actions on the arrows, and probably commute with each other in an appropriate sense.
  \item Squares compose horizontally and vertically like (cartesian) multicategories, and the compositions interchange.
    In particular, vertical arrows have horizontal identity squares that are closed under vertical composition, and vice versa.
  \end{itemize}
\end{defn}

Note that unlike other definitions of ``double multicategory'', this one is multi in \emph{both} directions.
We will be using ``double-ness'' in a different way, to separate the domains and codomains of sequents.
It's reasonable to expect that for syntax we can deal only with the cartesian case and ``adequacy away'' all the undesired junk if it's there, but for defining semantic examples we need to actually omit the cartesian structure when we don't want it.

\begin{eg}\label{eg:quintet}
  Any (cartesian) 2-multicategory $\M$ should give rise to a (cartesian) multi-double-category $\Q(\K)$ whose vertical and horizontal arrows are the same, and whose squares are 2-cells from the top-right composite to the left-bottom composite.
  This is a multicategorical version of the classical ``quintet'' construction of a double category from a 2-category, although in the multicategorical case there are no longer always five things involved (originally it was the four arrows and the 2-cell that determine a square in the double category).
  For instance, $\Q(\Cat)$ has as objects categories, as arrows multi-variable functors (in both directions), and as squares natural transformations.
\end{eg}

Unfortunately, the corresponding notion of generalized polycategory doesn't seem to be as elegant as a bifibration over a 2-multicategory.

\begin{defn}
  Let $\M$ be a (cartesian) multi-double-category, whose objects are called \emph{modes}.
  A \textbf{(cartesian) $\M$-polycategory} $\D$ consists of
  \begin{itemize}
  \item A set of objects, each assigned to a mode.
  \item For any lists of objects $\vec{A}$ and $\vec{B}$ with modes $\vec{p}$ and $\vec{q}$ respectively, and any vertical mode morphism $\vec{p} \xto{\alpha} r$ and horizontal mode morphism $\vec{q} \xto{\beta} r$, a set of morphisms $\DD\alpha\beta(\vec{A},\vec{B})$.
  \item (In the cartesian case) cartesian actions on these morphism sets ``over'' the cartesian actions on the domains of mode morphisms.
  \item For any object $A$ with mode $p$, an identity morphism $1_A \in \DD{1_p^v}{1_p^h}(A,A)$.
  \item For any square $e$ as above, and morphisms $f_i \in \DD{\phi_i}{\alpha_i}(\vec{A},\vec{B})$ and $g_j \in \DD{\gamma_j}{\psi_j}(\vec{B},\vec{C})$, a composite $\vec{g} \circ_e \vec{f} \in \DD{\delta\circ \vec{\phi}}{\beta\circ\vec{\psi}}(\vec{A},\vec{B})$.
  \item Composition is unital, using the identity squares in $\M$.
    That is, for any $f\in\DD{\alpha}{\beta}(\vec{A},\vec{B})$, we have $f = f \circ_{1_\alpha} 1_{\vec{A}}$ and $f = 1_{\vec{B}} \circ_{1_\beta} f$.
  \item Composition is associative, using the composition of squares in $\M$.
    This means $(h\circ_{e_3}g)\circ_{e_2 \circ e_1} f = h\circ_{e_3\circ e_2} (g\circ_{e_1} f)$ where $e_1,e_2,e_3$ are squares that can be composed like this:
    \[\begin{tikzcd}
       & . \ar[r] \ar[d]
      \ar[from=r,to=d,phantom,""{name=1,near start},""{name=2,near end}]
      \ar[Rightarrow,from=1,to=2,"e_1"]
      & .\ar[d] \\
      .\ar[r] \ar[d]
      \ar[from=r,to=d,phantom,""{name=3,near start},""{name=4,near end}]
      \ar[Rightarrow,from=3,to=4,"e_3"]
      & .\ar[r] \ar[d]
      \ar[from=r,to=d,phantom,""{name=5,near start},""{name=6,near end}]
      \ar[Rightarrow,from=5,to=6,"e_2"]
      & .\ar[d] \\
      .\ar[r] & .\ar[r] & .
    \end{tikzcd}\]
  \item (In the cartesian case) composition $g\circ_e f$ respects the cartesian actions.
    For the actions on the codomain of $g$ and the domain of $f$, what this means is obvious, and for \emph{permutation} actions on the domain of $g$ $=$ codomain of $f$ it makes sense, but I'm not sure whether there is a good condition to impose on diagonals and projections in the middle.
  \end{itemize}
\end{defn}

\section{Examples}
\label{sec:examples}

\begin{eg}\label{eg:degenerate}
  If the horizontal multicategory of $\M$ is discrete, then $\M$ should be essentially a (cartesian) 2-multicategory, and an $\M$-polycategory should be essentially a local discrete fibration $\D$ over it.
  The ``composition along squares'' operation incorporates together both the multicategorical composition in $\D$ and the local fibrational action of 2-cells in $\M$.
  I say ``should be essentially'' because in the cartesian case, the cartesian action on the horizontal multicategory means there is extra junk floating around, but hopefully we can ignore it at least in the syntax.
  Or we could define a version that is cartesian only in one direction.
\end{eg}

\begin{eg}\label{eg:linear}
  Suppose $\M$ is non-cartesian, and has one object $p$ that is a commutative monoid both horizontally and vertically.
  Let $\mu^n : (p,p,\dots,p) \to p$ denote the $n$-fold multiplication.
  Then the cut rule of classical linear logic
  \[
  \inferrule{\Gamma\vdash \Delta,A \\ A,\Phi\vdash \Psi}{\Gamma,\Phi \vdash \Delta,\Psi}
  \]
  can be generated by squares
  \[\begin{tikzcd}[column sep=huge]
    p^{n+1+m} \ar[r,"{(\mu^{n+1},1,\dots,1)}"] \ar[d,swap,"{(1,\dots,1,\mu^{1+m})}"] \drtwocell{e_{n,m}}
    & p^{1+m} \ar[d,"{\mu^{1+m}}"]
    \\ p^{n+1} \ar[r,swap,"{\mu^{n+1}}"] & p
  \end{tikzcd}\]
  by composing with identities, $(1,\dots,1,g) \circ_{e_{n,m}} (f,1,\dots,1)$.
  In fact, I believe all the above squares can be generated from the case $n=m=1$ by composing it with itself (and identities) horizontally and vertically.
  We should assert that such self-composites are associative, and interchanging.

  Note that the generating square $e_{1.1}$ is a sort of ``linear distributivity''.
  In fact I suspect that a functor from this $\M$ to $\Q(\Cat)$ is precisely a linearly distributive category.
  So just as we talk about modes in a (cartesian) 2-multicategory being monoids, commutative monoids, or cartesian monoids, we can talk about a mode in a multi-double-category being a ``linearly distributive monoid'', where the two monoidal structures involved in linear distributivity are horizontal and vertical.
  That is, the natural context in which to define a ``linearly distributive monoid'' is a multi-double-category.
  This is similar to how bimonoids are naturally defined in duoidal categories (see \cref{eg:duoidal}) and Frobenius objects are naturally defined in linearly distributive categories.
\end{eg}

\begin{eg}\label{eg:nonlinear}
  By making the previous example cartesian and adding ``contraction and weakening'' 2-cells, both vertically and horizontally, we should obtain some kind of classical nonlinear logic.
  For instance, horizontal contraction looks like
  \[\begin{tikzcd}
    p \ar[r,"\mu \Delta"] \ar[d,swap,"1_p"] \drtwocell{c^v}
    & p \ar[d,"1_p"]
    \\ p \ar[r,swap,"1_p"] & p
  \end{tikzcd}\]
  where $\mu\Delta$ means the cartesian contraction of $\mu:(p,p)\to p$ into a morphism $p\to p$.
  Then given a morphism like $f\in \DD{1}{\mu}(A,(B,B))$, by contraction-over-contraction we get $f\Delta \in \DD{1}{\mu\Delta}(A,B)$, and then by composing with an identity along the contraction 2-cell we get $1_B \circ_{c^v} f\Delta \in \DD{1}{1}(A,B)$.
  The possible degenerate-ness of this may relate to what equations, if any, we impose on the composites of these squares.
\end{eg}

\begin{eg}\label{eg:dirtt}
  Suppose we have a set of ``categories'' and let the objects of $\M$ (modes) be ``signed lists'' of categories such as $(A,B\o,C)$.
  A unary horizontal mode morphism consists of a pairing up of categories with their opposites in the \emph{domain only} and a bijection between the remaining categories (with variance) and the codomain.
  For instance, we have such a morphism $(A,B\o,C\o,B) \to (C\o,A)$ that pairs up the $B$'s and switches the $A$ and the $C\o$.
  There are no non-unary horizontal morphisms (except for those generated by weakening).
  But a \emph{vertical} mode morphism can be of arbitrary arity, and consists of removing the parentheses from the domain and then performing such a pairing.
  Thus, a morphism in $\D$ is ``intuitionistic'', with multiple modules in the domain and only one in the codomain, and parametrized by pairing some categories in the domains and some in the codomain and matching up the rest.
  The simplest squares look like this:
  \[\begin{tikzcd}[column sep=huge]
    (A,A\o,A) \ar[r,"\varepsilon^{12}_A"] \ar[d,swap,"\varepsilon^{23}_A"] \drtwocell{e_A}
    & A \ar[d,"1_A"]
    \\ A \ar[r,swap,"1_A"] & A
  \end{tikzcd}
  \qquad
  \begin{tikzcd}[column sep=huge]
    (A\o,A,A\o) \ar[r,"\varepsilon^{12}_A"] \ar[d,swap,"\varepsilon^{23}_A"] \drtwocell{e_{A\o}}
    & A\o \ar[d,"1_{A\o}"]
    \\ A\o \ar[r,swap,"1_{A\o}"] & A\o
  \end{tikzcd}\]
  where $\varepsilon^{12}$ pairs up the first two categories and $\varepsilon^{23}$ the second two.
  We need various generalizations of these too; I'm not sure how many we can get by composition and how many we need to put in by hand.
  This gives us the cut rule for directed type theory (without functors).
\end{eg}

Note that if we mentally reverse the direction of horizontal mode morphisms, and regard a square as an ``equation'' between its top-left ``composite'' and bottom-right ``composite'', then the generating squares $e_{1,1}$ of \cref{eg:linear} become the Frobenius axiom of a Frobenius algebra, while those of \cref{eg:dirtt} become the triangle identities of a dual pair.

\begin{eg}\label{eg:prop}
  Let $\M$ be generated by a vertical commutative monoid and a horizontal commutative monoid, as before, together with the following squares:
  \begin{mathpar}
    \begin{tikzcd}
      (p,p) \ar[r,"\mu"] \ar[d,"\mu",swap] \drtwocell{e_\mu} & p \ar[d,"1"] \\ p\ar[r,"1",swap] & p
    \end{tikzcd}
    \and
    \begin{tikzcd}
      (p,p) \ar[r,"{(1,1)}"] \ar[d,"{(1,1)}",swap] \drtwocell{e'_\mu} & (p,p) \ar[d,"\mu"] \\ (p,p)\ar[r,"\mu",swap] & p
    \end{tikzcd}
    \and
    \begin{tikzcd}
      () \ar[r,"\eta"] \ar[d,"\eta",swap] \drtwocell{e_\eta} & p \ar[d,"1"] \\ p\ar[r,"1",swap] & p
    \end{tikzcd}
    \and
    \begin{tikzcd}
      () \ar[r,"()"] \ar[d,"()",swap] \drtwocell{e'_\eta} & () \ar[d,"\eta"] \\ () \ar[r,"\eta",swap] & p
    \end{tikzcd}
  \end{mathpar}
  making the horizontal and vertical monoid structures into double-categorical ``companions''.
  Since a companion pair in $\Q(\K)$ is a pair of isomorphic arrows in $\K$, a functor $\M\to\Q(\K)$ is an object with two isomorphic commutative monoid structures, i.e.\ essentially only one such.

  An $\M$-polycategory, on the other hand, should be essentially a \emph{prop}.
  By composing the above generators we obtain corresponding squares for all $\mu^n$.
  The square $e_{\mu^n}$ then tells us that we can compose morphisms along $n$ objects at once, i.e.\ if $f\in\DD{\mu^m}{\mu^n}(\vec{A},\vec{B})$ and $g\in \DD{\mu^n}{\mu^k}(\vec{B},\vec{C})$ then $g\circ_{e_{\mu^n}} f \in \DD{\mu^m}{\mu^k}(\vec{A},\vec{C})$.
  And the square $e'_{\mu^n}$ gives us an ``$n$-ary identity'' $1_{\vec{A}} = \vec{1}_A \circ_{e'_{\mu^n}} \vec{1}_A\in\DD{\mu^n}{\mu^n}(\vec{A},\vec{A})$, which (by the companion axioms) acts as an identity for the $n$-ary composition.
\end{eg}

\begin{eg}\label{eg:duoidal}
  Let $\M$ be generated again by a vertical and a horizontal commutative monoid, but now the following generating squares:
  \begin{mathpar}
    \begin{tikzcd}
      (p,p,p,p) \ar[r,"{(1,\sigma,1)}"] \ar[d,"{(\mu,\mu)}",swap] \drrtwocell{e_{2,2}} &
      (p,p,p,p) \ar[r,"{(\mu,\mu)}"] & (p,p) \ar[d,"\mu"] \\
      (p,p) \ar[rr,"\mu",swap] & & p
    \end{tikzcd}
    \and
    \begin{tikzcd}
      () \ar[r,"()"] \ar[d,"()",swap] \drtwocell{e_{0,0}} & () \ar[d,"\eta"] \\ () \ar[r,"\eta",swap] & p
    \end{tikzcd}
    \and
    \begin{tikzcd}
      () \ar[r,"()"] \ar[d,"{(\eta,\eta)}",swap] \drtwocell{e_{0,2}} & () \ar[d,"\eta"] \\ (p,p) \ar[r,"\mu",swap] & p
    \end{tikzcd}
    \and
    \begin{tikzcd}
      () \ar[r,"{(\eta,\eta)}"] \ar[d,"()",swap] \drtwocell{e_{2,0}} & (p,p) \ar[d,"\mu"] \\ () \ar[r,"\eta",swap] & p
    \end{tikzcd}
  \end{mathpar}
  With appropriate axioms imposed, a functor $\M\to\Q(\Cat)$ is now a \emph{duoidal category}: a category with two monoidal structures $(\diamond,I)$ and $(\star,J)$ such that $(\star,J)$ is lax monoidal with respect to $(\diamond,I)$ (or equivalently the dual).
  An $\M$-polycategory is thus a structure that ``is to a duoidal category what a polycategory is to a linearly distributive category'': it is the composition structure induced on the sets of morphisms $A_1 \star \cdots \star A_m \to B_1\diamond \cdots \diamond B_n$ in a duoidal category.
\end{eg}

\section{Syntax}
\label{sec:syntax}

Syntactically, the judgments of the type theory for $\M$-polycategories will look something like $\Gamma \vdash_{\alpha\tot\beta} \Delta$, where $\alpha$ is a term describing a vertical mode morphism using variables from $\Gamma$ and $\beta$ is a term describing a horizontal mode morphism using variables from $\Delta$.
In \cref{eg:degenerate}, $\beta$ is trivial and this reduces to the previous syntax.
In \cref{eg:linear}, with $\mu$ written $x:p,y:p \vdash xy:p$, the sequents we are interested in look like
\[ x:A, y:B, z:C \vdash_{xyz \tot uvw} u:D, v:E, w:F \]
In \cref{eg:nonlinear} we get contraction by passing through other kinds of sequents like
\[ x:A, z:C \vdash_{xxz \tot uuw} u:D, w:F. \]
In \cref{eg:linear} these latter sequents are still there, but can presumably be ignored as junk since there are no squares allowing us to get anywhere else from them; the annotation $xxz\tot uuw$ says that we must ``use $x$ twice and $y$ once, and co-use $u$ twice and $w$ once''.
Finally, in \cref{eg:dirtt} with the morphisms written as $x:A, y:A\o \vdash \varepsilon(x,y) : ()$, we have extranatural judgments like
\[ x:A, y:B, z:B\o \vdash_{\varepsilon(y,z) \tot \varepsilon(u,v)} u:C, v:C\o, w:D \]
which are just a notational variant of the ``string diagrams'' syntax for directed type theory.

To incorporate substitution in first-order logics and functors in directed type theory, the natural thing to do is pass to \emph{triple} categories, where the third ``transversal'' direction describes substitution.
But I'll leave that for later.  \texttt{(-:}

\end{document}
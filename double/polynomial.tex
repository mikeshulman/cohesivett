\documentclass{article}
\usepackage{amsthm,mathtools,mathpartir,cleveref,mathrsfs,amssymb,url,paralist}
\newtheorem{thm}{Theorem}[section]
\newtheorem{lem}[thm]{Lemma}
\newtheorem{conj}[thm]{Conjecture}
\theoremstyle{definition}
\newtheorem{defn}{Definition}
\newtheorem{eg}{Example}
\usepackage{tikz}
\usepackage{tikz-cd}
\let\sto\looparrowright
\def\M{\mathcal{M}}
\def\K{\mathcal{K}}
\def\Q{\mathcal{Q}}
\def\SQ{\mathcal{S}\mathit{q}}
\def\Cat{\mathcal{C}\mathit{at}}
\def\Fib{\mathcal{F}\mathit{ib}}
\def\Prof{\mathcal{P}\mathit{rof}}
\def\D{\mathcal{D}}
\def\sC{\mathscr{C}}
\def\dom{\mathrm{dom}}
\def\cod{\mathrm{cod}}
\def\DD#1#2{\mathcal{D}^{#1}_{#2}}
\let\ot\leftarrow
\let\xto\xrightarrow
\let\xot\xleftarrow
\let\tot\leftrightarrow
\def\o{^{\circ}}
\def\type{\;\mathsf{type}}
\def\twocell#1#2#3{\ar[from=#1,to=#2,phantom,""{name=1,near start},""{name=2,near end}]\ar[Rightarrow,from=1,to=2,"#3"]}
\def\drtwocell{\twocell{r}{d}}
\def\drrtwocell{\twocell{rr}{d}}
\title{Type theory over polynomial monads}
\author{DL, PS, MS, MR}
\begin{document}
\maketitle

\section{Double categories and unary FOL}
\label{sec:double-categories}

For unary type theories in~\cite{ls:1var-adjoint-logic}, the mode theory was a 2-category.
For unary first-order logic, it is natural to generalize this to a double category $\M$, whose vertical category encodes substitution and whose horizontal category encodes structurality and judgment forms.
Semantically, the type theory over such a mode theory will be a vertical discrete fibration, i.e.\ a double functor $\D\to\M$ that is a discrete fibration on both the categories of objects-and-vertical-arrows and of horizontal-arrows-and-squares.

In the usual sort of first-order logic, the sorts are independent of the judgment structure, so we can construct $\M$ as a product.
On the one hand, take the free cartesian monoidal category generated by the cartesian multicategory of sorts and terms, whose objects are contexts of sorts, and regard it as a horizontally discrete double category.
On the other hand, take the 2-category describing some mode theory and regard it as a double category with only identity vertical arrows.
The product of these two double categories is the mode theory for the resulting first-order logic.

The fiber over an object of $\M$, which is a (context,mode) pair, is the set of propositions in that context with that mode.
Restriction along vertical arrows describes substitution of terms into propositions (which preserves the mode).
The fiber over a horizontal arrow of $\M$ is the set of implications between two propositions.
By construction of $\M$, both must be in the \emph{same} first-order context, while the relation of their modes is determined by the 2-category as before.
Restriction along squares describes both substitution of terms into implications and also structural actions on the modes.

We define $F$ and $U$ functors by horizontal fibration and opfibration conditions, imposing stability with respect to the vertical restrictions, so that substitution preserves the corresponding type formers.


\section{Polynomials}
\label{sec:polynomials}

A double category is an internal category in $\Cat$.
An internal category is equivalently a monad in a bicategory of spans.
The bicategory of spans in a category $\sC$ is equivalent to the bicategory of \emph{linear functors} between slice categories, a span $I \xot{q} A \xto{p} J$ corresponding to the functor $\sC/I \xto{q^*} \sC/A \xto{\Sigma_p} \sC/J$.
This sits inside the larger bicategory of \emph{polynomial functors}: those of the form
\[ \sC/I \xto{q^*} \sC/B \xto{\Pi_f} \sC/A \xto{\Sigma_p} \sC/J \]
for some \emph{polynomial} $I \xot{q} B \xto{f} A \xto{p} J$, where $f$ is an exponentiable morphism (i.e.\ $f^*$ has a right adjoint $\Pi_f$).
The natural ``transformations'' between polynomial functors are the cartesian ones (those whose naturality squares are all pullbacks); these correspond to diagrams 
\begin{equation}
\begin{tikzcd}
  I \ar[d] & B \ar[l] \ar[r]\ar[d] \ar[dr,phantom,near start,"\lrcorner"] & A \ar[r]\ar[d] & J\ar[d] \\
  I' & B' \ar[l] \ar[r] & A' \ar[r] & J'
\end{tikzcd}\label{eq:poly-cartmap}
\end{equation}
in which the middle square is a pullback.

This suggests that \emph{polynomial monads} in $\Cat$ (monads whose underlying functor is polynomial and whose unit and multiplication are cartesian) are a natural ``higher-arity'' version of double categories.
Let) us write such a polynomial as $\M_0 \xot{\mu_2} \M_2 \xto{\mu_1} \M_1 \xto{\mu_0} \M_0$.
Here $\M_0$ is the ``vertical category'' of objects and vertical arrows.
The objects and arrows of $\M_1$ are the horizontal arrows and the ``squares'', with $p$ assigning horizontal codomains.

The fiber of $\mu_1$ over a horizontal arrow is the ``arity'' of its horizontal domain.
Let us restrict to the case when all such fibers are finite discrete sets; thus $\mu_1$ and $\mu_2$ together assign to each horizontal arrow a family of objects indexed by a finite set as its horizontal domain.
Now the fiber of $\mu_1$ over a square $e:\alpha\to\beta$ is a \emph{span} $\dom(e):\dom(\alpha)\sto\dom(\beta)$, and $p$ assigns to each element of $\dom(e)$ a \emph{vertical arrow} between the objects it relates in $\dom(\alpha)$ and $\dom(\beta)$.
Composition in $\M_1$ allows us to compose squares vertically; the exponentiability of $\mu_1$ means exactly (in the fiberwise-discrete case) that the span morphism $\dom(e_2) \circ \dom(e_1) \to \dom(e_2\circ e_1)$ (arising from composition in $\M_2$) is an isomorphism.

The monad structure is what gives us horizontal composition.
On horizontal arrows, it says that if we have $\alpha : (y_i)_{i\in\dom(\alpha)} \to z$ and $\beta_i : (x_{ij})_{j\in \dom(\beta_i)} \to y_i$, then we have a morphism $\alpha \circ (\beta_i) : (\bar x_k)_{k\in \dom(\alpha \circ (\beta_i))} \to z$ and a bijection $\sigma : \sum_{i\in \dom(\alpha)} \dom(\beta_i) \to \dom(\alpha \circ (\beta_i))$ such that $\bar x_{\sigma(i,j)} = x_{ij}$.
If we assume furthermore that each fiber of $\mu_1$ is ordered and that the bijections $\sigma$ are order-preserving (with lexicographic order on their domains) --- which uniquely determines them --- then this becomes simply an ordinary non-symmetric multicategory structure.

Horizontal composition of squares is more complicated.
Suppose we have $\alpha : (y_i)_{i\in\dom(\alpha)} \to z$ and $\alpha' : (y'_i)_{i\in\dom(\alpha')} \to z'$, and also a square $e:\alpha\to \alpha'$, with $\dom(\alpha) \xot{s} \dom(e) \xto{t} \dom(\alpha')$ consisting of vertical arrows $(\phi_i : y_{s(i)} \to y'_{t(i)})_{i\in \dom(e)}$.
Now suppose we have $\beta_i : (x_{ij})_{j\in \dom(\beta_i)} \to y_i$ for each $i\in\dom(\alpha)$, and also $\beta'_i : (x'_{ij})_{j\in \dom(\beta'_i)} \to y'_i$ for each $i\in\dom(\alpha')$, so we can form $\alpha \circ (\beta_i)$ and $\alpha' \circ (\beta'_i)$.
Finally, suppose that for each $i\in \dom(e)$ we have a square $f_i : \beta_{s(i)} \to \beta_{t(i)}$ such that $\cod(f_i) = \phi_i$.
Then we get a composite square $e\circ (f_i) : \alpha \circ (\beta_i) \to \alpha' \circ (\beta'_i)$ with $\cod(e\circ (f_i)) = \cod(e)$ and $\dom(e\circ(f_i))$ bijectively related to $\sum_{i\in \dom(e)} \dom(f_i)$, with its vertical arrows assigned according to the domains of the $f_i$'s.

The idea of using a polynomial monad as a mode theory is that the objects are modes/sorts, the vertical arrows are for substitution, the horizontal arrows are judgment forms, and the squares implement substitution and structural action.
Suppose for simplicity that there are only identity vertical arrows, so we are back in the case of propositional logic.
Then the horizontal arrows are essentially a non-symmetric multicategory of modes, while the squares describe the allowable structural rules.

For instance, suppose there is one mode $p$ that is a horizontal monoid with multiplication $m:(p,p)\to p$ and unit $i:()\to p$.
On its own, this should give ordered logic.
Adding a square $s:m\to m$ such that $\dom(s) : (p,p) \sto (p,p)$ is the transposition automorphism allows an exchange rule.
Similarly, a square $c:1_p \to m$ such that $\dom(d): p \sto (p,p)$ is the full relation allows contraction, while a square $w:1_p \to i$ with $\dom(w):p \sto ()$ the unique span allows weakening.
The dual operations are just as easy to describe.

If $\mu_1$ is a discrete fibration, so that all the spans appearing in squares are the opposites of functions, and moreover $\M_0$ is discrete, then our polynomial can be identified with a \textbf{cartesian club} (a ``club over $\mathbf{S}^{\mathrm{op}}$'' in the sense of~\cite{kelly:mv-funct-calc,kelly:abst-coh}).
If the spans are functions or bijections, then we have a \textbf{cocartesian club} (a ``club over $\mathbf{S}$'') or a \textbf{symmetric club} (a ``club over $\mathbf{P}$'') respectively.
In the cartesian case there is an induced functor $\M_1 \to \mathbf{FinSet}^{\mathrm{op}}$; \textbf{cartesian 2-multicategories} as in~\cite{lsr:multi} can be identified with cartesian clubs for which this functor is a split fibration.

Since a polynomial monad is cartesian, it induces a notion of ``generalized multicategory'' in the sense of Leinster and Burroni~\cite{leinster:higher-opds,burroni:t-cats} using the double category of spans.
We can identify these with a slice category of polynomials:

\begin{lem}\label{thm:poly-multi}
  If $\M$ is a polynomial monad on a slice category $\sC/\M_0$ of a locally cartesian closed category $\sC$, then Leinster-Burroni $\M$-multicategories in $\sC/\M_0$ are equivalent to polynomial monads (on arbitrary slice categories of $\sC$) over $\M$ by a cartesian transformation as in~\eqref{eq:poly-cartmap}.
\end{lem}

In particular, cartesian clubs can be identified with $\mathcal{S}$-multicategories, where $\mathcal{S}$ is the monad for strictly associative finite products, and similarly for cocartesian and symmetric clubs.
However, more of the ``usual'' sorts of multicategories arise rather by considering monads on a double category of \emph{profunctors}~\cite{cs:multicats}.

\begin{lem}\label{thm:poly-multi-2}
  Suppose $\M$ is a polynomial monad in $\Cat$ where $\mu_1$ has discrete fibers, and assume furthermore that $\mu_0$ is a fibration and that the span associated to every $\mu_0$-cartesian arrow in $\M_1$ is a function (such as, for instance, if $\mu_1$ is an opfibration).
  Then the monad $\M$ induced on $\Cat/\M_0$ preserves (split) fibrations.
  Moreover, under the equivalence of \cref{thm:poly-multi}, the following properties correspond:
  \begin{enumerate}[(i)]
  \item Polynomial monads over $\M$ for which $\D_0\to\M_0$ and the induced functor $\D_1\to \Pi_{\mu_1} (\mu_2^* \D_0)$ (the adjunct of the map $\mu_1^* \D_1 \cong \D_2 \to \mu_2 ^* \D_0$ induced by the left-hand square in the morphism $\D\to \M$) are discrete fibrations.\label{item:pm1}
  \item $\M$-multicategories $\D_0 \leftarrow \D_1 \to \M(\D_0)$ for which $\D_0\to \M_0$ is a discrete fibration (hence $\M(\D_0)\to\M_0$ is a fibration) and $\D_0 \leftarrow \D_1 \to \M(\D_0)$ is a two-sided discrete fibration in $\Fib (\M_0)$.\label{item:pm2}
  \end{enumerate}
\end{lem}
\begin{proof}
  The first statement is a relativization of the fact that the dependent exponential of a fibration along an opfibration is a fibration (see for instance~\cite[Lemma 4.4.2]{weber:poly-pb}).
  To transport a dependent function backwards, we transport its argument forwards, apply the function, then transport the result backwards.

  For the second statement, we have $\M(\D_0) = \Pi_{\mu_1} (\mu_2^* \D_0)$ by construction of the monad $\M$.
  Moreover, since $\D_0$ is discrete in $\Fib(\M_0)$, any morphism with $\D_0$ as codomain is an opfibration, and a span out of $\D_0$ is a two-sided discrete fibration if and only if its other leg is a discrete fibration.
  Now standard arguments about composites of fibrations imply that if $\D_1 \to \M(\D_0)$ is a discrete fibration, then the composite $\D_1 \to \M(\D_0) \to \M_0$ is a fibration and $\D_1 \to \M(\D_0)$ is a discrete fibration in $\Fib(\M_0)$, while conversely if $\D_1 \to \M(\D_0)$ is a discrete fibration in $\Fib(\M_0)$ then it is also a discrete fibration in $\Cat$.
\end{proof}

\begin{conj}\label{thm:poly-multi-3}
  For a polynomial monad $\M$ as in \cref{thm:poly-multi-2}, $\M$-multicategories as in~(\ref{item:pm2}) can be identified with object-discrete $\M$-multicategories in $\Prof(\Fib(\M_0))$.
\end{conj}

The statement of~(\ref{item:pm2}) tells us that we have an object-discrete $\M$-span that represents a profunctor in $\Fib(\M_0)$.
The remaining content of the conjecture is nontrivial because $\M$ as a monad on $\Fib(\M_0)$ will not preserve two-sided discrete fibrations.
Indeed the extension of $\M$ to a monad on $\Prof(\Fib(\M_0))$ must be constructed in some other way before the conjecture is even fully precise.

It should be straightforward to formulate ``horizontal'' fibration and opfibration conditions on such a morphism $\D\to\M$.
The opfibration condition should correspond to being a \emph{representable} $\M$-multicategory as in~\cite{hermida:coh-univ,hermida:multicats,cs:multicats}, hence equivalent to a pseudo $\M$-algebra.

If \cref{thm:poly-multi-3} is false, I'm not sure which of the two things it wants to identify we ought to be considering as our intended semantics.
Even if \cref{thm:poly-multi-3} is true, I don't know how to ensure the condition on $\M$ in \cref{thm:poly-multi-2} as a property of the \emph{syntactic} mode theory, and without that condition I'm not even sure how to state a generalized-multicategory notion for the comparison, since the monad may not preserve fibrations.
This suggests that to be general we ought to consider ``discrete fibrations of polynomial monads'' as in~(\ref{item:pm1}) as the intended semantics --- though I'm not yet entirely reconciled to that conclusion, since generalized multicategories seem more natural to me from a categorical viewpoint.


\section{Type theory over polynomials}
\label{sec:tt-poly}

Here are some first thoughts about a type theory over polynomial monads.
For the mode theory, we have modes and horizontal mode morphisms forming an ordinary ordered logic.
We also have modes and vertical mode morphisms forming an ordinary unary logic, $x:p \vdash^v f(x):q$
Then we have a judgment for squares that looks like
\[
\left.{\textstyle\frac{x_1:p_1, \dots, x_n:p_n} {y_1:q_1,\dots,y_m:q_m}} \right| \dots z_i:y_{t(i)} = f_i(x_{s(i)}) \dots  \;\vdash\; e(\vec z_i) : {\beta(\vec y)} = g({\alpha(\vec x)})
\]
Here $\alpha : (p_1,\dots,p_n) \to r$ and $\beta: (q_1,\dots,q_m)\to s$ are terms for horizontal arrows, while $g:r\to s$ and the $f_i:p_{s(i)}\to q_{t(i)}$ are terms for vertical arrows.
The indices $i$ in the context of $z$'s label the elements of the span vertex in $\dom(e)$.

Horizontal composition is described by substitution into all three groups of variables.
Substitution into the $x$'s and $y$'s composes the horizontal-arrow domain and codomain, while substituting the $z$'s composes the squares.
For this to be well-defined we require that the types of the $z$'s match up when we perform the substitution of $x$'s and $y$'s into them.

Vertical composition is a bit less obvious, but basically consists of implementing span composition.
I am a little doubtful whether this will work (i.e.\ present a free polynomial monad) in full generality, due to the non-strict associativity of spans.
However, it ought at least to work in the following three cases, which I think cover (almost?)\ all previously studied type theories:
\begin{itemize}
\item All the spans are functions (a cocartesian club).
\item All the spans are opposites of functions (a cartesian club).
\item There are no equations to be imposed between mode squares (e.g.\ if we only care about provability).
\end{itemize}
Note that there is a partial cut-elimination theorem for symmetric clubs in~\cite{kelly:mv-funct-calc}, extended to the cartesian case in~\cite{kelly:club-doc}.

In the case of FOL and DirTT, the vertical mode theory also wants to include morphisms of higher arity.
We can incorporate this in a general way by starting with a prop type theory of ``sorts'', which might or might not include diagonals as generators.
Then the ``modes'' are defined to be contexts of sorts, and the vertical mode morphisms are the terms in the prop type theory of sorts.
To include propositional modal logic too, we could take the modes like ``true'' and ``valid'' to be the sorts (in which case we end up with junk modes like (true,valid)), or add an additional collection of ``pre-modes'' and let the modes be pairs of a pre-mode and a contexts of sorts.
(In the latter case it might be better to call the ``pre-modes'' simply ``modes'' and use a different word for the pairs.)

For the type theory over the mode theory, as before every type is assigned a mode, and the judgments are parametrized by a horizontal mode morphism.
If we pun the variables for types and their modes, as before, the judgments look about the same:
\begin{mathpar}
  A \type_p \and
  x_1:(A_1)_{p_1},\dots, x_n:(A_n)_{p_n} \vdash_{\alpha(x_1,\dots,x_n)} B_p
\end{mathpar}
when $x_1:p_1, \dots, x_n:p_n \vdash \alpha(x_1,\dots,x_n) : p$ is a horizontal mode morphism.
To include FOL and DirTT, however, it seems better to regard types as ``dependent on'' mode variables:
\begin{mathpar}
  x:p \vdash A(x) \type \and
  x_1:p_1, \dots ,x_n:p_n \mid u_1:A_1(x_1),\dots, u_n:A_n(x_n) \vdash B(\alpha(x_1,\dots,x_n))
\end{mathpar}
If we allow a prop type theory for the vertical mode theory, then each $x_n$ would actually be a context of sorts (plus perhaps a pre-mode).

We should have an (admissible?)\ rule of substitution along vertical mode morphisms into types:
\[
\inferrule{y:q \vdash A(y)\type \\ x:p \vdash^v f(x):q}{x:p \vdash A(f(x)) \type}
\]
We are abusing the notation a bit here: $B(\alpha(x_1,\dots,x_n))$ and $A(f(x))$ mean something different.
By $A(f(x))$ we mean the usual $A[f(x)/y]$, a syntactic substitution of $f(x)$ into $y$.
But $B(\alpha(x_1,\dots,x_n))$ is instead something formal, indicating the consequent and also the horizontal mode morphism parametrizing the judgment at once: we can't actually substitute a \emph{horizontal} mode term into a type.
Perhaps this is ill-advised and we should instead write something like
\[
  x_1:p_1, \dots ,x_n:p_n,y:p \mid u_1:A_1(x_1),\dots, u_n:A_n(x_n) \vdash_{y=\alpha(x_1,\dots,x_n)} B(y).
\]

We should also have a rule of substitution of term judgments along mode squares, something like this:
\begin{mathpar}
  \inferrule{
    \left.{\textstyle\frac{x_1:p_1, \dots, x_n:p_n} {y_1:q_1,\dots,y_m:q_m}} \right| \dots z_i:y_{t(i)} = f_i(x_{s(i)}) \dots  \;\vdash\; e(\vec z_i) : {\beta(\vec y)} = g({\alpha(\vec x)}) \\
    y_1:q_1, \dots ,y_m:q_m \mid A_1(y_1),\dots, A_m(y_m) \vdash B(\beta (\vec y))\\
    \forall i, C_{s(i)}(x_{s(i)}) \equiv A_{t(i)}(f_i(x_{s(i)}))
  }
  {x_1:p_1,\dots, x_n:p_n \mid \dots C_j(x_j)\dots \vdash B(g(\alpha(\vec x)))}
\end{mathpar}
This includes substitution of vertical mode morphisms into term judgments, and also structural actions by nontrivial 2-cells.
Here $B(g(\alpha(\vec x)))$ combines the above two kinds of ``substitution'': we actually substitute $g(u)$ into $B(v)$, then formally set $u$ to $\alpha(\vec x)$ to indicate both the consequent and the mode morphism.

If the condition of \cref{thm:poly-multi-2} doesn't hold (and I have no idea how to ensure that it does as a property of a syntactic mode theory), or if \cref{thm:poly-multi-3} fails, then possibly this rule should be different.

There is of course also a cut-over-cut rule, and so on.


\section{Unary classical logics}
\label{sec:unary-classical}

This is a bit weird, but to motivate the semantic structure to use for classical logics, let's consider ``unary classical logics''.
Of course this doesn't literally make much sense, since the distinction between classical and intuitionistic logic is precisely the presence of \emph{multiple} consquents.
What I mean is this: in unary adjoint logic~\cite{ls:1var-adjoint-logic} we parametrize a judgment by a morphism in the mode 2-category, but when we generalize this to the intuitionistic case~\cite{lsr:multi} it becomes a little more natural to think of this mode morphism as associated to the \emph{domain}, since that is where the variables that it uses come from, and where the structural action occurs.
In a classical logic we want to allow \emph{separate} structural actions on the domain and codomain, which suggests that we should have a \emph{separate} mode morphism for each of them, i.e.\ a sort of ``cospan'' of mode morphisms.
And this is something that makes sense already in the unary case.

In fact, because we want to allow different \emph{kinds} of structural action on the domain and codomain, we have to consider two different kinds of mode morphism.
Thus we have two 2-categories with the same objects; let us call the ``domain'' mode morphisms \emph{vertical} and the ``codomain'' ones \emph{horizontal}.
Then a judgment in the type theory will be parametrized by one of each kind of mode morphism, each with the same codomain, i.e.\ we will have $A_p \vdash_{\alpha = \beta} B_q$ where $\alpha : p\to r$ is vertical and $\beta:q\to r$ is horizontal.

In saying ``vertical'' and ``horizontal'' we of course suggest a double-categorical structure once again, but playing a different role: here both directions have to do with simple propositional logic.
The squares contain the information about the allowable cuts.
Specifically, if our ``double category of modes'' contains a square
\[
\begin{tikzcd}
  p \ar[r,"\alpha"] \ar[d,swap,"\beta"] \drtwocell{e} & q \ar[d,"\gamma"] \\ r \ar[r,swap,"\delta"] & s
\end{tikzcd}
\]
then we get a cut rule
\[
\inferrule{A_u \vdash_{\phi=\alpha} B_p \\ B_p \vdash_{\beta=\psi} C_v}{A_u \vdash_{\gamma\phi = \delta\psi} C_v}
\]

In fact the mode structure is, I think, not exactly an ordinary double category.
An ordinary double category does have 2-categories of vertical and horizontal morphisms, where the 2-cells are the squares with identities on the horizontal or vertical sides respectively.
However, here I believe we want to keep the ``horizontal 2-cells'' and ``vertical 2-cells'', which encode structural actions, separate from the squares, which encode cuts.
One reason is that this is what naturally arises semantically.
But another, better, reason is that if a structural rule like weakening on the codomain is encoded as the action of a square, then it seems that same square would be able to act on the domain as well and produce a corresponding ``strengthening'', which we don't want.

One possible semantic structure for the mode theory is a \textbf{double 2-category}: two 2-categories with the same objects, a double category whose vertical and horizontal arrows are those of these 2-categories, and actions of the 2-cells in the 2-categories on the squares in appropriate ways.
The corresponding notion of \emph{double bicategory}, where the 2-categories are weak, is due to~\cite{verity:base-change}; this is simply a stricter version that happens to be rather easier to define.

However, perhaps a better route is suggested by considering the first-order version, where the horizontal and vertical 2-categories are replaced by \emph{double} categories themselves, adding in a third direction of arrow that we call \emph{transversal}.
The transversal arrows are the same in both cases, since they tell us about substitution into types: thus we have \emph{three} categories forming the arrows of \emph{three} double categories in all possible ways.
The obvious way to see these is as a \textbf{triple category}, which adds a collection of ``cubes'' that can be composed in the expected ways, and ought to encode the interaction between substitution (of transversal mode morphisms into types) and cut.
Following~\cite{gp:intercategories-i,gp:intercategories-ii} we refer to the horizontal-transversal, vertical-transversal, and horizontal-vertical squares as \emph{horizontal}, \emph{vertical}, and \emph{basic cells} respectively.

A double 2-category is essentially a triple category with only identity transversal arrows and in which the horizontal and vertical cells ``act'' on the basic ones.
I don't see any obstacle to such an action --- it essentially means that from any ``allowable cut'' we can get a new one by incorporating structural actions on the cut formulas (is this perhaps something to do with the ``mix rule'' of classical logic?) --- but neither does it seem necessary.


\section{Double polynomials}
\label{sec:double-polynomials}

Now we need to combine the generalizations of \cref{sec:polynomials,sec:unary-classical}.
Just as an internal category in $\sC$ is a monad in the double category of spans, an internal \emph{double} category in $\sC$ is a ``double monad'' in the \emph{triple} category of spans.
The latter has objects of $\sC$ for objects, morphisms for transversal arrows, spans for horizontal and vertical arrows, and basic cells that look like
\[
\begin{tikzcd}
  \cdot & \cdot \ar[l] \ar[r] & \cdot \\
  \cdot \ar[u] \ar[d] & \cdot \ar[u] \ar[d] \ar[r] \ar[l] & \cdot \ar[u] \ar[d]\\
  \cdot & \cdot \ar[l] \ar[r] & \cdot
\end{tikzcd}
\]
Of course, just as the double category of spans is only a pseudo double category, this is only a pseudo triple category (pseudo in the horizontal and vertical directions); it is an \emph{intercategory} in the sense of~\cite{gp:intercategories-i,gp:intercategories-ii} whose interchangers are invertible.

A \textbf{double monad} (or \emph{intermonad} in the intercategory-case) in a triple category consists of a basic cell of the form
\[
\begin{tikzcd}
  A_{00} \ar[r,"{A_{10}}"] \ar[d,swap,"{A_{01}}"] \drtwocell{A_{11}} & A_{00} \ar[d,"{A_{01}}"]\\
  A_{00} \ar[r,swap,"{A_{10}}"] & A_{00}
\end{tikzcd}
\]
equipped with two unit cubes and two multiplication cubes (a horizontal one and a vertical one), satisfying evident associativity and interchange axioms.
The horizontal and vertical boundaries of these cubes tell us in particular that $A_{10}$ is a monad in the horizontal double category and $A_{01}$ is a monad in the vertical double category.

This suggests that our desired common generalization of \cref{sec:polynomials,sec:unary-classical} should be a double monad in some triple category whose horizontal and vertical morphisms are both polynomials in $\Cat$.
Such a triple category could be obtained by the ``quintets'' construction~\cite{pare:isotropic-intercats} from a triple category whose horizontal arrows are polynomials and whose non-identity vertical arrows we ignore (but whose basic cells we care about).

However, it's not yet clear to me what the correct choice is.
From the generalized-multicategory perspective, we should expect to have a ``profunctorial distributive law'' between our two polynomial monads as in~\cite{garner:polycats}, which suggests that the basic cells should consist of profunctors.
But this would be a generalization of the ``double 2-category'' approach in \cref{sec:unary-classical}, whereas I argued there that triple categories may be a better object in general.
A triple category is a double category internal to $\Cat$, and hence a double monad in the triple category of spans in $\Cat$; thus perhaps we should consider quintets in a triple category of polynomials in $\Cat$ and some kind of \emph{spans} in the other direction (rather than profunctors).
But what kind of spans I don't know.


\bibliography{../all.bib}
\bibliographystyle{alpha}

\end{document}
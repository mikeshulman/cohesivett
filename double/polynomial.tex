\documentclass{article}
\usepackage{amsthm,hyperref,mathtools,mathpartir,cleveref,mathrsfs,amssymb,url,paralist}
\newtheorem{thm}{Theorem}[section]
\newtheorem{lem}[thm]{Lemma}
\newtheorem{conj}[thm]{Conjecture}
\theoremstyle{definition}
\newtheorem{defn}[thm]{Definition}
\newtheorem{eg}[thm]{Example}
\theoremstyle{remark}
\newtheorem{rmk}[thm]{Remark}
\usepackage{tikz}
\usepackage{tikz-cd}
\let\sto\looparrowright
\def\M{\mathcal{M}}
\def\N{\mathcal{N}}
\def\K{\mathcal{K}}
\def\Q{\mathcal{Q}}
\def\Mh{\M_{\bullet 0}}
\def\Mv{\M_{0\bullet}}
\def\mult{\mathfrak{m}}
\def\pow{\mathfrak{p}}
\def\SQ{\mathcal{S}\mathit{q}}
\def\Cat{\mathcal{C}\mathit{at}}
\def\Fib{\mathcal{F}\mathit{ib}}
\def\Prof{\mathcal{P}\mathit{rof}}
\def\D{\mathcal{D}}
\def\sC{\mathscr{C}}
\def\dom{\mathrm{dom}}
\def\cod{\mathrm{cod}}
\def\DD#1#2{\mathcal{D}^{#1}_{#2}}
\def\Poly{\mathbb{P}\mathsf{oly}}
\def\Span{\mathsf{Span}}
\def\sZ{\mathcal{Z}}
\def\sQ{\mathcal{Q}}
\let\ot\leftarrow
\let\xto\xrightarrow
\let\xot\xleftarrow
\let\tot\leftrightarrow
\def\o{^{\circ}}
\def\type{\;\mathsf{type}}
\def\twocell#1#2#3{\ar[from=#1,to=#2,phantom,""{name=1,near start},""{name=2,near end}]\ar[Rightarrow,from=1,to=2,"#3"]}
\def\drtwocell{\twocell{r}{d}}
\def\drrtwocell{\twocell{rr}{d}}
\title{Type theory over polynomial monads}
\author{DL, PS, MS, MR}
\begin{document}
\maketitle

\section{Double categories and unary FOL}
\label{sec:double-categories}

For unary type theories in~\cite{ls:1var-adjoint-logic}, the mode theory was a 2-category.
For unary first-order logic, it is natural to generalize this to a double category $\M$, whose vertical category encodes substitution and whose horizontal category encodes structurality and judgment forms.
Semantically, the type theory over such a mode theory will be a vertical discrete fibration, i.e.\ a double functor $\D\to\M$ that is a discrete fibration on both the categories of objects-and-vertical-arrows and of horizontal-arrows-and-squares.

In the usual sort of first-order logic, the sorts are independent of the judgment structure, so we can construct $\M$ as a product.
On the one hand, take the free cartesian monoidal category generated by the cartesian multicategory of sorts and terms, whose objects are contexts of sorts, and regard it as a horizontally discrete double category.
On the other hand, take the 2-category describing some mode theory and regard it as a double category with only identity vertical arrows.
The product of these two double categories is the mode theory for the resulting first-order logic.

The fiber over an object of $\M$, which is a (context,mode) pair, is the set of propositions in that context with that mode.
Restriction along vertical arrows describes substitution of terms into propositions (which preserves the mode).
The fiber over a horizontal arrow of $\M$ is the set of implications between two propositions.
By construction of $\M$, both must be in the \emph{same} first-order context, while the relation of their modes is determined by the 2-category as before.
Restriction along squares describes both substitution of terms into implications and also structural actions on the modes.

We define $F$ and $U$ functors by horizontal fibration and opfibration conditions, imposing stability with respect to the vertical restrictions, so that substitution preserves the corresponding type formers.


\section{Polynomials}
\label{sec:polynomials}

A double category is an internal category in $\Cat$.
An internal category is equivalently a monad in a bicategory of spans.
The bicategory of spans in a category $\sC$ is equivalent to the bicategory of \emph{linear functors} between slice categories, a span $I \xot{q} A \xto{p} J$ corresponding to the functor $\sC/I \xto{q^*} \sC/A \xto{\Sigma_p} \sC/J$.
This sits inside the larger bicategory of \emph{polynomial functors}: those of the form
\[ \sC/I \xto{q^*} \sC/B \xto{\Pi_f} \sC/A \xto{\Sigma_p} \sC/J \]
for some \emph{polynomial} $I \xot{q} B \xto{f} A \xto{p} J$, where $f$ is an exponentiable morphism (i.e.\ $f^*$ has a right adjoint $\Pi_f$).
The most naturally-definable ``transformations'' between polynomial functors are the cartesian ones (those whose naturality squares are all pullbacks); these correspond to diagrams 
\begin{equation}
\begin{tikzcd}
  I \ar[d] & B \ar[l] \ar[r]\ar[d] \ar[dr,phantom,near start,"\lrcorner"] & A \ar[r]\ar[d] & J\ar[d] \\
  I' & B' \ar[l] \ar[r] & A' \ar[r] & J'
\end{tikzcd}\label{eq:poly-cartmap}
\end{equation}
in which the middle square is a pullback.
(Non-cartesian natural transformations can also be described in terms of polynomial data, but they are a bit more complicated.)

This suggests that \emph{polynomial monads} in $\Cat$ (monads whose underlying functor is polynomial and whose unit and multiplication are cartesian) are a natural ``higher-arity'' version of double categories.
Let) us write such a polynomial as $\M_0 \xot{\mu_2} \M_2 \xto{\mu_1} \M_1 \xto{\mu_0} \M_0$.
Here $\M_0$ is the ``vertical category'' of objects and vertical arrows.
The objects and arrows of $\M_1$ are the horizontal arrows and the ``squares'', with $p$ assigning horizontal codomains.

The fiber of $\mu_1$ over a horizontal arrow is the ``arity'' of its horizontal domain.
Let us restrict to the case when all such fibers are finite discrete sets; thus $\mu_1$ and $\mu_2$ together assign to each horizontal arrow a family of objects indexed by a finite set as its horizontal domain.
Now the fiber of $\mu_1$ over a square $e:\alpha\to\beta$ is a \emph{span} $\dom(e):\dom(\alpha)\sto\dom(\beta)$, and $p$ assigns to each element of $\dom(e)$ a \emph{vertical arrow} between the objects it relates in $\dom(\alpha)$ and $\dom(\beta)$.
Composition in $\M_1$ allows us to compose squares vertically; the exponentiability of $\mu_1$ means exactly (in the fiberwise-discrete case) that the span morphism $\dom(e_2) \circ \dom(e_1) \to \dom(e_2\circ e_1)$ (arising from composition in $\M_2$) is an isomorphism.

The monad structure is what gives us horizontal composition.
On horizontal arrows, it says that if we have $\alpha : (y_i)_{i\in\dom(\alpha)} \to z$ and $\beta_i : (x_{ij})_{j\in \dom(\beta_i)} \to y_i$, then we have a morphism $\alpha \circ (\beta_i) : (\bar x_k)_{k\in \dom(\alpha \circ (\beta_i))} \to z$ and a bijection $\sigma : \sum_{i\in \dom(\alpha)} \dom(\beta_i) \to \dom(\alpha \circ (\beta_i))$ such that $\bar x_{\sigma(i,j)} = x_{ij}$.
If we assume furthermore that each fiber of $\mu_1$ is ordered and that the bijections $\sigma$ are order-preserving (with lexicographic order on their domains) --- which uniquely determines them --- then this becomes simply an ordinary non-symmetric multicategory structure.

Horizontal composition of squares is more complicated.
Suppose we have $\alpha : (y_i)_{i\in\dom(\alpha)} \to z$ and $\alpha' : (y'_i)_{i\in\dom(\alpha')} \to z'$, and also a square $e:\alpha\to \alpha'$, with $\dom(\alpha) \xot{s} \dom(e) \xto{t} \dom(\alpha')$ consisting of vertical arrows $(\phi_i : y_{s(i)} \to y'_{t(i)})_{i\in \dom(e)}$.
Now suppose we have $\beta_i : (x_{ij})_{j\in \dom(\beta_i)} \to y_i$ for each $i\in\dom(\alpha)$, and also $\beta'_i : (x'_{ij})_{j\in \dom(\beta'_i)} \to y'_i$ for each $i\in\dom(\alpha')$, so we can form $\alpha \circ (\beta_i)$ and $\alpha' \circ (\beta'_i)$.
Finally, suppose that for each $i\in \dom(e)$ we have a square $f_i : \beta_{s(i)} \to \beta_{t(i)}$ such that $\cod(f_i) = \phi_i$.
Then we get a composite square $e\circ (f_i) : \alpha \circ (\beta_i) \to \alpha' \circ (\beta'_i)$ with $\cod(e\circ (f_i)) = \cod(e)$ and $\dom(e\circ(f_i))$ bijectively related to $\sum_{i\in \dom(e)} \dom(f_i)$, with its vertical arrows assigned according to the domains of the $f_i$'s.

The idea of using a polynomial monad as a mode theory is that the objects are modes/sorts, the vertical arrows are for substitution, the horizontal arrows are judgment forms, and the squares implement substitution and structural action.
Suppose for simplicity that there are only identity vertical arrows, so we are back in the case of propositional logic.
Then the horizontal arrows are essentially a non-symmetric multicategory of modes, while the squares describe the allowable structural rules.

For instance, suppose there is one mode $p$ that is a horizontal monoid with multiplication $m:(p,p)\to p$ and unit $i:()\to p$.
On its own, this should give ordered logic.
Adding a square $s:m\to m$ such that $\dom(s) : (p,p) \sto (p,p)$ is the transposition automorphism allows an exchange rule.
Similarly, a square $c:1_p \to m$ such that $\dom(d): p \sto (p,p)$ is the full relation allows contraction, while a square $w:1_p \to i$ with $\dom(w):p \sto ()$ the unique span allows weakening.
The dual operations are just as easy to describe.

If $\mu_1$ is a discrete fibration, so that all the spans appearing in squares are the opposites of functions, and moreover $\M_0$ is discrete, then our polynomial can be identified with a \textbf{cartesian club} (a ``club over $\mathbf{S}^{\mathrm{op}}$'' in the sense of~\cite{kelly:mv-funct-calc,kelly:abst-coh}).
If the spans are functions or bijections, then we have a \textbf{cocartesian club} (a ``club over $\mathbf{S}$'') or a \textbf{symmetric club} (a ``club over $\mathbf{P}$'') respectively.
In the cartesian case there is an induced functor $\M_1 \to \mathbf{FinSet}^{\mathrm{op}}$; \textbf{cartesian 2-multicategories} as in~\cite{lsr:multi} can be identified with cartesian clubs for which this functor is a split fibration.

Since a polynomial monad is cartesian, it induces a notion of ``generalized multicategory'' in the sense of Leinster~\cite{leinster:higher-opds} and Burroni~\cite{burroni:t-cats} using the double category of spans.
We can identify these with a slice category of polynomials:

\begin{lem}\label{thm:poly-multi}
  If $\M$ is a polynomial monad on a slice category $\sC/\M_0$ of a locally cartesian closed category $\sC$, then Leinster--Burroni $\M$-multicategories in $\sC/\M_0$ are equivalent to polynomial monads (on arbitrary slice categories of $\sC$) over $\M$ by a cartesian transformation as in~\eqref{eq:poly-cartmap}.
\end{lem}

In particular, cartesian clubs can be identified with $\mathcal{S}$-multicategories, where $\mathcal{S}$ is the monad for strictly associative finite products, and similarly for cocartesian and symmetric clubs.
More of the ``usual'' sorts of multicategories arise rather by considering monads on a double category of \emph{profunctors}~\cite{cs:multicats}.
At least in many cases, these can be identified with Leinster--Burroni multicategories whose underlying span is a ``two-sided discrete fibration'', i.e.\ $\D_1 \to \D_0$ is an opfibration (which is a trivial condition if $\D_0$ is discrete), $\D_1 \to \M(\D_0)$ is a fibration, and $\D_1 \to \D_0 \times \M(\D_0)$ has discrete fibers.
We can also identify this condition in terms of polynomials.

\begin{lem}\label{thm:poly-multi-2}
  Suppose $\M$ is a polynomial monad in $\Cat$ where $\mu_1$ has finite discrete fibers, and assume furthermore that $\mu_0$ is a fibration and that the arity span associated to every $\mu_0$-cartesian arrow in $\M_1$ (i.e.\ the fiber of $\mu_2$ over it) is a function (such as, for instance, if $\mu_1$ is an opfibration).
  Then the monad $\M$ induced on $\Cat/\M_0$ preserves (split) fibrations.
  Moreover, under the equivalence of \cref{thm:poly-multi}, the following properties correspond:
  \begin{enumerate}[(i)]
  \item Polynomial monads over $\M$ for which $\D_0\to\M_0$ and the induced functor $\D_1\to \Pi_{\mu_1} (\mu_2^* \D_0)$ (the adjunct of the map $\mu_1^* \D_1 \cong \D_2 \to \mu_2 ^* \D_0$ induced by the left-hand square in the morphism $\D\to \M$) are discrete fibrations.\label{item:pm1}
  \item $\M$-multicategories $\D_0 \leftarrow \D_1 \to \M(\D_0)$ for which $\D_0\to \M_0$ is a discrete fibration (hence $\M(\D_0)\to\M_0$ is a fibration) and $\D_0 \leftarrow \D_1 \to \M(\D_0)$ is a two-sided discrete fibration in $\Fib (\M_0)$.\label{item:pm2}
  \end{enumerate}
\end{lem}
\begin{proof}
  The first statement is a relativization of the fact that the dependent exponential of a fibration along an opfibration is a fibration (see for instance~\cite[Lemma 4.4.2]{weber:poly-pb}).
  To transport a dependent function backwards, we transport its argument forwards, apply the function, then transport the result backwards.

  For the second statement, we have $\M(\D_0) = \Pi_{\mu_1} (\mu_2^* \D_0)$ by construction of the monad $\M$.
  Moreover, since $\D_0$ is discrete in $\Fib(\M_0)$, any morphism with $\D_0$ as codomain is an opfibration, and a span out of $\D_0$ is a two-sided discrete fibration if and only if its other leg is a discrete fibration.
  Now standard arguments about composites of fibrations imply that if $\D_1 \to \M(\D_0)$ is a discrete fibration, then the composite $\D_1 \to \M(\D_0) \to \M_0$ is a fibration and $\D_1 \to \M(\D_0)$ is a discrete fibration in $\Fib(\M_0)$, while conversely if $\D_1 \to \M(\D_0)$ is a discrete fibration in $\Fib(\M_0)$ then it is also a discrete fibration in $\Cat$.
\end{proof}

It should be straightforward to formulate ``horizontal'' fibration and opfibration conditions on such a morphism $\D\to\M$.
The opfibration condition should correspond to being a \emph{representable} $\M$-multicategory as in~\cite{hermida:coh-univ,hermida:multicats,cs:multicats}, hence equivalent to a pseudo $\M$-algebra.

\begin{conj}\label{thm:poly-multi-3}
  For a polynomial monad $\M$ as in \cref{thm:poly-multi-2}, $\M$-multicategories as in~(\ref{item:pm2}) can be identified with object-discrete $\M$-multicategories in $\Prof(\Fib(\M_0))$.
\end{conj}

The statement of~(\ref{item:pm2}) tells us that we have an object-discrete $\M$-span that represents a profunctor in $\Fib(\M_0)$.
The remaining content of the conjecture is nontrivial because $\M$ as a monad on $\Fib(\M_0)$ will not preserve two-sided discrete fibrations.
Indeed the extension of $\M$ to a monad on $\Prof(\Fib(\M_0))$ must be constructed in some other way before the conjecture is even fully precise.

If \cref{thm:poly-multi-3} is false, I'm not sure which of the two things it wants to identify we ought to be considering as our intended semantics.
Even if \cref{thm:poly-multi-3} is true, I don't know how to ensure the condition on $\M$ in \cref{thm:poly-multi-2} as a property of the \emph{syntactic} mode theory.
This suggests that in general, our intended semantics ought to lie in Leinster--Burroni $\M$-multicategories, or equivalently polynomial monads over $\M$.
(Although it's not clear to me that there is anything very sensical to be said in the case when the condition of \cref{thm:poly-multi-2} fails, since then we can't really regard the monad as describing structure on fibrations the way we want to in other cases.)


\section{Type theory over polynomials}
\label{sec:tt-poly}

Here are some first thoughts about a type theory over polynomial monads.
For the mode theory, we have modes and horizontal mode morphisms forming an ordinary ordered logic.
We also have modes and vertical mode morphisms forming an ordinary unary logic, $x:p \vdash^v f(x):q$.
Then we have a judgment for squares that looks like
\[
\left.{\textstyle\frac{x_1:p_1, \dots, x_n:p_n} {y_1:q_1,\dots,y_m:q_m}} \right| \dots z_i:y_{t(i)} = f_i(x_{s(i)}) \dots  \;\vdash\; e(\vec z_i) : {\beta(\vec y)} = g({\alpha(\vec x)})
\]
Here $\alpha : (p_1,\dots,p_n) \to r$ and $\beta: (q_1,\dots,q_m)\to s$ are terms for horizontal arrows, while $g:r\to s$ and the $f_i:p_{s(i)}\to q_{t(i)}$ are terms for vertical arrows.
The indices $i$ in the context of $z$'s label the elements of the span vertex in $\dom(e)$.

Horizontal composition is described by substitution into all three groups of variables.
Substitution into the $x$'s and $y$'s composes the horizontal-arrow domain and codomain, while substituting the $z$'s composes the squares.
For this to be well-defined we require that the types of the $z$'s match up when we perform the substitution of $x$'s and $y$'s into them.

Vertical composition is a bit less obvious, but basically consists of implementing span composition.
I am a little doubtful whether this will work (i.e.\ present a free polynomial monad) in full generality, due to the non-strict associativity of spans.
However, it ought at least to work in the following three cases, which I think cover (almost?)\ all previously studied type theories:
\begin{itemize}
\item All the spans are functions (a cocartesian club).
\item All the spans are opposites of functions (a cartesian club).
\item There are no equations to be imposed between mode squares (e.g.\ if we only care about provability).
\end{itemize}
(In fact, nearly all examples seem to be cartesian clubs; contraction and weakening are much more common than their duals.)
Note that there is a partial cut-elimination theorem for symmetric clubs in~\cite{kelly:mv-funct-calc}, extended to the cartesian case in~\cite{kelly:club-doc}.

In the case of FOL and DirTT, the vertical mode theory also wants to include morphisms of higher arity.
We can incorporate this in a general way by starting with a prop type theory of ``sorts'', which might or might not include diagonals as generators.
Then the ``modes'' are defined to be contexts of sorts, and the vertical mode morphisms are the terms in the prop type theory of sorts.
To include propositional modal logic too, we could take the modes like ``true'' and ``valid'' to be the sorts (in which case we end up with junk modes like (true,valid)), or add an additional collection of ``pre-modes'' and let the modes be pairs of a pre-mode and a contexts of sorts.
(In the latter case it might be better to call the ``pre-modes'' simply ``modes'' and use a different word for the pairs.)

For the type theory over the mode theory, as before every type is assigned a mode, and the judgments are parametrized by a horizontal mode morphism.
If we pun the variables for types and their modes, as before, the judgments look about the same:
\begin{mathpar}
  A \type_p \and
  x_1:(A_1)_{p_1},\dots, x_n:(A_n)_{p_n} \vdash_{\alpha(x_1,\dots,x_n)} B_p
\end{mathpar}
when $x_1:p_1, \dots, x_n:p_n \vdash \alpha(x_1,\dots,x_n) : p$ is a horizontal mode morphism.
To include FOL and DirTT, however, it seems better to regard types as ``dependent on'' mode variables:
\begin{mathpar}
  x:p \vdash A(x) \type \and
  x_1:p_1, \dots ,x_n:p_n \mid u_1:A_1(x_1),\dots, u_n:A_n(x_n) \vdash_{\alpha(x_1,\dots,x_n)\tot v} B(v) \mid v:q
\end{mathpar}
If we allow a prop type theory for the vertical mode theory, then each $x_n$ would actually be a context of sorts (plus perhaps a pre-mode).

We should have an (admissible) rule of substitution along vertical mode morphisms into types:
\[
\inferrule{y:q \vdash A(y)\type \\ x:p \vdash^v f(x):q}{x:p \vdash A(f(x)) \type}
\]
And also a rule of substitution of term judgments along mode squares, something like this:
\begin{mathpar}
  \inferrule{
    \left.{\textstyle\frac{x_1:p_1, \dots, x_n:p_n} {y_1:q_1,\dots,y_m:q_m}} \right| \dots z_i:y_{t(i)} = f_i(x_{s(i)}) \dots  \;\vdash\; e(\vec z_i) : {\beta(\vec y)} = g({\alpha(\vec x)}) \\
    y_1:q_1, \dots ,y_m:q_m \mid A_1(y_1),\dots, A_m(y_m) \vdash_{\beta (\vec y)\tot v} B(v) \mid v:s\\
    \forall i, C_{s(i)}(x_{s(i)}) \equiv A_{t(i)}(f_i(x_{s(i)}))
  }
  {x_1:p_1,\dots, x_n:p_n \mid \cdots C_j(x_j)\dots \vdash_{\alpha(\vec x)\tot u} B(g(u)) \mid u:r}
\end{mathpar}
This includes substitution of vertical mode morphisms into term judgments, and also structural actions by nontrivial 2-cells.
Here $B(g(\alpha(\vec x)))$ combines the above two kinds of ``substitution'': we actually substitute $g(u)$ into $B(v)$, then formally set $u$ to $\alpha(\vec x)$ to indicate both the consequent and the mode morphism.

Note that unlike for cartesian 2-multicategories, to get a structural action like contraction
\[ \inferrule{x:A,y:A\vdash_{x\otimes y} B}{x:A\vdash_x B} \]
we don't need to pass through a ``junk'' sequent like $x:A \vdash_{x\otimes x} B$.
Contraction arises directly from a single mode square like so:
\begin{mathpar}
  \inferrule{\left.{\textstyle\frac{x:p}{y_1:p,y_2:p}}\right| z_1 : y_1 = x, z_2 : y_2 = x \;\vdash\; e(z_1,z_2) : y_1\otimes y_2 = x \\
    y_1:p, y_2:p \mid A_1(y_1), A_2(y_2) \vdash_{(y_1\otimes y_2) \tot u} B(u) \mid u:p \\
    C(x) \equiv A_1(x) \\ C(x) \equiv A_2(x)}
  {x:p \mid C(x) \vdash_{x\tot u} B(u) \mid u:p}
\end{mathpar}
I don't think this prevents us from talking about $n$-linear types either: we just need to put them in by hand.
For instance, suppose in addition to a commutative monoid structure $\mult:(p,p)\to p$ we have power operations $\pow^n:p\to p$, with laws like $\pow^n \circ \pow^k = \pow^{nk}$ and $\pow^n(\mult(x,y)) = \mult(\pow^n(x),\pow^n(y))$.
Then we can have a square allowing us to contract $\mult(x,y)$ to $\pow^2(x)$, and so on:
\begin{mathpar}
  \inferrule{\left.{\textstyle\frac{x:p}{y_1:p,y_2:p}}\right| z_1 : y_1 = x, z_2 : y_2 = x \;\vdash\; e(z_1,z_2) : y_1\otimes y_2 = \pow^2(x) \\
    y_1:p, y_2:p \mid A_1(y_1), A_2(y_2) \vdash_{(y_1\otimes y_2) \tot v} B(v) \mid v:p \\
    C(x) \equiv A_1(x) \\ C(x) \equiv A_2(x)}
  {x:p \mid C(x) \vdash_{\pow^2(x)\tot u} B(u) \mid u:p}
\end{mathpar}
Then an $F$-type for $\pow^n$ will be an $n$-linear type, and so on.

If the condition of \cref{thm:poly-multi-2} doesn't hold (and I have no idea how to ensure that it does as a property of a syntactic mode theory), or if \cref{thm:poly-multi-3} fails, then possibly the rule for the action of squares should be different.
However, that condition or something like it seems necessary in order for FOL-substitution to be admissible: the only way to substitute into an $F$-type is to distribute the substitution to the $F$ and to its arguments, and the span being a function seems necessary for the arguments of the substituted $F$ to be determined uniquely.
This suggests that perhaps some similar condition might be necessary to deal with $U$-types.
(If $\M_0$ is discrete, of course, there is nothing to worry about.)

There is of course also a cut-over-cut rule, and so on.
Note that~\cite{kelly:cutelim} contains a partial cut-elimination theorem for type theory over symmetric clubs with $\M_0$ discrete; I think his theorem is essentially eliminating only cuts at $U$-types.


\section{Unary classical logics}
\label{sec:unary-classical}

This is a bit weird, but to motivate the semantic structure to use for classical logics, let's consider ``unary classical logics''.
Of course this doesn't literally make much sense, since the distinction between classical and intuitionistic logic is precisely the presence of \emph{multiple} consquents.
What I mean is this: in unary adjoint logic~\cite{ls:1var-adjoint-logic} we parametrize a judgment by a morphism in the mode 2-category, but when we generalize this to the intuitionistic case~\cite{lsr:multi} it becomes a little more natural to think of this mode morphism as associated to the \emph{domain}, since that is where the variables that it uses come from, and where the structural action occurs.
In a classical logic we want to allow \emph{separate} structural actions on the domain and codomain, which suggests that we should have a \emph{separate} mode morphism for each of them, i.e.\ a sort of ``cospan'' of mode morphisms.
And this is something that makes sense already in the unary case.

In fact, because we want to allow different \emph{kinds} of structural action on the domain and codomain, we have to consider two different kinds of mode morphism.
Thus we have two 2-categories with the same objects; let us call the ``domain'' mode morphisms \emph{vertical} and the ``codomain'' ones \emph{horizontal}.
Then a judgment in the type theory will be parametrized by one of each kind of mode morphism, each with the same codomain, i.e.\ we will have $A_p \vdash_{\alpha = \beta} B_q$ where $\alpha : p\to r$ is vertical and $\beta:q\to r$ is horizontal.

In saying ``vertical'' and ``horizontal'' we of course suggest a double-categorical structure once again, but playing a different role: here both directions have to do with simple propositional logic.
The squares contain the information about the allowable cuts.
Specifically, if our ``double category of modes'' contains a square
\[
\begin{tikzcd}
  p \ar[r,"\alpha"] \ar[d,swap,"\beta"] \drtwocell{e} & q \ar[d,"\gamma"] \\ r \ar[r,swap,"\delta"] & s
\end{tikzcd}
\]
then we get a cut rule
\[
\inferrule{A_u \vdash_{\phi=\alpha} B_p \\ B_p \vdash_{\beta=\psi} C_v}{A_u \vdash_{\gamma\phi = \delta\psi} C_v}
\]

In fact the mode structure is, I think, not exactly an ordinary double category.
An ordinary double category does have 2-categories of vertical and horizontal morphisms, where the 2-cells are the squares with identities on the horizontal or vertical sides respectively.
However, here I believe we want to keep the ``horizontal 2-cells'' and ``vertical 2-cells'', which encode structural actions, separate from the squares, which encode cuts.
One reason is that this is what naturally arises semantically.
But another, better, reason is that if a structural rule like weakening on the codomain is encoded as the action of a square, then it seems that same square would be able to act on the domain as well and produce a corresponding ``strengthening'', which we don't want.

One possible semantic structure for the mode theory is a \textbf{double 2-category}: two 2-categories with the same objects, a double category whose vertical and horizontal arrows are those of these 2-categories, and actions of the 2-cells in the 2-categories on the squares in appropriate ways.
The corresponding notion of \emph{double bicategory}, where the 2-categories are weak, is due to~\cite{verity:base-change}; this is simply a stricter version that happens to be rather easier to define.

However, perhaps a better route is suggested by considering the first-order version, where the horizontal and vertical 2-categories are replaced by \emph{double} categories themselves, adding in a third direction of arrow that we call \emph{transversal}.
The transversal arrows are the same in both cases, since they tell us about substitution into types: thus we have \emph{three} categories forming the arrows of \emph{three} double categories in all possible ways.
The obvious way to see these is as a \textbf{triple category}, which adds a collection of ``cubes'' that can be composed in the expected ways, and ought to encode the interaction between substitution (of transversal mode morphisms into types) and cut.
Following~\cite{gp:intercategories-i,gp:intercategories-ii} we refer to the horizontal-transversal, vertical-transversal, and horizontal-vertical squares as \emph{horizontal}, \emph{vertical}, and \emph{basic cells} respectively.

A double 2-category is essentially a triple category with only identity transversal arrows and in which the horizontal and vertical cells ``act'' on the basic ones.
This would essentially mean that from any ``allowable cut'' we can get a new one by incorporating structural actions on the cut formulas (is this perhaps something to do with the ``mix rule'' of classical logic?).
This would be sensible to have, but also seems sensible to omit, and by omitting it the higher-ary case below is simplified.


\section{Double polynomial monads}
\label{sec:double-polynomials}

Now we need to combine the generalizations of \cref{sec:polynomials,sec:unary-classical}.
Just as an internal category in a category $\sC$ is a monad in the double category of spans, an internal \emph{double} category in $\sC$ is a ``double monad'' in the \emph{triple} category of spans.
The latter has objects of $\sC$ for objects, morphisms for transversal arrows, spans for horizontal and vertical arrows, and basic cells that look like
\[
\begin{tikzcd}
  \cdot & \cdot \ar[l] \ar[r] & \cdot \\
  \cdot \ar[u] \ar[d] & \cdot \ar[u] \ar[d] \ar[r] \ar[l] & \cdot \ar[u] \ar[d]\\
  \cdot & \cdot \ar[l] \ar[r] & \cdot
\end{tikzcd}
\]
Of course, just as the double category of spans is only a pseudo double category, this is only a pseudo triple category (pseudo in the horizontal and vertical directions); it is an \emph{intercategory} in the sense of~\cite{gp:intercategories-i,gp:intercategories-ii} whose interchangers are invertible.

A \textbf{double monad} (or \emph{intermonad} in the intercategory-case) in a triple category consists of a basic cell of the form
\[
\begin{tikzcd}
  A_{00} \ar[r,"{A_{10}}"] \ar[d,swap,"{A_{01}}"] \drtwocell{A_{11}} & A_{00} \ar[d,"{A_{01}}"]\\
  A_{00} \ar[r,swap,"{A_{10}}"] & A_{00}
\end{tikzcd}
\]
equipped with two unit cubes and two multiplication cubes (a horizontal one and a vertical one), satisfying evident associativity and interchange axioms.
The horizontal and vertical boundaries of these cubes tell us in particular that $A_{10}$ is a monad in the horizontal double category and $A_{01}$ is a monad in the vertical double category.

This suggests that our desired common generalization of \cref{sec:polynomials,sec:unary-classical} should be a double monad in some triple category whose horizontal and vertical morphisms are both polynomials in $\Cat$.
From the generalized-multicategory perspective, such a double monad will be a sort of \emph{distributive law} between our two polynomial monads.
Indeed, a double monad in the triple category of quintets in a bicategory is precisely a distributive law therein, and similarly for pseudo-distributive laws.

In~\cite{garner:polycats} such a pseudo-distributive law in the double category of \emph{profunctors} was used to describe ordinary polycategories.
But this would be a generalization of the ``double 2-category'' approach in \cref{sec:unary-classical}, whereas I argued there that triple categories may be a better object in general.
A triple category is a double category internal to $\Cat$, and hence a double monad in the triple category of spans in $\Cat$; thus it seems reasonable to ask the basic cells to be some kind of span between polynomials.
Here is one such construction.

By the arguments of~\cite[\S3.3]{weber:poly-pb}, the double category $\Poly$ of polynomials has ``strong pullbacks'', i.e.\ it is an internal category in the 2-category of categories with pullbacks.
Thus, by~\cite[\S6.1]{gp:intercategories-ii}, there is a triple category $\Span(\Poly)$ whose horizontal double category is that of polynomials in $\Cat$ and whose vertical structure consists of spans: vertical arrows are spans of functors, and basic cells are spans of polynomials as in
\[
\begin{tikzcd}
  \M_0 & \M_2 \ar[l] \ar[r] & \M_1 \ar[r] & \M_0\\
  \D_0 \ar[u] \ar[d] & \D_2 \ar[l] \ar[r] \ar[u] \ar[d] \ar[ur,phantom,near start,"\urcorner"] \ar[dr,phantom,near start,"\lrcorner"] & \D_1 \ar[r] \ar[u] \ar[d] & \D_0 \ar[u] \ar[d]\\
  \N_0 & \N_2 \ar[l] \ar[r] & \N_1 \ar[r] & \N_0
\end{tikzcd}
\]
It is easy to check that this is \emph{transversally invariant} in the sense of~\cite{pare:isotropic-intercats} (basic cells can be transported along transversal isomorphisms).
Thus we can discard the nonidentity vertical arrows and vertical cells to obtain a \emph{cylindrical} triple category $\sZ(\Span(\Poly))$, and then apply the \emph{quintets} construction from~\cite{pare:isotropic-intercats} to obtain a new triple category $\sQ(\sZ(\Span(\Poly)))$.
Its horizontal \emph{and} vertical double categories are both $\Poly$, and its basic cells
\[
\begin{tikzcd}
  A \ar[r,"\M"] \ar[d,swap,"\N"] \drtwocell{} & B \ar[d,"\mathcal{P}"] \\ C \ar[r,swap,"\mathcal{R}"] & D
\end{tikzcd}
\]
are spans of polynomials relating the polynomial composite $\mathcal{P}\circ\M$ to the composite $\mathcal{R}\circ\N$.
\[
\begin{tikzcd}
   & (\mathcal{P}\M)_2 \ar[dl] \ar[r] & (\mathcal{P}\M)_1 \ar[dr] & \\
  A & \mathcal{E}_2 \ar[l] \ar[r] \ar[u] \ar[d] \ar[ur,phantom,near start,"\urcorner"] \ar[dr,phantom,near start,"\lrcorner"] & \mathcal{E}_1 \ar[r] \ar[u] \ar[d] & D\\
   & (\mathcal{R}\N)_2 \ar[ul] \ar[r] & (\mathcal{R}\N)_1 \ar[ur] & 
\end{tikzcd}
\]

\begin{defn}
  A \textbf{double polynomial monad} (in $\Cat$) is a double monad in $\sQ(\sZ(\Span(\Poly)))$.
\end{defn}

Thus it consists of two polynomial monads $\Mv$ and $\Mh$ with the same category of objects $\M_{00}$, together with a span of polynomials
\[
\begin{tikzcd}
   & (\Mv\Mh)_2 \ar[dl] \ar[r] & (\Mv\Mh)_1 \ar[dr] & \\
  \M_{00} & \M_2 \ar[l] \ar[r] \ar[u] \ar[d] \ar[ur,phantom,near start,"\urcorner"] \ar[dr,phantom,near start,"\lrcorner"] & \M_{11} \ar[r] \ar[u] \ar[d] & \M_{00}\\
   & (\Mh\Mv)_2 \ar[ul] \ar[r] & (\Mh\Mv)_1 \ar[ur] & 
\end{tikzcd}
\]
with composition and unit structure.
Just as any polynomial induces a functor between slice categories and any transformation as in~\eqref{eq:poly-cartmap} induces a cartesian transformation, such a span of polynomials induces a pseudonatural transformation whose components are spans.
The double monad axioms tell us that this transformation is a pseudo-distributive law.

Concretely, we view $\Mv$ and $\Mh$ as ``horizontal'' and ``vertical'', with the morphisms of $\M_{00}$ being transversal arrows and the morphisms of $\M_{01}$ and $\M_{10}$ being horizontal and vertical cells respectively.
The objects of $\M_{11}$ are the ``basic cells''; their domains and codomains are ``formal composites'' of horizontal and vertical arrows, like so:
\begin{equation}\label{eq:dpcell}
\begin{tikzcd}
  & (p_{11},\dots,p_{nk_n}) \ar[drr,"{(\alpha_1,\dots,\alpha_n)}"]  \\
  (p'_{11},\dots,p_{m\ell_m}) \ar[ddr,swap,"{(\gamma_1,\dots,\gamma_m)}"] \ar[ur,"\cong"] & \twocell{rr}{dd}{} && (q_1,\dots,q_n) \ar[dd,"\delta"]\\
  \\
  & (r_1,\dots,r_m) \ar[rr,swap,"\beta"] && s
\end{tikzcd}
\end{equation}
Here $\delta:(q_1,\dots,q_n) \to s$ and each $\gamma_j : (p'_{j1},\dots,p'_{j\ell_j}) \to r_j$ are vertical arrows, while $\beta:(r_1,\dots,r_m)\to s$ and each $\alpha_i : (p_{i1},\dots,p_{ik_i})\to q_i$ are horizontal arrows.
We additionally have an \emph{isomorphism} rearranging the list $(p_{11},\dots,p_{nk_n})$ into the list $(p'_{11},\dots,p_{m\ell_m})$, coming from the pair of pullback squares above saying that $(\Mh\Mv)_2\to (\Mh\Mv)_1$ and $(\Mv\Mh)_2\to (\Mv\Mh)_1$ have the same fibers over $\M_{11}$.

The double monad structure says that these basic cells can be composed both horizontally and vertically, in a natural multicategorical way.
Furthermore we have cubes, which relate two such basic cells along a pair of formal composites of horizontal and vertical cells.
The spans in these formal composites must ``match up'' as if we were actually composing squares in an ordinary polynomial monad, and the composite spans must commute with the domain isomorphisms.

\begin{defn}
  For a double polynomial monad $\M$ as above, an \textbf{$\M$-polycategory} consists of:
  \begin{itemize}
  \item A discrete fibration $A_0 \to \M_0$,
  \item A discrete fibration $A_1 \to \Mh(A_0) \times_{M_0} \Mv(A_0)$,
  \item A composition map of spans
    \[
    \begin{tikzcd}[column sep=small]
      \Mh\Mh(A_0) \ar[d] & \Mh(A_1)\underset{\Mh\Mv(A_0)}{\times} \M_{11} \underset{\Mv\Mh(A_0)}\times \Mv(A_1) \ar[l] \ar[r] \ar[d] & \Mv\Mv(A_0) \ar[d]\\
      \Mh(A_0) & A_1 \ar[l] \ar[r] & \Mv(A_0)
    \end{tikzcd}
    \]
  \item A unit map of spans
    \[
    \begin{tikzcd}
      & A_0\ar[dl] \ar[d] \ar[dr] \\
      \Mh(A_0) & A_1 \ar[l] \ar[r] & \Mv(A_0)
    \end{tikzcd}
    \]
  \item Associativity and unit laws hold.
  \end{itemize}
  % It is a \textbf{local discrete fibration} if $A_0 \to \M_0$ is a discrete fibration and the span $\Mh(A_0) \leftarrow A_1 \to \Mv(A_0)$ is a two-sided discrete fibration.
\end{defn}

These should be the semantics for the type theory over a double polynomial monad as a mode theory.
Concretely, every object of $A_0$ is assigned a mode (an object of $\M_0$), in a discrete-fibrational way, and for any lists of objects $\vec{A}$ and $\vec{B}$ with modes $\vec{p}$ and $\vec{q}$ respectively, and any vertical mode morphism $\vec{p} \xto{\alpha} r$ and horizontal mode morphism $\vec{q} \xto{\beta} r$, we have a set of morphisms $\DD\alpha\beta(\vec{A},\vec{B})$.
These morphisms are acted on by the horizontal and vertical cells in $\M$ (the two-sided discrete fibration condition), and can be composed ``along a square'': for any square $e$ as in~\eqref{eq:dpcell}, and morphisms $f_i \in \DD{\phi_i}{\alpha_i}(\vec{A},\vec{B})$ and $g_j \in \DD{\gamma_j}{\psi_j}(\vec{B},\vec{C})$, we have a composite $\vec{g} \circ_e \vec{f} \in \DD{\delta\circ \vec{\phi}}{\beta\circ\vec{\psi}}(\vec{A},\vec{B})$.
We also have an identity morphism $1_A \in \DD{1_p^v}{1_p^h}(A,A)$ for any object $A$ with mode $p$, and identity and unit axioms saying that for $f\in\DD{\alpha}{\beta}(\vec{A},\vec{B})$, we have $f = f \circ_{1_\alpha} 1_{\vec{A}}$ and $f = 1_{\vec{B}} \circ_{1_\beta} f$, and that $(h\circ_{e_3}g)\circ_{e_2 \circ e_1} f = h\circ_{e_3\circ e_2} (g\circ_{e_1} f)$ where $e_1,e_2,e_3$ are squares that can be composed like this:
\[\begin{tikzcd}
  & . \ar[r] \ar[d]
  \ar[from=r,to=d,phantom,""{name=1,near start},""{name=2,near end}]
  \ar[Rightarrow,from=1,to=2,"e_1"]
  & .\ar[d] \\
  .\ar[r] \ar[d]
  \ar[from=r,to=d,phantom,""{name=3,near start},""{name=4,near end}]
  \ar[Rightarrow,from=3,to=4,"e_3"]
  & .\ar[r] \ar[d]
  \ar[from=r,to=d,phantom,""{name=5,near start},""{name=6,near end}]
  \ar[Rightarrow,from=5,to=6,"e_2"]
  & .\ar[d] \\
  .\ar[r] & .\ar[r] & .
\end{tikzcd}\]

The representable $\M$-polycategories (those having all $F$-functors) should correspond to \textbf{$\M$-algebras}, i.e.\ simultaneous $\Mh$- and $\Mv$-algebras together with 2-cells coming from the squares.

Note that unlike the situation with a profunctorial distributive law as in~\cite{garner:polycats}, the horizontal and vertical mode cells \emph{both} act contravariantly on the morphisms of an $\M$-polycategory: we have a fibration $A_1 \to \Mh(A_0) \times_{M_0} \Mv(A_0)$ instead of a two-sided fibration from one to the other.
This seems natural here because the morphisms in $\M_0$ must act contravariantly everywhere, and mixing the variance in $\Mh(A_0)$ would make the meaning of cubes rather opaque.

When written out explicitly, in the case when $\M_0$ is discrete, the above notions of double polynomial monad and generalized polycategory over them are almost the same as my previous idea using ``multi-double-categories''; the main difference is that now the structural transformations are ``club-like'' in the transversal direction, rather than using cartesian structure that acts directly on the morphisms and cells, and we have cubes to mediate their relationship to the basic cells.
The following examples can be mostly copied over from the previous suggestion.

\begin{eg}\label{eg:degenerate}
  If $\Mh$ is trivial (the identity monad), then we can take $\M_{11}$ to contain only identities, in which case an $\M$-polycategory reduces to an $\Mv$-multicategory as in \cref{sec:polynomials}.
\end{eg}

\begin{eg}\label{eg:linear}
  Suppose $\M$ has one object $p$ that is a commutative monoid both horizontally and vertically.
  Commutativity means that if $\mult:(p,p)\to p$ is either of the multiplications, we have a horizontal or vertical cell that is an isomorphism $\mult\cong \mult$ and whose arity span is the transposition $(p,p)\cong (p,p)$, satisfying the usual axioms of a symmetric monoidal category.

  Let $\mult^n : (p,p,\dots,p) \to p$ denote the $n$-fold multiplication.
  Then the cut rule of classical linear logic
  \[
  \inferrule{\Gamma\vdash \Delta,A \\ A,\Phi\vdash \Psi}{\Gamma,\Phi \vdash \Delta,\Psi}
  \]
  can be generated by squares
  \[\begin{tikzcd}[column sep=huge]
    p^{n+1+m} \ar[r,"{(\mult^{n+1},1,\dots,1)}"] \ar[d,swap,"{(1,\dots,1,\mult^{1+m})}"] \drtwocell{e_{n,m}}
    & p^{1+m} \ar[d,"{\mult^{1+m}}"]
    \\ p^{n+1} \ar[r,swap,"{\mult^{n+1}}"] & p
  \end{tikzcd}\]
  by composing with identities, $(1,\dots,1,g) \circ_{e_{n,m}} (f,1,\dots,1)$.
  In fact, I believe all the above squares can be generated from the case $n=m=1$ by composing it with itself (and identities) horizontally and vertically.
  We should assert that such self-composites are associative, and interchanging.
  Moreover, we should have cubes saying that these squares commute with the symmetric structure of the multiplications.

  Note that the generating square $e_{1.1}$ is a sort of ``linear distributivity''.
  In fact I suspect that an \emph{$\M$-algebra} is precisely a linearly distributive category.
  So just as we talk about modes in a 2-multicategory or a polynomial monad being monoids or commutative monoids, we can talk about a mode in a double polynomial monad being a ``linearly distributive monoid'', where the two monoidal structures involved in linear distributivity are horizontal and vertical.
  That is, the natural context in which to define a ``linearly distributive monoid'' is a double polynomial monad.
  This is similar to how bimonoids are naturally defined in duoidal categories (see \cref{eg:duoidal}) and Frobenius objects are naturally defined in linearly distributive categories.
\end{eg}

\begin{eg}\label{eg:nonlinear}
  If we enhance the vertical and horizontal monoid structures in the previous example to cartesian monoids, we should obtain some kind of classical nonlinear logic.
  The ``triple-categorical'' perspective allows the horizontal and vertical contraction and weakening cells to be essentially independent of the basic cells that describe the cut rule.
  However, if we want we can add in cubes that specify some kind of interaction.
\end{eg}

\begin{eg}\label{eg:dirtt}
  Suppose we have a set of ``categories'' and let the objects of $\M$ (modes) be ``signed lists'' of categories such as $(A,B\o,C)$.
  A unary horizontal mode morphism consists of a pairing up of categories with their opposites in the \emph{domain only} and a bijection between the remaining categories (with variance) and the codomain.
  For instance, we have such a morphism $(A,B\o,C\o,B) \to (C\o,A)$ that pairs up the $B$'s and switches the $A$ and the $C\o$.
  There are no non-unary horizontal morphisms (except for those generated by weakening).
  But a \emph{vertical} mode morphism can be of arbitrary arity, and consists of removing the parentheses from the domain and then performing such a pairing.
  Thus, a morphism in $\D$ is ``intuitionistic'', with multiple modules in the domain and only one in the codomain, and parametrized by pairing some categories in the domains and some in the codomain and matching up the rest.
  The simplest squares look like this:
  \[\begin{tikzcd}[column sep=huge]
    (A,A\o,A) \ar[r,"\varepsilon^{12}_A"] \ar[d,swap,"\varepsilon^{23}_A"] \drtwocell{e_A}
    & A \ar[d,"1_A"]
    \\ A \ar[r,swap,"1_A"] & A
  \end{tikzcd}
  \qquad
  \begin{tikzcd}[column sep=huge]
    (A\o,A,A\o) \ar[r,"\varepsilon^{12}_A"] \ar[d,swap,"\varepsilon^{23}_A"] \drtwocell{e_{A\o}}
    & A\o \ar[d,"1_{A\o}"]
    \\ A\o \ar[r,swap,"1_{A\o}"] & A\o
  \end{tikzcd}\]
  where $\varepsilon^{12}$ pairs up the first two categories and $\varepsilon^{23}$ the second two.
  We need various generalizations of these too; I'm not sure how many we can get by composition and how many we need to put in by hand.
  This gives us the cut rule for directed type theory (without functors).
\end{eg}

Note that if we mentally reverse the direction of horizontal mode morphisms, and regard a square as an ``equation'' between its top-left ``composite'' and bottom-right ``composite'', then the generating squares $e_{1,1}$ of \cref{eg:linear} become the Frobenius axiom of a Frobenius algebra, while those of \cref{eg:dirtt} become the triangle identities of a dual pair.

\begin{eg}\label{eg:prop}
  Let $\M$ be generated by a vertical commutative monoid and a horizontal commutative monoid, as before, together with the following squares:
  \begin{mathpar}
    \begin{tikzcd}
      (p,p) \ar[r,"\mult"] \ar[d,"\mult",swap] \drtwocell{e_\mult} & p \ar[d,"1"] \\ p\ar[r,"1",swap] & p
    \end{tikzcd}
    \and
    \begin{tikzcd}
      (p,p) \ar[r,"{(1,1)}"] \ar[d,"{(1,1)}",swap] \drtwocell{e'_\mult} & (p,p) \ar[d,"\mult"] \\ (p,p)\ar[r,"\mult",swap] & p
    \end{tikzcd}
    \and
    \begin{tikzcd}
      () \ar[r,"\eta"] \ar[d,"\eta",swap] \drtwocell{e_\eta} & p \ar[d,"1"] \\ p\ar[r,"1",swap] & p
    \end{tikzcd}
    \and
    \begin{tikzcd}
      () \ar[r,"()"] \ar[d,"()",swap] \drtwocell{e'_\eta} & () \ar[d,"\eta"] \\ () \ar[r,"\eta",swap] & p
    \end{tikzcd}
  \end{mathpar}
  making the horizontal and vertical monoid structures into double-categorical ``companions''.
  Then an $\M$-algebra is an object with two isomorphic commutative monoid structures, i.e.\ essentially only one such.

  An $\M$-polycategory, on the other hand, should be essentially a \emph{prop}.
  By composing the above generators we obtain corresponding squares for all $\mult^n$.
  The square $e_{\mult^n}$ then tells us that we can compose morphisms along $n$ objects at once, i.e.\ if $f\in\DD{\mult^m}{\mult^n}(\vec{A},\vec{B})$ and $g\in \DD{\mult^n}{\mult^k}(\vec{B},\vec{C})$ then $g\circ_{e_{\mult^n}} f \in \DD{\mult^m}{\mult^k}(\vec{A},\vec{C})$.
  And the square $e'_{\mult^n}$ gives us an ``$n$-ary identity'' $1_{\vec{A}} = \vec{1}_A \circ_{e'_{\mult^n}} \vec{1}_A\in\DD{\mult^n}{\mult^n}(\vec{A},\vec{A})$, which (by the companion axioms) acts as an identity for the $n$-ary composition.
\end{eg}

\begin{eg}\label{eg:duoidal}
  Let $\M$ be generated again by a vertical and a horizontal commutative monoid, but now the following generating squares:
  \begin{mathpar}
    \begin{tikzcd}
      (p,p,p,p) \ar[r,"{(1,\sigma,1)}"] \ar[d,"{(\mult,\mult)}",swap] \drrtwocell{e_{2,2}} &
      (p,p,p,p) \ar[r,"{(\mult,\mult)}"] & (p,p) \ar[d,"\mult"] \\
      (p,p) \ar[rr,"\mult",swap] & & p
    \end{tikzcd}
    \and
    \begin{tikzcd}
      () \ar[r,"()"] \ar[d,"()",swap] \drtwocell{e_{0,0}} & () \ar[d,"\eta"] \\ () \ar[r,"\eta",swap] & p
    \end{tikzcd}
    \and
    \begin{tikzcd}
      () \ar[r,"()"] \ar[d,"{(\eta,\eta)}",swap] \drtwocell{e_{0,2}} & () \ar[d,"\eta"] \\ (p,p) \ar[r,"\mult",swap] & p
    \end{tikzcd}
    \and
    \begin{tikzcd}
      () \ar[r,"{(\eta,\eta)}"] \ar[d,"()",swap] \drtwocell{e_{2,0}} & (p,p) \ar[d,"\mult"] \\ () \ar[r,"\eta",swap] & p
    \end{tikzcd}
  \end{mathpar}
  With appropriate axioms imposed, an $\M$-algebra is now a \emph{duoidal category}: a category with two monoidal structures $(\diamond,I)$ and $(\star,J)$ such that $(\star,J)$ is lax monoidal with respect to $(\diamond,I)$ (or equivalently the dual).
  An $\M$-polycategory is thus a structure that ``is to a duoidal category what a polycategory is to a linearly distributive category'': it is the composition structure induced on the sets of morphisms $A_1 \star \cdots \star A_m \to B_1\diamond \cdots \diamond B_n$ in a duoidal category.
\end{eg}

\begin{rmk}
  We could perform essentially the same constructions using cartesian clubs directly instead of polynomial monads, and lose little generality in the horizontal and vertical parts.
  This is because, as remarked above, contraction and weakening are much more common than their duals, and unlike in the profunctorial case of~\cite{garner:polycats}, our horizontal and vertical cells \emph{both} act contravariantly on polycategory morphisms.
  However, the extra generality of polynomial monads is useful even when all our monads are cartesian clubs, because not every morphism of such monads is a morphism of cartesian clubs, i.e.\ it may not respect the splitting of the fibrations.
  In particular, if we did the quintet construction for clubs instead of polynomials, the permutation in the domains of a basic cell would have to be an identity.
  This might not be a problem in some cases like \cref{eg:linear}, where we can always permute the domains or codomains of the inputs to a composition; but it would rule out the first generating square of \cref{eg:duoidal}, where the permutation on both sides ``mixes'' the modes in the domains of two morphisms.
\end{rmk}


\section{Type theory for generalized polycategories}
\label{sec:syntax}

Syntactically, the judgments of the type theory for $\M$-polycategories will look something like $\Gamma \vdash_{\alpha\tot\beta} \Delta$, where $\alpha$ is a term describing a vertical mode morphism using variables from $\Gamma$ and $\beta$ is a term describing a horizontal mode morphism using variables from $\Delta$.
In \cref{eg:degenerate}, $\beta$ is trivial and this reduces to the previous syntax.
In \cref{eg:linear}, with $\mult$ written as $\otimes$, the sequents we are interested in look like
\[ x:A, y:B, z:C \vdash_{x\otimes y\otimes z \tot u\otimes v\otimes w} u:D, v:E, w:F \]
As before, to deal with FOL and DirTT it may make more sense to have types dependent on modes rather than duplicating variables:
\[ x:p, y:q \mid A(x), B(y) \vdash_{x\otimes y \tot u\otimes v} D(u), E(v) \mid u:r, v:s \]
This allows the variables to belong to sorts, as before, with the modes being contexts of sorts (perhaps with a pre-mode too).
For instance, in \cref{eg:dirtt} with the ``pairing'' morphisms written as $x:A, y:A\o \vdash \varepsilon(x,y) : ()$, we have extranatural judgments like
\[ x:A, y:B, z:B\o \mid M(x,y),N(z) \vdash_{(x\mid \varepsilon(y,z)) \tot (w\mid\varepsilon(u,v))} P(u,v,w) \mid u:C, v:C\o, w:A \]
which are a somewhat more verbose notational variant of the ``string diagrams'' syntax for directed type theory.

The cut rule is parametrized by basic mode cells.
For instance, in \cref{eg:linear} the generating square $e_{1,1}$ induces the following cut rule
\begin{mathpar}
  \inferrule{\Gamma\vdash_{\mult^\Gamma \tot \mult(x,y)} x:B,y:A \\ \Delta \vdash_{\mult^\Delta \tot z} z:C \\\\ u:A,v:C\vdash_{\mult(u, v) \tot \mult^\Psi} \Psi \\ w:B \vdash_{w\tot \mult^\Phi} \Phi}{\Gamma,\Delta \vdash_{\mult(\mult^\Gamma,\mult^\Delta) \tot \mult(\mult^\Phi,\mult^\Psi)} \Phi,\Psi}
\end{mathpar}
And in \cref{eg:dirtt} the generating square $e_A$ induces the following cut rule:
\begin{mathpar}
  \inferrule{\Gamma \vdash_{\alpha \tot (z\mid \epsilon(y,x))} M(x,y,z) \mid x:A, y:A\o, z:A \\
    u:A, v:A\o, w:A \mid M(u,v,w) \vdash_{(u\mid \epsilon(v,w)) \tot \beta} \Delta}
  {\Gamma \vdash_{\alpha \tot \beta} \Delta}
\end{mathpar}

We should also have a substitution action of transversal mode morphisms on the types, and likewise an action of the horizontal and vertical mode cells on the above judgments.
The commutativity of this action with the cut (mediated by basic cells) should be specified by mode cubes.


\bibliography{../all.bib}
\bibliographystyle{alpha}

\end{document}
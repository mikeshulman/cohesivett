\documentclass[10pt]{article}
  \usepackage{xcolor}
  \definecolor{darkgreen}{rgb}{0,0.45,0} 
  \usepackage[pagebackref,colorlinks,citecolor=darkgreen,linkcolor=darkgreen]{hyperref}
  \usepackage{pdflscape}

\usepackage[sort]{natbib}
  
\usepackage{fullpage}
\usepackage{amssymb,amsthm,bbm}
\usepackage[centertags]{amsmath}
\usepackage[mathscr]{euscript}
\usepackage{dsfont}
\usepackage{fontawesome}
\usepackage{tikz-cd}
\usepackage{mathpartir}
\usepackage{enumitem}
\usepackage[status=draft,inline,nomargin]{fixme}
\FXRegisterAuthor{ms}{anms}{\color{blue}MS}
\FXRegisterAuthor{mvr}{anmvr}{\color{olive}MVR}
\FXRegisterAuthor{drl}{andrl}{\color{purple}DRL}
\usepackage{stmaryrd}
\usepackage{mathtools}

\newtheorem{theorem}{Theorem}
\newtheorem{proposition}{Proposition}
\newtheorem{lemma}{Lemma}
\newtheorem{corollary}{Corollary}
\newtheorem{problem}{Problem}
\newenvironment{constr}{\begin{proof}[Construction]}{\end{proof}}

\theoremstyle{definition}
\newtheorem{definition}{Definition}
\newtheorem{remark}{Remark}
\newtheorem{example}{Example}

\let\oldemptyset\emptyset%
\let\emptyset\varnothing

\newcommand\dsd[1]{\ensuremath{\mathsf{#1}}}

\newcommand{\yields}{\vdash}
\newcommand{\Yields}{\tcell}
\newcommand{\tcell}{\Rightarrow}
\newcommand{\cbar}{\, | \,}
\newcommand{\judge}{\mathcal{J}}

\newcommand{\Id}[3]{\mathsf{Id}_{{#1}}(#2,#3)}
\newcommand{\CTX}{\,\,\mathsf{Ctx}}
\newcommand{\ctx}{\,\,\mathsf{mctx}}
\newcommand{\TYPE}{\,\,\mathsf{Type}}
\newcommand{\type}{\,\,\mathsf{mode}}
\newcommand{\TELE}{\,\,\mathsf{Tele}}
\newcommand{\tele}{\,\,\mathsf{mtele}}
\newcommand{\ISFIB}{\,\,\mathsf{IsFib}}

\newcommand{\app}[2]{\ensuremath{#1 \: #2}}
\newcommand{\telety}[3]{\ensuremath{(#1{:}#2,#3)}}
\newcommand{\mt}[0]{\ensuremath{()}}
\newcommand{\sigmacl}[3]{\ensuremath{(#1{:}#2,#3)}}
\newcommand{\fst}[1]{\app{\dsd{fst}}{#1}}
\newcommand{\snd}[1]{\app{\dsd{snd}}{#1}}
\newcommand\extend[2]{\ensuremath{(#1,\id_{#2})}}
\newcommand\TeleE[4]{\ensuremath{\mathsf{let} \, (#2, #3) \, = \, {#1} \, \mathsf{in} \, #4}}

\newcommand{\id}{\mathsf{id}}
\DeclareMathOperator{\ob}{ob}

\newcommand{\rewrite}[2]{\overleftarrow{#1}(#2)}
\newcommand\Fsym{\ensuremath{\mathsf{F}}}
\newcommand\Usym{\ensuremath{\mathsf{U}}}
\newcommand\Esym{\ensuremath{\mathsf{E}}}
\newcommand\F[2]{\ensuremath{\mathsf{F}_{#1}(#2)}}
\newcommand\E[2]{\ensuremath{\mathsf{E}_{#1}(#2)}}
\newcommand\U[3]{\ensuremath{\mathsf{U}_{#1}(#2 \mid #3)}}
\newcommand\UE[2]{\ensuremath{#1(#2)}}
\newcommand\UI[2]{\ensuremath{\lambda #1.#2}}
\newcommand\St[2]{\ensuremath{{#1}^*(#2)}}
\newcommand\StI[2]{\ensuremath{\mathsf{st}_{#1}(#2)}}
\newcommand\UStI[2]{\ensuremath{\mathsf{ust}_{#1}(#2)}}
\newcommand\UnSt[2]{\ensuremath{\mathsf{unst}_{#1}(#2)}}
%\newcommand\StE[2]{\ensuremath{\mathsf{unst}(#1,#2)}}
\newcommand\StE[4]{\ensuremath{\mathsf{let} \, \StI{#1}{#3} \, = \, {#2} \, \mathsf{in} \, #4}}
\newcommand\FE[3]{\ensuremath{\mathsf{let} \, \mathsf{F}(#2) \, = \, {#1} \, \mathsf{in} \, #3}}
% With subscript:
\newcommand\FEs[4]{\ensuremath{\mathsf{let} \, \mathsf{F}_{#1}(#3) \, = \, {#2} \, \mathsf{in} \, #4}} 
\newcommand\FI[1]{\ensuremath{\mathsf{F}{(#1)}}}
\newcommand\FIs[2]{\ensuremath{\mathsf{F}_{#1}{(#2)}}}
\newcommand\EE[3]{\ensuremath{\mathsf{let} \, \mathsf{E}(#2) \, = \, {#1} \, \mathsf{in} \, #3}}
% With subscript:
\newcommand\EEs[4]{\ensuremath{\mathsf{let} \, \mathsf{E}_{#1}(#3) \, = \, {#2} \, \mathsf{in} \, #4}} 
\newcommand\EI[1]{\ensuremath{\mathsf{E}{(#1)}}}
\newcommand\EIs[2]{\ensuremath{\mathsf{E}_{#1}{(#2)}}}
\newcommand\TypeTwo[4]{\ensuremath{#1 \vdash #2 :  #3 \tcell #4}}
\newcommand\TeleTwo[4]{\ensuremath{#1 \vdash #2 : #3 \tcell #4}}
\newcommand\TermTwo[4]{\ensuremath{#1 \vdash #2 : #3 \tcell #4}}
\newcommand\TermTwoT[5]{\ensuremath{#1 \vdash {#2} : #3 \tcell_{#5} #4}}
%% \newcommand\TermTwoDisp[5]{\ensuremath{#1 \mid #3 \tcell_{\mathsf{disp}} #2 :_{#5} #4}}
%\newcommand\SubTwo[4]{\ensuremath{#1 \mid #3 \tcell #2 : #4}}
\newcommand\TrPlus[2]{\ensuremath{{#1}^+(#2)}}
\newcommand\TrCirc[2]{\ensuremath{{#1}^\circ(#2)}}

\newcommand\El[2]{\mathcal{T}_{#1}(#2)}
\newcommand\ApEl[2]{\mathcal{T}_{#1}\langle#2\rangle}
\newcommand\bdot[0]{\mathbin{.}}
\newcommand\bang[0]{\mathord{!}}

\newcommand\ap[2]{\ensuremath{#1 \langle #2 \rangle }}
\newcommand\ApPlus[2]{\ensuremath{{#1}^+ \langle #2 \rangle }}
\newcommand\ApCirc[2]{\ensuremath{{#1}^\circ \langle #2 \rangle }}

% Lemmas
\newcommand\ctxtuple[1]{(#1)}
\newcommand\pack[1]{\ensuremath{\mathsf{pack}_{#1}}}
\newcommand\unpack[2]{\ensuremath{\mathsf{unpack}_{#1}(#2)}}

% MLTT
\newcommand{\modeof}[1]{{#1}_p}
\newcommand{\modeofq}[1]{{#1}_q}
\newcommand{\tdot}{\ensuremath{\mathtt{dot}}}
\newcommand{\tempty}{\ensuremath{\mathtt{empty}}}
\newcommand{\tshape}[1]{\ensuremath{\mathtt{shape}_{#1}}}

\newcommand{\qyields}{\Vdash} 
\newcommand{\upstairs}[1]{\overline{#1}}
\newcommand{\downstairs}[1]{\underline{#1}}
\newcommand\proj[1]{\ensuremath{\mathsf{proj}_{#1}}}
\newcommand\qvar[1]{\ensuremath{\mathsf{var}_{#1}}}

\newcommand\One{\ensuremath{\mathds{1}}}
\newcommand\var[1]{\ensuremath{\mathtt{var}_{#1}}}
\newcommand\ApOne[1]{\ensuremath{\One_{\langle {#1} \rangle }}}

\newcommand\mtt[1]{\mathtt{#1}}
\newcommand\contract[1]{\ensuremath{\mathtt{contract}_{#1}}}
\newcommand\fibpair[1]{\ensuremath{\mathtt{fibpair}_{#1}}}
\newcommand\pair[1]{\ensuremath{\mathtt{pair}_{#1}}}
\newcommand\tsplit[1]{\ensuremath{\mathtt{split}_{#1}}}
\newcommand\pinv[1]{\ensuremath{\mathtt{pinv}_{#1}}}

\newcommand\qunitmatch[1]{\ensuremath{\mathsf{letunit}(#1)}}
\newcommand\qpair[1]{\ensuremath{\mathsf{pair}_{#1}}}
\newcommand\qsplit[1]{\ensuremath{\mathsf{split}_{#1}}}
\newcommand\qapp[1]{\ensuremath{\mathsf{app}({#1})}}
\newcommand\qlam[1]{\ensuremath{\mathsf{lam}({#1})}}

% Adjoint type theory
\newcommand\fone[1]{\ensuremath{\mathtt{fone}_{#1}}}
\newcommand\fibf[1]{\ensuremath{\mathtt{fibf}_{#1}}}
\newcommand\foneinv[1]{\ensuremath{\fone{#1}^{-1}}}
\newcommand\fdist[1]{\ensuremath{\mathtt{fdist}_{#1}}}
\newcommand\fdistinv[1]{\ensuremath{\fdist{#1}^{-1}}}

\newcommand\flatone[1]{\ensuremath{\mathtt{flatone}_{#1}}}
\newcommand\flatdist[1]{\ensuremath{\mathtt{flatdist}_{#1}}}
\newcommand\flatdistinv[1]{\ensuremath{\flatdist{#1}^{-1}}}

\newcommand{\lock}{\text{\faUnlock}}
\newcommand{\Rtype}[1]{\mathsf{R}{#1}}
\newcommand{\RI}[1]{\mathsf{shut}({#1})}
\newcommand{\RE}[1]{\mathsf{open}({#1})}

\newcommand{\Ltype}[1]{\mathsf{L}{#1}}
\newcommand{\LI}[1]{\mathsf{left}_{#1}}
\newcommand{\LE}[1]{\mathsf{letleft}({#1})}

% Spatial type theory
\newcommand\fcomult[1]{\ensuremath{\mathtt{comult}_{#1}}}
\newcommand\fcounit[1]{\ensuremath{\mathtt{counit}_{#1}}}
\newcommand{\counit}[1]{\mathsf{counit}_{#1}}
\newcommand{\comult}[1]{\mathsf{comult}_{#1}}
\newcommand{\Flattype}[1]{\flat{#1}}
\newcommand{\FlatI}[1]{{#1}^\flat}
\newcommand{\FlatE}[1]{\mathsf{letflat}({#1})}
\newcommand{\Sharptype}[1]{\sharp{#1}}
\newcommand{\SharpI}[1]{{#1}^\sharp}
\newcommand{\SharpE}[1]{{#1}_\sharp}
\newcommand\qcrispvar[1]{\ensuremath{\textsf{crisp-var}_{#1}}}

% Linear zone
\newcommand{\tfibshape}[1]{\ensuremath{\mathtt{fibshape}_{#1}}}
\newcommand{\linsnd}[1]{\mathtt{linsnd}_{#1}}
\newcommand{\linwk}[1]{\mathtt{linwk}_{#1}}
\newcommand{\frob}[1]{\mathtt{frob}_{#1}}
\newcommand\qlinvar[1]{\ensuremath{\mathsf{linvar}_{#1}}}
\newcommand\otimespair[1]{\ensuremath{\otimes\mathsf{pair}_{#1}}}
\newcommand\otimessplit[1]{\ensuremath{\otimes\mathsf{split}({#1})}}
\newcommand\linpair[1]{\ensuremath{\mathsf{linpair}_{#1}}}
\newcommand\linsplit[1]{\ensuremath{\mathsf{linsplit}({#1})}}
\newcommand\linapp[1]{\ensuremath{\mathsf{linapp}_{#1}}}
\newcommand\linlam{\ensuremath{\mathsf{linlam}}}
\newcommand{\qbang}[1]{\ensuremath{\mathsf{bang}_{#1}}}
\newcommand{\letbang}[1]{\mathsf{letbang}({#1})}

% Macros for semantics notation
\newcommand\mm[1]{\llbracket #1 \rrbracket}
\newcommand\op{^{\mathrm{op}}}
\newcommand\co{^{\mathrm{co}}}
\newcommand\coop{^{\mathrm{coop}}}
\newcommand\Cat{\mathrm{Cat}}
\newcommand\CAT{\mathrm{CAT}}
\newcommand\M{\mathcal{M}}
\newcommand\Mhat{\widehat{\mathcal{M}}}
\newcommand\Mty{{\mathrm{Ty}_{\M}}}
\newcommand\Mtm{{\mathrm{Tm}_{\M}}}
\newcommand\Mtyhat{{\widehat{\mathrm{Ty}}_{\M}}}
\newcommand\Mtmhat{{\widehat{\mathrm{Tm}}_{\M}}}
\newcommand\Ups{\Upsilon}
\newcommand\Upshat{{\widehat{\Upsilon}}}
\newcommand\C{\mathcal{C}}
\newcommand\Chat{{\widehat{\mathcal{C}}}}
\newcommand\Cty{\mathrm{Ty}_{\C}}
\newcommand\Ctm{\mathrm{Tm}_{\C}}
\newcommand\Ctyhat{{\widehat{\mathrm{Ty}}}_{\C}}
\newcommand\Ctmhat{{\widehat{\mathrm{Tm}}}_{\C}}
\newcommand\vp{\varpi}
\newcommand\vpst{\vp^*}
\newcommand\vpsh{\vp_!}
\newcommand\vptil{\widetilde{\vp}}
\newcommand\vpty{{\vp}_{\mathrm{Ty}}}
\newcommand\vptm{{\vp}_{\mathrm{Tm}}}
\newcommand\name[1]{\ulcorner #1\urcorner}
\newcommand{\Util}{\widetilde{U}}
\newcommand\ev{\mathrm{ev}}
\DeclareSymbolFont{bbold}{U}{bbold}{m}{n}
\DeclareSymbolFontAlphabet{\mathbbb}{bbold}
\newcommand\one{\mathbbb{1}}


\defcitealias{lsr17multi-extended}{LSR}	
\defcitealias{ls16adjoint-extended}{LS}	

\title{A Fibrational Framework for \\ Substructural and Modal Dependent Type Theories}
\author{Daniel R. Licata, Mitchell Riley, Michael Shulman}
\date{}

\begin{document}
\maketitle

\begin{abstract}
Several recent modal extensions of homotopy type theory extend the
synthetic style of formalizing mathematics to additional situations.
For example, real-cohesive homotopy type theory can describe types with
both a groupoid structure and a separate topological structure.  These
modal dependent type theories add new types to the syntax, which
typically are given universal properties relative to new judgement
forms.  To facilitate the design of such type theories, we introduce a
general framework for modal dependent type theories.  To describe a
particular modal type theory, the first step is to specify a signature
of desired modalities using a \emph{base} directed dependent type
theory; we call such a signature a \emph{mode theory}.  Then,
instantiating a \emph{top} type theory with a given a mode theory gives
rules for working with the modalities it describes.  As examples, we
give mode theories for adjunctions, monads, comonads, and idempotent
(co)monads, in the context of a standard structural dependent type
theory, as well as a linear logic dependent on a structural one.  While
the framework does not automatically produce ``optimized'' inference
rules for a particular modal discipline (where structural rules are as
admissible as possible), it does provide a convenient syntactic setting
for investigating such issues, including a general equational theory
governing the placement of structural rules in types and in terms.  We
show that the top type theory over the above example mode theories
encode the expected rules.  Finally, we give the framework a categorical
semantics for all mode theories at once, which saves some of the effort
involved in translating each type theory individually.
%% as examples, we give mode theories for ordinary
%% non-modal dependent type theory with $\Pi$ and $\Sigma$ types, for a
%% dependent adjoint pair of modalities, and for the spatial type theory
%% used in real-cohesion.
\end{abstract}

\tableofcontents

\section{Introduction}

%% cites: Awodey and Bauer, Nuyts bridge-path, Vakar and Krishnaswami

%% fibrational: base type theory for specifying mode theories.  top type
%% theory is parametrized by a particular mode theory, gives general rules
%% that can be instantiated.  

%% Framework = both base and top

%% Mode theory = a particular signature in the base

%% In previous
%% work~\citep*{ls16adjoint,ls16adjoint-extended,lsr17multi,lsr17multi-extended},
%% the mode theory was a 2-category, or cartesian 2-multicategory.
%% Objects/types of the mode theory represent ``modes of truth,'' or
%% categories of types.  Morphisms/terms of the mode theory generate type
%% constructors, e.g. modalities.  The further 2-cell data of the mode
%% theory corresponds to structural rules, such as weakening, exchange,
%% contraction for a product, or a (co)unit or (co)multiplication of a
%% modality.  Just as a 2-category or 2-multicategory is a directed
%% \emph{simple} type theory, here we use a directed \emph{dependent} type
%% theory as the mode theory.

%% choose your own adventure organization:
%% proof theorist, Section~\ref{sec:base-syntax} and \ref{sec:top-syntax};
%% category theoriest, Section~\ref{sec:semantics},
%% examples Sections \ref{sec:mode-examples}, \ref{sec:example-encodings}.


\section{Base Type Theory}
\label{sec:base-syntax}

A \emph{mode theory} describes the contexts and types of a particular
modal dependent type theory.  Mode theories are given by a signature of
constants and equations in a directed dependent type theory, which,
because of the fibrational perpective, we call the \emph{base} type
theory.  We refer to the types of the base type theory as \emph{mode
  types} or just \emph{modes}, and terms as \emph{mode terms}, to
differentiate them from the types and terms of the top type theory.

The base type theory consists of five judgements:
\begin{itemize}
\item $\gamma \ctx$ are mode contexts
\item $\gamma \yields p \type$ are mode types
\item $\gamma \yields \mu : p$ are mode terms
\item $\TypeTwo{\gamma}{s}{p}{q}$ are \emph{mode type morphisms}
\item $\TermTwoT{\gamma}{s}{\mu}{\nu}{p}$ are \emph{mode 2-cells}
\end{itemize}
as well as corresponding judgemental equality judgements for each.
The first three judgements have the familiar structure of a dependent
type theory, while the fourth and fifth add an additional notion of
morphism between types and morphism between terms.  The mode type
morphisms can be thought of as a special class of terms, as explained
below.  A \emph{mode theory} is a signature of constants for mode types,
mode terms, mode type morphisms, and mode 2-cells.

The mode theory can be thought of as a certain kind of ``2-category with
dependent types,'' generalizing the base type theories of our previous
work~\citepalias{ls16adjoint-extended,lsr17multi-extended}, which were 2
(multi-)categories.  We describe the syntax of the base type theory in
this section, though it may be more illuminating to read the example
mode theories in Section~\ref{sec:mode-examples} in parallel with the formalism.

\subsection{Mode Contexts}
Mode contexts are, as usual, given by iteratively adding variables of
mode types:
\begin{mathpar}
  \inferrule*{ }
             {\cdot \ctx}
             
  \inferrule*
    {\gamma \ctx \\
     \gamma \yields p \type}
    {\gamma,x:p \ctx}
\end{mathpar}  

\subsection{Mode Terms}

In addition to the constants introduced by specific mode theories, there
are two ways of making mode terms: 

\begin{mathpar}
\inferrule*{ }
             {\gamma,x : p, \gamma' \yields x : p}
             
\inferrule*
    {\gamma \yields \mu : q \\
     \TypeTwo{\gamma}{s}{p}{q}
    }
    {\gamma \yields \TrPlus{s}{\mu} : p}

\TrPlus{\id}{\mu} \equiv \mu \qquad
\TrPlus{s'}{\TrPlus{s}{\mu}} \equiv \TrPlus{(s';s)}{\mu} 
\end{mathpar}

The first is the usual variable rule.  The second says a mode type
morphism $\TypeTwo{\gamma}{s}{p}{q}$ acts contravariantly on mode terms,
inducing a ``function'' $q \to p$ (since we do not have function types
in the base type theory, this means a term with a free variable $x : q
\vdash \TrPlus{s}{x} : p$), and that this action is functorial in
identity and composition of mode type morphisms (defined next).

\subsection{Mode type morphisms}

In addition to the constants of a specific mode theory, mode type
morphisms are given by identity, composition, and whiskering
(``$\mathsf{ap}$'') of a mode type on a mode term 2-cell (which are
defined below):

\begin{mathpar}
    \inferrule*{ }
          {\TypeTwo{\gamma}{\id_p}{p}{p}}
    \qquad
    \inferrule*{{\TypeTwo{\gamma}{s_1}{p_1}{p_2}} \\
                {\TypeTwo{\gamma}{s_2}{p_2}{p_3}}
          }
          {\TypeTwo{\gamma}{s_1;s_2}{p_1}{p_3}}

\inferrule*{{\gamma,x:p} \vdash {q} \type \\
            \TermTwoT{\gamma}{t}{\mu}{\mu'}{p}\\
           } 
           {\TypeTwo{\gamma}{\ap {q} {t/x}}{q[\mu/x]}{q[\mu'/x]}}

\\
\id;s \equiv s \equiv s;\id \and
(s;s');s'' \equiv s;(s';s'') \\ 
\ap q {\id_{\mu}/x} \equiv \id_{q[\mu/x]} \and
\ap q {(s;t)/x} \equiv \ap q {s/x}; \ap q {t/x} \and
\ap q {s/\_} \equiv \id_q \\ 
\ap {(q[\mu/x])} {s/y} \equiv \ap q {\ap \mu {s/y}/x} \quad \text{where } \gamma,y:p' \vdash \mu : p \text{ and } \gamma,x:p \vdash q \type\\
t[\nu/x];\ap{q'}{s/x} \equiv \ap{q}{s/x};t[\nu'/x] \quad 
\text{where } \TypeTwo{\gamma,x:p}{t}{q}{q'} \text{ and } \TermTwoT{\gamma}{s}{\nu}{\nu'}{p}
%% subst: \id_\mu[\nu/x] = \id_{\mu[\nu/x]}
%% subst: s[x/x] = s
%% subst: (s;t)[\mu/x] = s[\mu/x];t[\mu/x]
%% subst: s[\mu[\nu/x]/x] = s[\mu/x][\nu/x]
%% subst: ap q (s [\mu/x]) = (ap q s)[\mu/x] and generalization
\end{mathpar}

Most of the equations are standard identity/associativity/projection
laws.  The first two say that composition is unital and associative.
The next two say that whiskering is functorial in the identity and
composition of mode type 2-cells (defined below). The next says that
whiskering with a constant function (i.e. when $x$ does not occur in
$q$) is the identity mode type morphism.  The next says that whiskering
a type given by substitution associates with the whiskering of a mode
term 2-cell by a term (defined below).  The final equation is a
naturality law saying that the two possible definitions of a horizontal
composition $\ap t {s/x} : q[\nu/x] \tcell q'[\nu'/x]$ are equal.
We will use the following notation for this:
\begin{mathpar}
  \inferrule*[Left=Derivable]
      {\TermTwoT{\gamma}{s}{\mu}{\mu'}{p} \\
        \TypeTwo{\gamma, x : p}{t}{q}{q'}}
      {\TypeTwo{\gamma}{\ap{t}{s/x} :\equiv t[\nu/x];\ap{q'}{s/x}}{q[\mu/x]}{q'[\mu'/x]}}
\\ 
\ap{\id_q}{s/x} \equiv \ap{q}{s/x} \and \ap{t}{\id_{\mu}/x} \equiv t[\mu/x]
\end{mathpar}

We sometimes write \ap{q}{s} for \ap{q(x)}{s/x}, eliding the variable
name when it is clear how to view $q$ as a term with a distinguished
variable.

\subsection{Mode term 2-cells}

Similarly, mode term 2-cells are given by identity, composition, and
post-whiskering (pre-whiskering is given by substitution), and
associated equations:
\begin{mathpar}
    \inferrule*{ }
          {\TermTwoT{\gamma}{\id_\mu}{\mu}{\mu}{p}}
    \quad
    \inferrule*{{\TermTwoT{\gamma}{s_1}{\mu_1}{\mu_2}{p}} \\
                {\TermTwoT{\gamma}{s_2}{\mu_2}{\mu_3}{p}}
          }
   {\TermTwoT{\gamma}{s_1;s_2}{\mu_1}{\mu_3}{p}}
\quad
\inferrule*{{\gamma,x:p} \yields {\nu} : {q} \\
            \TermTwoT{\gamma}{s}{\mu}{\mu'}{p}\\
           } 
           {\TermTwoT{\gamma}{\ap \nu {s/x}}{\nu[\mu/x]}{\TrPlus{\ap{q}{s/x}}{\nu[\mu'/x]}}{q[\mu/x]}}

\\           
\id;s \equiv s \equiv s;\id \and
(s;s');s'' \equiv s;(s';s'') \\ 
\ap \nu {\id_{\mu}/x} \equiv \id_{\nu[\mu/x]} \and
\ap \nu {(s;t)/x} \equiv \ap \nu {s/x} ; (\ap {(\TrPlus{\ap{q}{s/x}}{y})} {\ap \nu {t/x}/y}) \\ 
\ap x {s/x} \equiv s  \\ 
\ap {(\nu[\mu/x])} {s/y} \equiv \ap \nu {\ap \mu {s/y}/x} \quad
\text{where } \gamma,y:p' \vdash \mu : p \text{ and } \gamma,x:p \vdash \nu : q\\
\ap \nu {s/\_} \equiv \id_\nu \\
t[\mu/x];\ap{\nu'}{s/x} \equiv \ap{\nu}{s/x};\ap{(\TrPlus{\ap{q}{s/x}}{y})}{t[\mu'/x]/y} \quad
 \text{where } \TermTwoT{\gamma,x:p}{t}{\nu}{\nu'}{q} \text{ and } \TermTwoT{\gamma}{s}{\mu}{\mu'}{p} \\
\ap{(\TrPlus{s}{\mu})}{t/x} \equiv \ApPlus{(s[\nu/x])}{\ap{\mu}{t/x}}\quad 
\text{where } \TypeTwo{\gamma,x:p}{s}{q}{q'} \text{ and }
\TermTwoT{\gamma}{t}{\nu}{\nu'}{p} \text{ and } \gamma,x:p \vdash \mu : q'
\end{mathpar}

Identity and composition are standard.  When $q$ does not depend on $x$,
whiskering the ``function'' $x : p \vdash \nu : q$ onto the mode term
2-cell $s : \mu \tcell_p \mu'$ between two mode terms $\mu,\mu'$ of
mode $p$ gives a term 2-cell $\nu[\mu/x] \tcell_q \nu[\mu'/x]$.
However, in general $q$ might depend on $x$, in which case we have a
``dependent $\mathsf{ap}$'', which gives a mode term in $q[\mu/x]$
between $\nu[\mu/x]$ and the ``transport'' of $\nu[\mu'/x]$ along the
mode type morphism $\ap q {s/x} : q[\mu/x] \tcell q[\mu'/x]$ (in
the non-dependent case, the above equations say that $\ap q {s/x} \equiv
\id_q$, and that $\TrPlus{\id_q}{\nu[\mu'/x]} \equiv {\nu[\mu'/x]}$, so
the ``transport'' cancels).

The first two equations say that composition is associatve and unital.
The next two that whiskering is functorial in mode term 2-cell identity
and composition; the composition equation is the usual ``path over''
composition using whiskering, and both equations type check because of
the functoriality equations for $\TrPlus{s}{-}$.  The next two equations
say that whiskering is functorial in the function position: whiskering
with the identity function is the identity, and whiskering with a
composition is iterated whiskering.  The next equation says that
whiskering with a constant function ($x$ does not occur in $\nu$) is the
identity.

The next equation equates the two potential definitions of horizontal
composition $\ap t {s/x} : \nu[\mu/x] \tcell_{q[\mu/x]}
\TrPlus{\ap{q}{s/x}}{\nu'[\mu'/x]}$.  We write
\begin{mathpar}
  \inferrule*[Left=Derivable]
      {\TermTwoT{\gamma}{s}{\mu}{\mu'}{p} \\
    \TermTwoT{\gamma, x : p}{t}{\nu}{\nu'}{q}}
             {\TermTwoT{\gamma}{\ap{t}{s/x} :\equiv t[\mu/x];\ap{\nu'}{s/x}}{\nu[\mu/x]}{\TrPlus{\ap{q}{s/x}}{\nu'[\mu'/x]}}{q[\mu/x]}}
\\ 
\ap{\id_\nu}{s/x} \equiv \ap{\nu}{s/x} \and \ap{t}{\id_{\mu}/x} \equiv t[\mu/x]
\end{mathpar}
for this.

The final equation distributes the $\mathsf{ap}$ of a ``transport'' 
$\TrPlus{s}{\mu}$ into two $\mathsf{ap}$'s, and type checks because 
$s[\nu/x];\ap{q'}{t/x} \equiv \ap{q}{t/x};s[\nu'/x]$.  

We sometimes write \ap{\mu}{s} for \ap{\mu(x)}{s/x}, eliding the
variable name when it is clear how to view $\mu$ as a term with a
distinguished variable; e.g. $\ApPlus{s}{t}$ for
$\ap{\TrPlus{s}{x}}{t/x}$.  

\subsection{Mode Unit Types}

It will be useful to have the unit mode type in the base type theory:

\begin{mathpar}
  \inferrule*{ } { \gamma \yields 1 \type } \and
  
  \inferrule*{ }
             {\gamma \yields \mt : 1}
  \and 
  \mu \equiv \mt
  \and
s \equiv \id_{()} \text{ for } \yields s : () \tcell_1 ()
\end{mathpar}

\subsection{Mode $\Sigma$ types}
\label{sec:base-telescopes}

It will also be convenient to have $\Sigma$ mode types.  We write these
as telescopes $\sigmacl{x}{p}{q}$ to distinguish them from the $\Sigma$
types of an object language.  At the mode type and term level, the rules
are standard:
\begin{mathpar}
  \inferrule*{ \gamma \yields p \type \\ 
               \gamma,x:p \yields q \type }
             {\gamma \yields \sigmacl{x}{p}{q} \type} \\
             
\\
\inferrule*{
  \gamma \yields \mu : p \and
  \gamma \yields \nu : q[\mu/x]
    }
   {\gamma \yields (\mu,\nu) : \sigmacl{x}{p}{q}}
\and
\inferrule*
    {\gamma \yields \mu : \sigmacl{x}{p}{q}}
    {\gamma \yields \fst \mu : p}
\and
\inferrule*
    {\gamma \yields \mu : \sigmacl{x}{p}{q}}
    {\gamma \yields \snd \mu : q[\fst \mu / x]}
    \\
    \fst{(\mu,\nu)} \equiv \mu \and
    \snd{(\mu,\nu)} \equiv \nu \and
    p \equiv (\fst p, \snd p)
\end{mathpar}

Next, we assert a congruence rule for $\Sigma$-types on mode type
morphisms:
\begin{mathpar}
  \inferrule*
  {\TypeTwo{\gamma}{s}{p}{p'} \\
    \TypeTwo{\gamma,x':p'}{t}{q[\TrPlus{s}{x'}/x]}{q'}}
  {\TypeTwo{\gamma}{\sigmacl{x'}{s}{t}}{\sigmacl{x}{p}{q}}{\sigmacl{x'}{p'}{q'}}} 
  \\
    \sigmacl{x'}{\id_p}{\id_q} \equiv \id_{\sigmacl{x'}{p}{q}} \\
  (\sigmacl{x'}{s}{t});(\sigmacl{x''}{s'}{t'}) \equiv \sigmacl{x''}{(s;s')}{(t[\TrPlus{s'}{x''}/x'];t')} \\

  \ap{(\sigmacl{x'}{p}{q})}{s/(y:r)} \equiv
  \sigmacl{x'}{\ap{p}{s/y}}{\ap{({q[\fst z/x,\snd z/y]})}{\extend{s}{x'}/z:(\sigmacl{y}{r}{p})}}
  \\
  \TrPlus{(\sigmacl{x}{s}{t})}{\mu} \equiv (\TrPlus{s}{\fst \mu},\TrPlus{(t[\fst \mu/x])}{\snd \mu})
\end{mathpar}
The first three equations say that this interacts with identity,
compositions, and whiskering: The $\Sigma$ of two identities is the
identity.  The composition of two $\Sigma$ morphisms is the $\Sigma$ of
the composites.  The generic whiskering for mode type morphisms, when
instantiated with a $\Sigma$ mode, is equal to the appropriate instance
of this congruence rule (we write $\ap{q}{s/(x:p)}$ to indicate the type
of the variable involved in the $\mathsf{ap}$).  The final equation says
that the ``transport'' along the $\Sigma$ of two mode type morphisms
acts componentwise.

Finally, we add mode term 2-cells for $\Sigma$ mode types.  It suffices
to add the ``horizontal'' mode term 2-cells (non-trivial in the first
component), because the ``vertical'' ones (non-trivial in the second
component) are given by a whiskering: if $t : \nu \tcell_{q[\mu/x]}
\nu'$ then $\ap{(\mu,y)}{t/y} : (\mu,\nu) \tcell_{\sigmacl{x}{p}{q}}
(\mu,\nu')$.  The rules are:
\begin{mathpar}
\inferrule*
    {\TermTwoT{\gamma}{s}{\mu}{\mu'}{p} \and
      \gamma \vdash \nu' : q[\mu'/x]
    }
      {\TermTwoT{\gamma}{\extend{s}{\nu'}}{(\mu,\TrPlus{\ap{q}{s/x}}{\nu'})}{(\mu',\nu')}{\sigmacl{x}{p}{q}}}\\
\ap {\fst} {\extend{s}{\nu'}} \equiv s \and
\ap {\snd} {\extend{s}{\nu'}} \equiv \id_{\TrPlus{\ap{q}{s/x}}{\nu'}}  \and
s \equiv (\ap{\fst}{s}, \ap{\snd}{s}) %% \quad \text{where } \TermTwoT{\gamma}{s}{\mu}{\mu'}{\sigmacl{x}{p}{q}}
\\      
{\extend{\id_\mu}{\nu'}} \equiv \id_{(\mu,\nu')} \and
{\extend{(s;s')}{\nu''}} \equiv  \extend{s}{\TrPlus{\ap{q}{s'/x}}{\nu''}};\extend{s'}{\nu''}   \\
\extend{s}{\nu'} ; (\ap{(\mu',y)}{t/y}) \equiv
(\ap{(\mu,\TrPlus{(\ap{q}{s})}{y})}{t/y}); \extend{s}{\nu''} \qquad \text{where }\TermTwoT{\gamma}{t}{\nu'}{\nu''}{q[\mu'/x]}
\end{mathpar}

The equations make use of the following definable abbreviations for
pairing and projection 2-cells:
\begin{mathpar}
  \inferrule*[Left=Derivable]
      {\TermTwoT{\gamma}{s}{\mu}{\mu'}{p} \\
    \TermTwoT{\gamma}{t}{\nu}{\TrPlus{\ap{q}{s}}{\nu'}}{q[\mu/x]}}
             {\TermTwoT{\gamma}{(s,t) :\equiv \ap{(\mu,y)}{t/y};\extend{s}{\nu'}}{(\mu,\nu)}{(\mu',\nu')}{\sigmacl{x}{p}{q}}}
  \\
   \inferrule*[Left=Derivable]
              { {\TermTwoT{\gamma}{s}{\mu}{\mu'}{\sigmacl{x}{p}{q}}} }
              { {\TermTwoT{\gamma}{\ap{\fst}{s} := \ap{\fst(y)}{s/y}}{\fst{\mu}}{\fst{\mu'}}{p}} }
  \\
   \inferrule*[Left=Derivable]
              { {\TermTwoT{\gamma}{s}{\mu}{\mu'}{\sigmacl{x}{p}{q}}} }
              { {\TermTwoT{\gamma}{\ap{\snd}{s} := \ap{\snd(y)}{s/y}}{\snd{\mu}}{\TrPlus{\ap{(q(\fst y/x))}{s/y}}{\snd{\mu'}}}{q[\fst{\mu}/x]}} }
\end{mathpar}
%
The first two equation are the $\beta$ rules for morphisms in $\Sigma$
modes.  The next is an $\eta$ (the right-hand side expands to
\ap{(\fst{\mu},y)}{\ap{(\snd z)}{s/z}/y};\extend{\ap{(\fst{z})}{s/z}}{\snd{\mu'}}).
The next two equations give functoriality on identity and composition of
mode term 2-cells in $p$. The final naturality equation
reconciles the two possible ways of defining $(s,t)$.  

The derivable general pairing 2-cell interacts with identity and
composition: we have $(\id_\mu,\id_{\nu}) \equiv \id_{(\mu,\nu)}$ 
and, for mode term 2-cells 
\TermTwoT{\gamma}{s}{\mu}{\mu'}{p} and 
\TermTwoT{\gamma}{t}{\nu}{\TrPlus{\ap{q}{s}}{\nu'}}{q[\mu/x]} and 
\TermTwoT{\gamma}{s'}{\mu'}{\mu''}{p} and 
\TermTwoT{\gamma}{t'}{\nu'}{\TrPlus{\ap{q}{s'}}{\nu''}}{q[\mu'/x]}
we have
\begin{align*}
(s, t);(s', t') \equiv ((s;s'), (t;\ApPlus{\ap{q}{s}}{t'}))
\end{align*}
This follows by
\begin{align*}
(s, t);(s', t') 
&\equiv \ap{(\mu,y)}{t/y};\extend{s}{\nu'};\ap{(\mu',y)}{t'/y};\extend{s'}{\nu''} \\
&\equiv \ap{(\mu,y)}{t/y};\ap{(\mu, \TrPlus{\ap{q}{s}}{y})}{t'/y};\extend{s}{\TrPlus{\ap{q}{s'}}{\nu''}};\extend{s'}{\nu''} \\
&\equiv \ap{(\mu,y)}{t/y};\ap{(\mu, y)}{\ApPlus{\ap{q}{s}}{t'}/y};\extend{s}{\TrPlus{\ap{q}{s'}}{\nu''}};\extend{s'}{\nu''} \\
&\equiv \ap{(\mu,y)}{t;\ApPlus{\ap{q}{s}}{t'}/y};\extend{s;s'}{\nu''} \\
&\equiv ((s;s'), (t;\ApPlus{\ap{q}{s}}{t'}))
\end{align*}

\begin{remark}
An alternative to specifying the mode term morphisms in $\Sigma$-types
as above would be to have mode telescopes as part of the judgement
structure.  For example, if we had a primitive ``Frobenius'' whiskering
that acted on the middle of the context:
\[
\inferrule*{{\gamma,x:p,\gamma'} \vdash {q} \type \\
            \TermTwoT{\gamma}{t}{\mu}{\mu'}{p}\\
           } 
           {\TypeTwo{\gamma,\vec{y}:\gamma'[\mu'/x]}{\ap {q} {t/x}}{q[\mu/x,\TrPlus{\ap{\gamma'}{s}}{\vec{y}}/y]}{q[\mu'/x]}}
\]
(and similarly for terms) then $(s,\mathsf{id})$ would be a special
case.  Conversely, we can define these Frobenius whiskerings by packing
$x:p,\gamma'$ into a mode $\Sigma$ type.  Though the telscope approach
would be more judgemental, it adds a bit of weight to the presentation,
which we choose to avoid here.
\end{remark}

\subsection{Substitution}
  All judgements have an admissible substitution principle
\begin{mathpar}
  \inferrule*{\gamma,x:p,\gamma' \yields J \\
              \gamma \yields \mu : p
              }
             {\gamma,\gamma'[\mu/x] \yields J[\mu/x]} \\

J[\mu/x][\nu/y] \equiv J[\nu/y][\mu[\nu/y]/x]
\end{mathpar}

\section{Example Mode Theories}
\label{sec:mode-examples}
\mvrnote{TODO: Copy over and update}


\section{Type Theory over a Mode Theory}
\label{sec:top-syntax}

The role of the \emph{top type theory} in our framework is to provide
general type-theoretic rules for working ``inside'' the type theory
specified by a mode theory.  In the previous section, we showed
signatures for a variety of example substructural and modal dependent
type theories, and in the next section we will prove that the instances
of the top type theory with these mode theories support the familiar
rules for these examples.

The top type theory consists of four judgements: 
\begin{itemize}
\item $\yields_\gamma \Gamma \CTX$ where $\yields \gamma \ctx$.
\item $\Gamma \yields_p A \TYPE$ where $\gamma \yields p \type$.
\item $\Gamma \yields_p A \ISFIB$ where already $\Gamma \yields_p A \TYPE$
\item $\Gamma \yields_\mu M : A$ where $\gamma \yields \mu : p$ and
  $\Gamma \yields_p A \TYPE$.
\end{itemize}
along with corresponding equality judgements.  Because of the
fibrational perspective, we say that a judgement is ``over'' a
corresponding entity from the base type theory: a top context $\Gamma$
is over a mode context $\gamma$, a top type $A$ is over a mode type $p$,
and a top term $M$ is over a mode term $\mu$.  We also say that a type $A$
``has mode $p$'', and that a term $M$ ``has mode $\mu$''. 

We treat the well-formedness of the mode theory entities as a
presupposition of the top judgements, but sometimes write ``(over
\ldots)'' in inference rules to emphasize the typing that is happening
in the base.  We also treat the well-formedness of the non-principal
subjects of a judgement as presuppostions.

Using the common substructural logic parlance, we think of the base type
theory entities subscripting the turnstile as ``resources'', with the
mode theory dictating the ``resource usage policies'' of a particular
type theory.  For example, in logic with weakening and contraction, the
context $x : A \vdash J$ allows as many uses of $x$ as desired; but
without contraction, $x$ is a ``resource'' that can be used at most
once; while without weakening $x$ must be ``spent''.  This metaphor
extends to modal logics, where we think of modalities as additional
policies that must be obeyed; for example, if $f$ is a comonad, then the
resource $f(x)$ can be used as $x$ (by the counit), and can be used to
construct something else in the comonad; but a resource $x$ cannot be
used to construct something in the comonad.

\subsection{Top Contexts}

Top contexts are as usual, over the corresponding contexts of the base.  

\begin{mathpar}
  \inferrule*[Left = ctx-form]{ }
             {\yields_{\cdot} \cdot \CTX  } \and 

  \inferrule*[Left = ctx-form]{
    \yields_\gamma \Gamma \CTX \quad \text{over } (\yields \gamma \ctx) \\\\
    \Gamma \yields_p A \TYPE \quad \text{over }  (\gamma \yields p \type)}
  {\yields_{\gamma, x : p} \Gamma, x : A \CTX \quad \text{over } (\yields \gamma,x:p \ctx)  } \\
\end{mathpar}

We adopt a convention that the same variable names are used in the top
context and base context (which must, by these rules have the same
number of variables).  

\subsection{Top Structural Rules}
\label{sec:top-syntax-structural}

There are two derivable structural rules: 
\begin{mathpar}
  \inferrule*[Left = var]{
    % \yields \Gamma, x : A, \Gamma' \CTX_{\gamma, x : p, \gamma'}
  }
  {\Gamma, x : A, \Gamma' \yields_x x : A \quad (\text{over } \gamma,x:p,\gamma' \yields x : p)} \and

 \inferrule*[Left = rewrite]{
   \Gamma \yields_\mu M : A 
   \and \TermTwoT{\gamma}{s}{\nu}{\mu}{p}
  }
  {\Gamma \yields_\nu \rewrite{s}{M} : A} \\ \\
  
  \rewrite{\id_{\mu}}{M} \equiv M \and
  \rewrite{(s;t)}{M} \equiv \rewrite{s}{\rewrite{t}{M}} \and
  \rewrite{s}{M}[\rewrite{t}{N}/x] \equiv \rewrite{\ap{s}{t/x}}{\StI{\ap{q}{t/x}}{M[N/x]}}
\end{mathpar}

The first is the familiar variable rule, which is over the corresponding
variable rule in the base.  Intuitively, this says that to use an
assumption, the subscripting mode term must tell you that you can and
must use exactly that variable.  More permissive policies
(e.g. weakening away unused variables) will come from specific mode
theories.

The second rule says that mode term morphisms act contravariantly on
judgement subscripts, and the equations say that this action is
functorial, and commutes with substitution (the $\StI{}{}$ notation is
introduced in Section~\ref{sec:top-s-types} below).

The remaining rules make weakening, exchange, and substitution
admissible for the top type theory, where each rule is over the
corresponding structural rule of the base.  For example, we have
\begin{mathpar}
\inferrule*[Left = weaken-over, Right=admissible]
           {\Gamma,\Gamma' \yields_\mu M : A \quad (\text{over } \gamma,\gamma' \vdash \mu : p)}
           {\Gamma,y:B,\Gamma' \yields_\mu M : A \quad (\text{over } \gamma,y:q,\gamma' \vdash \mu : p)}

\inferrule*[Left = subst-over, Right=admissible]
           {\Gamma,x:A,\Gamma' \yields_\nu N : C \quad (\text{over } \gamma,x:p,\gamma' \vdash \nu : \gamma) \\\\
            \Gamma \vdash_\mu M : A \quad (\text{over } \gamma \vdash \mu : p)
           }
           {\Gamma,\Gamma'[M/x] \yields_{\nu[\mu/x]} N[M/x] : A[M/x] \quad (\text{over } \gamma,\gamma'[\mu/x] \vdash \nu[\mu/x] : p[\mu/x])}
\end{mathpar}

In terms of resources, weakening-over-weakening says that it is always
permissible to add a variable to the framework context, even when using
the framework to encode a type theory without weakening; this is because
adding $y$ to the framework context does not also add it to the
``resources'' $\mu$, which is a separate step that must be given in the
mode theory.  Substitution-over-substitution says that, in the mode
term, a substitution replaces all occurences of $x$ with the mode term
$\mu$ used to construct the top term $M$ that is plugged in for $x$.

\subsection{Top Unit Types}

We introduce a unit type in the top type theory, whose rules are over
the corresponding rules for the unit type in the base:

\begin{mathpar}
  \inferrule*[Left=$1$-form]{~}{\Gamma \yields_{1} 1 \TYPE} \and
  \inferrule*[Left=$1$-fib]{~}{\Gamma \yields_{1} 1 \ISFIB} \and
  \inferrule*[Left=$1$-intro]{~}{\Gamma \yields_{()} () : 1} \and
  M \equiv () \\
\end{mathpar}

Note that top unit types will play a rather structural role, and not, on
their own, model the unit types of a particular encoded language.  This
is because the unit type of an encoded language will have mode
$\mathsf{p}$ for some constant, or e.g. for a comprehension object
representing dependent type theory $\El{\mathsf p}{\alpha}$, and not
mode $1$.  

\subsection{Top Telescope Types}

Similarly, we introduce a $\Sigma$ type to the top type theory, over the
corresponding rules for mode $\Sigma$ types.  But as in the base type
theory, we refer to them as \emph{telescope types} to emphasize that
they play a structural role, rather than, on their own, represent the
$\Sigma$ types of an encoded language.  

\begin{mathpar}
  \inferrule*[Left=$()$-form]{ \Gamma \yields_p A \TYPE \\
               \Gamma,x:A \yields_q B \TYPE}
             { \Gamma \yields_{\sigmacl{x}{p}{q}} \telety{x}{A}{B} \TYPE}
  \and
  \inferrule*[Left=$()$-fib]{ \Gamma \yields_p A \ISFIB \\
               \Gamma,x:A \yields_q B \ISFIB}
             { \Gamma \yields_{\sigmacl{x}{p}{q}} \telety{x}{A}{B} \ISFIB}
  \\
  \inferrule*[Left=$()$-pair]{ \Gamma \yields_\mu M : A \\
               \Gamma \yields_\nu N : B[M/x]
             }
             { \Gamma \yields_{(\mu,\nu)} (M,N) : \telety{x}{A}{B}}
  \\             
  \inferrule*[Left=$()$-fst]{ \Gamma \yields_{\mu} M : \telety{x}{A}{B}}
             { \Gamma \yields_{\fst \mu} \fst{M} : A} 
  \and
  \inferrule*[Left=$()$-snd]{ \Gamma \yields_{\mu} M : \telety{x}{A}{B}}
             { \Gamma \yields_{\snd \mu} \snd{M} : B[\fst M/x]} 
  \\
             
    \fst{(M,N)} \equiv M \and
    \snd{(M,N)} \equiv N \and
    P \equiv (\fst P, \snd P)
\end{mathpar}

We present telescope types as negative $\Sigma$ types, with a
judgemental $\eta$ rule.  Each rule is over the corresponding rule of
the base type theory.

\subsection{Left Adjoint Modalities}

Left adjoint modalities, written $\F{x.\mu}{A}$, represent a
pushforward/cocartesian lift of a mode term $x : p \vdash \mu : q$, taking
types of mode $p$ to types of mode $q$.  $\mathsf{F}$-types will be used
to represent left adjoint modalities (e.g. $\flat$ in spatial type
theory).  In combination with top unit types and top telescope types,
these will also be used to represent encoded language unit and $\Sigma$
types.  For example, the $\Sigma_A B$ of standard dependent type theory
will be represented by $\F{z.\Sigma_1(\fst z, \snd
  z)}{\telety{x}{A}{B}}$: the telescope type has telescope mode
$\sigmacl{x}{\El{p}{\alpha}}{\El{p}{\alpha.x}}$, and then we use an
$\mathsf{F}$ type to push forward along the mode term $\Sigma_1$ that we
use to assert that a comprehension object supports $\Sigma$ types.
In \citepalias{lsr17multi-extended}, we used an $n$-ary $\mathsf{F}$ type in
place of having separate telescope types.

The rules are:

\begin{mathpar}
  \inferrule*[Left = F-form]{
    %% \yields_\gamma \Gamma \CTX \quad (\text{over } \yields \gamma \ctx)\\\\
    \Gamma \yields_p A \TYPE \quad (\text{over } \gamma \yields p \type) \\\\
    \gamma, x:p \yields \mu : q 
  }
  {\Gamma \yields_q \F{x.\mu}{A} \TYPE \quad (\text{over } \gamma \yields q \type) } \and
  \inferrule*[Left = F-fib]{
    \Gamma \yields_p A \ISFIB
  }
  {\Gamma \yields_q \F{x.\mu}{A} \ISFIB } \\
  
  \inferrule*[Left = F-intro]{
    \Gamma \yields_{\nu} M : A
    \quad (\text{over } \gamma \yields {\nu} : p)
    %% \and \gamma \yields \nu : q 
    %% \and \gamma \yields \mu[\theta] : q 
    %% \and \gamma \yields (\nu \Rightarrow \mu[\theta]) : q
  }
  {\Gamma \yields_{\mu[\nu/x]} \FI{M} : \F{x.\mu}{A} \quad (\text{over } \gamma \yields \mu[\nu/x] : q)} \\

  \inferrule*[Left = F-elim]{
    \Gamma, y : \F{x.\mu}{A} \yields_{r} C \TYPE \quad (\text{over } \gamma, y : q \yields r \type) \and \Gamma, y : \F{x.\mu}{A} \yields_{r} C \ISFIB \\\\
    \Gamma \yields_{\nu} M : \F{x.\mu}{A} \quad (\text{over } \gamma \yields \nu : q) \\\\
    \Gamma, x:A \yields_{\nu' [\mu / y]} N : C [\FI{x}/y]
    \quad (\text{over } \gamma, x:p \yields \nu' [\mu / y] : r [\mu / y] )}
  {\Gamma \yields_{\nu'[\nu/y]} \FE{M}{x}{N} : C[M/y]  \quad (\text{over }  \gamma \yields {\nu'[\nu/y]} : r[\nu/y])} \\
  \FE{\FI{M}}{x}{N} \equiv N[M/x] %\and
%  \text{(optionally:) }
%  \FE{M}{x}{N[\FI{x}/z]} \equiv N[M/z]
  \\ 

  \inferrule*[Left = F-Elim]{
    \gamma,y:q \yields r \type \\\\
    \Gamma \yields_{\nu} M : \F{x.\mu}{A} \quad (\text{over } \gamma \yields \nu : q) \\\\
    \Gamma, x:A \yields_{r [\mu / y]} C \TYPE
    \quad (\text{over } \gamma, x:p \yields r [\mu / y] \type )}
  {\Gamma \yields_{r[\nu/y]} \FE{M}{x}{C} \TYPE \quad (\text{over }  \gamma \yields {r[\nu/y]} \type)} \\
  \FE{\FI{M}}{x}{C} \equiv C[M/x] %\and
%  \text{(optionally:) }
%  \FE{M}{x}{C[\FI{x}/z]} \equiv C[M/z]
  \\
  
  (\FE{M}{x}{\rewrite{t[\mu/y]}{N}}) \equiv \rewrite{t[\nu/y]}{\FE{M}{x}{N}}
\end{mathpar}

$\mathsf{F}$ types are a positive type, and can be thought of as a
datatype with one constructor, the introduction rule $\FI{M}$.  The main
idea is that $\Gamma, x:A \vdash_{\mu(x)} \FI{x} : \F{x.\mu}{A}$;
i.e. to introduce $\F{x.\mu}{A}$, the mode term must be $\mu$.  The full
form of the introduction rule builds in a substitution for $x$ to make
substitution admissible.  The elimination rule patterm-matches on a term
of $\mathsf{F}$ type.  In sequent calculus style, it characterizes maps
out of $\mathsf{F}$ types:
\begin{mathpar}
  \inferrule*{
    \Gamma, y : \F{x.\mu}{A} \yields_{r} C \TYPE \quad (\text{over } \gamma, y : q \yields r \type) \\\\
    \Gamma, y : \F{x.\mu}{A}, x:A \yields_{\nu' [\mu / y]} N : C [\FI{x}/y]
    \quad (\text{over } \gamma, x:p \yields \nu' [\mu / y] : r [\mu / y] )}
  {\Gamma, y : \F{x.\mu}{A} \yields_{\nu'} \FE{y}{x}{N} : C  \quad (\text{over }  \gamma,y:q \yields {\nu'} : r)} 
\end{mathpar}
This says that an assumption of $\F{x.\mu}{A}$ can be replaced by an
assumption of $A$, but in the mode term we replace $y$ by $\mu$ to
remember the ``resources'' used to make $x$.

We treat $\mathsf{F}$ types like a positive type in dependent type
theory: we have a judgemental $\beta$ equality, but no $\eta$ (we intend
$\eta$ to be provable from the elimination rule with some sort of
identity type (which we have not yet investigated)).  We also include a
large elimination rule, i.e. a second elimination rule that maps into
types (which we do not identify with elements of a universe).
In general, the interaction of the structural rule applying a mode term
2-cell to a term, $\rewrite{t}{M}$, and the terms for various types
will follow from the equalities for $\rewrite{t}{-}$ in
Section~\ref{sec:top-syntax-structural} and the $\eta$ rules for the
types.  Because we do not have judgemental $\eta$ for $\mathsf{F}$
types, we assert the necessary interaction in the final equation in the
figure.  

\subsection{Right Adjoint Modalities}

A simple form of right adjoint types has the shape $\mathsf{U}_{c.\mu}(A
\TYPE_q) \TYPE_r$ when $c:r \vdash \mu : q$, and represents a
pullback/cartesian lift along $\mu$, and will be right adjoint to
$\F{c.\mu}{-}$.  We will generalize this by allowing a contravariant
position as well, so that right adjoint types include both right adjoint
modalities and dependent function types
($\Pi$)~\citep{atkey04separation,lsr17multi-extended}.

\begin{mathpar}
  \inferrule*[Left = U-form]{
    \Gamma \yields_p A \TYPE \quad (\text{over } \gamma \yields p \type)\\\\
    \and \Gamma,x:A \yields_q B \TYPE \quad (\text{over } \gamma,x:p \yields q \type)\\\\
    \and \gamma, x:p, c:r \yields \mu : q
  }{\Gamma \yields_r \U{c.\mu}{x:A}{B} \TYPE \quad (\text{over } \gamma \yields r \type)} \and
  \inferrule*[Left = U-fib]{
    \Gamma \yields_p A \ISFIB \\\\
    \and \Gamma,x:A \yields_q B \ISFIB
  }{\Gamma \yields_r \U{c.\mu}{x:A}{B} \ISFIB} \\

  \inferrule*[Left = U-intro]{
    \Gamma,x:A \yields_{\mu[\nu/c]} M : B \quad (\text{over } \gamma,x:p \yields {\mu[\nu/c]} : q)
  }
  {\Gamma \yields_{\nu} \UI {x}{M} : \U{c.\mu}{x:A}{B}
    \quad (\text{over } \gamma \yields \nu : r)
  } \\
  
  \inferrule*[Left = U-elim]{
    \Gamma \yields_{\nu_1} N_1 : \U{c.\mu}{x:A}{B} \quad (\text{over } \gamma \yields \nu_1 : r) \\\\
    \Gamma \yields_{\nu_2} N_2 : A \quad (\text{over } \gamma \yields \nu_2 : p)
  }{
    \Gamma \yields_{\mu[\nu_2/x,\nu_1/c]} \UE{N_1}{N_2} : B[N_2/x] \quad (\text{over } \gamma \yields \mu[\nu_2/x,\nu_1/c] : q)
  } \\

  \UE{(\UI{x}{M})}{N} \equiv M[N/x] \and 
  \UI{x}{\UE{N}{x}} \equiv N
\end{mathpar}

The type $\U{c.\mu}{x:A}{B}$ can be thought of as a modal verson of $\Pi
x:A.B$, where $\mu$ describes the resources under which the variable
$x$ can be used.  The simplified version discussed above is the special
case where $A$ is the top unit type $1$.  The variable $c$ in $\mu$ is a
placeholder for the ``current'' mode term: the $\lambda$ rule says that
to prove $\U{c.\mu}{x:A}{B}$ over $\nu$, you substitute $\nu$ for $c$ in
$\mu$ in the premise.  In the elimination rule, application, $\mu$ tells
you how to combine $\nu_1$ (the resources for the function) and
$\nu_2$ (the resources for the argument) when you apply a function.
For example, in the simply typed case~\citep*{lsr17multi-extended},
taking $\mu(c,x)$ to be $c \times x$ gives an $\to$ type right adjoint
to a product $\times$.

$\mathsf{U}$ types are a negative type, with judgemental $\beta$ and
$\eta$ equality rules.

\subsection{Strict Modalities}

\mvrnote{Text needs updating:}

The term $\rewrite{s}{a}$ (Section~\ref{sec:top-syntax-structural})
``transports'' a term $\Gamma \vdash_\mu a : A$ along a mode term 2-cell
$s : \mu \tcell_p \nu$ to make a term $\Gamma \vdash_\mu \rewrite{s}{a}
: A$.  In this section, we consider an analogous operation on types,
transporting a type $\Gamma \vdash_p A \TYPE$ along a mode type morphism
$s : q \tcell p$ to make a new type $\Gamma \vdash_q \St{s}{A} \TYPE$.
For example, if $s := \ApEl{p}{\pi} : \El p {\alpha.x} \tcell \El p
{\alpha}$ is the projection mode type morphism for a comprehension
object, then $\St{s}{A}$ represents weakening a type in the encoded
language ($\Gamma \vdash B \TYPE$ implies $\Gamma,A \vdash B \TYPE$).

In some sense, such an operation already exists, because the mode type
morphism $s$ contravariantly induces a mode term $x : p \vdash
\TrPlus{s}{x} : q$, and we can take the left adjoint type
$\F{x.\TrPlus{s}{x}}{A}$.  However, we will want the action of mode type
morphisms to obey certain judgemental equalities that we do not assert,
%% or do not even make sense
for a general mode term.  We call these
\emph{strict modalities}, written $\St{s}{A}$.  The introduction and
elimination rules for $\St{s}{A}$ are the same as those for
$\F{x.\TrPlus{s}{x}}{A}$, except we additionally assert a judgemental
$\eta$ rule that any map $N$ from $\St{s}{A}$ is equal to first doing
the elimination:
\[
\inferrule*{\Gamma,z:\St{s}{A} \vdash_\mu N : C \and
            \Gamma \vdash_\nu M : \St{s}{A} \and
           }
           {\Gamma \vdash_{\mu[\nu/z]} (\StE{s}{M}{x}{N[\StI{s}{x}/z]}) \equiv N[M/z] : C[\St{s}{A}/z]}
\]
(along with a corresponding judgemental $\eta$ for the large elimination
rule).  

In addition to this, we add four judgemental equalities
((\ref{eq:stype-id}) to \ref{eq:stype-pair}) of types (in the right-hand
column) and of terms with those types (in the left-hand column).  These
four equations give the four mode-theory-independent ways we have of
making mode type morphisms strict actions on types.  For non-strict left
adjoints $\F{x.\TrPlus{s}{x}}{A}$, these type equalities would only be an
``equivalence'': there are functions back and forth, but we would need
an identity type in order to use the induction principle for
$\mathsf{F}$ types to prove that the composites are the identity.

The first equation (\ref{eq:stype-id}) concerns the identity mode type
morphism $\id{p}$, which takes a type $\vdash_p A$ to another type of
mode $p$.  The equation in the right-hand column says that $\St{\id
  p}{A}$ is judgementally equal to $A$, and that, for $\Gamma \vdash_\mu
M : A$, the introduction form $\Gamma \vdash_{\TrPlus{\id p}{\mu}}
\StI{\id p}{M} : \St{\id p}{A}$ is equal to $M$ (both terms witness the
same judgement because of the right-hand equation and functoriality of
$\TrPlus{\id_p}{-}$ in the mode theory.  The second equation
(\ref{eq:stype-comp}) is the analogous rule for composition of mode type
morphisms.

%% It would type check to ask that, for general mode terms, $\F{x.x}{A}
%% \equiv A$ and $\F{g}{\F{f}{A}} \equiv \F{g \circ f}{A}$, but we do not
%% ask for this judgemental equality for general $\mathsf{F}$-types.

The next equation (\ref{eq:stype-subst}) is an analogous rule for the
whiskering operation that takes $t : \mu \tcell_p \mu'$ to $\ap q {t/x} :
q[\mu/x] \tcell q[\mu'/x]$.  Suppose $\Gamma,x:A \vdash_q B \TYPE$ and
$\Gamma \vdash_{\mu'} M : A$.  Then we could either first transport $M$,
getting $\Gamma \vdash_\mu \rewrite{t}{M} : A$, and then substitute into
$B$ to get $\Gamma \vdash_{q[\mu/x]} B[\rewrite{t}{M}/x] \TYPE$; or we
could first substitute to get $\Gamma \vdash_{q[\nu/x]} B[M/x] \TYPE$
and then transport the type along $\ap q {t/x}$, getting $\Gamma
\vdash_{q[\nu/x]} \St{\ap q {t/x}}{B[M/x]} \TYPE$.  The equation says
that these types are judgementally equal.  The left-hand column says
that if we have $x : A \vdash_\nu N : B$, then substituting
$\rewrite{t}{M}$ or substituting $M$ and then rewriting are equal.

Note that the corresponding equation for whiskering on the other side,
$(\St{s}{A})[M/x] \equiv \St{(s[\mu/x])}{A[M/x]}$, is one of the
defining equations for substitution.

For the final equation (\ref{eq:stype-pair}), the right-hand
equation interprets the action of a $\Sigma$ of mode type morphisms on
an upstairs telescope type as a componentwise action. Recall from
Section~\ref{sec:base-telescopes} that $\sigmacl{x'}{s}{t}$ type checks when
$\TypeTwo{\gamma}{s}{p}{p'}$ and
$\TypeTwo{\gamma,x':p'}{t}{q[\TrPlus{s}{x'}/x]}{q'}$,
and here we require
$\Gamma \vdash_{p'} A' \TYPE$ and $\Gamma,x':A' \vdash_{q'} B' \TYPE$,
so we have
$\Gamma \vdash_{\sigmacl{x}{p}{q}} \St{(\telety{x'}{s}{t})}{\telety{x'}{A'}{B'}}$
and
$\Gamma \vdash_{p} \St{s}{A'} \TYPE$
and 
$\Gamma,x':p' \vdash_{q[\TrPlus{s}{x'}/x]} \St{t}{B'} \TYPE$, so
$\Gamma,x:\St{s}{A'} \vdash_{q} \StE{s}{x}{x'}{\St{t}{B'}} \TYPE$, so
$\Gamma \vdash_{\sigmacl{x}{p}{q}} \telety{x}{\St{s}{A'}}{\StE{s}{x}{x'}{\St{t}{B'}}}$.
The
left-hand column says that, assuming $\Gamma
\vdash_\mu M : A'$ and $\Gamma \vdash_\nu N : B'[M/x']$, 
the term
$\Gamma \vdash_{\TrPlus{\sigmacl{x'}{s}{t}}{(\mu,\nu)}} \StI{\sigmacl{x'}{s}{t}}{(M, N)} : \St{(\telety{x'}{s}{t})}{\telety{x'}{A'}{B'}}$ is equal to
is equal to the term
$\Gamma \vdash_{(\TrPlus{s}{\mu},\TrPlus{t[\mu/x']}{\nu})} (\StI{s}{M}, \StI{t[\mu/x]}{N}) : \telety{x}{\St{s}{A'}}{\StE{s}{x}{x'}{\St{t}{B'}}}$,
using the equation between types and the equation for a $\Sigma$ mode
type morphism on mode terms.  

\begin{mathpar}
  \inferrule*[Left = E-form]{
    \Gamma \yields_p A \TYPE \quad (\text{over } \gamma \yields p \type) \\\\
    \gamma, x:p \yields \mu : q 
  }
  {\Gamma \yields_q \E{x.\mu}{A} \TYPE \quad (\text{over } \gamma \yields q \type) } \\
  \inferrule*[Left = E-intro]{
    \Gamma \yields_{\nu} M : A
    \quad (\text{over } \gamma \yields {\nu} : p)
    %% \and \gamma \yields \nu : q 
    %% \and \gamma \yields \mu[\theta] : q 
    %% \and \gamma \yields (\nu \Rightarrow \mu[\theta]) : q
  }
  {\Gamma \yields_{\mu[\nu/x]} \EI{M} : \E{x.\mu}{A} \quad (\text{over } \gamma \yields \mu[\nu/x] : q)} \\

  \inferrule*[Left = E-elim]{
    \Gamma, y : \E{x.\mu}{A} \yields_{r} C \TYPE \quad (\text{over } \gamma, y : q \yields r \type) \\\\
    \Gamma \yields_{\nu} M : \E{x.\mu}{A} \quad (\text{over } \gamma \yields \nu : q) \\\\
    \Gamma, x:A \yields_{\nu' [\mu / y]} N : C [\EI{x}/y]
    \quad (\text{over } \gamma, x:p \yields \nu' [\mu / y] : r [\mu / y] )}
  {\Gamma \yields_{\nu'[\nu/y]} \EE{M}{x}{N} : C[M/y]  \quad (\text{over }  \gamma \yields {\nu'[\nu/y]} : r[\nu/y])} \\
  \EEs{\mu}{\EI{M}}{x}{N} \equiv N[M/x] \and
  (\EEs{\mu}{M}{x}{N[\EI{x}/z]}) \equiv N[M/z]
  \\
  
\inferrule*[Left = E-Elim]{
    \gamma,y:q \yields r \type \\\\
    \Gamma \yields_{\nu} M : \E{x.\mu}{A} \quad (\text{over } \gamma \yields \nu : q) \\\\
    \Gamma, x:A \yields_{r [\mu / y]} C \TYPE
    \quad (\text{over } \gamma, x:p \yields r [\mu / y] \type )}
  {\Gamma \yields_{r[\nu/y]} \EEs{\mu}{M}{x}{C} \TYPE \quad (\text{over }  \gamma \yields {r[\nu/y]} \type)} \\
  \EEs{\mu}{\EI{M}}{x}{C} \equiv C[M/x] \and
  (\EEs{\mu}{M}{x}{C[\EI{x}/z]}) \equiv C[M/z]
\end{mathpar}

\begin{align}
\label{eq:etype-id} \EIs{x.x}{M} &\equiv M &\E{x.x}{A} &\equiv A \\
\label{eq:etype-comp} \EIs{x.\mu}{\EIs{x.\nu}{M}} &\equiv \EIs{x.\mu[\nu/x]}{M} &\EIs{x.\mu}{\EIs{x.\nu}{A}} &\equiv \EIs{x.\mu[\nu/x]}{A} \\
\label{eq:etype-pair}\EIs{(y,y').(\nu, \nu')}{(M, N)} &\equiv (\EIs{y.\nu}{M}, \EIs{y.\nu'[\mu/x]}{N}) &\E{(y,y').(\nu, \nu')}{x' : A', B'} & \equiv \telety{x}{\E{y.\nu}{A'}}{\EEs{\nu}{x}{x'}{\E{y.\nu'}{B'}}} 
%% other whiskering 'is a substitution rule:
%% \St{(s[\mu/x])}{B[M/x]} & \equiv (\St{s}{B})[M/x] 
\end{align}
Using $\EIs{(y,y').(\nu, \nu')}{\dots}$ as a short-hand for $\EIs{w.(\nu[\fst w/y], \nu'[\snd w/y'])}{\dots}$.

\mvrnote{s-types intro} An instance of $\mathsf{E}$-types that we will use frequently is the $\mathsf{E}$-type for the mode term $\TrPlus{s}{x}$, where $s : p \tcell q$ is a mode term morphism. We introduce some syntax for this situation:
\begin{align*}
\St{s}{A} &:\equiv \E{x.\TrPlus{s}{x}}{A} \\
\StI{s}{x} &:\equiv \EIs{x.\TrPlus{s}{x}}{x} \\
\StE{s}{M}{y}{N} &:\equiv \EEs{x.\TrPlus{s}{x}}{M}{y}{N}
\end{align*}

In combination with \mvrnote{mode theory equations}, the fusion equations for $\mathsf{E}$-types specialise to:
\begin{align}
\label{eq:stype-id} \StI{\id_p}{M} &\equiv M &\St{\id_p}{A} &\equiv A \\
\label{eq:stype-comp} \StI{s}{\StI{t}{M}} &\equiv \StI{s;t}{M} &\St{s}{\St{t}{A}} &\equiv \St{(s;t)}{A} \\
\label{eq:stype-pair}\StI{(s, t)}{(M, N)} &\equiv (\StI{s}{M}, \StI{t[\mu/x]}{N}) &\St{(\telety{x'}{s}{t})}{\telety{x'}{A'}{B'}} & \equiv \telety{x}{\St{s}{A'}}{\StE{s}{x}{x'}{\St{t}{B'}}} 
\end{align}

\subsection{Derivable Terms and Equations}

We end this section with a number of lemmas (which apply to all mode
theories) that will be used in example encodings.

\begin{lemma}
The \textsc{rewrite} rule commutes with pairing of telescope types:
\begin{align*}
%(\rewrite{s}{M},N) &\equiv \rewrite{(s, \varepsilon_\nu^{\ap{q}{s/x}})}{(M, \UnSt{\ap{q}{s/x}}{N}}) \\
%(M,\rewrite{t}{N}) &\equiv \rewrite{(\id_\mu, t)}{(M,N)} \\
\rewrite{(s, t)}{(M, N)} &\equiv (\rewrite{s}{M}, \rewrite{t}{\StI{\ap{q}{s/x}}{N}} ) 
\end{align*}
where
\begin{align*}
\TermTwoT{\gamma &}{s}{\mu}{\mu'}{p} \\
\TermTwoT{\gamma &}{t}{\nu}{\TrPlus{\ap{q}{s}}{\nu'}}{q[\mu/x]} \\
\Gamma &\yields_{\mu'} M : A \\
\Gamma &\yields_{\nu'} N : B[M/x]
\end{align*}
\end{lemma}
\begin{proof}
\begin{align*}
&\rewrite{(s, t)}{(M, N)} \\
&\equiv (\fst\rewrite{(s, t)}{(M, N)}, \snd \rewrite{(s, t)}{(M, N)} ) \\
&\equiv (\fst z [\rewrite{(s, t)}{(M, N)}/z], \snd z [\rewrite{(s, t)}{(M, N)}/z] ) \\
&\equiv (\rewrite{\ap{\fst}{(s, t)}}{\fst (M, N)}, \rewrite{\ap{\snd}{(s, t)}}{\StI{\ap{q[\fst z/x]}{(s, t)/z}}{\snd (M, N)}} ) \\
&\equiv (\rewrite{s}{M}, \rewrite{t}{\StI{\ap{q}{s/x}}{N}} )
\end{align*}
This type checks via Equation~\eqref{eq:stype-subst}; in the second component we have
\begin{align*}
&\snd \rewrite{(s, t)}{(M, N)} &&: B[\fst\rewrite{(s, t)}{(M, N)}/x] \\
\equiv~&\snd \rewrite{(s, t)}{(M, N)} &&: B[\rewrite{s}{M}/x] \\
\equiv~&\rewrite{t}{\StI{\ap{q}{s/x}}{N}} &&: \St{\ap{q}{s/x}}{B[M/x]}
\end{align*}
\end{proof}

\begin{lemma} \label{lem:rewrite-push}
The \textsc{rewrite} rule commutes with $\mathsf{F}$-, $\mathsf{U}$- and $E$- introduction and elimination.
\begin{align*}
\FI{\rewrite{s}{M}} &\equiv \rewrite{\ap{\mu}{s/x}}{\FI{M}} \\
(\FE{\rewrite{s}{M}}{x}{N}) &\equiv \rewrite{\ap{\nu'}{s/y}}{\StI{\ap{r}{s/y}}{\FE{M}{x}{N}}} \\
\UE{M}{\rewrite{s}{N}} &\equiv \rewrite{\ap{\mu[\nu_1/c]}{s/x}}{\StI{\ap{q}{s/x}}{\UE{M}{N}}} \\
\UE{(\rewrite{t}{M})}{N} &\equiv \rewrite{\ap{\mu[\nu_2/x]}{t/c}}{\UE{M}{N}} \\
\UI{x}{\rewrite{\ap{\mu}{s/c}}{M}}  &\equiv\rewrite{s}{\UI{x}{M}} \\
\EI{\rewrite{s}{M}} &\equiv \rewrite{\ap{\mu}{s/x}}{\EI{M}} \\
(\EE{\rewrite{s}{M}}{x}{N}) &\equiv \rewrite{\ap{\nu'}{s/y}}{\StI{\ap{r}{s/y}}{\EE{M}{x}{N}}} \\
(\EEs{\mu}{M}{x}{\rewrite{s[\mu/y]}{N}}) &\equiv \rewrite{s[\nu/y]}{\EEs{\mu}{M}{x}{N}} 
\end{align*}
\end{lemma}
\begin{proof}
For \textsf{F}-types:
\begin{align*}
\FI{\rewrite{s}{M}} 
&\equiv \rewrite{\id_{\mu[\nu/x]}}{\FI{x}}[\rewrite{s}{M}/x]  \\
&\equiv \rewrite{\ap{\mu}{s/x}}{\StI{\ap{q}{s/x}}{\FI{x}[M/x]}} \\
&\equiv \rewrite{\ap{\mu}{s/x}}{\StI{\ap{q}{s/x}}{\FI{M}}} \\
&\equiv \rewrite{\ap{\mu}{s/x}}{\FI{M}} && \text{As $x$ does not appear in $q$.}\\
(\FE{\rewrite{s}{M}}{x}{N})
&\equiv \rewrite{\id_{\nu'[\nu/y]}}{\FE{y}{x}{N}}[\rewrite{s}{M}/y] \\
&\equiv \rewrite{\ap{\nu'}{s/y}}{\StI{\ap{r}{s/y}}{(\FE{y}{x}{N})[M/y]}} \\
&\equiv \rewrite{\ap{\nu'}{s/y}}{\StI{\ap{r}{s/y}}{\FE{M}{x}{N}}}
\end{align*}

For \textsf{U}-types:
\begin{align*}
\UE{M}{\rewrite{s}{N}}
&\equiv \rewrite{\id_{\mu[z/x, \nu_1/c]}}{\UE{M}{z}}[\rewrite{s}{N}/z] \\
&\equiv \rewrite{\ap{\mu[\nu_1/c]}{s/x}}{\StI{\ap{q[\nu_1/c]}{s/x}}{\UE{M}{N}}} \\
&\equiv \rewrite{\ap{\mu[\nu_1/c]}{s/x}}{\StI{\ap{q}{s/x}}{\UE{M}{N}}} \\
\UE{(\rewrite{t}{M})}{N}
&\equiv \rewrite{\id_{\mu[\nu_2/x, z/c]}}{\UE{(z)}{N}}[\rewrite{t}{M}/z] \\
&\equiv \rewrite{\ap{\mu[\nu_2/x]}{t/c}}{\StI{\ap{q[\nu_2/x]}{t/c}}{\UE{M}{N}}} \\
&\equiv \rewrite{\ap{\mu[\nu_2/x]}{t/c}}{\UE{M}{N}} && \text{(As $c$ does not appear in $q$)}\\
\rewrite{s}{\UI{x}{M}}
&\equiv \UI{y}{\UE{(\rewrite{s}{\UI{x}{M}})}{y}} \\
&\equiv \UI{y}{\rewrite{\ap{\mu[y/x]}{s/c}}{\UE{(\UI{x}{M})}{y}}} \\
&\equiv \UI{y}{\rewrite{\ap{\mu[y/x]}{s/c}}{M[y/x]}} \\
&\equiv \UI{x}{\rewrite{\ap{\mu}{s/c}}{M}}
\end{align*}

The first two equations for $\mathsf{E}$-types follow by the same reasoning as for $\mathsf{F}$-types. For the third, we make use of the $\eta$-expansion that is not available for $\mathsf{F}$-types.
\begin{align*}
\rewrite{s[\nu/y]}{\EEs{\mu}{M}{x}{N}}
&\equiv \rewrite{s}{\EEs{\mu}{y}{x}{N}} [M/y] \\
&\equiv \EEs{\mu}{M}{z}{(\rewrite{s}{\EEs{\mu}{y}{x}{N}}[\EIs{\mu}{z}/y])} \\
&\equiv \EEs{\mu}{M}{z}{(\rewrite{s[\mu/y]}{\EEs{\mu}{\EIs{\mu}{z}}{x}{N}})} \\
&\equiv \EEs{\mu}{M}{z}{\rewrite{s[\mu/y]}{N}}
\end{align*}
\end{proof}

\begin{lemma}
The \textsc{rewrite} rule commutes with \Fsym- and \Esym-elimination into types:
\begin{align*}
(\FE{\rewrite{s}{M}}{x}{C}) &\equiv \St{\ap{r}{s/y}}{\FE{M}{x}{C}} \\
(\EE{\rewrite{s}{M}}{x}{C}) &\equiv \St{\ap{r}{s/y}}{\EE{M}{x}{C}}
\end{align*}
\end{lemma}
\begin{proof}
This is analogous to elimination into terms:
\begin{align*}
\FE{\rewrite{s}{M}}{x}{C}
&\equiv (\FE{y}{x}{C})[\rewrite{s}{M}/y] \\
&\equiv \St{\ap{r}{s/y}}{(\FE{y}{x}{C})[M/y]} \\
&\equiv \St{\ap{r}{s/y}}{\FE{M}{x}{C}}
\end{align*}
and similarly for \Esym-types.
\end{proof}

\begin{lemma}\label{lem:s-elim-s-elim}
\textsc{E-elim} commutes with \textsc{F-elim} and \textsc{E-elim}:
\begin{align*}
\FE{(\EEs{\mu}{M}{x'}{N'})}{x}{N} &\equiv \EEs{\mu}{M}{x'}{(\FE{N'}{x}{N})} \\
\EEs{\nu}{(\EEs{\mu}{M}{x'}{N'})}{x}{N} &\equiv \EEs{\mu}{M}{x'}{(\EEs{\nu}{N'}{x}{N})}
\end{align*}
\end{lemma}
\begin{proof}
Using $\eta$ for \Esym-types:
\begin{align*}
&\FE{(\EEs{\mu}{M}{x'}{N'})}{x}{N} \\
&\equiv (\FE{(\EEs{\mu}{z}{x'}{N'})}{x}{N})[M/z] \\
&\equiv \EEs{\mu}{M}{z'}{((\FE{(\EEs{\mu}{z}{x'}{N'})}{x}{N})[\EIs{\mu}{z'}/z])} \\
&\equiv \EEs{\mu}{M}{z'}{((\FE{(\EEs{\mu}{\EIs{\mu}{z'}}{x'}{N'})}{x}{N}))} \\
&\equiv \EEs{\mu}{M}{z'}{(\FE{N'[z'/x']}{x}{N})} \\
&\equiv \EEs{\mu}{M}{x'}{(\FE{N'}{x}{N})}
\end{align*}
and similarly for the second.
\end{proof}

\begin{lemma}\label{lem:s-elim-fusion}
Fusion for \textsc{E-elim} on composites:
\begin{align*}
\EEs{y.\mu[\nu/y]}{M}{x}{N} \equiv \EEs{y.\mu}{M}{x'}{\EEs{y.\nu}{x'}{x}{N}}
\end{align*}
\end{lemma}
\begin{proof}
\begin{align*}
\StE{s;t}{M}{x}{N}
&\equiv \EEs{y.\mu}{M}{x'}{\EEs{y.\mu[\nu/y]}{\EIs{y.\mu}{x'}}{x}{N}} \\
&\equiv \EEs{y.\mu}{M}{x'}{\EEs{y.\nu}{x'}{x''}{\EEs{y.\mu[\nu/y]}{\EIs{y.\mu}{\EIs{y.\nu}{x''}}}{x}{N}}} \\
&\equiv \EEs{y.\mu}{M}{x'}{\EEs{y.\nu}{x'}{x''}{\EEs{y.\mu[\nu/y]}{\EIs{y.\mu[\nu/y]}{x''}}{x}{N}}} \\
&\equiv \EEs{y.\mu}{M}{x'}{\EEs{y.\nu}{x'}{x''}{N[x''/x]}} \\
&\equiv \EEs{y.\mu}{M}{x'}{\EEs{y.\nu}{x'}{x}{N}}
\end{align*}
\end{proof}

%\begin{lemma}\label{lem:s-elim-tuple}
%Fusion for \textsc{s-elim} on tuples:
%\begin{align*}
%\StE{(x : s, t)}{(M, M'[M/x])}{w}{N} \equiv \StE{s}{M}{x}{\StE{t[\TrPlus{s}{x}/x]}{M'[\StI{s}{x}/x]}{y}{N[(x,y)/w]}}
%\end{align*}
%\end{lemma}
%\begin{proof}
%\begin{align*}
%&\StE{(x : s, t)}{(M, M'[M/x])}{w}{N} \\
%&\equiv \StE{s}{M}{x}{\StE{(s, t)}{(\StI{s}{x}, M'[\StI{s}{x}/x])}{w}{N}} \\
%&\equiv \StE{s}{M}{x}{(\StE{t[\TrPlus{s}{x}/x]}{M'[\StI{s}{x}/x]}{y}{\StE{(s, t)}{(\StI{s}{x}, \StI{t[\TrPlus{s}{x}/x]}{y})}{w}{N}})} \\
%&\equiv \StE{s}{M}{x}{(\StE{t[\TrPlus{s}{x}/x]}{M'[\StI{s}{x}/x]}{y}{\StE{(s, t)}{\StI{(x : s, t)}{(x,y)}}{w}{N}})} \\
%&\equiv \StE{s}{M}{x}{\StE{t[\TrPlus{s}{x}/x]}{M'[\StI{s}{x}/x]}{y}{N[(x,y)/w]}}
%\end{align*}
%\end{proof}

%\begin{lemma}
%\textsc{F-elim} fuses with \textsc{s-intro}.
%\end{lemma}
%\begin{proof}
%\begin{align*}
%&\FEs{\TrPlus{s}{\mu}}{\StI{s}{M}}{x}{N} \\
%&\equiv \StE{s}{\StI{s}{M}}{y}{(\FEs{\mu}{y}{x}{N})} \\
%&\equiv \FEs{\mu}{M}{x}{N}
%\end{align*}
%and
%\begin{align*}
%& \FEs{\mu[\TrPlus{s}{x}/x]}{M}{x}{N[\StI{s}{x}/x]} \\
%&\equiv \FEs{\mu}{M}{y}{(\StE{s}{y}{x}{N[\StI{s}{x}/x]})}  \\
%&\equiv \FEs{\mu}{M}{y}{N[y/x]} \\
%&\equiv \FEs{\mu}{M}{x}{N}
%\end{align*}
%\end{proof}

\begin{lemma}
Split for telescope types is derivable:
\begin{mathpar}
\inferrule*[Left=$()$-split]{\Gamma, w : \telety{x}{A}{B} \yields_r C \TYPE \\\\ \Gamma \yields_{\nu} M : \telety{x}{A}{B} \\\\ \Gamma, x : A, y : B \yields_{\nu'[(x,y)/w]} N : C[(x,y)/w]}{\Gamma \yields_{\nu'[\nu/w]} \TeleE{M}{x}{y}{N} : C[M/w]}
\end{mathpar}
\end{lemma}
\begin{proof}
\begin{mathpar}
\inferrule*[Left=cut]{\Gamma, x : A, y : B \yields_{\nu'[(x,y)/w]} N : C[(x,y)/w]}{\Gamma \yields_{\nu'[\nu/w]} N[\fst M/x, \snd M/y] : C[(x, y)/w][\fst M/x, \snd M/y]}
\end{mathpar}
And by eta for telescope types, $C[(x, y)/w][\fst M/x, \snd M/y] \equiv C[(\fst M, \snd M)/w] \equiv C[M/w]$.
\end{proof}

Given a term such as $\beta : \St{s}{\telety{\alpha}{A}{B}}$, we will often use $s$-elimination and then split on the pair. For clarity, rather than writing
\begin{align*}
\StE{s}{\beta}{w}{\TeleE{w}{x}{y}{(\dots)}}
\end{align*}
we will write
\begin{align*}
\StE{s}{\beta}{x, y}{(\dots)}
\end{align*}

\begin{lemma}\label{lem:e-elim-telescope-type}
\textsc{s-elim} into types commutes with telescope formation and $\mathsf{U}$-formation, in the following sense:
\begin{align*}
\EEs{y.\nu}{M}{x}{(z : A, B)} &\equiv (z : \EEs{y.\nu}{M}{x}{A}, \EEs{(y, y').(\nu, y')}{(M,z)}{x, z}{B}) \\
\EEs{y.\nu}{M}{x}{\U{c.\mu'[\nu/x]}{z : A}{B}} &\equiv \U{c.\mu'[\mu/x]}{z : \EEs{\nu}{M}{x}{A}}{\EEs{(y, y').(\nu, y')}{(M,z)}{x, z}{B}}
\end{align*}
\end{lemma}
\begin{proof}
For telescopes:
\begin{align*}
&\EEs{y.\nu}{M}{x}{(z : A, B)} \\
&\equiv \EEs{\nu}{M}{x}{(z : \EEs{\nu}{\EIs{\nu}{x}}{x'}{A[x'/x]}, \EEs{(y, y').(\nu, y')}{\EIs{(y, y').(\nu, y')}{x, z}}{x', z'}{B[x'/x, z'/z]})} \\
&\equiv \EEs{\nu}{M}{x}{(z : \EEs{\nu}{\EIs{\nu}{x}}{x'}{A[x'/x]}, \EEs{(y, y').(\nu, y')}{(\EIs{\nu}{x},z)}{x', z'}{B[x'/x, z'/z]})} \\
&\equiv (z : \EEs{\nu}{\EIs{\nu}{x}}{x'}{A[x'/x]}, \EEs{(y, y').(\nu, y')}{(M,z)}{x', z'}{B[x'/x, z'/z]}) \\
&\equiv (z : \EEs{\nu}{\EIs{\nu}{x}}{x}{A}, \EEs{(y, y').(\nu, y')}{(M,z)}{x, z}{B}) 
\end{align*}
and for $\mathsf{U}$-types the proof is similar.
\end{proof}

\begin{lemma}\label{lem:ctxtuple}
Any context $\Gamma$ can be tupled into an iterated telescope type $\ctxtuple{\Gamma}$ of mode $\ctxtuple{\gamma}$, so that substitutions $\Gamma \yields \Theta : \Delta$ correspond bijectively with terms of $\ctxtuple{\Delta}$.
\begin{mathpar}
\inferrule*{\yields_\gamma \Gamma}
             {\cdot \yields_{\ctxtuple{\gamma}} \ctxtuple{\Gamma} \TYPE}
\and
\inferrule*{~}
             {\sigma : \ctxtuple{\Gamma} \yields_{\unpack{\gamma}{\sigma}} \unpack{\Gamma}{\sigma} : \Gamma}
\and
\inferrule*{~}
             {\Gamma \yields_{\pack{\gamma}} \pack{\Gamma} : \ctxtuple{\Gamma}}
\\
\inferrule*{~}
             {\Gamma \yields \unpack{\Gamma}{\pack{\Gamma}} \equiv \id_\Gamma}
\and
\inferrule*{~}
             {\sigma : \ctxtuple{\Gamma} \yields \pack{\Gamma}[\unpack{\Gamma}{\sigma}] \equiv \sigma}
\end{mathpar}
\end{lemma}
\begin{proof}
$\ctxtuple{\Gamma}$ is defined inductively by
\begin{align*}
\ctxtuple{\cdot} &:\equiv 1 \\
\ctxtuple{\Gamma, x : A} &:\equiv \sigmacl{\sigma}{\ctxtuple \Gamma}{A[\unpack{\Gamma}{\sigma}]}\\
\unpack{(\cdot)}{\sigma} &:\equiv \cdot \\
\unpack{\Gamma, x : A}{\sigma} &:\equiv \unpack{\Gamma}{\fst \sigma}, \snd{\sigma}/x
\end{align*}
with each definition lying over the identical one downstairs. 
For $\pack{\Gamma}$ we simultaneously verify the equation $ \unpack{\Gamma}{\pack{\Gamma}} \equiv \id_\Gamma$, as we need it to hold for $(\pack{\Gamma}, x)$ to be well typed.
\begin{align*}
\pack{(\cdot)} &:\equiv \mt : 1 \\
\pack{\Gamma, x : A} &:\equiv (\pack{\Gamma}, x) : \sigmacl{\sigma}{\ctxtuple \Gamma}{A[\unpack{\Gamma}{\sigma}]} \\
\unpack{(\cdot)}{\pack{(\cdot)}} &\equiv  \cdot \equiv \id_{(\cdot)}\\
\unpack{\Gamma, x : A}{\pack{\Gamma, x : A}} 
&\equiv \unpack{\Gamma}{\fst \pack{\Gamma, x : A}}, \snd{\pack{\Gamma, x : A}}/x \\
&\equiv \unpack{\Gamma}{\pack{\Gamma}}, x/x \\
&\equiv \id_\Gamma, x/x \\
&\equiv \id_{\Gamma, x : A}
\end{align*}
For the second equation, we check inductively that
\begin{align*}
\pack{(\cdot)}[\unpack{(\cdot)}{\sigma}] 
&\equiv \mt[\id_{(\cdot)}] \\
&\equiv \sigma \\
\pack{\Gamma, x : A}[\unpack{\Gamma, x : A}{\sigma}]
&\equiv (\pack{\Gamma}, x)[\unpack{\Gamma}{\fst \sigma}, \snd{\sigma}/x] \\
&\equiv (\pack{\Gamma}[\unpack{\Gamma}{\fst \sigma}], \snd \sigma) \\
&\equiv (\fst \sigma, \snd \sigma) \\
&\equiv \sigma
\end{align*}
\end{proof}

%\begin{lemma}
%Tupling respects substitution: for $\gamma \yields \theta : \delta$ and $\delta \yields \kappa : \lambda$, we have:
%\begin{align*}
%\ctxtuple{(\kappa[\theta])} &\equiv (\ctxtuple{\kappa})[\theta]
%\end{align*}
%\end{lemma}
%\begin{proof}
%By induction on the length of $\lambda$:
%\begin{align*}
%\ctxtuple{((\cdot)[\theta])}
%&\equiv \ctxtuple{(\cdot)} \\
%&\equiv (\ctxtuple{(\cdot)})[\theta] \\
%\ctxtuple{((\kappa, M)[\theta])}
%&\equiv \ctxtuple{(\kappa[\theta], M[\theta])} \\
%&\equiv \ctxtuple{(\kappa[\theta]), M[\theta]} \\
%&\equiv (\ctxtuple \kappa)[\theta], M[\theta] \\
%&\equiv (\ctxtuple\kappa, M)[\theta]
%\end{align*}
%\end{proof}

%\begin{definition}
%A 2-cell between mode substitutions of shape $\gamma \yields t : \theta \tcell_\delta \theta' $ is specified by a mode term 2-cell \[\gamma \yields \ctxtuple t : \ctxtuple \theta \tcell_{\ctxtuple \delta} \ctxtuple \theta'. \]
%\end{definition}
%\mvrnote{A little confusing: for contexts and substitutions $\ctxtuple{}$ is an operation, here $\ctxtuple t$ is not an operation on $t$ but rather the underlying `implementation' of $t$.}

%\begin{lemma}\label{lem:n-ary-ap-rewrite}
%N-ary ap and rewrite are admissible:
%\begin{mathpar}
%\inferrule*{\delta \vdash {q} \type \\
%            \gamma \yields t : \theta \tcell_\delta \theta'
%           } 
%           {\TypeTwo{\gamma}{\ap {q} {t}}{q[\theta]}{q[\theta']}}
%\and
%\inferrule*{\delta \yields {\nu} : {q} \\
%            \gamma\yields t : \theta \tcell \theta'
%           } 
%           {\TermTwoT{\gamma}{\ap \nu {t}}{\nu[\theta]}{\TrPlus{\ap{q}{t}}{\nu[\theta]}}{q[\theta]}} \\
% \inferrule*[Left = rewrite]{
%   \Gamma \yields_{\theta'} \Theta : \Delta \and 
%   \gamma \yields t : \theta \tcell \theta'
%  }
%  {\Gamma \yields_{\theta} \rewrite{t}{\Theta} : \Delta}
%\end{mathpar}
%\end{lemma}
%\begin{proof}
%These are
%\begin{align*}
%\ap {q} {t} &:\equiv \ap{q[\unpack{\delta}{\sigma}]}{\ctxtuple t/\sigma} \\
%\ap {\nu} {t} &:\equiv \ap{\nu[\unpack{\delta}{\sigma}]}{\ctxtuple t/\sigma} \\
%\rewrite{t}{\Theta} &:\equiv \unpack{\Delta}{\rewrite{\ctxtuple t}{\ctxtuple \Theta}}
%\end{align*}
%which are well-typed by the equation $\theta \equiv \unpack{\delta}{\ctxtuple \theta}$.
%\end{proof}
%
%In particular, we have the following rules for building and using 2-cells between substitutions.
%\begin{mathpar}
%\inferrule{\gamma \yields t : \theta \tcell_\delta \theta' \and \gamma \yields s : \mu[\theta] \tcell_{p[\theta]} \TrPlus{\ap{p}{t}}{\mu'[\theta']}}
%{ \gamma \yields (s, t) : (\theta, \mu) \tcell_{(\delta,x:p)} (\theta', \mu')} \\
%%
%\inferrule{\gamma \yields t : (\theta, \mu) \tcell_{(\delta, x : p)} (\theta', \mu')}
%{\gamma \yields \ap{\fst}{t} : \theta \tcell_\delta \theta' } \and
%%
%\inferrule{\gamma \yields t : (\theta, \mu) \tcell_{(\delta, x : p)} (\theta', \mu')}
%{\gamma \yields \ap{\snd}{t} : \mu[\theta] \tcell_{p[\theta]} \TrPlus{\ap{p}{\ap{\fst}{t}}}{\mu'[\theta']}}
%\end{mathpar}%
%
%\begin{lemma}
%N-ary ap of a 2-cell on a substitution is admissible
%\begin{mathpar}
%\inferrule{\gamma \yields t : \theta \tcell_\delta \theta' \and \delta \yields \kappa : \lambda}
%{\gamma \yields \ap{\kappa}{t} : \kappa[\theta] \tcell_\lambda \kappa[\theta']}
%\end{mathpar} 
%\end{lemma}
%\begin{proof}
%Due to the equation $\ctxtuple(\kappa[\theta]) \equiv (\ctxtuple \kappa)[\theta]$ we can just define $\ctxtuple{(\ap{\kappa}{t})} :\equiv \ap{(\ctxtuple \kappa)}{t}$.
%\end{proof}

%\begin{lemma}
%N-ary associativity and interchange hold:
%\begin{align*}
%\ap {(\nu[\theta])} {t} &\equiv \ap \nu {\ap \theta {t}} \\
%s[\theta];\ap{\nu'}{t} &\equiv \ap{\nu}{t};\ap{(\TrPlus{\ap{q}{t}}{y})}{s[\theta']/y}
%\end{align*}
%\end{lemma}
%\begin{proof}
%Unwinding definitions we find:
%\begin{align*}
%\ap {(\nu[\theta])} {t}
%&\equiv \ap{\nu[\theta][\fan{\delta}]}{\ctxtuple t/\sigma} \\
%&\equiv \ap{\nu[\fan{\delta}][\ctxtuple \theta / \sigma][\fan{\gamma}]}{\ctxtuple t/\sigma} \\
%&\equiv \ap{\nu[\fan{\delta}]}{\ap{\ctxtuple \theta[\fan{\gamma}]}{\ctxtuple t/\sigma}/\sigma} \\
%&\equiv \ap{\nu[\fan{\delta}]}{\ap{(\ctxtuple \theta)}{t}/\sigma} \\
%&\equiv \ap{\nu[\fan{\delta}]}{\ctxtuple{(\ap \theta {t})}/\sigma} \\
%&\equiv \ap \nu {\ap \theta {t}}
%\end{align*}
%and
%\begin{align*}
%s[\theta];\ap{\nu'}{t} 
%&\equiv s[\theta];\ap{\nu'[\fan{\delta}]}{\ctxtuple t/\sigma} \\
%&\equiv \ap{\nu[\fan{\delta}]}{\ctxtuple t/\sigma} ; \ApPlus{(\ap{q[\fan{\delta}]}{\ctxtuple t/\sigma})}{s[\theta']} \\
%&\equiv \ap{\nu}{t};\ap{(\TrPlus{\ap{q}{t}}{y})}{s[\theta']/y}
%\end{align*}
%\end{proof}

%\begin{lemma}
%A N-ary versions of the equations concerning rewrites hold:
%\begin{align*}
%\St{(\ap{q}{t})}{B[\Theta]} &\equiv B[\rewrite{t}{\Theta}] \\
%\rewrite{\ap{\nu}{t}}{\StI{\ap{q}{t}}{N[\Theta]}} &\equiv N[\rewrite{t}{\Theta}] \\
%\rewrite{\ap{\kappa}{t}}{\Theta;\kappa} &\equiv \rewrite{t}{\Theta};\kappa
%\end{align*}
%\end{lemma}
%\begin{proof}
%\begin{align*}
%B[\rewrite{t}{\Theta}] 
%&\equiv B[\fan{\Delta}[\rewrite{\ctxtuple t}{\ctxtuple \Theta}/\sigma]] \\
%&\equiv B[\fan{\Delta}][\rewrite{\ctxtuple t}{\ctxtuple \Theta}/\sigma] \\
%&\equiv \St{\ap{q[\fan{\delta}]}{\ctxtuple t / \sigma}}{B[\fan{\Delta}][\ctxtuple \Theta/\sigma]} \\
%&\equiv \St{\ap{q}{t}}{B[\Theta]}
%\end{align*}
%And:
%\begin{align*}
%N[\rewrite{t}{\Theta}]
%&\equiv N[\fan{\Delta}[\rewrite{\ctxtuple t}{\ctxtuple \Theta}/\sigma]] \\
%&\equiv N[\fan{\Delta}][\rewrite{\ctxtuple t}{\ctxtuple \Theta}/\sigma] \\
%&\equiv \rewrite{\ap{\nu[\fan{\delta}]}{\ctxtuple t/\sigma}}{\StI{\ap{q[\fan{\delta}]}{\ctxtuple t/\sigma}}{N[\fan{\Delta}][\ctxtuple \Theta/\sigma]}} \\
%&\equiv \rewrite{\ap{\nu}{t}}{\StI{\ap{q}{t}}{N[\Theta]}} \\
%\end{align*}
%And:
%\begin{align*}
%\rewrite{\ap{\kappa}{t}}{\Theta;\kappa}
%&\equiv \fan{\Lambda}[\rewrite{\ctxtuple{(\ap{\kappa}{t})}}{\ctxtuple {(\Theta;\kappa)}}/\sigma] \\
%&\equiv \fan{\Lambda}[\rewrite{\ap{(\ctxtuple \kappa)}{t}}{(\ctxtuple \kappa)[\Theta]}/\sigma] \\
%&\equiv \fan{\Lambda}[(\ctxtuple \kappa)[\rewrite{t}{\Theta}]/\sigma] \\
%&\equiv \fan{\Lambda}[(\ctxtuple \kappa)/\sigma][\rewrite{t}{\Theta}] \\
%&\equiv \kappa[\rewrite{t}{\Theta}] \\
%&\equiv \rewrite{t}{\Theta};\kappa
%\end{align*}
%using the previous equation.
%\end{proof}

%\begin{lemma}
%Rewriting by a tuple is a tuple of rewritings:
%\begin{align*}
%\rewrite{(t, s/x)}{\Theta, M/x} \equiv (\rewrite{t}{\Theta}, \rewrite{s}{\StI{\ap{p}{t}}{M}})
%\end{align*}
%\end{lemma}
%\begin{proof}
%Unwinding definitions:
%\begin{align*}
%\rewrite{(t, s/x)}{\Theta, M/x}
%&\equiv \fan{\Delta, x : A}[\rewrite{\ctxtuple{(t, s/x)}}{\ctxtuple{(\Theta, M/x)}}/\sigma] \\
%&\equiv (\fan\Delta[\fst{\sigma}/\sigma], \snd{\sigma})[\rewrite{(\ctxtuple t, s)}{\ctxtuple\Theta, M}/\sigma] \\
%&\equiv (\fan\Delta[\fst{\sigma}/\sigma], \snd{\sigma})[(\rewrite{\ctxtuple t}{\ctxtuple\Theta}, \rewrite{s}{\StI{\ap{p[\fan{\delta}]}{\ctxtuple t/\sigma}}{N}})/\sigma] \\
%&\equiv (\fan\Delta[\rewrite{\ctxtuple t}{\ctxtuple\Theta}/\sigma], \rewrite{s}{\StI{\ap{p[\fan{\delta}]}{\ctxtuple t/\sigma}}{N}}) \\
%&\equiv (\rewrite{t}{\Theta}, \rewrite{s}{\StI{\ap{p}{t}}{M}})
%\end{align*}
%\end{proof}
%
%As a special case, note that we have
%\begin{align*}
%\rewrite{(\id_\Theta, s/x)}{\Theta, M/x} \equiv (\Theta, \rewrite{s}{M}/x)
%\end{align*}
%
%\mvrnote{Need to say something like: Because the above $n$-ary versions have been derived, from now on we use $\ap{\mu}{s/x, t/y}$ as shorthand for the $n$-ary version}

\bibliographystyle{abbrvnat}
\bibliography{../drl-common/cs}

\end{document}

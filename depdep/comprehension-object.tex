\documentclass[10pt]{article}

\usepackage{fullpage}
\usepackage{amssymb,amsthm,bbm}
\usepackage[centertags]{amsmath}
\usepackage{stackengine}
\stackMath
\usepackage[mathscr]{euscript}
\usepackage{tikz-cd}
\usepackage{mathpartir}
\usepackage[status=draft,inline,nomargin]{fixme}
\FXRegisterAuthor{ms}{anms}{\color{blue}MS}
\FXRegisterAuthor{mvr}{anmvr}{\color{olive}MVR}
\FXRegisterAuthor{drl}{andrl}{\color{purple}DRL}

\newtheorem{theorem}{Theorem}
\newtheorem{proposition}{Proposition}
\newtheorem{lemma}{Lemma}

\theoremstyle{definition}
\newtheorem{definition}{Definition}
\newtheorem{remark}{Remark}

\let\oldemptyset\emptyset%
\let\emptyset\varnothing

\newcommand{\TYPE}{\mathsf{type}}
\newcommand{\CTX}{\,\,\mathsf{ctx}}

\newcommand{\lolli}[3]{#1 \stackrel{#2}{\multimap} #3}

\newcommand{\C}[0]{\mathcal{C}}
\newcommand{\T}[0]{\mathcal{T}}
\newcommand{\fst}[0]{\mathsf{fst}}

\title{Mode Theory for Comprehension Categories}
%% \author{Daniel R. Licata, Mitchell Riley, Michael Shulman}
\date{}

\begin{document}
\maketitle

Here's an attempt to write out what's happening with the mode theory for
a comprehension object in a more familiar notation, and not taking
things to be so strict.  

The analogy I have in mind is ``? is to comprehension category'' as
``multicategory is to cartesian multicategory'' -- what is the basic
structure that you add some operations to (like terminal object and
diagonals) to get a ``normal'' one?

Suppose we have
\begin{mathpar}
  p \, \mathsf{mode} \and
  \emptyset : p \and 
  \alpha : p \vdash T(\alpha) \, \mathsf{mode} \and 
  \alpha : p, x : T(\alpha) \vdash \alpha.x : p \and
\end{mathpar}
This is supposed to start to specify an internal comprehension object in
some category, but if we think about it in $\mathit{Cat}$, then this
means that we have
\begin{itemize}
\item A category of contexts $\C$.  We'll write the objects of $\C$ as
  $\Gamma$ and the morphisms as $\Gamma \vdash \Delta$.
  \begin{mathpar}
      \inferrule*{ }
             {\Gamma \vdash \Gamma}

  \inferrule*
             {\Gamma \vdash \Delta \and
              \Delta \vdash \Psi
             }
             {\Gamma \vdash \Psi}
  \end{mathpar}
  with associativity and unit laws.  

\item A bifibration of types $\T \twoheadrightarrow \C$.  We'll write
  the objects in the fiber over $\Gamma$ as $A : \Gamma \vdash \TYPE$
  and the morphisms in the fiber between two such objects as $A
  \vdash_\Gamma B$.
  \begin{mathpar}
    \inferrule*{ }{A \vdash_\Gamma A}

    \inferrule*{A \vdash_\Gamma B \and B \vdash_\Gamma C}
               {A \vdash_\Gamma C}
  \end{mathpar}
  with associativity and unit laws.

\item Given $\alpha \vDash_p \beta$, we have $T(s)^+ : T(\beta) \to
  T(\alpha)$; we write this as substitution.  The fact that we can
  ``ap'' a function on a term 2-cell says that this acts on terms as
  well.
  \begin{mathpar}
  \inferrule*{A : \Gamma \vdash \TYPE \and
             \theta : \Delta \vdash \Gamma}
            {A[\theta] : \Gamma \vdash  \TYPE}
  \and 
  \inferrule*{A \vdash_\Gamma B \and \Delta \vdash \theta : \Gamma}
             {A[\theta] \vdash_\Delta B[\theta]}
  \and
  A[id] = A \and
  A[\theta][\theta'] = A[\theta';\theta] \\
  \and
  t[id] = t \and
  t[\theta][\theta'] = t[\theta';\theta]
  \and
  id[\theta] = id \and
  (s;t)[\theta] = s[\theta];t[\theta]
  \end{mathpar}
  with functoriality laws for both (the latter type checks because of
  the former).
  
\item Given $\alpha \vDash_p \beta$, we have $T(s)^\circ : T(\alpha) \to
  T(\beta)$; we write this as $\Sigma_{\theta} B$:
\begin{mathpar}
  \inferrule*{B : \Gamma \vdash \TYPE \and
              \theta : \Gamma \vdash \Delta 
  }
  {
    \Sigma_\theta B: \Delta \vdash  \TYPE
  }

  \inferrule*{A \vdash_\Gamma B}
            {\Sigma_\theta A \vdash_{\Delta} \Sigma_\theta B }
\end{mathpar}

The unit and counit of $s^\circ \dashv s^+$ are 
\begin{mathpar}
\inferrule*{ }{B \vdash_{\Gamma} (\Sigma_\theta B)[\theta]}
\and
\inferrule*{ }{\Sigma_\theta (B[\theta]) \vdash_{\Delta} B}
\end{mathpar}

By functoriality and the counit we get the following ``elim''
(c.f. universal morphism definition of adjunction):
\begin{mathpar}
\inferrule*[left=derivable]
           {B \vdash_{\Gamma} C[\theta]}
           {\Sigma_\theta B \vdash_{\Delta} C}
\end{mathpar}

\item The comprehension (and its functorial actions on term 2-cells)
  gives
  \begin{mathpar}
  \inferrule*{ }{\emptyset \CTX}
  
  \inferrule*{\Gamma \CTX \and A : \Gamma \vdash \TYPE}
             {\Gamma.A \CTX}

  \inferrule*{\theta : \Gamma \vdash \Delta \and t : A \vdash_\Gamma B[\theta]}
             {(\theta,t) : \Gamma.A \vdash \Delta.B}
\end{mathpar}
By default, $.$ is astructural, in the sense that when you want to make
a substitution like
\[
\emptyset.A.B \vdash \emptyset.D.E
\]
you don't have any choice about how you divide the variables into the
proofs of $D$ and $E$.

\item By default, $\emptyset$ isn't terminal, but we can make it so:
  \begin{mathpar}
    ! : \Gamma \vdash \emptyset \and
    \theta = !
  \end{mathpar}

\item A terminal object in each fiber:
  \begin{mathpar}
    \inferrule*{ }{1 : \Gamma \vdash \TYPE}
    \and
    \inferrule*{ }{! : A \vdash_\Gamma 1}
    \and
    t = !
    \and
    1_\Gamma[\theta] = 1_\Delta
    \and
  \end{mathpar}

\item There aren't any interesting $\Sigma$'s without some
  substitutions, so we can have projection with a naturality equation:
\begin{mathpar}
   \inferrule*{ }
              {\pi : \Gamma.A \vdash \Gamma}
   \and 
   (\theta,t);\pi^\Delta_B = \pi^\Gamma_A;\theta
\end{mathpar}

However, another way to do this is to make $\Gamma \cong \Gamma.1$
naturally in $\Gamma$:
\begin{mathpar}
\inferrule*{ }{o : \Gamma.1 \vdash \Gamma} \and
\inferrule*{ }{o^{-1} : \Gamma \vdash \Gamma.1} \and
o;o^{-1} = id \and
o^{-1};o = id \and
o_\Gamma;\theta = (\theta,id);o_\Delta
\end{mathpar}
Then we can define a general projection by
\begin{mathpar}
\pi^\Gamma_A := (id_\Gamma,!);o : \Gamma.A \vdash \Gamma.1 \vdash \Gamma  
\end{mathpar}
using the fact that 1 is terminal in the fiber.

TODO: check naturality.  

\item Now we can derive pairing of $\theta : \Gamma \vdash \Delta$ and
  $t : 1 \vdash_\Gamma A[\theta]$ to $(\theta.t) : \Gamma \vdash
  \Delta.A$ by 
  \begin{mathpar}
    (\theta.t) := o^{-1};(\theta,t) : \Gamma \vdash \Gamma.1 \vdash \Delta.A
  \and
  (\theta.t) ; \pi = \theta
  \end{mathpar}
  The equation should follow by naturality and the inverse law for $o$.

\item For the specific case of $\Sigma_{\pi^\Gamma_A} B$, we write
  $\Sigma_A B$, and assert that it's stable under substitution:
\begin{mathpar}
  \inferrule*{B : \Gamma.A \vdash \TYPE 
  }
  {
    \Sigma_A B: \Gamma \vdash  \TYPE
  }
  \and
  (\Sigma_A B)[\theta] = \Sigma_{A[\theta]} {B[\hat{\theta}]}

  \\
  \inferrule*{t : B \vdash_{\Gamma.A} B'}
             {\Sigma_A(t) : \Sigma_A B \vdash_{\Gamma} \Sigma_A B' }
  \and
  (\Sigma_A(t))[\theta] = \Sigma_{A[\theta]}(t[\hat{\theta}])
  \\ 
  \inferrule*{ }{\eta^{A,B} : B \vdash_{\Gamma.A} (\Sigma_A B)[\pi^\Gamma_A]}
  \and
  \eta^{A,B}[\hat{\theta}] = \eta^{A[\theta],B[\hat{\theta}]} : B[\hat{\theta}] \vdash_{\Delta.A[\theta]} (\Sigma_{A[\theta]}{B[\hat{\theta}]}) [\pi^{\Delta}_{A'}]
  \\
  \inferrule*{ }{\epsilon : \Sigma_A (B[\pi^\Gamma_A]) \vdash_{\Gamma} B}
  \and
  \epsilon^{A,B}[\theta] = \epsilon^{A[\theta],B[\theta]} : \Sigma_{A[\theta]} (B[\theta][\pi^{\Delta}_{A[\theta]}]) \vdash_{\Delta} B[\theta]
\end{mathpar}

To type check the third equation, we've used
\[
(\Sigma_A B)[\pi^\Gamma_A][\hat{\theta}]
= (\Sigma_A B)[\theta][\pi^\Delta_{A[\theta]}]
= \Sigma_{A[\theta]}{B[\hat{\theta}]} [\pi^\Delta_{A[\theta]}]
\]
The fourth is similar, using $B[\theta][\pi^{\Delta}_{A[\theta]}]
= B[\pi^{\Gamma}_A][\hat{\theta}]$.

We could also say these are stable under $\Sigma_\theta$ (TODO: type
check these)?  Are these true up to iso already?
\begin{mathpar}
\Sigma_{\hat{\theta}}(B[\pi_A]) \cong (\Sigma_\theta B)[\pi_A[\theta]] \\
\Sigma_{\hat \theta} \eta_B = \eta_{\Sigma_{\theta} B} \\
\Sigma_{\theta} \epsilon_B = \epsilon_{\Sigma_{\hat \theta} B} 
\end{mathpar}

\item We have a map
  \begin{mathpar}
    \inferrule*{\pi : \Gamma.A \vdash \Gamma \and B \vdash_{\Gamma.A} (\Sigma_A B)[\pi] }
               {(\pi,\eta) : \Gamma.A.B \vdash \Gamma.\Sigma_A B}
  \end{mathpar}
  but the converse doesn't necessarily exist.  So we could add it by
  saying that the above is an isomorphism:
  \begin{mathpar}
    \inferrule*{ }
               {pe^{-1} : \Gamma.\Sigma_A B \vdash \Gamma.A.B }
    \and
    pe^{-1}; (\pi,\eta) = id \and
    (\pi,\eta);pe^{-1} = id
  \end{mathpar}
TODO: check/add naturality equations (might already be true because
above is natural).

\item The final thing to add is that $\Sigma_A 1 \cong A$:
  \begin{mathpar}
    \inferrule*{ }
               {\fst : \Sigma_A 1_{\Gamma.A} \vdash_\Gamma A}
    \and 
    \inferrule*{ }
               {\fst^{-1} : A \vdash_\Gamma \Sigma_A 1}
    \and
    \fst;\fst^{-1} = id \and
    \fst^{-1};\fst = id \and
    \fst^{\Gamma,A}[\theta] = \fst^{\Delta,A[\theta]} \and
    \fst^{A};t = \hat{t};\fst^{B}
  \end{mathpar}
  where
  \[
  \inferrule*[left=Derivable]
             { t : A \vdash_\Gamma B }
             { \hat{t} : \Sigma_A 1 \vdash_\Gamma \Sigma_B 1 }
  \]

\item This gives variable projection
  \begin{mathpar}
    v : 1_{\Gamma.A} \vdash_{\Gamma.A} A[\pi]
  \end{mathpar}
  by transposing $\fst : (\Sigma_A 1) \vdash A$.

  More generally, it should give (indeed, be equivalent to) saying that
  $1 \vdash_{\Gamma.A} B[\pi]$ is naturally isomorphic to $A
  \vdash_{\Gamma} B$.  Writing $-^*$ and $-_*$ for the transposes, the
  ``substitution into a variable'' equation $v[\theta.t] = t$ should
  follow from something like: given $s : 1 \vdash_{\Gamma.A} B[\pi]$ and
  $t : A' \vdash_\Gamma A$, then $s[id,t] \equiv (t;s^*)_*$ (which makes
  sense without $\Gamma.1 \cong \Gamma$) and then maybe some other
  diagrams commuting involving that.  TODO: unwind this in terms of
  $\fst$.  

\item TODO: $\eta$ rule?
  
\end{itemize}

\end{document}



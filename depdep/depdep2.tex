\documentclass[10pt]{article}

\usepackage{fullpage}
\usepackage{amssymb,amsthm,bbm}
\usepackage[centertags]{amsmath}
\usepackage[mathscr]{euscript}
\usepackage{tikz-cd}
\usepackage{mathpartir}

\newcommand{\yields}{\vdash}
\newcommand{\cbar}{\, | \,}

\newcommand{\Id}[3]{\mathsf{Id}_{{#1}}(#2,#3)}
\newcommand{\CTX}{\,\,\mathsf{CTX}}
\newcommand{\ctx}{\,\,\mathsf{ctx}}
\newcommand{\TYPE}{\,\,\mathsf{TYPE}}
\newcommand{\type}{\,\,\mathsf{type}}
\newcommand{\TELE}{\,\,\mathsf{TELE}}
\newcommand{\tele}{\,\,\mathsf{tele}}

\newcommand\F[2]{\ensuremath{\mathsf{F}_{#1}(#2)}}
\newcommand\U[3]{\ensuremath{\mathsf{U}_{#1}(#2 \mid #3)}}
\newcommand\St[2]{\ensuremath{{#1}_*(#2)}}
\newcommand\StI[2]{\ensuremath{\mathsf{st}(#1,#2)}}
\newcommand\StE[2]{\ensuremath{\mathsf{unst}(#1,#2)}}
\newcommand\FE[3]{\ensuremath{\mathsf{split} \, #2 \, = \, {#1} \, \mathsf{in} \, #3}}
\newcommand\FI[1]{\ensuremath{\mathsf{F}{(#1)}}}
\newcommand\UE[2]{\ensuremath{#1(#2)}}
\newcommand\UI[2]{\ensuremath{\lambda #1.#2}}
\newcommand\Trd[2]{\ensuremath{#1_*(#2)}}

\newcommand\Set[0]{\ensuremath{\textbf{Set}}}
\newcommand\Hom[3]{\ensuremath{\textbf{hom}_{#1}(#2,#3)}}
\newcommand\just[1]{\ensuremath{\textsf{just}_{#1}}}
\newcommand\Dt[2]{\ensuremath{#1.#2}}

\newcommand\Push[3]{\ensuremath{#1 +_{#2} #3}}
\newcommand\Pushout[5]{\ensuremath{#1 +^{#4,#5}_{#2} #3}}

\title{Adjoint Type Theory}
\author{}
\date{}

\begin{document}
\maketitle

\section{Syntax}

\subsection{Overview of Judgements}

Mode theory judgements:
\begin{itemize}
\item $\gamma \ctx$ (empty, extension)
\item $\gamma \yields \alpha \type$ (generators)
\item $\gamma \yields \delta \tele$ (empty, extension)
\item $\gamma \yields \mu : \alpha$ (variables, generators, ``coercion''
  along $s : \alpha \Rightarrow \beta \type$)
\item $\gamma \yields \theta : \delta$ (tuples)
\item $\gamma \yields s : \mu \Rightarrow \nu : \alpha$ (identity,
  composition, horizontal composition, generators)
\item $\gamma \yields s : \theta \Rightarrow \theta' : \delta$ (tuples)
\item $\gamma \yields s : \alpha \Rightarrow \beta \type$ (identity,
  composition, ap of a type constant on a telescope of 2-cells)
\end{itemize}
Top judgements: 
\begin{itemize}
\item $\yields_\gamma \Gamma \CTX$ over $\yields \gamma \ctx$
\item $\Gamma \yields_\alpha A \TYPE$ over $\gamma \yields \alpha \type$
\item $\Gamma \yields_\mu M : A$ over $\gamma \yields \mu : \alpha$
\item Telescopes $\Gamma \yields_\delta \Delta \TELE$ over $\gamma \yields \delta \tele$
\item Substitutions $\Gamma \yields_\theta \Theta : \Delta$ over $\gamma \yields \theta : \delta$
\end{itemize}

Coercion along 2-cells $s : \alpha \Rightarrow \beta \type$ is
\emph{contravariant} in the mode theory, but $s : \alpha \Rightarrow
\beta \type$ and $s : \mu \Rightarrow \mu : \alpha$ act \emph{covariantly}
on the subcripts of the term.

We expect structurality to be admissible for the base, and structurality
over that to be admissible for the top, e.g.:

\begin{mathpar}
\inferrule*[Left = weaken-over]
           {\Gamma,\Gamma' \yields_\mu M : A (\text{where } \gamma,\gamma' \vdash \mu : \alpha)}
           {\Gamma,y:B,\Gamma' \yields_\mu M : A (\text{where } \gamma,y:\beta,\gamma' \vdash \mu : \alpha)}

\inferrule*[Left = subst-over]
           {\Gamma,x:A,\Gamma' \yields_\nu N : C \and (\text{where } \gamma,x:\alpha,\gamma' \vdash \nu : \gamma) \\\\
            \Gamma \vdash_\mu M : A \and (\text{where } \gamma \vdash \mu : \alpha)
           }
           {\Gamma,\Gamma'[M/x] \yields_{\nu[\mu/x]} N[M/x] : A[M/x] \and (\text{where } \gamma,\gamma'[\mu/x] \vdash \nu[\mu/x] : \alpha[\mu/x])}
\end{mathpar}

\subsection{Mode Theory}

Some of the less obvious rules:

\begin{mathpar}
\inferrule*
    {\gamma \yields \mu : \beta \\
     \gamma \yields \alpha \Rightarrow \beta : \type
    }
    {\gamma \yields s_*(\mu) : \alpha}

\inferrule*
    {\gamma \yields s : \theta \Rightarrow \theta' : \delta \\
      \gamma \yields t : \mu \Rightarrow (\alpha[s])_*(\mu') : \alpha[\mu]}
    {\gamma \yields (s,t/x) : (\theta,\mu/x) \Rightarrow (\theta', \mu'/x) : \delta,x:\alpha}

\inferrule*
    {\gamma, \delta \yields \alpha : \type \\
     \gamma \yields s : \theta \Rightarrow \theta' : \delta}
    {\gamma \yields \alpha[s] : \alpha[\theta] \Rightarrow \alpha[\theta'] : \type}

\inferrule*
    {\gamma, \delta \yields s_1 : \mu \Rightarrow \mu' : \alpha \\
      \gamma \yields s_2 : \theta \Rightarrow \theta' : \delta}
    {\gamma \yields s_1[s_2] : \mu[\theta] \Rightarrow (\alpha[s_1])_*(\mu'[\theta']) : \mu[\theta]}
\end{mathpar}

Equations for $s_*(\alpha)$ are identity, composition -- anything else?

\subsection{Contexts and telescopes (boring lifting to tuples)}

\begin{mathpar}
  \inferrule*[Left = ctx-form]{ }
  {\yields_{\cdot} \cdot \CTX  } \and 

  \inferrule*[Left = ctx-form]{
    \yields_\gamma \Gamma \CTX \and (\text{where } \yields \gamma \ctx) \\\\
    \Gamma \yields_\alpha A \TYPE \and (\text{where }  \gamma \yields \alpha \type)}
  {\yields_{\gamma, x : \alpha} \Gamma, x : A \CTX \and (\text{where } \yields \gamma,x:\alpha \ctx)  } \\

  \inferrule*[Left = tele-form]{ }
             {\Gamma \yields \cdot \TELE_{\cdot}  } \and

  \inferrule*[Left = tele-form]{
    \Gamma \yields_\delta \Delta \TELE \and  (\text{where } \gamma \yields \delta \tele) \\\\
    \Gamma,\Delta \yields_\alpha A \TYPE \and (\text{where } \gamma,\delta \yields \alpha \type)}
  {\Gamma \yields_{\delta, x : \alpha} \Delta, x : A \TELE  \and (\text{where } \gamma \yields \delta,x:\alpha \tele)} \\ \\

  \and

  \inferrule*[Left = sub1]{ }
             {\Gamma \yields_\cdot \cdot : {\cdot}  } \and 
  \inferrule*[Left = sub2]{
    \Gamma \yields_\theta \Theta : \Delta  \and (\text{where } \gamma \yields \theta : \delta) \\\\
    \Gamma \yields_{\mu} M : A[\Theta] \and (\text{where } \gamma,\delta \yields \mu : \alpha[\theta])}
  {\Gamma \yields_{\theta, \mu/x } (\Theta,M/x) : \Delta, x : A  \and (\text{where } \gamma \yields (\theta,\mu/x) : \delta,x:\alpha)} \\ \\
\end{mathpar}


\subsection{Types and Terms}

Note: we could build $s_*$ into the other rules, but there's not much
reason to in natural deduction.  

\begin{mathpar}
  \inferrule*[Left = var]{
    % \yields \Gamma, x : A, \Gamma' \CTX_{\gamma, x : \alpha, \gamma'}
  }
  {\Gamma, x : A, \Gamma' \yields_x x : A \and (\text{where } \gamma,x:\alpha,\gamma' \yields x : \alpha)} \and

 \inferrule*[Left = 2cell]{
   \Gamma \yields_\mu M : A 
   \and \gamma \yields s : (\mu \Rightarrow \nu) : \alpha
  }
  {\Gamma \yields_\nu s_*(M) : A} \\ \\

  \inferrule*[Left = *-form]{
    \Gamma, \yields_\alpha A \TYPE \and (\text{where } \gamma \yields \alpha \type)\\\\
    \and \gamma \yields s : \alpha \Rightarrow \beta : \type
  }{\Gamma \yields_\beta \St{s}{A} \TYPE \and (\text{where } \gamma \yields \beta \type)} \\

  \inferrule*[Left = *-intro]{
    \Gamma \yields_{s_*(\mu)} M : A \and (\text{where } \gamma \yields s_*(\mu) : \alpha)
  }
  {\Gamma \yields_{\mu} \StI{s}{M} : \St{s}{A}
    \and (\text{where } \gamma \yields \mu : \beta)
  } \\
  
  \inferrule*[Left = *-elim]{
    \Gamma \yields_\mu M : \St{s}{A} \and (\text{where } \gamma \yields \mu : \beta)
  }{
    \Gamma \yields_{s_*(\mu)} \StE{s}{M} : A \and (\text{where } \gamma \yields s_*(\mu) : \alpha)
  } \\
  
  \\
  \inferrule*[Left = F-form]{
    \yields_\gamma \Gamma \CTX \and (\text{where } \yields \gamma \ctx)\\\\
    \Gamma \yields_\delta \Delta \TELE \and (\text{where } \gamma \yields \delta \tele) \\\\
    \gamma, \delta \yields \mu : \beta 
  }
  {\Gamma \yields_\beta F_\mu(\Delta) \TYPE \and (\text{where } \gamma \yields \beta \type) } \\
  
  \inferrule*[Left = F-intro]{
    \Gamma \yields_{\theta} \Theta : \Delta 
    \and (\text{where } \gamma \yields {\theta} : \delta)
    %% \and \gamma \yields \nu : \beta 
    %% \and \gamma \yields \mu[\theta] : \beta 
    %% \and \gamma \yields (\nu \Rightarrow \mu[\theta]) : \beta
  }
  {\Gamma \yields_{\mu[\theta]} \FI{\Theta} : F_{\mu}(\Delta) \and (\text{where } \gamma \yields \mu[\theta] : \beta)} \\

  \inferrule*[Left = F-elim]{
    \Gamma, x : F_{\mu}(\Delta) \yields_{\alpha} C : \TYPE \and (\text{where } \gamma, x : \beta \yields \alpha : \type) \and \\
    \Gamma \yields_{\nu} M : F_{\mu}(\Delta) \and (\text{where } \gamma \yields \nu : \beta) \and \\
    \Gamma, \Delta \yields_{\nu' [\mu / x]} N : C [\FI{\Delta/\Delta}/x]
    \and (\text{where } \gamma, \delta \yields \nu' [\mu / x] : \alpha [\mu / x] )}
  {\Gamma \yields_{\nu'[\nu/x]} \FE{M}{\Delta}{N} : C[M/x]  \and (\text{where } \gamma, \beta \yields {\nu'[\nu/x]} : \alpha[\nu/x])} \\
    \\ \\

  \inferrule*[Left = U-form]{
    \Gamma \yields_\delta \Delta \TELE \and (\text{where } \gamma \yields \delta \tele)\\\\
    \Gamma, \Delta \yields_\alpha A \TYPE \and (\text{where } \gamma, \delta \yields \alpha \type)\\\\
    \and \gamma, \delta, c : \beta \yields \mu : \alpha
  }{\Gamma \yields_\beta U_{c.\mu}(\Delta \vert A) \TYPE \and (\text{where } \gamma \yields \beta \type)} \\

  \inferrule*[Left = U-intro]{
    \Gamma, \Delta \yields_{\mu[\nu/c]} N : A \and (\text{where } \gamma,\delta \yields {\mu[\nu/c]} : \alpha)
  }
  {\Gamma \yields_{\nu} \UI \Delta N : U_{c.\mu}(\Delta \vert A)
    \and (\text{where } \gamma \yields \nu : \beta)
  } \\
  
  \inferrule*[Left = U-elim]{
    \Gamma \yields_\nu M : U_{c.\mu}(\Delta \vert A) \and (\text{where } \gamma \yields \nu : \beta)\\\\
    \Gamma \yields_\theta \Theta : \Delta \and (\text{where } \gamma \yields \theta : \delta)\\
  }{
    \Gamma \yields_{\mu[\theta,\nu/c]} \UE{M}{\Theta} : A[\Theta] \and (\text{where } \gamma \yields \mu[\theta,\nu/c] : \alpha[\theta])
  } \\
\end{mathpar}

Note that \St{s}{A} and \U{x.s_*(x)}{\cdot}{A} have the same intro/elim,
but \St{s}{A} has more definitional equalities:
\[
\begin{array}{rcl}
\St{1}{A} & \equiv & A \\
\St{(s;t)}{A} & \equiv & \St{t}{\St{s}{A}}\\
\St{\alpha[s]}{A[\Theta]} & \equiv & A[s_*(\Theta)]
\end{array}
\]

We should also have term definitional equalities over these (like
$\StI{1}{M} \equiv M$, $\StE{1}{M} \equiv M$).

In the final equation, the typing is
\begin{mathpar}
\gamma,\delta \yields \alpha : \type \and 
\Gamma,\Delta \yields_\alpha A : \type \and 
\Gamma \yields s : \theta \Rightarrow \theta' : \delta \and
\Gamma \yields_\theta \Theta : \Delta 
\end{mathpar}
This equation uses the operation
\begin{mathpar}
 \inferrule*[Left = 2cell-subst]{
   \Gamma \yields_\theta \Theta : \Delta 
   \and \gamma \yields s : (\theta \Rightarrow \theta') : \delta
  }
  {\Gamma \yields_{\theta'} s_*(\Theta) : A} 
\end{mathpar}
is admissible. The extension case is given
\begin{mathpar}
(s,t/x) : (\theta,\mu/x) \Rightarrow (\theta',\mu'/x) : \delta,x:\alpha
\and
\yields_{\theta,\mu/x} (\Theta,M/x) : \Delta,x:A
\end{mathpar}
as in the typing rules for these above, and constructs
\[
(s_*(\Theta), \StI{\alpha[s]}{t_*(M)})
\]
Note that the second component uses the same third equation above (at a
smaller context) to type check, because the term is supposed to have
type $A[s_*(\Theta)]$, which we construct via the intro rule for
$\St{(\alpha[s])}{A[\Theta]}$.

\subsection{Mode Theories (Old)}

Here's an idea for how mode theories could work.

\paragraph{Contexts}

In the simply-typed framework, to make a mode $p$ have (tensor)
products, we made it an internal monoid with e.g. a 1-cell $\odot : p,p
\to p$.  Here, to make a mode $p$ ``have dependent types,'' we'll make
it an ``internal display category'' or something like that.

For normal dependent type theory, we'll use two base mode types $p_0
\type$ and $c : p_0 \yields p_1(c) \type$.  Think of $p_0$ as the closed
types/contexts associated with mode $p$, and $p_1(c)$ as the dependent
types (dependent on $c$).  The mode theory says that ``contexts'' are
generated by the empty context, context extension, and projection:
\begin{align*}
&\yields \emptyset_0 : p_0 \\
c : p_0, x : p_1(c) &\yields c.x : p_0 \\
c : p_0, x : p_1(c) & \yields w : c \Rightarrow (c.x) : p_0 \\
\end{align*}
(Note that weakening is in the ``inclusion'' rather than ``projection''
direction, which makes sense 2-cells act covariantly on the top
judgements.)

A typical object-language type formation judgement like 
\[
x : A, y : B, z : C \vdash D \type
\]
is over
\[
x : p_1(\emptyset_0), y : p_1(\emptyset.x), z : p_1(\emptyset.x.y) \vdash p_1(\emptyset.x.y.z) \type
\]
(``$x$ can't use anything, $y$ can use $x$, $z$ can use $x$ and $y$,'' etc.) 

\paragraph{$\Sigma$ types}
The $F$-type for context extension is a ``closed'' $\Sigma$-type:
$\yields_{p_0} \F{x.y}{x : A, y : B}$, if $\yields_{p_0} A \type$ and $x
: A \vdash_{p_1(x)} B \type$ ($A$ is a closed type and $B$ depends only
on it).  To get a $\Sigma$ in context, we need another mode morphism
\[
c : p_0, x : p_1(c), y : p_1(c.x) \yields \Sigma_c(x,y) : p_1(c) 
\]
and then we have $\vdash_{p_1(c)} \F{\Sigma_c(x,y)}{x : A, y : B} \type$
when $\yields_{p_1(c)} A \type$ and $x : A \vdash_{p_1(c.x)} B \type$
relative to an ambient context in which $c : p_0$.

Note:
\[
c : p_0, x : p_1(c), y : p_1(c.x) \yields (p_1[w : c \Rightarrow c.x])_*(y) : p_1(c) 
\]
has that type; is that a $\Sigma$ type?

However $\Sigma$ is defined, we'll probably want various equations about
these, such as associativity
\begin{align*}
c.\Sigma_c(x,y) &\equiv c.x.y \\
\Sigma_c(\Sigma_c(x,y),z) & \equiv \Sigma_c(x,\Sigma_{c.x}(y,z))
\end{align*}
(where the second uses the first to type check)
and that there is a unit 
\[
c : p_0 \yields \emptyset_0 : p_1(c) \\
\]
that has unit laws with $\Sigma$:  
\begin{align*}
c.(\emptyset_0) \equiv c\\
c : p_0, x : p_1(c) \yields \Sigma_c(\emptyset_0,x) \equiv x\\
c : p_0, x : p_1(c) \yields \Sigma_c(x,\emptyset_0) \equiv x
\end{align*}

TODO: cumulativity/projection

TODO: contraction
%% And we'll usually want contraction, e.g.
%% \[
%% c : p_0, x : p_1(c) \yields \Sigma_{c}(x,w_x^*(x)) \equiv x
%% \]
%% (or a directed cell if that's preferable).  

\paragraph{Modalities}

Suppose we have two modes $(p_0,p_1)$ and $(q_0,q_1)$ as above, then an
example modality between them has both 
\begin{align*}
c : p_0 \vdash f_0(c) : q_0\\
c : p_0, x : p_1(c) \vdash f_1(c,x) : q_1(f_0(c))
\end{align*}
$f_0$ says that the modality $f$ acts on contexts/closed types, and
$f_1$ says that the modality also acts on open types (we could omit
$f_1$ if it doesn't), and that the action on types depends through $f_0$
(but of course there's nothing that forces this; it just seems like a
common pattern).

For example, for a Pfenning-Davies style $\Box$ decomposed as
$F(U A)$ between modes $v$ and $t$, the $F$ and $U$ types for $f_1$ give
have formation rules like
\begin{mathpar}
\inferrule*{\Delta \yields A \type_v}
           {\Delta \mid \cdot \yields F(A) \type_t}

\inferrule*{\Delta \mid \cdot \yields A \type_t}
           {\Delta \yields U(A) \type_v}
\end{mathpar}
where $f_0(\Delta)$ is $\Delta \mid \cdot$

We could add product-preserving axioms like:
\begin{align*}
f_0(\emptyset_0) \equiv \emptyset_0\\
f_1(\emptyset_0) \equiv \emptyset_0\\
f_0(c.x) \equiv f_0(c).f_1(x) \\
f_1(\Sigma_c(x,y)) \equiv \Sigma_{f_0(c)}(f_1(x), f_1(y)) \\
\end{align*}

So how does this recover the coherence triangles for the modal stuff? Suppose we have:
\begin{align*}
x_0 : p_0 &\yields f_0(x_0) : p_0 \\ 
x_0 : p_0, x_1 : p_1(x_0) &\yields f_1(x_0, x_1) : p_1(f_0(x_0)) \\
\end{align*}
(These could land in some different $q_i$ if we wanted). And similarly
for $g_i$ and $h_i$. Now suppose we want to construct something
like \[x:p_0, y : p_1(f(x)), z : p_2(g(x), h(y))\] so $y$ depends on $x$
through $f$, and $z$ depends on $x$ and $y$ through $g$ and $h$. More
carefully, we can only directly get
\begin{align*}
x : p_0, y : p_1(f_0(x)) \yields h_1(f_0(x), y) : p_1(h_0(f_0(x))
\end{align*}
and 
\begin{align*}
x : p_0, y : p_1(g_0(x)) \yields p_1(g_0(x).y) \type
\end{align*}
so to write 
\begin{align*}
x : p_0, y : p_1(f_0(x)) \yields p_1(g_0(x).h_1(f_0(x), y))) \type
\end{align*}
we would need to be able to rewrite a $p_1(h_0(f_0(x))$ to a $p_1(g_0(x))$ using a 2-cell $h_0(f_0(x)) \Rightarrow g_0(x)$

\paragraph{Base Types}

I'm assuming that base types (and identity types, $\mathsf{El}(a)$ for
$a : U$, etc.) ``declare their variables'', e.g. $x : P, y : Q
\vdash_{p_1(\emptyset.x.y)} R(x,y) \type$.  Otherwise, the way that
$\Sigma$, $\Pi$ propogate contexts ($p_0$) wouldn't actually stop
anything from type checking, because you could make all of your types
elements of $p_1(\emptyset)$, which can be weakened to whatever you
want.

\paragraph{$\Pi$-types}

For $\Pi$ types, the guess is $\U{z.\Pi_c(z,x)}{x : A}{B}$ for something
of the following type:
\[ 
c : p_0, x : p_1(c), z : p_1(c) \yields \Pi_c(z,x) : p_1(c.x) 
\]

TODO: fill in triple-adjunction idea

\paragraph{Term Judgements}  

TODO: fill in

\subsection{Non-$p_0/p_1$ Try at Mode Theory}

We have the following judgements and intended semantics:

\begin{itemize}
\item $\gamma \ctx$ : $\gamma$ is a finite inverse category.  

\item We won't use it in the syntax, but the derived notion $\gamma
  \yields \theta : \gamma'$ of a morphism between contexts would be
  interpreted as a profunctor $\gamma \nrightarrow \gamma'$,
  i.e. $\gamma'^{op} \times \gamma \to \Set$

\item $\gamma \yields \alpha \type$ : a functor $\alpha : \gamma \to
  \Set$, which is equivalent to a profunctor $\gamma \nrightarrow 1$.  

\item $\gamma \yields \mu : \alpha$ : There is a projection profunctor
  $\gamma,x:\alpha \yields \mathsf{p} : \gamma$ and a term is
  equivalently (1) a section of this, or (2) this unpacks to a functor
  $\mu : \gamma \to \Set$ and a natural transformation $\alpha
  \Rightarrow \mu$.  (I.e. the term bundle bounds the type bundle, as
  Jason suggested way back at the beginning.)

\item $\gamma \yields s : \alpha \Rightarrow \beta \type$ : (guess)
  natural transformations $\alpha \Rightarrow \beta$ -- or is it
  supposed to be only some natural transformations?  Syntactically, we
  only want the ones that should induce structural rules upstairs.

\item $\gamma \yields s : \mu \Rightarrow \nu : \alpha$ : (guess) a
  natural transformation $s : \mu \Rightarrow \nu$ that makes a triangle
  with the $\alpha \Rightarrow \mu$ and $\alpha \Rightarrow \nu$ parts
  of the term.

%% TODO: 
\item $\gamma \yields \delta \tele$ (tuples, semantically the same as
  a type?)
\item $\gamma \yields \theta : \delta$ (tuples, semantically the same as
  a term?)
\item $\gamma \yields s : \theta \Rightarrow \theta' : \delta$ (tuples,
  semantically the same as a term 2-cell?)
\end{itemize}

Structural rules:
\begin{itemize}

\item The empty finite inverse category:
\begin{mathpar}
\inferrule{ } 
{\emptyset \ctx}
\end{mathpar}

\item The collage of $\alpha$:
\begin{mathpar}
\inferrule{ \gamma \ctx \and \gamma \yields \alpha}
          { \gamma, x:\alpha \ctx}
\end{mathpar}
I.e. the category whose objects are $\gamma + 1$, with
$\Hom{}{\mathsf{inr}()}{\mathsf{inl}(y)}$ given by $\alpha(y)$, no maps
in the other direction, only the identity for ${\mathsf{inr}()}$, and
agreeing with $\gamma$ on $\mathsf{inl}s$.  
Note that $\gamma,x:\alpha$ is disjunctive rather than conjunctive,
which is why some of the rules are weird.  I.e. adding a new object with
maps to things chosen from $\gamma$ by $\alpha$. 

\item Semantically, there is a \textbf{functor} $i : \gamma \rightarrow
  \gamma,x:\alpha$ given by the inclusion $\mathsf{inl}$ of $\gamma$
  into the collage of $\alpha$.  As with any functor, this determines
  profunctors
\begin{align*}
i : \gamma \nrightarrow (\gamma,x:\alpha) := (a : {\gamma,x:\alpha} ,b : \gamma) \mapsto \Hom{\gamma,x:\alpha}{a}{\mathsf{inl}(b)}\\
p :  (\gamma,x:\alpha) \nrightarrow \gamma := (a : \gamma, b : {\gamma,x:\alpha}) \mapsto \Hom{\gamma,x:\alpha}{\mathsf{inl}(a)}{b}
\end{align*}
with $i \dashv p$.  Using the definition of the collage, we can show
that $p \circ i : \gamma \nrightarrow \gamma$ is (naturally isomorphic
to?) the identity profunctor (hom) (while $i \circ p : (\gamma,x:\alpha)
\nrightarrow (\gamma,x:\alpha)$ is not, but it has a natural
transformation to the identity, which the counit of the adjunction).
Thus, $i$ is a section of $p$, and therefore should determine a term.

Syntactically, we make $p$ implicit by an admissible structural rule
\[
\inferrule*[Left = admissible]
          { \gamma,\delta \vdash J }
          { \gamma,x:\alpha,\delta \vdash J}
\]

We write $i$ as 
\[
\inferrule{ }
          {\gamma \vdash \just{\alpha} : \alpha}
\]
because, under the semantics of a term as a functor $\gamma \to \Set$
with a natural transformation to $\alpha$, it is $\alpha$ and the
identity natural transformation. (TODO: check this)

\item Tranformations between types act contravariantly:
\[
\inferrule{ \gamma \yields \mu : \beta \and \gamma \yields s : \alpha
  \Rightarrow \beta \type }
{ \gamma \yields s_*(\mu) : \alpha }
\]

Semantically, if we have $\beta \Rightarrow \mu$ and $\alpha \Rightarrow
\beta$ we can compose the two natural transformations.  That is, we are
weakening the ``type bundle'' without changing the ``term bundle''.

\item Since a term is (equivalent to) a section of the projection,
  substitution should act as usual:
\[
\inferrule*[Left = admissible]
           {\gamma,x:\alpha,\delta \yields J \and \gamma \yields \mu : \alpha}
           {\gamma,\delta[\mu/x] \yields J[\mu/x]}
\]
 
\item We should have identity, composition, and horizontal compostion
  rules for $\alpha \Rightarrow \beta \type$ and $\mu \Rightarrow \nu :
  \alpha$, since they're natural transformations, or triangles of
  natural transformations.
 
\end{itemize}

Types:
\begin{itemize}

\item The constantly $\emptyset$ functor
\begin{mathpar}
\inferrule{ }
          { \gamma \yields \emptyset_0 \type }

\inferrule{ }
          {! : \emptyset_0 \Rightarrow \alpha \type}

s \equiv \mathord{!} : \emptyset_0 \Rightarrow \alpha
\end{mathpar}

\item The semantics of a term judgement $\gamma \yields \mu : \alpha$ is
  the same as the semantics of a ``subtyping'' judgement $\gamma \yields
  \alpha \Rightarrow \beta$: a term $\mu$ denotes a functor $\mu :
  \gamma \to \Set$ which \emph{is} a type, along with a natural
  transformation $\alpha \Rightarrow \mu$.  That is, all terms determine
  types, or all types are universes?  We do this ``a la Tarski'' with

\begin{mathpar}
\inferrule{\gamma \yields \mu : \alpha}
          {\gamma \yields \Dt \alpha \mu \type}

\inferrule{ } 
          {\mathsf{w} : \alpha \Rightarrow \alpha.\mu \type}
\end{mathpar}

Note $\gamma,x:\alpha$ is the collage of $\alpha$, $\alpha.\mu$ is an
extended profunctor to 1, like $c.x$ in the old version.  The 2-cell
acts to give 
\[
\inferrule*[Left = derivable]
           {\gamma \yields \nu : \alpha.\mu }
           {\gamma \yields \mathsf{w}_*(\nu) : \alpha}
\]
i.e. $\alpha \Rightarrow \nu$ because $\alpha.\mu \Rightarrow \nu$ and
$\alpha \Rightarrow \alpha.\mu$.  

We probably have (not sure about strictness)
\[
\Dt{\alpha}{\just{\alpha}} \equiv \alpha
\]
because $\just{\alpha}$ denotes $\alpha \Rightarrow \alpha$ and
$\Dt{\alpha}{\just{\alpha}}$ takes the right-hand side of that natural
transformation.

We probably also have
\[
\alpha.s_*(\mu) \equiv \beta.\mu
\]
because both are just selecting the ``top'' of $\mu$.  

With this, we can do linear dependency, e.g. the context
\[
a : A, b : B(a), c : C(a,b), d : D(a,b,c) 
\]
is 
\[
a : (\emptyset_0), b : (\emptyset_0.a), c : (\emptyset_0.a.b), d : (\emptyset_0.a.b.c)
\]

I was trying to think of ways to make elements of $\alpha.\mu$.  Any
element of $\alpha$ that is bigger than $\mu$ is (determines) an
element of it:
\[
\inferrule*[Left = derivable]
           {\gamma \vdash \nu : \alpha \and \gamma \vdash s: \mu \Rightarrow \nu : \alpha}
           {\gamma \vdash (\alpha.s)_*(\just{\alpha.\nu}) : \alpha.\mu}
\]

\item Pushouts: Given a cospan of types and natural transformations, we
  can take the pushout in $\gamma \to \Set$:

\begin{mathpar}
\inferrule{\gamma \vdash \beta, \alpha_1, \alpha_2  \type \and
           \gamma \vdash s_i : \beta \Rightarrow \alpha_i}
          {\gamma \vdash \Pushout{\alpha_1}{\beta}{\alpha_2}{s_1}{s_2} \type}

\inferrule{ }
      {\gamma \vdash \mathsf{inl} : \alpha_1 \Rightarrow \Push{\alpha_1}{\beta}{\alpha_2}}

\inferrule{ }
      {\gamma \vdash \mathsf{inr} : \alpha_2 \Rightarrow \Push{\alpha_1}{\beta}{\alpha_2}}

\inferrule{ 
      \gamma \vdash s_1' : \alpha_1 \Rightarrow \beta \and 
      \gamma \vdash s_2' : \alpha_2 \Rightarrow \beta \and
      s_1;s_1' = s_2;s_2'
      }
      {\gamma \vdash {[s_1',s_2']} : \Pushout{\alpha_1}{\alpha}{\alpha_2}{s_1}{s_2} \Rightarrow \beta}
\end{mathpar}
Plus $\beta/\eta$ equations.  

TODO: Now we can do DAGs of dependencies, which do and don't (take
$\alpha = \emptyset_0$ to get a coproduct) commute.  

%% maybe this doesn't make sense:
%% We have a way of constructing terms that comes from the universal
%% property of the pushout: to give something bigger than $\alpha_1 +_\beta
%% \alpha_2$, we give two things bigger than $\alpha_1$ and $\alpha_2$
%% that agree:
%% \[
%% \inferrule{\gamma \vdash \mu_1 : \alpha_1 \and 
%%            \gamma \vdash \mu_2 : \alpha_2 \and 
%%            \gamma \vdash {s_1}_*(\mu_1) \equiv {s_2}_*(\mu_2) : \beta}
%%           {\gamma \vdash \mu_1 +_1 \mu_2 : \Pushout{\alpha_1}{\alpha}{\alpha_2}{s_1}{s_2}}
%% \]
\end{itemize}

Note: cell gluing seems like it could be
\[
\infer{\gamma \vdash \mu : \alpha \and
       \gamma \vdash s : \alpha \rightarrow \beta}
      { 
        \gamma \vdash (\beta +_s \mu) :=
        \Pushout{\beta}{\alpha}{(\alpha.\mu)}{s}{\mathsf{w}} \type
      }
\]

Write $\alpha + \beta$ for \Pushout{\alpha}{\emptyset_0}{\beta}{!}{!}.
From the universal properties of $\emptyset_0,+$ we get:
\begin{itemize}
\item ! : $\emptyset \Rightarrow \alpha$
\item associativity: $(\alpha_1 + \alpha_2) + \alpha_3  \Leftrightarrow \alpha_1 + (\alpha_2 + \alpha_3)$
\item unit: $\emptyset + \alpha \Leftrightarrow \alpha \Leftrightarrow \alpha + \emptyset$
\item exchange: $\alpha + \beta \Leftrightarrow \beta + \alpha$
\item contraction: $\mathsf{c} : \alpha + \alpha \Rightarrow \alpha$
\end{itemize}
so the mode theory has something that (because the variance is flipped)
will induce a structural product on types.  

We also have ``in context'' versions that could be used to give
fiberwise product types that depend through the corresponding products:
\begin{itemize}
\item $\just{\emptyset_0} : \emptyset_0$ (like $u : p_1(\cdot)$ from before)
\item $x : \alpha, y : \beta \vdash x +_1 y : \alpha + \beta$, where 
\[x + y := (\mathsf{w} + \mathsf{w})_* (\just{\alpha.x + \beta.y})
\]
using $\mathsf{w} + \mathsf{w} : \alpha + \beta \Rightarrow {\alpha.x +
  \beta.y}$ This satisifes $(\alpha+\beta).(x +_1 y) \equiv \alpha.x +
\beta.y$ because the 2-cell doesn't change the ``top'' of the \just{}.
\end{itemize}


  
%% c : p0, x : p1(c), y : p1(c) ⊢ ↑_x(y) : p1(c.x)
%% --> if x is the kind of thing that can be dotted with c,
%%     and y bounds c, then ↑_x(y) bounds c.x

%% Is this the pushout?  We have c ⇒ x and c ⇒ y and we want c.x = x ⇒
%% ↑_x(y), so we can pushout x and y under c?  In the sets = bool case
%% this would again just be the union of x and y.

\end{document}

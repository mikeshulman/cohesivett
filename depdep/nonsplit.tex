\documentclass[10pt]{article}

\usepackage{fullpage}
\usepackage{amssymb,amsthm,bbm}
\usepackage[centertags]{amsmath}
\usepackage{stackengine}
\stackMath
\usepackage[mathscr]{euscript}
\usepackage{tikz-cd}
\usepackage{mathpartir}
\usepackage[status=draft,inline,nomargin]{fixme}
\FXRegisterAuthor{ms}{anms}{\color{blue}MS}
\FXRegisterAuthor{mvr}{anmvr}{\color{olive}MVR}
\FXRegisterAuthor{drl}{andrl}{\color{red}DRL}

\newtheorem{theorem}{Theorem}
\newtheorem{proposition}{Proposition}
\newtheorem{lemma}{Lemma}

\let\oldemptyset\emptyset
\let\emptyset\varnothing

\newcommand{\yields}{\vdash}
\newcommand{\cbar}{\, | \,}

\newcommand{\Id}[3]{\mathsf{Id}_{{#1}}(#2,#3)}
\newcommand{\CTX}{\,\,\mathsf{Ctx}}
\newcommand{\TYPE}{\,\,\mathsf{Type}}
\newcommand{\TELE}{\,\,\mathsf{Tele}}

\newcommand{\id}{\mathsf{id}}
\DeclareMathOperator{\ob}{ob}

\newcommand\vsubst[2]{#1 \, \{ #2 \}}
\newcommand\p[1]{\mathsf{p}_{#1}}
\newcommand\ins[2]{\mathsf{in}_{#1}(#2)}
\newcommand\outs[2]{\mathsf{out}_{#1}(#2)}

\title{Non-Split Type Theory}
\author{}

\begin{document}
\maketitle

Suppose you have a non-split comprehension category; what does its
internal type theory look like?  We should have contexts $\Gamma$ and
substitutions $\Gamma \vdash \theta : \Delta$, where composition of
substitutions $\theta[\theta']$ is strictly unital and associative,
because the base category of contexts is a category.  We should have
dependent types $\Gamma \vdash A \TYPE$.  But the substitution into
types $A[\theta]$ (pullback of a type along a morphism in the base) is
only determined up to isomorphism, so it should be its own type
constructor, and there shouldn't be any definitional equalities relating
it to other types.  E.g. $(\Pi x:A.B)[\theta]$ should be isomorphic to
$\Pi x:A[\theta].B[\theta,x/x]$ but not definitionally equal to it.
On the other hand, the equational theory of \emph{terms} should be
exactly as usual, since that's happening in the base category of
contexts.   

A nice way to do this is to have virtual pullbacks in the judgement, so
that the explicit substitution types can be given a universal property
relative to those.  So a term judgement looks like
\[
\Gamma \vdash a : \vsubst A \theta \text { where } \Delta \vdash A \TYPE
\text{ and } \Gamma \vdash \theta : \Delta
\]
The idea is that the $\vsubst A \theta$ is judgemental punctuation
corresponding to the pullback $\theta^*(A)$.  In a non-virtual model,
such a term $a$ corresponds to a section of the projection
$\Gamma.A[\theta] \to \Gamma$, or equivalently a map from $\Gamma \to
\Delta.A$ that has $\theta$ in its first component.  

Each entry in $\Gamma$ also has a virtual pullback.

I think rules like this have been used before in explicit substitution
calculi (it reminds me of what I've heard people say about how the Twelf
implementation manages variables, which goes back to Bob's
implementation of LF; also of Thierry's weak type theory for the
semisimplicial model from the end of the IAS year), but I don't think
I've seen the virtual/judgemental approach written out.  And usually
people don't push the \emph{term} substitution under binders because
they're thinking of explicit subsitutions as closures.

\paragraph{Contexts}

\begin{mathpar}
\inferrule*{ }
           { \cdot \CTX }

\inferrule*{ \Gamma \CTX \and
             \Delta \CTX \and
             \Delta \vdash A \TYPE \and
             \Gamma \vdash \theta : \Delta
           }
           { \Gamma, x : \vsubst A \theta \CTX}
\end{mathpar}

\paragraph{Substitutions}

\begin{mathpar}
\inferrule*{ }
           { \Gamma \vdash \cdot : \cdot \CTX }

\inferrule*{ \Gamma \vdash \theta' : \Delta \and
             \Gamma \vdash a : \vsubst A {\theta[\theta']}
           }
           { \Gamma \vdash (\theta',a/x) : \Delta, x : \vsubst A {\theta}}
\end{mathpar}

These could be admissible, but we should have
\begin{mathpar}
  \inferrule*{ }
             {\Gamma \vdash 1 : \Gamma}

  \inferrule*{ \Gamma_1 \vdash \theta_1 : \Gamma_2 \and \Gamma_2 \vdash \theta_2 : \Gamma_3  }
  { \Gamma_1 \vdash \theta_2[\theta_1] : \Gamma_3 }
  \\
  1[\theta] \equiv \theta \and
  \theta[1] \equiv \theta \and
  (\theta_1[\theta_2])[\theta_3] \equiv (\theta_1[\theta_2[\theta_3]])
  \\
  \cdot[\theta] \equiv \cdot \and
  (\theta,a/x)[\theta'] \equiv (\theta[\theta'],a[\theta']/x)
\end{mathpar}

Projections (could also be admissible):
\begin{mathpar}
  \inferrule*{ }
             {\Gamma,\Gamma' \vdash \p{\Gamma'} : \Gamma}
  \\ 
  \p{x:\vsubst{A}{\theta}}[\theta',a] \equiv \theta' \and
  \theta : \cdot \equiv \cdot \and
  \theta' : (\Delta, x : \vsubst A {\theta}) \equiv (\p{x:\vsubst A \theta}{[\theta']}, x[\theta']/x)
\end{mathpar}


\paragraph{Dependent types}

\begin{mathpar}
  \inferrule*{ \Delta \vdash A \TYPE \and
               \Gamma \vdash \theta : \Delta
             }
             { \Gamma \vdash A[\theta] \TYPE }

\inferrule* { \Gamma \vdash A \TYPE \and
             \Gamma, x : \vsubst A 1 \vdash B \TYPE}
           { \Gamma \vdash \Pi x:A.B \TYPE }

\inferrule* { \Gamma \vdash A \TYPE \and
             \Gamma, x : \vsubst A 1 \vdash B \TYPE}
           { \Gamma \vdash \Sigma x:A.B \TYPE }
\end{mathpar}

We could also have a primitive compound type $\Pi x:A[\theta].B$ which
would avoid special-casing $1$ in the typing rule, and similarly for $\Sigma$.  

\paragraph{Terms}
Definitional equalities are always between two terms $\Gamma \vdash a
\equiv a' : \vsubst A \theta$ in the same virtual pullback. 

Substitution (could be admissible):
\begin{mathpar}
  \inferrule*{ \Gamma \vdash a : \vsubst A \theta   }
             { \Gamma \vdash a[\theta'] : \vsubst A {\theta[\theta']}  }

  a[\theta][\theta'] \equiv a[\theta[\theta']] \and
  a[1] \equiv a
\end{mathpar}

Variables:
\begin{mathpar}
\inferrule*{ }
           { \Gamma_1,x:\vsubst A \theta,\Gamma_2 \vdash x : \vsubst A {\theta[\p{x:\vsubst A \theta,\Gamma_2}]}  }
\and
x[\theta_1,a,\theta_2] \equiv a
\end{mathpar}

Explicit substitution types:
\begin{mathpar}
  \inferrule*{ \Gamma \vdash a : \vsubst A {\theta[\theta']}}
            { \Gamma \vdash \ins{\theta}{a} : \vsubst{A[\theta]}{\theta'} }

  \inferrule*{ \Gamma \vdash a : \vsubst{A[\theta]}{\theta'} }
             { \Gamma \vdash \outs{\theta}{a} : \vsubst A {\theta[\theta']} }
  \\
  \ins{\theta}{a}[\theta''] \equiv \ins{\theta}{a[\theta'']} \and
  \outs{\theta}{a}[\theta''] \equiv \outs{\theta}{a[\theta'']} \and
  \ins{\theta}{\outs{\theta}{a}} \equiv a \and 
  \outs{\theta}{\ins{\theta}{a}} \equiv a \and
\end{mathpar}
(Type checking the substitution equations uses strict associativity of
composition in $\Gamma$.)  These could also be treated positively,
e.g. if we wanted only weak $\eta$ for them
($\Gamma,\vsubst{A[\theta]}{\theta'} \vdash J$ if
 $\Gamma,\vsubst{A}{\theta[\theta']} \vdash J$ as the left rule).  

For $\Pi$, one way to do it is to build the $(\Pi x:A.B)[\theta] \cong
\Pi x:A[\theta].B[\theta,x]$ isomorphism into the rules (like how the
adjoint logic rules for coproducts that build in commuting with $F$):
  \begin{mathpar}
  \inferrule*{ \Gamma,x : \vsubst A {\theta}  \vdash b : \vsubst B {\theta[\p{x}], x/x} }
             { \Gamma \vdash \lambda x.b : \vsubst {\Pi x:A.B}{\theta} }
  \and
  \inferrule*{ \Gamma \vdash f : \vsubst {\Pi x:A.B}{\theta}
               \and
               \Gamma \vdash a : \vsubst A \theta
             }
             { \Gamma \vdash f(a) : \vsubst B {\theta,a/x} }
  \\
  (\lambda x.b)[\theta'] \equiv \lambda x.b[\theta',x/x] \and
  f(a) [\theta'] \equiv f[\theta'](a[\theta']) \and
  (\lambda x.b)(a) \equiv b[1_\Gamma,a/x] \and
  f \equiv \lambda x.(f[\p{x}])(x)
  \end{mathpar}
  (The $\lambda$ rule's premise uses strict unit $1[\theta]$ to type check.)
  An alternative is to have $\lambda$ and application only for $a :
  \vsubst{\Pi x:A.B}{1}$ and put in the isomorphism separately.

For $\Sigma$:
  \begin{mathpar}
  \inferrule*{ \Gamma \vdash a : \vsubst A {\theta}
               \and
               \Gamma \vdash b : \vsubst A {\theta,a/x}
             }
             { \Gamma \vdash (a,b) : \vsubst {\Sigma x:A.B} {\theta} }
  \and
  \inferrule*{ \Delta \vdash (\Sigma x:A.B) \TYPE \and
               \Delta, p : \vsubst{(\Sigma x:A.B)} {1} \vdash C \TYPE \and
               \Gamma \vdash \theta : \Delta
               \\ 
               \Gamma,x : \vsubst A {\theta}, y: \vsubst B {\theta[\p{x}],x/x} \vdash c : \vsubst C {\theta[\p{x,y}], (x,y)/p } \and
               \Gamma \vdash a : \vsubst {(\Sigma x:A.B)} {\theta} 
             }
             { \Gamma \vdash \mathsf{split}_{p.C}(x.y.c,a) : \vsubst {C}{\theta,a/p} }
  \\
  (a,b)[\theta'] \equiv (a[\theta'],b[\theta']) \and
  \mathsf{split}_{p.C}(x.y.c,a) [\theta'] \equiv \mathsf{split}_{p.C}(x.y.c[\theta'[\p{x,y}],x/x,y/y],a[\theta']) \and
  \mathsf{split}_{p.C}(x.y.c,(a,b)) \equiv c[1,a/x,b/y]
  \end{mathpar}
  
\end{document}

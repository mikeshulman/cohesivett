\documentclass[10pt]{article}
  \usepackage{xcolor}
  \definecolor{darkgreen}{rgb}{0,0.45,0} 
  \usepackage[pagebackref,colorlinks,citecolor=darkgreen,linkcolor=darkgreen]{hyperref}

\usepackage{fullpage}
\usepackage{amssymb,amsthm,bbm}
\usepackage[centertags]{amsmath}
\usepackage[mathscr]{euscript}
\usepackage{tikz-cd}
\usepackage{mathpartir}
\usepackage{enumitem}
\usepackage[status=draft,inline,nomargin]{fixme}
\FXRegisterAuthor{ms}{anms}{\color{blue}MS}
\FXRegisterAuthor{mvr}{anmvr}{\color{olive}MVR}
\FXRegisterAuthor{drl}{andrl}{\color{purple}DRL}
\usepackage{stmaryrd}
\usepackage{mathtools}

\newtheorem{theorem}{Theorem}
\newtheorem{proposition}{Proposition}
\newtheorem{lemma}{Lemma}
\newtheorem{corollary}{Corollary}
\newtheorem{problem}{Problem}
\newenvironment{constr}{\begin{proof}[Construction]}{\end{proof}}

\theoremstyle{definition}
\newtheorem{definition}{Definition}
\newtheorem{remark}{Remark}
\newtheorem{example}{Example}

\let\oldemptyset\emptyset%
\let\emptyset\varnothing

\newcommand\dsd[1]{\ensuremath{\mathsf{#1}}}

\newcommand{\yields}{\vdash}
\newcommand{\Yields}{\vDash}
\newcommand{\cbar}{\, | \,}
\newcommand{\judge}{\mathcal{J}}

\newcommand{\Id}[3]{\mathsf{Id}_{{#1}}(#2,#3)}
\newcommand{\CTX}{\,\,\mathsf{Ctx}}
\newcommand{\ctx}{\,\,\mathsf{mctx}}
\newcommand{\TYPE}{\,\,\mathsf{Type}}
\newcommand{\type}{\,\,\mathsf{mode}}
\newcommand{\TELE}{\,\,\mathsf{Tele}}
\newcommand{\tele}{\,\,\mathsf{mtele}}

\newcommand{\app}[2]{\ensuremath{#1 \: #2}}
\newcommand{\telety}[3]{\ensuremath{(#1{:}#2,#3)}}
\newcommand{\mt}[0]{\ensuremath{()}}
\newcommand{\sigmacl}[3]{\ensuremath{\textnormal{$\Sigma$}\,#1{:}#2.\,#3}}
\newcommand{\fst}[1]{\app{\dsd{fst}}{#1}}
\newcommand{\snd}[1]{\app{\dsd{snd}}{#1}}
\newcommand\extend[2]{\ensuremath{(#1,\id_{#2})}}

\newcommand\fan[1]{\ensuremath{\mathsf{fan}_{#1}}}

\newcommand{\id}{\mathsf{id}}
\DeclareMathOperator{\ob}{ob}

\newcommand{\rewrite}[2]{\overleftarrow{#1}(#2)}
\newcommand\F[2]{\ensuremath{\mathsf{F}_{#1}(#2)}}
\newcommand\U[3]{\ensuremath{\mathsf{U}_{#1}(#2 \mid #3)}}
\newcommand\UE[2]{\ensuremath{#1(#2)}}
\newcommand\UI[2]{\ensuremath{\lambda #1.#2}}
\newcommand\St[2]{\ensuremath{{#1}^*(#2)}}
\newcommand\StI[2]{\ensuremath{\mathsf{st}_{#1}(#2)}}
\newcommand\UStI[2]{\ensuremath{\mathsf{ust}_{#1}(#2)}}
\newcommand\UnSt[2]{\ensuremath{\mathsf{unst}_{#1}(#2)}}
%\newcommand\StE[2]{\ensuremath{\mathsf{unst}(#1,#2)}}
\newcommand\StE[4]{\ensuremath{\mathsf{let} \, \StI{#1}{#3} \, = \, {#2} \, \mathsf{in} \, #4}}
\newcommand\FE[3]{\ensuremath{\mathsf{let} \, \mathsf{F}(#2) \, = \, {#1} \, \mathsf{in} \, #3}}
% With subscript:
\newcommand\FEs[4]{\ensuremath{\mathsf{let} \, \mathsf{F}_{#1}(#3) \, = \, {#2} \, \mathsf{in} \, #4}} 
\newcommand\FI[1]{\ensuremath{\mathsf{F}{(#1)}}}
\newcommand\TypeTwo[4]{\ensuremath{#1 \mid #3 \vDash #2 : #4}}
\newcommand\TeleTwo[4]{\ensuremath{#1 \mid #3 \vDash #2 : #4}}
\newcommand\TermTwo[4]{\ensuremath{#1 \mid #3 \vDash #2 : #4}}
\newcommand\TermTwoT[5]{\ensuremath{#1 \mid #3 \vDash_{#5} #2 : #4}}
%% \newcommand\TermTwoDisp[5]{\ensuremath{#1 \mid #3 \vDash_{\mathsf{disp}} #2 :_{#5} #4}}
\newcommand\SubTwo[4]{\ensuremath{#1 \mid #3 \vDash #2 : #4}}
\newcommand\TrPlus[2]{\ensuremath{{#1}^+(#2)}}
\newcommand\TrCirc[2]{\ensuremath{{#1}^\circ(#2)}}

\newcommand\El[2]{\mathcal{T}_{#1}(#2)}
\newcommand\ApEl[2]{\mathcal{T}_{#1}\langle#2\rangle}

\newcommand\ap[2]{\ensuremath{#1 \langle #2 \rangle }}
\newcommand\ApPlus[2]{\ensuremath{{#1}^+ \langle #2 \rangle }}
\newcommand\ApCirc[2]{\ensuremath{{#1}^\circ \langle #2 \rangle }}

% Macros for semantics notation
\newcommand\mm[1]{\llbracket #1 \rrbracket}
\newcommand\op{^{\mathrm{op}}}
\newcommand\co{^{\mathrm{co}}}
\newcommand\coop{^{\mathrm{coop}}}
\newcommand\Cat{\mathrm{Cat}}
\newcommand\CAT{\mathrm{CAT}}
\newcommand\M{\mathcal{M}}
\newcommand\Mty{{\mathrm{Ty}_{\M}}}
\newcommand\Mtm{{\mathrm{Tm}_{\M}}}
\newcommand\C{\mathcal{C}}
\newcommand\Cty{\mathrm{Ty}_{\C}}
\newcommand\Ctm{\mathrm{Tm}_{\C}}
\newcommand\name[1]{\ulcorner #1\urcorner}
\newcommand{\Util}{\widetilde{U}}
\newcommand\ev{\mathrm{ev}}

\title{A Fibrational Framework for Modal Dependent Type Theories}
\author{Daniel R. Licata, Mitchell Riley, Michael Shulman}
\date{}

\begin{document}
\maketitle
\tableofcontents

\section{Syntax}

\subsection{Overview of Judgements}

Mode theory judgements:
\begin{enumerate}
\item $\gamma \ctx$ (empty, extension)
\item $\gamma \yields p \type$ 
\item $\TypeTwo{\gamma}{s}{p}{q}$ (horizontal and vertical composition, identities)
\item $\gamma \yields \mu : p$ (variables, action of mode type morphisms)
\item $\TermTwoT{\gamma}{s}{\mu}{\nu}{p}$ (horizontal and vertical
    composition, identities)
\end{enumerate}

Top judgements: 
\begin{itemize}
\item $\yields_\gamma \Gamma \CTX$ over $\yields \gamma \ctx$
\item $\Gamma \yields_p A \TYPE$ over $\gamma \yields p \type$
\item $\Gamma \yields_\mu M : A$ over $\gamma \yields \mu : p$
\end{itemize}

Mode type morphisms $\TypeTwo{\gamma}{s}{p}{q}$ induce 1-cells
in \emph{both directions} in the mode theory,
\msnote{That's not true any more, is it?}
but
$\TypeTwo{\gamma}{s}{p}{q}$ and $\TermTwo{\gamma}{s}{\mu}{\nu}$
act \emph{contravariantly} on the subscripts of upstairs terms.

We expect structurality to be admissible for the base, and structurality
over that to be admissible for the top, e.g.:
\begin{mathpar}
\inferrule*[Left = weaken-over]
           {\Gamma,\Gamma' \yields_\mu M : A \\ (\text{where } \gamma,\gamma' \vdash \mu : p)}
           {\Gamma,y:B,\Gamma' \yields_\mu M : A \\ (\text{where } \gamma,y:q,\gamma' \vdash \mu : p)}

\inferrule*[Left = subst-over]
           {\Gamma,x:A,\Gamma' \yields_\nu N : C \\ (\text{where } \gamma,x:p,\gamma' \vdash \nu : \gamma) \\\\
            \Gamma \vdash_\mu M : A \\ (\text{where } \gamma \vdash \mu : p)
           }
           {\Gamma,\Gamma'[M/x] \yields_{\nu[\mu/x]} N[M/x] : A[M/x] \\ (\text{where } \gamma,\gamma'[\mu/x] \vdash \nu[\mu/x] : p[\mu/x])}
\end{mathpar}


\subsection{Mode Theory}

\begin{enumerate}

\item Contexts are as usual:

\begin{mathpar}
  \inferrule*{ }
             {\cdot \ctx}
             
  \inferrule*
    {\gamma \ctx \\
     \gamma \yields p \type}
    {\gamma,x:p \ctx}
\end{mathpar}

\item We assume $1/\Sigma$ modes:
\begin{mathpar}
  \inferrule*{ } { \gamma \yields 1 \type }
  
  \inferrule*{ \gamma \yields p \type \\ 
               \gamma,x:p \yields q \type }
             {\gamma \yields \sigmacl{x}{p}{q} \type}

\end{mathpar}
  

\item In all mode theories, terms must have: 

\begin{mathpar}
\inferrule*{ }
             {\gamma,x : p, \gamma' \yields x : p}
             
\inferrule*
    {\gamma \yields \mu : q \\
     \TypeTwo{\gamma}{s}{p}{q}
    }
    {\gamma \yields \TrPlus{s}{\mu} : p}
\\
\TrPlus{\id}{\mu} \equiv \mu \qquad
\TrPlus{s'}{\TrPlus{s}{\mu}} \equiv \TrPlus{(s';s)}{\mu} 
\end{mathpar}

We also have the usual rules for $1/\Sigma$ modes:
\begin{mathpar}
  \inferrule*{ }
             {\gamma \yields \mt : 1}
  \and 
  \mu \equiv \mt
\\
\inferrule*{
  \gamma \yields \mu : p \and
  \gamma \yields \nu : q[\mu/x]
    }
   {\gamma \yields (\mu,\nu) : \sigmacl{x}{p}{q}}
\and
\inferrule*
    {\gamma \yields \mu : \sigmacl{x}{p}{q}}
    {\gamma \yields \fst \mu : p}
\and
\inferrule*
    {\gamma \yields \mu : \sigmacl{x}{p}{q}}
    {\gamma \yields \snd \mu : q[\fst \mu / x]}
    \\
    \fst{(\mu,\nu)} \equiv \mu \and
    \snd{(\mu,\nu)} \equiv \nu \and
    p \equiv (\fst p, \snd p)
\end{mathpar}

Equations for ``transport'' in $\Sigma$:
\begin{mathpar}
\TrPlus{(\sigmacl{x}{s}{t})}{\mu} \equiv (\TrPlus{s}{\fst \mu},\TrPlus{(t[\fst \mu/x])}{\snd \mu})
\end{mathpar}

\item Mode type morphisms:
\begin{mathpar}
    \inferrule*{ }
          {\TypeTwo{\gamma}{\id_p}{p}{p}}
    \qquad
    \inferrule*{{\TypeTwo{\gamma}{s_1}{p_1}{p_2}} \\
                {\TypeTwo{\gamma}{s_2}{p_2}{p_3}}
          }
          {\TypeTwo{\gamma}{s_1;s_2}{p_1}{p_3}}

\inferrule*{{\gamma,x:p} \vdash {q} \type \\
            \TermTwoT{\gamma}{t}{\mu}{\mu'}{p}\\
           } 
           {\TypeTwo{\gamma}{\ap {q} {t/x}}{q[\mu/x]}{q[\mu'/x]}}

\\
\id;s \equiv s \equiv s;\id \and
(s;s');s'' \equiv s;(s';s'') \\ 
\ap q {\id_{\mu}/x} \equiv \id_{q[\mu/x]} \and
\ap q {(s;t)/x} \equiv \ap q {s/x}; \ap q {t/x} \\ 
\ap q {s/\_} \equiv \id_q \\ 
\ap {(q[\mu/x])} {s/y} \equiv \ap q {\ap \mu {s/y}/x} \quad (\text{where } \gamma,y:p' \vdash \mu : p \text{ and } \gamma,x:p \vdash q \type)\\
s[\nu/x];\ap{q'}{t/x} \equiv \ap{q}{t/x};s[\nu'/x] \quad 
(\text{where } \TypeTwo{\gamma,x:p}{s}{q}{q'} \text{ and } \TermTwoT{\gamma}{t}{\nu}{\nu'}{p})

%% subst: \id_\mu[\nu/x] = \id_{\mu[\nu/x]}
%% subst: s[x/x] = s
%% subst: (s;t)[\mu/x] = s[\mu/x];t[\mu/x]
%% subst: s[\mu[\nu/x]/x] = s[\mu/x][\nu/x]
%% subst: ap q (s [\mu/x]) = (ap q s)[\mu/x] and generalization
\end{mathpar}

We write $\ap q {t/x}$ for whiskering (\dsd{ap} in book HoTT).

We need congruence for $\Sigma$ to be a rule (because we don't have ap
on a type variable/universes):
\begin{mathpar}
  \inferrule*
  {\TypeTwo{\gamma}{s}{p}{p'} \\
    \TypeTwo{\gamma,x':p'}{t}{q[\TrPlus{s}{x'}/x]}{q'}}
  {\TypeTwo{\gamma}{\sigmacl{x'}{s}{t}}{\sigmacl{x}{p}{q}}{\sigmacl{x'}{p'}{q'}}} \\

  \sigmacl{x'}{\id_p}{\id_q} \equiv \id_{\sigmacl{x'}{p}{q}} \and
  (\sigmacl{x'}{s}{t});(\sigmacl{x''}{s'}{t'}) \equiv \sigmacl{x''}{(s;s')}{(t[\TrPlus{s'}{x''}/x'];t')} \\

  \ap{(\sigmacl{x'}{p}{q})}{s/(y:r)} \equiv
  \sigmacl{x'}{\ap{p}{s/y}}{\ap{({q[\fst z/x,\snd z/y]})}{\extend{s}{x'}/(z:(\sigmacl{y}{r}{p}))}}
\end{mathpar}

\item 2-cells between terms.  First, we have
  identity/composition/whiskering and associated equations (whiskering
  on the other side is given by substitution):
\begin{mathpar}
    \inferrule*{ }
          {\TermTwoT{\gamma}{\id_\mu}{\mu}{\mu}{p}}
    \qquad
    \inferrule*{{\TermTwoT{\gamma}{s_1}{\mu_1}{\mu_2}{p}} \\
                {\TermTwoT{\gamma}{s_2}{\mu_2}{\mu_3}{p}}
          }
   {\TermTwoT{\gamma}{s_1;s_2}{\mu_1}{\mu_3}{p}}

\inferrule*{{\gamma,x:p} \yields {\nu} : {q} \\
            \TermTwoT{\gamma}{s}{\mu}{\mu'}{p}\\
           } 
           {\TermTwoT{\gamma}{\ap \nu {s/x}}{\nu[\mu/x]}{\TrPlus{\ap{q}{s/x}}{\nu[\mu'/x]}}{q[\mu/x]}}

\\           
\id;s \equiv s \equiv s;\id \and
(s;s');s'' \equiv s;(s';s'') \\ 
\ap \nu {\id_{\mu}/x} \equiv \id_{\nu[\mu/x]} \and
\ap \nu {(s;t)/x} \equiv \ap \nu {s/x} ; (\ap {(\TrPlus{\ap{q}{s/x}}{y})} {\ap \nu {t/x}/y}) \\ 
\ap x {s/x} \equiv s  \and
\ap \nu {s/\_} \equiv \id_\nu \and
\ap {(\nu[\mu/x])} {s/y} \equiv \ap \nu {\ap \mu {s/y}/x} \quad
(\text{where } \gamma,y:p' \vdash \mu : p \text{ and } \gamma,x:p \vdash \nu : q \type)\\
t[\mu/x];\ap{\nu'}{s/x} \equiv \ap{\nu}{s/x};\ap{(\TrPlus{\ap{q}{s/x}}{y})}{t[\mu'/x]/y} \quad
 (\text{where } \TermTwoT{\gamma,x:p}{t}{\nu}{\nu'}{q} \text{ and } \TermTwoT{\gamma}{s}{\mu}{\mu'}{p})
\end{mathpar}

Finally, we have the 2-cells for $\Sigma$-modes:
\begin{mathpar}
\inferrule*
    {\TermTwoT{\gamma}{s}{\mu}{\mu'}{p} \and
      \gamma \vdash \nu' : q[\mu'/x]
    }
      {\TermTwoT{\gamma}{\extend{s}{\nu'}}{(\mu,\TrPlus{\ap{q}{s/x}}{\nu'})}{(\mu',\nu')}{\sigmacl{x}{p}{q}}}\\
\ap {\fst(z)} {\extend{s}{\nu'}/z} \equiv s \and
\ap {\snd(z)} {\extend{s}{\nu'}/z} \equiv \id_{\TrPlus{\ap{q}{s/x}}{\nu'}}  \\
s \equiv \ap{(\fst{\mu},y)}{\ap{(\snd z)}{s/z}/y};\extend{\ap{(\fst{z})}{s/z}}{\snd{\mu'}} \quad (\text{where } \TermTwoT{\gamma}{s}{\mu}{\mu'}{\sigmacl{x}{p}{q}})
\\      
{\extend{\id_\mu}{\nu'}} \equiv \id_{(\mu,\nu')} \and
{\extend{(s;s')}{\nu''}} \equiv  \extend{s}{\TrPlus{\ap{q}{s'/x}}{\nu''}};\extend{s'}{\nu''}   \\
\extend{s}{\nu'} ; (\ap{(\mu',y)}{t/y}) \equiv
(\ap{(\mu,\TrPlus{(\ap{q}{s})}{y})}{t/y}); \extend{s}{\nu''} \qquad (\text{where }\TermTwoT{\gamma}{t}{\nu'}{\nu''}{q[\mu'/x]})
\end{mathpar}

\item
  All judgements have a substitution principle
\begin{mathpar}
  \inferrule*{\gamma,x:p,\gamma' \yields J \\
              \gamma \yields \mu : p
              }
             {\gamma,\gamma'[\mu/x] \yields J[\mu/x]} \\

J[\mu/x][\nu/y] \equiv J[\nu/y][\mu[\nu/y]/x]
\end{mathpar}

\drlnote{write out the usual rules defining this}
             
\end{enumerate}

We sometimes write \ap{\mu}{s} for \ap{\mu(x)}{s/x}, eliding the
variable name when it is clear how to view $\mu$ as a term with a
distinguished variable; e.g. $\ApPlus{s}{t}$ for
$\ap{\TrPlus{s}{x}}{t/x}$.

\subsubsection{Lemmas}

Horizontal composition:
\begin{mathpar}
  \inferrule*[Left=Derivable]
      {\TermTwoT{\gamma}{s}{\mu}{\mu'}{p} \\
    \TermTwoT{\gamma, x : p}{t}{\nu}{\nu'}{q}}
             {\TermTwoT{\gamma}{\ap{t}{s/x} :\equiv t[\mu/x];\ap{\nu'}{s/x}}{\nu[\mu/x]}{\TrPlus{\ap{q}{s/x}}{\nu'[\mu'/x]}}{q[\mu/x]}}
\\ 
\ap{\id_\nu}{s/x} \equiv \ap{\nu}{s/x} \and \ap{t}{\id_{\mu}/x} \equiv t[\mu/x]
\end{mathpar}

Pairing and projection 2-cells are definable:
\begin{mathpar}
  \inferrule*[Left=Derivable]
      {\TermTwoT{\gamma}{s}{\mu}{\mu'}{p} \\
    \TermTwoT{\gamma}{t}{\nu}{\TrPlus{\ap{q}{s}}{\nu'}}{q[\mu/x]}}
             {\TermTwoT{\gamma}{(s,t) :\equiv \ap{(\mu,y)}{t/y};\extend{s}{\nu'}}{(\mu,\nu)}{(\mu',\nu')}{\sigmacl{x}{p}{q}}}

   \inferrule*[Left=Deriv]
              { {\TermTwoT{\gamma}{s}{\mu}{\mu'}{\sigmacl{x}{p}{q}}} }
              { {\TermTwoT{\gamma}{\ap{\fst(y)}{s/y}}{\fst{\mu}}{\fst{\mu'}}{p}} }
   \and
   \inferrule*[Left=Deriv]
              { {\TermTwoT{\gamma}{s}{\mu}{\mu'}{\sigmacl{x}{p}{q}}} }
              { {\TermTwoT{\gamma}{\ap{\snd(y)}{s/y}}{\snd{\mu}}{\TrPlus{\ap{(q(\fst y/x))}{s/y}}{\snd{\mu'}}}{q[\fst{\mu}/x]}} }
\end{mathpar}

\drlnote{Check usual composition rule for pairing}



\subsection{Contexts}

\begin{mathpar}
  \inferrule*[Left = ctx-form]{ }
  {\yields_{\cdot} \cdot \CTX  } \and 

  \inferrule*[Left = ctx-form]{
    \yields_\gamma \Gamma \CTX \and (\text{where } \yields \gamma \ctx) \\\\
    \Gamma \yields_p A \TYPE \and (\text{where }  \gamma \yields p \type)}
  {\yields_{\gamma, x : p} \Gamma, x : A \CTX \and (\text{where } \yields \gamma,x:p \ctx)  } \\
\end{mathpar}

\subsection{Types and Terms}

\subsubsection{Structural Rules}

\begin{mathpar}
  \inferrule*[Left = var]{
    % \yields \Gamma, x : A, \Gamma' \CTX_{\gamma, x : p, \gamma'}
  }
  {\Gamma, x : A, \Gamma' \yields_x x : A \and (\text{where } \gamma,x:p,\gamma' \yields x : p)} \and

 \inferrule*[Left = rewrite]{
   \Gamma \yields_\mu M : A 
   \and \TermTwoT{\gamma}{s}{\nu}{\mu}{p}
  }
  {\Gamma \yields_\nu \rewrite{s}{M} : A} \\ \\
  
  \rewrite{\id_{\nu[\mu/x]}}{M} \equiv M \and
  \rewrite{(s;t)}{M} \equiv \rewrite{s}{\rewrite{t}{M}} \and
  \rewrite{s}{M}[\rewrite{t}{N}/x] \equiv \rewrite{\ap{s}{t/x}}{\StI{\ap{q}{t/x}}{M[N/x]}}
\end{mathpar}
In the absence of eta for $\mathsf{F}$, we also need to assert
\begin{mathpar}
(\FE{M}{x}{\rewrite{t[\mu/y]}{N}}) \equiv \rewrite{t[\nu/y]}{\FE{M}{x}{N}} %\\
%(\StE{t}{M}{x}{\rewrite{s[\TrPlus{t}{x}/y]}{N}}) \equiv \rewrite{s[\nu/y]}{\StE{t}{M}{x}{N}} 
\end{mathpar}

\subsubsection{Telescope Types}

\begin{mathpar}
  \inferrule*{~}{\Gamma \yields_{1} 1 \TYPE} \and
  \inferrule*{~}{\Gamma \yields_{()} () : 1} \and
  \inferrule*{ \Gamma \yields_p A \TYPE \\
               \Gamma,x:A \yields_q B \TYPE}
             { \Gamma \yields_{\sigmacl{x}{p}{q}} \telety{x}{A}{B} \TYPE}
  \and 
  \inferrule*{ \Gamma \yields_\mu M : A \\
               \Gamma \yields_\nu N : B[M/x]
             }
             { \Gamma \yields_{(\mu,\nu)} (M,N) : \telety{x}{A}{B}}
  \and
  \inferrule*{ \Gamma \yields_{\mu} M : \telety{x}{A}{B}}
             { \Gamma \yields_{\fst \mu} \fst{M} : A} 
  \and
  \inferrule*{ \Gamma \yields_{\mu} M : \telety{x}{A}{B}}
             { \Gamma \yields_{\snd \mu} \snd{M} : B[\fst M/x]} 

    \fst{(M,N)} \equiv M \and
    \snd{(M,N)} \equiv N \and
    P \equiv (\fst P, \snd P)
\end{mathpar}


\subsubsection{Modalities}

\begin{mathpar}
  \inferrule*[Left = F-form]{
    %% \yields_\gamma \Gamma \CTX \and (\text{where } \yields \gamma \ctx)\\\\
    \Gamma \yields_p A \TYPE \and (\text{where } \gamma \yields p \type) \\\\
    \gamma, x:p \yields \mu : q 
  }
  {\Gamma \yields_q \F{x.\mu}{A} \TYPE \and (\text{where } \gamma \yields q \type) } \\
  
  \inferrule*[Left = F-intro]{
    \Gamma \yields_{\nu} M : A
    \and (\text{where } \gamma \yields {\nu} : p)
    %% \and \gamma \yields \nu : q 
    %% \and \gamma \yields \mu[\theta] : q 
    %% \and \gamma \yields (\nu \Rightarrow \mu[\theta]) : q
  }
  {\Gamma \yields_{\mu[\nu/x]} \FI{M} : \F{x.\mu}{A} \and (\text{where } \gamma \yields \mu[\nu/x] : q)} \\

  \inferrule*[Left = F-elim]{
    \Gamma, y : \F{x.\mu}{A} \yields_{r} C \TYPE \and (\text{where } \gamma, y : q \yields r \type) \\\\
    \Gamma \yields_{\nu} M : \F{x.\mu}{A} \and (\text{where } \gamma \yields \nu : q) \\\\
    \Gamma, x:A \yields_{\nu' [\mu / y]} N : C [\FI{x}/y]
    \and (\text{where } \gamma, x:p \yields \nu' [\mu / y] : r [\mu / y] )}
  {\Gamma \yields_{\nu'[\nu/y]} \FE{M}{x}{N} : C[M/y]  \and (\text{where }  \gamma \yields {\nu'[\nu/y]} : r[\nu/y])} \\
  \FE{\FI{M}}{x}{N} \equiv N[M/x] \and
  \text{(optionally:) }
  \FE{M}{x}{N[\FI{x}/z]} \equiv N[M/z]
  \\ \\

  \inferrule*[Left = F-Elim]{
    \gamma,y:q \yields r \type \\\\
    \Gamma \yields_{\nu} M : \F{x.\mu}{A} \and (\text{where } \gamma \yields \nu : q) \\\\
    \Gamma, x:A \yields_{r [\mu / y]} C \TYPE
    \and (\text{where } \gamma, x:p \yields r [\mu / y] \type )}
  {\Gamma \yields_{r[\nu/y]} \FE{M}{x}{C} \TYPE \and (\text{where }  \gamma \yields {r[\nu/y]} \type)} \\
  \FE{\FI{M}}{x}{C} \equiv C[M/x] \and
  \text{(optionally:) }
  \FE{M}{x}{C[\FI{x}/z]} \equiv C[M/z]
\\ \\
  \inferrule*[Left = U-form]{
    \Gamma \yields_p A \TYPE \and (\text{where } \gamma \yields p \type)\\\\
    \and \Gamma,x:A \yields_q B \TYPE \and (\text{where } \gamma,x:p \yields q \type)\\\\
    \and \gamma, x:p, c:r \yields \mu : q
  }{\Gamma \yields_r \U{c.\mu}{A}{B} \TYPE \and (\text{where } \gamma \yields r \type)} \\

  \inferrule*[Left = U-intro]{
    \Gamma,x:A \yields_{\mu[\nu/c]} M : B \and (\text{where } \gamma,x:p \yields {\mu[\nu/c]} : q)
  }
  {\Gamma \yields_{\nu} \UI {x}{M} : \U{c.\mu}{x:A}{B}
    \and (\text{where } \gamma \yields \nu : r)
  } \\
  
  \inferrule*[Left = U-elim]{
    \Gamma \yields_{\nu_1} N_1 : \U{c.\mu}{x:A}{B} \and (\text{where } \gamma \yields \nu_1 : r) \\\\
    \Gamma \yields_{\nu_2} N_2 : A \and (\text{where } \gamma \yields \nu_2 : p)
  }{
    \Gamma \yields_{\mu[\nu_2/x,\nu_1/c]} \UE{N_1}{N_2} : B \and (\text{where } \gamma \yields \mu[\nu_2/x,\nu_1/c] : q)
  } \\

  \UE{(\UI{x}{M})}{N} \equiv M[N/x] \and 
  \UI{x}{\UE{N}{x}} \equiv N
\end{mathpar}

\subsubsection{Surprisingly Strict Modalities}

\begin{mathpar}
  \inferrule*[Left = s-form]{
    \Gamma \yields_p A \TYPE \and (\text{where } \gamma \yields p \type)\\\\
    \and \TypeTwo{\gamma}{s}{q}{p}
  }{\Gamma \yields_q \St{s}{A} \TYPE \and (\text{where } \gamma \yields q \type)} \\

  \inferrule*[Left = S-intro]{
    \Gamma \yields_{\mu} M : A
    \and (\text{where } \gamma \yields {\mu} : p)
  }
  {\Gamma \yields_{\TrPlus{s}{\mu}} \StI{s}{M} : \St{s}{A} \and (\text{where } \gamma \yields \TrPlus{s}{\mu} : q)} \\

  \inferrule*[Left = S-elim]{
    \Gamma, y : \St{s}{A} \yields_{r} C \TYPE \and (\text{where } \gamma, y : q \yields r \type) \and \\\\
    \Gamma \yields_{\nu} M : \St{s}{A} \and (\text{where } \gamma \yields \nu : q) \\\\
    \Gamma, x : A \yields_{\nu' [\TrPlus{s}{x} / y]} N : C [\StI{s}{x}/y]
    \and (\text{where } \gamma, x : p \yields \nu' [\TrPlus{s}{x} / y] : r[\TrPlus{s}{x} / y] )}
  {\Gamma \yields_{\nu'[\nu/y]} \StE{s}{M}{x}{N} : C[M/y]  \and (\text{where } \gamma \yields {\nu'[\nu/y]} : r[\nu/y])} \\
  \StE{s}{\StI{s}{M}}{x}{N} \equiv N[M/x] \and
  \StE{s}{M}{x}{N[\StI{s}{x}/z]} \equiv N[M/z]
  \\
  
  \inferrule*[Left = S-Elim]{
    \gamma,y:q \yields r \type \\\\
    \Gamma \yields_{\nu} M : \St{s}{A} \and (\text{where } \gamma \yields \nu : q) \\\\
    \Gamma, x:A \yields_{r [\TrPlus{s}{x} / y]} C \TYPE
    \and (\text{where } \gamma, x:p \yields r [\TrPlus{s}{x} / y] \type )}
  {\Gamma \yields_{r[\nu/y]} \StE{s}{M}{x}{C} \TYPE \and (\text{where }  \gamma \yields {r[\nu/y]} \type)} \\
  \StE{s}{\StI{s}{M}}{x}{C} \equiv C[M/x] \and
  \StE{s}{M}{x}{C[\StI{s}{x}/z]} \equiv C[M/z]
  \\ \\
\end{mathpar}

%\drlnote{Change the definition of $s$-types to $U$-types as primitive and derive $F$, so that having $\eta$ is less surprising.}

Term/type equalities:
\begin{align*}
\StI{s}{\FI{M}} &\equiv \FI{M} &\St{s}{\F{x.\mu}{A}} &\equiv \F{x.\TrPlus{s}{\mu}}{A} \\
\FI{\StI{s}{M}} &\equiv \FI{M} &\F{x.\mu}{\St{s}{A}} &\equiv \F{x.\mu[\TrPlus{s}{x}/x]}{A} \\
%\UStI{s}{\UI{x}{M}} &\equiv \UI{x}{M} &\St{s}{\U{c.\mu}{x:A}{B}} &\equiv \U{c.\mu[\TrCirc{s}{c}/c]}{x:A}{B} \\
%\UI{x}{\UStI{s}{M}} &\equiv \UI{x}{M} &\U{c.\mu}{x:A}{\St{s}{B}} &\equiv \U{c.\TrCirc{s}{\mu}}{x:A}{B} \\
\UI{x}{M} &\equiv \UI{x}{M[\StI{s}{x}/x]} &\U{c.\mu}{x:\St{s}{A}}{B} &\equiv \U{c.\mu[\TrPlus{s}{x}/x]}{x:A}{B[\StI{s}{x}/x]} \\
\StI{(s, t)}{(M, N)} &\equiv (\StI{s}{M}, \StI{t[\mu/x]}{N}) &\St{(\telety{x'}{s}{t})}{\telety{x'}{A'}{B'}} & \equiv \telety{x}{\St{s}{A'}}{\StE{s}{x}{x'}{\St{t}{B'}}} \\
\StI{s}{\StI{t}{M}} &\equiv \StI{s;t}{M} &\St{s}{\St{t}{A}} &\equiv \St{(s;t)}{A} \\
\StI{\id_p}{M} &\equiv M &\St{\id_p}{A} &\equiv A\\
\rewrite{\ap{\nu}{t/x}}{\StI{\ap{q}{t/x}}{N[M/x]}} &\equiv N[\rewrite{t}{M}/x]  &\St{(\ap{q}{t/x})}{B[M/x]} & \equiv B[\rewrite{t}{M}/x] \\
%% other whiskering is a substitution rule:
%% \St{(s[\mu/x])}{B[M/x]} & \equiv (\St{s}{B})[M/x] 
\end{align*}
Where the last term equation is a special case of rewrite on substitutions. The inputs are typed $\Gamma,x:A \vdash_q B \TYPE $\ and $\Gamma \vdash_{\mu'} M : A$ and $\TermTwo{\gamma}{t}{\mu}{\mu'}$.
\mvrnote{TODO: Typecheck the U ones...}

In the absence of \textsf{F}-eta, we also need these extra equalities so we can push $s$-types around:
\begin{align*}
\StI{s[\nu/y]}{\FE{M}{x}{N}} &\equiv \FE{M}{x}{\StI{s[\mu/y]}{N}} \\
\FEs{\TrPlus{s}{\mu}}{M}{x}{N} &\equiv \StE{s}{M}{y}{(\FEs{\mu}{y}{x}{N})} \\
\FEs{\mu[\TrPlus{s}{x}/x]}{M}{x}{N} &\equiv \FEs{\mu}{M}{y}{(\StE{s}{y}{x}{N})} \\
\end{align*}

\mvrnote{Interaction of $s$- and $U$-types pending}

\subsection{Lemmas}

\begin{lemma}
The \textsc{rewrite} rule pushes into terms of telescope types:
\begin{align*}
%(\rewrite{s}{M},N) &\equiv \rewrite{(s, \varepsilon_\nu^{\ap{q}{s/x}})}{(M, \UnSt{\ap{q}{s/x}}{N}}) \\
%(M,\rewrite{t}{N}) &\equiv \rewrite{(\id_\mu, t)}{(M,N)} \\
\rewrite{(s, t)}{(M, N)} &\equiv (\rewrite{s}{M}, \rewrite{t}{\StI{\ap{q}{s/x}}{N}} ) 
\end{align*}
\end{lemma}
\begin{proof}
\begin{align*}
\rewrite{(s, t)}{(M, N)} 
&\equiv (\fst\rewrite{(s, t)}{(M, N)}, \snd \rewrite{(s, t)}{(M, N)} ) \\
&\equiv (\fst z [\rewrite{(s, t)}{(M, N)}/z], \snd z [\rewrite{(s, t)}{(M, N)}/z] ) \\
&\equiv (\rewrite{\ap{\fst}{(s, t)}}{\fst (M, N)}, \rewrite{\ap{\snd}{(s, t)}}{\StI{\ap{q[\fst z/x]}{(s, t)/z}}{\snd (M, N)}} ) \\
&\equiv (\rewrite{s}{M}, \rewrite{t}{\StI{\ap{q}{s/x}}{N}} )
\end{align*}

\drlnote{Add ref to equalities being used from second to third line.
  Also write out the typing for the second half of the pair, to show how
  the $s$-type moves work.}

%\begin{align*}
%(\rewrite{s}{M},N) 
%&\equiv  \\
%(M,\rewrite{t}{N}) 
%&\equiv \rewrite{(\id_\mu, \id_y)}{(M, y)}[\rewrite{t}{N}/y] \\
%&\equiv \rewrite{(\id_\mu, \id_y)[t/y]}{(M,N)} \\
%&\equiv \rewrite{(\id_\mu, t)}{(M,N)}
%\end{align*}
\end{proof}

\begin{lemma}
The \textsc{rewrite} rule pushes into terms of $\mathsf{F}$-, $\mathsf{U}$- and $s$-types:
\begin{align*}
\FI{\rewrite{s}{M}} &\equiv \rewrite{\ap{\mu}{s/x}}{\FI{M}} \\
(\FE{\rewrite{s}{M}}{x}{N}) &\equiv \rewrite{\ap{\nu'}{s/y}}{\StI{\ap{r}{s/y}}{\FE{M}{x}{N}}} \\
\UI{x}{\rewrite{\ap{\mu}{s/c}}{M}}  &\equiv\rewrite{s}{\UI{x}{M}} \\
\StI{t}{\rewrite{s}{M}} &\equiv \rewrite{\ApPlus{t}{s}}{\StI{t}{M}} \\
(\StE{t}{\rewrite{s}{M}}{x}{N}) &\equiv \rewrite{\ap{\nu'}{s/y}}{\StI{\ap{r}{s/y}}{\StE{t}{M}{x}{N}}}
\intertext{If we have definitional eta-expansion for \textsf{F}- and $s$-types, we can also derive (rather than asserting)}
(\FE{M}{x}{\rewrite{s[\mu/y]}{N}}) &\equiv \rewrite{s[\nu/y]}{\FE{M}{x}{N}} \\
(\StE{t}{M}{x}{\rewrite{s[\TrPlus{t}{x}/y]}{N}}) &\equiv \rewrite{s[\nu/y]}{\StE{t}{M}{x}{N}} 
\end{align*}
\end{lemma}
\begin{proof}
For \textsf{F}-types:
\begin{align*}
\FI{\rewrite{s}{M}} 
&\equiv \rewrite{\id_{\mu[\nu/x]}}{\FI{x}}[\rewrite{s}{M}/x]  \\
&\equiv \rewrite{\ap{\mu}{s/x}}{\StI{\ap{q}{s/x}}{\FI{x}[M/x]}} \\
&\equiv \rewrite{\ap{\mu}{s/x}}{\StI{\ap{q}{s/x}}{\FI{M}}} \\
&\equiv \rewrite{\ap{\mu}{s/x}}{\FI{M}} && \text{As $x$ does not appear in $q$.}\\
(\FE{\rewrite{s}{M}}{x}{N})
&\equiv \rewrite{\id_{\nu'[\nu/y]}}{\FE{y}{x}{N}}[\rewrite{s}{M}/y] \\
&\equiv \rewrite{\ap{\nu'}{s/y}}{\StI{\ap{r}{s/y}}{(\FE{y}{x}{N})[M/y]}} \\
&\equiv \rewrite{\ap{\nu'}{s/y}}{\StI{\ap{r}{s/y}}{\FE{M}{x}{N}}}
\intertext{and assuming eta-expansion:}
\rewrite{s[\nu/y]}{\FE{M}{x}{N}}
&\equiv \rewrite{s}{\FE{y}{x}{N}} [M/y] \\
&\equiv \FE{M}{z}{(\rewrite{s}{\FE{y}{x}{N}}[\FI{z}/y])} \\
&\equiv \FE{M}{z}{(\rewrite{s[\mu/y]}{\FE{\FI{z}}{x}{N}})} \\
&\equiv \FE{M}{z}{\rewrite{s[\mu/y]}{N}}
\end{align*}

For \textsf{U}-types:
\begin{align*}
\UE{M}{\rewrite{s}{N}}
&\equiv \rewrite{\id_{\mu[z/x, \nu_1/c]}}{\UE{M}{z}}[\rewrite{s}{N}/z] \\
&\equiv \rewrite{\ap{\mu[\nu_1/c]}{s/x}}{\StI{\ap{q[\nu_1/c]}{s/x}}{\UE{M}{N}}} \\
&\equiv \rewrite{\ap{\mu[\nu_1/c]}{s/x}}{\StI{\ap{q}{s/x}}{\UE{M}{N}}} \\
\UE{(\rewrite{t}{M})}{N}
&\equiv \rewrite{\id_{\mu[\nu_2/x, z/c]}}{\UE{(z)}{N}}[\rewrite{t}{M}/z] \\
&\equiv \rewrite{\ap{\mu[\nu_2/x]}{t/c}}{\StI{\ap{q[\nu_2/x]}{t/c}}{\UE{M}{N}}} \\
&\equiv \rewrite{\ap{\mu[\nu_2/x]}{t/c}}{\UE{M}{N}} && \text{(As $c$ does not appear in $q$)}\\
\rewrite{s}{\UI{x}{M}}
&\equiv \UI{y}{\UE{(\rewrite{s}{\UI{x}{M}})}{y}} \\
&\equiv \UI{y}{\rewrite{\ap{\mu[y/x]}{s/c}}{\UE{(\UI{x}{M})}{y}}} \\
&\equiv \UI{y}{\rewrite{\ap{\mu[y/x]}{s/c}}{M[y/x]}} \\
&\equiv \UI{x}{\rewrite{\ap{\mu}{s/c}}{M}}
\end{align*}

And for $s$-types the reasoning is identical to that for \textsf{F}-types.
\end{proof}

\begin{lemma}
The \textsc{rewrite} rule pushes into types obtained by \textsf{F}- and $s$-elimination:
\begin{align*}
(\FE{\rewrite{s}{M}}{x}{C}) &\equiv \St{\ap{r}{s/y}}{\FE{M}{x}{C}} \\
(\StE{t}{\rewrite{s}{M}}{x}{C}) &\equiv \St{\ap{r}{s/y}}{\StE{t}{M}{x}{C}}
\end{align*}
\end{lemma}
\begin{proof}
This is analogous to elimination into terms:
\begin{align*}
\FE{\rewrite{s}{M}}{x}{C}
&\equiv (\FE{y}{x}{C})[\rewrite{s}{M}/y] \\
&\equiv \St{\ap{r}{s/y}}{(\FE{y}{x}{C})[M/y]} \\
&\equiv \St{\ap{r}{s/y}}{\FE{M}{x}{C}}
\end{align*}
and similarly for $s$-types.
\end{proof}

\begin{lemma}
\textsc{F-elim} commutes with \textsc{s-elim}.
\end{lemma}
\begin{proof}
Using eta for $s$-types:
\begin{align*}
&\FE{(\StE{s}{M}{x'}{N'})}{x}{N} \\
&\equiv (\FE{(\StE{s}{z}{x'}{N'})}{x}{N})[M/z] \\
&\equiv \StE{s}{M}{z'}{((\FE{(\StE{s}{z}{x'}{N'})}{x}{N})[\StI{s}{z'}/z])} \\
&\equiv \StE{s}{M}{z'}{((\FE{(\StE{s}{\StI{s}{z'}}{x'}{N'})}{x}{N}))} \\
&\equiv \StE{s}{M}{z'}{(\FE{N'[z'/x']}{x}{N})} \\
&\equiv \StE{s}{M}{x'}{(\FE{N'}{x}{N})}
\end{align*}
\end{proof}

\begin{lemma}
\textsc{F-elim} fuses with \textsc{s-intro}.
\end{lemma}
\begin{proof}
\begin{align*}
&\FEs{\TrPlus{s}{\mu}}{\StI{s}{M}}{x}{N} \\
&\equiv \StE{s}{\StI{s}{M}}{y}{(\FEs{\mu}{y}{x}{N})} \\
&\equiv \FEs{\mu}{M}{x}{N}
\end{align*}
and
\begin{align*}
& \FEs{\mu[\TrPlus{s}{x}/x]}{M}{x}{N[\StI{s}{x}/x]} \\
&\equiv \FEs{\mu}{M}{y}{(\StE{s}{y}{x}{N[\StI{s}{x}/x]})}  \\
&\equiv \FEs{\mu}{M}{y}{N[y/x]} \\
&\equiv \FEs{\mu}{M}{x}{N}
\end{align*}
\end{proof}

\begin{lemma}
Any context $\Gamma$ can be tupled into a $\Sigma$ type $\Sigma \Gamma$, so that substitutions $\Gamma \yields \Theta : \Delta$ correspond bijectively to terms $\Gamma \yields \Sigma \Theta : \Sigma \Delta$. \mvrnote{Generalising to telescopes seems doable} 
\mvrnote{I have been pretty inconsistent with the notation for telescope types through this whole document}
\begin{mathpar}
\inferrule*{\yields_\gamma \Gamma}
             {\cdot \yields_{\Sigma \gamma} \Sigma \Gamma \TYPE}
\and
\inferrule*{~}
             {\sigma : \Sigma \Gamma \yields_{\fan{\gamma}} \fan{\Gamma} : \Gamma}
\and
\inferrule*{\Gamma \yields_\theta \Theta : \Delta}
             {\Gamma \yields_{\Sigma \theta} \Sigma \Theta : \Sigma \Delta}
\and
\inferrule*{\Gamma \yields\Theta : \Delta}
             {\Gamma \yields \Theta \equiv \fan{\Delta}[\Sigma \Theta / \sigma]}
\end{mathpar}
\drlnote{Other round-trip}
\end{lemma}
\begin{proof}
$\Sigma \Gamma$ is defined inductively by
\begin{align*}
\Sigma (\cdot) &:\equiv 1 \\
\Sigma (\Gamma, x : A) &:\equiv \sum_{\sigma : \Sigma \Gamma} A[\fan{\Gamma}] \\
\fan{(\cdot)} &:\equiv \cdot \\
\fan{\Gamma, x : A} &:\equiv (\fan\Gamma[\fst{\sigma}/\sigma]), \snd{\sigma}
\end{align*}
with each definition lying over the identical one downstairs. For $\Sigma \Theta$ we simultaneously verify the equation $\Theta \equiv \fan{\Delta}[\Sigma \Theta / \sigma]$, as we need it to hold for $(\Sigma \Theta, M)$ to be well typed.
\begin{align*}
\Sigma(\cdot) &:\equiv \mt : 1\\
\Sigma(\Theta, M : A) &:\equiv (\Sigma \Theta, M) : \sum_{\sigma : \Sigma \Delta} A[\fan{\Delta}] \\
(\cdot)[()/\sigma]  &\equiv (\cdot) \\
\fan{\Delta, x : A}[\Sigma(\Theta, M : A)/\sigma] &\equiv ((\fan{\Delta}[\fst{\sigma}/\sigma]), \snd{\sigma})[(\Sigma \Theta, M)/\sigma] \\
&\equiv (\fan{\Delta}[\Sigma \Theta/\sigma]), M \\
&\equiv \Theta, M
\end{align*}
\end{proof}

\begin{lemma}
Tupling respects substitution: for $\gamma \yields \theta : \delta$ and $\delta \yields \kappa : \lambda$, we have:
\begin{mathpar}
\Sigma(\kappa[\theta]) \equiv (\Sigma \kappa)[\theta]
\end{mathpar}
\end{lemma}
\begin{proof}
By induction on the length of $\lambda$:
\begin{align*}
\Sigma((\cdot)[\theta]) 
&\equiv \Sigma(\cdot) \\
&\equiv (\Sigma (\cdot))[\theta] \\
\Sigma((\kappa, M)[\theta])
&\equiv \Sigma(\kappa[\theta], M[\theta]) \\
&\equiv \Sigma(\kappa[\theta]), M[\theta] \\
&\equiv (\Sigma\kappa)[\theta], M[\theta] \\
&\equiv (\Sigma\kappa, M)[\theta]
\end{align*}
\end{proof}

\begin{definition}
A 2-cell between mode substitutions $\gamma\yields t : (\theta \vDash \theta') : \delta$ is given by a mode term 2-cell \[\gamma \yields \Sigma t : (\Sigma \theta \vDash_{\Sigma \delta} \Sigma \theta')\].
\end{definition}

\drlnote{Maybe write $\gamma\yields t : \theta \vDash_\delta \theta'$ ?
In the rules, we're writing things like $\gamma \mid \mu \vDash s :
\mu'$; we should be consistent about that versus this notation for where
the $s$ goes, and change one or the other.  I find things like $\Gamma
\vdash s : \mu \vDash_p \nu$ with two turnstiles hard to parse.
}

\begin{lemma}
N-ary ap and rewrite are admissible:
\begin{mathpar}
\inferrule*{\delta \vdash {q} \type \\
            \gamma \yields t : (\theta \vDash \theta') : \delta
           } 
           {\TypeTwo{\gamma}{\ap {q} {t}}{q[\theta]}{q[\theta']}}
\and
\inferrule*{\delta \yields {\nu} : {q} \\
            \gamma\yields t : (\theta \vDash \theta') : \delta
           } 
           {\TermTwoT{\gamma}{\ap \nu {t}}{\nu[\theta]}{\TrPlus{\ap{q}{t}}{\nu[\theta]}}{q[\theta]}} \\
 \inferrule*[Left = rewrite]{
   \Gamma \yields_{\theta'} \Theta : \Delta \and 
   \gamma \yields t : (\theta \vDash \theta') : \delta
  }
  {\Gamma \yields_{\theta} \rewrite{t}{\Theta} : \Delta}
\end{mathpar}
\end{lemma}
\begin{proof}
These are
\begin{align*}
\ap {q} {t} &:\equiv \ap{q[\fan{\delta}]}{\Sigma t/\sigma} \\
\ap {\nu} {t} &:\equiv \ap{\nu[\fan{\delta}]}{\Sigma t/\sigma} \\
\rewrite{t}{\Theta} &:\equiv \fan{\Delta}[\rewrite{\Sigma t}{\Sigma \Theta}/\sigma]
\end{align*}
which are well-typed by the equation $\theta \equiv \fan{\delta}[\Sigma \theta / \sigma]$.
\end{proof}

In particular, we have the following rules for building and using 2-cells between substitutions.
\begin{mathpar}
\inferrule{\gamma \yields t : \theta \vDash \theta' : \delta \and \gamma \yields s : (\mu[\theta] \vDash \TrPlus{\ap{p}{t}}{\mu'[\theta']})}
{ \gamma \yields (s, t) : (\theta, \mu) \vDash (\theta', \mu') : (\delta,x:p)} \\
%
\inferrule{\gamma \yields t : ((\theta, \mu) \vDash (\theta', \mu')) : (\delta, x : p)}
{\gamma \yields \ap{\fst}{t} : \theta \vDash \theta' } \and
%
\inferrule{\gamma \yields t : ((\theta, \mu) \vDash (\theta', \mu')) : (\delta, x : p)}
{\gamma \yields \ap{\snd}{t} : (\mu[\theta] \vDash \TrPlus{\ap{p}{\ap{\fst}{t}}}{\mu'[\theta']})}
\end{mathpar}%

\begin{lemma}
N-ary ap of a 2-cell on a substitution is admissible
\begin{mathpar}
\inferrule{\gamma \yields t : \theta \vDash \theta' : \delta \and \delta \yields \kappa : \lambda}
{\gamma \yields \ap{\kappa}{t} : \kappa[\theta] \vDash \kappa[\theta'] : \lambda}
\end{mathpar} 
\end{lemma}
\begin{proof}
Due to the equation $\Sigma(\kappa[\theta]) \equiv (\Sigma \kappa)[\theta]$ we can just define $\Sigma(\ap{\kappa}{t}) :\equiv \ap{(\Sigma \kappa)}{t}$.
\end{proof}

\begin{lemma}
N-ary associativity/interchange/etc. holds
\begin{align*}
\ap {(\nu[\theta])} {t} &\equiv \ap \nu {\ap \theta {t}} \\
s[\theta];\ap{\nu'}{t} &\equiv \ap{\nu}{t};\ap{(\TrPlus{\ap{q}{t}}{y})}{s[\theta']/y}
\end{align*}
\end{lemma}
\begin{proof}
\mvrnote{todo}
\end{proof}

\begin{lemma}
A N-ary versions of the equations concerning rewrites hold:
\begin{align*}
\St{(\ap{q}{t})}{B[\Theta]} &\equiv B[\rewrite{t}{\Theta}] \\
\rewrite{\ap{\nu}{t}}{\StI{\ap{q}{t}}{N[\Theta]}} &\equiv N[\rewrite{t}{\Theta}] \\
\rewrite{\ap{\kappa}{t}}{\Theta;\kappa} &\equiv \rewrite{t}{\Theta};\kappa
\end{align*}
\end{lemma}
\begin{proof}
\begin{align*}
B[\rewrite{t}{\Theta}] 
&\equiv B[\fan{\Delta}[\rewrite{\Sigma t}{\Sigma \Theta}/\sigma]] \\
&\equiv B[\fan{\Delta}][\rewrite{\Sigma t}{\Sigma \Theta}/\sigma] \\
&\equiv \St{\ap{q[\fan{\delta}]}{\Sigma t / \sigma}}{B[\fan{\Delta}][\Sigma \Theta/\sigma]} \\
&\equiv \St{\ap{q}{t}}{B[\Theta]}
\end{align*}
And:
\begin{align*}
N[\rewrite{t}{\Theta}]
&\equiv N[\fan{\Delta}[\rewrite{\Sigma t}{\Sigma \Theta}/\sigma]] \\
&\equiv N[\fan{\Delta}][\rewrite{\Sigma t}{\Sigma \Theta}/\sigma] \\
&\equiv \rewrite{\ap{\nu[\fan{\delta}]}{\Sigma t/\sigma}}{\StI{\ap{q[\fan{\delta}]}{\Sigma t/\sigma}}{N[\fan{\Delta}][\Sigma \Theta/\sigma]}} \\
&\equiv \rewrite{\ap{\nu}{t}}{\StI{\ap{q}{t}}{N[\Theta]}} \\
\end{align*}
And:
\begin{align*}
\rewrite{\ap{\kappa}{t}}{\Theta;\kappa}
&\equiv \fan{\Lambda}[\rewrite{\Sigma(\ap{\kappa}{t})}{\Sigma (\Theta;\kappa)}/\sigma] \\
&\equiv \fan{\Lambda}[\rewrite{\ap{(\Sigma \kappa)}{t}}{(\Sigma \kappa)[\Theta]}/\sigma] \\
&\equiv \fan{\Lambda}[(\Sigma \kappa)[\rewrite{t}{\Theta}]/\sigma] \\
&\equiv \fan{\Lambda}[(\Sigma \kappa)/\sigma][\rewrite{t}{\Theta}] \\
&\equiv \kappa[\rewrite{t}{\Theta}] \\
&\equiv \rewrite{t}{\Theta};\kappa
\end{align*}
using the previous equation.
\end{proof}

\begin{lemma}
Rewriting by a tuple is a tuple of rewritings:
\begin{align*}
\rewrite{(t, s/x)}{\Theta, M/x} \equiv (\rewrite{t}{\Theta}, \rewrite{s}{\StI{\ap{p}{t}}{M}})
\end{align*}
\end{lemma}
\begin{proof}
Unwinding definitions:
\begin{align*}
\rewrite{(t, s/x)}{\Theta, M/x}
&\equiv \fan{\Delta, x : A}[\rewrite{\Sigma (t, s/x)}{\Sigma (\Theta, M/x)}/\sigma] \\
&\equiv (\fan\Delta[\fst{\sigma}/\sigma], \snd{\sigma})[\rewrite{(\Sigma t, s)}{\Sigma\Theta, M}/\sigma] \\
&\equiv (\fan\Delta[\fst{\sigma}/\sigma], \snd{\sigma})[(\rewrite{\Sigma t}{\Sigma\Theta}, \rewrite{s}{\StI{\ap{p[\fan{\delta}]}{\Sigma t/\sigma}}{N}})/\sigma] \\
&\equiv (\fan\Delta[\rewrite{\Sigma t}{\Sigma\Theta}/\sigma], \rewrite{s}{\StI{\ap{p[\fan{\delta}]}{\Sigma t/\sigma}}{N}}) \\
&\equiv (\rewrite{t}{\Theta}, \rewrite{s}{\StI{\ap{p}{t}}{M}})
\end{align*}
\end{proof}

As a special case, note that we have
\begin{align*}
\rewrite{(\id_\Theta, s/x)}{\Theta, M/x} \equiv (\Theta, \rewrite{s}{M}/x)
\end{align*}

\subsection{Examples}

\begin{itemize}
\item 
The previous two-argument \F{\mu}{x:A,y:B} (for $x :p, y:q \vdash \mu :
r$) is now \F{z.\mu[\fst z/x,\snd z/y]}{\telety{x}{A}{B}}, using the
upstairs $\Sigma$-type ${\telety{x}{A}{B}}$, which has mode
$\sigmacl{x}{p}{q}$.  Iterating $\telety{x}{A}{B}$ plays the role of a
longer telescope.  
\end{itemize}

\section{Mode Theories}

\subsection{Comprehension Object}

\subsection{Comprehension Object with $\Sigma$}

This doesn't necessarily need to go into the final paper, but here are
some notes that compare different ways of doing the mode theory for
$\Sigma$ types.

We assume a comprehension object $\chi$ given by a
comprehension-terminal adjunction (definition yet to be spelled out
without the automatic adjunction notation, but all the same stuff should
be defined).

\newcommand\mtt[1]{\mathtt{#1}}

\begin{enumerate}

\item \label{sigma:total-to-fiber0} Conjecture: In any comprehension
  object $\chi$, 2-cells $s : \alpha \vDash_p \alpha.x$ such that $s;\pi
  = \id_\alpha$ correspond bijectively to 2-cells $1_\alpha
  \vDash_{\El{p}{\alpha}} x$.

  First, given $s : \alpha \vDash_p \alpha.x$, take $\mtt{var} : 1_{\alpha.x}
  \vDash_{\El{p}{\alpha.x}} \TrPlus{(\pi^\alpha_x)}{x}$ and ap $s$ onto
  it to get $\TrPlus{s}{1} \vDash_{\El{p}{\alpha}}
  \TrPlus{s}{\TrPlus{(\pi^\alpha_x)}{x}}$.  Then fuse, reduce
  $\TrPlus{s}{1}$, and use the section property to show that this also
  has type
  $1_\alpha \vDash_{\El{p}{\alpha}} x$.

  Conversely, given $t : 1_\alpha \vDash_{\El{p}{\alpha}} x$,
  the pairing of $\chi$ gives $(\id, t) : \alpha \vDash \alpha.x$.

\item \label{sigma:total-to-fiber1} Conjecture: similarly, 2-cells $s :
  \alpha.x \vDash_p \alpha.y$ such that $s;\pi_y = \pi_x$ correspond
  bijectively to 2-cells $1_{\alpha.x} \vDash_{\El{p}{\alpha.x}}
  \TrPlus{\pi^\alpha_x}{y}$.

  First, given $s : \alpha.x \vDash_p \alpha.y$, take $\mtt{var} :
  1_{\alpha.y} \vDash_{\El{p}{\alpha.y}} \TrPlus{(\pi^\alpha_y)}{y}$ and
  ap $s$ onto it to get $\TrPlus{s}{1} \vDash_{\El{p}{\alpha.x}}
  \TrPlus{s}{\TrPlus{(\pi^\alpha_y)}{x}}$.  Then fuse, reduce
  $\TrPlus{s}{1}$, and use the commuting triangle to show that this also
  has type $1_{\alpha.x} \vDash_{\El{p}{\alpha.x}} \TrPlus{\pi_x} y$.

  Conversely, given $t : 1_{\alpha.x} \vDash_{\El{p}{\alpha.}}
  \TrPlus{\pi_x} y$, we have $\id.t := \ap{.}{(\id, t)} : \alpha.x \vDash \alpha.y$.

\item \label{sigma:total-to-fiber2} Conjecture: similarly, 2-cells $s :
  \alpha.x.y \vDash_p \alpha.z$ such that $s;\pi_z = \pi_y;\pi_x$
  correspond bijectively to 2-cells $1_{\alpha.x.y}
  \vDash_{\El{p}{\alpha.x}} \TrPlus{\pi^{\alpha.x}_y;\pi^\alpha_x}{y}$.

\item \label{sigma:full} Define \emph{fullness} of the comprehension
  object to mean that maps $s : \alpha.x \vDash_p \alpha.y$ such that
  $s;\pi_y = \pi_x$ correspond bijectively to maps in the fiber $x
  \vDash_{\El{p}{\alpha}} y$.  (Probably plus some naturality.)

  (Is this the same as $U$ from the comprehesion/slide adjunction before
  being full and faithful?)

\item \label{sigma:full-interesting} Define \emph{terminal fullness}
  (probably a better name) of the comprehension object to mean that maps
  maps $1_{\alpha.x} \vDash_{\El{p}{\alpha.x}} \TrPlus{\pi^\alpha_x}{y}$
  out of the terminal object are bjective with $x
  \vDash_{\El{p}{\alpha}} y$.

  One direction is automatic: given $s : x \vDash_{\El{p}{\alpha}} y$,
  we can make $(\id.s) : \alpha.x \vDash \alpha.y$ as above, and ap that
  onto $\mtt{var} : 1_{\alpha.y} \vDash \TrPlus{\pi_y}{y}$ to get
  $1_{\alpha.x} \vDash_{\El{p}{\alpha.x}} \TrPlus{\pi^\alpha_x}{y}$.

\item By \ref{sigma:total-to-fiber1}, fullness follows iff terminal
  fullness.  
  
\item \label{sigma:complete}

  A ``complete spec'' (cf. Jacobs REF) for $\Sigma$ types is
  \begin{itemize}
  \item $\alpha : p, x : \El{p}{\alpha}, y : \El{p}{\alpha.x} \vdash \Sigma_1(\alpha,x,y) : \El{p}{\alpha}$,
  \item $\Sigma_1(\alpha,x,-) \dashv \TrPlus{\pi^\alpha_x}{-}$
  \item $E : \Sigma_1(\alpha,x,1_{\alpha.x}) \vDash x$ given by
    transposing $\mtt{var}$ is an isomorphism
  \item The induced map $\mtt{pair} : \alpha.x.y \vDash
    \alpha.\Sigma_1(\alpha,x,y)$ given by $(\pi, \eta^\pi)$ is an
    isomorphism.
  \item A BC condition $\TrPlus{s}{(\Sigma_1(\alpha,x,y))} \equiv
    \Sigma_1(\beta,\TrPlus{s}{x}, \TrPlus{(s , \id)}{y})$
  \end{itemize}
  I'd be very surprised if this were not sufficient -- it's everything I
  think we could ask for.

\item Conjecture: given the adjunction $\Sigma_1(\alpha,x,-) \dashv
  \TrPlus{\pi^\alpha_x}{-}$ (actually, only the counit) and $E$ an iso,
  terminal fullness (and therefore fullness) holds.  The converse map
  for $1_{\alpha.x} \vDash_{\El{p}{\alpha.x}} \TrPlus{\pi^\alpha_x}{y}$
  is given by transposing to get $\Sigma_1(\alpha,x,1_{\alpha.x}) \vDash
  y$ and composing with $E^-1 : x \vDash
  \Sigma_1(\alpha,x,1_{\alpha.x})$.

  Conversely, given \ref{sigma:full-interesting}, do we get the counit
  $\Sigma_1(\alpha,x,\TrPlus{\pi^\alpha_x}{y}) \vdash_{\El{p}{\alpha}} y$?

\item \label{sigma:total-spec} Another possible spec for $\Sigma$ types
  would be
  \begin{itemize}
  \item $\alpha : p, x : \El{p}{\alpha}, y : \El{p}{\alpha.x} \vdash \Sigma_1(\alpha,x,y) : \El{p}{\alpha}$,
  \item A map $\mtt{pair} : \alpha.x.y \vDash
    \alpha.\Sigma_1(\alpha,x,y)$ that commutes over $\alpha$, probably
    with some naturality
  \item $\mtt{pair}$ is an isomorphism (split)
  \item A BC condition $\TrPlus{s}{(\Sigma_1(\alpha,x,y))} \equiv
    \Sigma_1(\beta,\TrPlus{s}{x}, \TrPlus{(s , \id)}{y})$?
  \end{itemize}

  This has all the axioms in $p$, rather than asserting the adjunction
  in the fiber.

\item Conjecture: given \ref{sigma:total-spec} (pairing and split in
  $p$) and fullness in the sense of \ref{sigma:full-interesting}, we
  can derive the adjunction $\Sigma_1(\alpha,x,-) \dashv
  \TrPlus{\pi^\alpha_x}{-}$.

  E.g. for the unit $y \vdash_{\alpha.x} \TrPlus{\pi^\alpha_x}{\Sigma_1(\alpha,x,y)}$
  we can get
  $1_{\alpha.x.y} \vdash_{\alpha.x.y} \TrPlus{\pi_y;\pi_x}{\Sigma_1(\alpha,x,y)}$
  from $\mtt{pair}$ by \ref{sigma:total-to-fiber2}.
  Then use \ref{sigma:full-interesting}.  

  For the counit
  $\Sigma_1(\alpha,x,\TrPlus{\pi^\alpha_x}{z}) \vdash_{\alpha} z$
  we have
  \[
  \alpha.\Sigma_1(\alpha,x,\TrPlus{\pi^\alpha_x}{z}) \vDash \alpha.x.\TrPlus{\pi^\alpha_x}{z} \vDash \alpha.z 
  \]
  and then can use the full version of fullness (\ref{sigma:full}).    

  Thus, if the comprehension category is already full, then pair and
  split are enough, which explains why we don't usually talk about the
  adjunction itself inside the type theory.  
  
\item The exact 2-cell that comes up in encoding pairing for
  object-language $\Sigma$-types below is a contraction:
  \[
  1_\alpha \vDash_{\El{p}{\alpha}} \Sigma_1(\alpha,1_\alpha,1_{\alpha.{1_\alpha}})
  \]
  This seemed odd, because it seems like it's less than full pairing.    

\item However, contraction does imply $\mtt{pair}$, because we already
  know that $\Sigma_1$ is a functor, so we have
  \[
  \ap{\Sigma_1}{(\pi^{\alpha.x}_y;\pi^\alpha_x, \mtt{var}, \mtt{var}?)} :
  \Sigma_1(\alpha,1_\alpha,1_{\alpha.1}) \vDash \TrPlus{(\pi^{\alpha.x}_y;\pi^\alpha_x)}{\Sigma_1(\alpha,x,y)}
  \]
  Then precompose with contraction, and then use
  \ref{sigma:total-to-fiber2}.

  (This is a more dependent version of deriving $\Gamma \vdash A$ and
  $\Gamma \vdash B$ implies $\Gamma \vdash A \times B$ from $\Gamma
  \times \Gamma \vdash A \times B$ by functoriality and then
  precomposing with the diagonal.)
  
\item Conversely, a constant $\mtt{pair} : \alpha.x.y \vDash
  \alpha.\Sigma_1(\alpha,x,y)$ (commuting over $\alpha$) implies
  contraction. An instance is $\alpha.1_{\alpha}.1_{\alpha.1} \vDash
  \alpha.\Sigma_1(\alpha,1_\alpha,1_{\alpha.1})$, and the comprehension
  object already has $\beta.1 \cong \beta$, so precompose, to get
  $\alpha \vDash \alpha.\Sigma_1(\alpha,1_\alpha,1_{\alpha.1})$, and
  then use \ref{sigma:total-to-fiber0}.

\item It remains to see exactly what we need to encode split (BC,
  $\mtt{pair}$ is an iso?), but it's plausible that we're avoiding
  needing the $\Sigma_1(\alpha,x,-) \dashv \TrPlus{\pi^\alpha_x}{-}$
  adjunction itself or fullness (which implies it) by only ever talking
  about maps from $1$ in the fiber.  
\end{enumerate}
  
\subsection{MLTT via Explicit Substitutions}
\newcommand{\qyields}{\Vdash} \newcommand{\varsof}[1]{{#1}^\dagger}
\newcommand{\upstairs}[1]{\overline{#1}}
\newcommand{\downstairs}[1]{\underline{#1}}
\newcommand{\asdep}[1]{{#1}_p}
\newcommand\proj[1]{\ensuremath{\mathsf{proj}_{#1}}}
\newcommand\var[1]{\ensuremath{\mathsf{var}_{#1}}}

In this section we interpret all the rules and equations of the presentation of type theory given in \cite{altenkirchkaposi16qit} via QITs. 

For consistency we switch from the Agda-style syntax used in the paper to ordinary inference rules. To distinguish judgements in this type theory from the ones in the framework I will use $\qyields$ as the turnstile. \mvrnote{Or some other symbol...} We have the following judgements:
\begin{mathpar}
\qyields \Gamma \CTX \and \Gamma \qyields A \TYPE \and \Gamma \qyields a : A \and \Gamma \qyields \Theta : \Delta 
\end{mathpar}
with the inference rules given in Figure~\ref{fig:qit-rules}. Note that some equations require earlier equations to hold in order to typecheck.

\begin{figure}
\begin{mathpar}
\inferrule*[left=ctx-empty]{~}{\cdot \CTX} \and
\inferrule*[left=ctx-ext]{\qyields \Gamma \CTX \and \Gamma \qyields A \TYPE}{\qyields \Gamma, A \CTX} \\
\inferrule*[left=type-sub]{\Delta \qyields A \TYPE \and \Gamma \qyields \Theta : \Delta}{\Gamma \qyields A[\Theta] \TYPE} \and
\inferrule*[left=term-sub]{\Delta \qyields a : A  \and \Gamma \qyields \Theta : \Delta}{\Gamma \qyields a[\Theta] : A[\Theta]} 
\\
\inferrule*[left=sub-empty]{~}{\Gamma \qyields \epsilon : \cdot} \and
\inferrule*[left=sub-ext]{\Gamma \qyields \Theta : \Delta \and \Gamma \qyields a : A[\Theta]}{\Gamma \qyields (\Theta, a) : \Delta, A} \\
\inferrule*[left=sub-id]{~}{\Gamma \qyields \id_\Gamma : \Gamma} \and
\inferrule*[left=sub-comp]{\Gamma \qyields \Theta : \Delta \and \Delta \qyields \kappa : \Lambda}{\Gamma \qyields \Theta ; \kappa : \Lambda} \\
\inferrule*[left=sub-proj]{~}{\Gamma, A \qyields \proj{\Gamma,A} : \Gamma} \and 
\inferrule*[left=var]{~}{\Gamma, A \qyields \var{\Gamma,A} : A[\proj{\Gamma,A}]} 
\end{mathpar}

\begin{align}
A[\id] &\equiv A \\
A[\Theta ; \kappa] &\equiv A[\kappa][\Theta] \\
\nonumber\\
a[\id] &\equiv a \\
a[\Theta ; \kappa] &\equiv a[\kappa][\Theta] \\
\nonumber\\
\id ; \Theta &\equiv \Theta \\
\Theta ; \id &\equiv \Theta \\
(\Theta; \kappa) ; \rho &\equiv \Theta ; (\kappa ; \rho) \\
\nonumber\\
\Theta ; (\kappa , a) &\equiv (\Theta ; \kappa) , a[\Theta] \\ 
(\Theta, a);\proj{\Gamma,A} &\equiv \Theta \\
\var{\Delta,A}[\Theta, a] &\equiv a \\
(\proj{\Gamma,A}, \var{\Gamma,A}) &\equiv \id_{\Gamma, A} \\
\Theta &\equiv \epsilon && \text{for } \Gamma \qyields \Theta : \cdot
\end{align}
\caption{Rules of MLTT via Explicit Substitutions}\label{fig:qit-rules}
\end{figure}

We can derive some useful operations that we will use later when defining $\Pi$ and $\Sigma$ types.
\begin{mathpar}
\inferrule*[left=derivable]{\Gamma \qyields \Theta : \Delta \and \Delta \qyields A \TYPE}{\Gamma, A[\Theta] \qyields \Theta \uparrow A : \Delta, A} \and
\inferrule*[left=derivable]{\Gamma \qyields a : A}{\Gamma \qyields \hat{a} : \Gamma, A}
\end{mathpar}
defined by:
\begin{align*}
\Theta \uparrow A &:\equiv (\proj{\Gamma, A[\Theta]}; \Theta) , \var{\Gamma, A[\Theta]} \\
\hat{a} &:\equiv \id_{\Gamma} , a
\end{align*}

Our goal is to interpret these rules in the framework. The mode theory will contain a comprehension object $p$, and the basic idea for representing the judgements of MLTT is as follows:

\begin{itemize}
\item For each object-language context $\Gamma$, there is a corresponding upstairs framework type $\upstairs{\Gamma}$ with mode $p$.

\item A type $\Gamma \qyields A \TYPE$ is represented by $g : \upstairs{\Gamma} \yields_{\El{p}{g}} \upstairs{A} \TYPE$.
  
\item A term $\Gamma \qyields a : A$ is represented by $g : \upstairs{\Gamma} \yields_{1_g} \upstairs{a} : \upstairs{A}$.

\item A substitution $\Gamma \qyields \Theta : \Delta$ is represented by a term $g : \upstairs{\Gamma} \yields_g \upstairs{\Theta} : \upstairs{\Delta}$.

\subsubsection{Structural Rules}

\begin{theorem}
The judgements and structural rules of MLTT can be interpreted in any comprehension object (Definition \ref{def:comprehension-object})
\end{theorem}

\begin{enumerate}
\item[\textsc{ctx-empty}] Define $\upstairs{(\cdot)} :\equiv \St{!_p}{\mt}$.
\item[\textsc{ctx-ext}] Given $\alpha : \upstairs{\Gamma} \yields_{\El{p}{g}} \upstairs{A} \TYPE$, define $\upstairs{\Gamma, A} :\equiv \St{\chi}{\telety{\alpha}{\upstairs{\Gamma}}{\upstairs{A}}}$
\item[\textsc{type-sub}] Given $\alpha : \upstairs{\Gamma} \yields_g \upstairs{\Theta} : \upstairs{\Delta}$ and $\alpha : \upstairs{\Delta} \yields_{\El{p}{\alpha}} \upstairs{A} \TYPE$ we can use framework substitution to form $\upstairs{A[\Theta]} :\equiv \upstairs{A}[\upstairs{\Theta}]$.
\item[\textsc{term-sub}] \mvrnote{Same as types.}
\item[\textsc{sub-ext}] We have
\begin{align*}
\alpha : \upstairs{\Gamma} &\yields_\alpha \upstairs{\Theta} : \upstairs{\Delta} \\
\alpha : \upstairs{\Gamma} &\yields_{1_\alpha} \upstairs{a} : \upstairs{A}[\upstairs{\Theta}]
\end{align*}
which are exactly what is needed to form
\begin{align*}
\upstairs{\Theta, a} :\equiv \rewrite{\eta^\chi_{\alpha}}{\StI{\chi}{(\upstairs{\Theta}, \upstairs{a})}}
\end{align*}
\item[\textsc{sub-id}] \mvrnote{Variable term}
\item[\textsc{sub-comp}] \mvrnote{Composition of substitutions}
\item[\textsc{sub-proj}] We are trying to construct a term
\begin{align*}
\beta : \St{\chi}{\telety{\alpha}{\upstairs{\Gamma}}{\upstairs{A}}} \yields_\beta \upstairs{\proj{\Gamma, A}} : \upstairs{\Gamma}
\end{align*}
Use:
\begin{align*}
\upstairs{\proj{\Gamma, A}} :\equiv \StE{\chi}{\beta}{w}{\rewrite{\pi^{\fst w}_{\snd w}}{\fst w}}
\end{align*}
\item[\textsc{var}] We are a building a term
\begin{align*}
\beta : \St{\chi}{\telety{\alpha}{\upstairs{\Gamma}}{\upstairs{A}}} \yields \upstairs{\var{\Gamma, A}} : \upstairs{A}[\upstairs{\proj{\Gamma, A}}]
\end{align*}
Note that:
\begin{align*}
&\upstairs{A}[\upstairs{\proj{\Gamma, A}}] \\
&\equiv \upstairs{A}[\StE{\chi}{\beta}{w}{\rewrite{\pi^{\fst w}_{\snd w}}{\fst w}}/\alpha] \\
&\equiv \StE{\chi}{\beta}{w}{\upstairs{A}[\rewrite{\pi^{\fst w}_{\snd w}}{\fst w}/\alpha]} \\
&\equiv \StE{\chi}{\beta}{w}{\St{\ApEl{p}{\pi^{\fst w}_{\snd w}}}{\upstairs{A}[\fst w/\alpha]}} 
\end{align*}
So we can build $\upstairs{\var{\Gamma, A}}$ by:
\begin{align*}
\upstairs{\var{\Gamma, A}} :\equiv \StE{\chi}{\beta}{w}{\rewrite{\var{\snd w}}{\StI{\ApEl{p}{\pi^{\fst w}_{\snd w}}}{\snd w}}}
\end{align*}
\end{enumerate}

Now we check that these translations satisfy the required equations. The first three blocks of equations follow quickly from the analogous properties in the framework. Here are the equations we might be worried about:
\begin{enumerate}[style = multiline, labelwidth = 80pt]
\item[{$A[\Theta ; \kappa] \equiv A[\kappa][\Theta]$}] 
\item[{$a[\Theta ; \kappa] \equiv a[\kappa][\Theta]$}] 
\item[{$\Theta ; (\kappa , a) \equiv (\Theta ; \kappa) , a[\Theta]$}] 
\item[{$(\Theta, a);\proj{\Gamma,A} \equiv \Theta$}]
\item[{$\var{\Delta,A}[\Theta, a] \equiv a$}] 
\item[{$(\proj{\Gamma,A}, \var{\Gamma,A}) \equiv \id_{\Gamma, A}$}]
\end{enumerate}
\end{itemize}

\begin{lemma}
There are framework substitutions:
\begin{align*}
\upstairs{\Gamma, A} &\yields \mathsf{outof} : \upstairs{\Gamma}, \upstairs{A} \\
\upstairs{\Gamma}, \upstairs{A} &\yields \mathsf{into} : \upstairs{\Gamma, A}
\end{align*}
that are almost but not quite inverses.
\end{lemma}
\begin{proof}
We can build the first substitution by:
\begin{align*}
\beta : \upstairs{\Gamma, A} &\yields \mathsf{outof} :\equiv (\upstairs{\proj{\Gamma, A}}, \upstairs{\var{\Gamma, A}}) : \upstairs{\Gamma}, \upstairs{A}
\end{align*}
For the second, we have
\begin{align*}
\alpha : \upstairs{\Gamma}, x : \upstairs{A} &\yields \mathsf{into} :\equiv \StI{\chi}{(\alpha, x)} : \upstairs{\Gamma, A}
\end{align*}
And we can check the composites in both directions:
%\StE{\chi}{\beta}{w}{\rewrite{\var{x}}{\StI{\ApEl{p}{\pi^{\fst w}_{\snd w}}}{\snd w}}}
\begin{align*}
\mathsf{outof}[\mathsf{into}]
&\equiv (\upstairs{\proj{\Gamma, A}}, \upstairs{\var{\Gamma, A}})[\StI{\chi}{(\alpha, x)}/\beta] \\
&\equiv (\StE{\chi}{\StI{\chi}{(\alpha, x)}}{w}{\rewrite{\pi^{\fst w}_{\snd w}}{\fst w}}, \StE{\chi}{\StI{\chi}{(\alpha, x)}}{w}{\rewrite{\var{\snd w}}{\StI{\ApEl{p}{\pi^{\fst w}_{\snd w}}}{\snd w}}}) \\
&\equiv (\rewrite{\pi^\alpha_x}{\alpha}, \rewrite{\var{x}}{\StI{\ApEl{p}{\pi^{\alpha}_x}}{x}}) \\
&\equiv \rewrite{(\pi^\alpha_x, \var{x})}{\alpha, x} \\
\mathsf{into}[\mathsf{outof}]
&\equiv \StI{\chi}{(\alpha, x)}[\StE{\chi}{\beta}{w}{\rewrite{\pi^{\fst w}_{\snd w}}{\fst w}}/\alpha, \StE{\chi}{\beta}{w}{\rewrite{\var{\snd w}}{\StI{\ApEl{p}{\pi^{\fst w}_{\snd w}}}{\snd w}}} / x] \\
&\equiv \StE{\chi}{\beta}{w}{\StI{\chi}{(\alpha, x)}[\rewrite{\pi^{\fst w}_{\snd w}}{\fst w}/\alpha, \rewrite{\var{\snd w}}{\StI{\ApEl{p}{\pi^{\fst w}_{\snd w}}}{\snd w}} / x]} \\
&\equiv \StE{\chi}{\beta}{w}{\StI{\chi}{(\rewrite{\pi^{\fst w}_{\snd w}}{\fst w}, \rewrite{\var{\snd w}}{\StI{\ApEl{p}{\pi^{\fst w}_{\snd w}}}{\snd w})}}} \\
&\equiv \StE{\chi}{\beta}{w}{\StI{\chi}{\rewrite{(\pi^\alpha_x, \var{x})}{(\fst w, \snd w)}}} \\
&\equiv \StE{\chi}{\beta}{w}{\StI{\chi}{\rewrite{(\pi^\alpha_x, \var{x})}{w}}} \\
&\equiv \StE{\chi}{\beta}{w}{\rewrite{\ApPlus{\chi}{(\pi^\alpha_x, \var{x})}}{\StI{\chi}{w}}} \\
&\equiv \rewrite{\ApPlus{\chi}{(\pi^\alpha_x, \var{x})}}{\beta}
\end{align*}
\mvrnote{The above is sloppy about using substitutions vs telescopes}
\end{proof}



\subsubsection{Rules for $\Sigma$-types}

The rules for $\Sigma$-types are given in Figure~\ref{fig:qit-sigma-rules}. There are two versions of the eliminator; a weak and strong version. The choice of one or the other yields \emph{weak} $\Sigma$-types or \emph{strong} $\Sigma$-types.

\begin{figure}
\begin{mathpar}
\inferrule*[left=$\Sigma$-form]{\Gamma \qyields A \TYPE \and \Gamma, A \qyields B \TYPE}{\Gamma \qyields \Sigma_A B \TYPE} \\
\inferrule*[left=$\Sigma$-pair]{~}{\Gamma, A, B \qyields \mathsf{pair}_{A, B} : (\Sigma_A B)[\proj{\Gamma, A, B};\proj{\Gamma, A}]} \and
\inferrule*[left=$\Sigma$-split-strong]{\Gamma, A, B \qyields c : C[(\proj{\Gamma, A, B};\proj{\Gamma, A}), \mathsf{pair}_{A,B}]}{\Gamma, \Sigma_A B \qyields \mathsf{split}_{A, B}(c) : C}
%\inferrule*[left=$\Sigma$-pair]{\Gamma \qyields a : A \and \Gamma \qyields b : B[\hat{a}]}{\Gamma \qyields (a, b) : \Sigma_A B} \and
%\inferrule*[left=$\Sigma$-$\pi_1$]{\Gamma \qyields p : \Sigma_A B}{\Gamma \qyields \pi_1(p) : A} \and
%\inferrule*[left=$\Sigma$-$\pi_2$]{\Gamma \qyields p : \Sigma_A B}{\Gamma \qyields \pi_2(p) : B[\widehat{\pi_1(p)}]} \and
\end{mathpar}
\begin{align}
(\Sigma_A B)[\Theta] &\equiv \Sigma_{A[\Theta]} B[\Theta \uparrow A] \\
\mathsf{pair}_{A, B}[\Theta \uparrow A \uparrow B] &\equiv \mathsf{pair}_{A[\Theta], B[\Theta \uparrow A]} \\
\mathsf{split}_{A,B}(c)[\Theta \uparrow \Sigma_A B] &\equiv \mathsf{split}_{A[\Theta],B[\Theta \uparrow A]}(c[\Theta \uparrow A \uparrow B])  \\
\nonumber \\
\mathsf{split}_{A,B}(c)[(\proj{\Gamma, A, B};\proj{\Gamma, A}), \mathsf{pair}_{A,B}] &\equiv c
\intertext{(If framework $\mathsf{F}$-types have eta:)}
\mathsf{split}_{A,B}(\mathsf{pair}_{A,B}) &\equiv \var{\Gamma, \Sigma_A B}
\end{align}
\mvrnote{that eta is not general enough}
\vspace{1cm}
\begin{mathpar}
\inferrule*[left=$\Sigma$-split-weak]{\Gamma, A, B \qyields c : C[\proj{\Gamma, A, B};\proj{\Gamma, A}]}{\Gamma, \Sigma_A B \qyields \mathsf{wsplit}_{A, B}(c) : C[\proj{\Gamma, \Sigma_A B}]}
\end{mathpar}
\begin{align}
\mathsf{wsplit}_{A,B}(\mathsf{pair}_{A,B}) &\equiv \var{\Gamma, \Sigma_A B} \\
\mathsf{wsplit}_{A,B}(c)[\Theta \uparrow \Sigma_A B] &\equiv \mathsf{wsplit}_{A[\Theta],B[\Theta \uparrow A]}(c[\Theta \uparrow A \uparrow B]) \\
\end{align}
%\begin{align}
%%\pi_1(a, b) &\equiv a \\
%%\pi_2(a, b) &\equiv b \\
%%(\pi_1(p), \pi_2(p)) &\equiv p \\
%%(a, b)[\Theta] &\equiv (a[\Theta], b[\Theta])
%\end{align}
\caption{Rules for $\Sigma$-types in MLTT via Explicit Substitutions}\label{fig:qit-sigma-rules}
\end{figure}

\begin{theorem}
The rules for strong $\Sigma$-types can be interpreted in any comprehension object that supports strong $\Sigma$-types (Definition \ref{def:supports-strong-sigmas}).
\end{theorem}

In the framework, $\Sigma$-types are the $\mathsf{F}$-types for the mode term
\begin{align*}
\alpha : p, x : \El{p}{\alpha}, y : \El{p}{\alpha.x} \yields \Sigma_1(\alpha,x,y) : \El{p}{\alpha}
\end{align*}
\mvrnote{TODO: What properties of this do we need?}

We now translate all the rules:
\begin{enumerate}[style = multiline, labelwidth = 80pt]
\item[\textsc{$\Sigma$-form}] We are given
\begin{align*}
\alpha : \upstairs{\Gamma} &\yields_{\El{p}{\alpha}} \upstairs{A} \TYPE \\
\beta : \St{\chi}{\telety{\alpha}{\upstairs{\Gamma}}{\upstairs{A}}} &\yields_{\El{p}{\beta}} \upstairs{B} \TYPE \\
\end{align*}
So form
\begin{align*}
\alpha : \upstairs{\Gamma} \yields_{\El{p}{\alpha}} \upstairs{\Sigma_A B} :\equiv \F{w. \Sigma_1(\alpha,\fst w, \snd w)}{\telety{x}{\upstairs{A}}{\upstairs{B}[\StI{\chi}{(\alpha, x)}/\beta]}} \TYPE
\end{align*}
Or written out:
\begin{mathpar}
\inferrule*[Left=F-form]
{\inferrule*[Left=$()$-form]
{\alpha : \upstairs{\Gamma} \yields_{\El{p}{\alpha}} : \upstairs{A} \TYPE \and 
\inferrule*[left=cut]{
\alpha : \upstairs{\Gamma}, x : \El{p}{\alpha} \yields \StI{\chi}{(\alpha, x)} : \St{\chi}{\telety{\alpha}{\upstairs{\Gamma}}{\upstairs{A}}}
\and 
\beta : \St{\chi}{\telety{\alpha}{\upstairs{\Gamma}}{\upstairs{A}}}\yields_{\El{p}{\beta}} \upstairs{B} \TYPE
}{
\alpha : \upstairs{\Gamma}, x : \El{p}{\alpha} \yields_{\El{p}{\alpha.x}} \upstairs{B}[\StI{\chi}{(\alpha, x)}/\beta] \TYPE
}
}
{\alpha : \upstairs{\Gamma} \yields_{\telety{x}{\El{p}{\alpha}}{\El{p}{\alpha.x}}} \telety{x}{\upstairs{A}}{\upstairs{B}[\StI{\chi}{(\alpha, x)}/\beta]} \TYPE}}
{\alpha : \upstairs{\Gamma} \yields_{\El{p}{\alpha}} \upstairs{\Sigma_A B} :\equiv \F{w. \Sigma_1(\alpha,\fst w, \snd w)}{\telety{x}{\upstairs{A}}{\upstairs{B}[\StI{\chi}{(\alpha, x)}/\beta]}} \TYPE}
\end{mathpar}
\item[\textsc{$\Sigma$-pair}] We are trying to construct a term of type
\begin{align*}
\delta :\upstairs{\Gamma, A, B} \yields \upstairs{(\Sigma_A B)[\proj{\Gamma, A, B};\proj{\Gamma, A}]} \TYPE
\end{align*}
Note that we are done if we can build a term of type
\begin{align*}
\alpha :\upstairs{\Gamma}, x : \upstairs{A}, y : \upstairs{B}[\StI{\chi}{(\alpha, x)}/\beta] \yields \upstairs{\Sigma_A B} \TYPE
\end{align*}
because we can then precompose with the substitution $[\mathsf{outof}_{\Gamma, A}, y/y][\mathsf{outof}_{\Gamma, A, B}]$ which has type:
\begin{align*}
\delta :\upstairs{\Gamma, A, B} \yields \alpha : \upstairs{\Gamma}, x : \upstairs{A}, y : \upstairs{B}[\StI{\chi}{(\alpha, x)}/\beta]
\end{align*}

Building the required term is easy:
\begin{align*}
\alpha : \upstairs{\Gamma}, x : \upstairs{A}, y : \upstairs{B}[\StI{\chi}{(\alpha, x)}/\beta] \yields \FI{(x, y)} : \F{w. \Sigma_1(\alpha,\fst w, \snd w)}{\telety{x}{\upstairs{A}}{\upstairs{B}[\StI{\chi}{(\alpha, x)}/\beta]}}
\end{align*}
%on the left:
%\begin{align*}
%\upstairs{\Gamma, A, B} \equiv \St{\chi}{\telety{\beta}{\St{\chi}{\telety{\alpha}{\upstairs{\Gamma}}{\upstairs{A}}}}{\upstairs{B}}}
%\end{align*}
%on the right:
%\begin{align*}
%&\upstairs{(\Sigma_A B)[\proj{\Gamma, A, B};\proj{\Gamma, A}]} \\ 
%&\equiv \F{w. \Sigma_1(\alpha,\fst w, \snd w)}{\telety{x}{\upstairs{A}}{\upstairs{B}[\StI{\chi}{(\alpha, x)}/\beta]}}\upstairs{[\proj{\Gamma, A, B};\proj{\Gamma, A}]} \\ 
%&\equiv \F{w. \Sigma_1(\delta,\fst w, \snd w)}{\telety{x}{\upstairs{A}\upstairs{[\proj{\Gamma, A, B};\proj{\Gamma, A}]}}{\upstairs{B}[\StI{\chi}{(\alpha, x)}/\beta]\upstairs{[\proj{\Gamma, A, B};\proj{\Gamma, A}]}}} \\ 
%\end{align*}
%Note that:
%\begin{align*}
%&\upstairs{B}[\StI{\chi}{(\alpha, x)}/\beta]\upstairs{[\proj{\Gamma, A}]} \\
%&\equiv \upstairs{B}[\StI{\chi}{(\alpha, x)}/\beta][\StE{\chi}{\beta}{w}{\rewrite{\pi^{\fst w}_{\snd w}}{\fst w}}/\alpha] \\
%&\equiv \upstairs{B}[\StI{\chi}{(\StE{\chi}{\beta}{w}{\rewrite{\pi^{\fst w}_{\snd w}}{\fst w}}, x)}/\beta] \\
%&\equiv \upstairs{B}[\StE{\chi}{\beta}{w}{\StI{\chi}{(\rewrite{\pi^{\fst w}_{\snd w}}{\fst w}, x)}}/\beta] \\
%\end{align*}

\item[\textsc{$\Sigma$-split-strong}] 
\end{enumerate}

Now the equations:
\begin{enumerate}[style = multiline, labelwidth = 80pt]
\item[{$(\Sigma_A B)[\Theta] \equiv \Sigma_{A[\Theta]} B[\Theta \uparrow A]$}:] 
\item[{$\mathsf{split}_{A,B}(c)\allowbreak[(\proj{\Gamma, A, B};\proj{\Gamma, A}), \allowbreak\mathsf{pair}_{A,B}] \equiv c$}:] 

\item[{$\mathsf{pair}_{A, B}[\Theta \uparrow A \uparrow B] \equiv \mathsf{pair}_{A[\Theta], B[\Theta \uparrow A]}$}:]
\item[{$\mathsf{split}_{A,B}(c)[\Theta \uparrow \Sigma_A B] \equiv \mathsf{split}_{A[\Theta],B[\Theta \uparrow A]}(c[\Theta \uparrow A \uparrow B])$}:] 
\end{enumerate}


\section{Semantics}
\label{sec:semantics}

\subsection{2-categories with families}
\label{sec:2cwfs}

The ``canonical'' semantics should interpret each judgment as follows:

Mode theory judgements:
\begin{enumerate}
\item $\mm{\gamma \ctx}$ is a category.
\item $\mm{\gamma \yields p \type}$ is a functor $\mm{\gamma}\op \to \Cat$.
\item $\mm{\TypeTwo{\gamma}{s}{p}{q}}$ is a natural transformation $\mm{\gamma \yields q} \Rightarrow \mm{\gamma \yields p}$ (note reversal of direction; this is because mode morphisms act contravariantly on mode terms and on upstairs subscripts).
\item $\mm{\gamma \yields \mu : p}$ is a section of the projection from the Grothendieck construction $\int\mm{\gamma\yields p} \to \mm{\gamma}$.
  In particular, it assigns to every object $x\in \mm{\gamma}$ an object $\mm{\gamma \yields \mu : p}(x)\in \mm{\gamma\yields p}(x)$.
\item $\mm{\TermTwoT{\gamma}{s}{\mu}{\nu}{p}}$ is a natural transformation of such sections over the identity, i.e.\ whose composite with the projection is the identity natural transformation of the identity functor.
\end{enumerate}

Top judgements: 
\begin{itemize}
\item $\mm{\yields_\gamma \Gamma \CTX}$ is an object of $\mm{\gamma}$
\item $\mm{\Gamma \yields_p A \TYPE}$ is an object of $\mm{\gamma \yields p}(\mm{\Gamma})$.
\item $\mm{\Gamma \yields_\mu M : A}$ is a morphism from $\mm{\gamma \yields \mu : p}(\mm{\Gamma})$ to $\mm{\Gamma \yields_p A}$ in $\mm{\gamma \yields p}(\mm{\Gamma})$.
\end{itemize}

The general categorical semantics is an abstraction of these structures --- categories, contravariant $\Cat$-valued functors, Grothendieck constructions, sections, natural transformations, objects, and morphisms.

\begin{enumerate}
\item Categories form a 2-category.
  In general we will stipulate an arbitrary (strict) 2-category $\M$.
\item Given a category $C$, the collection of functors and (strict) natural transformations $C\op \to \Cat$ forms a (large) category $[C\op, \Cat]$, and if we reverse the directions of the natural transformations we get $[C\op, \Cat]\op$.
  Moreover, precomposition with functors $C\to C'$ and natural transformations between them makes $[(-)\op, \Cat]\op$ into a strict 2-functor $\Cat\op \to \CAT$; syntactically these are $q[\mu/x]$ and $\ap{q}{\mu/x}$ respectively.

  Here $\CAT$ denotes the very large 2-category of large categories.
  Note that this 2-functor is covariant on 2-cells: the two $(-)\op$s cancel each other out at that level.
  In fact, this 2-functor can be identified with the representable $[-,\Cat\op]$.

  Thus, in general we will stipulate a strict 2-functor $\Mty:\M\op \to \CAT$.
  \addtocounter{enumi}{1}
\item Given a category $C$ and a functor $T:C\op\to Cat$, the sections of the projection $\int T \to C$, and natural transformations over the identity, form a category.
  Moreover, such sections also vary functorially as $C$ does.
  Thus, in general we will stipulate another strict 2-functor $\Mtm : \M\op\to\CAT$ with a strictly 2-natural projection map $\Mtm\to\Mty$.

  The contravariant action of mode morphisms $\TypeTwo{\gamma}{s}{p}{q}$ on mode terms tells us that the morphisms of $\Mty(C)$ must act on the objects of $\Mtm$ contravariantly.
  Moreover, this action is strictly functorial, and respected by substitution.
  Thus, we stipulate that $\Mtm \to \Mty$ is a \emph{split fibration} internal to the 2-category $[\M\coop,\CAT]$, which means that each functor $\Mtm(C) \to \Mty(C)$ is a split fibration and that all the naturality squares
  \begin{center}
    \begin{tikzcd}
      \Mtm(C) \ar[d] \ar[r] & \Mtm(C')\ar[d] \\
      \Mty(C) \ar[r] & \Mty(C')
    \end{tikzcd}
  \end{center}
  are strict morphisms of split fibrations (preserve the splittings on the nose).

  To represent the Grothendieck construction $\int T$ itself, we stipulate that this projection is additionally a \emph{representable morphism} in that for any $C\in \M$, if we form the pullback in $[\M\coop,\CAT]$
  \begin{center}
    \begin{tikzcd}
      \bullet \ar[d] \ar[r] & \Mtm \ar[d] \\
      y(C) \ar[r,"\name{T}"] & \Mty
    \end{tikzcd}
  \end{center}
  then the pullback object is also of the form $y(\int T)$ for some object $\int T\in \M$.
  Here $y(C)$ denotes the representable functor $y(C)(-) = \M(-,C) : \M\op\to\CAT$.
  The Yoneda lemma implies that (strict) 2-natural transformations $\name{T} : y(C) \to \Mty$ are in bijection with objects $T\in \Mty(C)$; thus every $T\in \Mty(C)$ induces an object $\int T$ with a projection $\int T \to C$ in $\M$, such that elements of $\Mtm(C)$ over $T$ are in bijection with sections of this projection.
  To make the notion algebraic, we require the object $\int T$ to be a specified function of $C$ and $T$.

  Since the above pullback is a pullback in a 2-category, it also has a universal property for 2-cells.
  Thus, morphisms in $\Mtm$ (over the identity in $\Mty$) are in bijection with 2-cells between sections (over the identity).

  The 2-categorical Yoneda lemma also tells us that morphisms $\mu : S\to T$ in $\Mty(C)$ correspond bijectively to modifications $\name{\mu}:\name{S} \to \name{T}$.
  Since $\Mtm\to\Mty$ is a fibration, such a 2-cell induces a universal map in the other direction $\int \mu : \int T\to \int S$ (see Hermida, Buckley, Johnstone), such that postcomposing with $\int \mu$ corresponds to the split (contravariant) fibrational action of $\mu$ on elements of $\Mtm$.
\end{enumerate}

Thus, the entire mode theory except for 1- and $\Sigma$-modes is encapsulated semantically by:

\begin{definition}
  A \textbf{2-category with families} is a 2-category $\M$ together with two 2-functors $\Mty,\Mtm : \M\op\to\CAT$ and a 2-natural transformation $\Upsilon:\Mtm\to \Mty$ that is both (1) an internal split fibration and (2) algebraically representable.
\end{definition}

Pleasingly, this is a straightforward categorification of the standard notion of category with families.
The only really new ingredient is the requirement that $\Mtm\to \Mty$ be an internal fibration, which has no analogue for 1-categories.

The ruminations above suggest that there should be a 2-category with families where $\M=\Cat$ and $\Mty(C) = [C\op,\Cat]\op$.
In fact we can construct this precisely as an instance of a much more general operation, similar to the ``global universe'' coherence method for ordinary type theory introduced by Voevodsky.
(There should also be a ``local universe'' method analogous to that of Lumsdaine--Warren, but we do not need that here.)

\begin{problem}
  Let $\M'$ be any 2-category containing an internal split fibration $\upsilon : \Util \to U$, and let $\M\subseteq \M'$ be a full subcategory such that if $C\in \M$ and $C'\to C$ is a pullback of $\upsilon$, then $C'\in \M$.
  Then $\M$ is a 2-category with families with
  \begin{align*}
    \Mty(C) &= \M'(C,U)\\
    \Mtm(C) &= \M'(C,\Util).
  \end{align*}
\end{problem}
\begin{constr}
  The definitions of $\Mty$ and $\Mtm$ are certainly 2-functors, since they are representable, and the morphism $\upsilon$ induces a 2-natural transformation $\Upsilon:\Mtm\to\Mty$.
  Since split fibrations can be defined in terms of finite limits and diagrammatic 2-categorical structure (adjunctions), they are preserved by the restricted Yoneda embedding; thus $\Upsilon$ is also a split fibration.
  Finally, since the restricted Yoneda embedding also preserves pullbacks, it preserves any pullback of $\upsilon$ to a representable; hence the vertex of any such pullback is again representable.
  Thus, $\Upsilon$ is a representable morphism (and by choosing pullbacks in $\M'$ we can make $\Upsilon$ algebraically representable).
\end{constr}

\begin{example}
  Let $\M'=\CAT$ and $\M=\Cat$, with $U = \Cat\op$ and $\Util$ the Grothendieck construction of the contravariant functor $(\Cat\op)\op \cong \Cat \hookrightarrow \CAT$.
  Thus, $\Cat$ becomes a 2-category with families where $\Mty(C) \cong [C,\Cat\op] \cong [C\op,\Cat]\op$, as in the above motivating discussion.
\end{example}

\subsection{$\Sigma$-modes}
\label{sec:2-sigmas}

Emboldened by this success, let's just try to write down a notion of $\Sigma$-types for 2-categories with families by categorifying the usual one for 1-categories.
As formulated by Awodey, $\Sigma$-types for a 1-category with families involve the dependent product of the presheaf $\mathrm{Ty}$ along the projection $\mathrm{Tm}\to \mathrm{Ty}$, to encode the notion of one type dependent on another one.
Unlike a presheaf 1-category, the presheaf 2-category $[\M\op,\CAT]$ is not locally cartesian closed, so not all dependent products exist; but fortunately, dependent products along fibrations do exist.

\begin{lemma}\label{thm:fib-exp}
  For a 2-category $\M$, any split fibration $\Upsilon : S\to T$ in $[\M\op,\CAT]$ is exponentiable.
\end{lemma}
\begin{proof}
  Suppose given $R\to S$; we want to define $\Pi_S[R]$ over $T$.
  Given an object $x\in T(C)$, corresponding to a morphism $\name{x}:y(C)\to T$, the objects of $\Pi_S[R](C)$ over $x$ must be bijective to the lifts of $\name{x}$ to $\Pi_S [R]$.
  The desired universal property of $\Pi_S [R]$ means that such lifts are bijective to maps $x^* S \to R$ over $S$.
  So we take the latter as the \emph{definition} the objects of $\Pi_S [R]$, i.e.\
  \[ \ob (\Pi_S [R](C)) = \sum_{x:\ob (T(C))} \ob (\M_{/S}(x^*S, R)). \]
  Next, given a morphism $\alpha :x\to y$ in $T(C)$ and objects $u,v$ of $\Pi_S[R]$ over $x,y$ respectively, morphisms $u\to v$ over $\alpha$ must correspond bijectively to lifts of the corresponding 2-cell $\name{\alpha} :\name{x} \to \name{y} :y(C) \to T$ to a 2-cell between $\name{u},\name{v} : y(C) \to \Pi_S[R]$.
  However, the desired universal property of $\Pi_S[R]$ as a dependent product doesn't obviously determine such lifts, because the 2-cell $\name{\alpha}$ doesn't live in the slice category $\M/T$.
  But we can use the fibration structure of $\Upsilon$, which induces a map $\alpha^*S : y^*S \to x^*S$ over $y(C)$ and a 2-cell
  \begin{center}
    \begin{tikzcd}
      x^*S \ar[drr] \\
      & \Downarrow\overline{\alpha} & S\\
      y^*S \ar[urr] \ar[uu,"\alpha^*S"]
    \end{tikzcd}
  \end{center}
  over $\alpha$ with a universal property (see Hermida, Buckley, Johnstone).
  Thus, if $u$ and $v$ are determined by maps $\hat{u}:x^*S \to R$ and $\hat{v}:y^*S\to R$ over $S$, we can define a morphism $u\to v$ over $\alpha$ to be determined by a lift $\hat{\alpha}$ of $\overline{\alpha}$ to $R$.

  The maps $\alpha^*S $ and 2-cells $\overline{\alpha}$ are functorial in $\alpha$ in all the ways one would hope (strictly so, since $\Upsilon$ is split --- although this is not really necessary), so it is straightforward to make $\Pi_S[R](C)$ thusly defined into a category and $\Pi_S[R]$ into a functor $\M\op\to\CAT$ over $T$.

  Note in particular that when $\alpha$ is an identity, so is $\overline{\alpha}$, so the fiber of $\Pi_S[R]$ over an object $x$ is the category $\M_{/S}(x^*S, R)$.
  This means that $\Pi_S[R]$ has the desired universal property with respect to maps out of representables, i.e.\ we have a natural isomorphism of hom-categories
  \[\M_{/T}(x:y(C)\to T,\Pi_S[R]) \cong \M_{/S}(x^*S, R).\]
  It is straightforward to extend this, using the Yoneda lemma, to the full universal property.
\end{proof}

Proceeding by analogy with the 1-categorical case, consider the dependent product $\Pi_\Upsilon[\Mty]$, where $\Mty$ denotes abusively the pullback $\Mtm \times \Mty \to \Mtm$ in $\M/\Mtm$.
The universal structure of $\Pi_\Upsilon[\Mty]$ consists of an ``evaluation'' map $\ev:\Pi_\Upsilon[\Mty] \times_\Mty \Mtm \to \Mty$.
Note that the projection $\Pi_\Upsilon[\Mty] \times_\Mty \Mtm \to \Pi_\Upsilon[\Mty]$ is a split fibration, since it is a pullback of $\Upsilon:\Mtm \to \Mty$.
And we can also pull $\Upsilon$ back along the evaluation map to get a further split fibration:
\begin{center}
  \begin{tikzcd}
    \ev^* \Mtm \ar[r] \ar[d,->>] \ar[dr,phantom,near start,"\lrcorner"] & \Mtm \ar[d,->>]\\
    \Pi_\Upsilon[\Mty] \times_\Mty \Mtm \ar[r] \ar[d,->>] \ar[ddr,phantom,near start,"\lrcorner"] \ar[dr] & \Mty\\
    \Pi_\Upsilon[\Mty] \ar[dr] & \Mtm \ar[d,->>] \\
    & \Mty
  \end{tikzcd}
\end{center}
Note that the composite of two split fibrations is again a split fibration.

\begin{definition}
  A 2-category with families has \textbf{$\Sigma$-types} if it is equipped with maps $\Pi_\Upsilon[\Mty] \to \Mty$ and $\ev^* \Mtm \to \Mtm$ such that the square
  \begin{center}
    \begin{tikzcd}
      \ev^* \Mtm \ar[r] \ar[d,->>] & \Mtm \ar[dd,->>]\\
    \Pi_\Upsilon[\Mty] \times_\Mty \Mtm \ar[d,->>] \\
    \Pi_\Upsilon[\Mty] \ar[r] & \Mty
    \end{tikzcd}
  \end{center}
  (1) commutes, (2) is a pullback, and (3) is a strict morphism of split fibrations.
\end{definition}

Let's unwind this to correspond to syntax.
By the construction in Lemma \ref{thm:fib-exp}, an object of $\Pi_\Upsilon[\Mty](\gamma)$ lives over a type $p\in \Mty(\gamma)$, and consists of the additional data of a map $p^*\Mtm \to \Mty$.
But since $\Upsilon$ is representable, $p^* \Mtm$ is also representable, by the extended context $\gamma.p$; thus this additional data is a type $\gamma,x:p \yields q \type$.
The morphism $\Pi_\Upsilon[\Mty] \to \Mty$ thus assigns to any such pair a mode type $\gamma \yields \sigmacl{x}{p}{q} \type$.

Similarly, an object of $\ev^* \Mtm$ over $(p,q)$ consists of a pair of a term $\gamma \yields \mu : p$ and $\gamma \yields \nu : q[\mu/x]$, and so the morphism $\ev^* \Mtm$ assigns to this the pair $(p,q):\sigmacl{x}{p}{q}$.
The fact that the square is a pullback (on objects) means that we have $\fst$ and $\snd$ collectively forming an inverse isomorphism, i.e.\ the $\beta$- and $\eta$-rules hold for $\Sigma$-modes.

The action of $\Pi_\Upsilon[\Mty] \to \Mty$ on morphisms gives the congruence rules for $\Sigma$ on mode type morphisms, along with its functoriality laws (since this map over each $\gamma$ is a functor) and its naturality law for substitution (since this map is a natural transformation as $\gamma$ varies).
Similarly, the action of $\ev^* \Mtm \to \Mtm$ on morphisms gives the 2-cell operation on $\Sigma$-modes, and the fact that the square is a pullback on morphisms gives the $\beta$- and $\eta$-rules for these.

Finally, the fact that this square is a strict morphism of split fibrations gives the equation for ``transport'' in $\Sigma$.


\end{document}

\documentclass[10pt]{article}
  \usepackage{xcolor}
  \definecolor{darkgreen}{rgb}{0,0.45,0} 
  \usepackage[pagebackref,colorlinks,citecolor=darkgreen,linkcolor=darkgreen]{hyperref}
  \usepackage{pdflscape}

\usepackage{fullpage}
\usepackage{amssymb,amsthm,bbm}
\usepackage[centertags]{amsmath}
\usepackage[mathscr]{euscript}
\usepackage{dsfont}
\usepackage{fontawesome}
\usepackage{tikz-cd}
\usepackage{mathpartir}
\usepackage{enumitem}
\usepackage[status=draft,inline,nomargin]{fixme}
\FXRegisterAuthor{ms}{anms}{\color{blue}MS}
\FXRegisterAuthor{mvr}{anmvr}{\color{olive}MVR}
\FXRegisterAuthor{drl}{andrl}{\color{purple}DRL}
\usepackage{stmaryrd}
\usepackage{mathtools}

\newtheorem{theorem}{Theorem}
\newtheorem{proposition}{Proposition}
\newtheorem{lemma}{Lemma}
\newtheorem{corollary}{Corollary}
\newtheorem{problem}{Problem}
\newenvironment{constr}{\begin{proof}[Construction]}{\end{proof}}

\theoremstyle{definition}
\newtheorem{definition}{Definition}
\newtheorem{remark}{Remark}
\newtheorem{example}{Example}

\let\oldemptyset\emptyset%
\let\emptyset\varnothing

\newcommand\dsd[1]{\ensuremath{\mathsf{#1}}}

\newcommand{\yields}{\vdash}
\newcommand{\Yields}{\vDash}
\newcommand{\cbar}{\, | \,}
\newcommand{\judge}{\mathcal{J}}

\newcommand{\Id}[3]{\mathsf{Id}_{{#1}}(#2,#3)}
\newcommand{\CTX}{\,\,\mathsf{Ctx}}
\newcommand{\ctx}{\,\,\mathsf{mctx}}
\newcommand{\TYPE}{\,\,\mathsf{Type}}
\newcommand{\type}{\,\,\mathsf{mode}}
\newcommand{\TELE}{\,\,\mathsf{Tele}}
\newcommand{\tele}{\,\,\mathsf{mtele}}

\newcommand{\app}[2]{\ensuremath{#1 \: #2}}
\newcommand{\telety}[3]{\ensuremath{(#1{:}#2,#3)}}
\newcommand{\mt}[0]{\ensuremath{()}}
\newcommand{\sigmacl}[3]{\ensuremath{\textnormal{$\Sigma$}\,#1{:}#2.\,#3}}
\newcommand{\fst}[1]{\app{\dsd{fst}}{#1}}
\newcommand{\snd}[1]{\app{\dsd{snd}}{#1}}
\newcommand\extend[2]{\ensuremath{(#1,\id_{#2})}}

\newcommand\fan[1]{\ensuremath{\mathsf{fan}_{#1}}}

\newcommand{\id}{\mathsf{id}}
\DeclareMathOperator{\ob}{ob}

\newcommand{\rewrite}[2]{\overleftarrow{#1}(#2)}
\newcommand\F[2]{\ensuremath{\mathsf{F}_{#1}(#2)}}
\newcommand\U[3]{\ensuremath{\mathsf{U}_{#1}(#2 \mid #3)}}
\newcommand\UE[2]{\ensuremath{#1(#2)}}
\newcommand\UI[2]{\ensuremath{\lambda #1.#2}}
\newcommand\St[2]{\ensuremath{{#1}^*(#2)}}
\newcommand\StI[2]{\ensuremath{\mathsf{st}_{#1}(#2)}}
\newcommand\UStI[2]{\ensuremath{\mathsf{ust}_{#1}(#2)}}
\newcommand\UnSt[2]{\ensuremath{\mathsf{unst}_{#1}(#2)}}
%\newcommand\StE[2]{\ensuremath{\mathsf{unst}(#1,#2)}}
\newcommand\StE[4]{\ensuremath{\mathsf{let} \, \StI{#1}{#3} \, = \, {#2} \, \mathsf{in} \, #4}}
\newcommand\FE[3]{\ensuremath{\mathsf{let} \, \mathsf{F}(#2) \, = \, {#1} \, \mathsf{in} \, #3}}
% With subscript:
\newcommand\FEs[4]{\ensuremath{\mathsf{let} \, \mathsf{F}_{#1}(#3) \, = \, {#2} \, \mathsf{in} \, #4}} 
\newcommand\FI[1]{\ensuremath{\mathsf{F}{(#1)}}}
\newcommand\FIs[2]{\ensuremath{\mathsf{F}_{#1}{(#2)}}}
\newcommand\TypeTwo[4]{\ensuremath{#1 \mid #3 \vDash #2 : #4}}
\newcommand\TeleTwo[4]{\ensuremath{#1 \mid #3 \vDash #2 : #4}}
\newcommand\TermTwo[4]{\ensuremath{#1 \mid #3 \vDash #2 : #4}}
\newcommand\TermTwoT[5]{\ensuremath{#1 \mid #3 \vDash_{#5} {#2} : #4}}
%% \newcommand\TermTwoDisp[5]{\ensuremath{#1 \mid #3 \vDash_{\mathsf{disp}} #2 :_{#5} #4}}
\newcommand\SubTwo[4]{\ensuremath{#1 \mid #3 \vDash #2 : #4}}
\newcommand\TrPlus[2]{\ensuremath{{#1}^+(#2)}}
\newcommand\TrCirc[2]{\ensuremath{{#1}^\circ(#2)}}

\newcommand\El[2]{\mathcal{T}_{#1}(#2)}
\newcommand\ApEl[2]{\mathcal{T}_{#1}\langle#2\rangle}
\newcommand\bdot[0]{\mathbin{.}}
\newcommand\bang[0]{\mathord{!}}

\newcommand\ap[2]{\ensuremath{#1 \langle #2 \rangle }}
\newcommand\ApPlus[2]{\ensuremath{{#1}^+ \langle #2 \rangle }}
\newcommand\ApCirc[2]{\ensuremath{{#1}^\circ \langle #2 \rangle }}

% Macros for semantics notation
\newcommand\mm[1]{\llbracket #1 \rrbracket}
\newcommand\op{^{\mathrm{op}}}
\newcommand\co{^{\mathrm{co}}}
\newcommand\coop{^{\mathrm{coop}}}
\newcommand\Cat{\mathrm{Cat}}
\newcommand\CAT{\mathrm{CAT}}
\newcommand\M{\mathcal{M}}
\newcommand\Mhat{\widehat{\mathcal{M}}}
\newcommand\Mty{{\mathrm{Ty}_{\M}}}
\newcommand\Mtm{{\mathrm{Tm}_{\M}}}
\newcommand\Mtyhat{{\widehat{\mathrm{Ty}}_{\M}}}
\newcommand\Mtmhat{{\widehat{\mathrm{Tm}}_{\M}}}
\newcommand\Ups{\Upsilon}
\newcommand\Upshat{{\widehat{\Upsilon}}}
\newcommand\C{\mathcal{C}}
\newcommand\Chat{{\widehat{\mathcal{C}}}}
\newcommand\Cty{\mathrm{Ty}_{\C}}
\newcommand\Ctm{\mathrm{Tm}_{\C}}
\newcommand\Ctyhat{{\widehat{\mathrm{Ty}}}_{\C}}
\newcommand\Ctmhat{{\widehat{\mathrm{Tm}}}_{\C}}
\newcommand\vp{\varpi}
\newcommand\vpst{\vp^*}
\newcommand\vpsh{\vp_!}
\newcommand\vptil{\widetilde{\vp}}
\newcommand\vpty{{\vp}_{\mathrm{Ty}}}
\newcommand\vptm{{\vp}_{\mathrm{Tm}}}
\newcommand\name[1]{\ulcorner #1\urcorner}
\newcommand{\Util}{\widetilde{U}}
\newcommand\ev{\mathrm{ev}}
\DeclareSymbolFont{bbold}{U}{bbold}{m}{n}
\DeclareSymbolFontAlphabet{\mathbbb}{bbold}
\newcommand\one{\mathbbb{1}}

\title{A Fibrational Framework for Modal Dependent Type Theories}
\author{Daniel R. Licata, Mitchell Riley, Michael Shulman}
\date{}

\begin{document}
\maketitle
\tableofcontents

\section{Syntax}

\subsection{Overview of Judgements}

Mode theory judgements:
\begin{enumerate}
\item $\gamma \ctx$ (empty, extension)
\item $\gamma \yields p \type$ 
\item $\TypeTwo{\gamma}{s}{p}{q}$ (horizontal and vertical composition, identities)
\item $\gamma \yields \mu : p$ (variables, action of mode type morphisms)
\item $\TermTwoT{\gamma}{s}{\mu}{\nu}{p}$ (horizontal and vertical
    composition, identities)
\end{enumerate}

Top judgements: 
\begin{itemize}
\item $\yields_\gamma \Gamma \CTX$ over $\yields \gamma \ctx$
\item $\Gamma \yields_p A \TYPE$ over $\gamma \yields p \type$
\item $\Gamma \yields_\mu M : A$ over $\gamma \yields \mu : p$
\end{itemize}

Mode type morphisms $\TypeTwo{\gamma}{s}{p}{q}$ induce 1-cells contravariantly in the mode theory and 
$\TypeTwo{\gamma}{s}{p}{q}$ and $\TermTwo{\gamma}{s}{\mu}{\nu}$
act \emph{contravariantly} on the subscripts of upstairs terms.

We expect structurality to be admissible for the base, and structurality
over that to be admissible for the top, e.g.:
\begin{mathpar}
\inferrule*[Left = weaken-over]
           {\Gamma,\Gamma' \yields_\mu M : A \\ (\text{where } \gamma,\gamma' \vdash \mu : p)}
           {\Gamma,y:B,\Gamma' \yields_\mu M : A \\ (\text{where } \gamma,y:q,\gamma' \vdash \mu : p)}

\inferrule*[Left = subst-over]
           {\Gamma,x:A,\Gamma' \yields_\nu N : C \\ (\text{where } \gamma,x:p,\gamma' \vdash \nu : \gamma) \\\\
            \Gamma \vdash_\mu M : A \\ (\text{where } \gamma \vdash \mu : p)
           }
           {\Gamma,\Gamma'[M/x] \yields_{\nu[\mu/x]} N[M/x] : A[M/x] \\ (\text{where } \gamma,\gamma'[\mu/x] \vdash \nu[\mu/x] : p[\mu/x])}
\end{mathpar}


\subsection{Mode Theory}

\begin{enumerate}

\item Contexts are as usual:

\begin{mathpar}
  \inferrule*{ }
             {\cdot \ctx}
             
  \inferrule*
    {\gamma \ctx \\
     \gamma \yields p \type}
    {\gamma,x:p \ctx}
\end{mathpar}  

\item In all mode theories, terms must have: 

\begin{mathpar}
\inferrule*{ }
             {\gamma,x : p, \gamma' \yields x : p}
             
\inferrule*
    {\gamma \yields \mu : q \\
     \TypeTwo{\gamma}{s}{p}{q}
    }
    {\gamma \yields \TrPlus{s}{\mu} : p}
\\
\TrPlus{\id}{\mu} \equiv \mu \qquad
\TrPlus{s'}{\TrPlus{s}{\mu}} \equiv \TrPlus{(s';s)}{\mu} 
\end{mathpar}

\item Mode type morphisms:
\begin{mathpar}
    \inferrule*{ }
          {\TypeTwo{\gamma}{\id_p}{p}{p}}
    \qquad
    \inferrule*{{\TypeTwo{\gamma}{s_1}{p_1}{p_2}} \\
                {\TypeTwo{\gamma}{s_2}{p_2}{p_3}}
          }
          {\TypeTwo{\gamma}{s_1;s_2}{p_1}{p_3}}

\inferrule*{{\gamma,x:p} \vdash {q} \type \\
            \TermTwoT{\gamma}{t}{\mu}{\mu'}{p}\\
           } 
           {\TypeTwo{\gamma}{\ap {q} {t/x}}{q[\mu/x]}{q[\mu'/x]}}

\\
\id;s \equiv s \equiv s;\id \and
(s;s');s'' \equiv s;(s';s'') \\ 
\ap q {\id_{\mu}/x} \equiv \id_{q[\mu/x]} \and
\ap q {(s;t)/x} \equiv \ap q {s/x}; \ap q {t/x} \\ 
\ap q {s/\_} \equiv \id_q \\ 
\ap {(q[\mu/x])} {s/y} \equiv \ap q {\ap \mu {s/y}/x} \quad (\text{where } \gamma,y:p' \vdash \mu : p \text{ and } \gamma,x:p \vdash q \type)\\
s[\nu/x];\ap{q'}{t/x} \equiv \ap{q}{t/x};s[\nu'/x] \quad 
(\text{where } \TypeTwo{\gamma,x:p}{s}{q}{q'} \text{ and } \TermTwoT{\gamma}{t}{\nu}{\nu'}{p})

%% subst: \id_\mu[\nu/x] = \id_{\mu[\nu/x]}
%% subst: s[x/x] = s
%% subst: (s;t)[\mu/x] = s[\mu/x];t[\mu/x]
%% subst: s[\mu[\nu/x]/x] = s[\mu/x][\nu/x]
%% subst: ap q (s [\mu/x]) = (ap q s)[\mu/x] and generalization
\end{mathpar}

We write $\ap q {t/x}$ for whiskering (\dsd{ap} in book HoTT).

\item 2-cells between terms.  First, we have
  identity/composition/whiskering and associated equations (whiskering
  on the other side is given by substitution):
\begin{mathpar}
    \inferrule*{ }
          {\TermTwoT{\gamma}{\id_\mu}{\mu}{\mu}{p}}
    \qquad
    \inferrule*{{\TermTwoT{\gamma}{s_1}{\mu_1}{\mu_2}{p}} \\
                {\TermTwoT{\gamma}{s_2}{\mu_2}{\mu_3}{p}}
          }
   {\TermTwoT{\gamma}{s_1;s_2}{\mu_1}{\mu_3}{p}}

\inferrule*{{\gamma,x:p} \yields {\nu} : {q} \\
            \TermTwoT{\gamma}{s}{\mu}{\mu'}{p}\\
           } 
           {\TermTwoT{\gamma}{\ap \nu {s/x}}{\nu[\mu/x]}{\TrPlus{\ap{q}{s/x}}{\nu[\mu'/x]}}{q[\mu/x]}}

\\           
\id;s \equiv s \equiv s;\id \and
(s;s');s'' \equiv s;(s';s'') \\ 
\ap \nu {\id_{\mu}/x} \equiv \id_{\nu[\mu/x]} \and
\ap \nu {(s;t)/x} \equiv \ap \nu {s/x} ; (\ap {(\TrPlus{\ap{q}{s/x}}{y})} {\ap \nu {t/x}/y}) \\ 
\ap x {s/x} \equiv s  \and
\ap \nu {s/\_} \equiv \id_\nu \and
\ap {(\nu[\mu/x])} {s/y} \equiv \ap \nu {\ap \mu {s/y}/x} \quad
(\text{where } \gamma,y:p' \vdash \mu : p \text{ and } \gamma,x:p \vdash \nu : q)\\
t[\mu/x];\ap{\nu'}{s/x} \equiv \ap{\nu}{s/x};\ap{(\TrPlus{\ap{q}{s/x}}{y})}{t[\mu'/x]/y} \quad
 (\text{where } \TermTwoT{\gamma,x:p}{t}{\nu}{\nu'}{q} \text{ and } \TermTwoT{\gamma}{s}{\mu}{\mu'}{p})
\end{mathpar}

\item We assume $1/\Sigma$ modes:

\begin{mathpar}
  \inferrule*{ } { \gamma \yields 1 \type } \and
  
  \inferrule*{ \gamma \yields p \type \\ 
               \gamma,x:p \yields q \type }
             {\gamma \yields \sigmacl{x}{p}{q} \type} \\
             
  \inferrule*{ }
             {\gamma \yields \mt : 1}
  \and 
  \mu \equiv \mt
\\
\inferrule*{
  \gamma \yields \mu : p \and
  \gamma \yields \nu : q[\mu/x]
    }
   {\gamma \yields (\mu,\nu) : \sigmacl{x}{p}{q}}
\and
\inferrule*
    {\gamma \yields \mu : \sigmacl{x}{p}{q}}
    {\gamma \yields \fst \mu : p}
\and
\inferrule*
    {\gamma \yields \mu : \sigmacl{x}{p}{q}}
    {\gamma \yields \snd \mu : q[\fst \mu / x]}
    \\
    \fst{(\mu,\nu)} \equiv \mu \and
    \snd{(\mu,\nu)} \equiv \nu \and
    p \equiv (\fst p, \snd p)
\end{mathpar}

Equations for ``transport'' in $\Sigma$:
\begin{mathpar}
\TrPlus{(\sigmacl{x}{s}{t})}{\mu} \equiv (\TrPlus{s}{\fst \mu},\TrPlus{(t[\fst \mu/x])}{\snd \mu})
\end{mathpar}

Mode type morphisms: We need congruence for $\Sigma$ to be a rule (because we don't have ap on a type variable/universes):
\begin{mathpar}
  \inferrule*
  {\TypeTwo{\gamma}{s}{p}{p'} \\
    \TypeTwo{\gamma,x':p'}{t}{q[\TrPlus{s}{x'}/x]}{q'}}
  {\TypeTwo{\gamma}{\sigmacl{x'}{s}{t}}{\sigmacl{x}{p}{q}}{\sigmacl{x'}{p'}{q'}}} \\

  \sigmacl{x'}{\id_p}{\id_q} \equiv \id_{\sigmacl{x'}{p}{q}} \and
  (\sigmacl{x'}{s}{t});(\sigmacl{x''}{s'}{t'}) \equiv \sigmacl{x''}{(s;s')}{(t[\TrPlus{s'}{x''}/x'];t')} \\

  \ap{(\sigmacl{x'}{p}{q})}{s/(y:r)} \equiv
  \sigmacl{x'}{\ap{p}{s/y}}{\ap{({q[\fst z/x,\snd z/y]})}{\extend{s}{x'}/(z:(\sigmacl{y}{r}{p}))}}
\end{mathpar}

Finally, we have the 2-cells for $1/\Sigma$-terms:
\begin{mathpar}
s \equiv \id_{()} \text{ for } s : () \vDash_1 ()
\\

\inferrule*
    {\TermTwoT{\gamma}{s}{\mu}{\mu'}{p} \and
      \gamma \vdash \nu' : q[\mu'/x]
    }
      {\TermTwoT{\gamma}{\extend{s}{\nu'}}{(\mu,\TrPlus{\ap{q}{s/x}}{\nu'})}{(\mu',\nu')}{\sigmacl{x}{p}{q}}}\\
\ap {\fst(z)} {\extend{s}{\nu'}/z} \equiv s \and
\ap {\snd(z)} {\extend{s}{\nu'}/z} \equiv \id_{\TrPlus{\ap{q}{s/x}}{\nu'}}  \\
s \equiv \ap{(\fst{\mu},y)}{\ap{(\snd z)}{s/z}/y};\extend{\ap{(\fst{z})}{s/z}}{\snd{\mu'}} \quad (\text{where } \TermTwoT{\gamma}{s}{\mu}{\mu'}{\sigmacl{x}{p}{q}})
\\      
{\extend{\id_\mu}{\nu'}} \equiv \id_{(\mu,\nu')} \and
{\extend{(s;s')}{\nu''}} \equiv  \extend{s}{\TrPlus{\ap{q}{s'/x}}{\nu''}};\extend{s'}{\nu''}   \\
\extend{s}{\nu'} ; (\ap{(\mu',y)}{t/y}) \equiv
(\ap{(\mu,\TrPlus{(\ap{q}{s})}{y})}{t/y}); \extend{s}{\nu''} \qquad (\text{where }\TermTwoT{\gamma}{t}{\nu'}{\nu''}{q[\mu'/x]})
\end{mathpar}

\item
  All judgements have a substitution principle
\begin{mathpar}
  \inferrule*{\gamma,x:p,\gamma' \yields J \\
              \gamma \yields \mu : p
              }
             {\gamma,\gamma'[\mu/x] \yields J[\mu/x]} \\

J[\mu/x][\nu/y] \equiv J[\nu/y][\mu[\nu/y]/x]
\end{mathpar}

\drlnote{write out the usual rules defining this}
             
\end{enumerate}

We sometimes write \ap{\mu}{s} for \ap{\mu(x)}{s/x}, eliding the
variable name when it is clear how to view $\mu$ as a term with a
distinguished variable; e.g. $\ApPlus{s}{t}$ for
$\ap{\TrPlus{s}{x}}{t/x}$.

\subsubsection{Lemmas}

Horizontal composition:
\begin{mathpar}
  \inferrule*[Left=Derivable]
      {\TermTwoT{\gamma}{s}{\mu}{\mu'}{p} \\
    \TermTwoT{\gamma, x : p}{t}{\nu}{\nu'}{q}}
             {\TermTwoT{\gamma}{\ap{t}{s/x} :\equiv t[\mu/x];\ap{\nu'}{s/x}}{\nu[\mu/x]}{\TrPlus{\ap{q}{s/x}}{\nu'[\mu'/x]}}{q[\mu/x]}}
\\ 
\ap{\id_\nu}{s/x} \equiv \ap{\nu}{s/x} \and \ap{t}{\id_{\mu}/x} \equiv t[\mu/x]
\end{mathpar}

Pairing and projection 2-cells are definable:
\begin{mathpar}
  \inferrule*[Left=Derivable]
      {\TermTwoT{\gamma}{s}{\mu}{\mu'}{p} \\
    \TermTwoT{\gamma}{t}{\nu}{\TrPlus{\ap{q}{s}}{\nu'}}{q[\mu/x]}}
             {\TermTwoT{\gamma}{(s,t) :\equiv \ap{(\mu,y)}{t/y};\extend{s}{\nu'}}{(\mu,\nu)}{(\mu',\nu')}{\sigmacl{x}{p}{q}}}

   \inferrule*[Left=Deriv]
              { {\TermTwoT{\gamma}{s}{\mu}{\mu'}{\sigmacl{x}{p}{q}}} }
              { {\TermTwoT{\gamma}{\ap{\fst(y)}{s/y}}{\fst{\mu}}{\fst{\mu'}}{p}} }
   \and
   \inferrule*[Left=Deriv]
              { {\TermTwoT{\gamma}{s}{\mu}{\mu'}{\sigmacl{x}{p}{q}}} }
              { {\TermTwoT{\gamma}{\ap{\snd(y)}{s/y}}{\snd{\mu}}{\TrPlus{\ap{(q(\fst y/x))}{s/y}}{\snd{\mu'}}}{q[\fst{\mu}/x]}} }
\end{mathpar}

\drlnote{Check usual composition rule for pairing}



\subsection{Contexts}

\begin{mathpar}
  \inferrule*[Left = ctx-form]{ }
  {\yields_{\cdot} \cdot \CTX  } \and 

  \inferrule*[Left = ctx-form]{
    \yields_\gamma \Gamma \CTX \and (\text{where } \yields \gamma \ctx) \\\\
    \Gamma \yields_p A \TYPE \and (\text{where }  \gamma \yields p \type)}
  {\yields_{\gamma, x : p} \Gamma, x : A \CTX \and (\text{where } \yields \gamma,x:p \ctx)  } \\
\end{mathpar}

\subsection{Types and Terms}

\subsubsection{Structural Rules}

\begin{mathpar}
  \inferrule*[Left = var]{
    % \yields \Gamma, x : A, \Gamma' \CTX_{\gamma, x : p, \gamma'}
  }
  {\Gamma, x : A, \Gamma' \yields_x x : A \and (\text{where } \gamma,x:p,\gamma' \yields x : p)} \and

 \inferrule*[Left = rewrite]{
   \Gamma \yields_\mu M : A 
   \and \TermTwoT{\gamma}{s}{\nu}{\mu}{p}
  }
  {\Gamma \yields_\nu \rewrite{s}{M} : A} \\ \\
  
  \rewrite{\id_{\mu}}{M} \equiv M \and
  \rewrite{(s;t)}{M} \equiv \rewrite{s}{\rewrite{t}{M}} \and
  \rewrite{s}{M}[\rewrite{t}{N}/x] \equiv \rewrite{\ap{s}{t/x}}{\StI{\ap{q}{t/x}}{M[N/x]}}
\end{mathpar}

\subsubsection{Telescope Types}

\begin{mathpar}
  \inferrule*{~}{\Gamma \yields_{1} 1 \TYPE} \and
  \inferrule*{~}{\Gamma \yields_{()} () : 1} \\
  M \equiv () \\
  \inferrule*{ \Gamma \yields_p A \TYPE \\
               \Gamma,x:A \yields_q B \TYPE}
             { \Gamma \yields_{\sigmacl{x}{p}{q}} \telety{x}{A}{B} \TYPE}
  \\
  \inferrule*{ \Gamma \yields_\mu M : A \\
               \Gamma \yields_\nu N : B[M/x]
             }
             { \Gamma \yields_{(\mu,\nu)} (M,N) : \telety{x}{A}{B}}
  \and
  \inferrule*{ \Gamma \yields_{\mu} M : \telety{x}{A}{B}}
             { \Gamma \yields_{\fst \mu} \fst{M} : A} 
  \and
  \inferrule*{ \Gamma \yields_{\mu} M : \telety{x}{A}{B}}
             { \Gamma \yields_{\snd \mu} \snd{M} : B[\fst M/x]} 

    \fst{(M,N)} \equiv M \and
    \snd{(M,N)} \equiv N \and
    P \equiv (\fst P, \snd P)
\end{mathpar}


\subsubsection{Modalities}

\begin{mathpar}
  \inferrule*[Left = F-form]{
    %% \yields_\gamma \Gamma \CTX \and (\text{where } \yields \gamma \ctx)\\\\
    \Gamma \yields_p A \TYPE \and (\text{where } \gamma \yields p \type) \\\\
    \gamma, x:p \yields \mu : q 
  }
  {\Gamma \yields_q \F{x.\mu}{A} \TYPE \and (\text{where } \gamma \yields q \type) } \\
  
  \inferrule*[Left = F-intro]{
    \Gamma \yields_{\nu} M : A
    \and (\text{where } \gamma \yields {\nu} : p)
    %% \and \gamma \yields \nu : q 
    %% \and \gamma \yields \mu[\theta] : q 
    %% \and \gamma \yields (\nu \Rightarrow \mu[\theta]) : q
  }
  {\Gamma \yields_{\mu[\nu/x]} \FI{M} : \F{x.\mu}{A} \and (\text{where } \gamma \yields \mu[\nu/x] : q)} \\

  \inferrule*[Left = F-elim]{
    \Gamma, y : \F{x.\mu}{A} \yields_{r} C \TYPE \and (\text{where } \gamma, y : q \yields r \type) \\\\
    \Gamma \yields_{\nu} M : \F{x.\mu}{A} \and (\text{where } \gamma \yields \nu : q) \\\\
    \Gamma, x:A \yields_{\nu' [\mu / y]} N : C [\FI{x}/y]
    \and (\text{where } \gamma, x:p \yields \nu' [\mu / y] : r [\mu / y] )}
  {\Gamma \yields_{\nu'[\nu/y]} \FE{M}{x}{N} : C[M/y]  \and (\text{where }  \gamma \yields {\nu'[\nu/y]} : r[\nu/y])} \\
  \FE{\FI{M}}{x}{N} \equiv N[M/x] %\and
%  \text{(optionally:) }
%  \FE{M}{x}{N[\FI{x}/z]} \equiv N[M/z]
  \\ \\

  \inferrule*[Left = F-Elim]{
    \gamma,y:q \yields r \type \\\\
    \Gamma \yields_{\nu} M : \F{x.\mu}{A} \and (\text{where } \gamma \yields \nu : q) \\\\
    \Gamma, x:A \yields_{r [\mu / y]} C \TYPE
    \and (\text{where } \gamma, x:p \yields r [\mu / y] \type )}
  {\Gamma \yields_{r[\nu/y]} \FE{M}{x}{C} \TYPE \and (\text{where }  \gamma \yields {r[\nu/y]} \type)} \\
  \FE{\FI{M}}{x}{C} \equiv C[M/x] %\and
%  \text{(optionally:) }
%  \FE{M}{x}{C[\FI{x}/z]} \equiv C[M/z]
\\ \\
  \inferrule*[Left = U-form]{
    \Gamma \yields_p A \TYPE \and (\text{where } \gamma \yields p \type)\\\\
    \and \Gamma,x:A \yields_q B \TYPE \and (\text{where } \gamma,x:p \yields q \type)\\\\
    \and \gamma, x:p, c:r \yields \mu : q
  }{\Gamma \yields_r \U{c.\mu}{A}{B} \TYPE \and (\text{where } \gamma \yields r \type)} \\

  \inferrule*[Left = U-intro]{
    \Gamma,x:A \yields_{\mu[\nu/c]} M : B \and (\text{where } \gamma,x:p \yields {\mu[\nu/c]} : q)
  }
  {\Gamma \yields_{\nu} \UI {x}{M} : \U{c.\mu}{x:A}{B}
    \and (\text{where } \gamma \yields \nu : r)
  } \\
  
  \inferrule*[Left = U-elim]{
    \Gamma \yields_{\nu_1} N_1 : \U{c.\mu}{x:A}{B} \and (\text{where } \gamma \yields \nu_1 : r) \\\\
    \Gamma \yields_{\nu_2} N_2 : A \and (\text{where } \gamma \yields \nu_2 : p)
  }{
    \Gamma \yields_{\mu[\nu_2/x,\nu_1/c]} \UE{N_1}{N_2} : B[N_2/x] \and (\text{where } \gamma \yields \mu[\nu_2/x,\nu_1/c] : q)
  } \\

  \UE{(\UI{x}{M})}{N} \equiv M[N/x] \and 
  \UI{x}{\UE{N}{x}} \equiv N
\end{mathpar}

\subsubsection{Surprisingly Strict Modalities}

\begin{mathpar}
  \inferrule*[Left = s-form]{
    \Gamma \yields_p A \TYPE \and (\text{where } \gamma \yields p \type)\\\\
    \and \TypeTwo{\gamma}{s}{q}{p}
  }{\Gamma \yields_q \St{s}{A} \TYPE \and (\text{where } \gamma \yields q \type)} \\

  \inferrule*[Left = S-intro]{
    \Gamma \yields_{\mu} M : A
    \and (\text{where } \gamma \yields {\mu} : p)
  }
  {\Gamma \yields_{\TrPlus{s}{\mu}} \StI{s}{M} : \St{s}{A} \and (\text{where } \gamma \yields \TrPlus{s}{\mu} : q)} \\

  \inferrule*[Left = S-elim]{
    \Gamma, y : \St{s}{A} \yields_{r} C \TYPE \and (\text{where } \gamma, y : q \yields r \type) \and \\\\
    \Gamma \yields_{\nu} M : \St{s}{A} \and (\text{where } \gamma \yields \nu : q) \\\\
    \Gamma, x : A \yields_{\nu' [\TrPlus{s}{x} / y]} N : C [\StI{s}{x}/y]
    \and (\text{where } \gamma, x : p \yields \nu' [\TrPlus{s}{x} / y] : r[\TrPlus{s}{x} / y] )}
  {\Gamma \yields_{\nu'[\nu/y]} \StE{s}{M}{x}{N} : C[M/y]  \and (\text{where } \gamma \yields {\nu'[\nu/y]} : r[\nu/y])} \\
  \StE{s}{\StI{s}{M}}{x}{N} \equiv N[M/x] \and
  \StE{s}{M}{x}{N[\StI{s}{x}/z]} \equiv N[M/z]
  \\
  
  \inferrule*[Left = S-Elim]{
    \gamma,y:q \yields r \type \\\\
    \Gamma \yields_{\nu} M : \St{s}{A} \and (\text{where } \gamma \yields \nu : q) \\\\
    \Gamma, x:A \yields_{r [\TrPlus{s}{x} / y]} C \TYPE
    \and (\text{where } \gamma, x:p \yields r [\TrPlus{s}{x} / y] \type )}
  {\Gamma \yields_{r[\nu/y]} \StE{s}{M}{x}{C} \TYPE \and (\text{where }  \gamma \yields {r[\nu/y]} \type)} \\
  \StE{s}{\StI{s}{M}}{x}{C} \equiv C[M/x] \and
  \StE{s}{M}{x}{C[\StI{s}{x}/z]} \equiv C[M/z]
  \\ \\
\end{mathpar}

%\drlnote{Change the definition of $s$-types to $U$-types as primitive and derive $F$, so that having $\eta$ is less surprising.}

Term/type equalities:
\begin{align}
%\StI{s}{\FI{M}} &\equiv \FI{M} &\St{s}{\F{x.\mu}{A}} &\equiv \F{x.\TrPlus{s}{\mu}}{A} \\
%\FI{\StI{s}{M}} &\equiv \FI{M} &\F{x.\mu}{\St{s}{A}} &\equiv \F{x.\mu[\TrPlus{s}{x}/x]}{A} \\
%%\UStI{s}{\UI{x}{M}} &\equiv \UI{x}{M} &\St{s}{\U{c.\mu}{x:A}{B}} &\equiv \U{c.\mu[\TrCirc{s}{c}/c]}{x:A}{B} \\
%%\UI{x}{\UStI{s}{M}} &\equiv \UI{x}{M} &\U{c.\mu}{x:A}{\St{s}{B}} &\equiv \U{c.\TrCirc{s}{\mu}}{x:A}{B} \\
%\UI{x}{M} &\equiv \UI{x}{M[\StI{s}{x}/x]} &\U{c.\mu}{x:\St{s}{A}}{B} &\equiv \U{c.\mu[\TrPlus{s}{x}/x]}{x:A}{B[\StI{s}{x}/x]} \\
\StI{(s, t)}{(M, N)} &\equiv (\StI{s}{M}, \StI{t[\mu/x]}{N}) &\St{(\telety{x'}{s}{t})}{\telety{x'}{A'}{B'}} & \equiv \telety{x}{\St{s}{A'}}{\StE{s}{x}{x'}{\St{t}{B'}}} \\
\StI{s}{\StI{t}{M}} &\equiv \StI{s;t}{M} &\St{s}{\St{t}{A}} &\equiv \St{(s;t)}{A} \\
\label{eqn:stype-subst} \StI{\id_p}{M} &\equiv M &\St{\id_p}{A} &\equiv A %\\
%\rewrite{\ap{\nu}{t/x}}{\StI{\ap{q}{t/x}}{N[M/x]}} &\equiv N[\rewrite{t}{M}/x]  &\St{(\ap{q}{t/x})}{B[M/x]} & \equiv B[\rewrite{t}{M}/x] \\
%% other whiskering is a substitution rule:
%% \St{(s[\mu/x])}{B[M/x]} & \equiv (\St{s}{B})[M/x] 
\end{align}
%Where the last term equation is a special case of rewrite on substitutions. The inputs are typed $\Gamma,x:A \vdash_q B \TYPE $\ and $\Gamma \vdash_{\mu'} M : A$ and $\TermTwo{\gamma}{t}{\mu}{\mu'}$.

In the absence of eta for $\mathsf{F}$-types, we also need to assert
\begin{mathpar}
(\FE{M}{x}{\rewrite{t[\mu/y]}{N}}) \equiv \rewrite{t[\nu/y]}{\FE{M}{x}{N}}
\end{mathpar}
\mvrnote{And the one for types?}

\subsection{Lemmas}

\begin{lemma}
The \textsc{rewrite} rule commutes with pairing of telescope types:
\begin{align*}
%(\rewrite{s}{M},N) &\equiv \rewrite{(s, \varepsilon_\nu^{\ap{q}{s/x}})}{(M, \UnSt{\ap{q}{s/x}}{N}}) \\
%(M,\rewrite{t}{N}) &\equiv \rewrite{(\id_\mu, t)}{(M,N)} \\
\rewrite{(s, t)}{(M, N)} &\equiv (\rewrite{s}{M}, \rewrite{t}{\StI{\ap{q}{s/x}}{N}} ) 
\end{align*}
\end{lemma}
\begin{proof}
\begin{align*}
&\rewrite{(s, t)}{(M, N)} \\
&\equiv (\fst\rewrite{(s, t)}{(M, N)}, \snd \rewrite{(s, t)}{(M, N)} ) \\
&\equiv (\fst z [\rewrite{(s, t)}{(M, N)}/z], \snd z [\rewrite{(s, t)}{(M, N)}/z] ) \\
&\equiv (\rewrite{\ap{\fst}{(s, t)}}{\fst (M, N)}, \rewrite{\ap{\snd}{(s, t)}}{\StI{\ap{q[\fst z/x]}{(s, t)/z}}{\snd (M, N)}} ) \\
&\equiv (\rewrite{s}{M}, \rewrite{t}{\StI{\ap{q}{s/x}}{N}} )
\end{align*}

\drlnote{Add ref to equalities being used from second to third line.
  Also write out the typing for the second half of the pair, to show how
  the $s$-type moves work.}
\end{proof}

\begin{lemma}
The \textsc{rewrite} rule pushes into terms of $\mathsf{F}$-, $\mathsf{U}$- and $s$-types:
\begin{align*}
\FI{\rewrite{s}{M}} &\equiv \rewrite{\ap{\mu}{s/x}}{\FI{M}} \\
(\FE{\rewrite{s}{M}}{x}{N}) &\equiv \rewrite{\ap{\nu'}{s/y}}{\StI{\ap{r}{s/y}}{\FE{M}{x}{N}}} \\
\UE{M}{\rewrite{s}{N}} &\equiv \rewrite{\ap{\mu[\nu_1/c]}{s/x}}{\StI{\ap{q}{s/x}}{\UE{M}{N}}} \\
\UE{(\rewrite{t}{M})}{N} &\equiv \rewrite{\ap{\mu[\nu_2/x]}{t/c}}{\UE{M}{N}} \\
\UI{x}{\rewrite{\ap{\mu}{s/c}}{M}}  &\equiv\rewrite{s}{\UI{x}{M}} \\
\StI{t}{\rewrite{s}{M}} &\equiv \rewrite{\ApPlus{t}{s}}{\StI{t}{M}} \\
(\StE{t}{\rewrite{s}{M}}{x}{N}) &\equiv \rewrite{\ap{\nu'}{s/y}}{\StI{\ap{r}{s/y}}{\StE{t}{M}{x}{N}}} \\
(\StE{t}{M}{x}{\rewrite{s[\TrPlus{t}{x}/y]}{N}}) &\equiv \rewrite{s[\nu/y]}{\StE{t}{M}{x}{N}} 
%\intertext{If we have definitional eta-expansion for \textsf{F}-types, we can also derive (rather than asserting)}
%(\FE{M}{x}{\rewrite{s[\mu/y]}{N}}) &\equiv \rewrite{s[\nu/y]}{\FE{M}{x}{N}}
\end{align*}
\end{lemma}
\begin{proof}
For \textsf{F}-types:
\begin{align*}
\FI{\rewrite{s}{M}} 
&\equiv \rewrite{\id_{\mu[\nu/x]}}{\FI{x}}[\rewrite{s}{M}/x]  \\
&\equiv \rewrite{\ap{\mu}{s/x}}{\StI{\ap{q}{s/x}}{\FI{x}[M/x]}} \\
&\equiv \rewrite{\ap{\mu}{s/x}}{\StI{\ap{q}{s/x}}{\FI{M}}} \\
&\equiv \rewrite{\ap{\mu}{s/x}}{\FI{M}} && \text{As $x$ does not appear in $q$.}\\
(\FE{\rewrite{s}{M}}{x}{N})
&\equiv \rewrite{\id_{\nu'[\nu/y]}}{\FE{y}{x}{N}}[\rewrite{s}{M}/y] \\
&\equiv \rewrite{\ap{\nu'}{s/y}}{\StI{\ap{r}{s/y}}{(\FE{y}{x}{N})[M/y]}} \\
&\equiv \rewrite{\ap{\nu'}{s/y}}{\StI{\ap{r}{s/y}}{\FE{M}{x}{N}}}
\end{align*}

For \textsf{U}-types:
\begin{align*}
\UE{M}{\rewrite{s}{N}}
&\equiv \rewrite{\id_{\mu[z/x, \nu_1/c]}}{\UE{M}{z}}[\rewrite{s}{N}/z] \\
&\equiv \rewrite{\ap{\mu[\nu_1/c]}{s/x}}{\StI{\ap{q[\nu_1/c]}{s/x}}{\UE{M}{N}}} \\
&\equiv \rewrite{\ap{\mu[\nu_1/c]}{s/x}}{\StI{\ap{q}{s/x}}{\UE{M}{N}}} \\
\UE{(\rewrite{t}{M})}{N}
&\equiv \rewrite{\id_{\mu[\nu_2/x, z/c]}}{\UE{(z)}{N}}[\rewrite{t}{M}/z] \\
&\equiv \rewrite{\ap{\mu[\nu_2/x]}{t/c}}{\StI{\ap{q[\nu_2/x]}{t/c}}{\UE{M}{N}}} \\
&\equiv \rewrite{\ap{\mu[\nu_2/x]}{t/c}}{\UE{M}{N}} && \text{(As $c$ does not appear in $q$)}\\
\rewrite{s}{\UI{x}{M}}
&\equiv \UI{y}{\UE{(\rewrite{s}{\UI{x}{M}})}{y}} \\
&\equiv \UI{y}{\rewrite{\ap{\mu[y/x]}{s/c}}{\UE{(\UI{x}{M})}{y}}} \\
&\equiv \UI{y}{\rewrite{\ap{\mu[y/x]}{s/c}}{M[y/x]}} \\
&\equiv \UI{x}{\rewrite{\ap{\mu}{s/c}}{M}}
\end{align*}

The first two equations for $s$-type follow by the same reasoning as for $\mathsf{F}$-types. For the third, we make use of the eta-expansion that is not available for $\mathsf{F}$-types.
\begin{align*}
\rewrite{s[\nu/y]}{\StE{t}{M}{x}{N}}
&\equiv \rewrite{s}{\StE{t}{y}{x}{N}} [M/y] \\
&\equiv \StE{t}{M}{z}{(\rewrite{s}{\StE{t}{y}{x}{N}}[\StI{t}{z}/y])} \\
&\equiv \StE{t}{M}{z}{(\rewrite{s[\TrPlus{t}{z}/y]}{\StE{t}{\StI{t}{z}}{x}{N}})} \\
&\equiv \StE{t}{M}{z}{\rewrite{s[\TrPlus{t}{z}/y]}{N}}
\end{align*}
\end{proof}

\begin{lemma}
The \textsc{rewrite} rule pushes into types obtained by \textsf{F}- and $s$-elimination:
\begin{align*}
(\FE{\rewrite{s}{M}}{x}{C}) &\equiv \St{\ap{r}{s/y}}{\FE{M}{x}{C}} \\
(\StE{t}{\rewrite{s}{M}}{x}{C}) &\equiv \St{\ap{r}{s/y}}{\StE{t}{M}{x}{C}}
\end{align*}
\end{lemma}
\begin{proof}
This is analogous to elimination into terms:
\begin{align*}
\FE{\rewrite{s}{M}}{x}{C}
&\equiv (\FE{y}{x}{C})[\rewrite{s}{M}/y] \\
&\equiv \St{\ap{r}{s/y}}{(\FE{y}{x}{C})[M/y]} \\
&\equiv \St{\ap{r}{s/y}}{\FE{M}{x}{C}}
\end{align*}
and similarly for $s$-types.
\end{proof}

\begin{lemma}
\textsc{s-elim} commutes with \textsc{F-elim} and \textsc{s-elim}:
\begin{align*}
\FE{(\StE{s}{M}{x'}{N'})}{x}{N} &\equiv \StE{s}{M}{x'}{(\FE{N'}{x}{N})} \\
\StE{t}{(\StE{s}{M}{x'}{N'})}{x}{N} &\equiv \StE{s}{M}{x'}{(\StE{t}{N'}{x}{N})}
\end{align*}
\end{lemma}
\begin{proof}
Using eta for $s$-types:
\begin{align*}
&\FE{(\StE{s}{M}{x'}{N'})}{x}{N} \\
&\equiv (\FE{(\StE{s}{z}{x'}{N'})}{x}{N})[M/z] \\
&\equiv \StE{s}{M}{z'}{((\FE{(\StE{s}{z}{x'}{N'})}{x}{N})[\StI{s}{z'}/z])} \\
&\equiv \StE{s}{M}{z'}{((\FE{(\StE{s}{\StI{s}{z'}}{x'}{N'})}{x}{N}))} \\
&\equiv \StE{s}{M}{z'}{(\FE{N'[z'/x']}{x}{N})} \\
&\equiv \StE{s}{M}{x'}{(\FE{N'}{x}{N})}
\end{align*}
and similarly for the second.
\end{proof}

%\begin{lemma}
%\textsc{F-elim} fuses with \textsc{s-intro}.
%\end{lemma}
%\begin{proof}
%\begin{align*}
%&\FEs{\TrPlus{s}{\mu}}{\StI{s}{M}}{x}{N} \\
%&\equiv \StE{s}{\StI{s}{M}}{y}{(\FEs{\mu}{y}{x}{N})} \\
%&\equiv \FEs{\mu}{M}{x}{N}
%\end{align*}
%and
%\begin{align*}
%& \FEs{\mu[\TrPlus{s}{x}/x]}{M}{x}{N[\StI{s}{x}/x]} \\
%&\equiv \FEs{\mu}{M}{y}{(\StE{s}{y}{x}{N[\StI{s}{x}/x]})}  \\
%&\equiv \FEs{\mu}{M}{y}{N[y/x]} \\
%&\equiv \FEs{\mu}{M}{x}{N}
%\end{align*}
%\end{proof}

\begin{lemma}
Any context $\Gamma$ can be tupled into a $\Sigma$ type $\Sigma \Gamma$, so that substitutions $\Gamma \yields \Theta : \Delta$ correspond bijectively to terms $\Gamma \yields \Sigma \Theta : \Sigma \Delta$. \mvrnote{Generalising to telescopes seems doable} 
\mvrnote{I have been pretty inconsistent with the notation for telescope types through this whole document}
\begin{mathpar}
\inferrule*{\yields_\gamma \Gamma}
             {\cdot \yields_{\Sigma \gamma} \Sigma \Gamma \TYPE}
\and
\inferrule*{~}
             {\sigma : \Sigma \Gamma \yields_{\fan{\gamma}} \fan{\Gamma} : \Gamma}
\and
\inferrule*{\Gamma \yields_\theta \Theta : \Delta}
             {\Gamma \yields_{\Sigma \theta} \Sigma \Theta : \Sigma \Delta}
\\
\inferrule*{\Gamma \yields\Theta : \Delta}
             {\Gamma \yields \Theta \equiv \fan{\Delta}[\Sigma \Theta / \sigma]}
\and
\inferrule*{\Gamma \yields a : \Sigma\Delta}
             {\Gamma \yields a \equiv \Sigma(\fan{\Delta}[a/\sigma])}
\end{mathpar}
\end{lemma}
\begin{proof}
$\Sigma \Gamma$ is defined inductively by
\begin{align*}
\Sigma (\cdot) &:\equiv 1 \\
\Sigma (\Gamma, x : A) &:\equiv \sum_{\sigma : \Sigma \Gamma} A[\fan{\Gamma}] \\
\fan{(\cdot)} &:\equiv \cdot \\
\fan{\Gamma, x : A} &:\equiv (\fan\Gamma[\fst{\sigma}/\sigma]), \snd{\sigma}
\end{align*}
with each definition lying over the identical one downstairs. For $\Sigma \Theta$ we simultaneously verify the equation $\Theta \equiv \fan{\Delta}[\Sigma \Theta / \sigma]$, as we need it to hold for $(\Sigma \Theta, M)$ to be well typed.
\begin{align*}
\Sigma(\cdot) &:\equiv \mt : 1\\
\Sigma(\Theta, M : A) &:\equiv (\Sigma \Theta, M) : \sum_{\sigma : \Sigma \Delta} A[\fan{\Delta}] \\
(\cdot)[()/\sigma]  &\equiv (\cdot) \\
\fan{\Delta, x : A}[\Sigma(\Theta, M : A)/\sigma] &\equiv ((\fan{\Delta}[\fst{\sigma}/\sigma]), \snd{\sigma})[(\Sigma \Theta, M)/\sigma] \\
&\equiv (\fan{\Delta}[\Sigma \Theta/\sigma]), M \\
&\equiv \Theta, M
\end{align*}
For the second equation, we check inductively that
\begin{align*}
\Sigma(\fan{(\cdot)}[a/\sigma]) 
&\equiv \Sigma(\cdot[a/\sigma]) \\
&\equiv \Sigma(\cdot) \\
&\equiv () \\
&\equiv a \\
\Sigma(\fan{\Delta, x : A}[a/\sigma]) 
&\equiv \Sigma((\fan{\Delta}[\fst \sigma/\sigma], \snd \sigma)[a/\sigma]) \\
&\equiv \Sigma(\fan{\Delta}[\fst a/\sigma], \snd a) \\
&\equiv \Sigma(\fan{\Delta}[\fst a/\sigma]), \snd a \\
&\equiv \fst a, \snd a \\
&\equiv a
\end{align*}
\end{proof}

\begin{lemma}
Tupling respects substitution: for $\gamma \yields \theta : \delta$ and $\delta \yields \kappa : \lambda$, we have:
\begin{mathpar}
\Sigma(\kappa[\theta]) \equiv (\Sigma \kappa)[\theta]
\end{mathpar}
\end{lemma}
\begin{proof}
By induction on the length of $\lambda$:
\begin{align*}
\Sigma((\cdot)[\theta]) 
&\equiv \Sigma(\cdot) \\
&\equiv (\Sigma (\cdot))[\theta] \\
\Sigma((\kappa, M)[\theta])
&\equiv \Sigma(\kappa[\theta], M[\theta]) \\
&\equiv \Sigma(\kappa[\theta]), M[\theta] \\
&\equiv (\Sigma\kappa)[\theta], M[\theta] \\
&\equiv (\Sigma\kappa, M)[\theta]
\end{align*}
\end{proof}

\begin{definition}
A 2-cell between mode substitutions $\gamma\yields t : (\theta \vDash \theta') : \delta$ is given by a mode term 2-cell \[\gamma \yields \Sigma t : (\Sigma \theta \vDash_{\Sigma \delta} \Sigma \theta')\].
\end{definition}

\drlnote{Maybe write $\gamma\yields t : \theta \vDash_\delta \theta'$ ?
In the rules, we're writing things like $\gamma \mid \mu \vDash s :
\mu'$; we should be consistent about that versus this notation for where
the $s$ goes, and change one or the other.  I find things like $\Gamma
\vdash s : \mu \vDash_p \nu$ with two turnstiles hard to parse.
}

\begin{lemma}\label{lem:n-ary-ap-rewrite}
N-ary ap and rewrite are admissible:
\begin{mathpar}
\inferrule*{\delta \vdash {q} \type \\
            \gamma \yields t : (\theta \vDash \theta') : \delta
           } 
           {\TypeTwo{\gamma}{\ap {q} {t}}{q[\theta]}{q[\theta']}}
\and
\inferrule*{\delta \yields {\nu} : {q} \\
            \gamma\yields t : (\theta \vDash \theta') : \delta
           } 
           {\TermTwoT{\gamma}{\ap \nu {t}}{\nu[\theta]}{\TrPlus{\ap{q}{t}}{\nu[\theta]}}{q[\theta]}} \\
 \inferrule*[Left = rewrite]{
   \Gamma \yields_{\theta'} \Theta : \Delta \and 
   \gamma \yields t : (\theta \vDash \theta') : \delta
  }
  {\Gamma \yields_{\theta} \rewrite{t}{\Theta} : \Delta}
\end{mathpar}
\end{lemma}
\begin{proof}
These are
\begin{align*}
\ap {q} {t} &:\equiv \ap{q[\fan{\delta}]}{\Sigma t/\sigma} \\
\ap {\nu} {t} &:\equiv \ap{\nu[\fan{\delta}]}{\Sigma t/\sigma} \\
\rewrite{t}{\Theta} &:\equiv \fan{\Delta}[\rewrite{\Sigma t}{\Sigma \Theta}/\sigma]
\end{align*}
which are well-typed by the equation $\theta \equiv \fan{\delta}[\Sigma \theta / \sigma]$.
\end{proof}

In particular, we have the following rules for building and using 2-cells between substitutions.
\begin{mathpar}
\inferrule{\gamma \yields t : \theta \vDash \theta' : \delta \and \gamma \yields s : (\mu[\theta] \vDash \TrPlus{\ap{p}{t}}{\mu'[\theta']})}
{ \gamma \yields (s, t) : (\theta, \mu) \vDash (\theta', \mu') : (\delta,x:p)} \\
%
\inferrule{\gamma \yields t : ((\theta, \mu) \vDash (\theta', \mu')) : (\delta, x : p)}
{\gamma \yields \ap{\fst}{t} : \theta \vDash \theta' } \and
%
\inferrule{\gamma \yields t : ((\theta, \mu) \vDash (\theta', \mu')) : (\delta, x : p)}
{\gamma \yields \ap{\snd}{t} : (\mu[\theta] \vDash \TrPlus{\ap{p}{\ap{\fst}{t}}}{\mu'[\theta']})}
\end{mathpar}%

\begin{lemma}
N-ary ap of a 2-cell on a substitution is admissible
\begin{mathpar}
\inferrule{\gamma \yields t : \theta \vDash \theta' : \delta \and \delta \yields \kappa : \lambda}
{\gamma \yields \ap{\kappa}{t} : \kappa[\theta] \vDash \kappa[\theta'] : \lambda}
\end{mathpar} 
\end{lemma}
\begin{proof}
Due to the equation $\Sigma(\kappa[\theta]) \equiv (\Sigma \kappa)[\theta]$ we can just define $\Sigma(\ap{\kappa}{t}) :\equiv \ap{(\Sigma \kappa)}{t}$.
\end{proof}

\begin{lemma}
N-ary associativity/interchange/etc. holds
\begin{align*}
\ap {(\nu[\theta])} {t} &\equiv \ap \nu {\ap \theta {t}} \\
s[\theta];\ap{\nu'}{t} &\equiv \ap{\nu}{t};\ap{(\TrPlus{\ap{q}{t}}{y})}{s[\theta']/y}
\end{align*}
\end{lemma}
\begin{proof}
\mvrnote{todo}
\end{proof}

\begin{lemma}
A N-ary versions of the equations concerning rewrites hold:
\begin{align*}
\St{(\ap{q}{t})}{B[\Theta]} &\equiv B[\rewrite{t}{\Theta}] \\
\rewrite{\ap{\nu}{t}}{\StI{\ap{q}{t}}{N[\Theta]}} &\equiv N[\rewrite{t}{\Theta}] \\
\rewrite{\ap{\kappa}{t}}{\Theta;\kappa} &\equiv \rewrite{t}{\Theta};\kappa
\end{align*}
\end{lemma}
\begin{proof}
\begin{align*}
B[\rewrite{t}{\Theta}] 
&\equiv B[\fan{\Delta}[\rewrite{\Sigma t}{\Sigma \Theta}/\sigma]] \\
&\equiv B[\fan{\Delta}][\rewrite{\Sigma t}{\Sigma \Theta}/\sigma] \\
&\equiv \St{\ap{q[\fan{\delta}]}{\Sigma t / \sigma}}{B[\fan{\Delta}][\Sigma \Theta/\sigma]} \\
&\equiv \St{\ap{q}{t}}{B[\Theta]}
\end{align*}
And:
\begin{align*}
N[\rewrite{t}{\Theta}]
&\equiv N[\fan{\Delta}[\rewrite{\Sigma t}{\Sigma \Theta}/\sigma]] \\
&\equiv N[\fan{\Delta}][\rewrite{\Sigma t}{\Sigma \Theta}/\sigma] \\
&\equiv \rewrite{\ap{\nu[\fan{\delta}]}{\Sigma t/\sigma}}{\StI{\ap{q[\fan{\delta}]}{\Sigma t/\sigma}}{N[\fan{\Delta}][\Sigma \Theta/\sigma]}} \\
&\equiv \rewrite{\ap{\nu}{t}}{\StI{\ap{q}{t}}{N[\Theta]}} \\
\end{align*}
And:
\begin{align*}
\rewrite{\ap{\kappa}{t}}{\Theta;\kappa}
&\equiv \fan{\Lambda}[\rewrite{\Sigma(\ap{\kappa}{t})}{\Sigma (\Theta;\kappa)}/\sigma] \\
&\equiv \fan{\Lambda}[\rewrite{\ap{(\Sigma \kappa)}{t}}{(\Sigma \kappa)[\Theta]}/\sigma] \\
&\equiv \fan{\Lambda}[(\Sigma \kappa)[\rewrite{t}{\Theta}]/\sigma] \\
&\equiv \fan{\Lambda}[(\Sigma \kappa)/\sigma][\rewrite{t}{\Theta}] \\
&\equiv \kappa[\rewrite{t}{\Theta}] \\
&\equiv \rewrite{t}{\Theta};\kappa
\end{align*}
using the previous equation.
\end{proof}

\begin{lemma}
Rewriting by a tuple is a tuple of rewritings:
\begin{align*}
\rewrite{(t, s/x)}{\Theta, M/x} \equiv (\rewrite{t}{\Theta}, \rewrite{s}{\StI{\ap{p}{t}}{M}})
\end{align*}
\end{lemma}
\begin{proof}
Unwinding definitions:
\begin{align*}
\rewrite{(t, s/x)}{\Theta, M/x}
&\equiv \fan{\Delta, x : A}[\rewrite{\Sigma (t, s/x)}{\Sigma (\Theta, M/x)}/\sigma] \\
&\equiv (\fan\Delta[\fst{\sigma}/\sigma], \snd{\sigma})[\rewrite{(\Sigma t, s)}{\Sigma\Theta, M}/\sigma] \\
&\equiv (\fan\Delta[\fst{\sigma}/\sigma], \snd{\sigma})[(\rewrite{\Sigma t}{\Sigma\Theta}, \rewrite{s}{\StI{\ap{p[\fan{\delta}]}{\Sigma t/\sigma}}{N}})/\sigma] \\
&\equiv (\fan\Delta[\rewrite{\Sigma t}{\Sigma\Theta}/\sigma], \rewrite{s}{\StI{\ap{p[\fan{\delta}]}{\Sigma t/\sigma}}{N}}) \\
&\equiv (\rewrite{t}{\Theta}, \rewrite{s}{\StI{\ap{p}{t}}{M}})
\end{align*}
\end{proof}

As a special case, note that we have
\begin{align*}
\rewrite{(\id_\Theta, s/x)}{\Theta, M/x} \equiv (\Theta, \rewrite{s}{M}/x)
\end{align*}

\subsection{Examples}

\begin{itemize}
\item 
The previous two-argument \F{\mu}{x:A,y:B} (for $x :p, y:q \vdash \mu :
r$) is now \F{z.\mu[\fst z/x,\snd z/y]}{\telety{x}{A}{B}}, using the
upstairs $\Sigma$-type ${\telety{x}{A}{B}}$, which has mode
$\sigmacl{x}{p}{q}$.  Iterating $\telety{x}{A}{B}$ plays the role of a
longer telescope.  
\end{itemize}

\section{Mode Theories}

\mvrnote{At some point near the start we should say:
For clarity we will often treat mode terms with a distinguished parameter as though they were functions. So if $\gamma, x : p \yields f : q$ we write $f(m)$ rather than $f[m/x]$.
}

\subsection{Adjoint Mode Terms}
\begin{definition}
A mode term $\gamma, y : q \yields u : p$ has a \emph{left adjoint} if there are specified constants:
\begin{align*}
\gamma, x : p &\yields f : q \\
\gamma, x : p &\yields (\eta_x : x \vDash_p u(f(x))) \\
\gamma, y : q &\yields (\varepsilon_y : f(u(y)) \vDash_q y)
\end{align*}
such that $\eta$ and $\varepsilon$ are natural:
\begin{align}
\eta_x ; \ap{u}{\ap{f}{s}} &\equiv s ; \eta_{x'} && \text{for } s : x \vDash_p	 x' \\
\ap{f}{\ap{u}{t}} ; \varepsilon_{y'}  &\equiv \varepsilon_y ; t && \text{for } t : y \vDash_q y'
\end{align}
and the triangle identities hold:
\begin{align}
\eta_{u(\nu)};\ap{u}{\varepsilon_\nu} &\equiv \id_{u(\nu)} \\
\ap{f}{\eta_\mu};\varepsilon_{f(\mu)} &\equiv \id_{f(\mu)}
\end{align}
\end{definition}

\subsection{Terminal Objects}
\newcommand\One{\ensuremath{\mathds{1}}}

\begin{definition}
For $p \type$, a \emph{terminal object} of $p$ is specified by
\begin{mathpar}
\vdash \emptyset : p \and
\TermTwoT{x : p}{!_x}{x}{\emptyset}{p}
\end{mathpar}
such that $t \equiv \bang_x$ for all $\TermTwoT{\gamma}{t}{\mu}{\emptyset}{p}$.
\end{definition}

\begin{definition}
For a dependent mode $x : p \vdash S(x) \type$, a \emph{fibred terminal object} term is specified by:
\begin{mathpar}
x : p \vdash \One_x : S(x) \and
\TermTwoT{x : p, \mu : S(x)}{!_\mu}{\mu}{\One_x}{S(x)}
\end{mathpar}
such that
\begin{align}
\label{bang-unique}
t & \equiv \bang_\mu && \text{where } \TermTwoT{\gamma}{t}{\mu}{\One_\alpha}{S(\alpha)}\\
\label{s-plus-one-strict}
\TrPlus{\ap{S}{s}}{\One_\alpha} &\equiv \One_\beta && \text{where } \TermTwoT{\gamma}{s}{\beta}{\alpha}{p}
\end{align}
\end{definition}

\subsection{Comprehension Object}
\newcommand\var[1]{\ensuremath{\mathtt{var}_{#1}}}
\newcommand\ApOne[1]{\ensuremath{\One_{\langle {#1} \rangle }}}

\begin{definition}[Comprehension Object]\label{def:comprehension-object}
  A \emph{comprehension object} is specified by the following
  constants:
  \begin{mathpar}
    p \type \and \alpha : p \yields \El{p}{\alpha} \type \and \alpha : p \yields \One_\alpha : \El{p}{\alpha}
    \\ 
    \TypeTwo{\cdot}{\chi}{p}{\sigmacl{\alpha}{p}{\El{p}{\alpha}}} \and
    \TypeTwo{\cdot}{\psi}{p}{1}
  \end{mathpar}
  such that
\begin{align*}
\alpha : p &\yields (\alpha, \One_\alpha) : \sigmacl{\alpha}{p}{\El{p}{\alpha}} \\
\alpha : p &\yields () : 1
\end{align*}
are left adjoints to $\chi^+$ and $\psi^+$ respectively.
\end{definition}

In particular, a comprehension object comes equipped with natural mode term morphisms
\begin{align*}
\eta^\chi_\alpha {}&: \alpha \vDash_p \TrPlus{\chi}{(\alpha, \One_\alpha)} \\
\varepsilon^\chi_{(\alpha, \mu)} {}&: (\TrPlus{\chi}{(\alpha, \mu)}, \One_{\TrPlus{\chi}{(\alpha, \mu)}}) \yields_{\sigmacl{\alpha}{p}{\El{p}{\alpha}}} (\alpha, \mu) \\
\eta^\psi_\alpha {}&: \alpha \vDash_p \TrPlus{\psi}{()} \\
\varepsilon^\psi_{x} {}&: x \vDash_1 ()
\end{align*}
satisfying the triangle identities. (Note that $\varepsilon^\psi_x$ is necessarily the unique morphism to $()$)

\begin{definition}
Any comprehension object supports the following derived forms (where $\gamma \yields \alpha : p$
and $\gamma \yields \beta : p$ and
$\TermTwoT{\gamma}{s}{\beta}{\alpha}{p}$ and $\gamma \yields \mu :
\alpha$ and $\gamma \yields \nu : \beta$):
  \begin{itemize}
  \item A distinguished term $\emptyset : p$ is given by
  \begin{mathpar}
  \cdot \yields \emptyset : p \and \emptyset :\equiv \TrPlus{\psi}{()}
  \end{mathpar}
  \item Comprehension is given by $\chi^+$:
  \begin{mathpar}
  \alpha : p, x : \El{p}{\alpha} \yields \alpha.x : p \and \alpha.x :\equiv \TrPlus{\chi}{(\alpha, x)}
  \end{mathpar}
  \item $\pi^\alpha_\mu$ and $\var{\mu}$ are defined via the counit of the adjunction $\chi^\circ \dashv \chi^+$.
  \begin{mathpar}
  {\TermTwoT{\alpha:p,x:\El{p}{\alpha}}{\pi^\alpha_x}{\alpha.x}{\alpha}{p}}
  \and
  \pi^\alpha_x :\equiv \ap \fst {\varepsilon^\chi_{(\alpha, x)}} \\
  {\TermTwoT{\alpha:p,x:\El{p}{\alpha}}{\var{x}}{\One_{\alpha.x}}{\TrPlus{\ApEl{p}{\pi^\alpha_x}}{x}}{\El{p}{\alpha.x}}} 
    \and
    \var{x} :\equiv \ap \snd {\varepsilon^\chi_{(\alpha, x)}}
  \end{mathpar}
  \item Pairing for the comprehension object is given by ap of
  $.$ on the pairing for $\Sigma$-modes:
  \begin{mathpar}
  \inferrule{\TermTwoT{\gamma}{s}{\alpha}{\beta}{p} \and
             \TermTwoT{\gamma}{m}{\mu}{\TrPlus{\ApEl{p}{s}}{\nu}}{\El{p}{\alpha}}}
            {\TermTwoT{\gamma}{s.m}{\alpha.\mu}{\beta.\nu}{p}} \and
  s \bdot m :\equiv \ApPlus{\chi}{(s, m)}
  \end{mathpar}
  \item We write $\ApOne{s}$ as a short hand for $\ap{\One_z}{s/z}$
  \end{itemize}
\end{definition}

The triangle identities for $\chi^\circ \dashv \chi^+$ take the following form:
\begin{align}
\eta^\chi_{\alpha.x};(\pi_x^\alpha \bdot \var{x}) &\equiv \id_{\alpha.x} \\
(\eta^\chi_\alpha ; \pi^\alpha_{\One_\alpha}, \ApOne{\eta^\chi_\alpha} ; \ApPlus{\ApEl{p}{\eta^\chi_\alpha}}{\var{\One_\alpha}}) &\equiv \id_{(\alpha, \One_\alpha)}
\end{align}

Requiring a left adjoint for $\psi^+$ is a cute way of stating the following:
\begin{lemma}
For $p$ a comprehension object, $\emptyset : p$ is terminal.
\end{lemma}
\begin{proof}
We always have a map $\eta^\psi_\alpha : \alpha \vDash_p \emptyset$. To show it is unique, note that one of the triangle identities is 
\begin{align*}
\eta^\psi_\emptyset ; \ApPlus{\psi}{\id_{()}} \equiv \eta^\psi_\emptyset \equiv \id_\emptyset
\end{align*}
so for any $s : \alpha \vDash_p \emptyset$ we check:
\begin{align*}
s 
&\equiv s ; \eta^\psi_\emptyset \\
&\equiv \eta^\psi_\alpha ; \ApPlus{\psi}{\ApCirc{\psi}{s}} \\
&\equiv \eta^\psi_\alpha ; \ApPlus{\psi}{\id_{()}} \\
&\equiv \eta^\psi_\alpha
\end{align*}
by naturality of $\eta^\psi$.
\end{proof}

Some other derivable equations:
\begin{itemize}
\item Fusion for $.$
\begin{align}
\label{dot-fusion}
    (s \bdot m);(s' \bdot m') \equiv ((s;s') \bdot (m;\ApPlus{\ApEl{p}{s}} {m'}))
\end{align}
follows by fusing the ap's and the corresponding fusion rule for morphisms in $\Sigma$ modes:
\begin{align*}
(s \bdot m);(s' \bdot m') &\equiv \ApPlus{\chi}{(s, m)} ; \ApPlus{\chi}{(s', m')} \\
&\equiv \ApPlus{\chi}{(s, m);(s', m')} \\
&\equiv ((s;s') \bdot (m;\ApPlus{\ApEl{p}{s}} {m'}))
\end{align*}

\item $\pi^\alpha_\mu$ is natural in $\mu$:
  \begin{align}
  \label{pi-naturality}
  (s \bdot m); \pi^\alpha_\mu &\equiv \pi^\beta_\nu;s && \text{where } \TermTwoT{\gamma}{m}{\mu}{\TrPlus{\ApEl{p}{s}}{\nu}}{\El{p}{\alpha}}
  \end{align}
  by
  \begin{align*}
  (s \bdot m); \pi^\alpha_\mu 
  &\equiv \ApPlus{\chi}{(s, m)} ; \ap \fst {\varepsilon^\chi_{(\alpha, \mu)}} \\  
  &\equiv \ap{\fst}{\ApCirc{\chi}{\ApPlus{\chi}{(s, m)}}} ; \ap \fst {\varepsilon^\chi_{(\alpha, \mu)}} \\
  &\equiv \ap{\fst}{\ApCirc{\chi}{\ApPlus{\chi}{(s, m)}} ; \varepsilon^\chi_{(\alpha, \mu)}}  \\
  &\equiv \ap{\fst}{\varepsilon^\chi_{(\beta, \nu)}; (s, m) } \\
  &\equiv \ap{\fst}{\varepsilon^\chi_{(\beta, \nu)}} ; s\\
  &\equiv \pi^\beta_\nu ; s
  \end{align*}

\item Beta reduction for first projection:
  \begin{align}
\label{beta-pi}
\eta_\beta;(s \bdot m);\pi^\alpha_\mu &\equiv s && \text{where } \TermTwoT{\gamma}{m}{\One_\alpha}{\TrPlus{\ApEl{p}{s}}{\mu}}{\alpha}
  \end{align}
follows from naturality and the second triangle equation by:
\begin{align*}
\eta_\beta;(s \bdot m);\pi^\alpha_\mu
&\equiv \eta_\beta;\pi^\beta_{\One_\beta};s \\
&\equiv s
\end{align*}

\item Naturality of $\var{}$:
\begin{align}
\label{beta-var}
\ApOne{(s \bdot m)};\ApPlus{\ApEl{p}{(s \bdot m)}}{\var{\mu}} &\equiv \var{\nu};\ApPlus{\ApEl{p}{\pi^\beta_\nu}}{m}  && \text{where } \TermTwoT{\gamma}{m}{\nu}{\TrPlus{\ApEl{p}{s}}{\mu}}{\El{p}{\beta}}
\end{align}
is derivable by:
\begin{align*}
\ApOne{(s \bdot m)};\ApPlus{\ApEl{p}{(s \bdot m)}}{\var{\mu}} 
&\equiv \ApOne{(s \bdot m)};\ApPlus{\ApEl{p}{\ApPlus{\chi}{(s, m)}}}{\ap \snd {\varepsilon^\chi_{(\alpha, \mu)}}} \\
&\equiv \ap \snd {((s \bdot m), \ApOne{(s \bdot m)});\varepsilon^\chi_{(\alpha, \mu)}} \\
&\equiv \ap \snd {\ap{\TrCirc{\chi}{\TrPlus{\chi}{z}}}{(s, m)/z};\varepsilon^\chi_{(\alpha, \mu)}} \\
&\equiv \ap \snd {\varepsilon^\chi_{(\beta, \nu)};(s, m)} \\
&\equiv \var{\nu}; \ApPlus{\ApEl{p}{\pi^\beta_{\nu}}}{m} 
\end{align*}

\item The eta principle for pairing:
\begin{align}
\label{eta-pi-var}
t &\equiv \eta_\beta;((t;\pi_\mu^\alpha) \bdot (\ApOne{t}; \ApPlus{\ApEl{p}{t}}{\var{\mu}})) && \text{where } \TermTwoT{\gamma}{t}{\beta}{\alpha.\mu}{p}
\end{align}
is derived by the first triangle equation followed by naturality of $\eta$:
\begin{align*}
t &\equiv t;\eta_{\alpha.\mu};(\pi_\mu^\alpha \bdot \var{\mu}) \\
&\equiv \eta_\beta;\ap{\TrPlus{\chi}{\TrCirc{\chi}{z}}}{t/z};(\pi_\mu^\alpha \bdot \var{\mu}) \\
&\equiv \eta_\beta;\ap{(z \bdot \One_z)}{t/z};(\pi_\mu^\alpha \bdot \var{\mu}) \\
&\equiv \eta_\beta;(t \bdot \ApOne{t});(\pi_\mu^\alpha \bdot \var{\mu}) \\
&\equiv \eta_\beta;((t;\pi_\mu^\alpha) \bdot (\ApOne{t}; \ApPlus{\ApEl{p}{t}}{\var{\mu}})) \\
\end{align*}

\item Naturality of $\eta^\chi_\beta$ takes the following form:
\begin{align*}
s;\eta^\chi_\mu \equiv \eta^\chi_{\mu'} ; (s \bdot \ApOne{s})
\end{align*}

\end{itemize}

We think of a term of mode $p$ as a ``context'', a mode term morphism of
mode $p$ as a ``substitution'', a term of mode $\El{p}{\alpha}$ as a
``dependent type'', and a mode term morphism $\One_\alpha
\Yields_{\El{p}{\alpha}} \mu$ as a ``term'' of ``type'' $\mu$.  For the
equations: $\TrPlus{\ApEl{p}{s}}{-}$ is supposed to act like
substitution; equation \eqref{s-plus-one-strict} says that substitution
into the unit type gives the unit type in a different ``context''.

Equation \eqref{beta-var} says that ``substituting'' into a ``variable'' is second projection on the ``substitution''.  Equation \eqref{eta-pi-var} is the usual $\eta$ principle for subsitutions.

\begin{lemma}\label{sigma:total-to-fiber0} 
For any comprehension object $p$, mode term morphisms $s : \alpha \vDash_p \alpha.x$ such that $s;\pi^\alpha_x \equiv \id_\alpha$ correspond bijectively to 2-cells $\One_\alpha \vDash_{\El{p}{\alpha}} x$.
\end{lemma}
\begin{proof}
Given such an $s$, we can define
\begin{align*}
\hat{s} &: \One_\alpha \vDash_{\El{p}{\alpha}} x \\
\hat{s} &:\equiv \ApOne{s};\ApPlus{s}{\var{x}}
\end{align*}
where the identity $s;\pi^\alpha_x \equiv \id_\alpha$ is used to verify that the codomain is $\TrPlus{s}{\TrPlus{\pi^\alpha_x}{x}} \equiv \TrPlus{(s;\pi^\alpha_x)}{x} \equiv x$.

Conversely, given $m : \One_\alpha \vDash_{\El{p}{\alpha}} x$ we have:
\begin{align*}
\tilde{m} &: \alpha \vDash_p \alpha.x \\
\tilde{m} &:\equiv \eta^\chi_\alpha ; (\id_\alpha \bdot m)
\end{align*}
And indeed $\tilde{m};\pi^\alpha_x \equiv \eta^\chi_\alpha ; (\id_\alpha \bdot m);\pi^\alpha_x \equiv \id_\alpha$ by Equation~\eqref{beta-pi}.

It is easy to check the round trips are the identity:
\begin{align*}
&\eta^\chi_\alpha ; (\id_\alpha \bdot \ApOne{s};\ApPlus{s}{\var{x}}) \\
&\equiv \eta^\chi_\alpha ; (s;\pi^\alpha_x \bdot \ApOne{s};\ApPlus{s}{\var{x}}) \\
&\equiv \eta^\chi_\alpha ; (s \bdot \ApOne{s});(\pi^\alpha_x \bdot \var{x}) \\
&\equiv s;\eta^\chi_{\alpha.x} ; (\pi^\alpha_x \bdot \var{x}) \\
&\equiv s
\end{align*}
And:
\begin{align*}
&\ApOne{\eta^\chi_\alpha ; (\id_\alpha \bdot m)};\ApPlus{(\eta^\chi_\alpha ; (\id_\alpha \bdot m))}{\var{x}} \\
&\equiv \ApOne{\eta^\chi_\alpha};\ApPlus{\eta^\chi_\alpha}{\ApOne{(\id_\alpha \bdot m)};\ApPlus{(\id_\alpha \bdot m)}{\var{x}}} \\
&\equiv \ApOne{\eta^\chi_\alpha};\ApPlus{\eta^\chi_\alpha}{\var{\One_\alpha};\ApPlus{\pi^\alpha_{\One_\alpha}}{m}} \\
&\equiv \ApOne{\eta^\chi_\alpha};\ApPlus{\eta^\chi_\alpha}{\var{\One_\alpha}};\ApPlus{\eta^\chi_\alpha}{\ApPlus{\pi^\alpha_{\One_\alpha}}{m}} \\
&\equiv \ApPlus{\eta^\chi_\alpha}{\ApPlus{\pi^\alpha_{\One_\alpha}}{m}} \\
&\equiv m
\end{align*}
\end{proof}

\begin{remark}
This axiomatization was inspired by the definition of comprehension
category with unit in \cite{ahman+16fibered}:
\msnote{I believe this definition is actually originally due to Ehrhard, and was reformulated later by Jacobs, before being picked up by AGP.  See \url{https://ncatlab.org/nlab/show/axiom+of+separation\#LawvereDefinition}.}
\begin{itemize}
\item A fibration $p : \mathcal{E} \to \mathcal{B}$
\item A functor $1 : \mathcal{B} \to \mathcal{E}$ that picks out the terminal object of each fiber,
\item A right adjoint $\{-\} : \mathcal{E} \to \mathcal{B}$ to the functor $1$.
\end{itemize}
In our notation:
\begin{itemize}
\item $\mathcal{B}$ corresponds to the mode $p$ and $\mathcal{E}$ to $\sigmacl{\alpha}{p}{\El{p}{\alpha}}$, with the morphisms in both categories given by the mode-term-2-cells in each.

\item The projection $p : \mathcal{E} \to \mathcal{B}$ is the function $\fst : \sigmacl{\alpha}{p}{\El{p}{\alpha}} \to p$. 

\item The functor $\{-\}$ is given by $\chi^+$ and the functor $1$ is given by $\chi^\circ$.
\end{itemize}
\end{remark}

\subsection{Comprehension Object with Unit}
\mvrnote{Conjecture:}
\begin{definition}
A comprehension object \emph{supports the unit type} if $\eta^\chi$ is an isomorphism, i.e.:
\begin{align}
\pi^\alpha_{\One_\alpha} ; \eta^\chi_\alpha \equiv \id_{\alpha.\One_\alpha}
\end{align}
\end{definition}
Composition in the other direction is already the identity, by one of the triangle identities for $\chi$.

\begin{lemma}
If $p$ supports $1$ then the following equations hold:
\begin{align}
\var{\One_\alpha} &\equiv \ApOne{\pi^\alpha_{\One_\alpha}} \\
\pi^{\alpha.\One_\alpha}_{\One_{\alpha.\One_\alpha}} &\equiv (\pi^\alpha_{\One_\alpha} \bdot \ApOne{\pi^\alpha_{\One_\alpha}}) \\
\pi^{\alpha.\One.\One}_\One;\pi^{\alpha.\One}_\One &\equiv (\pi^{\alpha.\One}_\One;\pi^{\alpha}_\One) \bdot (\ApOne{\pi^{\alpha.\One}_\One;\pi^{\alpha}_\One})
\end{align}
\end{lemma}
\begin{proof}
The first:
\begin{align*}
\var{\One_\alpha} 
&\equiv \ApPlus{\pi^\alpha_{\One_\alpha}}{\ApPlus{\eta^\chi_\alpha}{\var{\One_\alpha}}} \\
&\equiv \ApOne{\pi^\alpha_{\One_\alpha} ; \eta^\chi_\alpha};\ApPlus{\pi^\alpha_{\One_\alpha}}{\ApPlus{\eta^\chi_\alpha}{\var{\One_\alpha}}} \\
&\equiv \ApOne{\pi^\alpha_{\One_\alpha}};\ApPlus{\pi^\alpha_{\One_\alpha}}{\ApOne{\eta^\chi_\alpha}};\ApPlus{\pi^\alpha_{\One_\alpha}}{\ApPlus{\eta^\chi_\alpha}{\var{\One_\alpha}}} \\
&\equiv \ApOne{\pi^\alpha_{\One_\alpha}};\ApPlus{\pi^\alpha_{\One_\alpha}}{\ApOne{\eta^\chi_\alpha};\ApPlus{\eta^\chi_\alpha}{\var{\One_\alpha}}} \\
&\equiv \ApOne{\pi^\alpha_{\One_\alpha}};\ApPlus{\pi^\alpha_{\One_\alpha}}{\id_{\alpha}} \\
&\equiv \ApOne{\pi^\alpha_{\One_\alpha}}
\end{align*}
The second follows from naturality of $\eta^\chi$ by
\begin{align*}
\id_{\alpha.\One_\alpha} 
&\equiv \pi^\alpha_{\One_\alpha} ; \eta^\chi_\alpha \\
&\equiv \eta^\chi_{\alpha.\One_\alpha} ; (\pi^\alpha_{\One_\alpha} \bdot \ApOne{\pi^\alpha_{\One_\alpha}})
\end{align*}
and precomposing with $\pi^{\alpha.\One_\alpha}_{\One_{\alpha.\One_\alpha}}$.
And the third by applying the second equation to each $\pi$ and using fusion.
\end{proof}

\subsection{Comprehension Object with $\Sigma$}
\newcommand\mtt[1]{\mathtt{#1}}
\newcommand\contract[1]{\ensuremath{\mathtt{contract}_{#1}}}
\newcommand\fibpair[1]{\ensuremath{\mathtt{fibpair}_{#1}}}
\newcommand\pair[1]{\ensuremath{\mathtt{pair}_{#1}}}
\newcommand\tsplit[1]{\ensuremath{\mathtt{split}_{#1}}}

\begin{definition}\label{def:supports-sigmas}
A comprehension object \emph{supports $\Sigma$ types} if it supports $1$ and there is a specified mode term
\begin{align*}
\alpha : p, x : \El{p}{\alpha}, y : \El{p}{\alpha.x} \vdash \Sigma_1(\alpha,x,y) : \El{p}{\alpha}
\end{align*}
and mode term morphisms
\begin{align*}
\contract{\alpha} &: \One_\alpha \vDash_{\El{p}{\alpha}} \Sigma_1(\alpha,\One_\alpha,\One_{\alpha.{\One_\alpha}}) \\
\tsplit{\alpha,x,y} &: \alpha.\Sigma_1(\alpha,x,y) \vDash_{p} \alpha.x.y
\end{align*}
such that $\tsplit{\alpha,x,y}$ is an inverse to $\pair{\alpha,x,y}$, defined by:
\begin{align*}
\fibpair{\alpha,x,y} &: \One_{\alpha.x.y} \vDash_{\El{p}{\alpha.x.y}} \TrPlus{(\pi^{\alpha.x}_y;\pi^\alpha_x)}{\Sigma_1(\alpha,x,y)} \\
\fibpair{\alpha,x,y} &:\equiv \contract{\alpha.x.y};\ap{\Sigma_1(\alpha,x,y)}{(\pi^{\alpha.x}_y;\pi^{\alpha}_x,\ApOne{\pi^{\alpha.x}_y};\ApPlus{\ApEl{p}{\pi^{\alpha.x}_y}}{\var{x}}, \ApOne{\pi^{\alpha.x.y}_{\One_{\alpha.x.y}}};\ApPlus{\ApEl{p}{\pi^{\alpha.x.y}_{\One_{\alpha.x.y}}}}{\var{y}})/(\alpha,x,y)} \\
\pair{\alpha,x,y} &: \alpha.x.y \vDash_{p} \alpha.\Sigma_1(\alpha,x,y) \\
\pair{\alpha, x, y} &:\equiv \eta^\chi_{\alpha.x.y};((\pi^{\alpha.x}_y;\pi^\alpha_x) \bdot \fibpair{\alpha,x,y})
\end{align*}
and $\contract{\alpha}$ is natural:
\begin{align}
\contract{\alpha};\ap{\Sigma_1(\alpha,x,y)}{(s, \ApOne{s}, \ApOne{s \bdot \ApOne{s}})/(\alpha,x,y)} \equiv \ApOne{s};\ApPlus{\ApEl{p}{s}}{\contract{\beta}}
\end{align}
for any $s : \alpha \vDash \beta$.
\end{definition}

\begin{lemma}
A comprehension object that supports $\Sigma$s satisfies the following equations:
\begin{align}
\pair{\alpha,x,y};\pi^\alpha_{\Sigma_1(\alpha, x, y)} &\equiv \pi^{\alpha.x}_y;\pi^\alpha_x \\
\tsplit{\alpha,x,y};\pi^{\alpha.x}_y;\pi^\alpha_x &\equiv \pi^\alpha_{\Sigma_1(\alpha, x, y)} \\
\fibpair{\alpha,\One_\alpha,\One_{\alpha.\One_\alpha}} &\equiv \ApOne{\pi^{\alpha.\One_\alpha}_{\One_{\alpha.\One_\alpha}};\pi^\alpha_{\One_\alpha}};\ApPlus{(\pi^{\alpha.\One_\alpha}_{\One_{\alpha.\One_\alpha}};\pi^\alpha_{\One_\alpha})}{\contract{\alpha}} \\
(\id_\alpha \bdot \contract{\alpha});\tsplit{\alpha,\One_\alpha,\One_{\alpha.\One_\alpha}} &\equiv \eta^\chi_{\alpha.\One_\alpha}
\end{align}
\end{lemma}
\begin{proof}
The first follows from the definition of $\pair{}$ and Equation~\ref{beta-pi}. The second is immediate from the first, precomposing with $\tsplit{}$. For the third we use naturality of $\contract{}$, suppressing most of the subscripts of $1$ for readability:
\begin{align*}
&\fibpair{\alpha,\One_\alpha,\One_{\alpha.\One_\alpha}} \\
%&\equiv \contract{\alpha.\One_\alpha.\One_{\alpha.\One_\alpha}};\ap{\Sigma_\One(\alpha,x,y)}{(\pi^{\alpha.\One_\alpha}_{\One_{\alpha.\One_\alpha}};\pi^{\alpha}_{\One_\alpha},\ApOne{\pi^{\alpha.\One_\alpha}_{\One_{\alpha.\One_\alpha}}};\ApPlus{\ApEl{p}{\pi^{\alpha.\One_\alpha}_y}}{\var{\One_\alpha}}, \ApOne{\pi^{\alpha.\One_\alpha.{\One_{\alpha.\One_\alpha}}}_{\One_{\alpha.\One_\alpha.\One_{\alpha.\One_\alpha}}}};\ApPlus{\ApEl{p}{\pi^{\alpha.\One_\alpha.{\One_{\alpha.\One_\alpha}}}_{\One_{\alpha.\One_\alpha.{\One_{\alpha.\One_\alpha}}}}}}{\var{\One_{\alpha.\One_\alpha}}})/(\alpha,x,y)}
&\equiv \contract{\alpha.\One_\alpha.\One_{\alpha.\One_\alpha}};\ap{\Sigma_\One(\alpha,x,y)}{(\pi^{\alpha.\One}_\One;\pi^{\alpha}_\One,\ApOne{\pi^{\alpha.\One}_\One};\ApPlus{\ApEl{p}{\pi^{\alpha.\One}_\One}}{\var{\One_\alpha}}, \ApOne{\pi^{\alpha.\One.\One}_\One};\ApPlus{\ApEl{p}{\pi^{\alpha.\One.\One}_\One}}{\var{\One_{\alpha.\One_\alpha}}})/(\alpha,x,y)} \\
&\equiv \contract{\alpha.\One_\alpha.\One_{\alpha.\One_\alpha}};\ap{\Sigma_\One(\alpha,x,y)}{(\pi^{\alpha.\One}_\One;\pi^{\alpha}_\One,\ApOne{\pi^{\alpha.\One}_\One};\ApPlus{\ApEl{p}{\pi^{\alpha.\One}_\One}}{\ApOne{\pi^\alpha_\One}}, \ApOne{\pi^{\alpha.\One.\One}_\One};\ApPlus{\ApEl{p}{\pi^{\alpha.\One.\One}_\One}}{\ApOne{\pi^{\alpha.\One}_\One}})/(\alpha,x,y)} \\
&\equiv \contract{\alpha.\One_\alpha.\One_{\alpha.\One_\alpha}};\ap{\Sigma_\One(\alpha,x,y)}{(\pi^{\alpha.\One}_\One;\pi^{\alpha}_\One,\ApOne{\pi^{\alpha.\One}_\One;\pi^\alpha_\One}, \ApOne{\pi^{\alpha.\One.\One}_\One;\pi^{\alpha.\One}_\One})/(\alpha,x,y)} \\
&\equiv \contract{\alpha.\One_\alpha.\One_{\alpha.\One_\alpha}};\ap{\Sigma_\One(\alpha,x,y)}{(\pi^{\alpha.\One}_\One;\pi^{\alpha}_\One,\ApOne{\pi^{\alpha.\One}_\One;\pi^\alpha_\One}, \ApOne{(\pi^{\alpha.\One}_\One;\pi^{\alpha}_\One) \bdot (\ApOne{\pi^{\alpha.\One}_\One;\pi^{\alpha}_\One})})/(\alpha,x,y)} \\
&\equiv \ApOne{\pi^{\alpha.\One}_\One;\pi^{\alpha}_\One};\ApPlus{\ApEl{p}{\pi^{\alpha.\One}_\One;\pi^{\alpha}_\One}}{\contract{\alpha}}
\end{align*}
And the last follows from the third by:
\begin{align*}
&(\id_\alpha \bdot \contract{\alpha});\tsplit{\alpha,\One_\alpha,\One_{\alpha.\One_\alpha}} \\
&\equiv \eta^\chi_{\alpha.\One_\alpha} ; (\pi^\alpha_{\One_\alpha} \bdot \ApOne{\pi^\alpha_{\One_\alpha}}) ; (\id_\alpha \bdot \contract{\alpha});\tsplit{\alpha,\One_\alpha,\One_{\alpha.\One_\alpha}} \\
&\equiv \eta^\chi_{\alpha.\One_\alpha} ;  \eta^\chi_{\alpha.\One_\alpha.\One_{\alpha.\One_\alpha}} ; (\pi^{\alpha.\One_\alpha}_{\One_{\alpha.\One_\alpha}} \bdot \ApOne{\pi^{\alpha.\One_\alpha}_{\One_{\alpha.\One_\alpha}}}) ; (\pi^\alpha_{\One_\alpha} \bdot \ApOne{\pi^\alpha_{\One_\alpha}}) ; (\id_\alpha \bdot \contract{\alpha});\tsplit{\alpha,\One_\alpha,\One_{\alpha.\One_\alpha}} \\
&\equiv \eta^\chi_{\alpha.\One_\alpha} ;  \eta^\chi_{\alpha.\One_\alpha.\One_{\alpha.\One_\alpha}} ; ((\pi^{\alpha.\One_\alpha}_{\One_{\alpha.\One_\alpha}};\pi^\alpha_{\One_\alpha}) \bdot (\ApOne{\pi^{\alpha.\One_\alpha}_{\One_{\alpha.\One_\alpha}}};\ApPlus{\pi^{\alpha.\One_\alpha}_{\One_{\alpha.\One_\alpha}}}{\ApOne{\pi^\alpha_{\One_\alpha}}})) ; (\id_\alpha \bdot \contract{\alpha});\tsplit{\alpha,\One_\alpha,\One_{\alpha.\One_\alpha}} \\
&\equiv \eta^\chi_{\alpha.\One_\alpha};\eta^\chi_{\alpha.\One_\alpha.\One_{\alpha.\One_\alpha}};((\pi^{\alpha.\One_\alpha}_{\One_{\alpha.\One_\alpha}};\pi^\alpha_{\One_\alpha}) \bdot \ApOne{\pi^{\alpha.\One_\alpha}_{\One_{\alpha.\One_\alpha}};\pi^\alpha_{\One_\alpha}};\ApPlus{(\pi^{\alpha.\One_\alpha}_{\One_{\alpha.\One_\alpha}};\pi^\alpha_{\One_\alpha})}{\contract{\alpha}});\tsplit{\alpha,\One_\alpha,\One_{\alpha.\One_\alpha}} \\
&\equiv \eta^\chi_{\alpha.\One_\alpha};\eta^\chi_{\alpha.\One_\alpha.\One_{\alpha.\One_\alpha}};((\pi^{\alpha.\One_\alpha}_{\One_{\alpha.\One_\alpha}};\pi^\alpha_{\One_\alpha}) \bdot \fibpair{\alpha,\One_\alpha,\One_{\alpha.\One_\alpha}});\tsplit{\alpha,\One_\alpha,\One_{\alpha.\One_\alpha}} \\
&\equiv \eta^\chi_{\alpha.\One_\alpha};\pair{\alpha,\One_\alpha,\One_{\alpha.\One_\alpha}};\tsplit{\alpha,\One_\alpha,\One_{\alpha.\One_\alpha}} \\
&\equiv \eta^\chi_{\alpha.\One_\alpha}
\end{align*}
\end{proof}

\mvrnote{Is there some way of stating this by inserting a mode type morphism, similar to how we define a comprehension object? What if in the mode theory we have something like:}

\begin{align*}
\mathtt{whatever} : \sum_{\alpha : p} \El{p}{\alpha} \vDash \sum_{\alpha : p} \sum_{x : \El{p}{\alpha}} \El{p}{\alpha.x}
\end{align*}
with: 
\begin{align*}
\fst \TrPlus{\mathtt{whatever}}{\alpha, x, y} \equiv \alpha
\end{align*}
and then defining
\begin{align*}
\Sigma_1(\alpha, x, y) :\equiv \snd \TrPlus{\mathtt{whatever}}{\alpha, x, y} 
\end{align*}
Does this get us anything for free? Where does $\mathtt{contract}$ come from? Is it an adjunction? Is the $\tsplit{}$ condition saying that a unit/counit is the identity?
 
\mvrnote{Dan Notes:}

This doesn't necessarily need to go into the final paper, but here are
some notes that compare different ways of doing the mode theory for
$\Sigma$ types.

We assume a comprehension object $\chi$ given by a
comprehension-terminal adjunction (definition yet to be spelled out
without the automatic adjunction notation, but all the same stuff should
be defined).


\begin{enumerate}

\item \label{sigma:total-to-fiber0} Conjecture: In any comprehension
  object $\chi$, 2-cells $s : \alpha \vDash_p \alpha.x$ such that $s;\pi
  = \id_\alpha$ correspond bijectively to 2-cells $\One_\alpha
  \vDash_{\El{p}{\alpha}} x$.

  First, given $s : \alpha \vDash_p \alpha.x$, take $\mtt{var} : \One_{\alpha.x}
  \vDash_{\El{p}{\alpha.x}} \TrPlus{(\pi^\alpha_x)}{x}$ and ap $s$ onto
  it to get $\TrPlus{s}{1} \vDash_{\El{p}{\alpha}}
  \TrPlus{s}{\TrPlus{(\pi^\alpha_x)}{x}}$.  Then fuse, reduce
  $\TrPlus{s}{1}$, and use the section property to show that this also
  has type
  $\One_\alpha \vDash_{\El{p}{\alpha}} x$.

  Conversely, given $t : \One_\alpha \vDash_{\El{p}{\alpha}} x$,
  the pairing of $\chi$ gives $(\id, t) : \alpha \vDash \alpha.x$.

\item \label{sigma:total-to-fiber1} Conjecture: similarly, 2-cells $s :
  \alpha.x \vDash_p \alpha.y$ such that $s;\pi_y = \pi_x$ correspond
  bijectively to 2-cells $\One_{\alpha.x} \vDash_{\El{p}{\alpha.x}}
  \TrPlus{\pi^\alpha_x}{y}$.

  First, given $s : \alpha.x \vDash_p \alpha.y$, take $\mtt{var} :
  \One_{\alpha.y} \vDash_{\El{p}{\alpha.y}} \TrPlus{(\pi^\alpha_y)}{y}$ and
  ap $s$ onto it to get $\TrPlus{s}{1} \vDash_{\El{p}{\alpha.x}}
  \TrPlus{s}{\TrPlus{(\pi^\alpha_y)}{x}}$.  Then fuse, reduce
  $\TrPlus{s}{1}$, and use the commuting triangle to show that this also
  has type $\One_{\alpha.x} \vDash_{\El{p}{\alpha.x}} \TrPlus{\pi_x} y$.

  Conversely, given $t : \One_{\alpha.x} \vDash_{\El{p}{\alpha.}}
  \TrPlus{\pi_x} y$, we have $\id.t := \ap{.}{(\id, t)} : \alpha.x \vDash \alpha.y$.

\item \label{sigma:total-to-fiber2} Conjecture: similarly, 2-cells $s :
  \alpha.x.y \vDash_p \alpha.z$ such that $s;\pi_z = \pi_y;\pi_x$
  correspond bijectively to 2-cells $\One_{\alpha.x.y}
  \vDash_{\El{p}{\alpha.x}} \TrPlus{\pi^{\alpha.x}_y;\pi^\alpha_x}{y}$.

\item \label{sigma:full} Define \emph{fullness} of the comprehension
  object to mean that maps $s : \alpha.x \vDash_p \alpha.y$ such that
  $s;\pi_y = \pi_x$ correspond bijectively to maps in the fiber $x
  \vDash_{\El{p}{\alpha}} y$.  (Probably plus some naturality.)

  (Is this the same as $U$ from the comprehesion/slide adjunction before
  being full and faithful?)

\item \label{sigma:full-interesting} Define \emph{terminal fullness}
  (probably a better name) of the comprehension object to mean that maps
  maps $\One_{\alpha.x} \vDash_{\El{p}{\alpha.x}} \TrPlus{\pi^\alpha_x}{y}$
  out of the terminal object are bjective with $x
  \vDash_{\El{p}{\alpha}} y$.

  One direction is automatic: given $s : x \vDash_{\El{p}{\alpha}} y$,
  we can make $(\id.s) : \alpha.x \vDash \alpha.y$ as above, and ap that
  onto $\mtt{var} : \One_{\alpha.y} \vDash \TrPlus{\pi_y}{y}$ to get
  $\One_{\alpha.x} \vDash_{\El{p}{\alpha.x}} \TrPlus{\pi^\alpha_x}{y}$.

\item By \ref{sigma:total-to-fiber1}, fullness follows iff terminal
  fullness.  
  
\item \label{sigma:complete}

  A ``complete spec'' (cf. Jacobs REF) for $\Sigma$ types is
  \begin{itemize}
  \item $\alpha : p, x : \El{p}{\alpha}, y : \El{p}{\alpha.x} \vdash \Sigma_1(\alpha,x,y) : \El{p}{\alpha}$,
  \item $\Sigma_1(\alpha,x,-) \dashv \TrPlus{\pi^\alpha_x}{-}$
  \item $E : \Sigma_1(\alpha,x,\One_{\alpha.x}) \vDash x$ given by
    transposing $\mtt{var}$ is an isomorphism
  \item The induced map $\mtt{pair} : \alpha.x.y \vDash
    \alpha.\Sigma_1(\alpha,x,y)$ given by $(\pi, \eta^\pi)$ is an
    isomorphism.
  \item A BC condition $\TrPlus{s}{(\Sigma_1(\alpha,x,y))} \equiv
    \Sigma_1(\beta,\TrPlus{s}{x}, \TrPlus{(s . \id)}{y})$,
    where $s.\id : \beta.\TrPlus{s}{x} \vDash \alpha.x$
  \end{itemize}
  I'd be very surprised if this were not sufficient -- it's everything I
  think we could ask for.

\item Conjecture: given the adjunction $\Sigma_1(\alpha,x,-) \dashv
  \TrPlus{\pi^\alpha_x}{-}$ (actually, only the counit) and $E$ an iso,
  terminal fullness (and therefore fullness) holds.  The converse map
  for $\One_{\alpha.x} \vDash_{\El{p}{\alpha.x}} \TrPlus{\pi^\alpha_x}{y}$
  is given by transposing to get $\Sigma_1(\alpha,x,\One_{\alpha.x}) \vDash
  y$ and composing with $E^-1 : x \vDash
  \Sigma_1(\alpha,x,\One_{\alpha.x})$.

  Conversely, given \ref{sigma:full-interesting}, do we get the counit
  $\Sigma_1(\alpha,x,\TrPlus{\pi^\alpha_x}{y}) \vdash_{\El{p}{\alpha}} y$?

\item \label{sigma:total-spec} Another possible spec for $\Sigma$ types
  would be
  \begin{itemize}
  \item $\alpha : p, x : \El{p}{\alpha}, y : \El{p}{\alpha.x} \vdash \Sigma_1(\alpha,x,y) : \El{p}{\alpha}$,
  \item A map $\mtt{pair} : \alpha.x.y \vDash
    \alpha.\Sigma_1(\alpha,x,y)$ that commutes over $\alpha$, probably
    with some naturality
  \item $\mtt{pair}$ is an isomorphism (with inverse $\mtt{split}$)
  \item A BC condition $\TrPlus{s}{(\Sigma_1(\alpha,x,y))} \equiv
    \Sigma_1(\beta,\TrPlus{s}{x}, \TrPlus{(s . \id)}{y})$?
  \end{itemize}

  This has all the axioms in $p$, rather than asserting the adjunction
  in the fiber.

\item Conjecture: given \ref{sigma:total-spec} (pairing and split in
  $p$) and fullness in the sense of \ref{sigma:full-interesting}, we
  can derive the adjunction $\Sigma_1(\alpha,x,-) \dashv
  \TrPlus{\pi^\alpha_x}{-}$.

  E.g. for the unit $y \vdash_{\alpha.x} \TrPlus{\pi^\alpha_x}{\Sigma_1(\alpha,x,y)}$
  we can get
  $\One_{\alpha.x.y} \vdash_{\alpha.x.y} \TrPlus{\pi_y;\pi_x}{\Sigma_1(\alpha,x,y)}$
  from $\mtt{pair}$ by \ref{sigma:total-to-fiber2}.
  Then use \ref{sigma:full-interesting}.  

  For the counit
  $\Sigma_1(\alpha,x,\TrPlus{\pi^\alpha_x}{z}) \vdash_{\alpha} z$
  we have
  \[
  \alpha.\Sigma_1(\alpha,x,\TrPlus{\pi^\alpha_x}{z}) \vDash \alpha.x.\TrPlus{\pi^\alpha_x}{z} \vDash \alpha.z 
  \]
  and then can use the full version of fullness (\ref{sigma:full}).    

  Thus, if the comprehension category is already full, then pair and
  split are enough, which explains why we don't usually talk about the
  adjunction itself inside the type theory.  
  
\item The exact 2-cell that comes up in encoding pairing for
  object-language $\Sigma$-types below is a contraction:
  \[
  \One_\alpha \vDash_{\El{p}{\alpha}} \Sigma_1(\alpha,\One_\alpha,\One_{\alpha.{\One_\alpha}})
  \]
  This seemed odd, because it seems like it's less than full pairing.    

\item \label{sigma:get-pair} However, contraction does imply
  $\mtt{pair}$, because we already know that $\Sigma_1$ is a functor, so
  we have
  \[
  \ap{\Sigma_1}{(\pi^{\alpha.x}_y;\pi^\alpha_x, \mtt{var}, \mtt{var}?)} :
  \Sigma_1(\alpha,\One_\alpha,\One_{\alpha.\One}) \vDash \TrPlus{(\pi^{\alpha.x}_y;\pi^\alpha_x)}{\Sigma_1(\alpha,x,y)}
  \]
  Then precompose with contraction, and then use
  \ref{sigma:total-to-fiber2}.

  (This is a more dependent version of deriving $\Gamma \vdash A$ and
  $\Gamma \vdash B$ implies $\Gamma \vdash A \times B$ from $\Gamma
  \times \Gamma \vdash A \times B$ by functoriality and then
  precomposing with the diagonal.)
  
\item Conversely, a constant $\mtt{pair} : \alpha.x.y \vDash
  \alpha.\Sigma_1(\alpha,x,y)$ (commuting over $\alpha$) implies
  contraction. An instance is $\alpha.\One_{\alpha}.\One_{\alpha.\One} \vDash
  \alpha.\Sigma_1(\alpha,\One_\alpha,\One_{\alpha.\One})$, and the comprehension
  object already has $\beta.\One \cong \beta$, so precompose, to get
  $\alpha \vDash \alpha.\Sigma_1(\alpha,\One_\alpha,\One_{\alpha.\One})$, and
  then use \ref{sigma:total-to-fiber0}.

\item \label{sigma:spec} What we seem to need for $\Sigma$ types is just
  \begin{itemize}
  \item $\alpha : p, x : \El{p}{\alpha}, y : \El{p}{\alpha.x} \vdash \Sigma_1(\alpha,x,y) : \El{p}{\alpha}$,
  \item A map $\mtt{contract} : \One_\alpha \vDash_{\El{p}{\alpha}} \Sigma_1(\alpha,\One_\alpha,\One_{\alpha.{\One_\alpha}})$.
  \item such that the $\mtt{pair}$ map induced as in
    \ref{sigma:get-pair} is an isomorphism (split)
  \end{itemize}

\item It seems surprising that, in addition to not needing the
  adjunction (which is kind of explained by never touching maps out of
  anything other than $1$ in the fiber), we also don't need the BC
  condition.  A partial explanation is that the BC condition is
  (conjecture) true up to isomorphism, at least when phrased in terms of
  total spaces.

  Here are the maps:

  For 
  \[
  \Sigma_1(\beta,\TrPlus{\ApEl{p}{s}}{x}, \TrPlus{\ApEl{p}{s.\id}}{y}) \vDash \TrPlus{\ApEl{p}{s}}{(\Sigma_1(\alpha,x,y))} 
  \]
  observe that there is a 2-cell in the mode theory $\Sigma$-type
  \[
  (\beta,\TrPlus{\ApEl{p}{s}}{x}, \TrPlus{\ApEl{p}{(s . \id)}}{y}) \vDash_{\Sigma z:p.\Sigma x:\El{p}{z}.\El{p}{z.x}} (\alpha,x,y)
  \]
  given by $(s, \id, \id)$ (using the equation for ``ap of framework
  $\Sigma$ on a 2-cell'', and an associativity between $(\ap{w:(\Sigma
    z:p.\El{p}{z}).\El{p}{(\fst w).(\snd w)}}{s , \id})$ and
  $\ApEl{p}{s.\id}$).  $\Sigma_1$'s input is (an uncurried version of)
  this $\Sigma$ type, so
  \[
  \ap{\Sigma_1 (\fst(w), \fst({\snd (w)}), \snd{(\snd w)})}{(s, \id, \id)/w}
  \]
  has the desired type (using the equation for ``ap of \fst{} on a
  2-cell in $\Sigma$'').
  By \ref{sigma:total-to-fiber1} this induces a 2-cell (leaving off the $\ApEl{p}{-}$)
  \[
  \beta.\Sigma_1(\beta,\TrPlus{s}{x}, \TrPlus{s.\id}{y}) \vDash \beta.\TrPlus{s}{(\Sigma_1(\alpha,x,y))} 
  \]

  For the converse, we construct 
  \[
  \beta.\TrPlus{s}{(\Sigma_1(\alpha,x,y))} \vDash \beta.\Sigma_1(\beta,\TrPlus{s}{x}, \TrPlus{s.\id}{y}) 
  \]
  (I'm not sure if this one is actually true in the fiber without
  assuming fullness).
  By post-composing with $\mtt{pair}$, it suffices to give 
  \[
  \beta.\TrPlus{s}{(\Sigma_1(\alpha,x,y))} \vDash \beta.\TrPlus{s}{x}.\TrPlus{(s.\id)}{y}
  \]

  Lemma 1: In general, given $\alpha \vdash \alpha.x$ which is a section
  of $\pi^\alpha_x$, we can ``substitute'' any $s : \beta \vDash \alpha$
  into it to get $\beta \vdash \beta.\TrPlus{x}$, which is a section of
  $\pi^\beta_{\TrPlus{s}{x}}$.  For us, this is derived by using
  \ref{sigma:total-to-fiber0} to reduce the problem to going from $1
  \vDash_{\El{p}{\alpha}} x$ to $1 \vDash_{\El{p}{\beta}}
  \TrPlus{s}{x}$, which is just the ap of ${\TrPlus{s}{-}}$ together
  with it reducing on $1$.

  Lemma 2: similarly, given $\alpha.x \vdash \alpha.y$ (commuting over
  $\alpha$), we can ``substitute'' to get $\beta.\TrPlus{s}{x} \vdash
  \beta.\TrPlus{s}{y}$ (commuting over $\beta$), by using
  \ref{sigma:total-to-fiber1}, ap-ing $s$ in the fiber, reducing the ap
  on $1$, and using the commuting triangle with $\pi$.

  There is a map $\mtt{fst} : \alpha.\Sigma_1(\alpha,x,y) \vDash_p
  \alpha.x$ given by $\mtt{split};\pi$, which commutes over $\alpha$.

  There is also a section $\mtt{snd} : \alpha.\Sigma_1(\alpha,x,y)
  \vDash_p \alpha.\Sigma_1(\alpha,x,y).\TrPlus{\mtt{fst}}{y}$ (this is
  the total-space version of the usual second projection term).  In the
  fiber, we need $1 \vDash_{\alpha.\Sigma_1(\alpha,x,y)}
  \TrPlus{\mtt{fst}}{y}$, but we have the $\mtt{var}$ term $1
  \vDash_{\alpha.x.y} \TrPlus{\pi^{\alpha.x}_y}{y}$ and ap-ing
  $\mtt{split}$ gives the result.
  
  Applying Lemma 2 to $\mtt{fst}$ gives
  \[s^*{\mtt{fst}} : \beta.\TrPlus{s}{\Sigma_1(\alpha,x,y)} \vDash \beta.\TrPlus{s}{x}
  \]
  which commutes over $\beta$.

  Applying Lemma 1 to $\mtt{snd}$ gives $s^*{\snd{}} :
  \beta.\TrPlus{s}{\Sigma_1(\alpha,x,y)} \vDash_p
  \beta.\TrPlus{s}{\Sigma_1(\alpha,x,y)}.\TrPlus{s}{\TrPlus{\mtt{fst}}{y}}$
  commuting over $\alpha$.  (Some of the total space/fiber moves could
  be $\beta$-reduced here.)
  We also should have $s;\mtt{fst} = ;{s.\id}$, so this has type
  \[
  s^*{\snd{}} : \beta.\TrPlus{s}{\Sigma_1(\alpha,x,y)} \vDash_p
  \beta.\TrPlus{s}{\Sigma_1(\alpha,x,y)}.\TrPlus{s^*\mtt{fst}}{\TrPlus{s.\id}{y}}
  \]
  
  Using the comprehension-terminal adjunction to map into $.$, we can
  pair $s^*{\mtt{fst}}$, $s^*{\snd{}}$
  to get
  \[
  \beta.\TrPlus{s}{\Sigma_1(\alpha,x,y)} \vDash (\beta.\TrPlus{s}{x}).{\TrPlus{(s.\id)}{y}}
  \]
  
\end{enumerate}

\subsection{Comprehension Object with $\Pi$}
\newcommand\pinv[1]{\ensuremath{\mathtt{pinv}_{#1}}}
\mvrnote{Conjecture:}

\begin{definition}\label{def:supports-pis}
A comprehension object \emph{has $\Pi$s} if there is a morphism
\begin{align*}
\pinv{\alpha,x} : \TrPlus{\ApEl{p}{\pi^\alpha_x}}{\One_\alpha} \vDash_{\El{p}{\alpha.x}} \One_{\alpha.x} 
\end{align*}
such that
\begin{align*}
\ApPlus{\ApEl{p}{\eta^\chi_\alpha}}{\pinv{\alpha,\One_\alpha}} &\equiv  \ApOne{\eta^\chi_\alpha} \\
\pinv{\alpha,x};\ApPlus{\eta^\chi_{\alpha.x}}{\ap{\TrPlus{\pi^u_v}{\One_u}}{(\pi^\alpha_x,\var{x})/(u,v)}} &\equiv \id_{\TrPlus{\ApEl{p}{\pi^\alpha_x}}{\One_\alpha}}
\end{align*}
\mvrnote{Is that thing, $\ApPlus{\eta^\chi_{\alpha.x}}{\ap{\TrPlus{\pi^u_v}{\One_u}}{(\pi^\alpha_x,\var{x})/(u,v)}}$ equal to $\ApOne{\pi^\alpha_x}$? I couldn't tell...}
\end{definition}

\begin{lemma}
If $\One_\alpha : \El{p}{\alpha}$ is a fibred terminal object, then $p$ has $\Pi$s.
\end{lemma}
\begin{proof}
Take $\pinv{\alpha,x}$ to be the identity on $\One_{\alpha.x}$ and then the equations hold trivially.
\end{proof}

\subsection{Morphism of Comprehension Objects}
\newcommand\fone[1]{\ensuremath{\mathtt{fone}_{#1}}}
\newcommand\foneinv[1]{\ensuremath{\mathtt{foneinv}_{#1}}}

Suppose we have two comprehension objects $p$ and $q$.

\begin{definition}\label{def:morphism-comprehension-object}
A \emph{morphism of comprehension objects} $f$ from $p$ to $q$ consists of constants
\begin{align*}
f : q \vDash p \\
\alpha : p, x : \El{p}{\alpha} \yields f_1(x) : \El{q}{\TrPlus{f}{\alpha}} \\
\fone{\alpha} : f_1(\One_\alpha) \vDash_{\El{q}{\TrPlus{f}{\alpha}}} \One_{\TrPlus{f}{\alpha}} \\
\foneinv{\alpha} : \One_{\TrPlus{f}{\alpha}} \vDash_{\El{q}{\TrPlus{f}{\alpha}}} f_1(\One_\alpha)
\end{align*}
such that
\begin{align}
\text{$\fone{\alpha}$ and \foneinv{\alpha} are inverse}
\end{align}
\end{definition}

%\mvrnote{Another option is a mode type morphism $q \vDash p$ rather than a term $p \yields f : q$, and then using the $s$-type. (The direction switch is a little confusing). In real cohesive HoTT, however, $\flat$ has only propositional eta, suggesting that using a term is the right thing.}

\section{Object Languages}

\subsection{MLTT via Explicit Substitutions}
\newcommand{\qyields}{\Vdash} \newcommand{\varsof}[1]{{#1}^\dagger}
\newcommand{\upstairs}[1]{\overline{#1}}
\newcommand{\downstairs}[1]{\underline{#1}}
\newcommand{\asdep}[1]{{#1}_p}
\newcommand\proj[1]{\ensuremath{\mathsf{proj}_{#1}}}
\newcommand\qvar[1]{\ensuremath{\mathsf{var}_{#1}}}
\newcommand\outof[1]{\ensuremath{\mathsf{outof}_{#1}}}
\newcommand\into[1]{\ensuremath{\mathsf{into}_{#1}}}

In this section we interpret all the rules and equations of the presentation of type theory given in \cite{altenkirchkaposi16qit} via QITs. 

For consistency we switch from the Agda-style syntax used in the paper to ordinary inference rules. To distinguish judgements in this type theory from the ones in the framework I will use $\qyields$ as the turnstile. \mvrnote{Or some other symbol...} We have the following judgements:
\begin{mathpar}
\qyields \Gamma \CTX \and \Gamma \qyields A \TYPE \and \Gamma \qyields a : A \and \Gamma \qyields \Theta : \Delta 
\end{mathpar}
with the inference rules given in Figure~\ref{fig:qit-rules}. Note that some equations require earlier equations to hold in order to typecheck.

\begin{figure}
\begin{mathpar}
\inferrule*[left=ctx-empty]{~}{\cdot \CTX} \and
\inferrule*[left=ctx-ext]{\qyields \Gamma \CTX \and \Gamma \qyields A \TYPE}{\qyields \Gamma, A \CTX} \\
\inferrule*[left=type-sub]{\Delta \qyields A \TYPE \and \Gamma \qyields \Theta : \Delta}{\Gamma \qyields A[\Theta] \TYPE} \and
\inferrule*[left=term-sub]{\Delta \qyields a : A  \and \Gamma \qyields \Theta : \Delta}{\Gamma \qyields a[\Theta] : A[\Theta]} 
\\
\inferrule*[left=sub-empty]{~}{\Gamma \qyields \epsilon_\Gamma : \cdot} \and
\inferrule*[left=sub-ext]{\Gamma \qyields \Theta : \Delta \and \Gamma \qyields a : A[\Theta]}{\Gamma \qyields (\Theta, a) : \Delta, A} \\
\inferrule*[left=sub-id]{~}{\Gamma \qyields \id_\Gamma : \Gamma} \and
\inferrule*[left=sub-comp]{\Gamma \qyields \Theta : \Delta \and \Delta \qyields \kappa : \Lambda}{\Gamma \qyields \Theta ; \kappa : \Lambda} \\
\inferrule*[left=sub-proj]{~}{\Gamma, A \qyields \proj{\Gamma,A} : \Gamma} \and 
\inferrule*[left=var]{~}{\Gamma, A \qyields \qvar{\Gamma,A} : A[\proj{\Gamma,A}]} 
\end{mathpar}

\begin{align}
A[\id] &\equiv A \\
A[\Theta ; \kappa] &\equiv A[\kappa][\Theta] \\
\nonumber\\
a[\id] &\equiv a \\
a[\Theta ; \kappa] &\equiv a[\kappa][\Theta] \\
\nonumber\\
\id ; \Theta &\equiv \Theta \\
\Theta ; \id &\equiv \Theta \\
(\Theta; \kappa) ; \rho &\equiv \Theta ; (\kappa ; \rho) \\
\nonumber\\
\Theta ; (\kappa , a) &\equiv (\Theta ; \kappa) , a[\Theta] \\ 
(\Theta, a);\proj{\Gamma,A} &\equiv \Theta \\
\qvar{\Delta,A}[\Theta, a] &\equiv a \\
(\proj{\Gamma,A}, \qvar{\Gamma,A}) &\equiv \id_{\Gamma, A} \\
\Theta &\equiv \epsilon_\Gamma && \text{for } \Gamma \qyields \Theta : \cdot
\end{align}
\caption{Rules of MLTT via Explicit Substitutions}\label{fig:qit-rules}
\end{figure}

We can derive some useful operations that we will use later when defining $\Pi$ and $\Sigma$ types.
\begin{mathpar}
\inferrule*[left=derivable]{\Gamma \qyields \Theta : \Delta \and \Delta \qyields A \TYPE}{\Gamma, A[\Theta] \qyields \Theta \uparrow A : \Delta, A} \and
\inferrule*[left=derivable]{\Gamma \qyields a : A}{\Gamma \qyields \hat{a} : \Gamma, A}
\end{mathpar}
defined by:
\begin{align*}
\Theta \uparrow A &:\equiv (\proj{\Gamma, A[\Theta]}; \Theta) , \qvar{\Gamma, A[\Theta]} \\
\hat{a} &:\equiv \id_{\Gamma} , a
\end{align*}

Our goal is to interpret these rules in the framework. The mode theory will contain a comprehension object $p$, and the basic idea for representing the judgements of MLTT is as follows:

\begin{itemize}
\item For each object-language context $\Gamma$, there is a corresponding upstairs framework type $\upstairs{\Gamma}$ with mode $p$.

\item A type $\Gamma \qyields A \TYPE$ is represented by $g : \upstairs{\Gamma} \yields_{\El{p}{g}} \upstairs{A} \TYPE$.
  
\item A term $\Gamma \qyields a : A$ is represented by $g : \upstairs{\Gamma} \yields_{\One_g} \upstairs{a} : \upstairs{A}$.

\item A substitution $\Gamma \qyields \Theta : \Delta$ is represented by a term $g : \upstairs{\Gamma} \yields_g \upstairs{\Theta} : \upstairs{\Delta}$.
\end{itemize}

\subsubsection{Structural Rules}

\begin{theorem}
The judgements and structural rules of MLTT can be interpreted in any comprehension object (Definition \ref{def:comprehension-object})
\end{theorem}

\begin{enumerate}
\item[\textsc{ctx-empty}] Define $\upstairs{(\cdot)} :\equiv \St{\psi}{1}$.
\item[\textsc{ctx-ext}] Given $\alpha : \upstairs{\Gamma}
  \yields_{\El{p}{g}} \upstairs{A} \TYPE$, define $\upstairs{\Gamma, A}
  :\equiv \St{\chi}{\telety{\alpha}{\upstairs{\Gamma}}{\upstairs{A}}}$.
  Since $\cdot \vdash_{\Sigma \alpha:p.\El{p}{\alpha}} (\alpha :
  \upstairs{\Gamma}, \upstairs{A})$, and $\TypeTwo{\cdot}{\chi}{p}{\Sigma
    \alpha:p.\El{p}{\alpha}}$, this is a type of mode $p$.
  
\item[\textsc{type-sub}] Given $g : \upstairs{\Gamma} \yields_g \upstairs{\Theta} : \upstairs{\Delta}$ and $\alpha : \upstairs{\Delta} \yields_{\El{p}{\alpha}} \upstairs{A} \TYPE$ we can use framework substitution to form 
  $\upstairs{A[\Theta]} :\equiv \upstairs{A}[\upstairs{\Theta}]$,
  with $g : \upstairs{\Gamma} \yields_g \upstairs{A}[\upstairs{\Theta}] \TYPE$.
\item[\textsc{term-sub}]
  Given $g : \upstairs{\Gamma} \yields_g \upstairs{\Theta} :
  \upstairs{\Delta}$ and
  $\alpha : \upstairs{\Delta} \yields_{\One_{\alpha}} \upstairs{a} : \upstairs{A}$,
  we can use framework substitution to form $g : \Gamma \vdash_{\One_g}
  \upstairs{a[\Theta]} :\equiv \upstairs{a}[\upstairs{\Theta}] : \upstairs{A[\Theta]}$.

\item[\textsc{sub-empty}] For any $\upstairs{\Gamma}$ we have (recall
  that $\emptyset := \TrPlus{\psi}{}$):
\begin{mathpar}
\inferrule*[Left=rewrite]{
\inferrule*[Left=s-intro]{\alpha : \upstairs{\Gamma} \yields_{()} () : 1}{\alpha : \upstairs{\Gamma} \yields_{\emptyset} \StI{\psi}{()} : \upstairs{(.)}}
}{\alpha : \upstairs{\Gamma} \yields_{\alpha} \upstairs{\epsilon_\Gamma} :\equiv \rewrite{\eta^\psi_\alpha}{\StI{\psi}{()}} : \upstairs{(.)}}
\end{mathpar}

\item[\textsc{sub-ext}] We have
\begin{align*}
\alpha : \upstairs{\Gamma} &\yields_\alpha \upstairs{\Theta} : \upstairs{\Delta} \\
\alpha : \upstairs{\Gamma} &\yields_{\One_\alpha} \upstairs{a} : \upstairs{A}[\upstairs{\Theta}]
\end{align*}
which are exactly what is needed to form
\begin{align*}
  \inferrule*{\inferrule*{\alpha : \upstairs{\Gamma} \vdash_{(\alpha,\One_\alpha)} {(\upstairs{\Theta}, \upstairs{a})} : (\alpha : \upstairs{\Gamma}, \upstairs{A[\Theta]})}
                         {\alpha : \upstairs{\Gamma}
                           \vdash_{\alpha.\One_\alpha \equiv \TrPlus{\chi}{(\alpha,\One_\alpha)}} {\StI{\chi}{(\upstairs{\Theta}, \upstairs{a})}} : \St{\chi}{(\alpha : \upstairs{\Gamma}, \upstairs{A[\Theta]})}}
             }
             {\alpha : \upstairs{\Gamma} \vdash_{\alpha}
               \upstairs{\Theta, a} :\equiv
               \rewrite{\eta^\chi_{\alpha}}{\StI{\chi}{(\upstairs{\Theta},
                   \upstairs{a})}} : \upstairs{\Delta,A}}
\end{align*}

\item[\textsc{sub-id}] We have $\alpha : \upstairs{\Gamma} \vdash_\alpha \alpha : \upstairs{\Gamma}$

\item[\textsc{sub-comp}]
Given $g : \upstairs{\Gamma} \vdash_{g} \upstairs{\Theta} : \upstairs{\Delta}$
and
$d : \upstairs{\Delta} \vdash_{d} \upstairs{\kappa} :
\upstairs{\Lambda}$
we have
$g : \upstairs{\Gamma} \vdash_{g} \upstairs{\theta;\kappa} :\equiv \upstairs{\kappa}[\upstairs{\Theta}/d] : \upstairs{\Lambda}$
  
\item[\textsc{sub-proj}] We are trying to construct a term
\begin{align*}
\beta : \St{\chi}{\telety{\alpha}{\upstairs{\Gamma}}{\upstairs{A}}} \yields_\beta \upstairs{\proj{\Gamma, A}} : \upstairs{\Gamma}
\end{align*}
Use:
\begin{align*}
\upstairs{\proj{\Gamma, A}} :\equiv \StE{\chi}{\beta}{w}{\rewrite{\pi^{\fst w}_{\snd w}}{\fst w}}
\end{align*}

This type checks because
\[
\inferrule*{
            \inferrule*{{w : {\telety{\alpha}{\upstairs{\Gamma}}{\upstairs{A}}} } \vdash_{\fst w} {\fst w} : {\upstairs{\Gamma}}}
                       {w : {\telety{\alpha}{\upstairs{\Gamma}}{\upstairs{A}}} \vdash_{\fst w.\snd w \equiv \TrPlus{\chi}{w}} {\rewrite{\pi^{\fst w}_{\snd w}}{\fst w}} : \upstairs{\Gamma}}
           }
           {
             \beta : \St{\chi}{\telety{\alpha}{\upstairs{\Gamma}}{\upstairs{A}}} \yields_\beta \StE{\chi}{\beta}{w}{\rewrite{\pi^{\fst w}_{\snd w}}{\fst w}} : \upstairs{\Gamma}
           }
\]

\item[\textsc{var}] We are a building a term
\begin{align*}
\beta : \St{\chi}{\telety{\alpha}{\upstairs{\Gamma}}{\upstairs{A}}} \yields \upstairs{\qvar{\Gamma, A}} : \upstairs{A}[\upstairs{\proj{\Gamma, A}}]
\end{align*}
Note that:
\begin{align*}
&\upstairs{A}[\upstairs{\proj{\Gamma, A}}] \\
&\equiv \upstairs{A}[\StE{\chi}{\beta}{w}{\rewrite{\pi^{\fst w}_{\snd w}}{\fst w}}/\alpha] \\
&\equiv \StE{\chi}{\beta}{w}{\upstairs{A}[\rewrite{\pi^{\fst w}_{\snd w}}{\fst w}/\alpha]} & (\eta)\\
&\equiv \StE{\chi}{\beta}{w}{\St{\ApEl{p}{\pi^{\fst w}_{\snd w}}}{\upstairs{A}[\fst w/\alpha]}} & (\ref{eqn:stype-subst}) \\
\end{align*}
So we can build $\upstairs{\qvar{\Gamma, A}}$ by:
\begin{align*}
\upstairs{\qvar{\Gamma, A}} :\equiv \StE{\chi}{\beta}{w}{\rewrite{\var{\snd w}}{\StI{\ApEl{p}{\pi^{\fst w}_{\snd w}}}{\snd w}}}
\end{align*}
because
\[
\inferrule*{
  \inferrule*{ {w : {\telety{\alpha}{\upstairs{\Gamma}}{\upstairs{A}}} \vdash_{{\snd{w}}} {{{\snd w}}} : {{\upstairs{A}[\fst w/\alpha]}}}    }
      {\inferrule*{ {w : {\telety{\alpha}{\upstairs{\Gamma}}{\upstairs{A}}} \vdash_{\TrPlus{\ApEl{p}{\pi^{\fst w}_{\snd w}}}{\snd{w}}} {{\StI{\ApEl{p}{\pi^{\fst w}_{\snd w}}}{\snd w}}} : {\St{\ApEl{p}{\pi^{\fst w}_{\snd w}}}{\upstairs{A}[\fst w/\alpha]}}} }
                       {w : {\telety{\alpha}{\upstairs{\Gamma}}{\upstairs{A}}} \vdash_{\One_{\fst w.\snd w \equiv \TrPlus{\chi}{w}}} {\rewrite{\qvar{\snd w}}{\StI{\ApEl{p}{\pi^{\fst w}_{\snd w}}}{\snd w}}} : {\St{\ApEl{p}{\pi^{\fst w}_{\snd w}}}{\upstairs{A}[\fst w/\alpha]}}}}
           }
           {
             \beta : \St{\chi}{\telety{\alpha}{\upstairs{\Gamma}}{\upstairs{A}}} \yields_{\One_\beta} \StE{\chi}{\beta}{w}{\rewrite{\qvar{\snd w}}{\StI{\ApEl{p}{\pi^{\fst w}_{\snd w}}}{\snd w}}} : \StE{\chi}{\beta}{w}{\St{\ApEl{p}{\pi^{\fst w}_{\snd w}}}{\upstairs{A}[\fst w/\alpha]}}
           }
\]
\end{enumerate}

Now we check that these translations satisfy the required equations.

The first three blocks of equations follow directly from the analogous
properties in the framework, because framework substitution is
functorial and assoiative/unital, and object-language substitution is
translated to single-variable framework substitution into both types and
terms, and identity and composition of substitutions are translated to
identity and composition of open terms.

The remaining equations are:

\begin{enumerate}[style = multiline, labelwidth = 80pt]
\item[{$\Theta ; (\kappa , a) \equiv (\Theta ; \kappa) , a[\Theta]$}] 
\begin{align*}
\upstairs{\Theta ; (\kappa , a)}
&\equiv \upstairs{(\kappa, a)}[\Theta/\alpha] \\
&\equiv \rewrite{\eta^\chi_{\alpha}}{\StI{\chi}{(\upstairs{\kappa}, \upstairs{a})}}[\Theta/\alpha] \\
&\equiv \rewrite{\eta^\chi_{\alpha}}{\StI{\chi}{(\upstairs{\kappa}[\Theta/\alpha], \upstairs{a}[\Theta/\alpha])}} \\
&\equiv \upstairs{(\Theta ; \kappa) , a[\Theta]}
\end{align*}
\item[{$(\Theta, a);\proj{\Gamma,A} \equiv \Theta$}]
\begin{align*}
\upstairs{(\Theta, a);\proj{\Gamma,A}}
&\equiv \upstairs{\proj{\Gamma,A}}[\upstairs{(\Theta, a)}/\beta] \\
&\equiv (\StE{\chi}{\beta}{w}{\rewrite{\pi^{\fst w}_{\snd w}}{\fst w}})[\rewrite{\eta^\chi_{\alpha}}{\StI{\chi}{(\upstairs{\Theta}, \upstairs{a})}}/\beta] \\
&\equiv \StE{\chi}{\rewrite{\eta^\chi_{\alpha}}{\StI{\chi}{(\upstairs{\Theta}, \upstairs{a})}}}{w}{\rewrite{\pi^{\fst w}_{\snd w}}{\fst w}} \\
&\equiv \rewrite{\eta^\chi_{\alpha}}{\StE{\chi}{\StI{\chi}{(\upstairs{\Theta}, \upstairs{a})}}{w}{\rewrite{\pi^{\fst w}_{\snd w}}{\fst w}}} \\
&\equiv \rewrite{\eta^\chi_{\alpha}}{\rewrite{\pi^{\fst w}_{\snd w}}{\fst w}[(\upstairs{\Theta}, \upstairs{a})/w]} \\
&\equiv \rewrite{\eta^\chi_{\alpha}}{\rewrite{\pi^\alpha_{\One_\alpha}}{\upstairs{\Theta}}} \\
&\equiv\upstairs{\Theta} 
\end{align*}
\item[{$\qvar{\Delta,A}[\Theta, a] \equiv a$}] 
\begin{align*}
\upstairs{\qvar{\Delta,A}[\Theta, a]} 
&\equiv \upstairs{\qvar{\Delta,A}}[\upstairs{\Theta, a}/\beta] \\
&\equiv (\StE{\chi}{\beta}{w}{\rewrite{\var{\snd w}}{\StI{\ApEl{p}{\pi^{\fst w}_{\snd w}}}{\snd w}}})[\rewrite{\eta^\chi_{\alpha}}{\StI{\chi}{(\upstairs{\Theta}, \upstairs{a})}}/\beta] \\
&\equiv \StE{\chi}{\rewrite{\eta^\chi_{\alpha}}{\StI{\chi}{(\upstairs{\Theta}, \upstairs{a})}}}{w}{\rewrite{\var{\snd w}}{\StI{\ApEl{p}{\pi^{\fst w}_{\snd w}}}{\snd w}}} \\
&\equiv \rewrite{\One_{\langle \eta^\chi_\alpha \rangle}}{\StI{\ApEl{p}{\eta^\chi_\alpha}}{\StE{\chi}{\StI{\chi}{(\upstairs{\Theta}, \upstairs{a})}}{w}{\rewrite{\var{\snd w}}{\StI{\ApEl{p}{\pi^{\fst w}_{\snd w}}}{\snd w}}}}} \\
&\equiv \rewrite{\One_{\langle \eta^\chi_\alpha \rangle}}{\StI{\ApEl{p}{\eta^\chi_\alpha}}{\rewrite{\var{\snd w}}{\StI{\ApEl{p}{\pi^{\fst w}_{\snd w}}}{\snd w}}[(\upstairs{\Theta}, \upstairs{a})/w]}} \\
&\equiv \rewrite{\One_{\langle \eta^\chi_\alpha \rangle}}{\StI{\ApEl{p}{\eta^\chi_\alpha}}{\rewrite{\var{\One_\alpha}}{\StI{\ApEl{p}{\pi^{\alpha}_{\One_\alpha}}}{\upstairs{a}}}}} \\
&\equiv \rewrite{\One_{\langle \eta^\chi_\alpha \rangle}}{\rewrite{\ApPlus{\ApEl{p}{\eta^\chi_\alpha}}{\qvar{\One_\alpha}}}{\StI{\ApEl{p}{\eta^\chi_\alpha}}{\StI{\ApEl{p}{\pi^{\alpha}_{\One_\alpha}}}{\upstairs{a}}}}} \\
&\equiv \StI{\ApEl{p}{\eta^\chi_\alpha};\ApEl{p}{\pi^{\alpha}_{\One_\alpha}}}{\upstairs{a}} \\
&\equiv \upstairs{a}
\end{align*}
Where we have used that
\begin{align*}
\ApOne{\eta^\chi_\alpha};\ApPlus{\ApEl{p}{\eta^\chi_\alpha}}{\qvar{\One_\alpha}} \equiv \id_{\One_\alpha}
\end{align*}
by the triangle identity for $\chi$.
\item[{$(\proj{\Gamma,A}, \qvar{\Gamma,A}) \equiv \id_{\Gamma, A}$}] Using eta for $\St{\chi}{}$ and a triangle identity for $\chi$:
\begin{align*}
\upstairs{(\proj{\Gamma,A}, \qvar{\Gamma,A})}
&\equiv \rewrite{\eta^\chi_{\beta}}{\StI{\chi}{(\upstairs{\proj{\Gamma,A}}, \upstairs{\qvar{\Gamma,A}})}} \\
&\equiv \rewrite{\eta^\chi_{\beta}}{\StI{\chi}{(\StE{\chi}{\beta}{w}{\rewrite{\pi^{\fst w}_{\snd w}}{\fst w}}, \StE{\chi}{\beta}{w}{\rewrite{\var{\snd w}}{\StI{\ApEl{p}{\pi^{\fst w}_{\snd w}}}{\snd w}}})}} \\
&\equiv \StE{\chi}{\beta}{w}{\rewrite{\eta^\chi_{\TrPlus{\chi}{w}}}{\StI{\chi}{(\rewrite{\pi^{\fst w}_{\snd w}}{\fst w}, \rewrite{\var{\snd w}}{\StI{\ApEl{p}{\pi^{\fst w}_{\snd w}}}{\snd w}})}}} \\
&\equiv \StE{\chi}{\beta}{w}{\rewrite{\eta^\chi_{\TrPlus{\chi}{w}}}{\StI{\chi}{\rewrite{(\pi^{\fst w}_{\snd w},\qvar{\snd w})}{(\fst w, \snd w)}}}} \\
&\equiv \StE{\chi}{\beta}{w}{\rewrite{\eta^\chi_{\TrPlus{\chi}{w}}}{\StI{\chi}{\rewrite{\varepsilon^\chi_w}{w}}}} \\
&\equiv \StE{\chi}{\beta}{w}{\rewrite{\eta^\chi_{\TrPlus{\chi}{w}}}{\rewrite{\ApPlus{\chi}{\varepsilon^\chi_w}}{\StI{\chi}{w}}}} \\
&\equiv \StE{\chi}{\beta}{w}{\StI{\chi}{w}} \\
&\equiv \beta \\
&\equiv \upstairs{\id_{\Gamma, A}}
\end{align*}

\item[$\Theta \equiv \epsilon_\Gamma$] Using eta expansion for $\St{\psi}{}$ once in each direction:
\begin{align*}
\upstairs{\Theta} 
&\equiv \StE{\psi}{\upstairs{\Theta}}{x}{\StI{\psi}{x}} \\
&\equiv \StE{\psi}{\upstairs{\Theta}}{x}{\StI{\psi}{()}} \\
&\equiv \StE{\psi}{\upstairs{\Theta}}{x}{\rewrite{\eta^\psi_{\emptyset}}{\StI{\psi}{()}}} \\
&\equiv \StE{\psi}{\upstairs{\Theta}}{x}{\rewrite{\eta^\psi_{\TrPlus{\psi}{()}}}{\StI{\psi}{()}}} \\
&\equiv \StE{\psi}{\upstairs{\Theta}}{x}{\rewrite{\eta^\psi_{\TrPlus{\psi}{x}}}{\StI{\psi}{()}}} \\
&\equiv \rewrite{\eta^\psi_{\alpha}}{\StI{\psi}{()}} \\
&\equiv \upstairs{\epsilon_\Gamma}
\end{align*}
\end{enumerate}

\begin{lemma}\label{lem:into-identities}
Let 
\begin{align*}
\alpha : \upstairs{\Gamma}, x : \upstairs{A} &\yields_{\alpha.x} \into{\Gamma, A} :\equiv \StI{\chi}{(\alpha, x)} : \upstairs{\Gamma, A}
\end{align*}
There are identities:
\begin{align*}
\upstairs{\proj{\Gamma, A}}[\into{\Gamma, A}/\beta] &\equiv \rewrite{\pi^\alpha_x}{\alpha} \\
\upstairs{\qvar{\Gamma, A}}[\into{\Gamma, A}/\beta] &\equiv \rewrite{\var{x}}{\StI{\ApEl{p}{\pi^{\alpha}_x}}{x}} \\
\upstairs{\Theta \uparrow A}[\into{\Gamma, A[\Theta]}/\beta] &\equiv \into{\Delta, A}[\upstairs{\Theta}/\alpha, x / x]
\end{align*}
where $\Gamma \qyields \Theta : \Delta$ is any substitution.
\end{lemma}
\begin{proof}
These are straightforward to check:
\begin{align*}
\upstairs{\proj{\Gamma, A}}[\into{\Gamma, A}/\beta] 
&\equiv \StE{\chi}{\beta}{w}{\rewrite{\pi^{\fst w}_{\snd w}}{\fst w}}[\StI{\chi}{(\alpha, x)}/\beta] \\
&\equiv \StE{\chi}{\StI{\chi}{(\alpha, x)}}{w}{\rewrite{\pi^{\fst w}_{\snd w}}{\fst w}} \\
&\equiv \rewrite{\pi^\alpha_x}{\alpha} \\
\upstairs{\qvar{\Gamma, A}}[\into{\Gamma, A}/\beta]
&\equiv  \StE{\chi}{\beta}{w}{\rewrite{\var{\snd w}}{\StI{\ApEl{p}{\pi^{\fst w}_{\snd w}}}{\snd w}}}[\StI{\chi}{(\alpha, x)}/\beta]  \\
&\equiv \StE{\chi}{\StI{\chi}{(\alpha, x)}}{w}{\rewrite{\var{\snd w}}{\StI{\ApEl{p}{\pi^{\fst w}_{\snd w}}}{\snd w}}} \\
&\equiv \rewrite{\var{x}}{\StI{\ApEl{p}{\pi^{\alpha}_x}}{x}} \\
\end{align*}
For the last:
\begin{align*}
&\upstairs{\Theta \uparrow A}[\into{\Gamma, A[\Theta]}/\beta] \\
&\equiv \upstairs{(\proj{\Gamma, A[\Theta]}; \Theta) , \qvar{\Gamma, A[\Theta]}} [\into{\Gamma, A[\Theta]}/\beta] \\
&\equiv \rewrite{\eta^\chi_\beta}{\StI{\chi}{(\upstairs{\proj{\Gamma, A[\Theta]}; \Theta}, \upstairs{\qvar{\Gamma, A[\Theta]}})}}[\into{\Gamma, A[\Theta]}/\beta] \\
&\equiv \rewrite{\eta^\chi_\beta}{\StI{\chi}{(\upstairs{\Theta}[\upstairs{\proj{\Gamma, A[\Theta]}}/\alpha], \upstairs{\qvar{\Gamma, A[\Theta]}})}}[\into{\Gamma, A[\Theta]}/\beta] \\
&\equiv \rewrite{\eta^\chi_\beta}{\StI{\chi}{(\upstairs{\Theta}, x)}[\upstairs{\proj{\Gamma, A[\Theta]}}/\alpha, \upstairs{\qvar{\Gamma, A[\Theta]}}/x]}[\into{\Gamma, A[\Theta]}/\beta] \\
&\equiv \rewrite{\eta^\chi_{\alpha.x}}{\StI{\chi}{(\upstairs{\Theta}, x)}[\upstairs{\proj{\Gamma, A[\Theta]}}/\alpha, \upstairs{\qvar{\Gamma, A[\Theta]}}/x][\into{\Gamma, A[\Theta]}/\beta]} \\
&\equiv \rewrite{\eta^\chi_{\alpha.x}}{\StI{\chi}{(\upstairs{\Theta}, x)}[\rewrite{\varepsilon^\chi_{(\alpha,x)}}{(\alpha,x)}/(\alpha,x)]} \\
&\equiv \rewrite{\eta^\chi_{\alpha.x}}{\rewrite{\TrPlus{\chi}{\varepsilon^\chi_{(\alpha,x)}}}{\StI{\chi}{(\upstairs{\Theta}, x)}}} \\
&\equiv \StI{\chi}{(\upstairs{\Theta}, x)} \\
&\equiv \into{\Delta, A}[\upstairs{\Theta}/\alpha, x / x]
\end{align*}
\end{proof}

We can also define a 2-variable version of the above, which we will need when translating $\Sigma$-types. \mvrnote{Worth figuring out the n-variable case?}
\begin{lemma}\label{lem:double-into}
Suppose $p$ supports the unit type. \mvrnote{(I don't see how to avoid this)} There are terms:
\begin{align*}
\alpha : \upstairs{\Gamma}, x : \upstairs{A}, y : \upstairs{B}[\into{\Gamma, A} /\beta] &\yields_{\alpha.x.y} \into{\Gamma, A, B}[\into{\Gamma, A}/\beta, y/y] \\&: \upstairs{\Gamma,A,B} \\
\delta :\upstairs{\Gamma, A, B} &\yields_{\delta, \One_\delta, \One_{\delta.\One_\delta}} (\upstairs{\proj{\Gamma, A}}[\upstairs{\proj{\Gamma, A, B}}/\beta], \upstairs{\qvar{\Gamma, A}}[\upstairs{\proj{\Gamma, A, B}}/\beta], \upstairs{\qvar{\Gamma, A, B}}[\rewrite{\pi^\delta_{\One_\delta}}{\delta}/\delta]) \\&: (\alpha : \upstairs{\Gamma}, x : \upstairs{A},  \upstairs{B}[\into{\Gamma, A}/\beta])
\end{align*}
That compose to give:
\begin{align*}
&(\upstairs{\proj{\Gamma, A}}[\upstairs{\proj{\Gamma, A, B}}/\beta], \upstairs{\qvar{\Gamma, A}}[\upstairs{\proj{\Gamma, A, B}}/\beta], \upstairs{\qvar{\Gamma, A, B}}[\rewrite{\pi^\delta_{\One_\delta}}{\delta}/\delta])[\into{\Gamma, A, B}[\into{\Gamma, A}/\beta, y/y]/\delta] \\
&\equiv \rewrite{((\pi^{\alpha.x}_y;\pi^\alpha_x), (\ApOne{\pi^{\alpha.x}_y};\ApPlus{\pi^{\alpha.x}_y}{\var{x}}, \ApOne{\pi^{\alpha.x.y}_{\One_{\alpha.x.y}}};\ApPlus{\ApEl{p}{\pi^{\alpha.x.y}_{\One_{\alpha.x.y}}}}{\var{y}}))}{(\alpha, x,y)}
\end{align*}
in one direction and
\begin{align*}
&\into{\Gamma, A, B}[\into{\Gamma, A}/\beta, y/y][\upstairs{\proj{\Gamma, A, B};\proj{\Gamma, A}}/\alpha, \upstairs{\qvar{\Gamma, A}}[\upstairs{\proj{\Gamma, A, B}}/\beta]/x, \upstairs{\qvar{\Gamma, A, B}}[\rewrite{\pi^\delta_{\One_\delta}}{\delta}/\delta]/y] \\
&\equiv \rewrite{\pi^{\delta.\One_\delta}_{\One_{\delta.\One_\delta}};\pi^{\delta}_{\One_{\delta}}}{\delta}
\end{align*}
\end{lemma}
\begin{proof}
First let us check that the latter term has the right type. The typing of the first two components is clear. For the third, the type of 
\begin{align*}
\upstairs{\qvar{\Gamma, A, B}}[\rewrite{\pi^\delta_{\One_\delta}}{\delta}/\delta] : \upstairs{B[\proj{\Gamma, A, B}]}[\rewrite{\pi^\delta_{\One_\delta}}{\delta}/\delta]
\end{align*}
is equal to
\begin{align*}
&\upstairs{B[\proj{\Gamma, A, B}]}[\rewrite{\pi^\delta_{\One_\delta}}{\delta}/\delta] \\
&\equiv \St{\ApEl{p}{\pi^\delta_{\One_\delta}}}{\upstairs{B}[\upstairs{\proj{\Gamma, A, B}}/\beta]} \\
&\equiv \St{\ApEl{p}{\pi^\beta_{\One_\beta}}}{\upstairs{B}}[\upstairs{\proj{\Gamma, A, B}}/\beta] \\
&\equiv \St{\ApEl{p}{\pi^\beta_{\One_\beta}}}{\upstairs{B}}[\rewrite{\eta^\chi_\beta}{\into{\Gamma, A}[\upstairs{\proj{\Gamma, A}}/\alpha, \upstairs{\qvar{\Gamma, A}}/x]/\beta]}[\upstairs{\proj{\Gamma, A, B}}/\beta] \\
&\equiv \St{\ApEl{p}{\eta^\chi_\beta}}{\St{\ApEl{p}{\pi^\beta_{\One_\beta}}}{\upstairs{B}}[\into{\Gamma, A}[\upstairs{\proj{\Gamma, A}}/\alpha, \upstairs{\qvar{\Gamma, A}}/x]/\beta]}[\upstairs{\proj{\Gamma, A, B}}/\beta] \\
&\equiv \St{\ApEl{p}{\eta^\chi_\beta}}{\St{\ApEl{p}{\pi^\beta_{\One_\beta}}}{\upstairs{B}}}[\into{\Gamma, A}[\upstairs{\proj{\Gamma, A}}/\alpha, \upstairs{\qvar{\Gamma, A}}/x]/\beta][\upstairs{\proj{\Gamma, A, B}}/\beta] \\
&\equiv \upstairs{B}[\into{\Gamma, A}[\upstairs{\proj{\Gamma, A}}/\alpha, \upstairs{\qvar{\Gamma, A}}/x]/\beta][\upstairs{\proj{\Gamma, A, B}}/\beta] \\
&\equiv \upstairs{B}[\into{\Gamma, A}/\beta][\upstairs{\proj{\Gamma, A}}/\alpha, \upstairs{\qvar{\Gamma, A}}/x][\upstairs{\proj{\Gamma, A, B}}/\beta] \\
&\equiv \upstairs{B}[\into{\Gamma, A}/\beta][\upstairs{\proj{\Gamma, A}}[\upstairs{\proj{\Gamma, A, B}}/\beta]/\alpha, \upstairs{\qvar{\Gamma, A}}[\upstairs{\proj{\Gamma, A, B}}/\beta]/x]
\end{align*}
When we compose these terms, going componentwise we find:
\begin{align*}
\upstairs{\proj{\Gamma, A}}[\upstairs{\proj{\Gamma, A, B}}/\beta][\into{\Gamma, A, B}[\into{\Gamma, A}/\beta, y/y]/\delta] 
&\equiv \upstairs{\proj{\Gamma, A}}[\upstairs{\proj{\Gamma, A, B}}/\beta][\into{\Gamma, A, B}/\delta][\into{\Gamma, A}/\beta, y/y] \\
&\equiv \upstairs{\proj{\Gamma, A}}[\rewrite{\pi^\beta_y}{\beta}/\beta][\into{\Gamma, A}/\beta, y/y] \\
&\equiv \rewrite{\pi^\beta_y}{\upstairs{\proj{\Gamma, A}}}[\into{\Gamma, A}/\beta, y/y] \\
&\equiv \rewrite{\pi^{\alpha.x}_y}{\upstairs{\proj{\Gamma, A}}[\into{\Gamma, A}/\beta]} \\
&\equiv \rewrite{\pi^{\alpha.x}_y}{\rewrite{\pi^\alpha_x}{\alpha}} \\
&\equiv \rewrite{\pi^{\alpha.x}_y;\pi^\alpha_x}{\alpha} \\
%%%
\upstairs{\qvar{\Gamma, A}}[\upstairs{\proj{\Gamma, A, B}}/\beta][\into{\Gamma, A, B}[\into{\Gamma, A}/\beta, y/y]/\delta]
&\equiv \upstairs{\qvar{\Gamma, A}}[\upstairs{\proj{\Gamma, A, B}}/\beta][\into{\Gamma, A, B}/\delta][\into{\Gamma, A}/\beta, y/y] \\
&\equiv \upstairs{\qvar{\Gamma, A}}[\rewrite{\pi^\beta_y}{\beta}/\beta][\into{\Gamma, A}/\beta, y/y] \\
&\equiv \rewrite{\ApOne{\pi^\beta_y}}{\StI{\ApEl{p}{\pi^\beta_y}}{\upstairs{\qvar{\Gamma, A}}}}[\into{\Gamma, A}/\beta, y/y] \\
&\equiv \rewrite{\ApOne{\pi^{\alpha.x}_y}}{\StI{\ApEl{p}{\pi^{\alpha.x}_y}}{\upstairs{\qvar{\Gamma, A}}[\into{\Gamma, A}/\beta]}} \\
&\equiv \rewrite{\ApOne{\pi^{\alpha.x}_y}}{\StI{\ApEl{p}{\pi^{\alpha.x}_y}}{\rewrite{\var{x}}{\StI{\ApEl{p}{\pi^{\alpha}_x}}{x}}}} \\
&\equiv \rewrite{\ApOne{\pi^{\alpha.x}_y};\ApPlus{\pi^{\alpha.x}_y}{\var{x}}}{\StI{\ApEl{p}{\pi^{\alpha.x}_y;\pi^{\alpha}_x}}{x}} \\
%%%
\upstairs{\qvar{\Gamma, A, B}}[\rewrite{\pi^\delta_{\One_\delta}}{\delta}/\delta][\into{\Gamma, A, B}[\into{\Gamma, A}/\beta, y/y]/\delta]
&\equiv \rewrite{\ApOne{\pi^\delta_{\One_\delta}}}{\StI{\ApEl{p}{\pi^\delta_{\One_\delta}}}{\upstairs{\qvar{\Gamma, A, B}}}}[\into{\Gamma, A, B}[\into{\Gamma, A}/\beta, y/y]/\delta] \\
&\equiv \rewrite{\ApOne{\pi^{\alpha.x.y}_{\One_{\alpha.x.y}}}}{\StI{\ApEl{p}{\pi^{\alpha.x.y}_{\One_{\alpha.x.y}}}}{\upstairs{\qvar{\Gamma, A, B}}[\into{\Gamma, A, B}[\into{\Gamma, A}/\beta, y/y]/\delta]}} \\
&\equiv \rewrite{\ApOne{\pi^{\alpha.x.y}_{\One_{\alpha.x.y}}}}{\StI{\ApEl{p}{\pi^{\alpha.x.y}_{\One_{\alpha.x.y}}}}{\upstairs{\qvar{\Gamma, A, B}}[\into{\Gamma, A, B}/\delta][\into{\Gamma, A}/\beta, y/y]}} \\
&\equiv \rewrite{\ApOne{\pi^{\alpha.x.y}_{\One_{\alpha.x.y}}}}{\StI{\ApEl{p}{\pi^{\alpha.x.y}_{\One_{\alpha.x.y}}}}{\rewrite{\var{y}}{\StI{\ApEl{p}{\pi^{\beta}_y}}{y}}[\into{\Gamma, A}/\beta, y/y]}} \\
&\equiv \rewrite{\ApOne{\pi^{\alpha.x.y}_{\One_{\alpha.x.y}}}}{\StI{\ApEl{p}{\pi^{\alpha.x.y}_{\One_{\alpha.x.y}}}}{\rewrite{\var{y}}{\StI{\ApEl{p}{\pi^{\alpha.x}_y}}{y}}}} \\
&\equiv \rewrite{\ApOne{\pi^{\alpha.x.y}_{\One_{\alpha.x.y}}};\ApPlus{\pi^{\alpha.x.y}_{\One_{\alpha.x.y}}}{\var{y}}}{\StI{\ApEl{p}{\pi^{\alpha.x.y}_{\One_{\alpha.x.y}};\pi^{\alpha.x}_y}}{y}}
\end{align*}
So all together:
\begin{align*}
&(\upstairs{\proj{\Gamma, A}}[\upstairs{\proj{\Gamma, A, B}}/\beta], \upstairs{\qvar{\Gamma, A}}[\upstairs{\proj{\Gamma, A, B}}/\beta], \upstairs{\qvar{\Gamma, A, B}}[\rewrite{\pi^\delta_{\One_\delta}}{\delta}/\delta])[\into{\Gamma, A, B}[\into{\Gamma, A}/\beta, y/y]/\delta] \\
&\equiv (\rewrite{\pi^{\alpha.x}_y;\pi^\alpha_x}{\alpha}, \rewrite{\ApOne{\pi^{\alpha.x}_y};\ApPlus{\pi^{\alpha.x}_y}{\var{x}}}{\StI{\ApEl{p}{\pi^{\alpha.x}_y;\pi^{\alpha}_x}}{x}}, \rewrite{\ApOne{\pi^{\alpha.x.y}_{\One_{\alpha.x.y}}}}{\StI{\ApEl{p}{\pi^{\alpha.x.y}_{\One_{\alpha.x.y}}}}{\rewrite{\var{y}}{\StI{\ApEl{p}{\pi^{\alpha.x}_y}}{y}}}}) \\
&\equiv (\rewrite{((\pi^{\alpha.x}_y;\pi^\alpha_x), (\ApOne{\pi^{\alpha.x}_y};\ApPlus{\pi^{\alpha.x}_y}{\var{x}}))}{(\alpha, x)}, \rewrite{\ApOne{\pi^{\alpha.x.y}_{\One_{\alpha.x.y}}}}{\StI{\ApEl{p}{\pi^{\alpha.x.y}_{\One_{\alpha.x.y}}}}{\rewrite{\var{y}}{\StI{\ApEl{p}{\pi^{\alpha.x}_y}}{y}}}}) \\
&\equiv (\rewrite{((\pi^{\alpha.x}_y;\pi^\alpha_x), (\ApOne{\pi^{\alpha.x}_y};\ApPlus{\pi^{\alpha.x}_y}{\var{x}}))}{(\alpha, x)}, \rewrite{\ApOne{\pi^{\alpha.x.y}_{\One_{\alpha.x.y}}};\ApPlus{\ApEl{p}{\pi^{\alpha.x.y}_{\One_{\alpha.x.y}}}}{\var{y}}}{\StI{\ApEl{p}{\pi^{\alpha.x.y}_{\One_{\alpha.x.y}};\pi^{\alpha.x}_y}}{y}}) \\
&\equiv \rewrite{((\pi^{\alpha.x}_y;\pi^\alpha_x), (\ApOne{\pi^{\alpha.x}_y};\ApPlus{\pi^{\alpha.x}_y}{\var{x}}, \ApOne{\pi^{\alpha.x.y}_{\One_{\alpha.x.y}}};\ApPlus{\ApEl{p}{\pi^{\alpha.x.y}_{\One_{\alpha.x.y}}}}{\var{y}}))}{(\alpha, x,y)}
\end{align*}
In the last step we use that $p$ has unit, to show
\begin{align*}
&(\pi^{\alpha.x}_y;\pi^\alpha_x) \bdot (\ApOne{\pi^{\alpha.x}_y};\ApPlus{\pi^{\alpha.x}_y}{\var{x}}) \\
&\equiv (\pi^{\alpha.x}_y \bdot \ApOne{\pi^{\alpha.x}_y});(\pi^\alpha_x \bdot \var{x}) \\
&\equiv \pi^{\alpha.x.y}_{\One_{\alpha.x.y}};\eta^\chi_{\alpha.x.y};(\pi^{\alpha.x}_y \bdot \ApOne{\pi^{\alpha.x}_y});(\pi^\alpha_x \bdot \var{x}) \\
&\equiv \pi^{\alpha.x.y}_{\One_{\alpha.x.y}};\pi^{\alpha.x}_y;\eta^\chi_{\alpha.x};(\pi^\alpha_x \bdot \var{x}) \\
&\equiv \pi^{\alpha.x.y}_{\One_{\alpha.x.y}};\pi^{\alpha.x}_y
\end{align*}
Composing in the other direction we find:
\begin{align*}
&\into{\Gamma, A, B}[\into{\Gamma, A}/\beta, y/y][\upstairs{\proj{\Gamma, A, B};\proj{\Gamma, A}}/\alpha, \upstairs{\qvar{\Gamma, A}}[\upstairs{\proj{\Gamma, A, B}}/\beta]/x, \upstairs{\qvar{\Gamma, A, B}}[\rewrite{\pi^\delta_{\One_\delta}}{\delta}/\delta]/y] \\
&\equiv \into{\Gamma, A, B}[\into{\Gamma, A}[\upstairs{\proj{\Gamma, A, B};\proj{\Gamma, A}}/\alpha, \upstairs{\qvar{\Gamma, A}}[\upstairs{\proj{\Gamma, A, B}}/\beta]/x] /\beta, \upstairs{\qvar{\Gamma, A, B}}[\rewrite{\pi^\delta_{\One_\delta}}{\delta}/\delta]/y] \\
&\equiv \into{\Gamma, A, B}[\into{\Gamma, A}[\upstairs{\proj{\Gamma, A}}/\alpha, \upstairs{\qvar{\Gamma, A}}/x][\upstairs{\proj{\Gamma, A, B}}/\beta]/\beta, \upstairs{\qvar{\Gamma, A, B}}[\rewrite{\pi^\delta_{\One_\delta}}{\delta}/\delta]/y] \\
&\equiv \into{\Gamma, A, B}[\rewrite{\pi^{\beta}_{\One_{\beta}}}{\beta}[\upstairs{\proj{\Gamma, A, B}}/\beta]/\beta, \upstairs{\qvar{\Gamma, A, B}}[\rewrite{\pi^\delta_{\One_\delta}}{\delta}/\delta]/y] \\
&\equiv \into{\Gamma, A, B}[\rewrite{\pi^{\delta}_{\One_{\delta}}}{\upstairs{\proj{\Gamma, A, B}}}/\beta, \upstairs{\qvar{\Gamma, A, B}}[\rewrite{\pi^\delta_{\One_\delta}}{\delta}/\delta]/y]\\
&\equiv \into{\Gamma, A, B}[\rewrite{\pi^{\delta}_{\One_{\delta}}}{\upstairs{\proj{\Gamma, A, B}}}/\beta, \rewrite{\ApOne{\pi^{\delta}_{\One_{\delta}}}}{\StI{\ApEl{p}{\pi^{\delta}_{\One_{\delta}}}}{\upstairs{\qvar{\Gamma, A, B}}}}/y]\\
&\equiv \rewrite{(\pi^{\delta}_{\One_{\delta}} \bdot \ApOne{\pi^{\delta}_{\One_{\delta}}})}{\into{\Gamma, A, B}[\upstairs{\proj{\Gamma, A, B}}/\beta, \upstairs{\qvar{\Gamma, A, B}}/y]}  \\
&\equiv \rewrite{(\pi^{\delta}_{\One_{\delta}} \bdot \ApOne{\pi^{\delta}_{\One_{\delta}}})}{\rewrite{\pi^{\delta}_{\One_{\delta}}}{\delta}}  \\
&\equiv \rewrite{(\pi^{\delta}_{\One_{\delta}} \bdot \ApOne{\pi^{\delta}_{\One_{\delta}}});\pi^{\delta}_{\One_{\delta}}}{\delta}  \\
&\equiv \rewrite{\pi^{\delta.\One_\delta}_{\One_{\delta.\One_\delta}};\pi^{\delta}_{\One_{\delta}}}{\delta}
\end{align*}
\end{proof}

\subsubsection{Rules for the unit type}
\newcommand\qunitmatch[1]{\ensuremath{\mathsf{match}(#1)}}

The rules for the unit type are given in Figure~\ref{fig:qit-unit-rules}.

\begin{figure}
\begin{mathpar}
\inferrule*[left=1-form]{~}{\Gamma \qyields \One_\Gamma \TYPE} \and
\inferrule*[left=1-intro]{~}{\Gamma \qyields \star_\Gamma : \One_\Gamma} \and
\inferrule*[left=1-elim]{\Gamma \qyields c : C[\id_\Gamma, \star_\Gamma]}{\Gamma, \One_\Gamma \qyields \qunitmatch{c} : C}
\end{mathpar}
\begin{align}
\One_\Delta[\Theta] &\equiv \One_\Gamma \\
\star_\Delta[\Theta] &\equiv \star_\Gamma \\ 
\qunitmatch{c}[\Theta \uparrow \One_\Gamma] &\equiv \qunitmatch{c[\Theta]} \\
\nonumber \\
\qunitmatch{c}[\id_\Gamma, \star_\Gamma] &\equiv c
\end{align}
\caption{Rules for the unit type in MLTT via Explicit Substitutions}\label{fig:qit-unit-rules}
\end{figure}

\begin{theorem}
The rules for the unit type can be interpreted over any mode theory containing a comprehension object $p$ that has unit \mvrnote{and such that the $\mathsf{F}$-type for $\One_\alpha$ exists whenever needed}.
\end{theorem}

Translating the rules:
\begin{enumerate}[style = multiline, labelwidth = 80pt]
\item[\textsc{1-form}] For any $\upstairs{\Gamma}$ we can form
\begin{align*}
\alpha : \upstairs{\Gamma} \yields_{\El{p}{\alpha}} \upstairs{\One_\Gamma} :\equiv \F{w.\One_\alpha}{1} \TYPE
\end{align*}
\mvrnote{The mode $1$ vs the term $\One_\alpha$ vs the object language $\One_\Gamma$ could be confusing...}

\item[\textsc{1-intro}] We have
\begin{align*}
\alpha : \upstairs{\Gamma} \yields_{\One_\alpha} \upstairs{\star_\Gamma} :\equiv \FI{} : \upstairs{\One_\Gamma}
\end{align*}

\item[\textsc{1-elim}] We are given
\begin{align*}
\alpha : \upstairs{\Gamma} \yields_{\One_\alpha} \upstairs{c} :\upstairs{C[\id_\Gamma, \star_\Gamma]} 
\end{align*}
The type of $\upstairs{c}$ can be simplified to:
\begin{align*}
\upstairs{C[\id_\Gamma, \star_\Gamma]} 
&\equiv \upstairs{C}[\upstairs{\id_\Gamma, \star_\Gamma}/\beta] \\
&\equiv \upstairs{C}[\rewrite{\eta^\chi_{\alpha}}{\StI{\chi}{(\upstairs{\id_\Gamma},\upstairs{\star_\Gamma})}}/\beta] \\
&\equiv \upstairs{C}[\rewrite{\eta^\chi_{\alpha}}{\StI{\chi}{(\alpha, \FI{})}}/\beta] \\
&\equiv \St{\ApEl{p}{\eta^\chi_{\alpha}}}{\upstairs{C}[\StI{\chi}{(\alpha, \FI{})}/\beta]}
\end{align*}
So as our translation use:
\begin{mathpar}
\inferrule*[Left=s-intro+rewrite]{
\inferrule*[Left=cut]{
\inferrule*[Left=F-elim]{
\inferrule*[Left=s-intro+rewrite]{
\alpha : \upstairs{\Gamma} \yields_{\One_\alpha} \upstairs{c} :\St{\ApEl{p}{\eta^\chi_{\alpha}}}{\upstairs{C}[\StI{\chi}{(\alpha, \FI{})}/\beta]} }
{\alpha : \upstairs{\Gamma} \yields_{\One_{\alpha.\One_\alpha}} \rewrite{\ApOne{\pi^\alpha_{\One_\alpha}}}{\StI{\pi^\alpha_{\One_\alpha}}{\upstairs{c}}} : \upstairs{C}[\StI{\chi}{(\alpha, \FI{})}/\beta]}}
{\alpha : \upstairs{\Gamma}, s : \upstairs{\One_\Gamma} \yields_{\One_{\alpha.s}} \FEs{w.\One_\alpha}{s}{w}{\rewrite{\ApOne{\pi^\alpha_{\One_\alpha}}}{\StI{\pi^\alpha_{\One_\alpha}}{\upstairs{c}}}} : \upstairs{C}[\StI{\chi}{(\alpha, s)}/\beta]}}
{\beta : \upstairs{\Gamma, \One_\Gamma} \yields_{\One_{\beta.\One_\beta}} \FEs{w.\One_\alpha}{s}{w}{\rewrite{\ApOne{\pi^\alpha_{\One_\alpha}}}{\StI{\pi^\alpha_{\One_\alpha}}{\upstairs{c}}}}[\upstairs{\proj{\Gamma, \One_\Gamma}}/\alpha, \upstairs{\qvar{\Gamma, \One_\Gamma}}/s] : \upstairs{C}[\StI{\chi}{(\upstairs{\proj{\Gamma, \One_\Gamma}}, \upstairs{\qvar{\Gamma, \One_\Gamma}})}/\beta]}}
{\beta : \upstairs{\Gamma, \One_\Gamma} \yields_{\One_{\beta}} \rewrite{\ApOne{\eta^\chi_\beta}}{\StI{\eta^\chi_\beta}{\FEs{w.\One_\alpha}{s}{w}{\rewrite{\ApOne{\pi^\alpha_{\One_\alpha}}}{\StI{\pi^\alpha_{\One_\alpha}}{\upstairs{c}}}}[\upstairs{\proj{\Gamma, \One_\Gamma}}/\alpha, \upstairs{\qvar{\Gamma, \One_\Gamma}}/s]}} : \upstairs{C}}
\end{mathpar}
\end{enumerate}

Checking the equations hold is reasonably straightforward:
\begin{enumerate}[style = multiline, labelwidth = 80pt]
\item[{$\One_\Delta[\Theta] \equiv \One_\Gamma$}:] Easy!
\item[{$\star_\Delta[\Theta] \equiv \star_\Gamma $}:] Also easy!
\item[{$\qunitmatch{c}[\Theta \uparrow \One_\Gamma] \equiv \qunitmatch{c[\Theta]}$}:]
\begin{align*}
&\upstairs{\qunitmatch{c}[\Theta \uparrow \One_\Gamma]} \\
&\equiv \upstairs{\qunitmatch{c}}[\upstairs{\Theta \uparrow \One_\Gamma}/\beta] \\
&\equiv \rewrite{\ApOne{\eta^\chi_\beta}}{\StI{\eta^\chi_\beta}{\FEs{w.\One_\alpha}{s}{w}{\rewrite{\ApOne{\pi^\alpha_{\One_\alpha}}}{\StI{\pi^\alpha_{\One_\alpha}}{\upstairs{c}}}}[\upstairs{\proj{\Gamma, \One_\Gamma}}/\alpha, \upstairs{\qvar{\Gamma, \One_\Gamma}}/s]}}[\upstairs{\Theta \uparrow \One_\Gamma}/\beta] \\
&\equiv \rewrite{\ApOne{\eta^\chi_\beta}}{\StI{\eta^\chi_\beta}{\FEs{w.\One_\alpha}{s}{w}{\rewrite{\ApOne{\pi^\alpha_{\One_\alpha}}}{\StI{\pi^\alpha_{\One_\alpha}}{\upstairs{c}}}}[\upstairs{\proj{\Gamma, \One_\Gamma}}[\upstairs{\Theta \uparrow \One_\Gamma}/\beta]/\alpha, \upstairs{\qvar{\Gamma, \One_\Gamma}}[\upstairs{\Theta \uparrow \One_\Gamma}/\beta]/s]}} \\
&\equiv \rewrite{\ApOne{\eta^\chi_\beta}}{\StI{\eta^\chi_\beta}{\FEs{w.\One_\alpha}{s}{w}{\rewrite{\ApOne{\pi^\alpha_{\One_\alpha}}}{\StI{\pi^\alpha_{\One_\alpha}}{\upstairs{c}}}}[\upstairs{\Theta}[\upstairs{\proj{\Delta, \One_\Gamma[\Theta]}}/\alpha]/\alpha, \upstairs{\qvar{\Delta, \One_\Gamma[\Theta]}}/s]}} \\
&\equiv \rewrite{\ApOne{\eta^\chi_\beta}}{\StI{\eta^\chi_\beta}{\FEs{w.\One_\alpha}{s}{w}{\rewrite{\ApOne{\pi^\alpha_{\One_\alpha}}}{\StI{\pi^\alpha_{\One_\alpha}}{\upstairs{c}[\upstairs{\Theta}/\alpha]}}}[\upstairs{\proj{\Delta, \One_\Delta}}/\alpha, \upstairs{\qvar{\Delta, \One_\Delta}}/s]}} \\
&\equiv \upstairs{\qunitmatch{c[\Theta]}}
\end{align*}
\item[{$\qunitmatch{c}[\id_\Gamma, \star_\Gamma] \equiv c$}:] This is
\begin{align*}
&\upstairs{\qunitmatch{c}[\id_\Gamma, \star_\Gamma]} \\
&\equiv \rewrite{\ApOne{\eta^\chi_\beta}}{\StI{\eta^\chi_\beta}{\FEs{w.\One_\alpha}{s}{w}{\rewrite{\ApOne{\pi^\alpha_{\One_\alpha}}}{\StI{\pi^\alpha_{\One_\alpha}}{\upstairs{c}}}}[\upstairs{\proj{\Gamma, \One_\Gamma}}/\alpha, \upstairs{\qvar{\Gamma, \One_\Gamma}}/s]}}[\upstairs{\id_\Gamma, \star_\Gamma}/\beta] \\
&\equiv \rewrite{\ApOne{\eta^\chi_\beta}}{\StI{\eta^\chi_\beta}{\FEs{w.\One_\alpha}{s}{w}{\rewrite{\ApOne{\pi^\alpha_{\One_\alpha}}}{\StI{\pi^\alpha_{\One_\alpha}}{\upstairs{c}}}}[\upstairs{\proj{\Gamma, \One_\Gamma}}[\upstairs{\id_\Gamma, \star_\Gamma}/\beta]/\alpha, \upstairs{\qvar{\Gamma, \One_\Gamma}}[\upstairs{\id_\Gamma, \star_\Gamma}/\beta]/s]}} \\
&\equiv \rewrite{\ApOne{\eta^\chi_\beta}}{\StI{\eta^\chi_\beta}{\FEs{w.\One_\alpha}{s}{w}{\rewrite{\ApOne{\pi^\alpha_{\One_\alpha}}}{\StI{\pi^\alpha_{\One_\alpha}}{\upstairs{c}}}}[\upstairs{\id_\Gamma}/\alpha, \upstairs{\star_\Gamma}/s]}} \\
&\equiv \rewrite{\ApOne{\eta^\chi_\beta}}{\StI{\eta^\chi_\beta}{\FEs{w.\One_\alpha}{\upstairs{\star_\Gamma}}{w}{\rewrite{\ApOne{\pi^\alpha_{\One_\alpha}}}{\StI{\pi^\alpha_{\One_\alpha}}{\upstairs{c}}}}}} \\
&\equiv \rewrite{\ApOne{\eta^\chi_\beta}}{\StI{\eta^\chi_\beta}{\FEs{w.\One_\alpha}{\FI{}}{w}{\rewrite{\ApOne{\pi^\alpha_{\One_\alpha}}}{\StI{\pi^\alpha_{\One_\alpha}}{\upstairs{c}}}}}} \\
&\equiv \rewrite{\ApOne{\eta^\chi_\beta}}{\StI{\eta^\chi_\beta}{\rewrite{\ApOne{\pi^\alpha_{\One_\alpha}}}{\StI{\pi^\alpha_{\One_\alpha}}{\upstairs{c}}}}} \\
&\equiv \rewrite{\ApOne{\eta^\chi_\beta;\pi^\alpha_{\One_\alpha}}}{\StI{\eta^\chi_\beta;\pi^\alpha_{\One_\alpha}}{\upstairs{c}}} \\
&\equiv \upstairs{c}
\end{align*}
\end{enumerate}

\subsubsection{Rules for $\Sigma$-types}
\newcommand\qpair[1]{\ensuremath{\mathsf{pair}_{#1}}}
\newcommand\qsplit[1]{\ensuremath{\mathsf{split}_{#1}}}

The rules for $\Sigma$-types are given in Figure~\ref{fig:qit-sigma-rules}. There are two versions of the eliminator; a weak and strong version. The choice of one or the other yields \emph{weak} $\Sigma$-types or \emph{strong} $\Sigma$-types.

\begin{figure}
\begin{mathpar}
\inferrule*[left=$\Sigma$-form]{\Gamma \qyields A \TYPE \and \Gamma, A \qyields B \TYPE}{\Gamma \qyields \Sigma_A B \TYPE} \\
\inferrule*[left=$\Sigma$-pair]{~}{\Gamma, A, B \qyields \qpair{A, B} : (\Sigma_A B)[\proj{\Gamma, A, B};\proj{\Gamma, A}]} \and
\inferrule*[left=$\Sigma$-split]{\Gamma, A, B \qyields c : C[(\proj{\Gamma, A, B};\proj{\Gamma, A}), \qpair{A,B}]}{\Gamma, \Sigma_A B \qyields \qsplit{A, B}(c) : C}
%\inferrule*[left=$\Sigma$-pair]{\Gamma \qyields a : A \and \Gamma \qyields b : B[\hat{a}]}{\Gamma \qyields (a, b) : \Sigma_A B} \and
%\inferrule*[left=$\Sigma$-$\pi_1$]{\Gamma \qyields p : \Sigma_A B}{\Gamma \qyields \pi_1(p) : A} \and
%\inferrule*[left=$\Sigma$-$\pi_2$]{\Gamma \qyields p : \Sigma_A B}{\Gamma \qyields \pi_2(p) : B[\widehat{\pi_1(p)}]} \and
\end{mathpar}
\begin{align}
(\Sigma_A B)[\Theta] &\equiv \Sigma_{A[\Theta]} B[\Theta \uparrow A] \\
\qpair{A, B}[\Theta \uparrow A \uparrow B] &\equiv \qpair{A[\Theta], B[\Theta \uparrow A]} \\
\qsplit{A,B}(c)[\Theta \uparrow \Sigma_A B] &\equiv \qsplit{A[\Theta],B[\Theta \uparrow A]}(c[\Theta \uparrow A \uparrow B])  \\
\nonumber \\
\qsplit{A,B}(c)[(\proj{\Gamma, A, B};\proj{\Gamma, A}), \qpair{A,B}] &\equiv c
%\intertext{(If framework $\mathsf{F}$-types have eta:)}
%\qsplit{A,B}(\qpair{A,B}) &\equiv \qvar{\Gamma, \Sigma_A B}
\end{align}
%\mvrnote{that eta is not general enough}
%\vspace{1cm}
%\begin{mathpar}
%\inferrule*[left=$\Sigma$-split-weak]{\Gamma, A, B \qyields c : C[\proj{\Gamma, A, B};\proj{\Gamma, A}]}{\Gamma, \Sigma_A B \qyields \mathsf{wsplit}_{A, B}(c) : C[\proj{\Gamma, \Sigma_A B}]}
%\end{mathpar}
%\begin{align}
%\mathsf{wsplit}_{A,B}(\qpair{A,B}) &\equiv \qvar{\Gamma, \Sigma_A B} \\
%\mathsf{wsplit}_{A,B}(c)[\Theta \uparrow \Sigma_A B] &\equiv \mathsf{wsplit}_{A[\Theta],B[\Theta \uparrow A]}(c[\Theta \uparrow A \uparrow B]) \\
%\end{align}
%\begin{align}
%%\pi_1(a, b) &\equiv a \\
%%\pi_2(a, b) &\equiv b \\
%%(\pi_1(p), \pi_2(p)) &\equiv p \\
%%(a, b)[\Theta] &\equiv (a[\Theta], b[\Theta])
%\end{align}
\caption{Rules for $\Sigma$-types in MLTT via Explicit Substitutions}\label{fig:qit-sigma-rules}
\end{figure}

Suppose the comprehension object supports $\Sigma$ types in the sense of Definition~\ref{def:supports-sigmas}. In the framework, $\Sigma$-types are the $\mathsf{F}$-types for the mode term
\begin{align*}
\alpha : p, x : \El{p}{\alpha}, y : \El{p}{\alpha.x} \yields \Sigma_1(\alpha,x,y) : \El{p}{\alpha}
\end{align*}

\begin{theorem}
The rules for $\Sigma$-types can be interpreted over any mode theory containing a comprehension object that supports strong $\Sigma$-types (Definition \ref{def:supports-sigmas}), \mvrnote{and such that the $\mathsf{F}$-type for $\Sigma_1(\alpha,x,y)$ exists whenever needed}
\end{theorem}

We now translate all the rules:
\begin{enumerate}[style = multiline, labelwidth = 80pt]
\item[\textsc{$\Sigma$-form}] We are given
\begin{align*}
\alpha : \upstairs{\Gamma} &\yields_{\El{p}{\alpha}} \upstairs{A} \TYPE \\
\beta : \St{\chi}{\telety{\alpha}{\upstairs{\Gamma}}{\upstairs{A}}} &\yields_{\El{p}{\beta}} \upstairs{B} \TYPE \\
\end{align*}
So form
\begin{align*}
\alpha : \upstairs{\Gamma} \yields_{\El{p}{\alpha}} \upstairs{\Sigma_A B} :\equiv \F{w. \Sigma_1(\alpha,\fst w, \snd w)}{\telety{x}{\upstairs{A}}{\upstairs{B}[\into{\Gamma, A}/\beta]}} \TYPE
\end{align*}
Or written out:
\begin{mathpar}
\inferrule*[Left=F-form]
{\inferrule*[Left=$()$-form]
{\alpha : \upstairs{\Gamma} \yields_{\El{p}{\alpha}} \upstairs{A} \TYPE \and 
\inferrule*[left=cut]{
\alpha : \upstairs{\Gamma}, x : \upstairs{A} \yields \into{\Gamma, A} : \upstairs{\Gamma, A}
\and 
\beta : \upstairs{\Gamma, A} \yields_{\El{p}{\beta}} \upstairs{B} \TYPE
}{
\alpha : \upstairs{\Gamma}, x : \upstairs{A} \yields_{\El{p}{\alpha.x}} \upstairs{B}[\into{\Gamma,A}/\beta] \TYPE
}
}
{\alpha : \upstairs{\Gamma} \yields_{\telety{x}{\El{p}{\alpha}}{\El{p}{\alpha.x}}} \telety{x}{\upstairs{A}}{\upstairs{B}[\into{\Gamma,A}/\beta]} \TYPE}}
{\alpha : \upstairs{\Gamma} \yields_{\El{p}{\alpha}} \upstairs{\Sigma_A B} :\equiv \F{w. \Sigma_1(\alpha,\fst w, \snd w)}{\telety{x}{\upstairs{A}}{\upstairs{B}[\into{\Gamma,A}/\beta]}} \TYPE}
\end{mathpar}
\item[\textsc{$\Sigma$-pair}] We are trying to construct a term of type
\begin{align*}
\delta :\upstairs{\Gamma, A, B} \yields_{\El{p}{\delta}} \upstairs{(\Sigma_A B)[\proj{\Gamma, A, B};\proj{\Gamma, A}]} \TYPE
\end{align*}
Note that we are done if we can build a term of type
\begin{align*}
\alpha :\upstairs{\Gamma}, x : \upstairs{A}, y : \upstairs{B}[\into{\Gamma,A}/\beta] \yields_? ? : \upstairs{\Sigma_A B} 
\end{align*}
because we can then precompose with the substitution 
\begin{align*}
[\upstairs{\proj{\Gamma, A}}[\upstairs{\proj{\Gamma, A, B}}/\beta]/\alpha, \upstairs{\qvar{\Gamma, A}}[\upstairs{\proj{\Gamma, A, B}}/\beta]/x, \upstairs{\qvar{\Gamma, A, B}}[\rewrite{\pi^\delta_{\One_\delta}}{\delta}/\delta]/y]
\end{align*}
discussed earlier.

Building the required term is easy:
\begin{align*}
\alpha : \upstairs{\Gamma}, x : \upstairs{A}, y : \upstairs{B}[\into{\Gamma,A}/\beta] \yields_{\Sigma_1(\alpha, x, y)} \FI{x, y} : \F{w. \Sigma_1(\alpha,\fst w, \snd w)}{\telety{x}{\upstairs{A}}{\upstairs{B}[\into{\Gamma,A}/\beta]}}
\end{align*}
So all together:
\begin{align*}
\upstairs{\qpair{A,B}} :&\equiv \rewrite{\contract{\delta}}{\FIs{w. \Sigma_1(\alpha,\fst w, \snd w)}{(x, y)}[\upstairs{\proj{\Gamma, A}}[\upstairs{\proj{\Gamma, A, B}}/\beta]/\alpha, \upstairs{\qvar{\Gamma, A}}[\upstairs{\proj{\Gamma, A, B}}/\beta]/x, \upstairs{\qvar{\Gamma, A, B}}[\rewrite{\pi^\delta_{\One_\delta}}{\delta}/\delta]/y]} \\
&\equiv \rewrite{\contract{\delta}}{\FIs{w. \Sigma_1(\delta,\fst w, \snd w)}{(\upstairs{\qvar{\Gamma, A}}[\upstairs{\proj{\Gamma, A, B}}/\beta], \upstairs{\qvar{\Gamma, A, B}}[\rewrite{\pi^\delta_{\One_\delta}}{\delta}/\delta])}}
\end{align*}

\item[\textsc{$\Sigma$-split-strong}] We are given
\begin{align*}
\delta : \upstairs{\Gamma, A, B} \yields_{\One_\delta} \upstairs{c} :\upstairs{C[(\proj{\Gamma, A, B};\proj{\Gamma, A}), \qpair{A,B}]} 
\end{align*}
and we can substitute to get
\begin{align*}
\alpha : \upstairs{\Gamma}, x : \upstairs{A}, y : \upstairs{B}[\into{\Gamma,A}/\beta] \yields_{\One_{\alpha.x.y}} \upstairs{c}[\into{\Gamma, A, B}[\into{\Gamma, A}/\beta, y/y] /\delta] : \upstairs{C[(\proj{\Gamma, A, B};\proj{\Gamma, A}), \qpair{A,B}]}[\into{\Gamma, A, B}[\into{\Gamma, A}/\beta, y/y] /\delta]
\end{align*}
Let's simplify that type.
\begin{align*}
&\upstairs{C[(\proj{\Gamma, A, B};\proj{\Gamma, A}), \qpair{A,B}]}[\into{\Gamma, A, B}[\into{\Gamma, A}/\beta, y/y] /\delta] \\
&\equiv \upstairs{C}[\upstairs{(\proj{\Gamma, A, B};\proj{\Gamma, A}), \qpair{A,B}}/\beta][\into{\Gamma, A, B}[\into{\Gamma, A}/\beta, y/y] /\delta] \\
&\equiv \upstairs{C}[\rewrite{\eta^\chi_{\delta}}{\StI{\chi}{(\upstairs{\proj{\Gamma, A, B};\proj{\Gamma, A}}, \upstairs{\qpair{A,B}})}}/\beta][\into{\Gamma, A, B}[\into{\Gamma, A}/\beta, y/y] /\delta] \\
&\equiv \upstairs{C}[\rewrite{\eta^\chi_{\alpha.x.y}}{\StI{\chi}{(\upstairs{\proj{\Gamma, A, B};\proj{\Gamma, A}}[\into{\Gamma, A, B}[\into{\Gamma, A}/\beta, y/y] /\delta], \upstairs{\qpair{A,B}}[\into{\Gamma, A, B}[\into{\Gamma, A}/\beta, y/y] /\delta]/\beta])}}
\end{align*}
Going one component at a time, we checked in Lemma~\ref{lem:double-into} that
\begin{align*}
\upstairs{\proj{\Gamma, A, B};\proj{\Gamma, A}}[\into{\Gamma, A, B}[\into{\Gamma, A}/\beta, y/y] /\delta] 
&\equiv \rewrite{\pi^{\alpha.x}_y;\pi^\alpha_x}{\alpha}
\end{align*}
And in the other component we have
\begin{align*}
&\upstairs{\qpair{A,B}}[\into{\Gamma, A, B}[\into{\Gamma, A}/\beta, y/y] /\delta] \\
&\equiv \rewrite{\contract{\delta}}{\FIs{w. \Sigma_1(\alpha,\fst w, \snd w)}{(x, y)}[\upstairs{\proj{\Gamma, A}}[\upstairs{\proj{\Gamma, A, B}}/\beta]/\alpha, \upstairs{\qvar{\Gamma, A}}[\upstairs{\proj{\Gamma, A, B}}/\beta]/x, \upstairs{\qvar{\Gamma, A, B}}[\rewrite{\pi^\delta_{\One_\delta}}{\delta}/\delta]/y]}[\into{\Gamma, A, B}[\into{\Gamma, A}/\beta, y/y] /\delta] \\
&\equiv \rewrite{\contract{\alpha.x.y}}{\FI{x, y}[\rewrite{(\pi^{\alpha.x}_y;\pi^\alpha_x), (\ApOne{\pi^{\alpha.x}_y};\ApPlus{\pi^{\alpha.x}_y}{\var{x}}, \ApOne{\pi^{\alpha.x.y}_{\One_{\alpha.x.y}}};\ApPlus{\ApEl{p}{\pi^{\alpha.x.y}_{\One_{\alpha.x.y}}}}{\var{y}})}{(\alpha,x,y)}/(\alpha,x,y)]} \\
&\equiv \rewrite{\contract{\alpha.x.y}}{\rewrite{\ap{\Sigma_1(\alpha,x,y)}{(\pi^{\alpha.x}_y;\pi^\alpha_x), (\ApOne{\pi^{\alpha.x}_y};\ApPlus{\pi^{\alpha.x}_y}{\var{x}}, \ApOne{\pi^{\alpha.x.y}_{\One_{\alpha.x.y}}};\ApPlus{\ApEl{p}{\pi^{\alpha.x.y}_{\One_{\alpha.x.y}}}}{\var{y}})/(\alpha,x,y)}}{\StI{\ApEl{p}{\pi^{\alpha.x}_y;\pi^{\alpha}_x}}{\FI{x, y}}}} \\
&\equiv \rewrite{\fibpair{\alpha,x,y}}{\StI{\ApEl{p}{\pi^{\alpha.x}_y;\pi^{\alpha}_x}}{\FI{x, y}}}
\end{align*}
So continuing with $C$:
\begin{align*}
&\upstairs{C[(\proj{\Gamma, A, B};\proj{\Gamma, A}), \qpair{A,B}]}[\into{\Gamma, A, B}[\into{\Gamma, A}/\beta, y/y] /\delta] \\
&\equiv \upstairs{C}[\rewrite{\eta^\chi_{\alpha.x.y}}{\StI{\chi}{(\upstairs{\proj{\Gamma, A, B};\proj{\Gamma, A}}[\into{\Gamma, A, B}[\into{\Gamma, A}/\beta, y/y] /\delta], \upstairs{\qpair{A,B}}[\into{\Gamma, A, B}[\into{\Gamma, A}/\beta, y/y] /\delta]/\beta])}} \\
&\equiv \upstairs{C}[\rewrite{\eta^\chi_{\alpha.x.y}}{\StI{\chi}{(\rewrite{\pi^{\alpha.x}_y;\pi^\alpha_x}{\alpha}, \rewrite{\fibpair{\alpha,x,y}}{\StI{\ApEl{p}{\pi^{\alpha.x}_y;\pi^{\alpha}_x}}{\FI{x, y}}})}}/\beta] \\
&\equiv \upstairs{C}[\rewrite{\eta^\chi_{\alpha.x.y}}{\StI{\chi}{\rewrite{(\pi^{\alpha.x}_y;\pi^\alpha_x, \fibpair{\alpha,x,y})}{(\alpha, \FI{x, y})}}}/\beta] \\
&\equiv \upstairs{C}[\rewrite{\eta^\chi_{\alpha.x.y};\TrPlus{\chi}{\pi^{\alpha.x}_y;\pi^\alpha_x, \fibpair{\alpha,x,y}}}{\StI{\chi}{(\alpha, \FI{x, y})}}/\beta] \\
&\equiv \upstairs{C}[\rewrite{\pair{\alpha,x,y}}{\StI{\chi}{(\alpha, \FI{x, y})}}/\beta] \\
&\equiv \St{\ApEl{p}{\pair{\alpha,x,y}}}{\upstairs{C}[\StI{\chi}{(\alpha, \FI{x, y})}/\beta]}
\end{align*}

So as our translation use:
\begin{mathpar}
\inferrule*[Left=Cut]{
\inferrule*[Left=F-elim]{
\inferrule*[Left=s-intro+rewrite]{
\inferrule*[Left=Cut]
{\delta : \upstairs{\Gamma, A, B} \yields_{\One_\delta} \upstairs{c} : \upstairs{C[(\proj{\Gamma, A, B};\proj{\Gamma, A}), \qpair{A,B}]}}
{\alpha : \upstairs{\Gamma}, x : \upstairs{A}, y : \upstairs{B}[\into{\Gamma,A}/\beta] \yields_{\One_{\alpha.x.y}} \upstairs{c}[\into{\Gamma, A, B}[\into{\Gamma, A}/\beta, y/y] /\delta] : \St{\ApEl{p}{\pair{\alpha,x,y}}}{\upstairs{C}[\StI{\chi}{(\alpha, \FI{x, y})}/\beta]}}
}
{\alpha : \upstairs{\Gamma}, x : \upstairs{A}, y : \upstairs{B}[\into{\Gamma,A}/\beta] \yields_{\One_{\alpha.\Sigma_1(\alpha.x.y)}} \rewrite{\ApOne{\tsplit{\alpha,x,y}}}{\StI{\ApEl{p}{\tsplit{\alpha,x,y}}}{\upstairs{c}[\into{\Gamma, A, B}[\into{\Gamma, A}/\beta, y/y] /\delta]}} : \upstairs{C}[\StI{\chi}{(\alpha, \FI{x, y})}/\beta]}
}
{\alpha : \upstairs{\Gamma}, s : \upstairs{\Sigma_A B}\yields_{\One_{\alpha.s}} \FEs{w. \Sigma_1(\alpha,\fst w, \snd w)}{s}{w}{\rewrite{\ApOne{\tsplit{\alpha,x,y}}}{\StI{\ApEl{p}{\tsplit{\alpha,x,y}}}{\upstairs{c}[\into{\Gamma, A, B}[\into{\Gamma, A}/\beta, y/y] /\delta]}[\fst w/x, \snd w/y]}} : \upstairs{C}[\StI{\chi}{(\alpha, s)}/\beta]}
}
{
\beta : \upstairs{\Gamma, \Sigma_A B} \yields_{\One_\beta} \rewrite{\One_{\langle \eta^\chi_\beta \rangle}}{\StI{\ApEl{p}{\eta^\chi_\beta}}{\FEs{w. \Sigma_1(\alpha,\fst w, \snd w)}{s}{w}{\rewrite{\ApOne{\tsplit{\alpha,x,y}}}{\StI{\ApEl{p}{\tsplit{\alpha,x,y}}}{\upstairs{c}[\into{\Gamma, A, B}[\into{\Gamma, A}/\beta, y/y] /\delta]}[\fst w/x, \snd w/y]}})[\upstairs{\proj{\Gamma, \Sigma_A B}}/\alpha, \upstairs{\qvar{\Gamma, \Sigma_A B}}/s]}} : \upstairs{C}
}
\end{mathpar}
\mvrnote{Or because I can't figure out how to split lines in mathpar}
\begin{align*}
&\rewrite{\One_{\langle \eta^\chi_\beta \rangle}}{\StI{\ApEl{p}{\eta^\chi_\beta}}{\FEs{w. \Sigma_1(\alpha,\fst w, \snd w)}{s}{w}{\rewrite{\ApOne{\tsplit{\alpha,x,y}}}{\StI{\ApEl{p}{\tsplit{\alpha,x,y}}}{\\ &\upstairs{c}[\into{\Gamma, A, B}[\into{\Gamma, A}/\beta, y/y] /\delta]}[\fst w/x, \snd w/y]}})[\upstairs{\proj{\Gamma, \Sigma_A B}}/\alpha, \upstairs{\qvar{\Gamma, \Sigma_A B}}/s]}}
\end{align*}
\end{enumerate}

Now the equations:
\begin{enumerate}[style = multiline, labelwidth = 80pt]
\item[{$(\Sigma_A B)[\Theta] \equiv \Sigma_{A[\Theta]} B[\Theta \uparrow A]$}:] This is immediate, using Lemma~\ref{lem:into-identities}:
\begin{align*}
&\upstairs{\Sigma_{A[\Theta]} B[\Theta \uparrow A]} \\
&\equiv \F{w. \Sigma_1(\alpha,\fst w, \snd w)}{\telety{x}{\upstairs{A[\Theta]}}{\upstairs{B[\Theta \uparrow A]}[\into{\Gamma, A[\Theta]}/\beta]}} \\
&\equiv \F{w. \Sigma_1(\alpha,\fst w, \snd w)}{\telety{x}{\upstairs{A}[\upstairs{\Theta}/\alpha]}{\upstairs{B}[\into{\Delta, A}/\beta][\upstairs{\Theta}/\alpha, x / x]}} \\
&\equiv \F{w. \Sigma_1(\alpha,\fst w, \snd w)}{\telety{x}{\upstairs{A}}{\upstairs{B}[\into{\Delta, A}/\beta]}}[\upstairs{\Theta}/\alpha] \\
&\equiv \upstairs{\Sigma_A B}[\upstairs{\Theta}/\alpha] \\
&\equiv \upstairs{(\Sigma_A B)[\Theta]}
\end{align*}

\item[{$\qsplit{A,B}(c)\allowbreak[(\proj{\Gamma, A, B};\proj{\Gamma, A}), \allowbreak\qpair{A,B}] \equiv c$}:] 
This one should be `easy'!
\begin{align*}
&\upstairs{\qsplit{A,B}(c)[(\proj{\Gamma, A, B};\proj{\Gamma, A}),\qpair{A,B}]} \\
\equiv{} &\upstairs{\qsplit{A,B}(c)}[\upstairs{(\proj{\Gamma, A, B};\proj{\Gamma, A}),\qpair{A,B}}/\beta] \\
\equiv{} &\rewrite{\One_{\langle \eta^\chi_\beta \rangle}}{\StI{\ApEl{p}{\eta^\chi_\beta}}{\FEs{w. \Sigma_1(\alpha,\fst w, \snd w)}{s}{w}{\\&\rewrite{\ApOne{\tsplit{\alpha,x,y}}}{\StI{\ApEl{p}{\tsplit{\alpha,x,y}}}{\upstairs{c}[\into{\Gamma, A, B}[\into{\Gamma, A}/\beta, y/y] /\delta]}[\fst w/x, \snd w/y]}})\\&[\upstairs{\proj{\Gamma, \Sigma_A B}}/\alpha, \upstairs{\qvar{\Gamma, \Sigma_A B}}/s]}}[\upstairs{(\proj{\Gamma, A, B};\proj{\Gamma, A}),\qpair{A,B}}/\beta] \\
%%%
\equiv{} &\rewrite{\One_{\langle \eta^\chi_\delta \rangle}}{\StI{\ApEl{p}{\eta^\chi_\delta}}{\FEs{w. \Sigma_1(\alpha,\fst w, \snd w)}{s}{w}{\\&\rewrite{\ApOne{\tsplit{\alpha,x,y}}}{\StI{\ApEl{p}{\tsplit{\alpha,x,y}}}{\upstairs{c}[\into{\Gamma, A, B}[\into{\Gamma, A}/\beta, y/y] /\delta]}[\fst w/x, \snd w/y]}})\\&[\upstairs{\proj{\Gamma, \Sigma_A B}}[\upstairs{(\proj{\Gamma, A, B};\proj{\Gamma, A}),\qpair{A,B}}/\beta]/\alpha, \upstairs{\qvar{\Gamma, \Sigma_A B}}[\upstairs{(\proj{\Gamma, A, B};\proj{\Gamma, A}),\qpair{A,B}}/\beta]/s]}} \\
%%%
\equiv{} &\rewrite{\One_{\langle \eta^\chi_\delta \rangle}}{\StI{\ApEl{p}{\eta^\chi_\delta}}{\FEs{w. \Sigma_1(\alpha,\fst w, \snd w)}{s}{w}{\\&\rewrite{\ApOne{\tsplit{\alpha,x,y}}}{\StI{\ApEl{p}{\tsplit{\alpha,x,y}}}{\upstairs{c}[\into{\Gamma, A, B}[\into{\Gamma, A}/\beta, y/y] /\delta]}[\fst w/x, \snd w/y]}})\\&[\upstairs{\proj{\Gamma, A, B};\proj{\Gamma, A}}/\alpha, \upstairs{\qpair{A,B}}/s]}} \\
%%%
\equiv{} &\rewrite{\One_{\langle \eta^\chi_\delta \rangle}}{\StI{\ApEl{p}{\eta^\chi_\delta}}{\FEs{w. \Sigma_1(\alpha,\fst w, \snd w)}{\upstairs{\qpair{A,B}}}{w}{\\&\rewrite{\ApOne{\tsplit{\delta,x,y}}}{\StI{\ApEl{p}{\tsplit{\delta,x,y}}}{\upstairs{c}[\into{\Gamma, A, B}[\into{\Gamma, A}/\beta, y/y] /\delta][\upstairs{\proj{\Gamma, A, B};\proj{\Gamma, A}}/\alpha]}[\fst w/x, \snd w/y]}})}} \\
%%%
\equiv{} &\rewrite{\One_{\langle \eta^\chi_\delta \rangle}}{\StI{\ApEl{p}{\eta^\chi_\delta}}{\FEs{w. \Sigma_1(\alpha,\fst w, \snd w)}{\rewrite{\contract{\delta}}{\FIs{w. \Sigma_1(\delta,\fst w, \snd w)}{(\upstairs{\qvar{\Gamma, A}}[\upstairs{\proj{\Gamma, A, B}}/\beta], \upstairs{\qvar{\Gamma, A, B}}[\rewrite{\pi^\delta_{\One_\delta}}{\delta}/\delta])}}}{w}{\\&\rewrite{\ApOne{\tsplit{\delta,x,y}}}{\StI{\ApEl{p}{\tsplit{\delta,x,y}}}{\upstairs{c}[\into{\Gamma, A, B}[\into{\Gamma, A}/\beta, y/y] /\delta][\upstairs{\proj{\Gamma, A, B};\proj{\Gamma, A}}/\alpha]}[\fst w/x, \snd w/y]}})}} \\
%%%
\equiv{} &\rewrite{\One_{\langle \eta^\chi_\delta \rangle}}{\StI{\ApEl{p}{\eta^\chi_\delta}}{\rewrite{\One_{\langle \id_\delta \bdot \contract{\delta} \rangle}}{\StI{\ApEl{p}{\id_\delta \bdot \contract{\delta}}}{\FEs{w. \Sigma_1(\alpha,\fst w, \snd w)}{\FIs{w. \Sigma_1(\delta,\fst w, \snd w)}{(\upstairs{\qvar{\Gamma, A}}[\upstairs{\proj{\Gamma, A, B}}/\beta], \upstairs{\qvar{\Gamma, A, B}}[\rewrite{\pi^\delta_{\One_\delta}}{\delta}/\delta])}}{w}{\\&\rewrite{\ApOne{\tsplit{\delta,x,y}}}{\StI{\ApEl{p}{\tsplit{\delta,x,y}}}{\upstairs{c}[\into{\Gamma, A, B}[\into{\Gamma, A}/\beta, y/y] /\delta][\upstairs{\proj{\Gamma, A, B};\proj{\Gamma, A}}/\alpha]}[\fst w/x, \snd w/y]}})}}}} \\
%%%
\equiv{} &\rewrite{\One_{\langle \eta^\chi_\delta \rangle}}{\StI{\ApEl{p}{\eta^\chi_\delta}}{\rewrite{\One_{\langle \id_\delta \bdot \contract{\delta} \rangle}}{\StI{\ApEl{p}{\id_\delta \bdot \contract{\delta}}}{\\&\rewrite{\ApOne{\tsplit{\delta,x,y}}}{\StI{\ApEl{p}{\tsplit{\delta,x,y}}}{\upstairs{c}[\into{\Gamma, A, B}[\into{\Gamma, A}/\beta, y/y] /\delta][\upstairs{\proj{\Gamma, A, B};\proj{\Gamma, A}}/\alpha]}[\fst w/x, \snd w/y][(\upstairs{\qvar{\Gamma, A}}[\upstairs{\proj{\Gamma, A, B}}/\beta], \upstairs{\qvar{\Gamma, A, B}}[\rewrite{\pi^\delta_{\One_\delta}}{\delta}/\delta])/w]}})}}} \\
%%%
\equiv{} &\rewrite{\One_{\langle \eta^\chi_\delta \rangle}}{\StI{\ApEl{p}{\eta^\chi_\delta}}{\rewrite{\One_{\langle \id_\delta \bdot \contract{\delta} \rangle}}{\StI{\ApEl{p}{\id_\delta \bdot \contract{\delta}}}{\\&\rewrite{\ApOne{\tsplit{\delta,\One_\delta,\One_{\delta.\One_\delta}}}}{\StI{\ApEl{p}{\tsplit{\delta,\One_\delta,\One_{\delta.\One_\delta}}}}{\upstairs{c}[\into{\Gamma, A, B}[\into{\Gamma, A}/\beta, y/y] /\delta][\upstairs{\proj{\Gamma, A, B};\proj{\Gamma, A}}/\alpha, \upstairs{\qvar{\Gamma, A}}[\upstairs{\proj{\Gamma, A, B}}/\beta]/x, \upstairs{\qvar{\Gamma, A, B}}[\rewrite{\pi^\delta_{\One_\delta}}{\delta}/\delta]/y]}}})}}} \\
%%%
\equiv{} &\rewrite{\ApOne{\eta^\chi_\delta;(\id_\delta \bdot \contract{\delta});\tsplit{\delta,\One_\delta,\One_{\delta.\One_\delta}}}}{\StI{\ApEl{p}{\eta^\chi_\delta;(\id_\delta \bdot \contract{\delta});\tsplit{\delta,\One_\delta,\One_{\delta.\One_\delta}}}}{\\&\upstairs{c}[\into{\Gamma, A, B}[\into{\Gamma, A}/\beta, y/y] /\delta][\upstairs{\proj{\Gamma, A, B};\proj{\Gamma, A}}/\alpha, \upstairs{\qvar{\Gamma, A}}[\upstairs{\proj{\Gamma, A, B}}/\beta]/x, \upstairs{\qvar{\Gamma, A, B}}[\rewrite{\pi^\delta_{\One_\delta}}{\delta}/\delta]/y]}} \\
%%%
\equiv{} &\rewrite{\ApOne{\eta^\chi_\delta;\eta^\chi_{\delta.\One_\delta}}}{\StI{\ApEl{p}{\eta^\chi_\delta;\eta^\chi_{\delta.\One_\delta}}}{\\&\upstairs{c}[\into{\Gamma, A, B}[\into{\Gamma, A}/\beta, y/y] /\delta][\upstairs{\proj{\Gamma, A, B};\proj{\Gamma, A}}/\alpha, \upstairs{\qvar{\Gamma, A}}[\upstairs{\proj{\Gamma, A, B}}/\beta]/x, \upstairs{\qvar{\Gamma, A, B}}[\rewrite{\pi^\delta_{\One_\delta}}{\delta}/\delta]/y]}}
\end{align*}
By Lemma~\ref{lem:double-into}, we know the substitution is equal to:
\begin{align*}
&\into{\Gamma, A, B}[\into{\Gamma, A}/\beta, y/y][\upstairs{\proj{\Gamma, A, B};\proj{\Gamma, A}}/\alpha, \upstairs{\qvar{\Gamma, A}}[\upstairs{\proj{\Gamma, A, B}}/\beta]/x, \upstairs{\qvar{\Gamma, A, B}}[\rewrite{\pi^\delta_{\One_\delta}}{\delta}/\delta]/y] \\
&\equiv \rewrite{\pi^{\delta.\One_\delta}_{\One_{\delta.\One_\delta}};\pi^{\delta}_{\One_{\delta}}}{\delta}
\end{align*}
So finally:
\begin{align*}
&\upstairs{\qsplit{A,B}(c)[(\proj{\Gamma, A, B};\proj{\Gamma, A}), \allowbreak\qpair{A,B}]} \\
\equiv{} &\rewrite{\ApOne{\eta^\chi_\delta;\eta^\chi_{\delta.\One_\delta}}}{\StI{\ApEl{p}{\eta^\chi_\delta;\eta^\chi_{\delta.\One_\delta}}}{\\&\upstairs{c}[\into{\Gamma, A, B}[\into{\Gamma, A}/\beta, y/y] /\delta][\upstairs{\proj{\Gamma, A, B};\proj{\Gamma, A}}/\alpha, \upstairs{\qvar{\Gamma, A}}[\upstairs{\proj{\Gamma, A, B}}/\beta]/x, \upstairs{\qvar{\Gamma, A, B}}[\rewrite{\pi^\delta_{\One_\delta}}{\delta}/\delta]/y]}} \\
%%%
\equiv{} &\rewrite{\ApOne{\eta^\chi_\delta;\eta^\chi_{\delta.\One_\delta}}}{\StI{\ApEl{p}{\eta^\chi_\delta;\eta^\chi_{\delta.\One_\delta}}}{\upstairs{c}[\rewrite{\pi^{\delta.\One_\delta}_{\One_{\delta.\One_\delta}};\pi^{\delta}_{\One_{\delta}}}{\delta}/\delta]}} \\
%%%
\equiv{} &\upstairs{c}[\rewrite{\eta^\chi_\delta;\eta^\chi_{\delta.\One_\delta};\pi^{\delta.\One_\delta}_{\One_{\delta.\One_\delta}};\pi^{\delta}_{\One_{\delta}}}{\delta}/\delta]\\
%%%
\equiv{} &\upstairs{c}
\end{align*}

\item[{$\qpair{A, B}[\Theta \uparrow A \uparrow B] \equiv \qpair{A[\Theta], B[\Theta \uparrow A]}$}:]
Let's try it:
\begin{align*}
&\upstairs{\qpair{A, B}[\Theta \uparrow A \uparrow B]} \\
&\equiv \upstairs{\qpair{A, B}}[\upstairs{\Theta \uparrow A \uparrow B}/\delta] \\
&\equiv \rewrite{\contract{\delta}}{\FIs{w. \Sigma_1(\delta,\fst w, \snd w)}{(\upstairs{\qvar{\Delta, A}}[\upstairs{\proj{\Delta, A, B}}/\beta], \upstairs{\qvar{\Gamma, A, B}}[\rewrite{\pi^\delta_{\One_\delta}}{\delta}/\delta])}}[\upstairs{\Delta \uparrow A \uparrow B}/\delta] \\
&\equiv \rewrite{\contract{\delta}}{\FIs{w. \Sigma_1(\delta,\fst w, \snd w)}{(\upstairs{\qvar{\Delta, A}}[\upstairs{\proj{\Delta, A, B}}/\beta][\upstairs{\Theta \uparrow A \uparrow B}/\delta], \upstairs{\qvar{\Gamma, A, B}}[\rewrite{\pi^\delta_{\One_\delta}}{\delta}/\delta][\upstairs{\Delta \uparrow A \uparrow B}/\delta])}} \\
&\equiv \rewrite{\contract{\delta}}{\FIs{w. \Sigma_1(\delta,\fst w, \snd w)}{(\upstairs{\qvar{\Delta, A}}[\upstairs{\Theta \uparrow A}/\beta][\upstairs{\proj{\Gamma, A[\Theta], B[\Theta \uparrow A]}}/\beta], \upstairs{\qvar{\Gamma, A[\Theta], B[\Theta]}}[\rewrite{\pi^\delta_{\One_\delta}}{\delta}/\delta])}} \\
&\equiv \rewrite{\contract{\delta}}{\FIs{w. \Sigma_1(\delta,\fst w, \snd w)}{(\upstairs{\qvar{\Gamma, A[\Theta]}}[\upstairs{\proj{\Gamma, A[\Theta], B[\Theta \uparrow A]}}/\beta], \upstairs{\qvar{\Gamma, A[\Theta], B[\Theta]}}[\rewrite{\pi^\delta_{\One_\delta}}{\delta}/\delta])}} \\
&\equiv \upstairs{\qpair{A[\Theta], B[\Theta \uparrow A]}}
\end{align*}
Where the manipulations in the middle follow from the object-language equations that we checked above.

\item[{$\qsplit{A,B}(c)[\Theta \uparrow \Sigma_A B] \equiv \qsplit{A[\Theta],B[\Theta \uparrow A]}(c[\Theta \uparrow A \uparrow B])$}:]
\begin{align*}
&\upstairs{\qsplit{A,B}(c)[\Theta \uparrow \Sigma_A B]} \\
\equiv{} &\upstairs{\qsplit{A,B}(c)}[\upstairs{\Theta \uparrow \Sigma_A B}/\beta] \\
\equiv{} &\rewrite{\One_{\langle \eta^\chi_\beta \rangle}}{\StI{\ApEl{p}{\eta^\chi_\beta}}{\FEs{w. \Sigma_1(\alpha,\fst w, \snd w)}{s}{w}{\rewrite{\ApOne{\tsplit{\alpha,x,y}}}{\StI{\ApEl{p}{\tsplit{\alpha,x,y}}}{\\&\upstairs{c}[\into{\Gamma, A, B}[\into{\Gamma, A}/\beta, y/y] /\delta]}[\fst w/x, \snd w/y]}})[\upstairs{\proj{\Gamma, \Sigma_A B}}/\alpha, \upstairs{\qvar{\Gamma, \Sigma_A B}}/s]}}[\upstairs{\Theta \uparrow \Sigma_A B}/\beta] \\
\equiv{} &\rewrite{\One_{\langle \eta^\chi_\beta \rangle}}{\StI{\ApEl{p}{\eta^\chi_\beta}}{\FEs{w. \Sigma_1(\alpha,\fst w, \snd w)}{s}{w}{\rewrite{\ApOne{\tsplit{\alpha,x,y}}}{\StI{\ApEl{p}{\tsplit{\alpha,x,y}}}{\\&\upstairs{c}[\into{\Gamma, A, B}[\into{\Gamma, A}/\beta, y/y] /\delta]}[\fst w/x, \snd w/y]}})[\upstairs{\proj{\Gamma, \Sigma_A B}}[\upstairs{\Theta \uparrow \Sigma_A B}/\beta]/\alpha, \upstairs{\qvar{\Gamma, \Sigma_A B}}[\upstairs{\Theta \uparrow \Sigma_A B}/\beta]/s]}} \\
\equiv{} &\rewrite{\One_{\langle \eta^\chi_\beta \rangle}}{\StI{\ApEl{p}{\eta^\chi_\beta}}{\FEs{w. \Sigma_1(\alpha,\fst w, \snd w)}{s}{w}{\rewrite{\ApOne{\tsplit{\alpha,x,y}}}{\StI{\ApEl{p}{\tsplit{\alpha,x,y}}}{\\&\upstairs{c}[\into{\Gamma, A, B}[\into{\Gamma, A}/\beta, y/y] /\delta]}[\fst w/x, \snd w/y]}})[\upstairs{\Theta}[\upstairs{\proj{\Delta, \Sigma_A[\Theta] B[\Theta \uparrow A]}}/\alpha]/\alpha, \upstairs{\qvar{\Delta, \Sigma_A[\Theta] B[\Theta \uparrow A]}/s}]}} \\
\equiv{} &\rewrite{\One_{\langle \eta^\chi_\beta \rangle}}{\StI{\ApEl{p}{\eta^\chi_\beta}}{\FEs{w. \Sigma_1(\alpha,\fst w, \snd w)}{s}{w}{\rewrite{\ApOne{\tsplit{\alpha,x,y}}}{\StI{\ApEl{p}{\tsplit{\alpha,x,y}}}{\\&\upstairs{c}[\into{\Gamma, A, B}[\into{\Gamma, A}/\beta, y/y] /\delta][\upstairs{\Theta}/\alpha]}[\fst w/x, \snd w/y]}}[\upstairs{\proj{\Delta, \Sigma_A[\Theta] B[\Theta \uparrow A]}}/\alpha, \upstairs{\qvar{\Delta, \Sigma_A[\Theta] B[\Theta \uparrow A]}}/s]}} \\
\equiv{} &\rewrite{\One_{\langle \eta^\chi_\beta \rangle}}{\StI{\ApEl{p}{\eta^\chi_\beta}}{\FEs{w. \Sigma_1(\alpha,\fst w, \snd w)}{s}{w}{\rewrite{\ApOne{\tsplit{\alpha,x,y}}}{\StI{\ApEl{p}{\tsplit{\alpha,x,y}}}{\\&\upstairs{c}[\upstairs{\Theta \uparrow A \uparrow B}/\delta][\into{\Delta, A[\Theta], B[\Theta \uparrow A]}[\into{\Delta, A[\Theta]}/\beta, y/y] /\delta]}[\fst w/x, \snd w/y]}})[\upstairs{\proj{\Delta, \Sigma_A[\Theta] B[\Theta \uparrow A]}}/\alpha, \upstairs{\qvar{\Delta, \Sigma_A[\Theta] B[\Theta \uparrow A]}}/s]}} \\
\equiv{} & \upstairs{\qsplit{A[\Theta],B[\Theta \uparrow A]}(c[\Theta \uparrow A \uparrow B])}
\end{align*}
\end{enumerate}

\subsubsection{Rules for $\Pi$-types}
\newcommand\qapp[1]{\ensuremath{\mathsf{app}_{#1}}}
\newcommand\qlam{\ensuremath{\mathsf{lam}}}

The rules for $\Pi$ types are given in Figure~\ref{fig:qit-pi-rules}. Note that stability of $\mathsf{app}$ under substitution is derivable: \mvrnote{TODO}

\begin{figure}
\begin{mathpar}
\inferrule*[left=$\Pi$-form]{\Gamma \qyields A \TYPE \and \Gamma, A \qyields B \TYPE}{\Gamma \qyields \Pi_A B \TYPE} \\
\inferrule*[left=$\Pi$-app]{~}{\Gamma, A, (\Pi_A B)[\proj{\Gamma, A}] \qyields \qapp{A, B} : B[\proj{\Gamma, A, (\Pi_A B)[\proj{\Gamma, A}]}]} \and 
\inferrule*[left=$\Pi$-lam]{\Gamma, A \qyields b : B}{\Gamma \qyields \qlam(b) : \Pi_A B}
\end{mathpar}
\begin{align}
(\Pi_A B)[\Theta] &\equiv \Pi_{A[\Theta]} B[\Theta \uparrow A] \\
\nonumber \\
\qapp{A,B}[\id_{\Gamma, A}, \qlam(M)[\proj{\Gamma, A}]] &\equiv M \\
\qlam(\qapp{A,B}[\id_{\Gamma, A}, f[\proj{\Gamma, A}]]) &\equiv f \\
\qlam(b)[\Theta] &\equiv \qlam(b[\Theta \uparrow A])
\end{align}
\begin{mathpar}

\end{mathpar}
\caption{Rules for $\Pi$-types in MLTT via Explicit Substitutions}\label{fig:qit-pi-rules}
\end{figure}

\begin{theorem}
The rules for $\Pi$-types can be interpreted over any mode theory with a comprehension object that supports $\Pi$ types (Definition~\ref{def:supports-pis}, \mvrnote{and such that certain $\mathsf{U}$-types exist}
\end{theorem}

In the framework, $\Pi$-types are the $\mathsf{U}$-types for the mode term
\begin{align*}
\alpha : p, x : \El{p}{\alpha}, c : \El{p}{\alpha} \yields \Pi_1(\alpha,x,c) :\equiv \TrPlus{\ApEl{p}{\pi^\alpha_x}}{c} : \El{p}{\alpha.x}
\end{align*}

\begin{enumerate}
%\item[\textsc{a}]
\item[\textsc{$\Pi$-form}] We are given
\begin{align*}
\alpha : \upstairs{\Gamma} &\yields_{\El{p}{\alpha}} \upstairs{A} \TYPE \\
\beta : \St{\chi}{\telety{\alpha}{\upstairs{\Gamma}}{\upstairs{A}}} &\yields_{\El{p}{\beta}} \upstairs{B} \TYPE
\end{align*}
So form
\begin{align*}
\alpha : \upstairs{\Gamma} \yields_{\El{p}{\alpha}} \upstairs{\Pi_A B} :\equiv \U{c. \Pi_1(\alpha,x,c)}{\upstairs{A}}{\upstairs{B}[\into{\Gamma, A}/\beta]} \TYPE
\end{align*}

\item[\textsc{$\Pi$-app}] First note that we can pull the context apart using:
\begin{align*}
\delta : \upstairs{\Gamma, A, \Pi_A B[\proj{\Gamma, A}]} &\yields_{\One_\delta} \upstairs{\qvar{\Gamma, A, \Pi_A B[\proj{\Gamma, A}]}} : \upstairs{\Pi_A B[\proj{\Gamma, A}][\proj{\Gamma, A, \Pi_A B[\proj{\Gamma, A}]}]} \\ 
\delta : \upstairs{\Gamma, A, \Pi_A B[\proj{\Gamma, A}]} &\yields_{\One_\delta} \upstairs{\qvar{\Gamma, A}[\proj{\Gamma, A, \Pi_A B[\proj{\Gamma, A}]}]} : \upstairs{A[\proj{\Gamma, A}][\proj{\Gamma, A, \Pi_A B[\proj{\Gamma, A}]}]} 
\end{align*}
We check below that $\upstairs{(\Pi_A B)[\Theta]} \equiv \upstairs{\Pi_{A[\Theta]} B[\Theta \uparrow A]}$, so the type of the first is equal to:
\begin{align*}
&\upstairs{\Pi_A B[\proj{\Gamma, A}][\proj{\Gamma, A, \Pi_A B[\proj{\Gamma, A}]}]} \\
&\equiv \upstairs{\Pi_{A[\proj{\Gamma, A}][\proj{\Gamma, A, \Pi_A B[\proj{\Gamma, A}]}]} (B[(\proj{\Gamma, A, \Pi_A B[\proj{\Gamma, A}]};\proj{\Gamma, A}) \uparrow A)]} \\
&\equiv \U{c. \Pi_1(\alpha,x,c)}{\upstairs{A[\proj{\Gamma, A}][\proj{\Gamma, A, \Pi_A B[\proj{\Gamma, A}]}]}}{\upstairs{B[(\proj{\Gamma, A, \Pi_A B[\proj{\Gamma, A}]};\proj{\Gamma, A}) \uparrow A)]}[\into{\Gamma, A, \Pi_A B[\proj{\Gamma, A}], A[\proj{\Gamma, A}][\proj{\Gamma, A, \Pi_A B[\proj{\Gamma, A}]}]}/\beta]} \\
&\equiv \U{c. \Pi_1(\alpha,x,c)}{\upstairs{A[\proj{\Gamma, A}][\proj{\Gamma, A, \Pi_A B[\proj{\Gamma, A}]}]}}{\upstairs{B}[\into{\Gamma, A}/\beta][\upstairs{(\proj{\Gamma, A, \Pi_A B[\proj{\Gamma, A}]};\proj{\Gamma, A})}/\alpha, x/x]}
\end{align*}
So we can apply the second to the first and transport to get:
\begin{align*}
\delta : \upstairs{\Gamma, A, \Pi_A B[\proj{\Gamma, A}]} &\yields_{\One_\delta} \upstairs{\qapp{A,B}} :\equiv \StI{\ApEl{p}{\eta^\chi_\delta}}{\upstairs{\qvar{\Gamma, A, \Pi_A B[\proj{\Gamma, A}]}}(\upstairs{\qvar{\Gamma, A}[\proj{\Gamma, A, \Pi_A B[\proj{\Gamma, A}]}]})} \\&: \St{\ApEl{p}{\eta^\chi_\delta}}{\upstairs{B}[\into{\Gamma, A}/\beta][\upstairs{(\proj{\Gamma, A, \Pi_A B[\proj{\Gamma, A}]};\proj{\Gamma, A})}/\alpha, x/x][\upstairs{\qvar{\Gamma, A}[\proj{\Gamma, A, \Pi_A B[\proj{\Gamma, A}]}]}/x]}
\end{align*}
This is indeed the right type:
\begin{align*}
&\St{\ApEl{p}{\eta^\chi_\delta}}{\upstairs{B}[\into{\Gamma, A}/\beta][\upstairs{(\proj{\Gamma, A, \Pi_A B[\proj{\Gamma, A}]};\proj{\Gamma, A})}/\alpha, x/x][\upstairs{\qvar{\Gamma, A}[\proj{\Gamma, A, \Pi_A B[\proj{\Gamma, A}]}]}/x]} \\
&\equiv \St{\ApEl{p}{\eta^\chi_\delta}}{\upstairs{B}[\into{\Gamma, A}/\beta][\upstairs{(\proj{\Gamma, A, \Pi_A B[\proj{\Gamma, A}]};\proj{\Gamma, A})}/\alpha, \upstairs{\qvar{\Gamma, A}[\proj{\Gamma, A, \Pi_A B[\proj{\Gamma, A}]}]}/x]} \\
&\equiv \St{\ApEl{p}{\eta^\chi_\delta}}{\upstairs{B}[\into{\Gamma, A}/\beta][\upstairs{\proj{\Gamma, A}}/\alpha, \upstairs{\qvar{\Gamma, A}}/x][\upstairs{\proj{\Gamma, A, \Pi_A B[\proj{\Gamma, A}]}}/\beta]} \\
&\equiv \St{\ApEl{p}{\eta^\chi_\beta}}{\upstairs{B}[\into{\Gamma, A}[\upstairs{\proj{\Gamma, A}}/\alpha, \upstairs{\qvar{\Gamma, A}}/x]/\beta]}[\upstairs{\proj{\Gamma, A, \Pi_A B[\proj{\Gamma, A}]}}/\beta] \\
&\equiv \upstairs{B}[\rewrite{\eta^\chi_\beta}{\into{\Gamma, A}[\upstairs{\proj{\Gamma, A}}/\alpha, \upstairs{\qvar{\Gamma, A}}/x]}/\beta][\upstairs{\proj{\Gamma, A, \Pi_A B[\proj{\Gamma, A}]}}/\beta] \\
&\equiv \upstairs{B}[\upstairs{(\proj{\Gamma, A}, \qvar{\Gamma, A})}/\beta][\upstairs{\proj{\Gamma, A, \Pi_A B[\proj{\Gamma, A}]}}/\beta] \\
&\equiv \upstairs{B}[\upstairs{\proj{\Gamma, A, \Pi_A B[\proj{\Gamma, A}]}}/\beta] \\
&\equiv \upstairs{B[\proj{\Gamma, A, \Pi_A B[\proj{\Gamma, A}]}]}
\end{align*}


\item[\textsc{$\Pi$-lam}] Given
\begin{align*}
\beta : \upstairs{\Gamma, A} \yields_{\One_\beta} \upstairs{b} : \upstairs{B}
\end{align*}
So we can substitute with $\into{\Gamma, A}$ and rewrite by $\pinv{\alpha,x}$:
\begin{align*}
\alpha : \upstairs{\Gamma} \yields_{\TrPlus{\pi^\alpha_x}{\One_{\alpha}}} \rewrite{\pinv{\alpha,x}}{\upstairs{b}[\into{\Gamma, A}/\beta]} : \upstairs{B}[\into{\Gamma, A}/\beta]
\end{align*}
And then use $\mathsf{U}$-right: 
\begin{align*}
\upstairs{\qlam(b)} :\equiv \UI{x}{\rewrite{\pinv{\alpha,x}}{\upstairs{b}[\into{\Gamma, A}/\beta]}}
\end{align*}
\end{enumerate}


And the equations:
\begin{enumerate}[style = multiline, labelwidth = 80pt]
\item[{$(\Pi_A B)[\Theta] \equiv \Pi_{A[\Theta]} B[\Theta \uparrow A]$}:] Identical to the case for $\Sigma$:
\begin{align*}
\upstairs{(\Pi_A B)[\Theta]}
&\equiv \upstairs{\Pi_A B}[\upstairs{\Theta}/\alpha] \\
&\equiv \U{c.\Pi_1(\alpha,x,c)}{\upstairs{A}}{\upstairs{B}[\into{\Delta, A}/\beta]}[\upstairs{\Theta}/\alpha] \\
&\equiv \U{c.\Pi_1(\alpha,x,c)}{\upstairs{A}[\upstairs{\Theta}/\alpha]}{\upstairs{B}[\into{\Delta, A}/\beta][\upstairs{\Theta}/\alpha,x/x]} \\
&\equiv \U{c.\Pi_1(\alpha,x,c)}{\upstairs{A}[\upstairs{\Theta}/\alpha]}{\upstairs{B}[\upstairs{\Theta \uparrow A}/\beta][\into{\Gamma, A[\Theta]}/\beta]} \\
&\equiv \upstairs{\Pi_{A[\Theta]} B[\Theta \uparrow A]}
\end{align*}
using Lemma~\ref{lem:into-identities}.

\item[{$\qlam(b)[\Theta] \equiv \qlam(b[\Theta \uparrow A])$}:]
\begin{align*}
&\upstairs{\qlam(b)[\Theta]} \\
&\equiv (\UI{x}{\rewrite{\pinv{\alpha,x}}{\upstairs{b}[\into{\Delta, A}/\beta]}})[\upstairs{\Theta}/\alpha] \\
&\equiv \UI{x}{\rewrite{\pinv{\alpha,x}[\alpha/\alpha]}{\upstairs{b}[\into{\Delta, A}/\beta][\upstairs{\Theta}/\alpha, x / x]}} \\
&\equiv \UI{x}{\rewrite{\pinv{\alpha,x}}{\upstairs{b}[\upstairs{\Theta \uparrow A}/\beta][\into{\Gamma, A[\Theta]}/\beta]}} \\
&\equiv \upstairs{\qlam(b[\Theta \uparrow A])}
\end{align*}
again using Lemma~\ref{lem:into-identities}.

\item[{$\qapp{A,B}[\id_{\Gamma, A}, \qlam(M)[\proj{\Gamma, A}]] \equiv M$}:] 
\begin{align*}
&\upstairs{\qapp{A,B}[\id_{\Gamma, A}, \qlam(M)[\proj{\Gamma, A}]]} \\
\equiv{} &\upstairs{\qapp{A,B}}[\upstairs{\id_{\Gamma, A}, \qlam(M)[\proj{\Gamma, A}]}/\delta] \\
\equiv{} &\StI{\ApEl{p}{\eta^\chi_\beta}}{\upstairs{\qvar{\Gamma, A, \Pi_A B[\proj{\Gamma, A}]}}(\upstairs{\qvar{\Gamma, A}[\proj{\Gamma, A, \Pi_A B[\proj{\Gamma, A}]}]})}[\upstairs{\id_{\Gamma, A}, \qlam(M)[\proj{\Gamma, A}]}/\delta] \\
\equiv{} &\StI{\ApEl{p}{\eta^\chi_\beta}}{\upstairs{\qvar{\Gamma, A, \Pi_A B[\proj{\Gamma, A}]}}[\upstairs{\id_{\Gamma, A}, \qlam(M)[\proj{\Gamma, A}]}/\delta](\upstairs{\qvar{\Gamma, A}[\proj{\Gamma, A, \Pi_A B[\proj{\Gamma, A}]}]}[\upstairs{\id_{\Gamma, A}, \qlam(M)[\proj{\Gamma, A}]}/\delta])} \\
\equiv{} &\StI{\ApEl{p}{\eta^\chi_\beta}}{\upstairs{\qlam(M)[\proj{\Gamma, A}]}(\upstairs{\qvar{\Gamma, A}})} \\
\equiv{} &\StI{\ApEl{p}{\eta^\chi_\beta}}{\upstairs{\qlam(M[\proj{\Gamma, A} \uparrow A])}(\upstairs{\qvar{\Gamma, A}})} \\
\equiv{} &\StI{\ApEl{p}{\eta^\chi_\beta}}{(\UI{x}{\rewrite{\pinv{\beta,x}}{\upstairs{M[\proj{\Gamma, A} \uparrow A]}[\into{\Gamma, A, A[\proj{}]}/\beta]}})(\upstairs{\qvar{\Gamma, A}})} \\
\equiv{} &\StI{\ApEl{p}{\eta^\chi_\beta}}{(\UI{x}{\rewrite{\pinv{\beta,x}}{\upstairs{M}[\into{\Gamma, A}/\beta][\upstairs{\proj{\Gamma, A}}/\alpha, x/x]}})(\upstairs{\qvar{\Gamma, A}})} \\
\equiv{} &\StI{\ApEl{p}{\eta^\chi_\beta}}{\rewrite{\pinv{\beta,\One_\beta}}{\upstairs{M}[\into{\Gamma, A}/\beta][\upstairs{\proj{\Gamma, A}}/\alpha, \upstairs{\qvar{\Gamma, A}}/x]}} \\
\equiv{} &\rewrite{\ApPlus{\ApEl{p}{\eta^\chi_\beta}}{\pinv{\beta,\One_\beta}}}{\StI{\ApEl{p}{\eta^\chi_\beta}}{\upstairs{M}[\into{\Gamma, A}/\beta][\upstairs{\proj{\Gamma, A}}/\alpha, \upstairs{\qvar{\Gamma, A}}/x]}} \\
\equiv{} &\rewrite{\ApOne{\eta^\chi_\beta}}{\StI{\ApEl{p}{\eta^\chi_\beta}}{\upstairs{M}[\into{\Gamma, A}/\beta][\upstairs{\proj{\Gamma, A}}/\alpha, \upstairs{\qvar{\Gamma, A}}/x]}} \\
\equiv{} &\rewrite{\ApOne{\eta^\chi_\beta}}{\StI{\ApEl{p}{\eta^\chi_\beta}}{\upstairs{M}[\into{\Gamma, A}[\upstairs{\proj{\Gamma, A}}/\alpha, \upstairs{\qvar{\Gamma, A}}/x]/\beta]}} \\
\equiv{} &\upstairs{M}[\rewrite{\eta^\chi_\beta}{\into{\Gamma, A}[\upstairs{\proj{\Gamma, A}}/\alpha, \upstairs{\qvar{\Gamma, A}}/x]}/\beta] \\
\equiv{} &\upstairs{M}
\end{align*} 

\item[{$\qlam(\qapp{A,B}[\id_{\Gamma, A}, f[\proj{\Gamma, A}]]) \equiv f$}:] \begin{align*}
&\upstairs{\qlam(\qapp{A,B}[\id_{\Gamma, A}, f[\proj{\Gamma, A}]])} \\
\equiv{} &\UI{x}{\rewrite{\pinv{\alpha,x}}{\upstairs{\qapp{A,B}[\id_{\Gamma, A}, f[\proj{\Gamma, A}]]}[\into{\Gamma, A}/\beta]}} \\
\equiv{} &\UI{x}{\rewrite{\pinv{\alpha,x}}{\upstairs{\qapp{A,B}}[\upstairs{(\id_{\Gamma, A}, f[\proj{\Gamma, A}])}/\delta][\into{\Gamma, A}/\beta]}} \\
\equiv{} &\UI{x}{\rewrite{\pinv{\alpha,x}}{\StI{\eta^\chi_\delta}{\upstairs{\qvar{\Gamma, A, \Pi_A B[\proj{\Gamma, A}]}}(\upstairs{\qvar{\Gamma, A}[\proj{\Gamma, A, \Pi_A B[\proj{\Gamma, A}]}]})}[\upstairs{(\id_{\Gamma, A}, f[\proj{\Gamma, A}])}/\delta][\into{\Gamma, A}/\beta]}} \\
\equiv{} &\UI{x}{\rewrite{\pinv{\alpha,x}}{\StI{\eta^\chi_\beta}{\upstairs{\qvar{\Gamma, A, \Pi_A B[\proj{\Gamma, A}]}}[\upstairs{(\id_{\Gamma, A}, f[\proj{\Gamma, A}])}/\delta](\upstairs{\qvar{\Gamma, A}[\proj{\Gamma, A, \Pi_A B[\proj{\Gamma, A}]}]}[\upstairs{(\id_{\Gamma, A}, f[\proj{\Gamma, A}])}/\delta])}[\into{\Gamma, A}/\beta]}} \\
\equiv{} &\UI{x}{\rewrite{\pinv{\alpha,x}}{\StI{\eta^\chi_\beta}{\upstairs{f[\proj{\Gamma, A}]}(\upstairs{\qvar{\Gamma, A}})}[\into{\Gamma, A}/\beta]} \\
\equiv{} &\UI{x}{\rewrite{\pinv{\alpha,x}}{\StI{\eta^\chi_{\alpha.x}}{\upstairs{f[\proj{\Gamma, A}]}[\into{\Gamma, A}/\beta](\upstairs{\qvar{\Gamma, A}}[\into{\Gamma, A}/\beta]}}}} \\
\equiv{} &\UI{x}{\rewrite{\pinv{\alpha,x}}{\StI{\eta^\chi_{\alpha.x}}{\upstairs{f}[\rewrite{\pi^\alpha_x}{\alpha}/\alpha](\rewrite{\var{x}}{\StI{\ApEl{p}{\pi^{\alpha}_x}}{x}})}}} \\
\equiv{} &\UI{x}{\rewrite{\pinv{\alpha,x}}{\StI{\eta^\chi_{\alpha.x}}{\upstairs{f}(x)[\rewrite{\pi^\alpha_x}{\alpha}/\alpha,\rewrite{\var{x}}{\StI{\ApEl{p}{\pi^{\alpha}_x}}{x}}/x]}}} \\
\equiv{} &\UI{x}{\rewrite{\pinv{\alpha,x}}{\StI{\eta^\chi_{\alpha.x}}{\rewrite{\ap{\TrPlus{\pi^u_v}{\One_u}}{(\pi^\alpha_x,\var{x})/(u,v)}}{\StI{\ApEl{p}{\pi^\alpha_x \bdot \var{x}}}{\upstairs{f}(x)}}}}} \\
\equiv{} &\UI{x}{\rewrite{\pinv{\alpha,x}}{\rewrite{\ApPlus{\eta^\chi_{\alpha.x}}{\ap{\TrPlus{\pi^u_v}{\One_u}}{(\pi^\alpha_x,\var{x})/(u,v)}}}{\StI{\eta^\chi_{\alpha.x}}{\StI{\ApEl{p}{\pi^\alpha_x \bdot \var{x}}}{\upstairs{f}(x)}}}}} \\
\equiv{} &\UI{x}{\rewrite{\pinv{\alpha,x};\ApPlus{\eta^\chi_{\alpha.x}}{\ap{\TrPlus{\pi^u_v}{\One_u}}{(\pi^\alpha_x,\var{x})/(u,v)}}}{\StI{\eta^\chi_{\alpha.x}}{\StI{\ApEl{p}{\pi^\alpha_x \bdot \var{x}}}{\upstairs{f}(x)}}}} \\
\equiv{} &\UI{x}{\StI{\eta^\chi_{\alpha.x}}{\StI{\ApEl{p}{\pi^\alpha_x \bdot \var{x}}}{\upstairs{f}(x)}}} \\
\equiv{} &\UI{x}{\upstairs{f}(x)} \\
\equiv{} &\upstairs{f}
\end{align*}
\end{enumerate}

\subsubsection{Rules for a Dependent Right Adjoint}
\newcommand{\lock}{\mbox{\faUnlock}}
\newcommand{\Rtype}[1]{\mathsf{R}{#1}}
\newcommand{\RI}[1]{\mathsf{shut}({#1})}
\newcommand{\RE}[1]{\mathsf{open}({#1})}

Suppose our mode theory has two comprehension objects $p$ and $q$, with a morphism $f$ between them (Definition~\ref{def:morphism-comprehension-object}. The structural rules of MLTT can be interpreted in each comprehension object separately, so we mark turnstile with the comprehension object that the judgement corresponds to.

\begin{figure}
\begin{mathpar}
\inferrule*[left=ctx-$\lock$]{\qyields_p \Gamma \CTX}{\qyields_q \Gamma, \lock \CTX} \and
\inferrule*[left=sub-$\lock$]{\Gamma \qyields_p \Theta : \Delta}{\Gamma, \lock \qyields_q \Theta, \lock : \Delta, \lock} \\
\inferrule*[left=R-form]{\Gamma, \lock \qyields_q A \TYPE}{\Gamma \qyields_p \Rtype{A} \TYPE} \\
\inferrule*[left=R-intro]{\Gamma, \lock \qyields_q a : A}{\Gamma \qyields_p \RI{a} : \Rtype{A}} \and
\inferrule*[left=R-elim]{\Gamma \qyields_p b : \Rtype{B}}{\Gamma, \lock \qyields_q \RE{b} : B} \\
\end{mathpar}
\begin{align}
\id_\Gamma , \lock &\equiv \id_{\Gamma, \lock} \\
(\Theta, \lock) ; (\kappa, \lock) &\equiv (\Theta ; \kappa), \lock \\
\nonumber \\
\Rtype{A}[\Theta] &\equiv \Rtype{(A[\Theta, \lock])} \\
\RI{a}[\Theta] &\equiv \RI{a[\Theta, \lock]} \\
\nonumber \\
\RE{\RI{a}} &\equiv a \\
\RI{\RE{b}} &\equiv b
\end{align}
\caption{Rules for a Dependent Right Adjoint in MLTT via Explicit Substitutions}\label{fig:qit-pi-rules}
\end{figure}

\begin{enumerate}
\item[\textsc{ctx-$\lock$}] Given $\yields \upstairs{\Gamma} \TYPE_p$, we have $\yields \upstairs{\Gamma, \lock} :\equiv \St{f}{\upstairs{\Gamma}} \TYPE_q$

\item[\textsc{sub-$\lock$}] Follows because $\StI{f}{-}$ is a functor, explicitly:
\begin{align*}
\beta : \upstairs{\Gamma, \lock} \yields_\beta \upstairs{\Theta, \lock} :\equiv \StE{f}{\beta}{\alpha}{\StI{f}{\upstairs{\Theta}}} : \upstairs{\Delta, \lock}
\end{align*}

\item[\textsc{R-form}] We are given 
\begin{align*}
\beta : \St{f}{\upstairs{\Gamma}} \yields_{\El{q}{\beta}} \upstairs{A} \TYPE
\end{align*}
And define $\upstairs{\Rtype{A}}$ to be:
\begin{align*}
\upstairs{\Rtype{A}} :\equiv \U{f_1}{()}{\upstairs{A}[\StI{f}{\alpha}/\beta]}
\end{align*}

\item[\textsc{R-intro}] We are given 
\begin{align*}
\beta : \St{f}{\upstairs{\Gamma}} \yields_{\One_\beta} \upstairs{a} : \upstairs{A}
\end{align*}
Define:
\begin{mathpar}
\inferrule*[Left=U-intro]{
\inferrule*[Left=rewrite]{
\inferrule*[Left=Cut]
{\beta : \St{f}{\upstairs{\Gamma}} \yields_{\One_\beta} \upstairs{a} : \upstairs{A}}
{\alpha : \upstairs{\Gamma} \yields_{\One_{\TrPlus{f}{\alpha}}} \upstairs{a}[\StI{f}{\alpha}/\beta] : \upstairs{A}[\StI{f}{\alpha}/\beta]}}
{\alpha : \upstairs{\Gamma} \yields_{f_1(\One_\alpha)} \rewrite{\fone{\alpha}}{\upstairs{a}[\StI{f}{\alpha}/\beta]} : \upstairs{A}[\StI{f}{\alpha}/\beta]}}
{\alpha : \upstairs{\Gamma} \yields_{\One_\alpha} \UI{()}{\rewrite{\fone{\alpha}}{\upstairs{a}[\StI{f}{\alpha}/\beta]}} : \upstairs{\Rtype{A}}}
\end{mathpar}

\item[\textsc{R-elim}] Given
\begin{align*}
\alpha : \upstairs{\Gamma} \yields_{\One_\alpha} \upstairs{b} : \U{f_1}{()}{\upstairs{B}[\StI{f}{\alpha}/\beta]}
\end{align*}
we can do:
\begin{mathpar}
\inferrule*[Left=s-elim] {
\inferrule*[Left=rewrite] {
\inferrule*[Left=U-elim]
{\alpha : \upstairs{\Gamma} \yields_{\One_\alpha} \upstairs{b} : \U{f_1}{()}{\upstairs{B}[\StI{f}{\alpha}/\beta]}}
{\alpha : \upstairs{\Gamma} \yields_{f_1(\One_\alpha)} \UE{\upstairs{b}}{} : \upstairs{B}[\StI{f}{\alpha}/\beta]}}
{\alpha : \upstairs{\Gamma} \yields_{\One_{\TrPlus{f}{\alpha}}} \rewrite{\foneinv{\alpha}}{\UE{\upstairs{b}}{}} : \upstairs{B}[\StI{f}{\alpha}/\beta]}}
{\beta : \upstairs{\Gamma, \lock} \yields_{\One_\beta} \StE{f}{\beta}{\alpha}{\rewrite{\foneinv{\alpha}}{\UE{\upstairs{b}}{}}} : \upstairs{B}}
\end{mathpar}
\end{enumerate}

\begin{lemma}
If $f$ is a morphism of comprehension objects, then
\begin{align}
\StI{f}{\alpha}[\upstairs{\Theta}/\alpha] \equiv \upstairs{\Theta, \lock}[\StI{f}{\alpha}/\beta]
\end{align}
\end{lemma}
\begin{proof}
\begin{align*}
&\StI{f}{\alpha}[\upstairs{\Theta}/\alpha] \\
&\equiv \StI{f}{\upstairs{\Theta}} \\
&\equiv \StE{f}{\StI{f}{\alpha}}{\alpha}{\StI{f}{\upstairs{\Theta}}}\\
&\equiv \StE{f}{\beta}{\alpha}{\StI{f}{\upstairs{\Theta}}}[\StI{f}{\alpha}/\beta] \\
&\equiv \upstairs{\Theta, \lock}[\StI{f}{\alpha}/\beta] \\
\end{align*}
\end{proof}

\begin{enumerate}[style = multiline, labelwidth = 80pt]
\item[{$\id_\Gamma , \lock \equiv \id_{\Gamma, \lock}$}:] \mvrnote{This is why $f$ is given by a mode type morphism, we need definitional eta}
\mvrnote{These two should just be a general lemma about s-types being functorial}
\begin{align*}
\upstairs{\id_\Gamma , \lock} 
&\equiv \StE{f}{\beta}{\alpha}{\StI{f}{\upstairs{\id_\Gamma}}} \\ 
&\equiv \StE{f}{\beta}{\alpha}{\StI{f}{\alpha}} \\ 
&\equiv \beta \\
&\equiv \upstairs{\id_{\Gamma , \lock}} 
\end{align*}
\item[{$(\Theta, \lock) ; (\kappa, \lock) \equiv (\Theta ; \kappa), \lock$}:] 
\begin{align*}
&\upstairs{(\Theta, \lock) ; (\kappa, \lock)} \\
&\equiv \upstairs{\kappa, \lock}[\upstairs{\Theta, \lock}/\beta] \\
&\equiv \StE{f}{\beta}{\alpha}{\StI{f}{\upstairs{\kappa}}}[\upstairs{\Theta, \lock}/\beta] \\
&\equiv \StE{f}{\upstairs{\Theta, \lock}}{\alpha}{\StI{f}{\upstairs{\kappa}}} \\
&\equiv \StE{f}{(\StE{f}{\beta}{\alpha}{\StI{f}{\upstairs{\Theta}}})}{\alpha}{\StI{f}{\upstairs{\kappa}}} \\
&\equiv \StE{f}{\beta}{\alpha'}{\StE{f}{(\StE{f}{\StI{f}{\alpha'}}{\alpha}{\StI{f}{\upstairs{\Theta}}})}{\alpha}{\StI{f}{\upstairs{\kappa}}}} \\
&\equiv \StE{f}{\beta}{\alpha'}{\StE{f}{\StI{f}{\upstairs{\Theta}[\alpha'/\alpha]}}{\alpha}{\StI{f}{\upstairs{\kappa}}}} \\
&\equiv \StE{f}{\beta}{\alpha'}{\StI{f}{\upstairs{\kappa}}[\upstairs{\Theta}[\alpha'/\alpha]/\alpha]} \\
&\equiv \StE{f}{\beta}{\alpha}{\StI{f}{\upstairs{\kappa}}[\upstairs{\Theta}/\alpha]} \\
&\equiv \StE{f}{\beta}{\alpha}{\StI{f}{\upstairs{\Theta;\kappa}}} \\
&\equiv \upstairs{(\Theta;\kappa), \lock}
\end{align*}

\item[{$\Rtype{A}[\Theta] \equiv \Rtype{(A[\Theta, \lock])}$}:] 
\begin{align*}
\upstairs{\Rtype{A}[\Theta]}
&\equiv \upstairs{\Rtype{A}}[\upstairs{\Theta}/\alpha] \\
&\equiv \U{f_1}{()}{\upstairs{A}[\StI{f}{\alpha}/\beta]}[\upstairs{\Theta}/\alpha] \\
&\equiv \U{f_1}{()}{\upstairs{A}[\StI{f}{\alpha}/\beta][\upstairs{\Theta}/\alpha]} \\
&\equiv \U{f_1}{()}{\upstairs{A}[\upstairs{\Theta, \lock}/\beta][\StI{f}{\alpha}/\beta]} \\
&\equiv \U{f_1}{()}{\upstairs{A[\Theta, \lock]}[\StI{f}{\alpha}/\beta]} \\
&\equiv \upstairs{\Rtype{(A[\Theta, \lock])}} 
\end{align*}

\item[{$\RI{a}[\Theta] \equiv \RI{a[\Theta, \lock]}$}:] 
\begin{align*}
\upstairs{\RI{a}[\Theta]}
&\equiv \upstairs{\RI{a}}[\upstairs{\Theta}/\alpha] \\
&\equiv \UI{()}{\rewrite{\fone{\alpha}}{\upstairs{a}[\StI{f}{\alpha}/\beta]}}[\upstairs{\Theta}/\alpha] \\
&\equiv \UI{()}{\rewrite{\fone{\alpha}}{\upstairs{a[\Theta, \lock]}[\StI{f}{\alpha}/\beta]}} \\
&\equiv \upstairs{\RI{a[\Theta, \lock]}}
\end{align*}

\item[{$\RE{\RI{a}} \equiv a$}:] 
\begin{align*}
&\upstairs{\RE{\RI{a}}} \\
&\equiv \StE{f}{\beta}{\alpha}{\rewrite{\foneinv{\alpha}}{\UE{\upstairs{\RI{a}}}{}}} \\
&\equiv \StE{f}{\beta}{\alpha}{\rewrite{\foneinv{\alpha}}{\UE{\UI{()}{\rewrite{\fone{\alpha}}{\upstairs{a}[\StI{f}{\alpha}/\beta]}}}{}}} \\
&\equiv \StE{f}{\beta}{\alpha}{\rewrite{\foneinv{\alpha}}{\rewrite{\fone{\alpha}}{\upstairs{a}[\StI{f}{\alpha}/\beta]}}} \\
&\equiv \StE{f}{\beta}{\alpha}{\upstairs{a}[\StI{f}{\alpha}/\beta]} \\
&\equiv \upstairs{a}[\beta/\beta] \\
&\equiv \upstairs{a}
\end{align*}

\item[{$\RI{\RE{b}} \equiv b$}:] 
\begin{align*}
&\upstairs{\RI{\RE{b}}} \\
&\equiv \UI{()}{\rewrite{\fone{\alpha}}{\upstairs{\RE{b}}[\StI{f}{\alpha}/\beta]}} \\
&\equiv \UI{()}{\rewrite{\fone{\alpha}}{\StE{f}{\beta}{\alpha}{\rewrite{\foneinv{\alpha}}{\UE{\upstairs{b}}{}}}[\StI{f}{\alpha}/\beta]}} \\
&\equiv \UI{()}{\rewrite{\fone{\alpha}}{\StE{f}{\StI{f}{\alpha}}{\alpha}{\rewrite{\foneinv{\alpha}}{\UE{\upstairs{b}}{}}}}} \\
&\equiv \UI{()}{\rewrite{\fone{\alpha}}{\rewrite{\foneinv{\alpha}}{\UE{\upstairs{b}}{}}}} \\
&\equiv \UI{()}{\UE{\upstairs{b}}{}} \\
&\equiv \upstairs{b}
\end{align*}
\end{enumerate}

\section{Semantics}
\label{sec:semantics}

\subsection{2-categories with families}
\label{sec:2cwfs}

The ``canonical'' semantics should interpret each judgment as follows:

Mode theory judgements:
\begin{enumerate}
\item $\mm{\gamma \ctx}$ is a category.
\item $\mm{\gamma \yields p \type}$ is a functor $\mm{\gamma}\op \to \Cat$.
\item $\mm{\TypeTwo{\gamma}{s}{p}{q}}$ is a natural transformation $\mm{\gamma \yields q} \Rightarrow \mm{\gamma \yields p}$ (note reversal of direction; this is because mode morphisms act contravariantly on mode terms and on upstairs subscripts).
\item $\mm{\gamma \yields \mu : p}$ is a section of the projection from the Grothendieck construction $\int\mm{\gamma\yields p} \to \mm{\gamma}$.
  In particular, it assigns to every object $x\in \mm{\gamma}$ an object $\mm{\gamma \yields \mu : p}(x)\in \mm{\gamma\yields p}(x)$.
\item $\mm{\TermTwoT{\gamma}{s}{\mu}{\nu}{p}}$ is a natural transformation of such sections over the identity, i.e.\ whose composite with the projection is the identity natural transformation of the identity functor.
\end{enumerate}

Top judgements: 
\begin{itemize}
\item $\mm{\yields_\gamma \Gamma \CTX}$ is an object of $\mm{\gamma}$
\item $\mm{\Gamma \yields_p A \TYPE}$ is an object of $\mm{\gamma \yields p}(\mm{\Gamma})$.
\item $\mm{\Gamma \yields_\mu M : A}$ is a morphism from $\mm{\gamma \yields \mu : p}(\mm{\Gamma})$ to $\mm{\Gamma \yields_p A}$ in $\mm{\gamma \yields p}(\mm{\Gamma})$.
\end{itemize}

\drlnote{Are these right?}
\msnote{Yes!}
Total substitutions in the mode theory, 2-cells between them, and
upstairs substitutions above them are not part of the primitive syntax
of the framework, but can be defined by tupling terms.
These can be interpreted as: 
\begin{itemize}
\item $\mm{\gamma \yields \theta : \delta}$ is a functor from $\mm{\gamma}$ to $\mm{\delta}$
\item $\mm{\gamma \yields \theta_1 \vDash_\delta \theta_2}$ is a natural
  transformation from $\mm{\gamma \yields \theta_1 : \delta}$ to
  $\mm{\gamma \yields \theta_2 : \delta}$
\item $\mm{\Gamma_{\gamma} \yields_\theta \Theta : \Delta_\delta}$ is a
  morphism from $\mm{\theta}(\mm{\Gamma})$ to $\mm{\Delta}$ in
  $\mm{\delta}$.
  Natural transformations act contravariantly because $\theta$ is in the
  domain of the morphism.  
  \end{itemize}
So the semantics of contexts/substitutions is a lot like 
one-variable adjoint logic with only $F$ types~\cite{ls15adjoint}.  

The general categorical semantics is an abstraction of these structures --- categories, contravariant $\Cat$-valued functors, Grothendieck constructions, sections, natural transformations, objects, and morphisms.
In this subsection we will concern ourselves only with the ``downstairs'' mode theory, which means abstracting the behavior of Grothendieck constructions in $\Cat$; in \S\ref{sec:fib-2cwf} we will reintroduce the ``upstairs'' type theory by additionally abstracting the behavior of ``objects and morphisms''.

We work with two universes in the metatheory, leading to two categories of categories denoted $\Cat$ and $\CAT$, such that $\Cat$ is an object of $\CAT$.
If necessary, we refer to objects of $\Cat$ as \emph{small}, objects of $\CAT$ (including $\Cat$) as \emph{large}, and other categories (such as $\CAT$ itself) as \emph{very large}.
However, for the most part size issues can be ignored.

\begin{enumerate}
\item Categories form a 2-category.
  In general we will stipulate an arbitrary (strict) 2-category $\M$.
\item Given a category $C$, the collection of functors and (strict) natural transformations $C\op \to \Cat$ forms a (large) category $[C\op, \Cat]$, and if we reverse the directions of the natural transformations we get $[C\op, \Cat]\op$.
  Moreover, precomposition with functors $C\to C'$ and natural transformations between them makes $[(-)\op, \Cat]\op$ into a strict 2-functor $\Cat\op \to \CAT$; syntactically these are $q[\mu/x]$ and $\ap{q}{\mu/x}$ respectively.

  Here $\CAT$ denotes the very large 2-category of large categories.
  Note that this 2-functor is covariant on 2-cells: the two $(-)\op$s cancel each other out at that level.
  (In fact, this 2-functor can be identified with the representable $[-,\Cat\op]$; this will be useful below.)

  Thus, in general we will stipulate a strict 2-functor $\Mty:\M\op \to \CAT$.
  \addtocounter{enumi}{1}
\item Given a category $C$ and a functor $T:C\op\to Cat$, the sections of the projection $\int T \to C$, and natural transformations over the identity, form a category.
  Moreover, such sections also vary functorially as $C$ does.
  Thus, in general we will stipulate another strict 2-functor $\Mtm : \M\op\to\CAT$ with a strictly 2-natural projection map $\Mtm\to\Mty$.

  The contravariant action of mode morphisms $\TypeTwo{\gamma}{s}{p}{q}$ on mode terms tells us that the morphisms of $\Mty(C)$ must act on the objects of $\Mtm$ contravariantly.
  Moreover, this action is strictly functorial, and respected by substitution.
  Thus, we stipulate that $\Mtm \to \Mty$ is a \emph{split fibration} internal to the 2-category $[\M\op,\CAT]$, which means that each functor $\Mtm(C) \to \Mty(C)$ is a split fibration and that all the naturality squares
  \begin{center}
    \begin{tikzcd}
      \Mtm(C) \ar[d] \ar[r] & \Mtm(C')\ar[d] \\
      \Mty(C) \ar[r] & \Mty(C')
    \end{tikzcd}
  \end{center}
  are strict morphisms of split fibrations (preserve the splittings on the nose).

  To represent the Grothendieck construction $\int T$ itself, we stipulate that this projection is additionally a \emph{representable morphism} in that for any $C\in \M$, if we form the pullback in $[\M\op,\CAT]$
  \begin{equation}
    \begin{tikzcd}
      \int T \ar[d] \ar[r] & \Mtm \ar[d] \\
      y(C) \ar[r,"\name{T}"] & \Mty
    \end{tikzcd}\label{eq:rep-pb}
  \end{equation}
  then the pullback object is also of the form $y(\int T)$ for some object $\int T\in \M$.
  Here $y(C)$ denotes the representable functor $y(C)(-) = \M(-,C) : \M\op\to\CAT$.
  The Yoneda lemma implies that (strict) 2-natural transformations $\name{T} : y(C) \to \Mty$ are in bijection with objects $T\in \Mty(C)$; thus every $T\in \Mty(C)$ induces an object $\int T$ with a projection $\int T \to C$ in $\M$, such that elements of $\Mtm(C)$ over $T$ are in bijection with sections of this projection.
  To make the notion algebraic, we require the object $\int T$ to be a specified function of $C$ and $T$; we call this being \textbf{algebraically representable}.

  Since~\eqref{eq:rep-pb} is a pullback in a 2-category, it also has a universal property for 2-cells.
  Thus, morphisms in $\Mtm$ (over the identity in $\Mty$) are in bijection with 2-cells between sections (over the identity).

  The 2-categorical Yoneda lemma also says that morphisms $\mu : S\to T$ in $\Mty(C)$ correspond bijectively to modifications $\name{\mu}:\name{S} \to \name{T}$.
  Since $\Mtm\to\Mty$ is a fibration, such a $\name{\mu}$ induces a map in the other direction $\int \mu : \int T\to \int S$ (see Hermida, Buckley, Johnstone), such that postcomposing with $\int \mu$ corresponds to the split (contravariant) fibrational action of $\mu$ on elements of $\Mtm$.
\end{enumerate}

Thus, the entire mode theory except for 1- and $\Sigma$-modes is encapsulated semantically by:

\begin{definition}
  A \textbf{2-category with families} is a 2-category $\M$ together with two 2-functors $\Mty,\Mtm : \M\op\to\CAT$ and a 2-natural transformation $\Ups:\Mtm\to \Mty$ that is both (1) an internal split fibration and (2) algebraically representable.
\end{definition}

Pleasingly, this is a straightforward categorification of the standard notion of category with families.
The only really new ingredient is the requirement that $\Mtm\to \Mty$ be an internal fibration, which has no analogue for 1-categories.

The ruminations above suggest that there should be a 2-category with families where $\M=\Cat$ and $\Mty(C) = [C\op,\Cat]\op$.
In fact we can construct this precisely as an instance of a much more general operation, similar to the ``global universe'' coherence method for ordinary type theory introduced by Voevodsky.
(There should also be a ``local universe'' method analogous to that of Lumsdaine--Warren, but we do not need that here.)

\begin{definition}
  A \textbf{2-category with a universe} is a large 2-category $\M$ together with a fully faithful 2-functor $y:\M \hookrightarrow \Mhat$, where $\Mhat$ is a very large and locally large 2-category, and a morphism $\Upshat : \Mtmhat \to \Mtyhat$ in $\Mhat$ that is both (1) an internal split fibration and (2) algebraically $\M$-representable, in the sense that for any map $y(C)\to \Mtyhat$, where $C\in \M$, has a specified pullback that also lies in the image of $y$.
\end{definition}

Usually we will treat $y$ as an implicit coercion, identifying objects of $\M$ with their images in $\Mhat$.

Note that every 2-category with families is also a 2-category with a universe, taking $\Mhat = [\M\op,\CAT]$ and $\Mtyhat=\Mty$, $\Mtmhat = \Mtm$, $\Upshat = \Ups$.
Indeed a 2-category with families is precisely a 2-category with a universe such that $\M \hookrightarrow \Mhat$ is the Yoneda embedding.
Conversely we have:

\begin{theorem}\label{thm:2cwf-univ}
  Any 2-category with a universe has an underlying 2-category with families, defined by
  \begin{align*}
    \Mty(C) &= \Mhat(C,\Mtyhat)\\
    \Mtm(C) &= \Mhat(C,\Mtmhat)\\
    \Ups &= \Mhat(C,\Upshat)
  \end{align*}
\end{theorem}
\begin{proof}
  The definitions of $\Mty$ and $\Mtm$ are certainly 2-functors, since they are representable, and the morphism $\Upshat$ induces a 2-natural transformation $\Ups:\Mtm\to\Mty$.
  More abstractly, we are applying the \emph{restricted Yoneda embedding} $\Mhat \to [\M\op,\CAT]$ (the term is somewhat of a misnomer since, unlike the ordinary Yoneda embedding, it may not be fully faithful).
  Since split fibrations are defined representably, they are preserved by the restricted Yoneda embedding; thus $\Ups$ is also a split fibration.
  Finally, since the restricted Yoneda embedding also preserves pullbacks, it preserves any pullback of $\Upshat$ to an object of $\M$; hence the vertex of any such pullback is again a specified representable.
  Thus, $\Ups$ is algebraically representable.
\end{proof}

\begin{example}\label{eg:cat-2cwf}
  Let $\M=\Cat$ and $\Mhat=\CAT$, with $\Mtyhat = \Cat\op$, and $\Mtmhat$ the Grothendieck construction of the contravariant functor $(\Cat\op)\op \cong \Cat \hookrightarrow \CAT$.
  Explicitly, the objects of $\Mtmhat$ are categories $A$ equipped with an object $a\in A$, and its morphisms $(A,a) \to (B,b)$ are functors $f:B\to A$ paired with a morphism $\phi : a\to f(b)$.
  Thus, $\Cat$ becomes a 2-category with a universe, and hence a 2-category with families, as in the above motivating discussion.
  We have $\Mty(C) \cong [C,\Cat\op] \cong [C\op,\Cat]\op$, and $\Mtm(C)$ consists of functors $D:C\op\to\Cat$ together with a section $s:C\to \int D$ of their Grothendieck construction.
\end{example}

On the other hand, we should also have the syntactic model:

\begin{example}\label{eg:syn-2cwf}
  Let $\M$ be the 2-category whose:
  \begin{itemize}
  \item objects are mode contexts $\gamma \ctx$,
  \item morphisms are total mode substitutions $\gamma \yields \theta : \delta$, and
  \item 2-cells are total mode 2-cells $\gamma \yields \theta_1 \vDash_\delta \theta_2$.
  \end{itemize}
  We define $\Mty:\M\op\to\CAT$ such that
  \begin{itemize}
  \item the objects of $\Mty(\gamma)$ are modes $\gamma \yields p\type$, and
  \item the morphisms of $\Mty(\gamma)$ are mode morphisms $\TypeTwo{\gamma}{s}{p}{q}$,
  \item with functorial action by substitution.
  \end{itemize}
  Finally, we define $\Mtm:\M\op\to\CAT$ such that
  \begin{itemize}
  \item the objects of $\Mtm(\gamma)$ are mode terms $\gamma \yields \mu:p$, and
  \item the morphisms of $\Mtm(\gamma)$ are mode 2-cells $\TermTwoT{\gamma}{s}{\mu}{\nu}{p}$,
  \item with functorial action by substitution.
  \end{itemize}
  The projection $\Ups:\Mtm\to\Mty$ is clear.
  Its representability is given, as usual, by context extension $(\gamma, x:p) \ctx$; the universal property of this follows from the definition of total substitutions and 2-cells as tuples.
  Finally, its split fibration structure is given by the contravariant action of mode morphisms on mode terms ${\gamma \yields \TrPlus{s}{\mu} : p}$.
  Note that, as mentioned above, so far we are only seeing the ``downstairs'' mode theory.
\end{example}


\subsection{$\Sigma$-modes}
\label{sec:2-sigmas}

Emboldened by this success, let's just try to write down a notion of $\Sigma$-types for 2-categories with families by categorifying the usual one for 1-categories.
As formulated by Awodey, $\Sigma$-types for a 1-category with families involve the dependent product of the presheaf $\mathrm{Ty}$ along the projection $\mathrm{Tm}\to \mathrm{Ty}$, to encode the notion of one type dependent on another one.
Unlike a presheaf 1-category, the presheaf 2-category $[\M\op,\CAT]$ is not locally cartesian closed, so not all dependent products exist; but fortunately, dependent products along fibrations do exist.

\begin{lemma}\label{thm:fib-exp}
  For a 2-category $\M$, any split fibration $\Ups : S\to T$ in $[\M\op,\CAT]$ is exponentiable.
\end{lemma}
\begin{proof}
  Suppose given $R\to S$; we want to define $\Pi_S[R]$ over $T$.
  Given an object $x\in T(C)$, corresponding to a morphism $\name{x}:y(C)\to T$, the objects of $\Pi_S[R](C)$ over $x$ must be bijective to the lifts of $\name{x}$ to $\Pi_S [R]$.
  The desired universal property of $\Pi_S [R]$ means that such lifts are bijective to maps $x^* S \to R$ over $S$.
  So we take the latter as the \emph{definition} the objects of $\Pi_S [R]$, i.e.\
  \[ \ob (\Pi_S [R](C)) = \sum_{x:\ob (T(C))} \ob (\M_{/S}(x^*S, R)). \]
  Next, given a morphism $\alpha :x\to y$ in $T(C)$ and objects $u,v$ of $\Pi_S[R]$ over $x,y$ respectively, morphisms $u\to v$ over $\alpha$ must correspond bijectively to lifts of the corresponding 2-cell $\name{\alpha} :\name{x} \to \name{y} :y(C) \to T$ to a 2-cell between $\name{u},\name{v} : y(C) \to \Pi_S[R]$.
  However, the desired universal property of $\Pi_S[R]$ as a dependent product doesn't obviously determine such lifts, because the 2-cell $\name{\alpha}$ doesn't live in the slice category $\M/T$.
  But we can use the fibration structure of $\Ups$, which induces a map $\alpha^*S : y^*S \to x^*S$ over $y(C)$ and a 2-cell
  \begin{center}
    \begin{tikzcd}
      x^*S \ar[drr] \\
      & \Downarrow\overline{\alpha} & S\\
      y^*S \ar[urr] \ar[uu,"\alpha^*S"]
    \end{tikzcd}
  \end{center}
  over $\alpha$ with a universal property (see Hermida, Buckley, Johnstone).
  Thus, if $u$ and $v$ are determined by maps $\hat{u}:x^*S \to R$ and $\hat{v}:y^*S\to R$ over $S$, we can define a morphism $u\to v$ over $\alpha$ to be determined by a lift $\hat{\alpha}$ of $\overline{\alpha}$ to $R$.

  The maps $\alpha^*S $ and 2-cells $\overline{\alpha}$ are functorial in $\alpha$ in all the ways one would hope (strictly so, since $\Ups$ is split --- although this is not really necessary), so it is straightforward to make $\Pi_S[R](C)$ thusly defined into a category and $\Pi_S[R]$ into a functor $\M\op\to\CAT$ over $T$.

  Note in particular that when $\alpha$ is an identity, so is $\overline{\alpha}$, so the fiber of $\Pi_S[R]$ over an object $x$ is the category $\M_{/S}(x^*S, R)$.
  This means that $\Pi_S[R]$ has the desired universal property with respect to maps out of representables, i.e.\ we have a natural isomorphism of hom-categories
  \[\M_{/T}(x:y(C)\to T,\Pi_S[R]) \cong \M_{/S}(x^*S, R).\]
  It is straightforward to extend this, using the Yoneda lemma, to the full universal property.
\end{proof}

Proceeding by analogy with the 1-categorical case, consider the dependent product $\Pi_\Ups[\Mty]$, where $\Mty$ denotes abusively the pullback $\Mtm \times \Mty \to \Mtm$ in $\M/\Mtm$.
The universal structure of $\Pi_\Ups[\Mty]$ consists of an ``evaluation'' map $\ev:\Pi_\Ups[\Mty] \times_\Mty \Mtm \to \Mty$.
Note that the projection $\Pi_\Ups[\Mty] \times_\Mty \Mtm \to \Pi_\Ups[\Mty]$ is a split fibration, since it is a pullback of $\Ups:\Mtm \to \Mty$.
And we can also pull $\Ups$ back along the evaluation map to get a further split fibration:
\begin{center}
  \begin{tikzcd}
    \ev^* \Mtm \ar[r] \ar[d,->>] \ar[dr,phantom,near start,"\lrcorner"] & \Mtm \ar[d,->>]\\
    \Pi_\Ups[\Mty] \times_\Mty \Mtm \ar[r] \ar[d,->>] \ar[ddr,phantom,near start,"\lrcorner"] \ar[dr] & \Mty\\
    \Pi_\Ups[\Mty] \ar[dr] & \Mtm \ar[d,->>] \\
    & \Mty
  \end{tikzcd}
\end{center}
Note that the composite of two split fibrations is again a split fibration.

\begin{definition}
  A 2-category with a universe (such as a 2-category with families) has \textbf{$\Sigma$-types} if $\Upshat$ is exponentiable, and it is equipped with maps $\Pi_\Upshat[\Mtyhat] \to \Mtyhat$ and $\ev^* \Mtmhat \to \Mtmhat$ such that the square
  \begin{center}
    \begin{tikzcd}
      \ev^* \Mtmhat \ar[r] \ar[d,->>] & \Mtmhat \ar[dd,->>]\\
      \Pi_\Upshat[\Mtyhat] \times_\Mtyhat \Mtmhat \ar[d,->>] \\
      \Pi_\Upshat[\Mtyhat] \ar[r] & \Mtyhat
    \end{tikzcd}
  \end{center}
  (1) commutes, (2) is a pullback, and (3) is a strict morphism of split fibrations.
\end{definition}

\begin{example}\label{eg:syn-sig}
  Continuing Example \ref{eg:syn-2cwf}, we argue that the syntactic model has $\Sigma$-types in this sense if and only if it has $\Sigma$-modes in the syntactic sense.
By the construction in Lemma \ref{thm:fib-exp}, an object of $\Pi_\Ups[\Mty](\gamma)$ lives over a type $p\in \Mty(\gamma)$, and consists of the additional data of a map $p^*\Mtm \to \Mty$.
But since $\Ups$ is representable, $p^* \Mtm$ is also representable, by the extended context $\gamma.p$; thus this additional data is a type $\gamma,x:p \yields q \type$.
The morphism $\Pi_\Ups[\Mty] \to \Mty$ thus assigns to any such pair a mode type $\gamma \yields \sigmacl{x}{p}{q} \type$.

Similarly, an object of $\ev^* \Mtm$ over $(p,q)$ consists of a pair of a term $\gamma \yields \mu : p$ and $\gamma \yields \nu : q[\mu/x]$, and so the morphism $\ev^* \Mtm$ assigns to this the pair $(p,q):\sigmacl{x}{p}{q}$.
The fact that the square is a pullback (on objects) means that we have $\fst$ and $\snd$ collectively forming an inverse isomorphism, i.e.\ the $\beta$- and $\eta$-rules hold for $\Sigma$-modes.

The action of $\Pi_\Ups[\Mty] \to \Mty$ on morphisms gives the congruence rules for $\Sigma$ on mode type morphisms, along with its functoriality laws (since this map over each $\gamma$ is a functor) and its naturality law for substitution (since this map is a natural transformation as $\gamma$ varies).
Similarly, the action of $\ev^* \Mtm \to \Mtm$ on morphisms gives the 2-cell operation on $\Sigma$-modes (although the latter is currently written to only apply when one of the morphisms being paired is cartesian), and the fact that the square is a pullback on morphisms gives the $\beta$- and $\eta$-rules for these.

Finally, the fact that this square is a strict morphism of split fibrations gives the equation for ``transport'' in $\Sigma$.
\end{example}

To construct our canonical model, we again appeal to a universal case.

\begin{theorem}\label{thm:sig-univ}
  If a 2-category with a universe has $\Sigma$-types, then so does its underlying 2-category with families (as in Theorem \ref{thm:2cwf-univ}).
\end{theorem}
\begin{proof}
  All the structure involved in the definition of $\Sigma$-types (dependent products, pullbacks, morphisms of split fibrations) is preserved by any restricted Yoneda embedding.
\end{proof}

\begin{example}\label{eg:cat-sig}
  Continuing Example \ref{eg:cat-2cwf}, we want to show that $\Mtyhat=\Cat\op$ has $\Sigma$-types in the sense of Theorem \ref{thm:sig-univ}.
  The map $\Upshat:\Mtmhat\to \Mtyhat$ is exponentiable because it is a split fibration (all fibrations are exponentiable in $\CAT$).
  An object of $\Pi_\Upshat[\Mtyhat]$ consists of a category $A\in \Cat\op$ together with a functor from the fiber of $\Upshat$ over $A$ to $\Mtyhat$, which is to say simply a functor $A\to \Cat\op$, or equivalently $B:A\op\to\Cat$.
  Similarly, a morphism $(A,B)\to (A',B')$ in $\Pi_\Upshat[\Mtyhat]$ consists of a functor $f:A'\to A$ together with a natural transformation $g:B' \to B \circ f$.

  The fiber of $\Pi_\Upshat[\Mtyhat] \times_\Mtyhat \Mtmhat$ over $(A,B)\in \Pi_\Upshat[\Mtyhat]$ is just the category $A$, and its contravariant split fibrational action on $(f,g)$ is just the action of the functor $f$.
  Thus, a morphism therein from $(A,B,x)$ to $(A',B',x')$ consists of $f:A'\to A$ and $g:B' \to B \circ f$ together with $\xi : x \to f(x')$.

  The functor $\ev$ sends $(A,B,x)$ to the category $B(x)$.
  Thus, the fiber of $\ev^*\Mtmhat$ over $(A,B,x)$ is just the category $B(x)$, with contravariant split fibrational action on $(f,g,\xi)$ given by the components of $g$.
  So a morphism from $(A,B,x,y)$ to $(A',B',x',y')$ consists of $f,g,\xi$ and $\zeta : y \to B(\xi)(g_{x'}(y'))$.

  It follows that the fiber of the composite $\ev^*\Mtmhat \to \Pi_\Upshat[\Mtyhat] \times_\Mtyhat \Mtmhat \to \Pi_\Upshat[\Mtyhat]$ over $(A,B)$ is (isomorphic to) the category of pairs $(x,y)$ with $x\in A$ and $y\in B(x)$, and with morphisms $(x,y)\to (x',y')$ being pairs $(\xi,\zeta)$ where $\xi:x\to x'$ and $\beta: y \to B(\xi)(y')$.
  The split fibrational action of a morphism $(f:A'\to A,g:B' \to B \circ f)$ sends $(x',y')$ to $(f(x'),g(y'))$, with a similar action on morphisms.
  Thus, we can define the desired functor $\Pi_\Upshat[\Mtyhat] \to \Mtyhat = \Cat\op$ by sending $(A,B)$ to precisely this category of pairs $(x,y)$, which is none other than the ordinary Grothendieck construction of $B:A\op\to \Cat$.
  Defining this functor on morphisms by the above split fibrational action, we find tautologically that the desired square is both a pullback and a morphism of split fibrations.
  (More abstractly, we are simply using the fact that $\Mtmhat\to \Mtyhat$ is a classifier in $\CAT$ of split fibrations with small fibers.)

  Thus, the canonical 2-category with families structure on $\Cat$ has $\Sigma$-types.
\end{example}


\subsection{Fibered 2-categories with families}
\label{sec:fib-2cwf}

As in our previous work, with the 2-categorical analogue of the downstairs mode theory in place, the upstairs type theory corresponds to a ``local fibration'' over it.
To define this, we first need to talk about morphisms between 2-categories with families, or more generally with universes.

\begin{definition}\label{defn:fibmor}
  Let $\vp : \mathcal{A}\to \mathcal{B}$ be any 2-functor, and $p:S\to T$ and $q:U\to V$ be split fibrations in $\mathcal{A}$ and $\mathcal{B}$ respectively.
  A \textbf{strict $\vp$-morphism from $p$ to $q$} consists of a commutative square in $\mathcal{B}$:
  \begin{equation}\label{eq:fibmor}
    \begin{tikzcd}
      \vp S \ar[r] \ar[d] & U\ar[d]\\
      \vp T \ar[r] & V
    \end{tikzcd}
  \end{equation}
  such that for any $X\in \mathcal{A}$, the induced commutative square of categories
  \begin{equation}\label{eq:fibmor2}
    \begin{tikzcd}
      \mathcal{A}(X,S) \ar[r] \ar[d] & \mathcal{B}(\vp X, \vp S) \ar[r] \ar[d] & \mathcal{B}(\vp X, U) \ar[d] \\
      \mathcal{A}(X,T) \ar[r] & \mathcal{B}(\vp X, \vp T) \ar[r] & \mathcal{B}(\vp X, V)
    \end{tikzcd}
  \end{equation}
  is a strict morphism of split fibrations.
\end{definition}

If $\vp$ has a right adjoint $\vpst$, then~\eqref{eq:fibmor} is equivalent to a square in $\mathcal{A}$:
\begin{equation}\label{eq:fibmor-mate}
  \begin{tikzcd}
    S \ar[r] \ar[d] & \vpst U\ar[d]\\
    T \ar[r] & \vpst V
  \end{tikzcd}
\end{equation}
and the condition in Definition \ref{defn:fibmor} is equivalent to asking that~\eqref{eq:fibmor-mate} be a strict morphism of internal split fibrations in $\mathcal{A}$.
This makes sense since $\vpst$, as a right adjoint, preserves limits and hence split fibrations, while an analogous phrasing for~\eqref{eq:fibmor} would not since $\vp$ is not assumed to preserve fibrations.
However, if $\vp$ \emph{does} preserve split fibrations, in the algebraic sense that it preserves the chosen cartesian lifts, and if~\eqref{eq:fibmor} is a strict morphism of split fibrations internal to $\M$, then it is also a strict $\vp$-morphism (but the converse need not hold).

% Suppose $\C$ and $\M$ are 2-categories with universes and $\vp:\C\to\M$ is a 2-functor.
% Since $\C$ is large and $\Mhat$ has large colimits, the composite $\C \xrightarrow{\vp} \M \hookrightarrow \Mhat$ has a pointwise left Kan extension along the inclusion $\C\hookrightarrow \Chat$; we denote it by $\vp_! : \Chat \to \Mhat$.
% Since $\C\hookrightarrow \Chat$ is fully faithful, $\vp_!$ is an honest extension, i.e.\ it restricts to $\vp$ on $\C$.

% If $\C$ is a 2-category with families, $\vp_!$ has a right adjoint $\vpst : \Mhat \to \Chat = [\C\op,\CAT]$ by precomposition: $\vpst(X)(C) = \Mhat(\vp(C),X)$.
% In general, any right adjoint of $\vp_!$ comes with a mate natural transformation
% \begin{equation}
%   \begin{tikzcd}
%     \C \ar[r] \ar[d,"\vp"'] \ar[dr,phantom,"{\Downarrow^{\vptil}}"] & \Chat\\
%     \M \ar[r] & \Mhat. \ar[u,"\vpst"']
%   \end{tikzcd}\label{eq:vpmate}
% \end{equation}
% If $\C$ and $\M$ are 2-categories with families, then $\vptil$ is just the action of $\vp$ on homs, $\C(-,C) \to \M(\vp(-),\vp(C))$.

\begin{definition}\label{defn:mor-2cwu}
  Let $\C$ and $\M$ be two 2-categories with universes.
  A \textbf{morphism of 2-categories with universes} consists of:
  \begin{enumerate}
  \item A commutative square of 2-functors
    \begin{equation}\label{eq:mor-2cwu}
      \begin{tikzcd}
        \C \ar[r] \ar[d,"\vp"'] & \Chat \ar[d,"\vpsh"] \\
        \M \ar[r] & \Mhat
      \end{tikzcd}
    \end{equation}
  \item A $\vpsh$-morphism from $\Upshat:\Ctmhat \to \Ctyhat$ to $\Upshat : \Mtmhat \to \Mtyhat$ as in Definition \ref{defn:fibmor}.
  % \item Morphisms $\vpty$ and $\vptm$ forming a commutative square in $\Chat$:
  %   \begin{center}
  %     \begin{tikzcd}
  %       \Ctmhat \ar[r,"\vptm"] \ar[d,->>,"\Upshat"'] & \vpst \Mtmhat\ar[d,->>,"\vpst(\Upshat)"] \\
  %       \Ctyhat \ar[r,"\vpty"] & \vpst \Mtyhat
  %     \end{tikzcd}
  %   \end{center}
  %   Note that the right-hand vertical morphism is still a split fibration, since split fibration structure is defined using 2-categorical limits, and $\vpst$ is a right adjoint and hence preserves all such structure.
  % \item This commutative square is a strict morphism of split fibrations.
  \item This morphism furthermore ``preserves the algebraic representations'' in the following sense: given $C\in \C$ and $T:C \to \Ctyhat$, with specified pullback $\int T$, the composite square
    \begin{equation}
      \begin{tikzcd}
        \vp(\int T) \ar[r] \ar[d] & \vpsh \Ctmhat \ar[r] \ar[d] & \Mtmhat \ar[d]\\
        \vp(C) \ar[r,"\vpsh(T)"'] & \vpsh \Ctyhat \ar[r] & \Mtyhat
      \end{tikzcd}\label{eq:mor-presrep}
    \end{equation}
    is also a specified pullback.
% , the composite $\vpty T : C \to \vpst \Mtyhat$ has an adjunct $\overline{\vpty T}:\vp(C) = \vp_!(C) \to \Mtyhat$, and thus two algebraically specified pullback squares
%     \begin{center}
%       \begin{tikzcd}
%         \int T \ar[r] \ar[d] \ar[dr,phantom,near start,"\lrcorner"] & \Ctmhat \ar[d] &
%         \int \overline{ \vpty T} \ar[r] \ar[d] \ar[dr,phantom,near start,"\lrcorner"] & \Mtmhat \ar[d] \\
%         C \ar[r,"T"'] & \Ctyhat &
%         \vp(C) \ar[r,"{\overline{\vpty T}}"'] & \Mtyhat
%       \end{tikzcd}
%     \end{center}
%     Since $\vpst$ preserves pullbacks, the left-hand composite square below factors uniquely as on the right:
%     \begin{center}
%       \begin{tikzcd}
%         \int T \ar[r] \ar[d] &
%         \Ctmhat \ar[r,"\vptm"] \ar[d] & \vpst \Mtmhat\ar[d] \ar[dr,phantom,"="] &
%         \int T \ar[r,dashed]\ar[d] \ar[dr,phantom,"(*)"] &
%         \vpst \int \overline{\vpty T} \ar[r] \ar[d] \ar[dr,phantom,near start,"\lrcorner"] & \vpst\Mtmhat \ar[d] \\
%         C \ar[r] & \Ctyhat \ar[r,"\vpty"] & \vpst \Mtyhat &
%         C \ar[r,"\vptil"'] & \vpst \vp (C) \ar[r] & \vpst \Mtyhat
%       \end{tikzcd}
%     \end{center}
%     where $\vptil$ denotes a component of~\eqref{eq:vpmate}.
%     The condition we require is then that the square $(*)$ above coincides with the naturality square of $\vptil$:
%     \begin{center}
%       \begin{tikzcd}
%         \int T \ar[r,"\vptil"] \ar[d] & \vpst\vp(\int T) \ar[d] \\
%         C \ar[r,"\vptil"] & \vpst\vp(C)
%       \end{tikzcd}
%     \end{center}
%     (hence, in particular, that $\int \overline{\vpty T} = \vp(\int T)$).
  \end{enumerate}
\end{definition}

If $\M$ is a 2-category with a universe for which $\Mhat$ has large colimits, $\C$ is a 2-category with families, and $\vp:\C\to\M$ is any 2-functor, then the composite $\C \xrightarrow{\vp} \M \hookrightarrow \Mhat$ has a left Kan extension along $\C\hookrightarrow \Chat$.
Since the latter inclusion is fully faithful, $\vpsh$ is an honest extension, i.e.\ the universal square~\eqref{eq:mor-2cwu} commutes (up to isomorphism, at least, and $\vpsh$ can be chosen to make it commute strictly).
Moreover, in this case $\vpsh$ always has a right adjoint $\vpst : \Mhat \to \Chat = [\C\op,\CAT]$ defined by precomposition: $\vpst(X)(C) = \Mhat(\vp(C),X)$.

\begin{definition}
  If $\C$ and $\M$ are 2-categories with families, a \textbf{morphism of 2-categories with families} is a morphism of 2-categories with universes such that $\vpsh$ is this left Kan extension.
\end{definition}

\begin{remark}
  If $\M$ is a 2-category with a universe for which $\Mhat$ has large colimits, and $\M'$ denotes the 2-category with families constructed from it as in Theorem \ref{thm:2cwf-univ}, then the identity 2-functor $\M'\to\M$ is a morphism of 2-categories with universes with $\vpsh$ a left Kan extension as above.
  Morally, this should be some kind of ``coreflection'', but we will not try to make this precise.
\end{remark}

\begin{definition}\label{thm:2cwf-ldf}
  A morphism of 2-categories with universes $\vp : \C\to\M$ is a \textbf{local discrete fibration} if
  \begin{enumerate}
  \item The 2-functor $\vp:\C\to\M$ is a local discrete fibration, i.e.\ the induced functors on hom-categories $\C(X,Y) \to \M(\vp(X),\vp(Y))$ are discrete fibrations.\footnote{If they were non-discrete fibrations, there would be an additional compatibility condition on composition, but in the discrete case this is automatic.}\label{item:ldf1}
  \item The horizontal functors in the relevant instance of~\eqref{eq:fibmor2}:
    \begin{gather*}
      \Chat(X,\Ctmhat) \to \Mhat(\vp X, \vpsh \Ctmhat) \to \Mhat(\vp X, \Mtmhat)\\
      \Chat(X,\Ctyhat) \to \Mhat(\vp X, \vpsh \Ctyhat) \to \Mhat(\vp X, \Mtyhat)
    \end{gather*}
    are discrete fibrations.\label{item:ldf2}
  \end{enumerate}
  In this case we refer to $\C$ as a \textbf{fibered 2-category with a universe} (or \textbf{with families}, if $\C$ and $\M$ are such and $\vp$ is a morphism of such) over $\M$.
\end{definition}

If $\vpsh$ has a right adjoint $\vpst$ (such as if $\vp$ is a morphism of 2-categories with families), then condition \ref{item:ldf2} is equivalent to asking that the adjunct morphisms $\Ctmhat \to \vpst \Mtmhat$ and $\Ctyhat \to \vpst \Mtyhat$ are internal discrete fibrations in $\Chat$.

\begin{example}\label{eg:syn-fib-2cwf}
  Returning to Example \ref{eg:syn-2cwf}, we construct a fibered 2-category with families over the syntactic one $\M$, now using the ``upstairs'' type theory.
  \begin{itemize}
  \item An object of $\C$ is a context $\yields_\gamma \Gamma \CTX$.
    Of course, its image in $\M$ is $\yields \gamma \ctx$.
  \item A morphism in $\C$ is a total substitution $\Gamma_{\gamma} \yields_\theta \Theta : \Delta_\delta$, lying over $\gamma \yields \theta : \delta$.
  \item The fibrational action of 2-cells in $\M$ on morphisms in $\C$ is given by N-ary rewrite (Lemma \ref{lem:n-ary-ap-rewrite}).
  \item An object of $\Cty(\Gamma)$ is a type $\Gamma_\gamma \yields_p A \TYPE$, lying over $\gamma \yields p\type$ in $\Mty(\gamma) = \Mty(\vp(\Gamma))$.
    The fibrational action of morphisms in $\Mty$ is given by the $\St{s}{A}$ types.
  \item An object of $\Ctm(\Gamma)$ over $A\in \Cty(\Gamma)$ is a term $\Gamma_\gamma \yields_\mu M:A_p$, lying over $\gamma \yields \mu:p$ in $\Mtm$.
    To act on such a term by a morphism in $\Mtm(\gamma)$, we first factor the latter morphism as a mode 2-cell in a fiber $\TermTwoT{\gamma}{s}{\mu}{\TrPlus{t}{\nu}}{p}$ followed by the cartesian arrow corresponding to the action of a mode morphism $\TypeTwo{\gamma}{t}{p}{q}$ on $\nu$.
    The cartesian morphism then acts on $M$ by S-intro, $\StI{t}{M} : \St{t}{A}$, and then we rewrite with the fiber 2-cell $\rewrite{s}{\StI{t}{M}}$.
  \end{itemize}
\end{example}

To construct the canonical model, we appeal to a more general ``co-universe'' construction, which we decompose into three pieces.

\begin{theorem}\label{thm:corefl-ldfib}
  If $\vp:\C\to \M$ is a local discrete fibration of 2-categories with universes, then its coreflection into 2-categories with families is also a local discrete fibration.
\end{theorem}
\begin{proof}
  We first have to check that it is actually a morphism of 2-categories with families.
  The morphism $\vpsh \Ctyhat \to \Mtyhat$ induces by composition
  \begin{equation}
    \Chat(X,\Ctyhat) \to \Mhat(\vp X, \vpsh \Ctyhat) \to \Mhat(\vp X,\Mtyhat)\label{eq:corefl-map}
  \end{equation}
  and hence a map of the induced presheaves $\Cty \to \vpst \Mty$, and similarly for the $\Ctm \to \vpst \Mtm$.
  Moreover,~\eqref{eq:fibmor2} refers only to the functors represented by $p$ and $q$, so we can restrict it to $X\in\C$ to obtain the analogous condition for these underlying presheaves.
  And~\eqref{eq:mor-presrep} is defined in terms of~\eqref{eq:corefl-map}, so it carries over to the analogous condition as well.

  Finally, condition~\ref{item:ldf1} of Definition \ref{thm:2cwf-ldf} is evident since the coreflection doesn't change the underlying categories, while condition \ref{item:ldf2} follows since it is also defined in terms of~\eqref{eq:corefl-map}.
\end{proof}

\begin{theorem}\label{thm:ldf-lift}
  Suppose $\M$ is a 2-category with a universe, and we have a pullback square
  \begin{equation}\label{eq:ldf-mor}
    \begin{tikzcd}
      \C \ar[r] \ar[d,"\vp"'] & \Chat \ar[d,"\vpsh"] \\
      \M \ar[r] & \Mhat
    \end{tikzcd}
  \end{equation}
  such that $\vpsh$ is a local discrete fibration.
  Moreover, suppose we are given a lifting of $\Upshat : \Mtmhat \to \Mtyhat$ to a morphism $\Ctmhat \to \Ctyhat$ of $\Chat$, and that $\vpsh$ creates pullbacks of $\Ctmhat \to \Ctyhat$.
  Then $\C\hookrightarrow\Chat$ is a 2-category with a universe and~\eqref{eq:ldf-mor} is a local discrete fibration of 2-categories with universes.
\end{theorem}
\begin{proof}
  Since fully faithful 2-functors and local discrete fibrations are closed under pullback, $\C \to \Chat$ is fully faithful and $\vp$ is a local discrete fibration (of 2-categories only, for now).
  To show that $\Ctmhat \to \Ctyhat$ is a split fibration, suppose given the following data in $\Chat$, mapping to $\Mhat$ as shown on the right:
  \begin{equation*}
    \begin{tikzcd}[row sep=large]
      C \ar[d,equals] \ar[r,bend right,near end,"g'"'] & \Ctmhat \ar[d] \ar[dr,phantom,"\mapsto"] &
      \vpsh C \ar[d,equals] \ar[r,bend right,near end,"\vpsh g'"'] & \Mtmhat \ar[d] \\
      C \ar[r,bend left,"f"] \ar[r,bend right,"g"'] \ar[r,phantom,"\Downarrow^\alpha"] & \Ctyhat &
      \vpsh C \ar[r,bend left,"\vpsh f"] \ar[r,bend right,"\vpsh g"'] \ar[r,phantom,"\Downarrow^{\vpsh \alpha}"] & \Mtyhat
    \end{tikzcd}
  \end{equation*}
  We then have a specified cartesian lift $\gamma : h \to \vpsh g'$ in $\Mhat$.
  Since $\vpsh$ is a local discrete fibration, $\gamma$ has a unique lift $\beta : f' \to g'$ in $\Chat$.
  To show $\beta$ is cartesian, suppose given $\theta : k\to g'$ in $\Chat$ with $\Upshat\circ \theta = \alpha *\mu$ for some $\mu$\footnote{We are writing $\circ$ for composition of 1- and 2-morphsims along objects, including whiskering, and $*$ for composition of 2-morphisms along 1-morphisms}.
  Then $\Upshat \circ \vpsh\theta = \vpsh\alpha *\vpsh\mu$, so $\vpsh\theta$ factors uniquely as $\gamma* \nu$ since $\gamma$ is cartesian.
  Now since $\vpsh$ is a local discrete fibration, $\nu$ has a unique lift $\eta : k' \to f'$; but then $\beta*\eta : k' \to g'$ is a lift of $\vpsh\theta$, hence equal by uniqueness to $\theta$.
  Uniqueness of these factorizations follows similarly, and splitness and functoriality is immediate.
  Furthermore, by construction the identities $\vpsh\Ctyhat = \Mtyhat$ and $\vpsh\Ctmhat =\Mtmhat$ are a strict $\vpsh$-morphism of split fibrations, and since $\vpsh$ (not just $\vp$) is a local discrete fibration, condition~\ref{item:ldf2} of Definition \ref{thm:2cwf-ldf} is automatic.

  It remains to show that $\Ctmhat \to \Ctyhat$ is algebraically $\C$-representable and that $\vp$ preserves these representations.
  Since the right-hand square in~\eqref{eq:mor-presrep} is an identity, it suffices to show that given $T:C\to \Ctyhat$ with $C\in \C$, the specified pullback $\int (\vpsh T)$ lifts to a pullback in $\Chat$ with vertex in $\C$.
  But by assumption $\vpsh$ creates pullbacks of $\Ctmhat\to\Ctyhat$, and the vertex of such a pullback must lie in $\C$ since~\eqref{eq:ldf-mor} is a pullback of 2-categories.
\end{proof}

\begin{theorem}\label{thm:laxslice-2cwf}
  Let $\M$ be a 2-category with a universe, and let $\one$ be an arbitrary object of $\M$ equipped with a commutative triangle
  \begin{equation}
    \begin{tikzcd}
      & \Mtmhat \ar[d,"\Upshat"] \\
      \one \ar[ur,"t"] \ar[r,"\top"'] & \Mtyhat.
    \end{tikzcd}\label{eq:laxslice-tri}
  \end{equation}
  Let $\C = \one\sslash \M$ be the lax slice 2-category of $\M$ under $\one$: its objects are morphisms $c:\one\to C$ in $\M$, its morphisms are 2-cells inhabiting triangles
  \begin{equation}\label{eq:laxslice-mor}
    \begin{tikzcd}
      & \one \ar[dl,"a"'] \ar[dr,"b"] \ar[d,phantom,"\overset{\phi}{\Rightarrow}"] \\
      A \ar[rr,"f"'] & {}& B
    \end{tikzcd}
  \end{equation}
  and its 2-cells are 2-cells $\alpha :f\to g$ in $\M$ such that $\alpha a \circ \phi = \psi$.
  Similarly, let $\Chat = \one\sslash\Mhat$, with the evident inclusion $\C\hookrightarrow\Chat$.
  Let $\Ctyhat = (\top:\one \to \Mtyhat)$ and $\Ctmhat = (t:\one\to\Mtmhat)$, with~\eqref{eq:laxslice-tri} defining a map $\Ctmhat \to \Ctyhat$ in $\Chat$.
  Then $\C$ is a 2-category with a universe, and the forgetful 2-functor $\vp:\C\to\M$ is a local discrete fibration.
\end{theorem}
\begin{proof}
  First note that the forgetful functor $\vp:\C=(\one\sslash\M)\to\M$ is a local discrete fibration of 2-categories: given a morphism $(f,\phi):(A,a) \to (B,b)$ as in~\eqref{eq:laxslice-mor} and a 2-cell $\alpha:\psi\to\phi$ in $\M$, the unique lift is $\alpha$ with domain $(f,\phi * (\alpha \circ a))$.
  Similarly, the forgetful functor $\vpsh:\Chat=(\one\sslash\Mhat)\to\Mhat$ is a local discrete fibration, and $\C$ is the pullback of $\Chat$ and $\M$ over $\Mhat$ (the objects of $\C$ are just those of $\Chat$ whose underlying objects in $\Mhat$ lie in $\M$).
  Thus, to apply Theorem \ref{thm:ldf-lift} it remains only to check that $\vpsh$ creates pullbacks of~\eqref{eq:laxslice-tri} regarded as a morphism in $\one\sslash\Mhat$.

  For brevity, write $q:X\to Y$ for $\Upshat :\Mtmhat \to \Mtyhat$; the only thing we will use about $q$ is that it is a fibration.
  Note that we have assumed~\eqref{eq:laxslice-tri} to be commutative rather than simply inhabited by a 2-cell, so regarded as a lifting of $q$ it is special in this way as well among morphisms of $\one\sslash\Mhat$.
  Now suppose we have a morphism $(f,\phi) : (A,a) \to (Y,\top)$ in $\one\sslash\Mhat$, and a pullback in $\Mhat$:
  \begin{center}
    \begin{tikzcd}
      P \ar[r,"j"] \ar[d,"k"'] & X \ar[d,"q"]\\
      A \ar[r,"f"'] & Y.
    \end{tikzcd}
  \end{center}
  Let $\overline{\phi} : i \to t$ be a cartesian 2-cell lifting $\phi : f \circ a \to \top$.
  Then $q \circ i = f\circ a$, so the universal property of $P$ induces a map $p:\one\to P$ with $j\circ p = i$ and $k\circ p = a$.
  The latter identity makes $k$ into a morphism $(P,p) \to (A,a)$ in $\one\sslash\Mhat$, while $j\circ p = i \xrightarrow{\overline{\phi}} t$ makes $j$ into a morphism $(P,p) \to (X,t)$.
  We claim the resulting square
  \begin{center}
    \begin{tikzcd}
      (P,p) \ar[r,"{(j,\overline{\phi})}"] \ar[d,"{(k,1)}"'] & (X,t) \ar[d,"{(q,1)}"]\\
      (A,a) \ar[r,"{(f,\phi)}"'] & (Y,\top)
    \end{tikzcd}
  \end{center}
  (which commutes because $q \circ \overline{\phi} = \phi$) is a pullback in $\one\sslash\Mhat$.
  To show this, suppose given $(Z,z:\one\to Z)$ and morphisms $(g,\gamma):(Z,z) \to (X,t)$ and $(h,\theta):(Z,z) \to (A,a)$ forming a commutative square, which means $q\circ g = f\circ h$ and $q\circ \gamma = \phi * (f\circ \theta)$.
  The former equality induces, by the universal property of $P$, a unique map $\ell : Z\to P$ such that $j\circ \ell = g$ and $k\circ \ell = h$.

  To make $\ell$ into a morphism in $\one\sslash\Mhat$, we need a 2-cell $\ell \circ z \to p$.
  By the 2-cell universal property of $P$, it suffices to give 2-cells $g\circ z \to j\circ p = i$ and $h\circ z \to k \circ p = a$ that agree in $Y$.
  For the latter, we can simply take $\theta$.
  For the first, note that $\overline{\phi}:i\to t$ is cartesian over $\phi$, so there is a unique 2-cell $\eta : g \circ z \to i$ such that  $\overline{\phi} * \eta = \gamma$ and $q \circ \eta = f\circ \theta$.

  The latter equation is the requisite agreement in $Y$, so we have a unique 2-cell $\lambda:\ell \circ z \to p$ such that $k\circ \lambda = \theta$ and $j\circ \lambda = \eta$.
  The latter equation is equivalent, by uniqueness of $\eta$, to $q\circ j \circ \lambda = f\circ \theta$ (which follows from $k\circ \lambda = \theta$ and $q\circ j = f\circ k$) and $\overline{\phi} * (j\circ \lambda) = \gamma$.
  Thus, these conditions are precisely the statements that $(j,\overline{\phi})\circ (\ell,\lambda) = (g,\gamma)$  and $(k,1)\circ (\ell,\lambda) = (h,\theta)$.
  This shows the 1-cell universal property of $(P,p)$ in $\one\sslash\Mhat$.
  The 2-cell universal property follows immediately from the 2-cell universal property of $P$.
\end{proof}

Combining Theorems \ref{thm:corefl-ldfib} and \ref{thm:laxslice-2cwf}, we can construct local discrete fibrations of 2-categories with families.

\begin{example}
  Consider the 2-category with a universe from Example \ref{eg:cat-2cwf}, with $\M=\Cat$, $\Mhat = \CAT$, and $\Mtyhat = \Cat\op$.
  Let $\one$ be the terminal category in $\M=\Cat$, with $\top:\one \to \Cat\op$ picking out the terminal category and $t$ the unique object of the latter.
  Thus we obtain a 2-category with a universe $\C$ where:
  \begin{itemize}
  \item The objects of $\C$ (resp.\ $\Chat$) are large (resp.\ very large) categories $C$ equipped with a chosen object $c\in C$.
  \item The morphisms $(C,c) \to (D,d)$ are functors $f:C\to D$ together with a morphism $\phi :f(c) \to d$.
  \item Natural transformations act contravariantly on such pairs by precomposition $f(c) \xrightarrow{\alpha_c} g(c) \to d$.
  \item $\Ctyhat$ is the category $\Mtyhat = \Cat\op$ equipped with the terminal category $\top$.
  \item Similarly, $\Ctmhat$ is the category $\Mtmhat$ of pointed categories $(A,a)$ with morphisms $(A,a) \to (B,b)$ being functors $f:B\to A$ paired with a morphism $\phi : a\to f(b)$, equipped with the terminal pointed category $(\top,t)$.
  \end{itemize}
  If we now apply the coreflection into 2-categories with families, we get a local discrete fibration over the ``canonical'' 2-category with families from Example \ref{eg:cat-2cwf}, where:
  \begin{itemize}
  \item The objects of $\Cty(C,c)$ are morphisms $(C,c) \to \Ctyhat = (\Cat\op,\top)$ in $Chat = \one\sslash\CAT$, which is to say functors $D:C\op\to \Cat$ equipped with a morphism $D(c) \to \top$ in $\Cat\op$, i.e.\ a functor $\top \to D(c)$, which is to say an object $d\in D(c)$.
  \item The objects of $\Ctm(C,c)$ are such pairs $(D,d)$ with a section $s:C\to \int D$ of the Grothendieck construction, along with an object $d\in D(c)$ and a morphism $s(c) \to d$ in $D(c)$.
  \end{itemize}
  Thus, this reproduces our informal expectation from \S\ref{sec:2cwfs}.
  Note that while $\Cty$ and $\Ctm$ in this example simply ``pick out the objects and morphisms'' of the categories $C\in\M$, in general we can regard the fibered 2-category with families $\C$ as an ``abstraction'' of the notions of ``objects and morphisms'' applicable to any 2-category with families $\M$.
\end{example}



\end{document}

\documentclass[10pt]{article}

\usepackage{fullpage}
\usepackage{amssymb,amsthm,bbm,stmaryrd}
\usepackage[centertags]{amsmath}
\usepackage{mathrsfs}
\usepackage{mathtools}
\usepackage{tikz-cd}
\usepackage{mathpartir}
\usepackage[status=draft,inline,nomargin]{fixme}
\FXRegisterAuthor{ms}{anms}{\color{blue}MS}
\FXRegisterAuthor{todo}{andrl}{\color{red} }
\FXRegisterAuthor{mvr}{anmvr}{\color{olive}MVR}

\newtheorem{theorem}{Theorem}
\newtheorem{proposition}{Proposition}
\newtheorem{lemma}{Lemma}

\let\oldemptyset\emptyset
\let\emptyset\varnothing

\newcommand{\yields}{\vdash}
\newcommand{\cbar}{\mid}
\newcommand{\sub}[2]{{#1}\llbracket {#2}\rrbracket}
\newcommand{\xt}{\triangleright}
\let\ec\diamond

%\newcommand{\Id}[3]{\mathsf{Id}_{{#1}}(#2,#3)}
\newcommand{\CTX}{\,\,\mathsf{CTX}}
\newcommand{\ctx}{\,\,\mathsf{ctx}}
\newcommand{\TYPE}{\,\,\mathsf{TYPE}}
\newcommand{\type}{\,\,\mathsf{type}}
\newcommand{\mode}{\,\,\mathsf{mode}}
\newcommand{\asp}{\,\,\mathsf{asp}}
\newcommand{\TELE}{\,\,\mathsf{TELE}}
\newcommand{\tele}{\,\,\mathsf{tele}}

\newcommand\F[2]{\ensuremath{\mathsf{F}_{#1}(#2)}}
\newcommand\U[3]{\ensuremath{\mathsf{U}_{#1}(#2 \mid #3)}}
\newcommand\St[2]{\ensuremath{{#1}^*(#2)}}
\newcommand\StI[2]{\ensuremath{\mathsf{st}(#1,#2)}}
\newcommand\StE[2]{\ensuremath{\mathsf{unst}(#1,#2)}}
\newcommand\FE[3]{\ensuremath{\mathsf{split} \, #2 \, = \, {#1} \, \mathsf{in} \, #3}}
\newcommand\FI[1]{\ensuremath{\mathsf{F}{(#1)}}}
\newcommand\UE[2]{\ensuremath{#1(#2)}}
\newcommand\UI[2]{\ensuremath{\lambda #1.#2}}
\newcommand\Trd[2]{\ensuremath{#1_*(#2)}}

\newcommand\Set[0]{\ensuremath{\textbf{Set}}}
\newcommand\Hom[3]{\ensuremath{\textbf{hom}_{#1}(#2,#3)}}
\newcommand\just[1]{\ensuremath{\textsf{just}_{#1}}}
\newcommand\Dt[2]{\ensuremath{#1.#2}}

\newcommand\Push[3]{\ensuremath{#1 +_{#2} #3}}
\newcommand\Pushout[5]{\ensuremath{#1 +^{#4,#5}_{#2} #3}}
\newcommand{\case}{\mathsf{case}}
\newcommand{\inl}{\mathsf{inl}}
\newcommand{\inr}{\mathsf{inr}}
\newcommand{\ini}{\mathsf{in}}

\newcommand{\id}{\mathsf{id}}
\DeclareMathOperator{\ob}{ob}

\newcommand{\A}{\mathscr{A}}
\newcommand{\B}{\mathscr{B}}
\newcommand{\C}{\mathscr{C}}
\newcommand{\T}{\mathscr{T}}
\newcommand{\atA}{A}
\newcommand{\atB}{B}
\newcommand{\atC}{C}
\newcommand{\atG}{G}
\newcommand{\atX}{X}
\newcommand{\atY}{Y}
\newcommand{\atZ}{Z}
\newcommand{\tma}{a}
\newcommand{\unit}{\mathbf{1}}
\newcommand{\Id}{\mathsf{Id}}

\title{Complicial syntax for dependent type 2-theory}
\author{}
\date{}

\begin{document}
\maketitle

\section{Overview}
\label{sec:overview}

As I described in another file, we can incorporate the ``type theory over the mode theory'' into the mode theory itself as the empty-context fragment, leading to a 2-simplex-based syntax mode theory.
Here is an attempt at doing this for dependent type theory.

Even if we don't want to discard the separate ``type theory over the mode theory'', we could still use this presentation for the mode theory itself.
It might have advantages over the ``globular'' presentation, perhaps in terms of its own computational / cut-elimination behavior.
The separate ``type theory over the mode theory'' has the advantage of being able to represent type theories in which not all \textsf{F}-types exist (which incorporating it into the mode theory cannot), and might also be easier to make substructural.

The attempt below also returns to the ``universe'' approach to recovering ordinary type theory, where instead of rules for building up types as profunctors we have a ``comprehension category object'' inside our comprehension bicategory.
I don't remember any more exactly why I was against this approach before, but now I am for it again.
And I think I understand it better now, mainly due to the following additional change:

I'm not thinking about finite inverse categories and profunctors any more, at least not very much.
I brought those in because I was thinking about the modes as directly encoding ``dependency structures'', but now I think that that was too syntactic.
Instead, now I want to think about the semantics going all the way to $\mathbf{Cat}$.
That is, objects of the comprehension bicategory (``mode contexts'') correspond to categories (which, in the ordinary case, will be categories of \emph{diagrams} on finite inverse categories), and dependent objects (previously ``modes'', which I'm now experimenting with calling ``aspects'') represent fibrations over them.
This resolves the weird backwardness as everything now goes in the ``real'' direction, and it also makes it clear to me how to represent term dependencies using $\Sigma$s (which we had trouble with before).


\section{Syntax}
\label{sec:syntax}

\begin{itemize}
\item $\aleph \ctx$ (empty, extension).
\item $\aleph \yields \A \asp$ (generators).  We call $\A$ an ``aspect''.
\item $\aleph \yields \beth \tele$ (empty, extension). %  We call $\beth$ an ``aspect telescope''.
\item $\aleph \yields \atA : \A$ (variables, generators, substitutions of terms -- below).  We call $a:\A$ an ``attribute''.
\item $\aleph \yields \Gamma : \beth$ (tuples). % We call $\Gamma$ an ``attribute context''.
\item $\aleph \cbar \Gamma_\beth \yields_B \tma : \atA_\A$ (generators, $\mathsf{F}/\mathsf{U}$ intro and elim -- below).  We call $\tma$ a ``term''.
\item $\aleph \cbar \Gamma_\beth \yields_\Delta \omega : \Theta_\daleth$ (tuples).
\end{itemize}
In the term judgment $\aleph \cbar \Gamma_\beth \yields_B \tma : \atA_\A$ the preconditions are:
\begin{mathpar}
  \aleph\ctx \and
  \aleph \yields \beth \tele\and
  \aleph \yields \Gamma:\beth \and
  \aleph,\beth \yields \A\asp\and
  \aleph,\beth \yields \atB:\A\and
  \aleph \yields \atA : \A[\Gamma].
\end{mathpar}
In particular, note that $\Gamma$ represents a tuple of \emph{attributes}.
To express the dependence of $\tma$ on $\Gamma$ (and indeed of later attributes in $\Gamma$ on earlier ones), each of these attributes should also be assigned a variable.
%It would be nice if we could pun these with the variables in $\beth$, but I'm doubtful.
This judgment represents a 2-simplex
\[
\begin{tikzcd}
  & \aleph.\beth \ar[dr,"\atB"] \ar[d,phantom,"\Downarrow\scriptstyle \tma"] \\
  \aleph \ar[ur,"\Gamma"] \ar[rr,"\Gamma.\atA"'] & {} & \aleph.\beth.\A
\end{tikzcd}
\]
subject to commutativity with the projections of the context extensions.
It includes as particular cases both ``globular mode 2-cells $\aleph \yields \tma : \atB \Rightarrow \atA$'', represented as:
\[
\aleph \cbar \cdot \yields_\atB \tma : \atA_\A \qquad
\begin{tikzcd}
  & \aleph \ar[dr,"\atB"] \ar[d,phantom,"\Downarrow\scriptstyle \tma"] \\
  \aleph \ar[ur,equals] \ar[rr,"\Gamma.\atA"'] & {} & \aleph.\A
\end{tikzcd}
\]
and also ``terms in the type theory over the mode theory'', represented as:
\[
\cdot \cbar \Gamma_\beth \yields_\atB \tma : \atA_\A \qquad
\begin{tikzcd}
  & \beth \ar[dr,"\atB"] \ar[d,phantom,"\Downarrow\scriptstyle \tma"] \\
  1 \ar[ur,"\Gamma"] \ar[rr,"\Gamma.\atA"'] & {} & \beth.\A
\end{tikzcd}
\]
This latter is the reason for the funny notation: I'm trying to keep the common letters $\Gamma$ for contexts of types (= tuples of attributes in the empty aspect context), $\atA,\atB$ for types, and $\tma$ for terms in such types.

Similarly, the tuple judgment $\aleph \cbar \Gamma_\beth \yields_\Delta \omega : \Theta_\daleth$ requires
\begin{mathpar}
  \aleph\ctx \and
  \aleph \yields \beth \tele\and
  \aleph \yields \Gamma:\beth \and
  \aleph,\beth \yields \daleth\tele\and
  \aleph,\beth \yields \Delta:\daleth\and
  \aleph \yields \Theta : \daleth[\Gamma].
\end{mathpar}
The tupling rule for terms into tuples requires substitution of terms into attributes, which is explained below.

Substitution of attributes into aspects and attributes is as usual, and should be admissible.
Substitution of terms into terms, which should also be admissible, is something like this:
\begin{mathpar}
  \inferrule{\aleph \cbar \Gamma_\beth \yields_\Delta \omega:\Theta_\daleth \\ 
    \aleph \cbar \Gamma_\beth, \Delta_{\daleth} \yields_\atB \tma: \atA_\A}
  { \aleph \cbar \Gamma_\beth \yields_{B[\Delta]} \tma[\omega] : \atA[\Theta]_{\A[\Theta]} }
\end{mathpar}
This represents a 2-simplex composite
\[
\begin{tikzcd}[row sep=large]
  \aleph.\beth \ar[rr,"\Delta"] & {} &
  \aleph.\beth.\daleth \ar[d,"\atB"] \ar[dl,phantom,"\Downarrow\scriptstyle \tma"] \\
  \aleph \ar[u,"\Gamma"] \ar[ur,phantom,"\Downarrow\scriptstyle\omega"]
  \ar[urr,"\Gamma.\Theta" description] \ar[rr,"\Gamma.\Theta.\atA"'] & {} & \aleph.\beth.\daleth.\A
\end{tikzcd}
\qquad\leadsto\qquad
\begin{tikzcd}
  & \aleph.\beth \ar[dr,"{\atB[\Delta]}"] \ar[d,phantom,"{\Downarrow\scriptstyle \tma[\omega]}"] \\
  \aleph \ar[ur,"\Gamma"] \ar[rr,"{\Gamma.\atA[\Theta]}"'] & {} & \aleph.\beth.\A[\Theta]
\end{tikzcd}
\]
(The notation is hard to get right, so it's probably wrong somewhere.)

Since the judgment $\aleph \yields \atA:\A$ includes the types in the object language, $\mathsf{F}$- and $\mathsf{U}$-types are given by operations on such judgments in our 2-theory.
The \textsf{F}-types are the simplest: they are just \emph{substitution}.
That is, given
\begin{mathpar}
  \cdot \yields \atA:\A \and \atX:\A \yields \atC:\C
\end{mathpar}
with $\atA$ regarded as a type of mode $\A$, and $\atX.\atC$ regarded as a mode morphism, the \textsf{F}-type $\F{\atX.\atC}{\atA}$ is just $\cdot \yields \atC[\atA/\atX]:\C$.
The intro and elim rules for terms in \textsf{F}-types then generalize to rules for terms that look something like these:
\begin{mathpar}
  \inferrule{
    \aleph \yields \Gamma:\beth \\
    \aleph,\beth \yields \C\asp\\
    \aleph,\beth \yields \atC:\C}
  {\aleph \cbar \Gamma_\beth \yields_\atC \mathsf{FI} : \atC[\Gamma]_\C}
  \and
  \inferrule{
    \aleph \yields \Gamma:\beth\\
    \aleph,\beth,\C \yields \atB:\A\\
    \aleph,\beth \yields \atC : \C\\
    \aleph \yields \atA : \A[\atC][\Gamma]\\
    \aleph \cbar \Gamma_\beth \yields_{\atB[\atC]} a : \atA_{\A[\atC]}
  }{\aleph \cbar \Gamma_\beth,\atC[\Gamma]_{\C[\Gamma]} \yields_{\atB} \mathsf{FE}(a) : \atA_\A}
\end{mathpar}
The eventual \textsf{F}-intro rule should also build in a substitution and a 2-cell action (below) to make them admissible, but the above is simpler to see first.
It represents a ``universal'' composition 2-simplex
\[
\begin{tikzcd}
  & \aleph.\beth \ar[dr,"\atC"] \ar[d,phantom,"\Downarrow\scriptstyle\mathsf{id}"] \\
  \aleph \ar[ur,"\Gamma"] \ar[rr,"{\Gamma.\atC}"'] & {} & \aleph.\beth.\C
\end{tikzcd}
\]
The \textsf{F}-elim rule represents the other kind of 2-simplex composite with one of these universal 2-simplices:
\[
\begin{tikzcd}[row sep=large]
  \aleph.\beth \ar[drr,"{\atC.\atB}" description] \ar[rr,"\atC"] \ar[dr,phantom,"\Downarrow\scriptstyle\mathsf{id}.a"] &
  {} \ar[dr,phantom,"\Downarrow\scriptstyle\mathsf{id}"] & \aleph.\beth.\C \ar[d,"\atB"] \\
  \aleph \ar[u,"\Gamma"] \ar[rr,"{\Gamma.\atC.\atA}"'] & {} & \aleph.\beth.\C.\A
\end{tikzcd}
\qquad\leadsto\qquad
\begin{tikzcd}
  & \aleph.\beth.\C \ar[dr,"\atB"] \ar[d,phantom,"\Downarrow\scriptstyle\mathsf{FE}(a)"] \\
  \aleph \ar[ur,"{\Gamma.\atC}"] \ar[rr,"\Gamma.\atC.\atA"'] & {} & \aleph.\beth.\C.\A
\end{tikzcd}
\]
Of course, there are $\beta$ and $\eta$ rules connecting these two.
(Actually, I suspect we should probably assert all of these rules only when $\atB$ is a generator, and obtain the one for general $\atB$ admissibly.)

To get a 2-categorical structure, we need to ensure that \emph{all} the 2-simplex composites of this form exist, not just those with universal 2-simplices.
Since any 2-simplex can be factored as a universal one composed with a ``globular'' one, it remains to deal with the globular case:
\[
\begin{tikzcd}[row sep=large]
  \aleph.\beth \ar[drr,"{b}" description] \ar[rr,equals] \ar[dr,phantom,"\Downarrow\scriptstyle\alpha"] &
  {} \ar[dr,phantom,"\Downarrow\scriptstyle\beta"] & \aleph.\beth \ar[d,"c"] \\
  \aleph \ar[u,"\Gamma"] \ar[rr,"{\Gamma.a}"'] & {} & \aleph.\beth.\A
\end{tikzcd}
 \]
This yields the following ``structural 2-cell action'' rule, familiar from LS and LSR modal type theory, which should also be admissible (as long as we build it into things like \textsf{F}-intro):
\[ \inferrule{
  \aleph \cbar \Gamma_\beth \yields_b \alpha: a_\A \\
  \aleph,\beth \cbar \cdot \yields_c \beta: b_\A
}{\aleph \cbar \Gamma_\beth \yields_c \beta^*\alpha : a_\A}
\]
Of course, this action should satisfy functoriality laws and so on.

It remains to explain how to substitute terms into attributes.
As a special case, this includes substitution of terms into types, so it should be admissible as long as we build it into things sufficiently.
But it also includes as a different special case a ``structural 2-cell action'' on \emph{types}, which in earlier versions was called ``coercion''.
The general rule is the following:
\begin{mathpar}
  \inferrule{
    \aleph \cbar \Gamma_\beth \yields_\Delta \omega : \Theta_\daleth\\
    \aleph,\beth,\daleth \yields \A \asp \\
    \aleph \yields \atA:\A[\Gamma][\Theta]
  }{ \aleph \yields \sub \atA\omega : \A[\Delta][\Gamma] }
  \and
  % a lifted 2-cell (elim)
  % a factorization through it (intro)
  % beta/eta
\end{mathpar}
If $\aleph$ is empty, this yields substitution of terms into types.
But if $\Gamma$ is empty, it instead yields the ``coercion'' $\A[\Theta]\to \A[\Delta]$ induced by the globular 2-cell $\omega : \Delta\Rightarrow\Theta$.
Thus this coercion is an ``$\mathsf{F}$-type'' associated to this morphism $\A[\Theta]\to \A[\Delta]$, but since it's not a generator it's not ``really'' an $\mathsf{F}$-type (i.e.\ its intro and elim rules are probably not primitive).

I'm not sure whether $\sub\atA\omega$ really deserves a different notation than any other kind of substitution, but for now I'm keeping it distinct to avoid confusing myself.
To make it admissible (if that is possible), we will probably need to be careful about having ``base'' attributes express their dependency on a term variable, i.e.\ building this kind of substitution into the right places.

With substitution of terms into attributes, we can now explain how terms get tupled up to land in contexts:
\[
\inferrule{\aleph \cbar \Gamma_\beth \yields_\Delta \omega : \Theta_\daleth \\
  \aleph,\beth,\daleth \yields \A\asp\\
  \aleph,\beth \yields \atB : \A[\Delta] \\
  \aleph \yields \atA : \A[\Gamma][\Theta]\\
  \aleph \cbar \Gamma_\beth \yields_{\atB} \tma : {\sub \atA \omega}_{\A[\Delta]}
}{\aleph \cbar \Gamma_\beth \yields_{(\Delta,\atB)} (\omega,\tma) : (\Theta,\atA)_{(\daleth,\A)}}
\]

Categorically, substituting terms into attributes is what ensures that the projections $\aleph.\A\to \aleph$ are internal fibrations.
Johnstone's characterization of such fibrations using a ``semi-lax right adjoint'' involves a universal 2-cell lift that should look something like this:
\[
\begin{tikzcd}
  & \aleph.\beth.\A[\Delta][\Gamma] \ar[dr,"\Delta.\mathsf{id}"] \ar[d,phantom,"\Downarrow\scriptstyle\omega.\mathsf{SE}"] \\
  \aleph.\A[\Gamma][\Theta] \ar[ur,"{\Gamma.(\lambda \atX.\sub \atX\omega)}"] \ar[rr,"\Theta.\mathsf{id}" description] \ar[dd,dashed] &
  {} \ar[d,dashed] & 
  \aleph.\beth.\daleth.\A \ar[dd,dashed]\\
  & \aleph.\beth \ar[dr,"\Delta"] \ar[d,phantom,"\Downarrow\scriptstyle\omega"] \\
  \aleph \ar[ur,"\Gamma"] \ar[rr,"\Theta"'] & {} & \aleph.\beth.\daleth
\end{tikzcd}
\]
If we write this out, using the above rule to deduce what the type of \textsf{SE} should be in order to permit this tupling, we get
\[
  \inferrule{
    \aleph \cbar \Gamma_\beth \yields_\Delta \omega : \Theta_\daleth\\
    \aleph,\beth,\daleth \yields \A \asp
  }{\aleph, \atX:\A[\Gamma][\Theta] \cbar \Gamma_\beth, \sub \atX\omega_{\atY:\A[\Delta][\Gamma]} \yields_{\atY} \mathsf{SE} : {\sub \atX \omega}_{\A[\Delta]}}
\]
This looks it should be just a ``variable use'', with the $\sub X\omega$ in the context matching that in the conclusion.
Which makes sense: in the universal pullback square for a substitution in ordinary type theory
\[
\begin{tikzcd}
  \Gamma'.A[\gamma] \ar[r] \ar[d] & \Gamma.A \ar[d] \\
  \Gamma' \ar[r,"\gamma"'] & \Gamma
\end{tikzcd}
\]
the top arrow is a tupling of $\Gamma \yields \gamma :\Gamma$ with the variable use $\Gamma',x:A[\gamma] \yields x:A[\Gamma]$.
It seems like if we take this seriously, then the syntax ought automatically to ensure the universal property of the above universal 2-cell lift, just as it ensures that the ordinary square above is a pullback; but I have not checked that.



\section{Recovering ordinary type theory}
\label{sec:ordinary}

Recalling that the intended semantics is now directly in $\mathbf{Cat}$, we don't need to turn anything around in defining an internal comprehension-category object.
However, I do want to shift our point of view on a comprehension category a bit.

Normally, we regard the \emph{terms} $\Gamma\yields a:A$ in the type theory presented by a comprehension category $\T \to \C^{\mathbf{2}}$ as represented by the sections of the comprehension projection $\pi_A : \Gamma.A \to \Gamma$, which is in particular a morphism in the base ``category of contexts'' $\C$.
However, in natural semantic models, such sections are in natural bijection with morphisms $1 \to A$ in the fiber category $\T(\Gamma)$.
I want to shift perspective and regard \emph{these} as the terms.

This does require assuming that the fibers have terminal objects (unit types) $\unit_\Gamma \in\T(\Gamma)$.
In previous language, these unit types are the ``generic section'', which we use to parametrize the dependency of the ``generic'' term.

To express more complex dependencies, we need our comprehension-category object to have $\Sigma$-types and $\Id$-types.
This makes sense semantically because just as a ``mode'' in LSR ``has finite products'' as an object in the mode 2-category, here each ``mode'' should ``have finite limits'' in an appropriate sense.
However, our changed perspective on terms means that the rules for $\Sigma$- and $\Id$-types should be different.
In fact, I think they become the more natural category-theoretic rules, namely that $\Sigma_A : \T(\Gamma.A) \to \T(\Gamma)$ is left adjoint to weakening $\pi_A^* : \T(\Gamma) \to \T(\Gamma.A)$, and likewise $\Id_A \in \T(\Gamma.A.A)$ is the image of $\unit_{\Gamma.A}$ under a left adjoint to contraction $\Delta_A^* : \T(\Gamma.A.A) \to \T(\Gamma.A)$.
We probably need to assert Beck-Chevalley conditions as well, and we may need to assert that these adjoints are preserved by comprehension, i.e.\ that the projection $\Gamma.\Sigma_A(B) \to \Gamma$ is (isomorphic to) the composite $\Gamma.A.B \to \Gamma.A \to \Gamma$.

I'm not going to try to write down the complete definition of a comprehension-category object with this structure right now.
However, the basic judgments are:
\begin{mathpar}
  \cdot\yields \C\asp\and
  \cdot \yields \ec:\C\and
  \atG:\C \yields \T(\atG) \asp\and
  \atG:\C, \atX:\T(\atG) \yields \atG.\atX : \C\and
  \atG:\C, \atX:\T(\atG) \yields \pi_\atX : \atG.\atX \Rightarrow \atG : \C\and
  \atG:\C \yields \unit_{\atG} : \T(\atG)\and
  \atG:\C, \atX:\T(\atG), \atY:\T(\atG.\atX) \yields \Sigma_\atX(\atY) : \T(\atG)\and
  \atG:\C, \atX:\T(\atG) \yields \Id_\atX : \T(\atG.\atX.\atX)
\end{mathpar}
Of course, as ``generators'' these should build in substitutions (i.e.\ substitutions into them are stuck).
But for substitution of \emph{terms} into attributes, we should probably make them compute:
\[
\inferrule{
  \aleph \cbar \Gamma_\beth \yields_\Delta g : \Theta_\C
}{ \aleph \yields \sub{\unit_\Theta}{g} \equiv \unit_\Delta : \T(\Delta) }
\]

Now the ``standard'' sort of dependency in ordinary DTT:
\[ x:A, y:B(x), z:C(x,y) \yields t(x,y,z) : D(x,y,z) \]
is represented by
\[ \cdot \cbar x:A_{\atX:\T(\ec)}, y:B(x)_{\atY:\T(\ec.\atX)}, z:C(x,y)_{\atZ:\T(\ec.\atX.\atY)} \yields_{\unit_{\ec.\atX.\atY.\atZ}} t(x,y,z) : D(x,y,z)_{\T(\ec.\atX.\atY.\atZ)}
\]
(I think the notation $B(x)$ means that a substitution of terms-into-attributes may be applied to $B$, being built in at base types.)
For a term that depends on some type more times than its own type does (corresponding to a non-surjective map of profunctors):
\[ x:A,\, y,y':B(x) \yields t(x,y,y') : C(x,y) \]
we can use $\Sigma$-types:
\[ \cdot \cbar x:A_{\atX:\T(\ec)}, y:B(x)_{\atY:\T(\ec.\atX)} \yields_{\Sigma_{\atY}(\unit_{\ec.\atX.\atY.\atY})} t(x,y,y) : C(x,y)_{\T(\ec.\atX.\atY)}
\]
I think that $\Id$-types are for terms that collapse some dependencies of their type (corresponding to a non-injective map of profunctors), but I haven't quite made that out in syntax yet.
I also think that by putting all these together we should be able to show that we can represent all profunctors if we need to.


\end{document}


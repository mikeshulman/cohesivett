Recently, several modal extensions of homotopy type theory have been
investigated, with the goal of extending the synthetic style of
formalizing mathematics to additional situations.  For example,
real-cohesive homotopy type theory can describe types with both a
groupoid structure and a separate topological structure.  These modal
dependent type theories add new type operators to the syntax, which
typically are given universal properties relative to new judgement
forms.  To facilitate the design of such type theories, we introduce a
general framework for modal dependent type theories, building on our
previous work for simple type theories.  The framework consists first of
a base directed dependent type theory, which serves as a language for
specifying a signature of desired modalities, which we call a mode
theory.  This mode theory is the parameter to a second type theory,
which gives general rules for working with the modalities it describes.
The mode theory language is flexible enough to describe a variety of
modalities, including adjunctions, monads, comonads, idempotent
(co)monads, and so on; as examples, we give mode theories for ordinary
non-modal dependent type theory with $\Pi$ and $\Sigma$ types, for a
dependent adjoint pair of modalities, and for the spatial type theory
used in real-cohesion.  One advantage of our framework is that we can
give it a categorical semantics for all mode theories at once, which
saves some of the effort involved in translating each type theory
individually, and we describe a category-with-families-like semantics.
While the framework does not automatically produce ``optimized''
inference rules for a particular modal discipline (where structural
rules are as admissible as possible), it does provide a convenient
syntactic setting for investigating such issues, including a general
equational theory governing the placement of structural rules in types
and in terms.

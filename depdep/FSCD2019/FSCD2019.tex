
\documentclass[UKenglish,letterpaper,cleveref,autoref]{lipics-v2019}
%This is a template for producing LIPIcs articles. 
%See lipics-manual.pdf for further information.
%for A4 paper format use option "a4paper", for US-letter use option "letterpaper"
%for british hyphenation rules use option "UKenglish", for american hyphenation rules use option "USenglish"
%for section-numbered lemmas etc., use "numberwithinsect"
%for enabling cleveref support, use "cleveref"
%for enabling cleveref support, use "autoref"

  \usepackage{xcolor}
  \definecolor{darkgreen}{rgb}{0,0.45,0} 
  %% \usepackage[pagebackref,colorlinks,citecolor=darkgreen,linkcolor=darkgreen]{hyperref}
  %% \usepackage{pdflscape}

\usepackage{amssymb,amsthm,bbm}
\usepackage{amsmath}
%% \usepackage[mathscr]{euscript}
\usepackage{dsfont}
 \usepackage{fontawesome}
 \usepackage{tikz-cd}
\usepackage{mathpartir}
\usepackage{enumitem}
\usepackage[status=draft,inline,nomargin]{fixme}
\FXRegisterAuthor{ms}{anms}{\color{blue}MS}
\FXRegisterAuthor{mvr}{anmvr}{\color{olive}MVR}
\FXRegisterAuthor{drl}{andrl}{\color{purple}DRL}
\usepackage{stmaryrd}
\usepackage{mathtools}

%% \newtheorem{theorem}{Theorem}
%% \newtheorem{proposition}{Proposition}
%% \newtheorem{lemma}{Lemma}
%% \newtheorem{corollary}{Corollary}
\newtheorem{problem}{Problem}
\newenvironment{constr}{\begin{proof}[Construction]}{\end{proof}}

\theoremstyle{definition}
%% \newtheorem{definition}{Definition}
%% \newtheorem{remark}{Remark}
%% \newtheorem{example}{Example}

\let\oldemptyset\emptyset%
\let\emptyset\varnothing

\newcommand\dsd[1]{\ensuremath{\mathsf{#1}}}

\newcommand{\yields}{\vdash}
\newcommand{\Yields}{\tcell}
\newcommand{\tcell}{\Rightarrow}
\newcommand{\cbar}{\, | \,}
\newcommand{\judge}{\mathcal{J}}

\newcommand{\Id}[3]{\mathsf{Id}_{{#1}}(#2,#3)}
\newcommand{\CTX}{\,\,\mathsf{Ctx}}
\newcommand{\ctx}{\,\,\mathsf{mctx}}
\newcommand{\TYPE}{\,\,\mathsf{Type}}
\newcommand{\type}{\,\,\mathsf{mode}}
\newcommand{\TELE}{\,\,\mathsf{Tele}}
\newcommand{\tele}{\,\,\mathsf{mtele}}

\newcommand{\app}[2]{\ensuremath{#1 \: #2}}
\newcommand{\telety}[3]{\ensuremath{(#1{:}#2,#3)}}
\newcommand{\mt}[0]{\ensuremath{()}}
\newcommand{\sigmacl}[3]{\ensuremath{\textnormal{$\Sigma$}\,#1{:}#2.\,#3}}
\newcommand{\fst}[1]{\app{\dsd{fst}}{#1}}
\newcommand{\snd}[1]{\app{\dsd{snd}}{#1}}
\newcommand\extend[2]{\ensuremath{(#1,\id_{#2})}}

\newcommand\fan[1]{\ensuremath{\mathsf{fan}_{#1}}}

\newcommand{\id}{\mathsf{id}}
\DeclareMathOperator{\ob}{ob}

\newcommand{\rewrite}[2]{\overleftarrow{#1}(#2)}
\newcommand\F[2]{\ensuremath{\mathsf{F}_{#1}(#2)}}
\newcommand\U[3]{\ensuremath{\mathsf{U}_{#1}(#2 \mid #3)}}
\newcommand\UE[2]{\ensuremath{#1(#2)}}
\newcommand\UI[2]{\ensuremath{\lambda #1.#2}}
\newcommand\St[2]{\ensuremath{{#1}^*(#2)}}
\newcommand\StI[2]{\ensuremath{\mathsf{st}_{#1}(#2)}}
\newcommand\UStI[2]{\ensuremath{\mathsf{ust}_{#1}(#2)}}
\newcommand\UnSt[2]{\ensuremath{\mathsf{unst}_{#1}(#2)}}
%\newcommand\StE[2]{\ensuremath{\mathsf{unst}(#1,#2)}}
\newcommand\StE[4]{\ensuremath{\mathsf{let} \, \StI{#1}{#3} \, = \, {#2} \, \mathsf{in} \, #4}}
\newcommand\FE[3]{\ensuremath{\mathsf{let} \, \mathsf{F}(#2) \, = \, {#1} \, \mathsf{in} \, #3}}
% With subscript:
\newcommand\FEs[4]{\ensuremath{\mathsf{let} \, \mathsf{F}_{#1}(#3) \, = \, {#2} \, \mathsf{in} \, #4}} 
\newcommand\FI[1]{\ensuremath{\mathsf{F}{(#1)}}}
\newcommand\FIs[2]{\ensuremath{\mathsf{F}_{#1}{(#2)}}}
\newcommand\TypeTwo[4]{\ensuremath{#1 \vdash #2 :  #3 \tcell #4}}
\newcommand\TeleTwo[4]{\ensuremath{#1 \vdash #2 : #3 \tcell #4}}
\newcommand\TermTwo[4]{\ensuremath{#1 \vdash #2 : #3 \tcell #4}}
\newcommand\TermTwoT[5]{\ensuremath{#1 \vdash {#2} : #3 \tcell_{#5} #4}}
%% \newcommand\TermTwoDisp[5]{\ensuremath{#1 \mid #3 \tcell_{\mathsf{disp}} #2 :_{#5} #4}}
%\newcommand\SubTwo[4]{\ensuremath{#1 \mid #3 \tcell #2 : #4}}
\newcommand\TrPlus[2]{\ensuremath{{#1}^+(#2)}}
\newcommand\TrCirc[2]{\ensuremath{{#1}^\circ(#2)}}

\newcommand\El[2]{\mathcal{T}_{#1}(#2)}
\newcommand\ApEl[2]{\mathcal{T}_{#1}\langle#2\rangle}
\newcommand\bdot[0]{\mathbin{.}}
\newcommand\bang[0]{\mathord{!}}

\newcommand\ap[2]{\ensuremath{#1 \langle #2 \rangle }}
\newcommand\ApPlus[2]{\ensuremath{{#1}^+ \langle #2 \rangle }}
\newcommand\ApCirc[2]{\ensuremath{{#1}^\circ \langle #2 \rangle }}

% Macros for semantics notation
\newcommand\mm[1]{\llbracket #1 \rrbracket}
\newcommand\op{^{\mathrm{op}}}
\newcommand\co{^{\mathrm{co}}}
\newcommand\coop{^{\mathrm{coop}}}
\newcommand\Cat{\mathrm{Cat}}
\newcommand\CAT{\mathrm{CAT}}
\newcommand\M{\mathcal{M}}
\newcommand\Mhat{\widehat{\mathcal{M}}}
\newcommand\Mty{{\mathrm{Ty}_{\M}}}
\newcommand\Mtm{{\mathrm{Tm}_{\M}}}
\newcommand\Mtyhat{{\widehat{\mathrm{Ty}}_{\M}}}
\newcommand\Mtmhat{{\widehat{\mathrm{Tm}}_{\M}}}
\newcommand\Ups{\Upsilon}
\newcommand\Upshat{{\widehat{\Upsilon}}}
\newcommand\C{\mathcal{C}}
\newcommand\Chat{{\widehat{\mathcal{C}}}}
\newcommand\Cty{\mathrm{Ty}_{\C}}
\newcommand\Ctm{\mathrm{Tm}_{\C}}
\newcommand\Ctyhat{{\widehat{\mathrm{Ty}}}_{\C}}
\newcommand\Ctmhat{{\widehat{\mathrm{Tm}}}_{\C}}
\newcommand\vp{\varpi}
\newcommand\vpst{\vp^*}
\newcommand\vpsh{\vp_!}
\newcommand\vptil{\widetilde{\vp}}
\newcommand\vpty{{\vp}_{\mathrm{Ty}}}
\newcommand\vptm{{\vp}_{\mathrm{Tm}}}
\newcommand\name[1]{\ulcorner #1\urcorner}
\newcommand{\Util}{\widetilde{U}}
\newcommand\ev{\mathrm{ev}}
\DeclareSymbolFont{bbold}{U}{bbold}{m}{n}
\DeclareSymbolFontAlphabet{\mathbbb}{bbold}
\newcommand\one{\mathbbb{1}}

%\graphicspath{{./graphics/}}%helpful if your graphic files are in another directory

\bibliographystyle{plainurl}% the mandatory bibstyle

\title{A Fibrational Framework for Modal Dependent Type Theories}
%\titlerunning{}%optional, please use if title is longer than one line

\author{Daniel R. Licata}{Wesleyan University}{dlicata@wesleyan.edu}{https://orcid.org/0000-0003-0697-7405}{}

\author{Mitchell Riley}{Wesleyan University}{mvriley@wesleyan.edu}{?}
       {}%TODO mandatory, please use full name; only 1 author per \author macro; first two parameters are mandatory, other parameters can be empty. Please provide at least the name of the affiliation and the country. The full address is optional

\author{Michael Shulman}{University of San Diego}{shulman@sandiego.edu}{?}
       {}%TODO mandatory, please use full name; only 1 author per \author macro; first two parameters are mandatory, other parameters can be empty. Please provide at least the name of the affiliation and the country. The full address is optional

\authorrunning{D.\,R. Licata and M. Riley and M. Shulman}%TODO mandatory. First: Use abbreviated first/middle names. Second (only in severe cases): Use first author plus 'et al.'

\Copyright{Daniel R. Licata and Mitchell Riley and Michael Shulman}%TODO mandatory, please use full first names. LIPIcs license is "CC-BY";  http://creativecommons.org/licenses/by/3.0/

\ccsdesc[500]{Theory of computation~Type theory}%TODO mandatory: Please choose ACM 2012 classifications from https://dl.acm.org/ccs/ccs_flat.cfm 

\keywords{homotopy type theory, modal logic
  % bicubical sets, dependent type theory,  synthetic \infty-categories,
  % model categories, Reedy categories, internal language, univalent
  % foundations
}%TODO mandatory; please add comma-separated list of keywords

%% \category{}%optional, e.g. invited paper

%% \relatedversion{}%optional, e.g. full version hosted on arXiv, HAL, or other respository/website
%% %\relatedversion{A full version of the paper is available at \url{...}.}

%% \supplement{}%optional, e.g. related research data, source code, ... hosted on a repository like zenodo, figshare, GitHub, ...

%% %\funding{(Optional) general funding statement \dots}%optional, to capture a funding statement, which applies to all authors. Please enter author specific funding statements as fifth argument of the \author macro.

%% \acknowledgements{I want to thank \dots}%optional

%% %\nolinenumbers %uncomment to disable line numbering

%% %\hideLIPIcs  %uncomment to remove references to LIPIcs series (logo, DOI, ...), e.g. when preparing a pre-final version to be uploaded to arXiv or another public repository

%% %Editor-only macros:: begin (do not touch as author)%%%%%%%%%%%%%%%%%%%%%%%%%%%%%%%%%%
%% \EventEditors{John Q. Open and Joan R. Access}
%% \EventNoEds{2}
%% \EventLongTitle{42nd Conference on Very Important Topics (CVIT 2016)}
%% \EventShortTitle{CVIT 2016}
%% \EventAcronym{CVIT}
%% \EventYear{2016}
%% \EventDate{December 24--27, 2016}
%% \EventLocation{Little Whinging, United Kingdom}
%% \EventLogo{}
%% \SeriesVolume{42}
%% \ArticleNo{23}
%%%%%%%%%%%%%%%%%%%%%%%%%%%%%%%%%%%%%%%%%%%%%%%%%%%%%%

\begin{document}

\maketitle

%TODO mandatory: add short abstract of the document
\begin{abstract}
Recently, several modal extensions of homotopy type theory have been
investigated, with the goal of extending the synthetic style of
formalizing mathematics to additional situations.  For example,
real-cohesive homotopy type theory can describe types with both a
groupoid structure and a separate topological structure.  These modal
dependent type theories add new type operators to the syntax, which
typically are given universal properties relative to new judgement
forms.  To facilitate the design of such type theories, we introduce a
general framework for modal dependent type theories, building on our
previous work for simple type theories.  The framework consists first of
a base directed dependent type theory, which serves as a language for
specifying a signature of desired modalities, which we call a mode
theory.  This mode theory is the parameter to a second type theory,
which gives general rules for working with the modalities it describes.
The mode theory language is flexible enough to describe a variety of
modalities, including adjunctions, monads, comonads, idempotent
(co)monads, and so on; as examples, we give mode theories for ordinary
non-modal dependent type theory with $\Pi$ and $\Sigma$ types, for a
dependent adjoint pair of modalities, and for the spatial type theory
used in real-cohesion.  One advantage of our framework is that we can
give it a categorical semantics for all mode theories at once, which
saves some of the effort involved in translating each type theory
individually, and we describe a category-with-families-like semantics.
While the framework does not automatically produce ``optimized''
inference rules for a particular modal discipline (where structural
rules are as admissible as possible), it does provide a convenient
syntactic setting for investigating such issues, including a general
equational theory governing the placement of structural rules in types
and in terms.

\end{abstract}

\section{Introduction}

\section{Introduction}

In ordinary intuitionistic logic or $\lambda$-calculus, assumptions or
variables can go unused (weakening), be used in any order (exchange), be
used more than once (contraction), and be used in any position in a
term.  \emph{Substructural} logics, such as linear logic, ordered logic,
relevant logic, and affine logic, omit some of these structural
properties of weakening, exchange, and contraction, while \emph{modal
  logics} place restrictions on where variables may be used---e.g. a
formula $\Bx{} C$ can only be proved using assumptions of $\Bx{} A$,
while an assumption of $\Dia{}{A}$ can only be used when the conclusion
is $\Dia{}{C}$.  Substructural and modal logics have had many
applications to both functional and logic programming (modeling concepts
such state, staging, distribution, and concurrency, to name just a few).
They are also used as \emph{internal languages} of categories: one uses
an appropriate logical language to do constructions ``inside'' a
particular mathematical setting, which often leads to shorter statements
than working ``externally''.  For example, to define a function
externally in domains, one must first define the underlying
set-theoretic function, and then prove that it is continuous; but when
using untyped $\lambda$-calculus as an internal language of domains,
there is no need to prove that a function described by a $\lambda$-term
is continuous, because all terms are shown to denote continuous
functions.  Substructural logics extend this idea to
various forms of monoidal categories, while modal logics describe monads
and comonads.  Recently,
\citet{schreibershulman12cohesive,shulman15realcohesion} proposed using
modal operators to add a notion of \emph{cohesion} to homotopy type
theory/univalent foundations~\citep{voevodsky06homotopy,uf13hott-book}.
Without going into the precise details of this application, the idea is
to add a triple $\sh{} \la \Flat{} \la \Sharp{}$ of type operators,
where for example $\Sharp{}$ is a monad (like a modal possibility
$\Diamond$ or $\bigcirc$), $\Flat{}$ is a comonad (like a modal
necessity $\Box$), and there is an adjunction structure between them
(e.g. $\flat{A} \to B$ is the same as $A \to \Sharp{B}$).  This raised
the question of how to best add modalities with these properties to type
theory.

Because other similar applications have different monads and comonads
with different properties, we would like general tools for going from a
semantic situation of interest to a well-behaved logic/type theory for
it, e.g. one with cut and identity admissibility (normalization,
$\eta$-expansion).  In previous work~\citep{ls16adjoint}, we considered
the special case of a single-assumption logic, building most directly on
the adjoint logics of
\citet{benton94mixed,bentonwadler96adjoint,reed09adjoint}.  Here we
extend this previous work to the multi-assumption case.  The resulting
framework is quite general and covers many existing intuitionistic
substructural and modal connectives: cartesian, linear, affine,
relevant, ordered, bunched~\citep{ohearnpym99bunched} and
non-associative products and implications; $n$-linear
variables~\citep{reed08namessubstructural,abel15modal,mcbride16nuttin};
the comonadic $\Box$ and linear exponential $!$ and
subexponentials~\citep{nigammiller09subexponentials,danos+93subexponentials};
monadic $\Diamond$ and $\bigcirc$ modalities; and adjoint logic $F$ and
$G$~\citep{benton94mixed,bentonwadler96adjoint,reed09adjoint}, including
the single-assumption 2-categorical version from our previous
work~\citep{ls16adjoint}.  It also supports variations on these, such as
non-monoidal comonads and non-strong monads.  A central syntactic result
is that cut and identity are admissible for our framework
itself, which implies cut admissibility for any logic that can be
described in the framework, including all of the above, as well as any
new logics that one designs using it.  When we view the derivations in
the framework as terms in a type theory, this gives an immediate
normalization (and $\eta$-expansion) result.
%% While it is not too surprising
%% that this is possible, given that cut proofs for these logics all follow
%% a similar template, it is nonetheless satisfying to codify this pattern
%% as an abstraction.

At a high level, the framework is based on the idea that all of the
above logics / type theories are a restriction on how variables can be
used in ordinary structural/cartesian proofs.  We express these
restrictions using a first layer, which is a simple type theory for what
we will call \emph{modes} and \emph{context descriptors}.  The modes are
just a collection of base types, which we write as $p,q,r$, while a
context descriptor $\alpha$ is a term built from variables and
constants.  The next layer is the main logic.  Each proposition/type is
assigned a mode, and the basic sequent is \seq{x_1 : A_1, \ldots, x_n :
  A_n}{\alpha}{C}, where if $A_i$ has mode $p_i$, and $C$ has mode $q$,
then $\oftp{x_1 : p_1,\ldots, x_n : p_n}{\alpha}{q}$.  We use a sequent
calculus to concisely describe cut-free derivations/normal forms, but
everything can be translated to natural deduction in the usual way.
$\Gamma$ itself behaves like an ordinary structural/cartesian context,
while the substructural and modal aspects are enforced by the
\emph{term} $\alpha$, which constrains how the resources from $\Gamma$
may be used.  For example, in linear logic/ordered logic/BI, the context
is usually taken to be a multiset/list/tree.  We represent this by a
pair of an ordinary structural context $\Gamma$, together with a term
$\alpha$ that describes the multiset or list or tree structure, labeled
with variables from the ordinary context at the leaves.  We pronounce a
sequent \seq{\Gamma}{\alpha}{A} as ``$\Gamma$ proves $A$ \{along,over\}
$\alpha$''.

For example, suppose we have one mode $\dsd{n}$, together with a context
descriptor constant $x : \dsd{n}, y:\dsd{n} \vdash x \odot y : \dsd{n}$.
Then an example sequent \seq{x:A, y:B, z:C, w:D}{(y \odot x) \odot z}{E}
should be read as saying that we must prove $E$ using the resources $y$
and $x$ and $z$ (but not $w$) according to the particular tree structure
${(y \odot x) \odot z}$.  If we say nothing else, the framework will
treat $\odot$ as describing a non-associative, linear, ordered context
as in Lambek calculus~\citep{lambek58calculus}: if we have a
product-like type $A \odot B$ internalizing this context
operation,\footnote{We overload binary operations to refer both to
  context descriptors and propositional connectives, because it is clear
  from whether it is applied to variables $x,y,z$ or propositions
  $A,B,C$ which we mean.}  then we will \emph{not} be able to prove
associativity ($(A \odot B) \odot C \dashv\vdash A \odot (B \odot C)$)
or exchange ($A \odot B \vdash B \odot A$) etc.  
\ifthenelse{\boolean{short}}{}{ 
%% para break
}
To get from this basic structure to a linear or affine or relevant or
cartesian system, we provide a way to add structural properties governing
the context descriptor term $\alpha$.  We analyze structural properties
as \emph{equations}, or more generally \emph{directed transformations},
on such terms.  For example, to specify linear logic, we will add a unit
element $1 : \dsd{n}$ together with equations making $(\odot,1)$ into a
commutative monoid ($x \odot (y \odot z) = (x \odot y) \odot z$ and 
$x \odot 1 = x = 1 \odot x$ and 
$x \odot y = y \odot x$)
so that the context descriptors ignore associativity and order.  To get
BI, we add an additional commutative monoid $(\times,\top)$ (with
weakening and contraction, as discussed below), so that a BI context
tree $(x:A,y:B);(z:C,w:D)$ can be represented by the ordinary context
$x:A,y:B,z:C,w:D$ with the term $(x \odot y) \times (z \odot w)$
describing the tree.  Because the context descriptors are themselves
ordinary structural/cartesian terms, the same variable can occur more
than once or not at all.  A descriptor such as $x \odot x$ captures the
idea that we can use the \emph{same} variable $x$ twice, expressing
$n$-linear types.  Thus, we can express contraction for a particular
context descriptor $\odot$ as a transformation $x \spr x \odot x$ (one
use of $x$ allows two).  Weakening, on the other hand, is represented by
a transformation $x \spr 1$, which is oriented to allow throwing away an
allowed use of $x$, but not creating an allowed use from nothing.  We
refer to these as \emph{structural transformations}, to evoke their use
in representing the structural properties of object logics that are
embedded in our framework.  The main sequent $\seq{\Gamma}{\alpha}{A}$
respects the specified structural properties in the sense that when
$\alpha = \beta$, we regard $\seq{\Gamma}{\alpha}{A}$ and
$\seq{\Gamma}{\beta}{A}$ as the same sequent, while when $\alpha \spr
\beta$, there will be an operation that takes a derivation of
\seq{\Gamma}{\beta}{A} to a derivation of \seq{\Gamma}{\alpha}{A}.

Modal logics will generally involve a mode theory with more than one
mode.  For example, a context descriptor $x : \dsd{c} \vdash \dsd{f}(x)
: \dsd{l}$ will generate an adjoint pair of functors between the two
modes, as in the adjoint syntax for linear logic's
$!$~\citep{bentonwadler96adjoint} or other modal
operators~\citep{reed09adjoint}.  Structural transformations are used to
describe how these modal operators interact with each other and with the
product structures, and in some cases~\citep{ls16adjoint} it is
important that there can be more than one structural transformation
between a given pair of context descriptors.

A guiding principle of the framework is a meta-level notion of
\emph{structurality over structurality}.  For example, we always have
\emph{weakening over weakening}: if \seq{\Gamma}{\alpha}{A} then
\seq{\Gamma,y:B}{\alpha}{A}, where $\alpha$ itself is weakened with $y$.
This does not prevent encodings of e.g. linear logic: it is permissible
to weaken a derivation of \seq{\Gamma}{x_1 \odot \ldots \odot x_n}{A}
(``use $x_1$ through $x_n$'') to a derivation of \seq{\Gamma,y:B}{x_1
  \odot \ldots \odot x_n}{A} because the (weakened) context descriptor
still disallows the use of $y$.  Similarly, we have exchange over
exchange and contraction over contraction.  The \emph{identity-over-identity}
principle says that we should be able to prove $A$ using exactly an
assumption $x:A$ ({\seq{\Gamma,x:A}{x}{A}}).  The cut principle says
that from \seq{\Gamma,x:A}{\beta}{B} and \seq{\Gamma}{\alpha}{A} we get
{\seq{\Gamma}{\subst{\beta}{\alpha}{x}}{B}}---the context descriptor for
the result of the cut is the substitution of the context descriptor used
to prove $A$ into the one used to prove $B$.  For example, together with
weakening-over-weakening, this captures the usual cut principle of
linear logic, which says that cutting $\Gamma,x:A \vdash B$ and $\Delta
\vdash A$ yields $\Gamma,\Delta \vdash B$.  If $\Gamma$ binds
$x_1,\ldots,x_n$ and $\Delta$ binds $y_1,\ldots,y_n$, then we will
represent the two derivations to be cut together by sequents with $\beta
= x_1 \odot \ldots \odot x_n \odot x$ and $\alpha = y_1 \odot \ldots
\odot y_n$ then $\beta[\alpha/x] = x_1 \odot \ldots \odot x_n \odot y_1
\odot \ldots \odot y_n$ correctly deletes $x$ and replaces it with the
variables from $\Delta$.  In more subtle situations such as BI, the
substitution will insert the resources used to prove the cut formula in
the correct place in the tree.

The framework has two main logical connectives / type constructors.  The
first, \F{\alpha}{\Delta}, generalizes the \dsd{F} of adjoint logic and
the multiplicative products (e.g. $\otimes$ of linear logic).  The
second, \U{x.\alpha}{\Delta}{A}, generalizes the $\dsd{G}/\dsd{U}$ of
adjoint logic and implication (e.g. $A \lolli B$ in linear logic).  Here
$\Delta$ is a context of assumptions $x_i:A_i$, and trivializing the
context descriptors (i.e. adding an equation $\alpha = \beta$ for all
$\alpha$ and $\beta$) degenerates $\F{\alpha}{\Delta}$ into the ordinary
intuitionistic product $A_1 \times \ldots \times A_n$, while
\U{x.\alpha}{\Delta}{A} becomes $A_1 \to \ldots \to A_n \to A$.  
As one would expect, \dsd{F} is left-invertible and \dsd{U} is right-invertible.
%%  and we conjecture that focusing works with the
%% polarization that one would expect based on these degeneracies
%% ($\F{\alpha}{\Delta^{\mathord{+}}}^{\mathord{+}}$ and
%% $\U{x.\alpha}{\Delta^{\mathord{+}}}{A^{\mathord{-}}}^{\mathord{-}}$).
In linear logic terms, our \dsd{F} and \dsd{U} cover both the
multiplicatives and exponentials; additives can be added separately by
the usual rules.  We discuss many examples of \emph{logical adequacy}
theorems, showing that a sequent can be proved in a standard sequent
calculus for a logic iff its embedding using these connectives can be
proved in the framework.

%% In summary, to specify a particular substructural or modal logic / type
%% theory, one gives constants generating context descriptors $\alpha$,
%% with equations $\alpha = \beta$ and transformations $\alpha \spr \beta$
%% expressing structural properties.  

Being a very general theory, our framework treats the object-logic
structural properties in a general but na\"ive way, allowing an
arbitrary structural transformation to be applied at the non-invertible
rules for $\dsd{F}$ and $\dsd{U}$ and at the leaves of a derivation.
For specific embedded logics, there is often a more refined discipline
that suffices---e.g. for cartesian logic, always contract all
assumptions in all premises, and only weaken at the leaves.  We view our
framework as a tool for bridging the gap between an intended semantic
situation (such as the cohesion example mentioned, ``a comonad and a
monad which are themselves adjoint'') and a proof theory: the framework
gives \emph{some} proof theory for the semantics, and the placement of
structural rules can then be optimized purely in syntax.  To support
this mode of use, we give an equational theory on sequent derivations
that identifies different placements of the same structural rules, which
can be used to prove correctness of such optimizations not just at the
level of provability, but also identity of derivations---which matters
for our intended applications to internal languages.  We discuss some
preliminary work on \emph{equational adequacy}, which extends the
logical correspondence to isomorphisms of definitional-equality-classes
of derivations.

Semantically, the logic corresponds to a functor between
\emph{2-dimensional cartesian multicategories} which is a fibration in
various senses.  Multicategories are a generalization of categories
which allow more than one object in the domain, and cartesianness means
that the multiple domain objects are treated structurally.  The
2-dimensionality supplies a notion of morphism between (multi)morphisms.
A \emph{mode theory} specifying context descriptors and structural
properties is analyzed as a cartesian 2-multicategory, with the
descriptors as 1-cells and the structural properties as 2-cells.  The
functor relates the sequent judgement to the mode theory, specifying the
mode of each proposition and the context descriptor of a sequent.  The
fibration conditions (similar to
\citep{hermida02fibrations,hormann15multicategories}) give respect for
the structural transformations and the presence of \dsd{F} and \dsd{U}
types.  We prove that the sequent calculus and the equational theory are
sound and complete for this semantics: the syntax can be interpreted in
any bifibration, and itself determines one.  This semantics shows that
an interesting class of type theories can be identified with a class of
more mathematical objects, fibrations of cartesian 2-multicategories,
thus providing some progress towards characterizing substructural
and modal type theories in mathematical terms.

\ifthenelse{\boolean{short}}{}{Our framework builds on many approaches to substructural and modal logic
in the literature.  Logical rules that act at a leaf of a
tree-structured context go back to the Lambek
calculus~\citep{lambek58calculus}.  A rich collection of context
structures that correspond to type constructors plays a central role in
display logic~\citep{belnap82display}.  \citet{atkey04separation}'s
$\lambda$-calculus for resource separation is similar to mode theories
with one mode, where there is at most one 2-cell between a given pair of
1-cells; at the logical level, our calculus is a unification of this
with the multimode adjoint logic of
\citet{reed09adjoint}.  Algebraic resource annotations on variables are
used to track modalities in Agda's implementation~\citep{abel15modal}
and in \citet{mcbride16nuttin}'s approach to linear dependent types.  LF
representations of modal or substructural logics work by restricting the
use of cartesian variables~\citep{crary10substructural}.  Relative to
all of these approaches, we believe that the analysis of the context
structures/resources as a \emph{term} in a base type theory, and the
fibrational structure of the derivations over them, is a new and useful
observation.  For example, rather than needing extra-logical conditions
on proof rules to ensure cut admissibility, as in display logic, the
conditions are encoded in the language of context descriptors and the
definition of types from them.  Moreover, none of these existing
approaches allow for proof-relevant 2-cells/structural rules, and their
presence (and the equational theory we give for them) is important for
our applications to extensions of homotopy type theory.  A point of
contrast with substructural logical
frameworks~\citep{cervesatopfenning02llf,watkins+03clf-tr,reed09thesis}
is that logics are ``embedded'' in our calculus (giving a type
translation such that provability in the object logic corresponds to
provability in ours), rather than ``encoding'' the structure of
derivations.  This way, we obtain cut elimination for object languages
as a corollary of framework cut elimination.
}

%% Similar semantic structures have come up recently
%% in~\citep{zeilberger,mellieszeilberger,johann}.  
 


\section{Mode Theory}

Mode theory judgements:
\begin{enumerate}
\item $\gamma \ctx$ (empty, extension)
\item $\gamma \yields p \type$ 
\item $\TypeTwo{\gamma}{s}{p}{q}$ (horizontal and vertical composition, identities)
\item $\gamma \yields \mu : p$ (variables, action of mode type morphisms)
\item $\TermTwoT{\gamma}{s}{\mu}{\nu}{p}$ (horizontal and vertical
    composition, identities)
\end{enumerate}

\begin{enumerate}

\item Contexts are as usual:

\begin{mathpar}
  \inferrule*{ }
             {\cdot \ctx}
             
  \inferrule*
    {\gamma \ctx \\
     \gamma \yields p \type}
    {\gamma,x:p \ctx}
\end{mathpar}  

\item In all mode theories, terms must have: 

\begin{mathpar}
\inferrule*{ }
             {\gamma,x : p, \gamma' \yields x : p}
             
\inferrule*
    {\gamma \yields \mu : q \\
     \TypeTwo{\gamma}{s}{p}{q}
    }
    {\gamma \yields \TrPlus{s}{\mu} : p}
\\
\TrPlus{\id}{\mu} \equiv \mu \qquad
\TrPlus{s'}{\TrPlus{s}{\mu}} \equiv \TrPlus{(s';s)}{\mu} 
\end{mathpar}

\item Mode type morphisms:
\begin{mathpar}
    \inferrule*{ }
          {\TypeTwo{\gamma}{\id_p}{p}{p}}
    \qquad
    \inferrule*{{\TypeTwo{\gamma}{s_1}{p_1}{p_2}} \\
                {\TypeTwo{\gamma}{s_2}{p_2}{p_3}}
          }
          {\TypeTwo{\gamma}{s_1;s_2}{p_1}{p_3}}

\inferrule*{{\gamma,x:p} \vdash {q} \type \\
            \TermTwoT{\gamma}{t}{\mu}{\mu'}{p}\\
           } 
           {\TypeTwo{\gamma}{\ap {q} {t/x}}{q[\mu/x]}{q[\mu'/x]}}

\\
\id;s \equiv s \equiv s;\id \and
(s;s');s'' \equiv s;(s';s'') \\ 
\ap q {\id_{\mu}/x} \equiv \id_{q[\mu/x]} \and
\ap q {(s;t)/x} \equiv \ap q {s/x}; \ap q {t/x} \\ 
\ap q {s/\_} \equiv \id_q \\ 
\ap {(q[\mu/x])} {s/y} \equiv \ap q {\ap \mu {s/y}/x} \quad (\text{where } \gamma,y:p' \vdash \mu : p \text{ and } \gamma,x:p \vdash q \type)\\
s[\nu/x];\ap{q'}{t/x} \equiv \ap{q}{t/x};s[\nu'/x] \quad 
(\text{where } \TypeTwo{\gamma,x:p}{s}{q}{q'} \text{ and } \TermTwoT{\gamma}{t}{\nu}{\nu'}{p})

%% subst: \id_\mu[\nu/x] = \id_{\mu[\nu/x]}
%% subst: s[x/x] = s
%% subst: (s;t)[\mu/x] = s[\mu/x];t[\mu/x]
%% subst: s[\mu[\nu/x]/x] = s[\mu/x][\nu/x]
%% subst: ap q (s [\mu/x]) = (ap q s)[\mu/x] and generalization
\end{mathpar}

We write $\ap q {t/x}$ for whiskering (\dsd{ap} in book HoTT).

\item 2-cells between terms.  First, we have
  identity/composition/whiskering and associated equations (whiskering
  on the other side is given by substitution):
\begin{mathpar}
    \inferrule*{ }
          {\TermTwoT{\gamma}{\id_\mu}{\mu}{\mu}{p}}
    \qquad
    \inferrule*{{\TermTwoT{\gamma}{s_1}{\mu_1}{\mu_2}{p}} \\
                {\TermTwoT{\gamma}{s_2}{\mu_2}{\mu_3}{p}}
          }
   {\TermTwoT{\gamma}{s_1;s_2}{\mu_1}{\mu_3}{p}}

\inferrule*{{\gamma,x:p} \yields {\nu} : {q} \\
            \TermTwoT{\gamma}{s}{\mu}{\mu'}{p}\\
           } 
           {\TermTwoT{\gamma}{\ap \nu {s/x}}{\nu[\mu/x]}{\TrPlus{\ap{q}{s/x}}{\nu[\mu'/x]}}{q[\mu/x]}}

\\           
\id;s \equiv s \equiv s;\id \and
(s;s');s'' \equiv s;(s';s'') \\ 
\ap \nu {\id_{\mu}/x} \equiv \id_{\nu[\mu/x]} \and
\ap \nu {(s;t)/x} \equiv \ap \nu {s/x} ; (\ap {(\TrPlus{\ap{q}{s/x}}{y})} {\ap \nu {t/x}/y}) \\ 
\ap x {s/x} \equiv s  \and
\ap \nu {s/\_} \equiv \id_\nu \and
\ap {(\nu[\mu/x])} {s/y} \equiv \ap \nu {\ap \mu {s/y}/x} \quad
(\text{where } \gamma,y:p' \vdash \mu : p \text{ and } \gamma,x:p \vdash \nu : q)\\
t[\mu/x];\ap{\nu'}{s/x} \equiv \ap{\nu}{s/x};\ApPlus{\ap{q}{s/x}}{t[\mu'/x]} \quad
 (\text{where } \TermTwoT{\gamma,x:p}{t}{\nu}{\nu'}{q} \text{ and } \TermTwoT{\gamma}{s}{\mu}{\mu'}{p}) \\
\ap{\TrPlus{s}{\mu}}{t/x} \equiv \ApPlus{(s[\nu/x])}{\ap{\mu}{t/x}}\quad 
(\text{where } \TypeTwo{\gamma,x:p}{s}{q}{q'} \text{ and } \TermTwoT{\gamma}{t}{\nu}{\nu'}{p})
\end{mathpar}

\item We assume $1/\Sigma$ modes:

\begin{mathpar}
  \inferrule*{ } { \gamma \yields 1 \type } \and
  
  \inferrule*{ \gamma \yields p \type \\ 
               \gamma,x:p \yields q \type }
             {\gamma \yields \sigmacl{x}{p}{q} \type} \\
             
  \inferrule*{ }
             {\gamma \yields \mt : 1}
  \and 
  \mu \equiv \mt
\\
\inferrule*{
  \gamma \yields \mu : p \and
  \gamma \yields \nu : q[\mu/x]
    }
   {\gamma \yields (\mu,\nu) : \sigmacl{x}{p}{q}}
\and
\inferrule*
    {\gamma \yields \mu : \sigmacl{x}{p}{q}}
    {\gamma \yields \fst \mu : p}
\and
\inferrule*
    {\gamma \yields \mu : \sigmacl{x}{p}{q}}
    {\gamma \yields \snd \mu : q[\fst \mu / x]}
    \\
    \fst{(\mu,\nu)} \equiv \mu \and
    \snd{(\mu,\nu)} \equiv \nu \and
    p \equiv (\fst p, \snd p)
\end{mathpar}

Equations for ``transport'' in $\Sigma$:
\begin{mathpar}
\TrPlus{(\sigmacl{x}{s}{t})}{\mu} \equiv (\TrPlus{s}{\fst \mu},\TrPlus{(t[\fst \mu/x])}{\snd \mu})
\end{mathpar}

Mode type morphisms: We need congruence for $\Sigma$ to be a rule (because we don't have ap on a type variable/universes):
\begin{mathpar}
  \inferrule*
  {\TypeTwo{\gamma}{s}{p}{p'} \\
    \TypeTwo{\gamma,x':p'}{t}{q[\TrPlus{s}{x'}/x]}{q'}}
  {\TypeTwo{\gamma}{\sigmacl{x'}{s}{t}}{\sigmacl{x}{p}{q}}{\sigmacl{x'}{p'}{q'}}} \\

  \sigmacl{x'}{\id_p}{\id_q} \equiv \id_{\sigmacl{x'}{p}{q}} \and
  (\sigmacl{x'}{s}{t});(\sigmacl{x''}{s'}{t'}) \equiv \sigmacl{x''}{(s;s')}{(t[\TrPlus{s'}{x''}/x'];t')} \\

  \ap{(\sigmacl{x'}{p}{q})}{s/(y:r)} \equiv
  \sigmacl{x'}{\ap{p}{s/y}}{\ap{({q[\fst z/x,\snd z/y]})}{\extend{s}{x'}/(z:(\sigmacl{y}{r}{p}))}}
\end{mathpar}

Finally, we have the 2-cells for $1/\Sigma$-terms:
\begin{mathpar}
s \equiv \id_{()} \text{ for } \yields_1 s : () \tcell ()
\\

\inferrule*
    {\TermTwoT{\gamma}{s}{\mu}{\mu'}{p} \and
      \gamma \vdash \nu' : q[\mu'/x]
    }
      {\TermTwoT{\gamma}{\extend{s}{\nu'}}{(\mu,\TrPlus{\ap{q}{s/x}}{\nu'})}{(\mu',\nu')}{\sigmacl{x}{p}{q}}}\\
\ap {\fst(z)} {\extend{s}{\nu'}/z} \equiv s \and
\ap {\snd(z)} {\extend{s}{\nu'}/z} \equiv \id_{\TrPlus{\ap{q}{s/x}}{\nu'}}  \\
s \equiv \ap{(\fst{\mu},y)}{\ap{(\snd z)}{s/z}/y};\extend{\ap{(\fst{z})}{s/z}}{\snd{\mu'}} \quad (\text{where } \TermTwoT{\gamma}{s}{\mu}{\mu'}{\sigmacl{x}{p}{q}})
\\      
{\extend{\id_\mu}{\nu'}} \equiv \id_{(\mu,\nu')} \and
{\extend{(s;s')}{\nu''}} \equiv  \extend{s}{\TrPlus{\ap{q}{s'/x}}{\nu''}};\extend{s'}{\nu''}   \\
\extend{s}{\nu'} ; (\ap{(\mu',y)}{t/y}) \equiv
(\ap{(\mu,\TrPlus{(\ap{q}{s})}{y})}{t/y}); \extend{s}{\nu''} \qquad (\text{where }\TermTwoT{\gamma}{t}{\nu'}{\nu''}{q[\mu'/x]})
\end{mathpar}

\item
  All judgements have a substitution principle
\begin{mathpar}
  \inferrule*{\gamma,x:p,\gamma' \yields J \\
              \gamma \yields \mu : p
              }
             {\gamma,\gamma'[\mu/x] \yields J[\mu/x]} \\

J[\mu/x][\nu/y] \equiv J[\nu/y][\mu[\nu/y]/x]
\end{mathpar}


We sometimes write \ap{\mu}{s} for \ap{\mu(x)}{s/x}, eliding the
variable name when it is clear how to view $\mu$ as a term with a
distinguished variable; e.g. $\ApPlus{s}{t}$ for
$\ap{\TrPlus{s}{x}}{t/x}$.

\subsubsection{Lemmas}

Horizontal composition:
\begin{mathpar}
  \inferrule*[Left=Derivable]
      {\TermTwoT{\gamma}{s}{\mu}{\mu'}{p} \\
    \TermTwoT{\gamma, x : p}{t}{\nu}{\nu'}{q}}
             {\TermTwoT{\gamma}{\ap{t}{s/x} :\equiv t[\mu/x];\ap{\nu'}{s/x}}{\nu[\mu/x]}{\TrPlus{\ap{q}{s/x}}{\nu'[\mu'/x]}}{q[\mu/x]}}
\\ 
\ap{\id_\nu}{s/x} \equiv \ap{\nu}{s/x} \and \ap{t}{\id_{\mu}/x} \equiv t[\mu/x]
\end{mathpar}

Pairing and projection 2-cells are definable:
\begin{mathpar}
  \inferrule*[Left=Derivable]
      {\TermTwoT{\gamma}{s}{\mu}{\mu'}{p} \\
    \TermTwoT{\gamma}{t}{\nu}{\TrPlus{\ap{q}{s}}{\nu'}}{q[\mu/x]}}
             {\TermTwoT{\gamma}{(s,t) :\equiv \ap{(\mu,y)}{t/y};\extend{s}{\nu'}}{(\mu,\nu)}{(\mu',\nu')}{\sigmacl{x}{p}{q}}}

   \inferrule*[Left=Deriv]
              { {\TermTwoT{\gamma}{s}{\mu}{\mu'}{\sigmacl{x}{p}{q}}} }
              { {\TermTwoT{\gamma}{\ap{\fst(y)}{s/y}}{\fst{\mu}}{\fst{\mu'}}{p}} }
   \and
   \inferrule*[Left=Deriv]
              { {\TermTwoT{\gamma}{s}{\mu}{\mu'}{\sigmacl{x}{p}{q}}} }
              { {\TermTwoT{\gamma}{\ap{\snd(y)}{s/y}}{\snd{\mu}}{\TrPlus{\ap{(q(\fst y/x))}{s/y}}{\snd{\mu'}}}{q[\fst{\mu}/x]}} }
\end{mathpar}


\begin{lemma}
For mode term morphisms
\begin{align*}
\TermTwoT{\gamma &}{s}{\mu}{\mu'}{p} \\
\TermTwoT{\gamma &}{t}{\nu}{\TrPlus{\ap{q}{s}}{\nu'}}{q[\mu/x]} \\
\TermTwoT{\gamma &}{s'}{\mu'}{\mu''}{p} \\
\TermTwoT{\gamma &}{t'}{\nu'}{\TrPlus{\ap{q}{s'}}{\nu''}}{q[\mu'/x]}
\end{align*}
we have
\begin{align*}
(s, t);(s', t') \equiv ((s;s'), (t;\ApPlus{\ap{q}{s}}{t'}))
\end{align*}
\end{lemma}
\begin{proof}
Follows by
\begin{align*}
(s, t);(s', t') 
&\equiv \ap{(\mu,y)}{t/y};\extend{s}{\nu'};\ap{(\mu',y)}{t'/y};\extend{s'}{\nu''} \\
&\equiv \ap{(\mu,y)}{t/y};\ap{(\mu, \TrPlus{\ap{q}{s}}{y})}{t'/y};\extend{s}{\TrPlus{\ap{q}{s'}}{\nu''}};\extend{s'}{\nu''} \\
&\equiv \ap{(\mu,y)}{t/y};\ap{(\mu, y)}{\ApPlus{\ap{q}{s}}{t'}/y};\extend{s}{\TrPlus{\ap{q}{s'}}{\nu''}};\extend{s'}{\nu''} \\
&\equiv \ap{(\mu,y)}{t;\ApPlus{\ap{q}{s}}{t'}/y};\extend{s;s'}{\nu''} \\
&\equiv ((s;s'), (t;\ApPlus{\ap{q}{s}}{t'}))
\end{align*}
\end{proof}


\end{enumerate}



\section{Framework}\label{sec:dua-ret}
\subsection{Contexts}

\begin{mathpar}
  \inferrule*[Left = ctx-form]{ }
  {\yields_{\cdot} \cdot \CTX  } \and 

  \inferrule*[Left = ctx-form]{
    \yields_\gamma \Gamma \CTX \and (\text{where } \yields \gamma \ctx) \\\\
    \Gamma \yields_p A \TYPE \and (\text{where }  \gamma \yields p \type)}
  {\yields_{\gamma, x : p} \Gamma, x : A \CTX \and (\text{where } \yields \gamma,x:p \ctx)  } \\
\end{mathpar}

\subsection{Types and Terms}

\subsubsection{Structural Rules}

\begin{mathpar}
  \inferrule*[Left = var]{
    % \yields \Gamma, x : A, \Gamma' \CTX_{\gamma, x : p, \gamma'}
  }
  {\Gamma, x : A, \Gamma' \yields_x x : A \and (\text{where } \gamma,x:p,\gamma' \yields x : p)} \and

 \inferrule*[Left = rewrite]{
   \Gamma \yields_\mu M : A 
   \and \TermTwoT{\gamma}{s}{\nu}{\mu}{p}
  }
  {\Gamma \yields_\nu \rewrite{s}{M} : A} \\ \\
  
  \rewrite{\id_{\mu}}{M} \equiv M \and
  \rewrite{(s;t)}{M} \equiv \rewrite{s}{\rewrite{t}{M}} \and
  \rewrite{s}{M}[\rewrite{t}{N}/x] \equiv \rewrite{\ap{s}{t/x}}{\StI{\ap{q}{t/x}}{M[N/x]}}
\end{mathpar}

\subsubsection{Telescope Types}

\begin{mathpar}
  \inferrule*{~}{\Gamma \yields_{1} 1 \TYPE} \and
  \inferrule*{~}{\Gamma \yields_{()} () : 1} \\
  M \equiv () \\
  \inferrule*{ \Gamma \yields_p A \TYPE \\
               \Gamma,x:A \yields_q B \TYPE}
             { \Gamma \yields_{\sigmacl{x}{p}{q}} \telety{x}{A}{B} \TYPE}
  \\
  \inferrule*{ \Gamma \yields_\mu M : A \\
               \Gamma \yields_\nu N : B[M/x]
             }
             { \Gamma \yields_{(\mu,\nu)} (M,N) : \telety{x}{A}{B}}
  \and
  \inferrule*{ \Gamma \yields_{\mu} M : \telety{x}{A}{B}}
             { \Gamma \yields_{\fst \mu} \fst{M} : A} 
  \and
  \inferrule*{ \Gamma \yields_{\mu} M : \telety{x}{A}{B}}
             { \Gamma \yields_{\snd \mu} \snd{M} : B[\fst M/x]} 

    \fst{(M,N)} \equiv M \and
    \snd{(M,N)} \equiv N \and
    P \equiv (\fst P, \snd P)
\end{mathpar}


\subsubsection{Modalities}

\begin{mathpar}
  \inferrule*[Left = F-form]{
    %% \yields_\gamma \Gamma \CTX \and (\text{where } \yields \gamma \ctx)\\\\
    \Gamma \yields_p A \TYPE \and (\text{where } \gamma \yields p \type) \\\\
    \gamma, x:p \yields \mu : q 
  }
  {\Gamma \yields_q \F{x.\mu}{A} \TYPE \and (\text{where } \gamma \yields q \type) } \\
  
  \inferrule*[Left = F-intro]{
    \Gamma \yields_{\nu} M : A
    \and (\text{where } \gamma \yields {\nu} : p)
    %% \and \gamma \yields \nu : q 
    %% \and \gamma \yields \mu[\theta] : q 
    %% \and \gamma \yields (\nu \Rightarrow \mu[\theta]) : q
  }
  {\Gamma \yields_{\mu[\nu/x]} \FI{M} : \F{x.\mu}{A} \and (\text{where } \gamma \yields \mu[\nu/x] : q)} \\

  \inferrule*[Left = F-elim]{
    \Gamma, y : \F{x.\mu}{A} \yields_{r} C \TYPE \and (\text{where } \gamma, y : q \yields r \type) \\\\
    \Gamma \yields_{\nu} M : \F{x.\mu}{A} \and (\text{where } \gamma \yields \nu : q) \\\\
    \Gamma, x:A \yields_{\nu' [\mu / y]} N : C [\FI{x}/y]
    \and (\text{where } \gamma, x:p \yields \nu' [\mu / y] : r [\mu / y] )}
  {\Gamma \yields_{\nu'[\nu/y]} \FE{M}{x}{N} : C[M/y]  \and (\text{where }  \gamma \yields {\nu'[\nu/y]} : r[\nu/y])} \\
  \FE{\FI{M}}{x}{N} \equiv N[M/x] %\and
%  \text{(optionally:) }
%  \FE{M}{x}{N[\FI{x}/z]} \equiv N[M/z]
  \\ \\

  \inferrule*[Left = F-Elim]{
    \gamma,y:q \yields r \type \\\\
    \Gamma \yields_{\nu} M : \F{x.\mu}{A} \and (\text{where } \gamma \yields \nu : q) \\\\
    \Gamma, x:A \yields_{r [\mu / y]} C \TYPE
    \and (\text{where } \gamma, x:p \yields r [\mu / y] \type )}
  {\Gamma \yields_{r[\nu/y]} \FE{M}{x}{C} \TYPE \and (\text{where }  \gamma \yields {r[\nu/y]} \type)} \\
  \FE{\FI{M}}{x}{C} \equiv C[M/x] %\and
%  \text{(optionally:) }
%  \FE{M}{x}{C[\FI{x}/z]} \equiv C[M/z]
\\ \\
  \inferrule*[Left = U-form]{
    \Gamma \yields_p A \TYPE \and (\text{where } \gamma \yields p \type)\\\\
    \and \Gamma,x:A \yields_q B \TYPE \and (\text{where } \gamma,x:p \yields q \type)\\\\
    \and \gamma, x:p, c:r \yields \mu : q
  }{\Gamma \yields_r \U{c.\mu}{A}{B} \TYPE \and (\text{where } \gamma \yields r \type)} \\

  \inferrule*[Left = U-intro]{
    \Gamma,x:A \yields_{\mu[\nu/c]} M : B \and (\text{where } \gamma,x:p \yields {\mu[\nu/c]} : q)
  }
  {\Gamma \yields_{\nu} \UI {x}{M} : \U{c.\mu}{x:A}{B}
    \and (\text{where } \gamma \yields \nu : r)
  } \\
  
  \inferrule*[Left = U-elim]{
    \Gamma \yields_{\nu_1} N_1 : \U{c.\mu}{x:A}{B} \and (\text{where } \gamma \yields \nu_1 : r) \\\\
    \Gamma \yields_{\nu_2} N_2 : A \and (\text{where } \gamma \yields \nu_2 : p)
  }{
    \Gamma \yields_{\mu[\nu_2/x,\nu_1/c]} \UE{N_1}{N_2} : B[N_2/x] \and (\text{where } \gamma \yields \mu[\nu_2/x,\nu_1/c] : q)
  } \\

  \UE{(\UI{x}{M})}{N} \equiv M[N/x] \and 
  \UI{x}{\UE{N}{x}} \equiv N
\end{mathpar}

\subsubsection{Surprisingly Strict Modalities}

\begin{mathpar}
  \inferrule*[Left = s-form]{
    \Gamma \yields_p A \TYPE \and (\text{where } \gamma \yields p \type)\\\\
    \and \TypeTwo{\gamma}{s}{q}{p}
  }{\Gamma \yields_q \St{s}{A} \TYPE \and (\text{where } \gamma \yields q \type)} \\

  \inferrule*[Left = S-intro]{
    \Gamma \yields_{\mu} M : A
    \and (\text{where } \gamma \yields {\mu} : p)
  }
  {\Gamma \yields_{\TrPlus{s}{\mu}} \StI{s}{M} : \St{s}{A} \and (\text{where } \gamma \yields \TrPlus{s}{\mu} : q)} \\

  \inferrule*[Left = S-elim]{
    \Gamma, y : \St{s}{A} \yields_{r} C \TYPE \and (\text{where } \gamma, y : q \yields r \type) \and \\\\
    \Gamma \yields_{\nu} M : \St{s}{A} \and (\text{where } \gamma \yields \nu : q) \\\\
    \Gamma, x : A \yields_{\nu' [\TrPlus{s}{x} / y]} N : C [\StI{s}{x}/y]
    \and (\text{where } \gamma, x : p \yields \nu' [\TrPlus{s}{x} / y] : r[\TrPlus{s}{x} / y] )}
  {\Gamma \yields_{\nu'[\nu/y]} \StE{s}{M}{x}{N} : C[M/y]  \and (\text{where } \gamma \yields {\nu'[\nu/y]} : r[\nu/y])} \\
  \StE{s}{\StI{s}{M}}{x}{N} \equiv N[M/x] \and
  \StE{s}{M}{x}{N[\StI{s}{x}/z]} \equiv N[M/z]
  \\
  
  \inferrule*[Left = S-Elim]{
    \gamma,y:q \yields r \type \\\\
    \Gamma \yields_{\nu} M : \St{s}{A} \and (\text{where } \gamma \yields \nu : q) \\\\
    \Gamma, x:A \yields_{r [\TrPlus{s}{x} / y]} C \TYPE
    \and (\text{where } \gamma, x:p \yields r [\TrPlus{s}{x} / y] \type )}
  {\Gamma \yields_{r[\nu/y]} \StE{s}{M}{x}{C} \TYPE \and (\text{where }  \gamma \yields {r[\nu/y]} \type)} \\
  \StE{s}{\StI{s}{M}}{x}{C} \equiv C[M/x] \and
  \StE{s}{M}{x}{C[\StI{s}{x}/z]} \equiv C[M/z]
  \\ \\
\end{mathpar}

%\drlnote{Change the definition of $s$-types to $U$-types as primitive and derive $F$, so that having $\eta$ is less surprising.}

Term/type equalities:
\begin{align}
%\StI{s}{\FI{M}} &\equiv \FI{M} &\St{s}{\F{x.\mu}{A}} &\equiv \F{x.\TrPlus{s}{\mu}}{A} \\
%\FI{\StI{s}{M}} &\equiv \FI{M} &\F{x.\mu}{\St{s}{A}} &\equiv \F{x.\mu[\TrPlus{s}{x}/x]}{A} \\
%%\UStI{s}{\UI{x}{M}} &\equiv \UI{x}{M} &\St{s}{\U{c.\mu}{x:A}{B}} &\equiv \U{c.\mu[\TrCirc{s}{c}/c]}{x:A}{B} \\
%%\UI{x}{\UStI{s}{M}} &\equiv \UI{x}{M} &\U{c.\mu}{x:A}{\St{s}{B}} &\equiv \U{c.\TrCirc{s}{\mu}}{x:A}{B} \\
%\UI{x}{M} &\equiv \UI{x}{M[\StI{s}{x}/x]} &\U{c.\mu}{x:\St{s}{A}}{B} &\equiv \U{c.\mu[\TrPlus{s}{x}/x]}{x:A}{B[\StI{s}{x}/x]} \\
\label{eq:stype-pair}\StI{(s, t)}{(M, N)} &\equiv (\StI{s}{M}, \StI{t[\mu/x]}{N}) &\St{(\telety{x'}{s}{t})}{\telety{x'}{A'}{B'}} & \equiv \telety{x}{\St{s}{A'}}{\StE{s}{x}{x'}{\St{t}{B'}}} \\
\StI{s}{\StI{t}{M}} &\equiv \StI{s;t}{M} &\St{s}{\St{t}{A}} &\equiv \St{(s;t)}{A} \\
\StI{\id_p}{M} &\equiv M &\St{\id_p}{A} &\equiv A \\
\label{eq:stype-subst} \rewrite{\ap{\nu}{t/x}}{\StI{\ap{q}{t/x}}{N[M/x]}} &\equiv N[\rewrite{t}{M}/x]  &\St{(\ap{q}{t/x})}{B[M/x]} & \equiv B[\rewrite{t}{M}/x] 
%% other whiskering is a substitution rule:
%% \St{(s[\mu/x])}{B[M/x]} & \equiv (\St{s}{B})[M/x] 
\end{align}
%Where the last term equation is a special case of rewrite on substitutions. The inputs are typed $\Gamma,x:A \vdash_q B \TYPE $\ and $\Gamma \vdash_{\mu'} M : A$ and $\TermTwo{\gamma}{t}{\mu}{\mu'}$.

Due to the lack of the eta rule for $\mathsf{F}$-types, we also need to assert
\begin{mathpar}
(\FE{M}{x}{\rewrite{t[\mu/y]}{N}}) \equiv \rewrite{t[\nu/y]}{\FE{M}{x}{N}}
\end{mathpar}




\section{Examples}

TODO: move ../depdep-examples-short here


\section{Semantics}
\documentclass{amsart}
\usepackage{mathpartir,latexsym,amssymb,stmaryrd,mathtools}
\usepackage[all]{xy}
\title{Something involving bifibrations of cartesian 2-multicategories and generalized substructural type theories}
\author{Daniel R. Licata \and Michael Shulman}
\thanks{
This material is based on research sponsored by The United States Air
Force Research Laboratory under agreement number FA9550-15-1-0053. The
U.S. Government is authorized to reproduce and distribute reprints for
Governmental purposes notwithstanding any copyright notation thereon.
The views and conclusions contained herein are those of the authors and
should not be interpreted as necessarily representing the official
policies or endorsements, either expressed or implied, of the United
States Air Force Research Laboratory, the U.S. Government, or Carnegie
Mellon University.
}
\newtheorem{thm}{Theorem}[section]
\theoremstyle{definition}
\newtheorem{defn}[thm]{Definition}
\def\M{\mathcal{M}}
\def\D{\mathcal{D}}
\def\toiso{\xrightarrow{\sim}}
\let\To\Rightarrow
%\newcommand\compo[2]{\ensuremath{#1 \circ #2}}
\newcommand\compv[2]{\ensuremath{#1 \cdot #2}}
\newcommand\comph[2]{\ensuremath{#1 \mathbin{\circ_2} #2}}
\newcommand\wftp[2]{\ensuremath{#1 \,\,\, \mathsf{type}_{#2}}}
\newcommand\seq[3]{\ensuremath{#1 \, [ #2 ] \, \vdash \, #3}}
\begin{document}
\maketitle

\section{Bifibrations of cartesian 2-multicategories}
\label{sec:bifib-cart-2multi}

\begin{defn}
  A \textbf{(strict) cartesian 2-multicategory} consists of
  \begin{enumerate}
  \item A set $\M_0$ of \emph{objects}.
  \item For every object $B$ and every finite list of objects $(A_1,\dots,A_n)$, a category $\M(A_1,\dots,A_n;B)$.
    The objects of this category are \emph{1-morphisms} and its morphisms are \emph{2-morphisms}; we write composition of 2-morphisms as $\compv{e_1}{e_2}$.
  \item For each object $A$, an identity arrow $1_A\in\M(A;A)$.
  \item For any object $C$ and lists of objects $(B_1,\dots,B_m)$ and $(A_{i1},\dots,A_{in_i})$ for $1\le i\le m$, a composition functor
    \begin{align*}
      \M(B_1,\dots,B_m;C) \times \prod_{i=1}^m \M(A_{i1},\dots,A_{in_i};B_i) &\longrightarrow \M(A_{11},\dots,A_{mn_m};C)\\
      (g,(f_1,\dots,f_m)) &\mapsto g\circ (f_1,\dots,f_m)
    \end{align*}
    We write the action of this functor on 2-cells as $\comph{d}{(e_1,\dots,e_m)}$.
  \item For any $f\in\M(A_1,\dots,A_n;B)$ we have natural equalities (i.e.\ natural transformations whose components are identities)
    \begin{mathpar}
      1_B \circ (f) = f\and
      f\circ (1_{A_1},\dots,1_{A_n}) = f.
    \end{mathpar}
  \item For any $h,g_i,f_{ij}$ we have natural equalities
    \begin{multline*}
      (h\circ (g_1,\dots,g_m))\circ (f_{11},\dots,f_{mn_m}) =\\
      h \circ (g_1\circ (f_{11},\dots,f_{1n_1}), \dots, g_m \circ (f_{m1},\dots,f_{mn_m}))
    \end{multline*}
  \item For any function $\sigma : \{1,\dots,m\} \to \{1,\dots,n\}$ and objects $A_1,\dots,A_n,B$, a functor
    \begin{align*}
      \M(A_{\sigma 1},\dots,A_{\sigma m}; B) &\to \M(A_1,\dots,A_n;B)\\
      f &\mapsto f\sigma^*
    \end{align*}
  \item The functors $\sigma^*$ satisfy the following natural equalities:
    \begin{gather*}
      f \sigma^* \tau^* = f(\tau\sigma)^*\\
      f (1_n)^* = f\\
      g\circ (f_1 \sigma_1^* ,\dots, f_n \sigma_n^*) = (g \circ (f_1,\dots,f_n))(\sigma_1\sqcup \cdots \sqcup \sigma_n)^*\\
      g\sigma^* \circ (f_1,\dots,f_n) = (g\circ (f_{\sigma 1},\dots, f_{\sigma m}))(\sigma \wr (k_1,\dots,k_n))^*
    \end{gather*}
    In the last equation, $k_i$ is the arity of $f_i$, and $\sigma \wr (k_1,\dots,k_n)$ denotes the composite function
    \begin{equation*}
      \{1,\dots,\textstyle\sum_{i=1}^m k_{\sigma i} \}
      \toiso \bigsqcup_{i=1}^m \{1,\dots,k_{\sigma i}\}
      \xrightarrow{\widehat{\sigma}} \bigsqcup_{j=1}^n \{1,\dots,k_{j}\}
      \toiso \{1,\dots,\textstyle\sum_{j=1}^n k_j \}
    \end{equation*}
    where $\widehat{\sigma}$ acts as the identity from the $i^{\mathrm{th}}$ summand to the $(\sigma i)^{\mathrm{th}}$ summand.
  \end{enumerate}
\end{defn}

A cartesian 2-multicategory encodes the structure of the modes.
We will generally be interested in cartesian 2-multicategories given by \emph{presentations} consisting of some objects (``modes''), some generating 1-morphisms, some equalities and some 2-cells between composites of those (possibly including $\sigma$-actions), and some equalities between composites of the generating 2-cells.
For example:
\begin{itemize}
\item Suppose we have one mode $p$, generating morphisms $\mu:(p,p)\to p$ and $\eta:()\to p$, and equations $\mu\circ (1,\eta) = 1 = \mu\circ (\eta,1)$ and $\mu\circ (1,\mu) = \mu \circ (\mu,1)$.
  This is the free cartesian 2-multicategory containing a monoid, and the type theory it generates is noncommutative intuitionistic linear logic.
\item Suppose we add the additional axiom $\mu\sigma^* = \mu$, where $\sigma$ is the nonidentity automorphism of $\{1,2\}$.
  This is the free cartesian 2-multicategory containing a commutative monoid, and the type theory it generates is (commutative) intuitionistic linear logic.
\item Suppose we also add a generating 2-cell $t:1_p \To \eta\tau^*$, where $\tau:\emptyset \to \{1\}$ is the unique function, so that $\tau^* : \M(;p) \to \M(p;p)$, and also an axiom $\comph{t}{1_\eta} = 1_\eta$.
  This is the free cartesian 2-multicategory containing a commutative ``semicartesian'' monoid (whose unit is a ``terminal object''), and the type theory it generates is intuitionistic affine logic.
\item Suppose we also add a generating 2-cell $d:1_p \to \mu\delta^*$, where $\delta:\{1,2\} \to \{1\}$, is the unique function, so that $\delta^* : \M(p,p;p) \to \M(p;p)$, and also some axioms.
  Then we get the free cartesian 2-multicategory containing a cartesian object, and the type theory it generates is ordinary intuitionistic logic.
\item Starting over, suppose the same mode $p$ has both a commutative monoid structure and an unrelated cartesian monoid structure.
  This should generate Bunched Implication.
\item With two modes $p,q$, where $p$ is a cartesian monoid and $q$ a commutative monoid, and a morphism $p\to q$ respecting the monoid structures, we should get the adjoint decomposition of the linear exponential.
\item In a similar case but where both $p$ and $q$ are cartesian, we should get the adjoint decomposition of Pfenning--Davies $\Box$.
\item With one mode $p$ having a cartesian monoid structure, and an idempotent monad on it respecting the monoid structure, we should get a simply-typed multi-variable version of spatial type theory.
\end{itemize}

Now, as a matter of fact, type theory itself excels at presenting categorical structures by generators and relations, especially those of the cartesian multicategory sort!
Ordinary simple type theory with no type operations (but with generators and axioms) yields presentations of ordinary cartesian multicategories.
Thus, we only need to enhance this with some judgments for 2-cells in order to give convenient presentations of cartesian 2-multicategories.
[I'm not sure exactly what those judgments should look like, but you were using something like them on your slides, so maybe you have a better idea.]

In any case, if we wrap up the entire type theory into one, it will have judgments for modes, mode terms, mode 2-cells, types, and type terms/sequents (and perhaps equalities of those).
The first three judgments present a cartesian 2-multicategory; the latter two or three give the following structure over it.

\begin{defn}
  A functor of cartesian 2-multicategories $\pi:\D\to\M$ is a \textbf{local discrete fibration} if each induced functor of ordinary categories
  \[\D(A_1,\dots,A_n;B)\to\M(\pi A_1,\dots,\pi A_n;\pi B)\]
  is a discrete fibration.
\end{defn}

We write $\D_f(A_1,\dots,A_n;B)$ for the fiber of this functor over $f\in \M(\pi A_1,\dots,\pi A_n;\pi B)$; the assumption ensures that this fiber is a discrete set.

Local discrete fibrations over a fixed $\M$ are the categorical structure corresponding to the \emph{judgmental structure} of the type theory generated by $\M$, as follows:
\begin{itemize}
\item Each type of the type theory is assigned an object of $\M$ as its mode, written $\wftp{A}{p}$; these correspond to objects of $\D$ living over $p$ via $\pi$.
\item Each sequent
  \[ \seq{A_1,\dots,A_n}{\alpha}{B}\]
  is labeled by a morphism $\alpha:(p_1,\dots,p_n) \to q$ of $\M$, where $\wftp{A_i}{p_i}$ and $\wftp{B}{q}$; these correspond to the elements of $\D_\alpha(A_1,\dots,A_n;B)$.
\item Composition in $\D$ corresponds to a cut rule as usual for multicategories, and functoriality of $\pi$ means that the cut rule is annotated by composition of morphisms in $\M$;
  \begin{mathpar}
    \inferrule{\seq{A_1,\dots,A_n}{\alpha}{B} \\ \seq{\Gamma_i}{\beta_i}{A_i}}{\seq{\Gamma_1,\dots\Gamma_n}{\alpha\circ(\beta_1,\dots,\beta_n)}{B}}
  \end{mathpar}
  Similarly, identities in $\D$ correspond to the variable rule $\seq{A}{1_p}{A}$, where \wftp{A}{p}.
\item The local fibration structure of $\pi$ corresponds to the pull-back rule
  \begin{mathpar}
    \inferrule{\seq{A_1,\dots,A_n}{\beta}{B} \\ e:\alpha\To\beta}{\seq{A_1,\dots,A_n}{\alpha}{B}}
  \end{mathpar}
  (As in one-variable adjoint logic, to make this admissible, one has to build it into some of the rules, including the variable rule.
  Note that the variance of this is reversed from the one-variable adjoint logic paper.
  Categorically, the reason for this is that when representing a fully \emph{covariant} 2-functor $\M\to\mathit{Cat}$ by its Grothendieck construction, the latter is nevertheless an opfibration on 1-cells but a \emph{fibration} on 2-cells.)
\item Finally, the $\sigma^*$ actions on $\D$ correspond to the structural rules of exchange, contraction, and weakening ``with resource-usage labels''.
\end{itemize}

This last point deserves more explanation.
Suppose $p\in\M$ has a monoid structure $\mu:(p,p)\to p$, and we have $\wftp{A}{p}$ and $\wftp{B}{p}$ and a sequent $\seq{A,A}{\mu}{B}$.
Since $\M$ is a cartesian multicategory, we can act on $\mu$ by $\delta:\{1,2\} \to \{1\}$ to get $\mu\delta^*:p \to p$, which (without further assumptions on $p$) is unrelated to $1_p:p\to p$.

Similarly, we can act by $\delta$ on our sequent $\seq{A,A}{\mu}{B}$ to get $\seq{A}{\mu\delta^*}{B}$.
Since $\mu\delta^*$ is not the same as $1_p$, this is different from asserting $\seq{A}{1_p}{B}$; the mode morphism $\mu\delta^*$ is still ``recording the fact that $A$ is used twice''.
But if it should happen that $p$ has a \emph{cartesian} monoid structure, with a 2-cell $d:1_p \to \mu\delta^*$, then we can apply the pull-back rule to $\seq{A}{\mu\delta^*}{B}$ to get $\seq{A}{1_p}{B}$.
This is the ``real'' contraction rule that is only allowed when $p$ is cartesian.

In sum, type-theoretically we have three possible sequents:
\begin{mathpar}
  \seq{A,A}{\mu}{B}\and
  \seq{A}{\mu\delta^*}{B}\and
  \seq{A}{1_p}{B}
\end{mathpar}
which are represented in a local discrete fibration by morphisms $(A,A)\to B$ over $\mu$, morphisms $A\to B$ over $\mu\delta^*$, and morphisms $A\to B$ over $1_p$ respectively.
The passage from the first to the second, and from the second to the third, are both a sort of ``contraction''; but the first is always possible whereas the second depends on the mode being cartesian.

In most naturally-occurring categorical models, however, we have only two kinds of morphism:
\begin{mathpar}
  A\otimes A \to B\and
  A\to B
\end{mathpar}
where we can get from the first to the second if $\otimes$ is cartesian.
This suggests that we should regard the intermediate sequent $\seq{A}{\mu\delta^*}{B}$ as a different type-theoretic representation of a morphism $A\otimes A\to B$ that ``remembers the double occurrence of $A$'', rather than anything to do with a morphism $A\to B$.
Semantically, the observation is that many naturally-occurring local discrete fibrations (though not the free ones built from syntax) satisfy the following property:

\begin{defn}
  A local discrete fibration $\pi:\D\to\M$ has \textbf{honest resources} (what a terrible name, there must be a better name for this) if
  for each function $\sigma : \{1,\dots,m\} \to \{1,\dots,n\}$, objects $A_1,\dots,A_n,B \in \D$, and $f\in\M(\pi A_{\sigma 1},\dots,\pi A_{\sigma m}; \pi B)$, the functor
  \[\D_f(A_{\sigma 1},\dots,A_{\sigma m}; B) \to \D_{f\sigma^*}(A_1,\dots,A_n;B)\]
  is an isomorphism.
  (Since $\pi$ is a local discrete fibration, this functor is actually a mere function between discrete sets.)
\end{defn}

Finally, the connectives that ``internalize'' the judgmental structure are represented categorically by cartesian and opcartesian arrows.

\begin{defn}
  If $\pi:\D\to\M$ is a local discrete fibration, then a morphism $\xi\in\D(A_1,\dots,A_n;B)$ is \textbf{opcartesian} if all diagrams of the following form are pullbacks of categories:
  \[ \xymatrix{
    \D(\vec C,B,\vec D;E) \ar[r]^-{-\circ \xi} \ar[d]_\pi &
    \D(\vec C,\vec A,\vec D;E) \ar[d]^\pi \\
    \M(\pi\vec C,\pi B, \pi\vec D; \pi E) \ar[r]_-{-\circ \pi\xi} &
    \M(\pi\vec C,\pi\vec A,\pi\vec D;\pi E)
  }\]
  Dually, a morphism $\xi\in\D(\vec C,B,\vec D;E)$ is \textbf{cartesian at $B$} if all diagrams of the following form are pullbacks of categories:
  \[ \xymatrix{
    \D(\vec A;B) \ar[r]^-{\xi\circ -} \ar[d]_\pi &
    \D(\vec C,\vec A,\vec D;E) \ar[d]^\pi \\
    \M(\pi\vec A;\pi B) \ar[r]_-{\pi\xi\circ -} &
    \D(\pi\vec C,\pi\vec A,\pi\vec D;\pi E)
  }\]
  Given $\mu:(p_1,\dots,p_n) \to q$ in $\M$, we say that $\pi$ \textbf{has $\mu$-products} if for any $A_i$ with $\pi A_i = p_i$, there exists a $B$ with $\pi B = q$ and an opcartesian morphism in $\D_\mu(A_1,\dots,A_n;B)$.
  Dually, we say $\pi$ \textbf{has $\mu$-homs} if for any $i$, any $B$ with $\pi B = q$, and any $A_j$ with $\pi A_j = p_j$ for $j\neq i$, there exists an $A_i$ with $\pi A_i = p_i$ and a cartesian morphism in $\D_\mu(A_1,\dots,A_n;B)$.
\end{defn}

References: Hermida, H\"{o}rmann.

For example:
\begin{itemize}
\item The intuitionistic linear logic $A \otimes B$ is a $\mu$-product for the multiplication $\mu:(p,p)\to p$ of a (commutative) monoid mode.
  Its adjoint $A\multimap B$ is a $\mu$-hom for the same $\mu$.
  Similarly, the linear unit $\mathbf{1}$ is an $\eta$-product for the unit $\eta :()\to p$.
\item The cartesian product $A\times B$ is a $\mu$-product for the multiplication of a cartesian monoid mode, and its right adjoint $A\to B$ is similarly a $\mu$-hom.
\item If the linear logic $!$ is written as $F U$, then $F$ and $U$ are the $\alpha$-product and $\alpha$-hom for the (unary!)\ mode morphism $p\to q$ from the cartesian to the linear mode.
  The situation with Pfenning--Davies $\Box$ is similar.
\item The spatial type theory $\flat$ is an $\alpha$-product where $\alpha$ is an idempotent monad, and similarly $\sharp$ is an $\alpha$-hom for the same $\alpha$.
\end{itemize}


\section{Generalized multicategories}
\label{sec:genmulti}

Let $\M$ be a cartesian 2-multicategory, with object set $\M_0$.
Then it induces a 2-monad $T_\M$ on $\mathit{Cat}^{\M_0}$ whose algebras are a sort of ``strong pseudofunctor'' $\M\to \mathit{Cat}$, where $\mathit{Cat}$ has a cartesian 2-multicategory structure arising from its finite products, that preserve composition strictly but the $\sigma^*$ actions only up to coherent isomorphism.
The latter condition is necessary to get from a (strict) commutative monoid object in $\M$ to a non-strict symmetric monoidal category in the $T_\M$-algebras.
Similarly, we can check that a cartesian monoid in $\M$ yields a cartesian monoidal category, and so on.

A bit more explicitly, $T_\M$ is defined by starting with
\begin{equation*}
  \left(\coprod_{p_1,\dots,p_n} \left(\M(p_1,\dots,p_n;q) \times \prod_{i=1}^n X_{p_i} \right)\right)
\end{equation*}
and adding natural isomorphisms
\[ (f\sigma^*,(x_1,\dots,x_n))\cong (f,(x_{\sigma 1},\dots,x_{\sigma m})) \]
satisfying some natural coherence axioms.

We extend $T_\M$ to act on profunctors in the usual way.
I then claim that \emph{virtual $T_\M$-algebras} (the sort of ``generalized multicategory'' corresponding to $T_\M$) are equivalent to \emph{local discrete fibrations with honest resources} over $\M$.
By Theorem~8.6 of Cruttwell--Shulman, a virtual $T_\M$-algebras can be described by a discrete family of \emph{sets} $D\in \mathit{Set}^{\M_0}$ together with a family of profunctors $\D_q:(T_\M D)_q^{\mathrm{op}} \times D_q \to \mathit{Set}$, with composition and unit structure.

The sets $D_p$ will be the objects of $\D$ with $\pi A = p$.
If we ignore the extra isomorphisms in $T_\M D$ for a moment, we see that the profunctor $\D$ is determined by functors
\[ \M(p_1,\dots,p_n;q)^{\mathrm{op}} \times D_{p_1}\times\cdots\times D_{p_n} \times D_q \to \mathit{Set} \]
for all $p_1,\dots, p_n$.
In other words, for each $p_1,\dots, p_n,q$ we have a family of hom-sets $\D_\alpha(A_1,\dots,A_n;B)$ indexed by objects $A_i$ and $B$ with $\pi A_i = p_i$ and $\pi B = q$ and also by a mode morphism $\alpha:(p_1,\dots,p_n)\to q$, and these sets depend contravariantly functorially on $\alpha$, just as required by the local discrete fibration condition.
The composition and identities of a generalized multicategory give these hom-sets multicategorical structure.
Finally, the extra isomorphisms in the definition of $T_\M D$ yield the bijective $\sigma^*$ actions on $\D$.

It is now straightforward to check that a virtual $T_\M$-algebra is representable, i.e.\ is a $T_\M$-algebra, exactly when its corresponding local discrete fibration is an opfibration (has $\mu$-products for all $\mu$).
Thus, the present type theory (assuming it works) provides some fairly general evidence for the correspondence between type theories and generalized multicategories.
Of course, not every 2-monad on a power of $\mathit{Cat}$ can be obtained in this way from a cartesian 2-multicategory, and not even every finitary one can (consider the 2-monad for categories with equalizers).
However, many important ones can, and this seems to be particularly true of those that extend ``canonically'' to profunctors.


\end{document}

\section{Conclusion}

\section{Conclusion}

We have described a sequent calculus that can express a variety of
substructural and modal logics through a suitable choice of mode theory.
The framework itself enjoys identity and cut admissibility for all mode
theories, and these properties are inherited by the logics that are
represented in it.  The logic corresponds semantically to a fibration
between 2-dimensional cartesian multicategories, and so gives both a
syntactic and semantic account of the idea that substructural and modal
logics are constraints on structural proofs.  


\ifthenelse{\boolean{short}}{
In future work, we plan to continue a preliminary investigation of
equational adequacy that is discussed in the extended version,
investigating whether that the logical adequacy proofs are an
isomorphism on $\beta\eta$-classes of derivations.  It is generally easy
to show that object-language equations are true in the framework.  We
conjecture that the converse is true for the mode theories we have
described here, which says that the ``extra'' types and judgements
available in the framework do not add to the equations between terms in
the image of encoded sequents.  
Proving this is challenging because the
equational theory of Section~\ref{sec:equational} does not itself
obviously have the subformula property---we can in principle prove
equations by introducing and then eliminating cuts.  We have sketched a
proof of equational adequacy for the simplest case (ordered logic
products), assuming a lemma that the equational theory from
Section~\ref{sec:equational} can be characterized as some ``permuting
conversions'' on cut-free derivations (i.e. that one can first
$\beta$-reduce and then rearrange the normal form).  We proved this
lemma for one-variable sequents~\citep{ls16adjoint}, and have checked
one direction here.

}{
In future work, we plan to continue the preliminary investigation of
equational adequacy that is discussed
Section~\ref{sec:adequacy-equational}.  
}
Additionally, we plan to apply our framework to investigate more
extensions of homotopy type theory like the spatial type theory
considered here; in current work with Eric Finster, we are designing a
variant of cohesion that is an internal language for spectra.  We also
plan to consider encodings of programming-focused type theories, such as
specialized effect calculi.  Finally, our adequacy proofs require
reasoning about the 1- and 2-cells in the mode theory, which we have
currently done entirely na\"ively; we would like to investigate using
techniques from higher-dimensional rewriting to simplify and possibly
automate these proofs.



%% In future work, we will
%% investigate the issue of equality of derivations, leading to an
%% initiality theorem with respect to the semantic structure, and to a
%% better understanding of how the equality of derivations in the framework
%% corresponds to object-language equality of proofs/programs/morphisms.



%%
%% Bibliography
%%

%% Please use bibtex, 

{
  \bibliography{../../drl-common/cs}
}

\end{document}

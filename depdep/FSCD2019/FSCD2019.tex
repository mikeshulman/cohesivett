
\documentclass[UKenglish,letterpaper,cleveref,autoref]{lipics-v2019}
%This is a template for producing LIPIcs articles. 
%See lipics-manual.pdf for further information.
%for A4 paper format use option "a4paper", for US-letter use option "letterpaper"
%for british hyphenation rules use option "UKenglish", for american hyphenation rules use option "USenglish"
%for section-numbered lemmas etc., use "numberwithinsect"
%for enabling cleveref support, use "cleveref"
%for enabling cleveref support, use "autoref"

  \usepackage{xcolor}
  \definecolor{darkgreen}{rgb}{0,0.45,0} 
  %% \usepackage[pagebackref,colorlinks,citecolor=darkgreen,linkcolor=darkgreen]{hyperref}
  %% \usepackage{pdflscape}

\usepackage{amssymb,amsthm,bbm}
\usepackage{amsmath}
%% \usepackage[mathscr]{euscript}
\usepackage{dsfont}
 \usepackage{fontawesome}
 \usepackage{tikz-cd}
\usepackage{mathpartir}
\usepackage{enumitem}
\usepackage[status=draft,inline,nomargin]{fixme}
\FXRegisterAuthor{ms}{anms}{\color{blue}MS}
\FXRegisterAuthor{mvr}{anmvr}{\color{olive}MVR}
\FXRegisterAuthor{drl}{andrl}{\color{purple}DRL}
\usepackage{stmaryrd}
\usepackage{mathtools}

%% \newtheorem{theorem}{Theorem}
%% \newtheorem{proposition}{Proposition}
%% \newtheorem{lemma}{Lemma}
%% \newtheorem{corollary}{Corollary}
\newtheorem{problem}{Problem}
\newenvironment{constr}{\begin{proof}[Construction]}{\end{proof}}

\theoremstyle{definition}
%% \newtheorem{definition}{Definition}
%% \newtheorem{remark}{Remark}
%% \newtheorem{example}{Example}

\let\oldemptyset\emptyset%
\let\emptyset\varnothing

\newcommand\dsd[1]{\ensuremath{\mathsf{#1}}}

\newcommand{\yields}{\vdash}
\newcommand{\Yields}{\tcell}
\newcommand{\tcell}{\Rightarrow}
\newcommand{\cbar}{\, | \,}
\newcommand{\judge}{\mathcal{J}}

\newcommand{\Id}[3]{\mathsf{Id}_{{#1}}(#2,#3)}
\newcommand{\CTX}{\,\,\mathsf{Ctx}}
\newcommand{\ctx}{\,\,\mathsf{mctx}}
\newcommand{\TYPE}{\,\,\mathsf{Type}}
\newcommand{\type}{\,\,\mathsf{mode}}
\newcommand{\TELE}{\,\,\mathsf{Tele}}
\newcommand{\tele}{\,\,\mathsf{mtele}}

\newcommand{\app}[2]{\ensuremath{#1 \: #2}}
\newcommand{\telety}[3]{\ensuremath{(#1{:}#2,#3)}}
\newcommand{\mt}[0]{\ensuremath{()}}
\newcommand{\sigmacl}[3]{\ensuremath{\textnormal{$\Sigma$}\,#1{:}#2.\,#3}}
\newcommand{\fst}[1]{\app{\dsd{fst}}{#1}}
\newcommand{\snd}[1]{\app{\dsd{snd}}{#1}}
\newcommand\extend[2]{\ensuremath{(#1,\id_{#2})}}

\newcommand\fan[1]{\ensuremath{\mathsf{fan}_{#1}}}

\newcommand{\id}{\mathsf{id}}
\DeclareMathOperator{\ob}{ob}

\newcommand{\rewrite}[2]{\overleftarrow{#1}(#2)}
\newcommand\F[2]{\ensuremath{\mathsf{F}_{#1}(#2)}}
\newcommand\U[3]{\ensuremath{\mathsf{U}_{#1}(#2 \mid #3)}}
\newcommand\UE[2]{\ensuremath{#1(#2)}}
\newcommand\UI[2]{\ensuremath{\lambda #1.#2}}
\newcommand\St[2]{\ensuremath{{#1}^*(#2)}}
\newcommand\StI[2]{\ensuremath{\mathsf{st}_{#1}(#2)}}
\newcommand\UStI[2]{\ensuremath{\mathsf{ust}_{#1}(#2)}}
\newcommand\UnSt[2]{\ensuremath{\mathsf{unst}_{#1}(#2)}}
%\newcommand\StE[2]{\ensuremath{\mathsf{unst}(#1,#2)}}
\newcommand\StE[4]{\ensuremath{\mathsf{let} \, \StI{#1}{#3} \, = \, {#2} \, \mathsf{in} \, #4}}
\newcommand\FE[3]{\ensuremath{\mathsf{let} \, \mathsf{F}(#2) \, = \, {#1} \, \mathsf{in} \, #3}}
% With subscript:
\newcommand\FEs[4]{\ensuremath{\mathsf{let} \, \mathsf{F}_{#1}(#3) \, = \, {#2} \, \mathsf{in} \, #4}} 
\newcommand\FI[1]{\ensuremath{\mathsf{F}{(#1)}}}
\newcommand\FIs[2]{\ensuremath{\mathsf{F}_{#1}{(#2)}}}
\newcommand\TypeTwo[4]{\ensuremath{#1 \vdash #2 :  #3 \tcell #4}}
\newcommand\TeleTwo[4]{\ensuremath{#1 \vdash #2 : #3 \tcell #4}}
\newcommand\TermTwo[4]{\ensuremath{#1 \vdash #2 : #3 \tcell #4}}
\newcommand\TermTwoT[5]{\ensuremath{#1 \vdash {#2} : #3 \tcell_{#5} #4}}
%% \newcommand\TermTwoDisp[5]{\ensuremath{#1 \mid #3 \tcell_{\mathsf{disp}} #2 :_{#5} #4}}
%\newcommand\SubTwo[4]{\ensuremath{#1 \mid #3 \tcell #2 : #4}}
\newcommand\TrPlus[2]{\ensuremath{{#1}^+(#2)}}
\newcommand\TrCirc[2]{\ensuremath{{#1}^\circ(#2)}}

\newcommand\El[2]{\mathcal{T}_{#1}(#2)}
\newcommand\ApEl[2]{\mathcal{T}_{#1}\langle#2\rangle}
\newcommand\bdot[0]{\mathbin{.}}
\newcommand\bang[0]{\mathord{!}}

\newcommand\ap[2]{\ensuremath{#1 \langle #2 \rangle }}
\newcommand\ApPlus[2]{\ensuremath{{#1}^+ \langle #2 \rangle }}
\newcommand\ApCirc[2]{\ensuremath{{#1}^\circ \langle #2 \rangle }}

% MLTT
\newcommand{\qyields}{\Vdash} 
\newcommand{\upstairs}[1]{\overline{#1}}
\newcommand\proj[1]{\ensuremath{\mathsf{proj}_{#1}}}
\newcommand\qvar[1]{\ensuremath{\mathsf{var}_{#1}}}
\newcommand\into[1]{\ensuremath{\mathsf{into}_{#1}}}

\newcommand\One{\ensuremath{\mathds{1}}}
\newcommand\var[1]{\ensuremath{\mathtt{var}_{#1}}}
\newcommand\ApOne[1]{\ensuremath{\One_{\langle {#1} \rangle }}}

\newcommand\mtt[1]{\mathtt{#1}}
\newcommand\contract[1]{\ensuremath{\mathtt{contract}_{#1}}}
\newcommand\fibpair[1]{\ensuremath{\mathtt{fibpair}_{#1}}}
\newcommand\pair[1]{\ensuremath{\mathtt{pair}_{#1}}}
\newcommand\tsplit[1]{\ensuremath{\mathtt{split}_{#1}}}
\newcommand\pinv[1]{\ensuremath{\mathtt{pinv}_{#1}}}

% Adjoint type theory
\newcommand\fone[1]{\ensuremath{\mathtt{fone}_{#1}}}
\newcommand\foneinv[1]{\ensuremath{\fone{#1}^{-1}}}
\newcommand\fdist[1]{\ensuremath{\mathtt{fdist}_{#1}}}
\newcommand\fdistinv[1]{\ensuremath{\fdist{#1}^{-1}}}

\newcommand{\lock}{\text{\faUnlock}}
\newcommand{\Rtype}[1]{\mathsf{R}{#1}}
\newcommand{\RI}[1]{\mathsf{shut}({#1})}
\newcommand{\RE}[1]{\mathsf{open}({#1})}

\newcommand{\Ltype}[1]{\mathsf{L}{#1}}
\newcommand{\LI}[1]{\mathsf{left}_{#1}}
\newcommand{\LE}[1]{\mathsf{letleft}({#1})}

% Spatial type theory
\newcommand\fcomult[1]{\ensuremath{\mathtt{comult}_{#1}}}
\newcommand\fcounit[1]{\ensuremath{\mathtt{counit}_{#1}}}
\newcommand{\counit}[1]{\mathsf{counit}_{#1}}
\newcommand{\comult}[1]{\mathsf{comult}_{#1}}
\newcommand{\Flattype}[1]{\flat{#1}}
\newcommand{\FlatI}[1]{{#1}^\flat}
\newcommand{\FlatE}[1]{\mathsf{letflat}({#1})}
\newcommand{\Sharptype}[1]{\sharp{#1}}
\newcommand{\SharpI}[1]{{#1}^\sharp}
\newcommand{\SharpE}[1]{{#1}_\sharp}
\newcommand\qcrispvar[1]{\ensuremath{\textsf{crisp-var}_{#1}}}

% Macros for semantics notation
\newcommand\mm[1]{\llbracket #1 \rrbracket}
\newcommand\op{^{\mathrm{op}}}
\newcommand\co{^{\mathrm{co}}}
\newcommand\coop{^{\mathrm{coop}}}
\newcommand\Cat{\mathrm{Cat}}
\newcommand\CAT{\mathrm{CAT}}
\newcommand\M{\mathcal{M}}
\newcommand\Mhat{\widehat{\mathcal{M}}}
\newcommand\Mty{{\mathrm{Ty}_{\M}}}
\newcommand\Mtm{{\mathrm{Tm}_{\M}}}
\newcommand\Mtyhat{{\widehat{\mathrm{Ty}}_{\M}}}
\newcommand\Mtmhat{{\widehat{\mathrm{Tm}}_{\M}}}
\newcommand\Ups{\Upsilon}
\newcommand\Upshat{{\widehat{\Upsilon}}}
\newcommand\C{\mathcal{C}}
\newcommand\Chat{{\widehat{\mathcal{C}}}}
\newcommand\Cty{\mathrm{Ty}_{\C}}
\newcommand\Ctm{\mathrm{Tm}_{\C}}
\newcommand\Ctyhat{{\widehat{\mathrm{Ty}}}_{\C}}
\newcommand\Ctmhat{{\widehat{\mathrm{Tm}}}_{\C}}
\newcommand\vp{\varpi}
\newcommand\vpst{\vp^*}
\newcommand\vpsh{\vp_!}
\newcommand\vptil{\widetilde{\vp}}
\newcommand\vpty{{\vp}_{\mathrm{Ty}}}
\newcommand\vptm{{\vp}_{\mathrm{Tm}}}
\newcommand\name[1]{\ulcorner #1\urcorner}
\newcommand{\Util}{\widetilde{U}}
\newcommand\ev{\mathrm{ev}}
\DeclareSymbolFont{bbold}{U}{bbold}{m}{n}
\DeclareSymbolFontAlphabet{\mathbbb}{bbold}
\newcommand\one{\mathbbb{1}}

%\graphicspath{{./graphics/}}%helpful if your graphic files are in another directory

\bibliographystyle{plainurl}% the mandatory bibstyle

\title{A Fibrational Framework for Modal Dependent Type Theories}
%\titlerunning{}%optional, please use if title is longer than one line

\author{Daniel R. Licata}{Wesleyan University}{dlicata@wesleyan.edu}{https://orcid.org/0000-0003-0697-7405}{}

\author{Mitchell Riley}{Wesleyan University}{mvriley@wesleyan.edu}{?}
       {}%TODO mandatory, please use full name; only 1 author per \author macro; first two parameters are mandatory, other parameters can be empty. Please provide at least the name of the affiliation and the country. The full address is optional

\author{Michael Shulman}{University of San Diego}{shulman@sandiego.edu}{?}
       {}%TODO mandatory, please use full name; only 1 author per \author macro; first two parameters are mandatory, other parameters can be empty. Please provide at least the name of the affiliation and the country. The full address is optional

\authorrunning{D.\,R. Licata and M. Riley and M. Shulman}%TODO mandatory. First: Use abbreviated first/middle names. Second (only in severe cases): Use first author plus 'et al.'

\Copyright{Daniel R. Licata and Mitchell Riley and Michael Shulman}%TODO mandatory, please use full first names. LIPIcs license is "CC-BY";  http://creativecommons.org/licenses/by/3.0/

\ccsdesc[500]{Theory of computation~Type theory}%TODO mandatory: Please choose ACM 2012 classifications from https://dl.acm.org/ccs/ccs_flat.cfm 

\keywords{homotopy type theory, modal logic
  % bicubical sets, dependent type theory,  synthetic \infty-categories,
  % model categories, Reedy categories, internal language, univalent
  % foundations
}%TODO mandatory; please add comma-separated list of keywords

%% \category{}%optional, e.g. invited paper

%% \relatedversion{}%optional, e.g. full version hosted on arXiv, HAL, or other respository/website
%% %\relatedversion{A full version of the paper is available at \url{...}.}

%% \supplement{}%optional, e.g. related research data, source code, ... hosted on a repository like zenodo, figshare, GitHub, ...

%% %\funding{(Optional) general funding statement \dots}%optional, to capture a funding statement, which applies to all authors. Please enter author specific funding statements as fifth argument of the \author macro.

%% \acknowledgements{I want to thank \dots}%optional

%% %\nolinenumbers %uncomment to disable line numbering

%% %\hideLIPIcs  %uncomment to remove references to LIPIcs series (logo, DOI, ...), e.g. when preparing a pre-final version to be uploaded to arXiv or another public repository

%% %Editor-only macros:: begin (do not touch as author)%%%%%%%%%%%%%%%%%%%%%%%%%%%%%%%%%%
%% \EventEditors{John Q. Open and Joan R. Access}
%% \EventNoEds{2}
%% \EventLongTitle{42nd Conference on Very Important Topics (CVIT 2016)}
%% \EventShortTitle{CVIT 2016}
%% \EventAcronym{CVIT}
%% \EventYear{2016}
%% \EventDate{December 24--27, 2016}
%% \EventLocation{Little Whinging, United Kingdom}
%% \EventLogo{}
%% \SeriesVolume{42}
%% \ArticleNo{23}
%%%%%%%%%%%%%%%%%%%%%%%%%%%%%%%%%%%%%%%%%%%%%%%%%%%%%%

\begin{document}

\maketitle

%TODO mandatory: add short abstract of the document
\begin{abstract}
Recently, several modal extensions of homotopy type theory have been
investigated, with the goal of extending the synthetic style of
formalizing mathematics to additional situations.  For example,
real-cohesive homotopy type theory can describe types with both a
groupoid structure and a separate topological structure.  These modal
dependent type theories add new type operators to the syntax, which
typically are given universal properties relative to new judgement
forms.  To facilitate the design of such type theories, we introduce a
general framework for modal dependent type theories, building on our
previous work for simple type theories.  The framework consists first of
a base directed dependent type theory, which serves as a language for
specifying a signature of desired modalities, which we call a mode
theory.  This mode theory is the parameter to a second type theory,
which gives general rules for working with the modalities it describes.
The mode theory language is flexible enough to describe a variety of
modalities, including adjunctions, monads, comonads, idempotent
(co)monads, and so on; as examples, we give mode theories for ordinary
non-modal dependent type theory with $\Pi$ and $\Sigma$ types, for a
dependent adjoint pair of modalities, and for the spatial type theory
used in real-cohesion.  One advantage of our framework is that we can
give it a categorical semantics for all mode theories at once, which
saves some of the effort involved in translating each type theory
individually, and we describe a category-with-families-like semantics.
While the framework does not automatically produce ``optimized''
inference rules for a particular modal discipline (where structural
rules are as admissible as possible), it does provide a convenient
syntactic setting for investigating such issues, including a general
equational theory governing the placement of structural rules in types
and in terms.

\end{abstract}

\section{Introduction}

\section{Introduction}

In ordinary intuitionistic logic or $\lambda$-calculus, assumptions or
variables can go unused (weakening), be used in any order (exchange), be
used more than once (contraction), and be used in any position in a
term.  \emph{Substructural} logics, such as linear logic, ordered logic,
relevant logic, and affine logic, drop some of these structural
properties of weakening, exchange, and contraction, while \emph{modal
  logics} place restrictions on where variables may be used---e.g. a
formula $\Bx{} C$ can only be proved using assumptions of $\Bx{} A$,
while an assumption of $\Dia{}{A}$ can only be used when the conclusion
is $\Dia{}{C}$.  Substructural and modal logics have had many
applications to both functional and logic programming (modeling concepts
such state, staged computation, distribution, and concurrency, to name
just a few).

Substructural and modal logics can also be used as \emph{internal
  languages} of categories, where one uses an appropriate logical
language to do constructions ``inside'' a particular mathematical
setting, which often leads to shorter statements than working
``externally''.  For example, to define a function when working
``externally'' in domains, one must first define the underlying
set-theoretic function, and then prove that it is continuous.  But when
using untyped $\lambda$-calculus as an internal language of domains,
there is no need to prove that a function described by a $\lambda$-term
is continuous, because all terms are shown to denote continous functions
once and for all.  Substructural logics extend this idea to various
forms of monoidal categories, while modal logics describe monads and
comonads.  Recently,
\citet{schreibershulman12cohesive,shulman15realcohesion} proposed using
modal operators to add a notion of \emph{cohesion} to homotopy type
theory/univalent foundations~\citep{,voevodsky06homotopy,uf13hott-book}.
Without going into the precise details of this application, the idea is
to add a triple $\sh{} \la \Flat{} \la \Sharp{}$ of type operators,
where for example $\Sharp A$ is a monad (like a modal possibility
$\Diamond$ or $\bigcirc$), $\Flat A$ is a comonad (like a modal
necessity $\Box$), and there is an adjunction structure between them
(e.g. $\flat{A} \to B$ is the same as $A \to \Sharp{B}$).  This raised
the question of how to best add modalities with these properties to type
theory.

Because other similar applications would have different monads and
comonads with different properties, we would like general tools for
going from a semantic situation of interest to a ``nice'' type
theory/logic for it, e.g. one with cut and identity admissibility and
the subformula property. In previous work~\citep{ls16adjoint}, we
considered the special case of a single-assumption logic, building most
directly on the adjoint logics of
\citet{benton94mixed,bentonwadler96adjoint,reed09adjoint}.  Here we
extend this previous work to the multi-assumption case.  The resulting
framework is quite general and covers many existing intuitionistic
substructural and modal connectives: cartesian, linear, affine,
relevant, ordered, bunched~\citep{ohearnpym99bunched} and
non-associative products and implications; $n$-linear
variables~\citep{reed08namessubstructural}; the comonadic $\Box$ and
linear exponential $!$ and
subexponentials~\citep{nigammiller09subexponentials,danos+93subexponentials};
monadic $\Diamond$ and $\bigcirc$ modalities; and adjoint logic $F$ and
$G$~\citep{benton94mixed,bentonwadler96adjoint,reed09adjoint}, including
the single-assumption 2-categorical version from our previous
work~\citep{ls16adjoint}.  It also supports variations on these, such as
non-monoidal comonads and non-strong monads.  We show that a single,
simple structural~\citep{pfenning94cut} proof of cut (and identity)
admissibility applies to all of these logics, as well as any new logics
that can be described in the framework.
%% While it is not too surprising
%% that this is possible, given that cut proofs for these logics all follow
%% a similar template, it is nonetheless satisfying to codify this pattern
%% as an abstraction.

At a high level, the framework expresses the idea that all of the above
logics are a restriction on how variables can be used in ordinary
structural/cartesian proofs.  We express these restrictions using a
first layer of the logic, which is a simple type theory for what we will
call \emph{modes} and \emph{context descriptors}.  The modes are just a
collection of base types, which we write as $p,q,r$, while a context
descriptor is a term built from variables and constants.  The next layer
is the main logic.  Each proposition of the logic is assigned a mode,
and the basic sequent is \seq{x_1 : A_1, \ldots, x_n : A_n}{\alpha}{C},
where if $A_i$ has mode $p_i$, and $C$ has mode $q$, then $\oftp{x_1 :
  p_1,\ldots, x_n : p_n}{\alpha}{q}$.  
%% In a sequent
%% \seq{\Gamma}{\alpha}{A}, the idea is that $\Gamma$ binds some variables
%% for use both in $\alpha$ and in the derivation.  
$\Gamma$ itself behaves like an ordinary structural/cartesian context,
while the substructural and modal aspects are enforced by the
\emph{term} $\alpha$, which describes how the resources from $\Gamma$
are allowed to be used.  For example, in linear logic/ordered logic/BI,
the context is usually taken to be a multiset/list/tree (respectively).
We represent the multiset or list or tree using a pair of an ordinary
structural context $\Gamma$, together with a term $\alpha$ that
describes the multiset or list or tree structure, labeled with variables
from the ordinary context at the leaves.  We pronounce a sequent
\seq{\Gamma}{\alpha}{A} as ``$\Gamma$ proves $A$ {along,over} $\alpha$''
or ``$\Gamma$ structured according to $\alpha$ proves $A$''.

For example, suppose we have one mode $\dsd{n}$, together with a context
descriptor constant
\[
x : \dsd{n}, y:\dsd{n} \vdash x \odot y : \dsd{n}
\]
Then an example sequent
\[
\seq{x:A, y:B, z:C, w:D}{(y \odot x) \odot z}{E}
\]
should be read as saying that we must prove $E$ using the resources $y$
and $x$ and $z$ (but not $w$) according to the particular tree structure
${(y \odot x) \odot z}$.  If we say nothing else, the framework will
treat $\odot$ as describing a non-associative, linear, ordered context:
if we have a product-like type $A \odot B$ internalizing this context
operation,\footnote{We overload binary operations to refer both to
  context descriptors and propositional connectives, because it is clear
  from whether it is applied to variables $x,y,z$ or propositions
  $A,B,C$ which we mean.}  then we will \emph{not} be able to prove
associativity ($(A \odot B) \odot C \dashv\vdash A \odot (B \odot C)$)
or contraction ($A \vdash A \odot A$) or exchange ($A \odot B \vdash B
\odot A$) etc.

To get from this basic structure to linear or affine or relevant or
cartesian logic, we need to add some structural properties to the
context descriptor term $\alpha$.  We analyze structural properties as
\emph{equations}, or more generally \emph{directed transformations}, on
such terms.  For example, to specify linear logic, we will add a unit
element $1 : \dsd{n}$ together with equations making $(\odot,1)$ into a
commutative monoid:
\[
\begin{array}{c}
x \odot (y \odot z) = (x \odot y) \odot z\\
x \odot 1 = x = 1 \odot x\\
x \odot y = y \odot x
\end{array}
\]
so that the context descriptors ignore associativity and order.  To get
BI, we add an additional commutative monoid $(\times,\top)$ (with
weakening and contraction, as discussed below), so that a BI context
tree $(x:A,y:B);(z:C,w:D)$ can be represented by the ordinary context
$x:A,y:B,z:C,w:D$ with the term $(x \odot y) \times (z \odot w)$
describing the tree.  Because the context descriptors are themselves
ordinary structural/cartesian terms, the same variable can occur more
than once or not at all.  A descriptor such as $x \odot x$ captures the
idea that we can use the \emph{same} variable $x$ twice, expressing
$n$-linear types~\citep{reed08namessubstructural}.  Thus, we can express
contraction for a particular context descriptor $\odot$ as an equation
$x = x \odot x$ (one use of $x$ is the same as two, or $\odot$ is an
idempotent binary operation).  However, weakening cannot be represented
as an equation between context descriptors: an equation $x = 1$ would
trivialize the logic to ordinary intuitionistic logic.  Instead, to
express weakening, we use a directed transformation $x \spr 1$, which is
oriented to allow throwing away an allowed use of $x$, but not creating
an allowed use from nothing.  We refer to these as \emph{structural
  transformations}, to evoke their use in representing the structural
properties of object logics that are embedded in our framework.
Structural transformations are also used to describe relationships
between adjunctions~\citep{ls16adjoint}.

In summary, to specify a particular substructural or modal logic, one
gives constants generating context descriptors $\alpha$, with equations
$\alpha = \beta$ and transformations $\alpha \spr \beta$ expressing
structural properties.  The main sequent $\seq{\Gamma}{\alpha}{A}$
respects the specified structural properties in the sense that when
$\alpha = \beta$, we regard $\seq{\Gamma}{\alpha}{A}$ and
$\seq{\Gamma}{\beta}{A}$ as the same sequent, while when $\alpha \spr
\beta$, there will be an operation that takes a derivation of
\seq{\Gamma}{\beta}{A} to a derivation of \seq{\Gamma}{\alpha}{A}.

A guiding principle of the framework is a meta-level notion of
\emph{structurality over structurality}.  For example, we always have
\emph{weakening over weakening}: if \seq{\Gamma}{\alpha}{A} then
\seq{\Gamma,y:B}{\alpha}{A}, where $\alpha$ itself is weakened with $y$.
This does not prevent encodings of e.g. linear logic: it is permissible
to weaken a derivation of \seq{\Gamma}{x_1 \odot \ldots \odot x_n}{A}
(``use $x_1$ through $x_n$'') to a derivation of \seq{\Gamma,y:B}{x_1
  \odot \ldots \odot x_n}{A} because the (weakened) context descriptor
still disallows the use of $y$.  Similarly, we always have exchange over
exchange and contraction over contraction.  The identity and and cut
principles are analogous:
\[
\infer{\seq{\Gamma,x:A}{x}{A}}{}
\qquad
\infer{\seq{\Gamma}{\subst{\beta}{\alpha}{x}}{B}}
    {\seq{\Gamma,x:A}{\beta}{B} &
     \seq{\Gamma}{\alpha}{A}}
\]
The identity-over-identity principle says that we should be able to
prove $A$ using exactly an assumption $x:A$.  The cut principle says
that the context descriptor for the result of the cut is the
substitution of the context descriptor used to prove $A$ into the one
used to prove $B$.  For example, together with weakening-over-weakening,
this captures the usual cut principle of linear logic, which says that
cutting $\Gamma,x:A \vdash B$ and $\Delta \vdash A$ yields
$\Gamma,\Delta \vdash B$.  If $\Gamma$ binds $x_1,\ldots,x_n$ and
$\Delta$ binds $y_1,\ldots,y_n$, then we will represent the two
derivations to be cut together by sequents with
\[
\begin{array}{l}
\beta = x_1 \odot \ldots \odot x_n \odot x\\
\alpha = y_1 \odot \ldots \odot y_n
\end{array}
\]
so
\[
\beta[\alpha/x] = x_1 \odot \ldots \odot x_n \odot y_1 \odot \ldots \odot y_n
\]
correctly deletes $x$ and replaces it with the variables from $\Delta$.
Moreover, in more subtle situations such as BI, the substitution will
insert the resources used to prove the cut formula in the correct place
in the tree.

The framework has two main logical connectives.  The first,
\F{\alpha}{\Delta}, generalizes the \dsd{F} of adjoint
logic~\citep{bentonwadler96adjoint,reed09adjoint} and the tensor
($\otimes$) of linear logic.  The second, \U{x.\alpha}{\Delta}{A},
generalizes the $\dsd{G}/\dsd{U}$ of adjoint logic and the implication
$A \lolli B$ of linear logic.  Here $\Delta$ is a context of assumptions
$x_i:A_i$, and trivializing the context descriptors (i.e. adding an
equation $\alpha = \beta$ for all $\alpha$ and $\beta$) degenerates
$\F{\alpha}{\Delta}$ into the ordinary intuititionistic product $A_1
\times \ldots \times A_n$, while \U{x.\alpha}{\Delta}{A} becomes $A_1
\to \ldots \to A_n \to A$.  Though we do not give a full
polarized/focused proof theory in this paper, we do prove that \dsd{F}
is left-invertible and \dsd{U} is right-invertible, and we conjecture
that focusing works with the polarization that one would expect based on
these degeneracies ($\F{\alpha}{\Delta^{\mathord{+}}}^{\mathord{+}}$ and
$\U{x.\alpha}{\Delta^{\mathord{+}}}{A^{\mathord{-}}}^{\mathord{-}}$).
In linear logic terms, our \dsd{F} and \dsd{U} cover both the
multiplicatives and exponentials; additives can be added separately by
essentially the usual rules.

Being a very general theory, our framework treats the structural
properties in a general but na\"ive way, allowing an arbitrary
structural transformation to be applied at the non-invertible rules for
$\dsd{F}$ and $\dsd{U}$ and at the leaves of a derivation.  For specific
embedded logics, there will often be a more refined discipline that
suffices---e.g. for cartesian logic, always contract all assumptions at
in all premises, rather than choosing which assumptions to contract.  We
view our framework as a tool for bridging the gap between an intended
semantic situation such as the cohesion example mentioned above (``a
comonad and a monad which are themselves adjoint'') and a proof theory:
the framework gives \emph{some} proof theory for the semantics, and the
placement of structural rules can then be optimized purely in syntax.
To support this mode of use, we give an equational theory on sequent
derivations that identifies different placements of the same structural
rules.  This equational theory is used to prove correctness of such
optimizations not just at the level of provability, but also identity of
derivations---which matters for our intended applications to internal
languages.

Semantically, the logic corresponds to a functor between
\emph{2-dimensional cartesian multicategories} which is a fibration in
various senses.  Multicategories are a generalization of categories
which allow more than one object in the domain, and cartesianness means
that the multiple domain objects are treated structurally.  The
2-dimensionality supplies a notion of morphism between (multi)morphisms,
which correspond to the structural transformations.  The functor
specifies the mode of each proposition and the context descriptor of a
sequent.  The fibration conditions (similar to \citep{hermida,hormann})
specify respect for the structural transformations and the presence of
\dsd{F} and \dsd{U} types.

The remainder of this paper is organized as follows.  FIXME

FIXME: comparison with display logic, L/CLF, what else?  



\section{Mode Theory}

The first (``bottom'') level of our framework, called the \emph{mode
  theory}, is used to give a signature of type constructors.  For a
particular mode theory, the second (``top'') level, described in the
next section, allows for constructions using those type constructors.
In previous work~\cite{ls15adjoint,lsr17fibrational}, the mode theory
was a 2-category, or cartesian 2-multicategory.  Objects/types of the
mode theory represent ``modes of truth,'' or categories of types.
Morphisms/terms of the mode theory generate type constructors,
e.g. modalities.  The further 2-cell data of the mode theory corresponds
to structural rules, such as weakening, exchange, contraction for a
product, or a (co)unit or (co)multiplication of a modality.  Just as a
2-category is a directed simple type theory, here we use a directed
dependent type theory as the mode theory.  We write mode theory entities
with lowercase Greek letters.

The mode theory consists of five judgements:
\begin{itemize}
\item $\gamma \ctx$ are mode theory contexts
\item $\gamma \yields p \type$ are mode theory types
\item $\gamma \yields \mu : p$ are mode theory terms
\item $\TypeTwo{\gamma}{s}{p}{q}$ are \emph{mode type morphisms}
\item $\TermTwoT{\gamma}{s}{\mu}{\nu}{p}$ are \emph{mode 2-cells}
\end{itemize}
The first three judgements have the familiar structure of a dependent
type theory, while the fourth and fifth add an additional notion of
morphism between types and morphism between terms.  The mode type
morphisms can be thought of as a special class of terms, as explained
below.

In this section, we describe the rules for the mode theory that are
general to all examples. In Section~\ref{sec:examples}, we show examples
of specific mode theories.  

\subsection{Mode Contexts}

Mode contexts have the usual empty and context extension structure:

\begin{mathpar}
  \inferrule*{ }
             {\cdot \ctx}
             
  \inferrule*
    {\gamma \ctx \\
     \gamma \yields p \type}
    {\gamma,x:p \ctx}
\end{mathpar}  

\subsection{Mode Terms}

The mode terms not specific to any particular types are

\begin{mathpar}
\inferrule*{ }
             {\gamma,x : p, \gamma' \yields x : p}
\quad
\inferrule*
    {\gamma \yields \mu : q \\
     \TypeTwo{\gamma}{s}{p}{q}
    }
    {\gamma \yields \TrPlus{s}{\mu} : p}
\quad
\TrPlus{\id}{\mu} \equiv \mu \quad
\TrPlus{s'}{\TrPlus{s}{\mu}} \equiv \TrPlus{(s';s)}{\mu} 
\end{mathpar}

In addition to the usual variable rule, we have the term
$\TrPlus{s}{\mu}$, which says that mode type morphisms act
contravariantly on mode terms.  The equations say this action is
functorial ($\id$ and $s;s'$ are identity and composition for mode type
morphisms; see below).

\subsection{Mode type morphisms}

\begin{mathpar}
    \inferrule*{ }
          {\TypeTwo{\gamma}{\id_p}{p}{p}}
    \qquad
    \inferrule*{{\TypeTwo{\gamma}{s_1}{p_1}{p_2}} 
                {\TypeTwo{\gamma}{s_2}{p_2}{p_3}}
          }
          {\TypeTwo{\gamma}{s_1;s_2}{p_1}{p_3}}
\qquad
\inferrule*{{\gamma,x:p} \vdash {q} \type \\
            \TermTwoT{\gamma}{t}{\mu}{\mu'}{p}\\
           } 
           {\TypeTwo{\gamma}{\ap {q} {t/x}}{q[\mu/x]}{q[\mu'/x]}}

\end{mathpar}


\id;s \equiv s \equiv s;\id \and
(s;s');s'' \equiv s;(s';s'') \\ 
\ap q {\id_{\mu}/x} \equiv \id_{q[\mu/x]} \and
\ap q {(s;t)/x} \equiv \ap q {s/x}; \ap q {t/x} \\ 
\ap q {s/\_} \equiv \id_q \\ 
\ap {(q[\mu/x])} {s/y} \equiv \ap q {\ap \mu {s/y}/x} \quad (\text{where } \gamma,y:p' \vdash \mu : p \text{ and } \gamma,x:p \vdash q \type)\\
s[\nu/x];\ap{q'}{t/x} \equiv \ap{q}{t/x};s[\nu'/x] \quad 
(\text{where } \TypeTwo{\gamma,x:p}{s}{q}{q'} \text{ and } \TermTwoT{\gamma}{t}{\nu}{\nu'}{p})

%% subst: \id_\mu[\nu/x] = \id_{\mu[\nu/x]}
%% subst: s[x/x] = s
%% subst: (s;t)[\mu/x] = s[\mu/x];t[\mu/x]
%% subst: s[\mu[\nu/x]/x] = s[\mu/x][\nu/x]
%% subst: ap q (s [\mu/x]) = (ap q s)[\mu/x] and generalization
\end{mathpar}

We write $\ap q {t/x}$ for whiskering (\dsd{ap} in book HoTT).

\subsection{2-cells between terms.}
First, we have
  identity/composition/whiskering and associated equations (whiskering
  on the other side is given by substitution):
\begin{mathpar}
    \inferrule*{ }
          {\TermTwoT{\gamma}{\id_\mu}{\mu}{\mu}{p}}
    \qquad
    \inferrule*{{\TermTwoT{\gamma}{s_1}{\mu_1}{\mu_2}{p}} \\
                {\TermTwoT{\gamma}{s_2}{\mu_2}{\mu_3}{p}}
          }
   {\TermTwoT{\gamma}{s_1;s_2}{\mu_1}{\mu_3}{p}}

\inferrule*{{\gamma,x:p} \yields {\nu} : {q} \\
            \TermTwoT{\gamma}{s}{\mu}{\mu'}{p}\\
           } 
           {\TermTwoT{\gamma}{\ap \nu {s/x}}{\nu[\mu/x]}{\TrPlus{\ap{q}{s/x}}{\nu[\mu'/x]}}{q[\mu/x]}}

\\           
\id;s \equiv s \equiv s;\id \and
(s;s');s'' \equiv s;(s';s'') \\ 
\ap \nu {\id_{\mu}/x} \equiv \id_{\nu[\mu/x]} \and
\ap \nu {(s;t)/x} \equiv \ap \nu {s/x} ; (\ap {(\TrPlus{\ap{q}{s/x}}{y})} {\ap \nu {t/x}/y}) \\ 
\ap x {s/x} \equiv s  \and
\ap \nu {s/\_} \equiv \id_\nu \and
\ap {(\nu[\mu/x])} {s/y} \equiv \ap \nu {\ap \mu {s/y}/x} \quad
(\text{where } \gamma,y:p' \vdash \mu : p \text{ and } \gamma,x:p \vdash \nu : q)\\
t[\mu/x];\ap{\nu'}{s/x} \equiv \ap{\nu}{s/x};\ApPlus{\ap{q}{s/x}}{t[\mu'/x]} \quad
 (\text{where } \TermTwoT{\gamma,x:p}{t}{\nu}{\nu'}{q} \text{ and } \TermTwoT{\gamma}{s}{\mu}{\mu'}{p}) \\
\ap{\TrPlus{s}{\mu}}{t/x} \equiv \ApPlus{(s[\nu/x])}{\ap{\mu}{t/x}}\quad 
(\text{where } \TypeTwo{\gamma,x:p}{s}{q}{q'} \text{ and } \TermTwoT{\gamma}{t}{\nu}{\nu'}{p})
\end{mathpar}

\subsection{Product modes}

We assume $1/\Sigma$ modes:

\begin{mathpar}
  \inferrule*{ } { \gamma \yields 1 \type } \and
  
  \inferrule*{ \gamma \yields p \type \\ 
               \gamma,x:p \yields q \type }
             {\gamma \yields \sigmacl{x}{p}{q} \type} \\
             
  \inferrule*{ }
             {\gamma \yields \mt : 1}
  \and 
  \mu \equiv \mt
\\
\inferrule*{
  \gamma \yields \mu : p \and
  \gamma \yields \nu : q[\mu/x]
    }
   {\gamma \yields (\mu,\nu) : \sigmacl{x}{p}{q}}
\and
\inferrule*
    {\gamma \yields \mu : \sigmacl{x}{p}{q}}
    {\gamma \yields \fst \mu : p}
\and
\inferrule*
    {\gamma \yields \mu : \sigmacl{x}{p}{q}}
    {\gamma \yields \snd \mu : q[\fst \mu / x]}
    \\
    \fst{(\mu,\nu)} \equiv \mu \and
    \snd{(\mu,\nu)} \equiv \nu \and
    p \equiv (\fst p, \snd p)
\end{mathpar}

Equations for ``transport'' in $\Sigma$:
\begin{mathpar}
\TrPlus{(\sigmacl{x}{s}{t})}{\mu} \equiv (\TrPlus{s}{\fst \mu},\TrPlus{(t[\fst \mu/x])}{\snd \mu})
\end{mathpar}

Mode type morphisms: We need congruence for $\Sigma$ to be a rule (because we don't have ap on a type variable/universes):
\begin{mathpar}
  \inferrule*
  {\TypeTwo{\gamma}{s}{p}{p'} \\
    \TypeTwo{\gamma,x':p'}{t}{q[\TrPlus{s}{x'}/x]}{q'}}
  {\TypeTwo{\gamma}{\sigmacl{x'}{s}{t}}{\sigmacl{x}{p}{q}}{\sigmacl{x'}{p'}{q'}}} \\

  \sigmacl{x'}{\id_p}{\id_q} \equiv \id_{\sigmacl{x'}{p}{q}} \and
  (\sigmacl{x'}{s}{t});(\sigmacl{x''}{s'}{t'}) \equiv \sigmacl{x''}{(s;s')}{(t[\TrPlus{s'}{x''}/x'];t')} \\

  \ap{(\sigmacl{x'}{p}{q})}{s/(y:r)} \equiv
  \sigmacl{x'}{\ap{p}{s/y}}{\ap{({q[\fst z/x,\snd z/y]})}{\extend{s}{x'}/(z:(\sigmacl{y}{r}{p}))}}
\end{mathpar}

Finally, we have the 2-cells for $1/\Sigma$-terms:
\begin{mathpar}
s \equiv \id_{()} \text{ for } \yields_1 s : () \tcell ()
\\

\inferrule*
    {\TermTwoT{\gamma}{s}{\mu}{\mu'}{p} \and
      \gamma \vdash \nu' : q[\mu'/x]
    }
      {\TermTwoT{\gamma}{\extend{s}{\nu'}}{(\mu,\TrPlus{\ap{q}{s/x}}{\nu'})}{(\mu',\nu')}{\sigmacl{x}{p}{q}}}\\
\ap {\fst(z)} {\extend{s}{\nu'}/z} \equiv s \and
\ap {\snd(z)} {\extend{s}{\nu'}/z} \equiv \id_{\TrPlus{\ap{q}{s/x}}{\nu'}}  \\
s \equiv \ap{(\fst{\mu},y)}{\ap{(\snd z)}{s/z}/y};\extend{\ap{(\fst{z})}{s/z}}{\snd{\mu'}} \quad (\text{where } \TermTwoT{\gamma}{s}{\mu}{\mu'}{\sigmacl{x}{p}{q}})
\\      
{\extend{\id_\mu}{\nu'}} \equiv \id_{(\mu,\nu')} \and
{\extend{(s;s')}{\nu''}} \equiv  \extend{s}{\TrPlus{\ap{q}{s'/x}}{\nu''}};\extend{s'}{\nu''}   \\
\extend{s}{\nu'} ; (\ap{(\mu',y)}{t/y}) \equiv
(\ap{(\mu,\TrPlus{(\ap{q}{s})}{y})}{t/y}); \extend{s}{\nu''} \qquad (\text{where }\TermTwoT{\gamma}{t}{\nu'}{\nu''}{q[\mu'/x]})
\end{mathpar}

\subsection{Admissible Rules}

  All judgements have a substitution principle
\begin{mathpar}
  \inferrule*{\gamma,x:p,\gamma' \yields J \\
              \gamma \yields \mu : p
              }
             {\gamma,\gamma'[\mu/x] \yields J[\mu/x]} \\

J[\mu/x][\nu/y] \equiv J[\nu/y][\mu[\nu/y]/x]
\end{mathpar}

We sometimes write \ap{\mu}{s} for \ap{\mu(x)}{s/x}, eliding the
variable name when it is clear how to view $\mu$ as a term with a
distinguished variable; e.g. $\ApPlus{s}{t}$ for
$\ap{\TrPlus{s}{x}}{t/x}$.

\subsection{Mode Theory Signatures}

We allow extensions of the above judgements with generating
constants/axiomn for mode types, mode terms, mode type morphisms, mode
2-cells, and equations between all four of these classes (i.e.,
all judgements but mode contexts).  

\subsubsection{Lemmas}

Horizontal composition:
\begin{mathpar}
  \inferrule*[Left=Derivable]
      {\TermTwoT{\gamma}{s}{\mu}{\mu'}{p} \\
    \TermTwoT{\gamma, x : p}{t}{\nu}{\nu'}{q}}
             {\TermTwoT{\gamma}{\ap{t}{s/x} :\equiv t[\mu/x];\ap{\nu'}{s/x}}{\nu[\mu/x]}{\TrPlus{\ap{q}{s/x}}{\nu'[\mu'/x]}}{q[\mu/x]}}
\\ 
\ap{\id_\nu}{s/x} \equiv \ap{\nu}{s/x} \and \ap{t}{\id_{\mu}/x} \equiv t[\mu/x]
\end{mathpar}

Pairing and projection 2-cells are definable:
\begin{mathpar}
  \inferrule*[Left=Derivable]
      {\TermTwoT{\gamma}{s}{\mu}{\mu'}{p} \\
    \TermTwoT{\gamma}{t}{\nu}{\TrPlus{\ap{q}{s}}{\nu'}}{q[\mu/x]}}
             {\TermTwoT{\gamma}{(s,t) :\equiv \ap{(\mu,y)}{t/y};\extend{s}{\nu'}}{(\mu,\nu)}{(\mu',\nu')}{\sigmacl{x}{p}{q}}}

   \inferrule*[Left=Deriv]
              { {\TermTwoT{\gamma}{s}{\mu}{\mu'}{\sigmacl{x}{p}{q}}} }
              { {\TermTwoT{\gamma}{\ap{\fst(y)}{s/y}}{\fst{\mu}}{\fst{\mu'}}{p}} }
   \and
   \inferrule*[Left=Deriv]
              { {\TermTwoT{\gamma}{s}{\mu}{\mu'}{\sigmacl{x}{p}{q}}} }
              { {\TermTwoT{\gamma}{\ap{\snd(y)}{s/y}}{\snd{\mu}}{\TrPlus{\ap{(q(\fst y/x))}{s/y}}{\snd{\mu'}}}{q[\fst{\mu}/x]}} }
\end{mathpar}


\begin{lemma}
For mode term morphisms
\begin{align*}
\TermTwoT{\gamma &}{s}{\mu}{\mu'}{p} \\
\TermTwoT{\gamma &}{t}{\nu}{\TrPlus{\ap{q}{s}}{\nu'}}{q[\mu/x]} \\
\TermTwoT{\gamma &}{s'}{\mu'}{\mu''}{p} \\
\TermTwoT{\gamma &}{t'}{\nu'}{\TrPlus{\ap{q}{s'}}{\nu''}}{q[\mu'/x]}
\end{align*}
we have
\begin{align*}
(s, t);(s', t') \equiv ((s;s'), (t;\ApPlus{\ap{q}{s}}{t'}))
\end{align*}
\end{lemma}
\begin{proof}
Follows by
\begin{align*}
(s, t);(s', t') 
&\equiv \ap{(\mu,y)}{t/y};\extend{s}{\nu'};\ap{(\mu',y)}{t'/y};\extend{s'}{\nu''} \\
&\equiv \ap{(\mu,y)}{t/y};\ap{(\mu, \TrPlus{\ap{q}{s}}{y})}{t'/y};\extend{s}{\TrPlus{\ap{q}{s'}}{\nu''}};\extend{s'}{\nu''} \\
&\equiv \ap{(\mu,y)}{t/y};\ap{(\mu, y)}{\ApPlus{\ap{q}{s}}{t'}/y};\extend{s}{\TrPlus{\ap{q}{s'}}{\nu''}};\extend{s'}{\nu''} \\
&\equiv \ap{(\mu,y)}{t;\ApPlus{\ap{q}{s}}{t'}/y};\extend{s;s'}{\nu''} \\
&\equiv ((s;s'), (t;\ApPlus{\ap{q}{s}}{t'}))
\end{align*}
\end{proof}






\section{Framework}\label{sec:dua-ret}
\subsection{Contexts}

\begin{mathpar}
  \inferrule*[Left = ctx-form]{ }
  {\yields_{\cdot} \cdot \CTX  } \and 

  \inferrule*[Left = ctx-form]{
    \yields_\gamma \Gamma \CTX \and (\text{where } \yields \gamma \ctx) \\\\
    \Gamma \yields_p A \TYPE \and (\text{where }  \gamma \yields p \type)}
  {\yields_{\gamma, x : p} \Gamma, x : A \CTX \and (\text{where } \yields \gamma,x:p \ctx)  } \\
\end{mathpar}

\subsection{Types and Terms}

\subsubsection{Structural Rules}

\begin{mathpar}
  \inferrule*[Left = var]{
    % \yields \Gamma, x : A, \Gamma' \CTX_{\gamma, x : p, \gamma'}
  }
  {\Gamma, x : A, \Gamma' \yields_x x : A \and (\text{where } \gamma,x:p,\gamma' \yields x : p)} \and

 \inferrule*[Left = rewrite]{
   \Gamma \yields_\mu M : A 
   \and \TermTwoT{\gamma}{s}{\nu}{\mu}{p}
  }
  {\Gamma \yields_\nu \rewrite{s}{M} : A} \\ \\
  
  \rewrite{\id_{\mu}}{M} \equiv M \and
  \rewrite{(s;t)}{M} \equiv \rewrite{s}{\rewrite{t}{M}} \and
  \rewrite{s}{M}[\rewrite{t}{N}/x] \equiv \rewrite{\ap{s}{t/x}}{\StI{\ap{q}{t/x}}{M[N/x]}}
\end{mathpar}

\subsubsection{Telescope Types}

\begin{mathpar}
  \inferrule*{~}{\Gamma \yields_{1} 1 \TYPE} \and
  \inferrule*{~}{\Gamma \yields_{()} () : 1} \\
  M \equiv () \\
  \inferrule*{ \Gamma \yields_p A \TYPE \\
               \Gamma,x:A \yields_q B \TYPE}
             { \Gamma \yields_{\sigmacl{x}{p}{q}} \telety{x}{A}{B} \TYPE}
  \\
  \inferrule*{ \Gamma \yields_\mu M : A \\
               \Gamma \yields_\nu N : B[M/x]
             }
             { \Gamma \yields_{(\mu,\nu)} (M,N) : \telety{x}{A}{B}}
  \and
  \inferrule*{ \Gamma \yields_{\mu} M : \telety{x}{A}{B}}
             { \Gamma \yields_{\fst \mu} \fst{M} : A} 
  \and
  \inferrule*{ \Gamma \yields_{\mu} M : \telety{x}{A}{B}}
             { \Gamma \yields_{\snd \mu} \snd{M} : B[\fst M/x]} 

    \fst{(M,N)} \equiv M \and
    \snd{(M,N)} \equiv N \and
    P \equiv (\fst P, \snd P)
\end{mathpar}


\subsubsection{Modalities}

\begin{mathpar}
  \inferrule*[Left = F-form]{
    %% \yields_\gamma \Gamma \CTX \and (\text{where } \yields \gamma \ctx)\\\\
    \Gamma \yields_p A \TYPE \and (\text{where } \gamma \yields p \type) \\\\
    \gamma, x:p \yields \mu : q 
  }
  {\Gamma \yields_q \F{x.\mu}{A} \TYPE \and (\text{where } \gamma \yields q \type) } \\
  
  \inferrule*[Left = F-intro]{
    \Gamma \yields_{\nu} M : A
    \and (\text{where } \gamma \yields {\nu} : p)
    %% \and \gamma \yields \nu : q 
    %% \and \gamma \yields \mu[\theta] : q 
    %% \and \gamma \yields (\nu \Rightarrow \mu[\theta]) : q
  }
  {\Gamma \yields_{\mu[\nu/x]} \FI{M} : \F{x.\mu}{A} \and (\text{where } \gamma \yields \mu[\nu/x] : q)} \\

  \inferrule*[Left = F-elim]{
    \Gamma, y : \F{x.\mu}{A} \yields_{r} C \TYPE \and (\text{where } \gamma, y : q \yields r \type) \\\\
    \Gamma \yields_{\nu} M : \F{x.\mu}{A} \and (\text{where } \gamma \yields \nu : q) \\\\
    \Gamma, x:A \yields_{\nu' [\mu / y]} N : C [\FI{x}/y]
    \and (\text{where } \gamma, x:p \yields \nu' [\mu / y] : r [\mu / y] )}
  {\Gamma \yields_{\nu'[\nu/y]} \FE{M}{x}{N} : C[M/y]  \and (\text{where }  \gamma \yields {\nu'[\nu/y]} : r[\nu/y])} \\
  \FE{\FI{M}}{x}{N} \equiv N[M/x] %\and
%  \text{(optionally:) }
%  \FE{M}{x}{N[\FI{x}/z]} \equiv N[M/z]
  \\ \\

  \inferrule*[Left = F-Elim]{
    \gamma,y:q \yields r \type \\\\
    \Gamma \yields_{\nu} M : \F{x.\mu}{A} \and (\text{where } \gamma \yields \nu : q) \\\\
    \Gamma, x:A \yields_{r [\mu / y]} C \TYPE
    \and (\text{where } \gamma, x:p \yields r [\mu / y] \type )}
  {\Gamma \yields_{r[\nu/y]} \FE{M}{x}{C} \TYPE \and (\text{where }  \gamma \yields {r[\nu/y]} \type)} \\
  \FE{\FI{M}}{x}{C} \equiv C[M/x] %\and
%  \text{(optionally:) }
%  \FE{M}{x}{C[\FI{x}/z]} \equiv C[M/z]
\\ \\
  \inferrule*[Left = U-form]{
    \Gamma \yields_p A \TYPE \and (\text{where } \gamma \yields p \type)\\\\
    \and \Gamma,x:A \yields_q B \TYPE \and (\text{where } \gamma,x:p \yields q \type)\\\\
    \and \gamma, x:p, c:r \yields \mu : q
  }{\Gamma \yields_r \U{c.\mu}{A}{B} \TYPE \and (\text{where } \gamma \yields r \type)} \\

  \inferrule*[Left = U-intro]{
    \Gamma,x:A \yields_{\mu[\nu/c]} M : B \and (\text{where } \gamma,x:p \yields {\mu[\nu/c]} : q)
  }
  {\Gamma \yields_{\nu} \UI {x}{M} : \U{c.\mu}{x:A}{B}
    \and (\text{where } \gamma \yields \nu : r)
  } \\
  
  \inferrule*[Left = U-elim]{
    \Gamma \yields_{\nu_1} N_1 : \U{c.\mu}{x:A}{B} \and (\text{where } \gamma \yields \nu_1 : r) \\\\
    \Gamma \yields_{\nu_2} N_2 : A \and (\text{where } \gamma \yields \nu_2 : p)
  }{
    \Gamma \yields_{\mu[\nu_2/x,\nu_1/c]} \UE{N_1}{N_2} : B[N_2/x] \and (\text{where } \gamma \yields \mu[\nu_2/x,\nu_1/c] : q)
  } \\

  \UE{(\UI{x}{M})}{N} \equiv M[N/x] \and 
  \UI{x}{\UE{N}{x}} \equiv N
\end{mathpar}

\subsubsection{Surprisingly Strict Modalities}

\begin{mathpar}
  \inferrule*[Left = s-form]{
    \Gamma \yields_p A \TYPE \and (\text{where } \gamma \yields p \type)\\\\
    \and \TypeTwo{\gamma}{s}{q}{p}
  }{\Gamma \yields_q \St{s}{A} \TYPE \and (\text{where } \gamma \yields q \type)} \\

  \inferrule*[Left = S-intro]{
    \Gamma \yields_{\mu} M : A
    \and (\text{where } \gamma \yields {\mu} : p)
  }
  {\Gamma \yields_{\TrPlus{s}{\mu}} \StI{s}{M} : \St{s}{A} \and (\text{where } \gamma \yields \TrPlus{s}{\mu} : q)} \\

  \inferrule*[Left = S-elim]{
    \Gamma, y : \St{s}{A} \yields_{r} C \TYPE \and (\text{where } \gamma, y : q \yields r \type) \and \\\\
    \Gamma \yields_{\nu} M : \St{s}{A} \and (\text{where } \gamma \yields \nu : q) \\\\
    \Gamma, x : A \yields_{\nu' [\TrPlus{s}{x} / y]} N : C [\StI{s}{x}/y]
    \and (\text{where } \gamma, x : p \yields \nu' [\TrPlus{s}{x} / y] : r[\TrPlus{s}{x} / y] )}
  {\Gamma \yields_{\nu'[\nu/y]} \StE{s}{M}{x}{N} : C[M/y]  \and (\text{where } \gamma \yields {\nu'[\nu/y]} : r[\nu/y])} \\
  \StE{s}{\StI{s}{M}}{x}{N} \equiv N[M/x] \and
  \StE{s}{M}{x}{N[\StI{s}{x}/z]} \equiv N[M/z]
  \\
  
  \inferrule*[Left = S-Elim]{
    \gamma,y:q \yields r \type \\\\
    \Gamma \yields_{\nu} M : \St{s}{A} \and (\text{where } \gamma \yields \nu : q) \\\\
    \Gamma, x:A \yields_{r [\TrPlus{s}{x} / y]} C \TYPE
    \and (\text{where } \gamma, x:p \yields r [\TrPlus{s}{x} / y] \type )}
  {\Gamma \yields_{r[\nu/y]} \StE{s}{M}{x}{C} \TYPE \and (\text{where }  \gamma \yields {r[\nu/y]} \type)} \\
  \StE{s}{\StI{s}{M}}{x}{C} \equiv C[M/x] \and
  \StE{s}{M}{x}{C[\StI{s}{x}/z]} \equiv C[M/z]
  \\ \\
\end{mathpar}

%\drlnote{Change the definition of $s$-types to $U$-types as primitive and derive $F$, so that having $\eta$ is less surprising.}

Term/type equalities:
\begin{align}
%\StI{s}{\FI{M}} &\equiv \FI{M} &\St{s}{\F{x.\mu}{A}} &\equiv \F{x.\TrPlus{s}{\mu}}{A} \\
%\FI{\StI{s}{M}} &\equiv \FI{M} &\F{x.\mu}{\St{s}{A}} &\equiv \F{x.\mu[\TrPlus{s}{x}/x]}{A} \\
%%\UStI{s}{\UI{x}{M}} &\equiv \UI{x}{M} &\St{s}{\U{c.\mu}{x:A}{B}} &\equiv \U{c.\mu[\TrCirc{s}{c}/c]}{x:A}{B} \\
%%\UI{x}{\UStI{s}{M}} &\equiv \UI{x}{M} &\U{c.\mu}{x:A}{\St{s}{B}} &\equiv \U{c.\TrCirc{s}{\mu}}{x:A}{B} \\
%\UI{x}{M} &\equiv \UI{x}{M[\StI{s}{x}/x]} &\U{c.\mu}{x:\St{s}{A}}{B} &\equiv \U{c.\mu[\TrPlus{s}{x}/x]}{x:A}{B[\StI{s}{x}/x]} \\
\label{eq:stype-pair}\StI{(s, t)}{(M, N)} &\equiv (\StI{s}{M}, \StI{t[\mu/x]}{N}) &\St{(\telety{x'}{s}{t})}{\telety{x'}{A'}{B'}} & \equiv \telety{x}{\St{s}{A'}}{\StE{s}{x}{x'}{\St{t}{B'}}} \\
\StI{s}{\StI{t}{M}} &\equiv \StI{s;t}{M} &\St{s}{\St{t}{A}} &\equiv \St{(s;t)}{A} \\
\StI{\id_p}{M} &\equiv M &\St{\id_p}{A} &\equiv A \\
\label{eq:stype-subst} \rewrite{\ap{\nu}{t/x}}{\StI{\ap{q}{t/x}}{N[M/x]}} &\equiv N[\rewrite{t}{M}/x]  &\St{(\ap{q}{t/x})}{B[M/x]} & \equiv B[\rewrite{t}{M}/x] 
%% other whiskering is a substitution rule:
%% \St{(s[\mu/x])}{B[M/x]} & \equiv (\St{s}{B})[M/x] 
\end{align}
%Where the last term equation is a special case of rewrite on substitutions. The inputs are typed $\Gamma,x:A \vdash_q B \TYPE $\ and $\Gamma \vdash_{\mu'} M : A$ and $\TermTwo{\gamma}{t}{\mu}{\mu'}$.

Due to the lack of the eta rule for $\mathsf{F}$-types, we also need to assert
\begin{mathpar}
(\FE{M}{x}{\rewrite{t[\mu/y]}{N}}) \equiv \rewrite{t[\nu/y]}{\FE{M}{x}{N}}
\end{mathpar}




\section{Examples}
\newcommand\truej[1]{#1 \,\, \dsd{true}}
\newcommand\possj[1]{#1 \,\, \dsd{poss}}
\newcommand\validj[1]{#1 \,\, \dsd{valid}}
\newcommand\crispj[1]{#1 \,\, \dsd{crisp}}
\newcommand\cohesivej[1]{#1 \,\, \dsd{coh}}

\section{Examples}
\label{sec:exampleencodings}

In this section, we give some examples of logical connectives that can
be represented by mode theories in this framework, and explain
informally why they have the desired behavior with respect to
provability.
%% An additional level to these embeddings is making the
%% equational theory of derivations of \seq{\Gamma}{\alpha}{A} match a
%% desired notion of equality of maps/morphisms, and for this it is often
%% necessary to add some additional equations between structural properties
%% $s \deq s' : \alpha \spr \beta$.

\subsection{Non-associative products}

A mode theory with one mode \dsd{m} and a constant
\[
\begin{array}{c}
\oftp{x : \dsd{m}, y : \dsd{m}}{x \odot y}{\dsd{m}}\\
\end{array}
\]
specifies a completely astructural context (no weakening, exchange,
contraction, associtivity).  If we write $A \odot B$ for \F{x \odot
  y}{x:A,y:B} we \emph{cannot}, for example, derive associativity $A
\odot (B \odot C) \vdash (A \odot B) \odot C$.

To attempt a derivation, we can (without loss of generality) begin by
applying the invertible (Lemma~\ref{lem:Finv}) \FL\/ rule twice, at
which point no further left rules are possible, so we must try to apply
\FR:

\begin{footnotesize}
\[
\infer[\FL]
{\seq{a:\F{x \odot p}{x:A,p:\F{y \odot z}{y:B,z:C}}}
  {a}
  {\F{q \odot z}{q:\F{x \odot y}{x:A,y:B},z:C}}}
{
  \infer[\FL]
        {\seq{x:A,p:\F{y \odot z}{y:B,z:C}}{x \times p}{{\F{q \odot z}{q:\F{x \odot y}{x:A,y:B},z:C}}}}
        {\infer[\FR]
          {\seq{x:A,y:B,z:C}{x \times (y \odot z)}{{\F{q \odot z}{q:\F{x \odot y}{x:A,y:B},z:C}}}}
          {\begin{array}{l}
              {x \odot (y \odot z)} \spr (q \odot z)[\alpha_1/q,\alpha_2/z] \\
              \seq{x:A,y:B,z:C}{\alpha_1}{\F{x \odot y}{x:A,y:B}}\\
              \seq{x:A,y:B,z:C}{\alpha_2}{C}\\
            \end{array}
        }}
}
\]
\end{footnotesize}

\noindent To apply \FR, we need to find a substitution for $\alpha_1/q$ and
$\alpha_2/z$ with a structural transformations as above.  In the absence of any
equational or transformation axioms, the only possible choice is $x/q, (y \odot
z)/z$, so we need to show
\[
\seq{x:A,y:B,z:C}{x}{A \odot B}
\qquad
\seq{x:A,y:B,z:C}{y \odot z}{C}
\]
This is not possible because the context is not divded correctly.  

\subsection{Ordered Products and Implications}

We extend the above mode theory with a constant $1 : \dsd{m}$ and
equations
\[
\begin{array}{c}
x \odot (y \odot z) \deq (x \odot y) \odot z\\
x \odot 1 \deq x \deq 1 \odot x
\end{array}
\]
making $(\odot,1)$ into a monoid.  This makes the context behave like
ordered logic, which has associativity but none of exchange, weakening,
and contraction---a monoidal product that is not symmetric monoidal.

We can complete the above proof of associativity of $\odot$: where we
need to find a substitution such that ${x \odot (y \odot z)} \spr (q
\odot z)[\alpha_1/q,\alpha_2/z]$, we can now choose $(x \odot y)/q,
z/z$ because
\[
{x \odot (y \odot z)} \deq {(x \odot y) \odot z} = (q \odot z)[x \odot y/q, z/z]
\]
Thus, the subgoals are
\[
\seq{x:A,y:B,z:C}{x \odot y}{A \odot B}
\qquad
\seq{x:A,y:B,z:C}{z}{C}
\]
The latter is identity-over-identity (Theorem~\ref{thm:identity}), and
the former is a further \FR\/ and then identities:
\[
\infer{\seq{x:A,y:B,z:C}{x \odot y}{\F{x' \odot y'}{x':A,y':B}}}
      { \begin{array}{l}
          x \otimes y \spr (x' \odot y')[x/x',y/y'] \\
          \seq{x:A,y:B,z:C}{x}{A} \\
          \seq{x:A,y:B,z:C}{y}{B} 
        \end{array}
      }
\]
However, we cannot prove commutativity:

\begin{small}
\[
\infer[\FL]{\seq{p:\F{x\odot y}{x:A,y:B}}{p}{\F{z\odot w}{z:B,w:A}}}
      {\infer[\FR]{\seq{x:A,y:B}{x \odot y}{\F{z\odot w}{z:B,w:A}}}
        {
            x \odot y \spr (z \odot w) [\alpha_1/z,\alpha_2/w] &
            \seq{x:A,y:B}{\alpha_1}{B} &
            \seq{x:A,y:B}{\alpha_2}{A} 
      }}
\]
\end{small}

\noindent because the only choice is $\alpha_1 = x$ and $\alpha_2 = y$, which
sends the wrong resource to each branch.  

Ordered logic has two different implications, one that adds to the left
of the context, and one that adds to the right; the expected rules are

\begin{small}
\[
\begin{array}{l}
\infer{\seql{\Gamma}{o}{ A \rightharpoonup B}}
      {\seql{\Gamma,A}{o}B}
~~
\infer{\seql{\Gamma,A \rightharpoonup B,\Delta,\Gamma'}{o}{C}}
      {\seql{\Delta}{o}{A} &
       \seql{\Gamma,B,\Gamma'}{o}{C}
      }
~~
\infer{\seql{\Gamma}{o}{A \leftharpoonup B}}
      {\seql{A,\Gamma}{o}{B}}
~~
\infer{\seql{\Gamma,\Delta,A \leftharpoonup B,\Gamma'}{o}{C}}
      {\seql{\Delta}{o}{A} &
        \seql{\Gamma,B,\Gamma'}{o}{C}
      }
\end{array}
\]
\end{small}

We represent these by 
\[
\begin{array}{ll}
A \rightharpoonup B := \U{c.c \odot x}{x:A}{B} &
A \leftharpoonup B := \U{c.x \odot c}{x:A}{B}
\end{array}
\]
These have the expected right rules, putting $x$ on the left or right of
the current context descriptor, by the substitution $\beta/c$ in \UR:
\[
\infer{\seq{\Gamma}{\beta}{\U{c.c \odot x}{x:A}{B}}}
      {\seq{\Gamma,x:A}{\beta \odot x}{B}}
\qquad
\infer{\seq{\Gamma}{\beta}{\U{c.x \odot c}{x:A}{B}}}
      {\seq{\Gamma,x:A}{x \odot \beta}{B}}
\]
The instances of \UL\/ are
\[
\begin{array}{l}
\infer{\seq{\Gamma} {\beta} {C}}
      {\begin{array}{l}
          c:\U{c.c \odot x}{x:A}{B} \in \Gamma \\
          \beta \spr \beta'[c \odot \alpha/z] \\
          \seq{\Gamma}{\alpha}{A} \\
          \seq{\Gamma,z:A}{\beta'}{C}
        \end{array}
      }
\qquad
\infer{\seq{\Gamma} {\beta} {C}}
      {\begin{array}{l}
          c:\U{c.x \odot c}{x:A}{B} \in \Gamma \\
          \beta \spr \beta'[\alpha \odot c/z] \\
          \seq{\Gamma}{\alpha}{A} \\
          \seq{\Gamma,z:A}{\beta'}{C}
       \end{array}
      }
\end{array}
\]
Suppose that $\beta$ is of the form $x_1 \odot \ldots c \ldots \odot
x_n$ for distinct variables $x_i$, and consider the rule on the left,
for $\rightharpoonup$.  Because the only structural transformations are
the associativity and unit equations, the transformation must
reassociate $\beta$ as $\beta_1 \odot (c \odot \alpha) \odot \beta_2$,
with $\beta' = \beta_1 \odot z \odot \beta_2$, for some $\beta_1$ and
$\beta_2$.  Here $\alpha$ plays the role of $\Delta$ in the ordered
logic rule---the resources used to prove $A$, which occur to the right
of the implication being eliminated.  Reading the substitution
backwards, the resources $\beta'$ used for the continuation are
``$\beta$ with $c \odot \alpha$ replaced by the result of the
implication,'' as desired.  While $c$ and any variables used in $\alpha$
are still in $\Gamma$, permission to use them has been removed from
$\beta'$---and there is no way to restore such permissions in this mode
theory.  The rule for $\leftharpoonup$ is the same, but with $\alpha$ on
the opposite side of $c$.

More formally, for an ordered logic formula built from $\odot
\leftharpoonup \rightharpoonup$ and atoms, write $A^*$ for the
translation to the above encodings, and extend this pointwise to
$\Gamma^*$ for an ordered logic context $\Gamma$.  Further, define
$\vars{x_1:A_1,\ldots,x_n:A_n} = x_1 \odot \ldots \odot x_n$.  Then the
encoding of ordered logic is adequate in the sense that
$\seql{\Gamma}{o}{A}$ iff $\seq{\Gamma^*}{\vars{\Gamma}}{A^*}$.  The
proof is very similar to the proof for affine logic discussed in
Example~\ref{sec:ex:affine}.  The analogous translation of types and
judgements and adequacy statement is used for
Examples~\ref{sec:ex:linear},\ref{sec:ex:affine},\ref{sec:ex:relevant-cartesian}.

\subsection{Linear products and implication}
\label{sec:ex:linear}

Linear logic is ordered logic with exchange, so to model this we add a
commutativity equation
\[
x \otimes y \deq y \otimes x
\]
(and switch notation from $\odot$ to $\otimes$).  For example, we can
derive {\seq{p : A \otimes B}{p}{B \otimes A}}:
\[
\infer[\FL]
      {\seq{p:\F{x\otimes y}{x:A,y:B}}{p}{\F{z\otimes w}{z:B,w:A}}}
      {\infer[\FR]{\seq{x:A,y:B}{x \otimes y}{\F{z\otimes w}{z:B,w:A}}}
        {
            x \otimes y \spr (z \otimes w) [y/z,x/w] &
            \seq{x:A,y:B}{y}{B} &
            \seq{x:A,y:B}{x}{A} 
      }}
\]
where the first premise is exactly $x \otimes y = y \otimes x$.

For this mode theory, \U{c.c \odot x}{x:A}{B} and \U{c.x \odot
  c}{x:A}{B} are equal types (because comutativity is an equation, and
types are parametrized by equivalence-classes of context descriptors),
and both represent $A \lolli B$.  

%% The linear logic $\multimap$-left rule (where contexts are implicitly
%% treated modulo exchange) is
%% \[
%% \infer{\Gamma,\Delta,A \multimap B \vdash C}
%%       {\Delta \vdash A &
%%        \Gamma,B \vdash C}
%% \]
%% We have
%% \[
%% \infer{\seq{\Gamma} {\beta} {C}}
%%       {c:\U{c.c \otimes x}{x:A}{B} \in \Gamma &
%%        \beta \deq \beta'[c \odot \alpha/z] &
%%        \seq{\Gamma}{\alpha}{A} &
%%        \seq{\Gamma,z:A}{\beta'}{C}
%%       }
%% \]

\subsection{Multi-use variables}

An $n$-use variable (see \citep{reed08namessubstructural} for example)
is like a linear variable, but instead of being used ``exactly once''
(modulo additives), it is used ``exactly $n$ times.''  In the above
work, $0$-use variables were used in an encoding of nominal techniques;
another application of $n$-use variables is static analysis of
functional programs\footnote{Andreas Abel, personal communication}
(e.g. counting how many times a variable occurs to decide whether it
will be efficient to unfold a substitution).

%% n-use functions [Wright, Momigliano]
%% • Other 0-use (“irrelevant”) functions [Pfenning, Ley-Wild]
%% • RLF [Ishtiaq, Pym]
%% • HLF
%% – Designed for statement of metatheorems for Linear LF.
%% – Does n-linear Πs above, and more (e.g. some of BI)
%% – Prototype implementation

We use the following sequent calculus rules for $n$-linear functions 
\begin{small}
\[
\infer{{0\cdot \Gamma,x:^1 P} \vdash {P}}
      {}
\qquad
\infer{\Gamma \vdash A \to^n B}
      {{\Gamma, x :^n A} \vdash {B}}
\qquad
\infer{\Gamma + f:^k A \to^n B + (nk \cdot \Delta) \vdash C}
      {\Delta \vdash A &
       {\Gamma, z :^k B} \vdash {C}}
\]
\end{small}%
\noindent where $\Gamma + \Delta$ acts pointwise by $x :^{n} A + x :^{m}
A = x :^{n+m} A$ and $n \cdot \Delta$ acts pointwise by $n \cdot x^{m} A
= x :^{nm} A$.  In the left rule, $\Gamma$ and $\Delta$ have the same
underlying variables and types (but potentially different counts), and
$f:^kA \to^n B$ abbreviates a context with the same variables and types
but $0$'s for all counts besides $f$'s.  The left rule says that if you
spend $k$ occurences of a function that takes $n$ occurences of an
argument, then you need $nk$ occurrences of whatever you use to
construct the argument, in order to get $k$ occurneces of the result.  

We can model this in the linear mode theory by using context descriptors
that are themselves non-linear:
\[
\begin{array}{rcl}
x^0 & := & 1 \\
x^{n+1} & := & x^n \otimes x \\
A \to^n B & := & \U{c.c \otimes (x^n)}{x:A}{B} \\
\end{array}
\]

This has the following instances of \UL{}{} and \UR{}: 
\[
\infer{\seq{\Gamma}{\beta}{A \to^n B}}
      {\seq{\Gamma, x:A}{\beta \otimes x^n}{B}}
\qquad
\infer{\seq{\Gamma}{\beta}{C}}
      {\begin{array}{l}
          f : \U{f.f \otimes x^n}{x : A}{B} \in \Gamma \\
          \beta \spr \beta'[f \otimes (\alpha)^n/z] \\
          \seq{\Gamma}{\alpha}{A} \\
          \seq{\Gamma, z:B}{\beta'}{C} 
       \end{array}
      }
\]
For this mode theory, the only transformations are the commutative
monoid equations, and we can commute $\beta'$ to the form $\beta''
\otimes z^k$ for some $\beta''$ not mentioning $z$ and $k$ because any context descriptor
is a polynomial of variables. Thus the premise is really of form $\beta
\deq (\beta'' \otimes z^k) [f \otimes (\alpha)^n/z]$, which is equal to
$\beta'' \otimes f^k \otimes (\alpha)^{nk}$.  Here $\beta''$
corresponds to the $\Gamma$ in the above left rule (the resources used
in the continuation, besides $z^k$) and $\alpha$ corresponds to $\Delta$.
Overall, we have $x_1:^{k_1} A_1,\ldots,x_n :^{k_n} A_n \vdash C$ iff
\seq{x_1:A_1^*,\ldots,x_n:A_n^*}{x_1^{k_1} \otimes \ldots \otimes
  x_n^{k_n}}{C^*} (where $A^*$ translates atoms to themselves and each
$A \to^n B$ as indicated above).

We can also consider an $n$-use product 
\[
A^n := \F{x^n}{x:A}
\]
as a positive type, which will decompose $A \to^n B$ as $A^n \lolli B$
(by Lemma~\ref{lem:fusion}).  This has a map \seq{p:A^n}{p}{A \otimes
  \ldots \otimes A} but not a converse map \seq{p:A \otimes \ldots \otimes
  A}{p}{A^n}.  For example,
\[
\infer {\seq{p:\F{x \otimes x}{x:A}}{p}{\F{x \otimes y}{x:A,y:A}}}
       {\infer {\seq {x:A}{x \otimes x}{\F{x \otimes y}{x:A,y:A}}}
               {x \otimes x \deq (y \otimes z)[x/y,x/z] &
                \seq{x:A}{x}{A} &
                \seq{x:A}{x}{A}}
       }
\]
the essence of which is the contraction in the substitution
$[x/y,x/z]$.  However,
\[
\infer {\seq{p:\F{y \otimes z}{y:A,z:A}}{p}{\F{x \otimes x}{x:A}}}
       {\infer {\seq {y:A,z:A}{y \otimes z}{\F{x \otimes x}{x:A}}}
               {y \otimes z \deq (x \otimes x)[?/x] &
                \ldots
               }
       }
\]
is not derivable, because there is no substitution into $x \otimes x$
that makes it equal to $y \otimes z$ for distinct $y$ and $z$.
Conceptually, we think of $A^2$ as expressing a notion of identity: it
is a \emph{single} $A$ that can be used twice, which is stronger than
having two potentially different $A$'s.
%% This might be useful for
%% applications of linear logic to imperative or concurrent programming,
%% where there is a notion of identity of memory cells or resources.  

\subsection{Affine products and implications}
\label{sec:ex:affine}

If we extend the linear logic mode theory with our first directed
structural transformation $\dsd{w} :: x \spr 1$ then we get weakening.
For example, we can define a projection
\[
\infer[\FL]{\seq{p : A \otimes B}{p}{A}}
           {\infer[Lemma~\ref{lem:respectspr}]
             {\seq{x:A,y:B}{x \otimes y}{A}}
             {\infer{x \otimes y \spr x}
                    {y \spr 1}
               &
               \infer[Theorem~\ref{thm:identity}]{\seq{x:A,y:B}{x}{A}}{}
             }}
\]

We give the full adequacy proof for this example in the appendix.
Inspecting this proof, we can see that the translation from a ``native''
sequent proof in affine logic to our framework and back is the identity
on cut-free derivations.  The other round-trip is not the identity,
because the framework allows two things that the native sequent calculus
does not.  First, the framework allows weakening at the non-invertible
rules, rather than pushing it to the leaves.  For example, we have
the following two derivations of $P,Q,R \vdash P \otimes R$.

\begin{footnotesize}
\[
\infer[\FR]
      {\seq{x:P,y:Q,z:R}{x \otimes y \otimes z}{\F{x' \otimes z'}{x':P,z':R}}}
      {x \otimes y \otimes z \spr ((x \otimes y) \otimes z) &
        \infer[\dsd{v}]
              {\seq{x:A,y:B,z:C}{x \otimes y}{C}}
              {(x \otimes y) \spr x} &
        \infer[\dsd{v}]
              {\seq{x:A,y:B,z:C}{z}{C}}
              {z \spr z}
      }
\]
\end{footnotesize}
\begin{footnotesize}
\[
\infer[\FR]
      {\seq{x:P,y:Q,z:R}{x \otimes y \otimes z}{\F{x' \otimes z'}{x':P,z':R}}}
      {x \otimes y \otimes z \spr (x \otimes z) &
        \infer[\dsd{v}]
              {\seq{x:A,y:B,z:C}{x}{C}}
              {x \spr x} &
        \infer[\dsd{v}]
              {\seq{x:A,y:B,z:C}{z}{C}}
              {z \spr z}
      }
\]
\end{footnotesize}%

\noindent The second is that a derivation may perform a left rule on a
$0$-linear (in the sense of the previous section) variable, i.e. one
that does not occur in the context descriptor.  Such variables arise
because \UL\/ ``removes a variable from the context'' by marking it as
0-use, not by actually removing it.  For this mode theory (and the other
ones we consider, besides the previous section), these left rules
produce only other 0-use variables, which ultimately cannot be used, and
can be strengthed away (see Lemma~\ref{lem:0-use-strengthening}).

\subsection{Relevant and Cartesian products and implications}
\label{sec:ex:relevant-cartesian}

Next, we consider a logic with contraction, which should result in a map
$A \vdash (A \otimes A)$.  We always have the left and right components
of the chain
\[
\begin{array}{llllllll}
A & \cong & \F{x}{x:A}  & \vdash^? & \F{x \otimes x}{x:A} & \vdash & \F{x \otimes y}{x:A, y : A}
\end{array}
\]
The left isomorphism is just \FL/\FR, while the right map was discussed
above.  So it suffices to give $\F{x}{x:A} \vdash \F{x
  \otimes x}{x:A}$.  Since \dsd{F} is covariant on structural properties
(Lemma~\ref{lem:typespr}), it suffices to add a structural transformation
$\dsd{c} :: x \spr x \otimes x$.  Then we have $A \vdash A^2 \vdash (A
\otimes A)$ but neither of the converses.

Moreover, if we have both $\dsd{w} :: x \spr 1$ and $\dsd{c} :: x \spr x
\otimes x$, then $x \otimes y$ will behave like a cartesian product in
the mode theory (with projections $x \otimes y \spr x$ and $x \otimes y
\spr y$ and pairing of $z \spr x$ and $z \spr y$ ad $z \spr x \otimes
y$), and consequently $A \otimes B$ will behave like a cartesian product
type, and $\U{c.c \otimes x}{x:A}{B}$ like the usual structural $A \to B$.
We refer to this mode theory as an
\emph{cartesian monoid} and write $(\times,\top)$ for it.

%% In this setting, we have both $A \vdash \F{x \otimes x}{x:A}$ (by
%% contraction) and $\F{x \otimes x}{x:A} \vdash A$ (by projection).  It
%% seems reasonable to make this an isomorphism, rather than just an
%% interprovability, expressing the idea that in a cartesian setting, a
%% one-use $A$ is exactly the same as a two-use $A$.  (On the other hand,
%% we do not want $A \cong (A \times A)$, because a single $A$ is not the
%% same as two potentially different $A$'s).  To accomplish this, we can
%% take contraction to be an equation $x \deq x \otimes x$ (idempotence of
%% $\otimes$) rather than a directed transformation.  This makes $A \cong
%% A^2 \dashv\vdash A \times A$.  We refer to the mode of theory for an
%% idempotent commutative monoid with a weakening transformation as an
%% \emph{idempotent cartesian monoid} and write $(\times,\top)$ for it.

\subsection{Bunched Implication (BI)} Bunched implication~\citep{ohearnpym99bunched}
has two context-forming operations $\Gamma,\Gamma'$ and
$\Gamma;\Gamma'$, along with corresponding products and implications.
Both are associative, unitial, and commutative, but $;$ has weakening
and contraction while $,$ does not.  A context is represented by a tree
such as $(x:A, y:B);(z : C, w : D)$ (considered modulo the laws), and
the notation $\Gamma[\Delta]$ is used to refer to a tree with a hole
$\Gamma[-]$ that has $\Delta$ as a subtree at the hole.  In sequent
calculus style, the rules for the product and implication corresponding
to $,$ are
\[
\begin{array}{l}
\infer{\Gamma[A * B] \vdash C}
      {\Gamma[A , B] \vdash C}
\quad
\infer{\Gamma,\Delta \vdash A * B}
      {\Gamma \vdash A &
       \Delta \vdash B}
\quad
\infer{\Gamma \vdash A \magicwand B}
      {\Gamma, A \vdash B}
\quad
\infer{\Gamma[A \magicwand B, \Delta] \vdash C}
      {\Delta \vdash A &
       \Gamma[B] \vdash C}
\end{array}
\]
There are similar rules for a product and implication for $;$ as well as
structural rules of weakening and contraction for it.

We can model BI by a mode \dsd{m} with a commutative monoid $(*,I)$ and
a cartesian monoid $(\times,\top)$.  
%% \[
%% \begin{array}{l}
%% x  : \dsd{m}, y  : \dsd{m} \vdash x \times y : \dsd{m} \\
%% \cdot \vdash \top : \dsd{m} \\
%% x  : \dsd{m}, y  : \dsd{m} \vdash x * y : \dsd{m} \\
%% \cdot \vdash \dsd{I} : \dsd{m} \\
%% \end{array}
%% \]
%% where both $(\times,\top)$ and $(*,I)$ are commutative monoids, $\times$
%% is idempotent, and $\top$ (but not $I$) is terminal ($x \spr \top$).  
We define the BI products and implications using the monoids:
\[
\begin{array}{ll}
A * B := \F{x * y}{x : A, y : B}  &
A \magicwand B := \U{c.c * x}{x : A}{B} \\
A \times B := \F{x \times y}{x : A, y : B} &
A \to B := \U{c.c \times x}{x : A}{B}\\
\end{array}
\]
A context descriptor such as $(x \times y) * (z \times w)$ captures
the ``bunched'' structure of a BI context, and substitution for a
variable models the hole-filling operation $\Gamma[\Delta]$.  The left
rule for $*$ (and similarly $\times$) acts on a leaf
\[
\infer{\seq{\Gamma,z:A*B,\Gamma'}{\beta}{C}}
      {\seq{\Gamma,\Gamma',x:A,y:B}{\subst{\beta}{x * y}{z}}{C}}
\]
and replaces the leaf where $z$ occurs in the tree $\beta$ with the
correct bunch $x*y$, The left rule for $\magicwand$ (and similarly for
$\to$)
\[
\infer{\seq{\Gamma}{\beta}{C}}
      {
        c : A \magicwand B \in \Gamma &
        \beta \spr \beta'[ c * \alpha / z] & 
        \seq{\Gamma}{\alpha}{A} &
        \seq{\Gamma,z:B}{\beta'}{C} 
      }
\]
isolates a subtree containing the implication $c$ and resources $*$'ed
with it, uses those resources to prove $A$, and then replaces the
subtree with the variable $z$ standing for the result of the
implication.

We assume the BI sequent is given as a judgement $\Gamma \vdash A$ where
$\Gamma$ is a tree and there are explicit equality premises for the
algebraic laws on bunches.  Then we define $\Gamma^*$ as an in-order
flattening of the tree into one of our contexts (e.g.  $x:A^* = x:A$ and
$(\Gamma,\Delta)^* = (\Gamma;\Delta)^*=\Gamma^*,\Delta^*$), while we
define $\vars{\Gamma}$ as a context descriptor that preserves the tree
structure (e.g. $x:A^* = x$ and $\vars{(\Gamma,\Delta)} =
\vars{\Gamma}*\vars{\Delta}$ and
$\vars{\Gamma;\Delta}=\vars{\Gamma}\times\vars{\Delta}$).  Then we have
the usual adequacy statement $\Gamma \vdash A$ iff
\seq{\Gamma^*}{\vars{\Gamma}}{A^*}.

\subsection{Adjoint decomposition of !}  
\label{sec:example:bang}

Following \citet{benton94mixed,bentonwadler96adjoint}, we decompose the
$!$ exponential of intuitionistic linear logic as the comonad of an
adjunction between ``linear'' and ``cartesian'' categories.  We start
with two modes \dsd{l} (linear) and \dsd{c} (cartesian), along with a
commutative monoid $(\otimes,1)$ on \dsd{l} and a cartesian
monoid $(\times,\top)$ on \dsd{c}.  Next, we add a context descriptor
from \dsd{c} to \dsd{l}:
\[
x : \dsd{c} \vdash \dsd{f}(x) : \dsd{l}
\]
that we think of as including a cartesian context in a linear context.
This generates types 
\[
\wftype {\F{\dsd{f}(x)}{x : A_{\dsd{c}}}}{\dsd{l}}
\qquad
\wftype {\U{x.\dsd{f}(x)}{\cdot}{A_{\dsd{l}}}}{\dsd{c}}
\]
which are adjoint $\F{\dsd{f}(x)}{x:-} \la
{\U{x.\dsd{f}(x)}{\cdot}{-}}$.  The bijection on hom-sets is defined
using \FL\/ and \FR\/ and their invertibility
(Corollary~\ref{cor:Uinv}, Lemma~\ref{lem:Finv}):
\[
\infer={\seq{p:\F{\dsd{f}(x)}{x:A}}{p}{B}}
       {\infer={\seq{x:A}{\dsd{f}(x)}{B}}
               {\seq{x:A}{x}{\U{x.\dsd{f}(x)}{\cdot}{B}}}}
\]
The comonad of the adjunction
\F{\dsd{f}(x)}{x:\U{c.\dsd{f}(c)}{\cdot}{A}} is the linear logic $!A$.

In the LNL models and sequent calculus~\citep{benton94mixed}, $F(A
\times B) \cong F(A) \otimes F(B)$ and $F(\top) \cong 1$, which we can
add to the mode theory by equations 
\[
\dsd{f}(x \times y) \deq \dsd{f}(x) \otimes \dsd{f}(y)
\qquad \dsd{f}(\top) \deq 1
\]
These equations then extend to isomorphisms using Lemma~\ref{lem:fusion}
because all of $F,\otimes,\times$ are represented by \Fsymb-types in our
framework.  These properties of \dsd{f} are necessary to prove that $!
A$ has weakening and contraction (with respect to $\otimes$) and $!A
\otimes !B \vdash !(A \otimes B)$, for example.  Omitting these
equations allows us to describe non-monoidal (or lax monoidal, if we add
only one direction) left adjoints.

In general, we translate $F(A)^* = \F{\dsd{f}(x)}{x:C^*}$ and $G(A)^* =
\U{x.\dsd{f}(x)}{\cdot}{A}$ and products and functions as usual.
Then a sequent $x_1:C_1,\ldots,x_n:C_n \vdash C$ in the cartesian
category is represented by a sequent
\seq{x_1:C_1^*,\ldots,x_n:C_n^*}{x_1 \times \ldots \times x_n}{C^*}, 
and 
a mixed sequent with cartesian and linear assumptions and a linear
conclusion  $x_1:C_1,\ldots,x_n:C_n;y_1:A_1,\ldots,y_m:A_m \vdash A$ 
by 
\seq{x_1:C_1^*,\ldots,y_1:A_1^*,\ldots}{\dsd{f}(x_1) \otimes\ldots\otimes
  \dsd{f}(x_n)  \otimes y_1 \otimes \ldots \otimes y_n}{A^*}.


%% The $ \dsd{f}(x) \otimes \dsd{f}(y) \spr \dsd{f}(x \times y)$
%% direction is used to prove the purely linear logic entailment $!A
%% \otimes !B \vdash !(A \otimes B)$, for example.

%% For example, for contraction we can begin
%% \[
%% \infer{\seq{p : ! A}{p}{! A \otimes ! A}}
%%       {\infer{\seq{{x:\U{c.\dsd{f}(c)}{\cdot}{A}}}{f(x)}{{! A \otimes ! A}}}
%%              {\begin{array}{l}
%%                  f(x) \spr (x' \otimes y') [\dsd{f}(x) / x' , \dsd{f}(x) / y'] \\
%%                  \seq{x:\U{c.\dsd{f}(c)}{\cdot}{A}}{\dsd{f}(x)}{! A} \\
%%                  \seq{x:\U{c.\dsd{f}(c)}{\cdot}{A}}{\dsd{f}(x)}{! A} 
%%                \end{array}
%%              }}
%% \]
%% and we can derive \seq{x:\U{c.\dsd{f}(c)}{\cdot}{A}}{\dsd{f}(x)}{! A}
%% by \FR\/ (it is of the ``axiomatic'' form
%% $\seq{x:C}{f(x)}{\F{\dsd{f}(x)}{x:C}}$).  The key point is that the first
%% premise, which reduces to
%% \[
%% \dsd{f}(x) \spr \dsd{f}(x) \otimes \dsd{f}(x)
%% \]
%% can be deduced as
%% \[
%% \dsd{f}(x) \deq \dsd{f}(x \times x) \deq \dsd{f}(x) \otimes \dsd{f}(x)
%% \]
%% by contraction for $\times$ \emph{if we add an axiom that \dsd{f}
%%   (strictly) preserves the monoidal product}
%% \[
%% \dsd{f}(x \times y) \deq \dsd{f}(x) \otimes \dsd{f}(y)
%% \]

%% Similarly, to get weakening we take $\dsd{f}(\top) \deq 1$.  

\subsection{Adjoint decomposition of $\Box$}  
\label{sec:example:box}

The modal S4 \Bx{}{} as in \citet{pfenningdavies} is similar to !.  We
call the two modes \dsd{t}ruth and \dsd{v}alidity and have 
cartesian monoids on both (we write $(\times,\top)$ for the \dsd{t} one
and $(\times_v,\top_v)$ for the \dsd{v} one) along with $x : \dsd{v}
\vdash \dsd{f}(x) : \dsd{t}$.  Here, following the analysis of \Bx{}{}
as a monoidal comonad~\citep{alechina+01categoricals4}, we have only lax
monoid-preservation axioms
\[
\dsd{f}(x) \times \dsd{f}(y) \spr \dsd{f}(x \times_v y) \\
\qquad
\top \spr \dsd{f}(\top_v)
\]
though the difference is only at the level of equality of
derivations.\footnote{Because the context monoids are cartesian
  products, there are always converse maps, e.g.  $\dsd{f}(x \times_v y)
  \spr \dsd{f}(x) \times \dsd{f}(y)$ defined by pairing, projection, and
  congruence.  However, in the equational theory of proofs in
  S4~\citep{pfenningdavies}, there is a section-retraction $(\Box A
  \times \Box B) \rightarrowtail \Box (A \times B) \twoheadrightarrow
  (\Box A \times \Box B)$ but not an isomorphism. If we had equalities
  above, they would generate type isomorphisms $\dsd{F}(A \times_v B)
  \cong \dsd{F}(A) \times \dsd{F}(B)$, and because the right-adjoint
  $\dsd{U}$ preserves products, we would have $\dsd{F} \dsd{U} (A \times
  B) \cong \dsd{F}(U A \times_v U B) \cong (\dsd{FU}(A) \times
  \dsd{FU}(B))$, which does not match the existing theory---though it is
  a reasonable alternative to consider.
%% I think it reduces to p : Box A =?= letbox x = p in letbox y = p in box(fst x, snd y)
}
We represent a sequent 
\[
 x_1:\validj{A_1},\ldots,x_n:\validj{A_n};y_1:\truej{B_1},\ldots \vdash \truej{C}
\]
by 
\[
\seq{x_1:\Uempty{\dsd{f}}{A_1^*},\ldots,x_n:\Uempty{\dsd{f}}{A_n^*};y_1:B_1^*,\ldots}
    {\dsd{f}(x_1) \times\ldots\times \dsd{f}(x_n) \times y_1 \times \ldots \times y_n}{C^*}
\]


\subsection{Subexponentials}

Subexponentials~\citep{danos+93subexponentials,nigammiller09subexponentials} extend linear logic with a family of
comonads $!_a A$.  All of the comonads are monoidal ($!_a A \otimes !_a
B \vdash !_a(A \otimes B)$ and $1 \vdash !_a A$), and there is a
preorder $a \le b$ such that $!_b A \vdash !_a A$.  Each $!_a$ is
allowed to have weakening and/or contraction, subject to the constraint
that when $a \le b$, $b$ must be at least as structural as $a$.

We illustrate the embedding on a specific example of the diamond
preorder generated by $\dsd i < \dsd j,\dsd k < \dsd m$.  Following
\citep[Example 4.3]{reed09adjoint}, we identify each subexponential $a$
with a mode, and have an additional mode \dsd{l} for basic linear truth,
all with commutative monoids $(\otimes_a,1_a)$.  We add context
descriptor constants \oftp{x : b}{ba(x)}{a} for each $a < b$ (so, in
this example, \dsd{mk}, \dsd{mj}, \dsd{ji}, \dsd{ki}), with an
additional \oftp{x:\dsd i}{\dsd{il}(x)}{\dsd l}.  These include each
``higher'' mode into the immediately ``lower'' ones, and the lowest ones
into \dsd{l}.  We add an equation $\dsd{ji}(\dsd{mj}(x)) \deq
\dsd{ki}(\dsd{mk}(x))$ that the diamond commutes.  Then $!_b A$ is the
comonad $\F{b{\dsd l}(x)} {x : \U{x.b\dsd{l}(x)}{\cdot}{A}}$ for the
unique $\oftp{x : b}{b{\dsd l}}{\dsd{l}}$ generated by these constants.
For example, $!_k$ is the comonad of $\oftp{x :
  \dsd{k}}{\dsd{il}(\dsd{ki}(x))}{\dsd{l}}$.

This mode theory is constructed so that every mode has a unique map to
\dsd{l}.  When $a \le b$, we have a morphism \oftp{x:b}{ab(x)}{a}, so
the morphism \oftp{x:b}{b\dsd{l}(x)}{\dsd{l}} is equal to
\oftp{x:b}{a\dsd{l}(ba(x))}{\dsd{l}}.  Thus, by Lemma~\ref{lem:fusion},
we have
\[
!_b A = \dsd{F}_{{b\dsd{l}}}(\dsd{U}_{{b\dsd{l}}} A) \cong \dsd{F}_{{a\dsd{l}}}\dsd{F}_{{ba}} \dsd{U}_{{ba}} \dsd{U}_{{a\dsd{l}}} A
\]
The map $!_b A \vdash !_a A$ can thus be defined as the conunit for the
$\dsd{F}_{{ba}} \dsd{U}_{{ba}} A \vdash A$ for the comonad in the middle.

We add equations $ba(x \otimes_b y) \deq ba(x) \otimes_a ba(y)$ and
$ba(1_b) \deq 1_a$ making each generator strictly monoidal.  This
ensures that each $!_b$ is monoidal and that $!_b A$ can be weakned or
contracted if $(\otimes_b,1_b)$ has weakening or contraction (and more
generally that \F{ba}{B} can be weakened or contracted for any $B$, not
just $\dsd{U}_{ba}{(A)}$).  Thus, we add weakening or contraction to
a particular subexponential $a$ by adding them to $(\otimes_a,1_a)$.
%% assuming only monoid axioms (not that one is cartesian), 
%% - !_b A \otimes !_b B \vdash !_b(A \otimes B) uses both directions of ba(x \otimes_b y) <=> ba(x) \otimes_a ba(y)
%% - 1 \vdash !_b(1) uses both directions of ba(1_b) \deq 1_a
%% - contraction uses f(a . b) => f(a) . f(b) direction
%% - weakening uses f(1) => 1 direction

Interestingly, when $a \le b$, it does not seem that we need a condition
that $(\otimes_b,1_b)$ has whatever structural properties
$(\otimes_a,1_a)$ has in order to get that $!_b A$ is at least as
structural as $!_a A$.  As argued above $!_b A$ factors into the form
$\dsd{F}_{a{\dsd l}}(C)$, which has whatever structural properties mode $a$ has.

%% Following \citep[Example 4.3]{reed09adjoint}, we identify each
%% subexponential $i$ with a mode.  We put a commutative monoid
%% $(\otimes_i,1_i)$ on each mode, and an additional mode \dsd{l} with
%% $(\otimes,1)$ for non-modal truth.  We assume that the relation ($i \le
%% j \cup (\forall k, \dsd{l} \le k)$ is the reflexive, transitive closure
%% of a simple graph, and add a context descriptor $\oftp{x :
%%   \dsd{j}}{\dsd{ji}(x)}{i}$ for each edge of this graph, with equations
%% $\alpha \deq \beta$ when $\alpha$ and $\beta$ both correspond to paths
%% between the same two nodes in the graph.  Then the subexponential $!_i
%% A$ is defined to be $\F{\alpha} {x : \U{x.\alpha}{\cdot}{A}}$ for the
%% $\oftp{x : i}{\alpha}{\dsd{l}}$.

\subsection{Monads}

Consider a \Dia{}{A} modality with rules in the style of
\citet{pfenningdavies}: 
\[
\infer{\Gamma \vdash \possj{A}}
      {\Gamma \vdash \truej{A}}
\qquad
\infer{\Gamma \vdash \truej{\Dia{}{A}}}
      {\Gamma \vdash \possj{A}}
\qquad
\infer{\Gamma,\truej{\Dia{}{A}} \vdash \possj{C}}
      {\truej{A} \vdash \possj{C}}
\]

We can model this using a mode theory with two modes \dsd{t} and \dsd{p}
and context descriptor \oftp{x:\dsd{t}}{\dsd{g}(x)}{\dsd{p}}, defining
the type $\Dia{}{A} :=
\U{c.\dsd{g}(c)}{\cdot}{\F{\dsd{g}(x)}{x:A}}$.  This is always a monad,
but it does not automatically have a tensorial strength, which
corresponds to the context-clearing in the left rule.

For example, if we have a monoid $(\otimes_\dsd{t},1_{\dsd t})$ on mode
\dsd{t} and try to derive
\begin{small}
\[
\infer[\UR]
      {\seq{x : A, y : \Dia{\dsd{g}}{B}}{x \otimes_{\dsd t} y}{\Dia{\dsd{g}}{(A \otimes_{\dsd t} B)}}}
      {\infer[\UL]
        {\seq{x : A, y : \Dia{\dsd{g}}{B}}{\dsd{g}(x \otimes_{\dsd t} y)}{\F{\dsd{g}}{A \otimes_{\dsd t} B}}}
        {\dsd{g}(x \otimes_{\dsd t} y) \spr \subst{\beta'}{\dsd{g}(y)}{z} &
          \seq{x:A,y : \Dia{\dsd{g}}{B},z:\F{\dsd{g}}{B}}{\beta'}{\F{\dsd{g}}{A \otimes_{\dsd t} B}}
        }}
\]
\end{small}
we are stuck, because there is no way to rewrite $\dsd{g}(x
\otimes_{\dsd t} y)$ as a term containing $\dsd{g}(y)$.  If
$(\otimes_t,1_t)$ is affine, then we can weaken away $x$ and take
$\beta' = z$---the context clearing of in the left rule---but then in
the right-hand premise we will only have access to $z$, not $x$, and
cannot complete the derivation.

In general, we translate all formulas at mode \dsd{t} and represent
\Dia{}{A} as above, and translate a sequent $\truej{A_1}, \ldots,
\truej{A_1} \vdash \truej{C}$ by
\seq{x_1:A_1^*,\ldots,x_1:A_n^*}{x_1\otimes\ldots\otimes x_n}{C^*} and a
sequent $\truej{A_1}, \ldots, \truej{A_b} \vdash \possj{C}$ by
\seq{x_1:A_1^*,\ldots,x_1:A_n^*}{\dsd{g}(x_1\otimes\ldots\otimes
  x_n)}{\F{\dsd{g}}{C^*}}.  Then the three ``native'' rules above are
\FR, \UR, and a composite of \UL\/ followed by \FL, respectively.

\newcommand\ttp[2]{#1 \otimes_{\dsd {tp}} #2}
\newcommand\tvp[2]{#1 \otimes_{\dsd {vp}} #2}

Some monads, such as the \Crc{}{A} of \citep{pfenningdavies} and those
used to encapsulate effects in functional programming are strong.  
One way to axiomatize the strength is using an asymmetric product of a
\dsd{t}-mode and \dsd{p}-mode context:
\[
\begin{array}{ll}
\oftp{x : \dsd{t}, y : \dsd{p}}{\ttp x y}{\dsd{p}}
& \dsd{g}(x \otimes_{\dsd t} y) \deq \ttp x {\dsd{g}(y)}\\
\ttp {(x \otimes_{\dsd t} y)} z \deq \ttp x {(\ttp y z)}
& \ttp {\dsd{1}} y \deq y
\end{array}
\]
The equations make this into a monoid action of the \dsd{t}-contexts on
the \dsd{p}-contexts, and allow for ``isolating'' any one $x_i$ in
$\dsd{g}(x_1 \otimes_{\dsd t} \ldots \otimes_{\dsd t} x_n)$ as the
designated variable under a \dsd{g}.  Using this (and switching notation
from \Dia{\dsd{g}}{A} to \Crc{\dsd{g}}{A}), we can prove

\begin{footnotesize}
\[
\infer[\UR]
      {\seq{x : A, y : \Crc{\dsd{g}}{B}}{x \otimes_{\dsd t} y}{\Crc{\dsd{g}}{(A \otimes_{\dsd t} B)}}}
      {\infer[\UL]
        {\seq{x : A, y : \Crc{\dsd{g}}{B}}{\dsd{g}(x \otimes_{\dsd t} y)}{\F{\dsd{g}}{A \otimes_{\dsd t} B}}}
        {\dsd{g}(x \otimes_{\dsd t} y) \spr \subst{(\ttp x z)}{\dsd{g}(y)}{z} &
          \infer[\FL]
                {\seq{x:A,y : \Crc{\dsd{g}}{B},z:\F{\dsd{g}}{B}}{\ttp x z}{\F{\dsd{g}}{A \otimes_{\dsd t} B}}}
                {\infer[\UL]
                       {\seq{x:A,y : \Crc{\dsd{g}}{B},z':B}{\ttp{x}{\dsd{g}(z')}}{\F{\dsd{g}}{A \otimes_{\dsd t} B}}}
                       { {\ttp{x}{\dsd{g}(z')}} \spr \dsd{g}(x \otimes_{\dsd t} z') & 
                         \infer[\FR]{\seq{\ldots}{x \otimes_{\dsd t} z'}{{A \otimes_{\dsd t} B}}}{}
                       }
        }}}
\]
\end{footnotesize}%


An analogous description can be given for the ``$\Box$-strong
$\Diamond$''~\citep{pfenningdavies,alechina+01categoricals4}, which has
a strength only for boxed formulas ($\Bx{} A \otimes \Dia{} B \vdash
\Dia{}(\Bx{} A \otimes B)$).  We use 3 modes \dsd{v},\dsd{t},\dsd{p} and
represent the $\Box$ as the comonad of a context descriptor
\oftp{x:\dsd{v}}{\dsd{f}(x)}{\dsd{t}} (with cartesian monoids
on \dsd{v} and \dsd{t} and \dsd{f} laxly monoidal as above), and the
$\Diamond$ as the monad of a \oftp{x:\dsd{t}}{\dsd{g}(x)}{\dsd{p}}.  We
have a mixed-mode product between \dsd{v} and \dsd{p}
\[
\begin{array}{ll}
\oftp{x : \dsd{v}, y : \dsd{p}}{\tvp x y}{\dsd{p}}
& \dsd{g}(\dsd{f}(x) \times_{\dsd t} y) \deq \tvp x {\dsd{g}(y)}\\
\ttp {(x \times_{\dsd v} y)} z \deq \ttp x {\tvp y z}
& \tvp {\dsd{1}} y \deq y
\end{array}
\]

We represent the truth-conclused sequent as in
Example~\ref{sec:example:box}, and $x_1:A_1 \dsd{valid},\ldots;y_1:B_1
\dsd{true},\ldots \vdash C \, \dsd{poss}$ by
\[
\seq{x_1:\Uempty{\dsd{f}}{A_1},\ldots,y_1:B_1,\ldots}
    {\dsd{g}(\dsd{f}(x_1) \times_{\dsd t} \dsd{f}(x_2) \times_{\dsd t} \ldots y_1 \times_{\dsd t} \ldots )}
    {\F{\dsd{g}}{C}}
\]
The left rule
\[
\infer{\Delta ; \Gamma, \truej{z:\Dia{}{A}} \vdash \possj C}
      {\Delta ; w':\truej A \vdash \possj C}
\]
that keeps the valid assumptions and discards the true ones is derivable
by
\begin{footnotesize}
\[
\infer%[\UL]
      {\seq{x_i:\Uempty{\dsd{f}}{A_i},y_i:B_i,z:\Dia{\dsd{g}}{A}}{\dsd{g}(\dsd{f}{(x_i)} \times y_i \times z)}{\F{\dsd{f}}{C}}}
      {
        \dsd{g}(\dsd{f}{(x_i)} \times y_i \times z) \spr \tvp{(x_1 \times_v \ldots x_n)} {\dsd{g}(z)} & 
        \infer%[\FL]
            {\seq{\ldots,w:\F{\dsd{g}}{A}}{(\tvp{(x_1 \times_v \ldots_v \times x_n)} {w})}{{\F{\dsd{f}}{C}}}}
            {\seq{\ldots,w':A}{(\tvp{(x_1 \times_v \ldots \times_v x_n)} {\dsd{g}(w')})}{{\F{\dsd{f}}{C}}}}
      }
\]
\end{footnotesize}
\noindent The transformation is given by weakening away $y_i$ and
using the monoidalness of \dsd{f} and the isolation equation:
\[
\begin{array}{ll}
& \dsd{g}(\dsd{f}(x_1) \times \ldots \times \dsd{f}(x_n) \times y_1 \times \ldots \times z)\\
\spr & \dsd{g}(\dsd{f}(x_1) \times \ldots \times \dsd{f}(x_n) \times z)\\
\deq & \dsd{g}(\dsd{f}(x_1 \times_v x_n) \times z)\\
\deq & \tvp {(x_1 \times_v x_n)} z
\end{array}
\]
The right-hand premise is the encoding of the premise of the rule, using
the isolation equation and monoidalness of \dsd{f} in the other
direction.  The restriction of the isolation equation to \dsd{f}
prevents keeping any additional \dsd{true}\/ variables in the premise.

\subsection{Spatial Type Theory}

The spatial type theory for cohesion~\citep{shulman15realcohesion}
(which motivated this work) has an adjoint pair $\flat \la \sharp$,
where $\flat$ is a comonad and $\sharp$ is a monad, with some additional
properties.  In the one-variable case~\citep{ls16adjoint}, we analyzed
this as arising from an idempotent comonad\footnote{There it was an
  idempotent monad; the variance of \dsd{F} and \dsd{U} has been flipped
  in paper.} in the mode theory: we have a mode \dsd{c} with a cartesian
monoid $(\times,\top)$ and a context descriptor
\oftp{x:\dsd{c}}{\dsd{r}(x)}{\dsd{c}} such that $\dsd{r}(\dsd{r}(x))
\deq \dsd{r}(x)$ and there is a directed transformation $\dsd{r}(x) \spr
x$.  Then we define $\flat A := \F{\dsd{r}}{A}$ and $\sharp A :=
\Uempty{\dsd{r}}{A}$. These are adjoint as discussed in
Example~\ref{sec:example:bang}, and the transformation gives the counit
$\F{\dsd{r}}{A} \vdash A$ and the unit $A \vdash \Uempty{\dsd{r}}{A}$ by
Lemma~\ref{lem:typespr}.  Now that we have a multi-assumptioned logic,
we can model the fact that $\flat{A}$ preserves products by the equation
equation $\dsd{r}(x \times y) \deq \dsd{r}(x) \times \dsd{r}(y)$.
Overall, we encode a simply-typed spatial type theory judgement $x_1 :
\crispj{A_1},\ldots;y_1:\cohesivej{B_1} \vdash \cohesivej{C}$ as
$\seq{x_1:A_1,\ldots,y_1:B_1,\ldots}{\dsd{r}(x_1)\times\ldots\times
  y_1\times\ldots}{C}$.  This corresponds to the following native rules:

\begin{footnotesize}
\[
\begin{array}{c}
\infer{\Delta;\Gamma \vdash C}
      {A \in \Delta &
       \Delta;\Gamma,A \vdash C}
\quad
\infer{\Delta; \Gamma \vdash {\Flat A}}
      {\Delta; \cdot \vdash {A}}
\quad
\infer{\Delta; \Gamma,\Flat{A} \vdash C}
      {\Delta,A; \Gamma \vdash C}
\quad
\infer{\Delta;\Gamma \vdash {\Sharp C}}
      {\Delta,\Gamma; \cdot \vdash C}
\quad
\infer{\Delta;\Gamma \vdash C}
      {\Sharp A \in \Delta &
        \Delta;\Gamma,A \vdash {C}}
\quad
\end{array}
\]
\end{footnotesize}%

\noindent In order, these correspond to (1) the action of the
contraction and $\dsd{r}(x) \spr x$ transformations; (2) \FR\/ with
weakening, using monoidalness of \dsd{r} in one direction; (3) \FL; (4)
\UR, using monoidalness of \dsd{r} in the other direction and
idempotence; (5) \UL, with contraction.  This provides a satisfying
explanation for the unusual features of these rules, such as promoting
all cohesive variables to crisp in \Sharp{}-right, and eliminating a
crisp \Sharp{} in \Sharp{}-left.  


\section{Semantics}



\section{Conclusion}

We have described a sequent calculus that can express a variety of
substructural and modal logics through a suitable choice of mode theory.
The framework itself enjoys identity and cut admissibility for all mode
theories, and these properties are inherited by the logics that are
represented in it.  The logic corresponds semantically to a fibration
between 2-dimensional cartesian multicategories, and so gives both a
syntactic and semantic account of the idea that substructural and modal
logics are constraints on structural proofs.  

%% In future work, we will
%% investigate the issue of equality of derivations, leading to an
%% initiality theorem with respect to the semantic structure, and to a
%% better understanding of how the equality of derivations in the framework
%% corresponds to object-language equality of proofs/programs/morphisms.



%%
%% Bibliography
%%

%% Please use bibtex, 

{
  \bibliography{../../drl-common/cs}
}

\end{document}

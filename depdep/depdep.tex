\documentclass[10pt]{article}

\usepackage{fullpage}
\usepackage{amssymb,amsthm,bbm}
\usepackage[centertags]{amsmath}
\usepackage[mathscr]{euscript}
\usepackage{tikz-cd}
\usepackage{mathpartir}

\newcommand{\yields}{\vdash}
\newcommand{\cbar}{\, | \,}

\newcommand{\Id}[3]{\mathsf{Id}_{{#1}}(#2,#3)}
\newcommand{\CTX}{\,\,\mathsf{CTX}}
\newcommand{\ctx}{\,\,\mathsf{ctx}}
\newcommand{\TYPE}{\,\,\mathsf{TYPE}}
\newcommand{\type}{\,\,\mathsf{type}}
\newcommand{\TELE}{\,\,\mathsf{TELE}}
\newcommand{\tele}{\,\,\mathsf{tele}}

\newcommand\FE[3]{\ensuremath{\mathsf{split} \, #2 \, = \, {#1} \, \mathsf{in} \, #3}}
\newcommand\FI[1]{\ensuremath{\mathsf{F}{(#1)}}}
\newcommand\UE[2]{\ensuremath{#1(#2)}}
\newcommand\UI[2]{\ensuremath{\lambda #1.#2}}
\newcommand\Trd[2]{\ensuremath{#1_*(#2)}}

\title{Adjoint Type Theory}
\author{}
\date{}

\begin{document}
\maketitle

Capital things live above, lowercase things live below. Not sure how to do the annotations and keep them legible.

Judgements: 
\begin{itemize}
\item $\yields_\gamma \Gamma \CTX$ lives over $\yields \gamma \ctx$
\item $\Gamma \yields_\alpha A \TYPE$ lives over $\gamma \yields \alpha \type$
\item $\Gamma \yields_\mu M : A$ lives over $\gamma \yields \mu : \alpha$
\item Telescopes $\Gamma \yields_\delta \Delta \TELE$ lying over $\gamma \yields \delta \tele$
\item Substitutions $\Gamma \yields_\theta \Theta : \Delta$ lying over $\gamma \yields \theta : \delta$
\item 2-cells $\gamma \yields s : (\mu \Rightarrow \nu) : \alpha$
\end{itemize}

Note: we could build $s_*$ into the other rules, but not much
reason to in natural deduction.  

Some intended admissible rules:
\begin{mathpar}
\inferrule*[Left = weaken-over]
           {\Gamma,\Gamma' \yields_\mu M : A (\text{where } \gamma,\gamma' \vdash \mu : \alpha)}
           {\Gamma,y:B,\Gamma' \yields_\mu M : A (\text{where } \gamma,y:\beta,\gamma' \vdash \mu : \alpha)}

\inferrule*[Left = subst-over]
           {\Gamma,x:A,\Gamma' \yields_\nu N : C \and (\text{where } \gamma,x:\alpha,\gamma' \vdash \nu : \gamma) \\\\
            \Gamma \vdash_\mu M : A \and (\text{where } \gamma \vdash \mu : \alpha)
           }
           {\Gamma,\Gamma'[M/x] \yields_{\nu[\mu/x]} N[M/x] : A[M/x] \and (\text{where } \gamma,\gamma'[\mu/x] \vdash \nu[\mu/x] : \alpha[\mu/x])}
           
\end{mathpar}

\begin{mathpar}
  \inferrule*[Left = ctx-form]{ }
  {\yields_{\cdot} \cdot \CTX  } \and 

  \inferrule*[Left = ctx-form]{
    \yields_\gamma \Gamma \CTX \and (\text{where } \yields \gamma \ctx) \\\\
    \Gamma \yields_\alpha A \TYPE \and (\text{where }  \gamma \yields \alpha \type)}
  {\yields_{\gamma, x : \alpha} \Gamma, x : A \CTX \and (\text{where } \yields \gamma,x:\alpha \ctx)  } \\

  \inferrule*[Left = tele-form]{ }
             {\Gamma \yields \cdot \TELE_{\cdot}  } \and

  \inferrule*[Left = tele-form]{
    \Gamma \yields_\delta \Delta \TELE \and  (\text{where } \gamma \yields \delta \tele) \\\\
    \Gamma,\Delta \yields_\alpha A \TYPE \and (\text{where } \gamma,\delta \yields \alpha \type)}
  {\Gamma \yields_{\delta, x : \alpha} \Delta, x : A \TELE  \and (\text{where } \gamma \yields \delta,x:\alpha \tele)} \\ \\

  \and

  \inferrule*[Left = sub1]{ }
             {\Gamma \yields_\cdot \cdot : {\cdot}  } \and 
  \inferrule*[Left = sub2]{
    \Gamma \yields_\theta \Theta : \Delta  \and (\text{where } \gamma \yields \theta : \delta) \\\\
    \Gamma \yields_{\mu} M : A[\Theta] \and (\text{where } \gamma,\delta \yields \mu : \alpha[\theta])}
  {\Gamma \yields_{\theta, \mu/x } (\Theta,M/x) : \Delta, x : A  \and (\text{where } \gamma \yields (\theta,\mu/x) : \delta,x:\alpha)} \\ \\
  
  \inferrule*[Left = var]{
    % \yields \Gamma, x : A, \Gamma' \CTX_{\gamma, x : \alpha, \gamma'}
  }
  {\Gamma, x : A, \Gamma' \yields_x x : A \and (\text{where } \gamma,x:\alpha,\gamma' \yields x : \alpha)} \and

 \inferrule*[Left = 2cell]{
   \Gamma \yields_\mu M : A 
   \and \gamma \yields s : (\nu \Rightarrow \mu) : \alpha
  }
  {\Gamma \yields_\nu s_*(M) : A} \\ \\

  \inferrule*[Left = F-form]{
    \yields_\gamma \Gamma \CTX \and (\text{where } \yields \gamma \ctx)\\\\
    \Gamma \yields_\delta \Delta \TELE \and (\text{where } \gamma \yields \delta \tele) \\\\
    \gamma, \delta \yields \mu : \beta 
  }
  {\Gamma \yields_\beta F_\mu(\Delta) \TYPE \and (\text{where } \gamma \yields \beta \type) } \\
  
  \inferrule*[Left = F-intro]{
    \Gamma \yields_{\theta} \Theta : \Delta 
    \and (\text{where } \gamma \yields {\theta} : \delta)
    %% \and \gamma \yields \nu : \beta 
    %% \and \gamma \yields \mu[\theta] : \beta 
    %% \and \gamma \yields (\nu \Rightarrow \mu[\theta]) : \beta
  }
  {\Gamma \yields_{\mu[\theta]} \FI{\Theta} : F_{\mu}(\Delta) \and (\text{where } \gamma \yields \mu[\theta] : \beta)} \\

  \inferrule*[Left = F-elim]{
    \Gamma, x : F_{\mu}(\Delta) \yields_{\alpha} C : \TYPE \and (\text{where } \gamma, x : \beta \yields \alpha : \type) \and \\
    \Gamma \yields_{\nu} M : F_{\mu}(\Delta) \and (\text{where } \gamma \yields \nu : \beta) \and \\
    \Gamma, \Delta \yields_{\nu' [\mu / x]} N : C [\FI{\Delta/\Delta}/x]
    \and (\text{where } \gamma, \delta \yields \nu' [\mu / x] : \alpha [\mu / x] )}
  {\Gamma \yields_{\nu'[\nu/x]} \FE{M}{\Delta}{N} : C[M/x]  \and (\text{where } \gamma, \beta \yields {\nu'[\nu/x]} : \alpha[\nu/x])} \\
    \\ \\

  \inferrule*[Left = U-form]{
    \Gamma \yields_\delta \Delta \TELE \and (\text{where } \gamma \yields \delta \tele)\\\\
    \Gamma, \Delta \yields_\alpha A \TYPE \and (\text{where } \gamma, \delta \yields \alpha \type)\\\\
    \and \gamma, \delta, c : \beta \yields \mu : \alpha
  }{\Gamma \yields_\beta U_{c.\mu}(\Delta \vert A) \TYPE \and (\text{where } \gamma \yields \beta \type)} \\

  \inferrule*[Left = U-intro]{
    \Gamma, \Delta \yields_{\mu[\nu/c]} N : A \and (\text{where } \gamma,\delta \yields {\mu[\nu/c]} : \alpha)
  }
  {\Gamma \yields_{\nu} \UI \Delta N : U_{c.\mu}(\Delta \vert A)
    \and (\text{where } \gamma \yields \nu : \beta)
  } \\
  
  \inferrule*[Left = U-elim]{
    \Gamma \yields_\nu M : U_{c.\mu}(\Delta \vert A) \and (\text{where } \gamma \yields \nu : \beta)\\\\
    \Gamma \yields_\theta \Theta : \Delta \and (\text{where } \gamma \yields \theta : \delta)\\
  }{
    \Gamma \yields_{\mu[\theta,\nu/c]} \UE{M}{\Theta} : A[\Theta] \and (\text{where } \gamma \yields \mu[\theta,\nu/c] : \alpha[\theta])
  } \\
\end{mathpar}
Note that at the moment in $U$-FORM, $\beta$ can only depend on things in $\gamma$. Not sure if this is the right thing to do?

How do we recover ordinary dependent type theory with $\Pi$ and $\Sigma$? Idea:
\begin{align*}
&\yields p_0 \type \\
x_0 : p_0 &\yields p_1(x_0) \type \\
x_0 : p_0, x_1 : p_1(x_0) &\yields p_2(x_0, x_1) \type \\
&\vdots \\
x_0 : p_0, x_1 : p_1(x_0) &\yields x_0 \times x_1 : p_0 \\
x_0 : p_0, x_1 : p_1(x_0), x_2 : p_2(x_0, x_1) &\yields x_1 \times x_2 : p_1(x_0) \\
&\vdots \\
x_0 : p_0, x_1 : p_1(x_0) &\yields (x_0 \times x_1 \Rightarrow x_0) : p_0 \\
x_0 : p_0, x_1 : p_1(x_0), x_2 : p_2(x_0, x_1) &\yields (x_1 \times x_2 \Rightarrow x_1) : p_1(x_0) \\
&\vdots \\
x_0 : p_0, x_1 : p_1(x_0), x'_0 : p_0,  &\yields w(x'_0) : p_1(x_0) \\
x_0 : p_0, x_1 : p_1(x_0), x_2 : p_2(x_0, x_1), x'_1 : p_1(x_0) &\yields w(x'_1) : p_2(x_0, x_1) \\
&\vdots
\end{align*}
So $p_n \type$ is the mode of types that depend on exactly $n$ previous things. Then $\times$ hopefully gives the correct rules for $\Pi$ and $\Sigma$, and weakening the dependency using $w(-)$ allows not depending on a variable if you don't want to, and $(x \times y \Rightarrow x)$ allows the ordinary kind of weakening in the context.

So how does this recover the coherence triangles for the modal stuff? Suppose we add:
\begin{align*}
x_0 : p_0 &\yields f_0(x_0) : p_0 \\ 
x_0 : p_0, x_1 : p_1(x_0) &\yields f_1(x_0, x_1) : p_1(f_0(x_0)) \\
&\vdots
\end{align*}
(These could land in some different $q_i$ if we wanted). And similarly for $g_i$ and $h_i$. Now suppose we want to construct something like \[x:p_0, y : p_1(f(x)), z : p_2(g(x), h(y))\] so $y$ depends on $x$ through $f$, and $z$ depends on $x$ and $y$ through $g$ and $h$. More carefully, we can only directly get
\begin{align*}
x : p_0, y : p_1(f_0(x)) \yields h_1(f_0(x), y) : p_1(h_0(f_0(x))
\end{align*}
and 
\begin{align*}
x : p_0, y : p_1(g_0(x)) \yields p_2(g_0(x), y) \type
\end{align*}
so to write 
\begin{align*}
x : p_0, y : p_1(f_0(x)) \yields p_2(g_0(x), h_1(f_0(x), y))) \type
\end{align*}
we would need to be able to rewrite a $p_1(h_0(f_0(x))$ to a $p_1(g_0(x))$ using a 2-cell $h_0(f_0(x)) \Rightarrow g_0(x)$

\end{document}

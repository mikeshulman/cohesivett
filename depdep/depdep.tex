\documentclass[10pt]{article}

\usepackage{fullpage}
\usepackage{amssymb,amsthm,bbm}
\usepackage[centertags]{amsmath}
\usepackage[mathscr]{euscript}
\usepackage{tikz-cd}
\usepackage{mathpartir}

\newcommand{\yields}{\vdash}
\newcommand{\cbar}{\, | \,}

\newcommand{\Id}[3]{\mathsf{Id}_{{#1}}(#2,#3)}
\newcommand{\CTX}{\,\,\mathsf{CTX}}
\newcommand{\ctx}{\,\,\mathsf{ctx}}
\newcommand{\TYPE}{\,\,\mathsf{TYPE}}
\newcommand{\type}{\,\,\mathsf{type}}
\newcommand{\TELE}{\,\,\mathsf{TELE}}
\newcommand{\tele}{\,\,\mathsf{tele}}

\newcommand\F[2]{\ensuremath{\mathsf{F}_{#1}(#2)}}
\newcommand\U[3]{\ensuremath{\mathsf{U}_{#1}(#2 \mid #3)}}
\newcommand\FE[3]{\ensuremath{\mathsf{split} \, #2 \, = \, {#1} \, \mathsf{in} \, #3}}
\newcommand\FI[1]{\ensuremath{\mathsf{F}{(#1)}}}
\newcommand\UE[2]{\ensuremath{#1(#2)}}
\newcommand\UI[2]{\ensuremath{\lambda #1.#2}}
\newcommand\Trd[2]{\ensuremath{#1_*(#2)}}

\title{Adjoint Type Theory}
\author{}
\date{}

\begin{document}
\maketitle

Capital things live above, lowercase things live below. Not sure how to do the annotations and keep them legible.

Judgements: 
\begin{itemize}
\item $\yields_\gamma \Gamma \CTX$ lives over $\yields \gamma \ctx$
\item $\Gamma \yields_\alpha A \TYPE$ lives over $\gamma \yields \alpha \type$
\item $\Gamma \yields_\mu M : A$ lives over $\gamma \yields \mu : \alpha$
\item Telescopes $\Gamma \yields_\delta \Delta \TELE$ lying over $\gamma \yields \delta \tele$
\item Substitutions $\Gamma \yields_\theta \Theta : \Delta$ lying over $\gamma \yields \theta : \delta$
\item 2-cells $\gamma \yields s : (\mu \Rightarrow \nu) : \alpha$
\end{itemize}

Note: we could build $s_*$ into the other rules, but not much
reason to in natural deduction.  

Some intended admissible rules:
\begin{mathpar}
\inferrule*[Left = weaken-over]
           {\Gamma,\Gamma' \yields_\mu M : A (\text{where } \gamma,\gamma' \vdash \mu : \alpha)}
           {\Gamma,y:B,\Gamma' \yields_\mu M : A (\text{where } \gamma,y:\beta,\gamma' \vdash \mu : \alpha)}

\inferrule*[Left = subst-over]
           {\Gamma,x:A,\Gamma' \yields_\nu N : C \and (\text{where } \gamma,x:\alpha,\gamma' \vdash \nu : \gamma) \\\\
            \Gamma \vdash_\mu M : A \and (\text{where } \gamma \vdash \mu : \alpha)
           }
           {\Gamma,\Gamma'[M/x] \yields_{\nu[\mu/x]} N[M/x] : A[M/x] \and (\text{where } \gamma,\gamma'[\mu/x] \vdash \nu[\mu/x] : \alpha[\mu/x])}
           
\end{mathpar}

\begin{mathpar}
  \inferrule*[Left = ctx-form]{ }
  {\yields_{\cdot} \cdot \CTX  } \and 

  \inferrule*[Left = ctx-form]{
    \yields_\gamma \Gamma \CTX \and (\text{where } \yields \gamma \ctx) \\\\
    \Gamma \yields_\alpha A \TYPE \and (\text{where }  \gamma \yields \alpha \type)}
  {\yields_{\gamma, x : \alpha} \Gamma, x : A \CTX \and (\text{where } \yields \gamma,x:\alpha \ctx)  } \\

  \inferrule*[Left = tele-form]{ }
             {\Gamma \yields \cdot \TELE_{\cdot}  } \and

  \inferrule*[Left = tele-form]{
    \Gamma \yields_\delta \Delta \TELE \and  (\text{where } \gamma \yields \delta \tele) \\\\
    \Gamma,\Delta \yields_\alpha A \TYPE \and (\text{where } \gamma,\delta \yields \alpha \type)}
  {\Gamma \yields_{\delta, x : \alpha} \Delta, x : A \TELE  \and (\text{where } \gamma \yields \delta,x:\alpha \tele)} \\ \\

  \and

  \inferrule*[Left = sub1]{ }
             {\Gamma \yields_\cdot \cdot : {\cdot}  } \and 
  \inferrule*[Left = sub2]{
    \Gamma \yields_\theta \Theta : \Delta  \and (\text{where } \gamma \yields \theta : \delta) \\\\
    \Gamma \yields_{\mu} M : A[\Theta] \and (\text{where } \gamma,\delta \yields \mu : \alpha[\theta])}
  {\Gamma \yields_{\theta, \mu/x } (\Theta,M/x) : \Delta, x : A  \and (\text{where } \gamma \yields (\theta,\mu/x) : \delta,x:\alpha)} \\ \\
  
  \inferrule*[Left = var]{
    % \yields \Gamma, x : A, \Gamma' \CTX_{\gamma, x : \alpha, \gamma'}
  }
  {\Gamma, x : A, \Gamma' \yields_x x : A \and (\text{where } \gamma,x:\alpha,\gamma' \yields x : \alpha)} \and

 \inferrule*[Left = 2cell]{
   \Gamma \yields_\mu M : A 
   \and \gamma \yields s : (\nu \Rightarrow \mu) : \alpha
  }
  {\Gamma \yields_\nu s_*(M) : A} \\ \\

\text{FIXME: missing the other transport rules from Mike's notes}\\\\

  \inferrule*[Left = F-form]{
    \yields_\gamma \Gamma \CTX \and (\text{where } \yields \gamma \ctx)\\\\
    \Gamma \yields_\delta \Delta \TELE \and (\text{where } \gamma \yields \delta \tele) \\\\
    \gamma, \delta \yields \mu : \beta 
  }
  {\Gamma \yields_\beta F_\mu(\Delta) \TYPE \and (\text{where } \gamma \yields \beta \type) } \\
  
  \inferrule*[Left = F-intro]{
    \Gamma \yields_{\theta} \Theta : \Delta 
    \and (\text{where } \gamma \yields {\theta} : \delta)
    %% \and \gamma \yields \nu : \beta 
    %% \and \gamma \yields \mu[\theta] : \beta 
    %% \and \gamma \yields (\nu \Rightarrow \mu[\theta]) : \beta
  }
  {\Gamma \yields_{\mu[\theta]} \FI{\Theta} : F_{\mu}(\Delta) \and (\text{where } \gamma \yields \mu[\theta] : \beta)} \\

  \inferrule*[Left = F-elim]{
    \Gamma, x : F_{\mu}(\Delta) \yields_{\alpha} C : \TYPE \and (\text{where } \gamma, x : \beta \yields \alpha : \type) \and \\
    \Gamma \yields_{\nu} M : F_{\mu}(\Delta) \and (\text{where } \gamma \yields \nu : \beta) \and \\
    \Gamma, \Delta \yields_{\nu' [\mu / x]} N : C [\FI{\Delta/\Delta}/x]
    \and (\text{where } \gamma, \delta \yields \nu' [\mu / x] : \alpha [\mu / x] )}
  {\Gamma \yields_{\nu'[\nu/x]} \FE{M}{\Delta}{N} : C[M/x]  \and (\text{where } \gamma, \beta \yields {\nu'[\nu/x]} : \alpha[\nu/x])} \\
    \\ \\

  \inferrule*[Left = U-form]{
    \Gamma \yields_\delta \Delta \TELE \and (\text{where } \gamma \yields \delta \tele)\\\\
    \Gamma, \Delta \yields_\alpha A \TYPE \and (\text{where } \gamma, \delta \yields \alpha \type)\\\\
    \and \gamma, \delta, c : \beta \yields \mu : \alpha
  }{\Gamma \yields_\beta U_{c.\mu}(\Delta \vert A) \TYPE \and (\text{where } \gamma \yields \beta \type)} \\

  \inferrule*[Left = U-intro]{
    \Gamma, \Delta \yields_{\mu[\nu/c]} N : A \and (\text{where } \gamma,\delta \yields {\mu[\nu/c]} : \alpha)
  }
  {\Gamma \yields_{\nu} \UI \Delta N : U_{c.\mu}(\Delta \vert A)
    \and (\text{where } \gamma \yields \nu : \beta)
  } \\
  
  \inferrule*[Left = U-elim]{
    \Gamma \yields_\nu M : U_{c.\mu}(\Delta \vert A) \and (\text{where } \gamma \yields \nu : \beta)\\\\
    \Gamma \yields_\theta \Theta : \Delta \and (\text{where } \gamma \yields \theta : \delta)\\
  }{
    \Gamma \yields_{\mu[\theta,\nu/c]} \UE{M}{\Theta} : A[\Theta] \and (\text{where } \gamma \yields \mu[\theta,\nu/c] : \alpha[\theta])
  } \\
\end{mathpar}
Note that at the moment in $F$-FORM/$U$-FORM, $\beta$ can only depend on things in $\gamma$. Not sure if this is the right thing to do?

\subsection*{Mode Theories}

Here's an idea for how mode theories could work.

\paragraph{Contexts}

In the simply-typed framework, to make a mode $p$ have (tensor)
products, we made it an internal monoid with e.g. a 1-cell $\odot : p,p
\to p$.  Here, to make a mode $p$ ``have dependent types,'' we'll make
it an ``internal display category'' or something like that.

For normal dependent type theory, we'll use two base mode types $p_0
\type$ and $c : p_0 \yields p_1(c) \type$.  Think of $p_0$ as the closed
types/contexts associated with mode $p$, and $p_1(c)$ as the dependent
types (dependent on $c$).  The mode theory says that ``contexts'' are
generated by the empty context, context extension, and projection:
\begin{align*}
&\yields \emptyset_0 : p_0 \\
c : p_0, x : p_1(c) &\yields c.x : p_0 \\
c : p_0, x : p_1(c) & \yields w : (c.x) \Rightarrow c : p_0 \\
\end{align*}

A typical object-language type formation judgement like 
\[
x : A, y : B, z : C \vdash D \type
\]
is over
\[
x : p_1(\emptyset_0), y : p_1(\emptyset.x), z : p_1(\emptyset.x.y) \vdash p_1(\emptyset.x.y.z) \type
\]
(``$x$ can't use anything, $y$ can use $x$, $z$ can use $x$ and $y$,'' etc.) 

I'm assuming that weakening acts \emph{contravariantly} in the base here
(in Mike's notes, why is the variance different in the base and on the
subscripts of the top?); could flip if that's wrong:
\[
c : p_0, x : p_1(c), y : p_1(c) \vdash (w_x)^*(y) : p_1(c.x)
\]

\paragraph{$\Sigma$ types}
The $F$-type for context extension is a ``closed'' $\Sigma$-type:
$\yields_{p_0} \F{x.y}{x : A, y : B}$, if $\yields_{p_0} A \type$ and $x
: A \vdash_{p_1(x)} B \type$ ($A$ is a closed type and $B$ depends only
on it).  To get a $\Sigma$ in context, we need another mode morphism
\[
c : p_0, x : p_1(c), y : p_1(c.x) \yields \Sigma_c(x,y) : p_1(c) 
\]
and then we have $\vdash_{p_1(c)} \F{\Sigma_c(x,y)}{x : A, y : B} \type$
when $\yields_{p_1(c)} A \type$ and $x : A \vdash_{p_1(c.x)} B \type$
relative to an ambient context in which $c : p_0$.

We'll probably want various equations about these, such as 
associativity
\begin{align*}
c.\Sigma_c(x,y) &\equiv c.x.y \\
\Sigma_c(\Sigma_c(x,y),z) & \equiv \Sigma_c(x,\Sigma_{c.x}(y,z))
\end{align*}
(where the second uses the first to type check)
and that there is a unit 
\[
\yields \emptyset_1 : p_1(\emptyset_0) \\
\]
that has unit laws with $\Sigma$:
\begin{align*}
c.(w_c^*(\emptyset_1)) \equiv c\\
c : p_0, x : p_1(c) \yields \Sigma_c(w_c^*(\emptyset_1),x) \equiv x\\
c : p_0, x : p_1(c) \yields \Sigma_c(x,w_{c.x}^*(\emptyset_1)) \equiv x
\end{align*}

We'll also need that weakenings/projections interact appropriately with
it (maybe this kind of naturality is a general principle?), like:
\[
c : p_0, z : p_1(c), x : p_1(c), y : p_1(c.x) \yields
(w_z)^*(\Sigma_c(x,y)) \equiv 
\Sigma_{c.z}((w_z)^* x,(w_z)^* y)
 : p_1(c.z) \\
\]

And we'll usually want contraction, e.g.
\[
c : p_0, x : p_1(c) \yields \Sigma_{c}(x,w_x^*(x)) \equiv x
\]
(or a directed cell if that's preferable).  

\paragraph{Modalities}

Suppose we have two modes $(p_0,p_1)$ and $(q_0,q_1)$ as above, then an
example modality between them has both 
\begin{align*}
c : p_0 \vdash f_0(c) : q_0\\
c : p_0, x : p_1(c) \vdash f_1(c,x) : q_1(f_0(c))
\end{align*}
$f_0$ says that the modality $f$ acts on contexts/closed types, and
$f_1$ says that the modality also acts on open types (we could omit
$f_1$ if it doesn't), and that the action on types depends through $f_0$
(but of course there's nothing that forces this; it just seems like a
common pattern).

For example, for a Pfenning-Davies style $\Box$ decomposed as
$F(U A)$ between modes $v$ and $t$, the $F$ and $U$ types for $f_1$ give
have formation rules like
\begin{mathpar}
\inferrule*{\Delta \yields A \type_v}
           {\Delta \mid \cdot \yields F(A) \type_t}

\inferrule*{\Delta \mid \cdot \yields A \type_t}
           {\Delta \yields U(A) \type_v}
\end{mathpar}
where $f_0(\Delta)$ is $\Delta \mid \cdot$

We could add product-preserving axioms like:
\begin{align*}
f_0(\emptyset_0) \equiv \emptyset_0\\
f_1(\emptyset_1) \equiv \emptyset_1\\
f_0(c.x) \equiv f_0(c).f_1(x) \\
f_1(\Sigma_c(x,y)) \equiv \Sigma_{f_0(c)}(f_1(x), f_1(y)) \\
\end{align*}

So how does this recover the coherence triangles for the modal stuff? Suppose we have:
\begin{align*}
x_0 : p_0 &\yields f_0(x_0) : p_0 \\ 
x_0 : p_0, x_1 : p_1(x_0) &\yields f_1(x_0, x_1) : p_1(f_0(x_0)) \\
\end{align*}
(These could land in some different $q_i$ if we wanted). And similarly
for $g_i$ and $h_i$. Now suppose we want to construct something
like \[x:p_0, y : p_1(f(x)), z : p_2(g(x), h(y))\] so $y$ depends on $x$
through $f$, and $z$ depends on $x$ and $y$ through $g$ and $h$. More
carefully, we can only directly get
\begin{align*}
x : p_0, y : p_1(f_0(x)) \yields h_1(f_0(x), y) : p_1(h_0(f_0(x))
\end{align*}
and 
\begin{align*}
x : p_0, y : p_1(g_0(x)) \yields p_1(g_0(x).y) \type
\end{align*}
so to write 
\begin{align*}
x : p_0, y : p_1(f_0(x)) \yields p_1(g_0(x).h_1(f_0(x), y))) \type
\end{align*}
we would need to be able to rewrite a $p_1(h_0(f_0(x))$ to a $p_1(g_0(x))$ using a 2-cell $h_0(f_0(x)) \Rightarrow g_0(x)$

\paragraph{$\Pi$-types}

For $\Pi$ types, $\U{z.\Pi_c(z,x)}{x : A}{B}$ for something of the
following type encodes the usual formation rule:
\[ 
c : p_0, x : p_1(c), z : p_1(c) \yields \Pi_c(z,x) : p_1(c.x) 
\]

For example, we can take $\Pi_c(z,x)$ to be ``$z \times x$, weakened
with $x$'':
\[
\Pi_c(z,x) := w_x(\Sigma_c(z,w_z(x)))
\]
Whether this is a correct 1-cell to use depends on the encoding of term
judgements; see the discussion below.

\paragraph{Base Types}

I'm assuming that base types (and identity types, $\mathsf{El}(a)$ for
$a : U$, etc.) ``declare their variables'', e.g. $x : P, y : Q
\vdash_{p_1(\emptyset.x.y)} R(x,y) \type$.  Otherwise, the way that
$\Sigma$, $\Pi$ propogate contexts ($p_0$) wouldn't actually stop
anything from type checking, because you could make all of your types
elements of $p_1(\emptyset)$, which can be weakened to whatever you
want.

\paragraph{Term Judgements}  

The above discussion seems to give some reasonable type formation
rules.  To see if the usual intro/elim rules are derivable, we need to
first specify how a term judgement is represented. 

A typical judgement like 
\[
x : A_{p_1(\emptyset)}, y : B_{p_1(\emptyset.x)}, z : C_{p_1(\emptyset.x.y)} \vdash_{\mu} d : D_{p_1(\emptyset.x.y.z)}
\]
needs to be over a term 
\[
x : p_1(\emptyset), y : p_1(\emptyset.x), z : p_1(\emptyset.x.y) \vdash \mu : p_1(\emptyset.x.y.z) 
\]

It seems like there are a couple of different choices for what could go
here:
\begin{itemize}

\item $w_{\emptyset.x.y.z}^*(\Sigma_{\emptyset}(x,\Sigma_{\emptyset.x}(y,z)))$,
where the depenency of $y$ on $x$, and $z$ on $y$, is internal to the
term, and the variables that can occur in the type are weakened  

\item $\Sigma_{\emptyset.x.y.z}(w^*(x), \Sigma_{\emptyset.x.y.z.w^*(x)}(w^*(y), w^*(z)))$
where the dependencies of $x,y,z$ all point ``outside'', e.g. 
the weakening of $y$ is $w_{\emptyset.x \Rightarrow \emptyset.x.y.w^*(x)}^*(y)$,
given by projecting away the final $y$ and $w^*(x)$, rather than adding
the middle $x,y$.  

\end{itemize}

It feels like the standard practice is to have 
\[
x : p_1(\emptyset), y : p_1(\emptyset.x), z : p_1(\emptyset.x.y) \vdash \mu : p_0
\]
instead of $p_1$, thinking of the term subscripting a turnstile as
representing a \emph{context}, rather than a type, and these two choices
are artifacts of the fact that we have $p_1(\emptyset.x.y.z)$ as the
goal instead.  In that case $\emptyset.x.y.z$ would be the obvious
choice for $\mu$, and both of the above are (equivalent?) ways of
weakening this context to a type.

If we equate/relate these two (which seems reasonable with contraction),
I think the $F/U$ types instantiated as above will give the usual rules
for $\Sigma$ and $\Pi$ (though I haven't checked things in detail,
especially regarding substitution, which will need the 
transport-over-transport rule from Mike's notes).  

It would be good to have an example of an unusual dependent type system
where the extra flexibility is helpful, or else it seems like something
might be wrong with the setup.  What it's letting you do is to have a
type depend on some variables, in a term that is not allowed to use all
of those variables.  For example, the subscript
\[
x : p_1(\emptyset), y : p_1(\emptyset.x), z : p_1(\emptyset.x.y) \vdash w_{z}^*(z) : p_1(\emptyset.x.y.z) 
\]
would have the type dependency as above, but allow the term to only use
$z$, not $x,y$.  

Some other questions:
\begin{itemize}

\item In this setup, the way that the type uses variables is completely
  decoupled (in terms of what judgements can be \emph{stated}) from the
  way the term uses the variables.  For example, supposing two
  modalities,
  \begin{align*}
    c : p_0 \vdash f_0(c) : p_0 \\
    c : p_0 \vdash g_0(c) : p_0 \\
    c : p_0, x : p_1(c) \vdash f_1(x) : p_1(f_0(c)) \\
    c : p_0, x : p_1(c) \vdash g_1(x) : p_1(g_0(c))
  \end{align*}
  then
  \[
  c : p_0, x : p_1(c), y : p_1(c.x) \vdash w_{\emptyset \Rightarrow
    (\emptyset.g_1(x).g_1(y))}^*(\Sigma_{\emptyset}(f_1(x),f_1(y))) :
  p_1(\emptyset.g_1(x).g_1(y))
  \]
  would mean ``the type uses $x$ and $y$ through $f$, but the term uses
  them through $g$''.  Is this a feature or a bug?  
  
\item The setup is compatible with contraction-free dependent types; can
  we do something useful with that?

\end{itemize}

\end{document}

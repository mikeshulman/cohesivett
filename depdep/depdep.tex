\documentclass[10pt]{article}

\usepackage{amssymb,amsthm,bbm}
\usepackage[centertags]{amsmath}
\usepackage[mathscr]{euscript}
\usepackage{tikz-cd}
\usepackage{mathpartir}

\newcommand{\yields}{\vdash}
\newcommand{\cbar}{\, | \,}

\newcommand{\Id}[3]{\mathsf{Id}_{{#1}}(#2,#3)}
\newcommand{\CTX}{\,\,\mathsf{CTX}}
\newcommand{\ctx}{\,\,\mathsf{ctx}}
\newcommand{\TYPE}{\,\,\mathsf{TYPE}}
\newcommand{\type}{\,\,\mathsf{type}}
\newcommand{\TELE}{\,\,\mathsf{TELE}}
\newcommand{\tele}{\,\,\mathsf{tele}}

\title{Dependent Adjoint Logic}
\author{}
\date{}

\begin{document}
\maketitle

Capital things live above, lowercase things live below. Not sure how to do the annotations and keep them legible.

Judgements: 
\begin{itemize}
\item $\yields \Gamma \CTX_\gamma$ lives over $\yields \gamma \ctx$
\item $\Gamma \yields_\alpha A \TYPE$ lives over $\gamma \yields \alpha \type$
\item $\Gamma \yields_\mu M : A$ lives over $\gamma \yields \mu : \alpha$
\item Telescopes $\Gamma \yields_\delta \Delta \TELE$ lying over $\gamma \yields \delta \tele$
\item Substitutions $\Gamma \yields_\theta \Theta : \Delta$ lying over $\gamma \yields \theta : \delta$
\item 2-cells $\gamma \yields (\mu \Rightarrow \nu) : \alpha$
\end{itemize}

Rules so far:
\begin{mathpar}
  \inferrule*[Left = ctx-form]{
    \yields \Gamma \CTX_\gamma 
    \and \gamma \yields \alpha \type
  	\and \Gamma \yields_\alpha A \TYPE }
  {\yields \Gamma, x : A \CTX_{\gamma, x : \alpha}  } \\
  
  \inferrule*[Left = var]{
    \yields \Gamma, x : A, \Gamma' \CTX_{\gamma, x : \alpha, \gamma'}
    \and \gamma, x : \alpha, \gamma' \yields (\mu \Rightarrow x) : \alpha
  }
  {\Gamma, x : A, \Gamma' \yields_\mu x : A} \\  

  \inferrule*[Left = F-form]{
    \yields \Gamma \CTX_\gamma 
    \and \gamma, \delta \yields \mu : \beta 
    \and \Gamma \yields_\delta \Delta \TELE}
  {\Gamma \yields_\beta F_\mu(\Delta) \TYPE  } \\
  
  \inferrule*[Left = F-left]{
    \Gamma, \Delta \yields_{\nu [\mu / x]} N : C [FR(\Delta)/x] 
    \and (\text{where } \gamma, \delta \yields \nu [\mu / x] : \alpha [\mu / x] )}
  {\Gamma, x : F_\mu(\Delta) \yields_\nu \mathrm{let...} : C  \and (\text{where } \gamma, \beta \yields \nu : \alpha)} \\
  
  \inferrule*[Left = F-right]{
    \Gamma \yields_{\theta} \Theta : \Delta 
    \and \gamma \yields \nu : \beta 
    \and \gamma \yields \mu[\theta] : \beta 
    \and \gamma \yields (\nu \Rightarrow \mu[\theta]) : \beta}
  {\Gamma \yields_\nu ... : F_{\mu}(\Delta) \and (\text{where } \gamma \yields \nu : \beta)} 
  
  \inferrule*[Left = U-form]{
    \Gamma \yields \Delta \TELE 
  	\and \Gamma, \Delta \yields A \TYPE
  	\and \gamma \yields \beta \type
  	\and \gamma, \delta, c : \beta \yields \mu : \alpha
  	}{\Gamma \yields_\beta U_{c.\mu}(\Delta \vert A) \TYPE} \\

  \inferrule*[Left = U-intro]{
    \Gamma, \Delta \yields_{\mu[\nu/c]} A}
  {\Gamma \yields_{\nu} (\lambda x.\, ...) : U_{c.\mu}(\Delta \vert A)} \\
  
  \inferrule*[Left = U-elim??]{
    \Gamma \yields_\nu M : U_{c.\mu}(\Delta \vert A)
    \and \Gamma \yields_\theta \Theta : \Delta
    \and \gamma \yields (\mu' \Rightarrow \mu[\nu/c][\theta]) : \alpha[\theta]
  }{
    \Gamma \yields_{\mu'} M \Theta : A[\Theta] 
  } \\
\end{mathpar}
Note that at the moment in $U$-FORM, $\beta$ can only depend on things in $\gamma$. Not sure if this is the right thing to do?

Some intended admissible rules:
\begin{mathpar}
 \inferrule*{
   \Gamma \yields_\mu M : A 
   \and \gamma \yields s : (\nu \Rightarrow \mu)
  }
  {\Gamma \yields_\nu s_*(M) : A} \\  
\end{mathpar}

How do we recover ordinary dependent type theory with $\Pi$ and $\Sigma$? Idea:
\begin{align*}
&\yields p_0 \type \\
x_0 : p_0 &\yields p_1(x_0) \type \\
x_0 : p_0, x_1 : p_1(x_0) &\yields p_2(x_0, x_1) \type \\
&\vdots \\
x_0 : p_0, x_1 : p_1(x_0) &\yields x_0 \times x_1 : p_0 \\
x_0 : p_0, x_1 : p_1(x_0), x_2 : p_2(x_0, x_1) &\yields x_1 \times x_2 : p_1(x_0) \\
&\vdots \\
x_0 : p_0, x_1 : p_1(x_0) &\yields (x_0 \times x_1 \Rightarrow x_0) : p_0 \\
x_0 : p_0, x_1 : p_1(x_0), x_2 : p_2(x_0, x_1) &\yields (x_1 \times x_2 \Rightarrow x_1) : p_1(x_0) \\
&\vdots \\
x_0 : p_0, x_1 : p_1(x_0), x'_0 : p_0,  &\yields w(x'_0) : p_1(x_0) \\
x_0 : p_0, x_1 : p_1(x_0), x_2 : p_2(x_0, x_1), x'_1 : p_1(x_0) &\yields w(x'_1) : p_2(x_0, x_1) \\
&\vdots
\end{align*}
So $p_n \type$ is the mode of types that depend on exactly $n$ previous things. Then $\times$ hopefully gives the correct rules for $\Pi$ and $\Sigma$, and weakening the dependency using $w(-)$ allows not depending on a variable if you don't want to, and $(x \times y \Rightarrow x)$ allows the ordinary kind of weakening in the context.

So how does this recover the coherence triangles for the modal stuff? Suppose we add:
\begin{align*}
x_0 : p_0 &\yields f_0(x_0) : p_0 \\ 
x_0 : p_0, x_1 : p_1(x_0) &\yields f_1(x_0, x_1) : p_1(f_0(x_0)) \\
&\vdots
\end{align*}
(These could land in some different $q_i$ if we wanted). And similarly for $g_i$ and $h_i$. Now suppose we want to construct something like \[x:p_0, y : p_1(f(x)), z : p_2(g(x), h(y))\] so $y$ depends on $x$ through $f$, and $z$ depends on $x$ and $y$ through $g$ and $h$. More carefully, we can only directly get
\begin{align*}
x : p_0, y : p_1(f_0(x)) \yields h_1(f_0(x), y) : p_1(h_0(f_0(x))
\end{align*}
and 
\begin{align*}
x : p_0, y : p_1(g_0(x)) \yields p_2(g_0(x), y) \type
\end{align*}
so to write 
\begin{align*}
x : p_0, y : p_1(f_0(x)) \yields p_2(g_0(x), h_1(f_0(x), y))) \type
\end{align*}
we would need to be able to rewrite a $p_1(h_0(f_0(x))$ to a $p_1(g_0(x))$ using a 2-cell $h_0(f_0(x)) \Rightarrow g_0(x)$

\end{document}
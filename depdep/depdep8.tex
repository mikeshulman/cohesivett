\documentclass[10pt]{article}
  \usepackage{xcolor}
  \definecolor{darkgreen}{rgb}{0,0.45,0} 
  \usepackage[pagebackref,colorlinks,citecolor=darkgreen,linkcolor=darkgreen]{hyperref}
  \usepackage{pdflscape}

\usepackage{fullpage}
\usepackage{amssymb,amsthm,bbm}
\usepackage[centertags]{amsmath}
\usepackage[mathscr]{euscript}
\usepackage{dsfont}
\usepackage{fontawesome}
\usepackage{tikz-cd}
\usepackage{mathpartir}
\usepackage{enumitem}
\usepackage[status=draft,inline,nomargin]{fixme}
\FXRegisterAuthor{ms}{anms}{\color{blue}MS}
\FXRegisterAuthor{mvr}{anmvr}{\color{olive}MVR}
\FXRegisterAuthor{drl}{andrl}{\color{purple}DRL}
\usepackage{stmaryrd}
\usepackage{mathtools}

\newtheorem{theorem}{Theorem}
\newtheorem{proposition}{Proposition}
\newtheorem{lemma}{Lemma}
\newtheorem{corollary}{Corollary}
\newtheorem{problem}{Problem}
\newenvironment{constr}{\begin{proof}[Construction]}{\end{proof}}

\theoremstyle{definition}
\newtheorem{definition}{Definition}
\newtheorem{remark}{Remark}
\newtheorem{example}{Example}

\let\oldemptyset\emptyset%
\let\emptyset\varnothing

\newcommand\dsd[1]{\ensuremath{\mathsf{#1}}}

\newcommand{\yields}{\vdash}
\newcommand{\Yields}{\tcell}
\newcommand{\tcell}{\Rightarrow}
\newcommand{\cbar}{\, | \,}
\newcommand{\judge}{\mathcal{J}}

\newcommand{\Id}[3]{\mathsf{Id}_{{#1}}(#2,#3)}
\newcommand{\CTX}{\,\,\mathsf{Ctx}}
\newcommand{\ctx}{\,\,\mathsf{mctx}}
\newcommand{\TYPE}{\,\,\mathsf{Type}}
\newcommand{\type}{\,\,\mathsf{mode}}
\newcommand{\TELE}{\,\,\mathsf{Tele}}
\newcommand{\tele}{\,\,\mathsf{mtele}}

\newcommand{\app}[2]{\ensuremath{#1 \: #2}}
\newcommand{\telety}[3]{\ensuremath{(#1{:}#2,#3)}}
\newcommand{\mt}[0]{\ensuremath{()}}
\newcommand{\sigmacl}[3]{\ensuremath{(#1{:}#2,#3)}}
\newcommand{\fst}[1]{\app{\dsd{fst}}{#1}}
\newcommand{\snd}[1]{\app{\dsd{snd}}{#1}}
\newcommand\extend[2]{\ensuremath{(#1,\id_{#2})}}
\newcommand\TeleE[4]{\ensuremath{\mathsf{let} \, (#2, #3) \, = \, {#1} \, \mathsf{in} \, #4}}

\newcommand{\id}{\mathsf{id}}
\DeclareMathOperator{\ob}{ob}

\newcommand{\rewrite}[2]{\overleftarrow{#1}(#2)}
\newcommand\F[2]{\ensuremath{\mathsf{F}_{#1}(#2)}}
\newcommand\U[3]{\ensuremath{\mathsf{U}_{#1}(#2 \mid #3)}}
\newcommand\UE[2]{\ensuremath{#1(#2)}}
\newcommand\UI[2]{\ensuremath{\lambda #1.#2}}
\newcommand\St[2]{\ensuremath{{#1}^*(#2)}}
\newcommand\StI[2]{\ensuremath{\mathsf{st}_{#1}(#2)}}
\newcommand\UStI[2]{\ensuremath{\mathsf{ust}_{#1}(#2)}}
\newcommand\UnSt[2]{\ensuremath{\mathsf{unst}_{#1}(#2)}}
%\newcommand\StE[2]{\ensuremath{\mathsf{unst}(#1,#2)}}
\newcommand\StE[4]{\ensuremath{\mathsf{let} \, \StI{#1}{#3} \, = \, {#2} \, \mathsf{in} \, #4}}
\newcommand\FE[3]{\ensuremath{\mathsf{let} \, \mathsf{F}(#2) \, = \, {#1} \, \mathsf{in} \, #3}}
% With subscript:
\newcommand\FEs[4]{\ensuremath{\mathsf{let} \, \mathsf{F}_{#1}(#3) \, = \, {#2} \, \mathsf{in} \, #4}} 
\newcommand\FI[1]{\ensuremath{\mathsf{F}{(#1)}}}
\newcommand\FIs[2]{\ensuremath{\mathsf{F}_{#1}{(#2)}}}
\newcommand\TypeTwo[4]{\ensuremath{#1 \vdash #2 :  #3 \tcell #4}}
\newcommand\TeleTwo[4]{\ensuremath{#1 \vdash #2 : #3 \tcell #4}}
\newcommand\TermTwo[4]{\ensuremath{#1 \vdash #2 : #3 \tcell #4}}
\newcommand\TermTwoT[5]{\ensuremath{#1 \vdash {#2} : #3 \tcell_{#5} #4}}
%% \newcommand\TermTwoDisp[5]{\ensuremath{#1 \mid #3 \tcell_{\mathsf{disp}} #2 :_{#5} #4}}
%\newcommand\SubTwo[4]{\ensuremath{#1 \mid #3 \tcell #2 : #4}}
\newcommand\TrPlus[2]{\ensuremath{{#1}^+(#2)}}
\newcommand\TrCirc[2]{\ensuremath{{#1}^\circ(#2)}}

\newcommand\El[2]{\mathcal{T}_{#1}(#2)}
\newcommand\ApEl[2]{\mathcal{T}_{#1}\langle#2\rangle}
\newcommand\bdot[0]{\mathbin{.}}
\newcommand\bang[0]{\mathord{!}}

\newcommand\ap[2]{\ensuremath{#1 \langle #2 \rangle }}
\newcommand\ApPlus[2]{\ensuremath{{#1}^+ \langle #2 \rangle }}
\newcommand\ApCirc[2]{\ensuremath{{#1}^\circ \langle #2 \rangle }}

% Lemmas
\newcommand\ctxtuple[1]{(#1)}
\newcommand\pack[1]{\ensuremath{\mathsf{pack}_{#1}}}
\newcommand\unpack[2]{\ensuremath{\mathsf{unpack}_{#1}(#2)}}

% MLTT
\newcommand{\modeof}[1]{{#1}_p}
\newcommand{\modeofq}[1]{{#1}_q}
\newcommand{\tdot}{\ensuremath{\mathtt{dot}}}
\newcommand{\tempty}{\ensuremath{\mathtt{empty}}}
\newcommand{\tshape}[1]{\ensuremath{\mathtt{shape}_{#1}}}

\newcommand{\qyields}{\Vdash} 
\newcommand{\upstairs}[1]{\overline{#1}}
\newcommand{\downstairs}[1]{\underline{#1}}
\newcommand\proj[1]{\ensuremath{\mathsf{proj}_{#1}}}
\newcommand\qvar[1]{\ensuremath{\mathsf{var}_{#1}}}

\newcommand\One{\ensuremath{\mathds{1}}}
\newcommand\var[1]{\ensuremath{\mathtt{var}_{#1}}}
\newcommand\ApOne[1]{\ensuremath{\One_{\langle {#1} \rangle }}}

\newcommand\mtt[1]{\mathtt{#1}}
\newcommand\contract[1]{\ensuremath{\mathtt{contract}_{#1}}}
\newcommand\fibpair[1]{\ensuremath{\mathtt{fibpair}_{#1}}}
\newcommand\pair[1]{\ensuremath{\mathtt{pair}_{#1}}}
\newcommand\tsplit[1]{\ensuremath{\mathtt{split}_{#1}}}
\newcommand\pinv[1]{\ensuremath{\mathtt{pinv}_{#1}}}

\newcommand\qunitmatch[1]{\ensuremath{\mathsf{letunit}(#1)}}
\newcommand\qpair[1]{\ensuremath{\mathsf{pair}_{#1}}}
\newcommand\qsplit[1]{\ensuremath{\mathsf{split}_{#1}}}
\newcommand\qapp[1]{\ensuremath{\mathsf{app}({#1})}}
\newcommand\qlam[1]{\ensuremath{\mathsf{lam}({#1})}}

% Adjoint type theory
\newcommand\fone[1]{\ensuremath{\mathtt{fone}_{#1}}}
\newcommand\fibf[1]{\ensuremath{\mathtt{fibf}_{#1}}}
\newcommand\foneinv[1]{\ensuremath{\fone{#1}^{-1}}}
\newcommand\fdist[1]{\ensuremath{\mathtt{fdist}_{#1}}}
\newcommand\fdistinv[1]{\ensuremath{\fdist{#1}^{-1}}}

\newcommand\flatone[1]{\ensuremath{\mathtt{flatone}_{#1}}}
\newcommand\flatdist[1]{\ensuremath{\mathtt{flatdist}_{#1}}}
\newcommand\flatdistinv[1]{\ensuremath{\flatdist{#1}^{-1}}}

\newcommand{\lock}{\text{\faUnlock}}
\newcommand{\Rtype}[1]{\mathsf{R}{#1}}
\newcommand{\RI}[1]{\mathsf{shut}({#1})}
\newcommand{\RE}[1]{\mathsf{open}({#1})}

\newcommand{\Ltype}[1]{\mathsf{L}{#1}}
\newcommand{\LI}[1]{\mathsf{left}_{#1}}
\newcommand{\LE}[1]{\mathsf{letleft}({#1})}

% Spatial type theory
\newcommand\fcomult[1]{\ensuremath{\mathtt{comult}_{#1}}}
\newcommand\fcounit[1]{\ensuremath{\mathtt{counit}_{#1}}}
\newcommand{\counit}[1]{\mathsf{counit}_{#1}}
\newcommand{\comult}[1]{\mathsf{comult}_{#1}}
\newcommand{\Flattype}[1]{\flat{#1}}
\newcommand{\FlatI}[1]{{#1}^\flat}
\newcommand{\FlatE}[1]{\mathsf{letflat}({#1})}
\newcommand{\Sharptype}[1]{\sharp{#1}}
\newcommand{\SharpI}[1]{{#1}^\sharp}
\newcommand{\SharpE}[1]{{#1}_\sharp}
\newcommand\qcrispvar[1]{\ensuremath{\textsf{crisp-var}_{#1}}}

% Linear zone
\newcommand{\tfibshape}[1]{\ensuremath{\mathtt{fibshape}_{#1}}}
\newcommand{\linsnd}[1]{\mathtt{linsnd}_{#1}}
\newcommand{\linwk}[1]{\mathtt{linwk}_{#1}}
\newcommand{\frob}[1]{\mathtt{frob}_{#1}}
\newcommand\qlinvar[1]{\ensuremath{\mathsf{linvar}_{#1}}}
\newcommand\otimespair[1]{\ensuremath{\otimes\mathsf{pair}_{#1}}}
\newcommand\otimessplit[1]{\ensuremath{\otimes\mathsf{split}({#1})}}
\newcommand\linpair[1]{\ensuremath{\mathsf{linpair}_{#1}}}
\newcommand\linsplit[1]{\ensuremath{\mathsf{linsplit}({#1})}}
\newcommand\linapp[1]{\ensuremath{\mathsf{linapp}_{#1}}}
\newcommand\linlam{\ensuremath{\mathsf{linlam}}}
\newcommand{\qbang}[1]{\ensuremath{\mathsf{bang}_{#1}}}
\newcommand{\letbang}[1]{\mathsf{letbang}({#1})}

% Macros for semantics notation
\newcommand\mm[1]{\llbracket #1 \rrbracket}
\newcommand\op{^{\mathrm{op}}}
\newcommand\co{^{\mathrm{co}}}
\newcommand\coop{^{\mathrm{coop}}}
\newcommand\Cat{\mathrm{Cat}}
\newcommand\CAT{\mathrm{CAT}}
\newcommand\M{\mathcal{M}}
\newcommand\Mhat{\widehat{\mathcal{M}}}
\newcommand\Mty{{\mathrm{Ty}_{\M}}}
\newcommand\Mtm{{\mathrm{Tm}_{\M}}}
\newcommand\Mtyhat{{\widehat{\mathrm{Ty}}_{\M}}}
\newcommand\Mtmhat{{\widehat{\mathrm{Tm}}_{\M}}}
\newcommand\Ups{\Upsilon}
\newcommand\Upshat{{\widehat{\Upsilon}}}
\newcommand\C{\mathcal{C}}
\newcommand\Chat{{\widehat{\mathcal{C}}}}
\newcommand\Cty{\mathrm{Ty}_{\C}}
\newcommand\Ctm{\mathrm{Tm}_{\C}}
\newcommand\Ctyhat{{\widehat{\mathrm{Ty}}}_{\C}}
\newcommand\Ctmhat{{\widehat{\mathrm{Tm}}}_{\C}}
\newcommand\vp{\varpi}
\newcommand\vpst{\vp^*}
\newcommand\vpsh{\vp_!}
\newcommand\vptil{\widetilde{\vp}}
\newcommand\vpty{{\vp}_{\mathrm{Ty}}}
\newcommand\vptm{{\vp}_{\mathrm{Tm}}}
\newcommand\name[1]{\ulcorner #1\urcorner}
\newcommand{\Util}{\widetilde{U}}
\newcommand\ev{\mathrm{ev}}
\DeclareSymbolFont{bbold}{U}{bbold}{m}{n}
\DeclareSymbolFontAlphabet{\mathbbb}{bbold}
\newcommand\one{\mathbbb{1}}

\title{A Fibrational Framework for \\ Substructural and Modal Dependent Type Theories}
\author{Daniel R. Licata, Mitchell Riley, Michael Shulman}
\date{}

\begin{document}
\maketitle
\tableofcontents

\begin{abstract}
Recently, several modal extensions of homotopy type theory have been
investigated, with the goal of extending the synthetic style of
formalizing mathematics to additional situations.  For example,
real-cohesive homotopy type theory can describe types with both a
groupoid structure and a separate topological structure.  These modal
dependent type theories add new type operators to the syntax, which
typically are given universal properties relative to new judgement
forms.  To facilitate the design of such type theories, we introduce a
general framework for modal dependent type theories, building on our
previous work for simple type theories.  The framework consists first of
a base directed dependent type theory, which serves as a language for
specifying a signature of desired modalities, which we call a mode
theory.  This mode theory is the parameter to a second type theory,
which gives general rules for working with the modalities it describes.
The mode theory language is flexible enough to describe a variety of
modalities, including adjunctions, monads, comonads, idempotent
(co)monads, and so on; as examples, we give mode theories for ordinary
non-modal dependent type theory with $\Pi$ and $\Sigma$ types, for a
dependent adjoint pair of modalities, and for the spatial type theory
used in real-cohesion.  One advantage of our framework is that we can
give it a categorical semantics for all mode theories at once, which
saves some of the effort involved in translating each type theory
individually, and we describe a category-with-families-like semantics.
While the framework does not automatically produce ``optimized''
inference rules for a particular modal discipline (where structural
rules are as admissible as possible), it does provide a convenient
syntactic setting for investigating such issues, including a general
equational theory governing the placement of structural rules in types
and in terms.
\end{abstract}

\section{Syntax}

\subsection{Overview of Judgements}

Mode theory judgements:
\begin{enumerate}
\item $\gamma \ctx$ (empty, extension)
\item $\gamma \yields p \type$ 
\item $\TypeTwo{\gamma}{s}{p}{q}$ (horizontal and vertical composition, identities)
\item $\gamma \yields \mu : p$ (variables, action of mode type morphisms)
\item $\TermTwoT{\gamma}{s}{\mu}{\nu}{p}$ (horizontal and vertical
    composition, identities)
\end{enumerate}

Top judgements: 
\begin{itemize}
\item $\yields_\gamma \Gamma \CTX$ over $\yields \gamma \ctx$
\item $\Gamma \yields_p A \TYPE$ over $\gamma \yields p \type$
\item $\Gamma \yields_\mu M : A$ over $\gamma \yields \mu : p$
\end{itemize}

Mode type morphisms $\TypeTwo{\gamma}{s}{p}{q}$ induce 1-cells contravariantly in the mode theory and 
$\TypeTwo{\gamma}{s}{p}{q}$ and $\TermTwoT{\gamma}{s}{\mu}{\nu}{p}$
act \emph{contravariantly} on the subscripts of upstairs terms.

We expect structurality to be admissible for the base, and structurality
over that to be admissible for the top, e.g.:
\begin{mathpar}
\inferrule*[Left = weaken-over]
           {\Gamma,\Gamma' \yields_\mu M : A \\ (\text{where } \gamma,\gamma' \vdash \mu : p)}
           {\Gamma,y:B,\Gamma' \yields_\mu M : A \\ (\text{where } \gamma,y:q,\gamma' \vdash \mu : p)}

\inferrule*[Left = subst-over]
           {\Gamma,x:A,\Gamma' \yields_\nu N : C \\ (\text{where } \gamma,x:p,\gamma' \vdash \nu : \gamma) \\\\
            \Gamma \vdash_\mu M : A \\ (\text{where } \gamma \vdash \mu : p)
           }
           {\Gamma,\Gamma'[M/x] \yields_{\nu[\mu/x]} N[M/x] : A[M/x] \\ (\text{where } \gamma,\gamma'[\mu/x] \vdash \nu[\mu/x] : p[\mu/x])}
\end{mathpar}


\subsection{Mode Theory}

\begin{enumerate}

\item Contexts are as usual:

\begin{mathpar}
  \inferrule*{ }
             {\cdot \ctx}
             
  \inferrule*
    {\gamma \ctx \\
     \gamma \yields p \type}
    {\gamma,x:p \ctx}
\end{mathpar}  

\item In all mode theories, terms must have: 

\begin{mathpar}
\inferrule*{ }
             {\gamma,x : p, \gamma' \yields x : p}
             
\inferrule*
    {\gamma \yields \mu : q \\
     \TypeTwo{\gamma}{s}{p}{q}
    }
    {\gamma \yields \TrPlus{s}{\mu} : p}
\\
\TrPlus{\id}{\mu} \equiv \mu \qquad
\TrPlus{s'}{\TrPlus{s}{\mu}} \equiv \TrPlus{(s';s)}{\mu} 
\end{mathpar}

\item Mode type morphisms:
\begin{mathpar}
    \inferrule*{ }
          {\TypeTwo{\gamma}{\id_p}{p}{p}}
    \qquad
    \inferrule*{{\TypeTwo{\gamma}{s_1}{p_1}{p_2}} \\
                {\TypeTwo{\gamma}{s_2}{p_2}{p_3}}
          }
          {\TypeTwo{\gamma}{s_1;s_2}{p_1}{p_3}}

\inferrule*{{\gamma,x:p} \vdash {q} \type \\
            \TermTwoT{\gamma}{t}{\mu}{\mu'}{p}\\
           } 
           {\TypeTwo{\gamma}{\ap {q} {t/x}}{q[\mu/x]}{q[\mu'/x]}}

\\
\id;s \equiv s \equiv s;\id \and
(s;s');s'' \equiv s;(s';s'') \\ 
\ap q {\id_{\mu}/x} \equiv \id_{q[\mu/x]} \and
\ap q {(s;t)/x} \equiv \ap q {s/x}; \ap q {t/x} \\ 
\ap q {s/\_} \equiv \id_q \\ 
\ap {(q[\mu/x])} {s/y} \equiv \ap q {\ap \mu {s/y}/x} \quad (\text{where } \gamma,y:p' \vdash \mu : p \text{ and } \gamma,x:p \vdash q \type)\\
s[\nu/x];\ap{q'}{t/x} \equiv \ap{q}{t/x};s[\nu'/x] \quad 
(\text{where } \TypeTwo{\gamma,x:p}{s}{q}{q'} \text{ and } \TermTwoT{\gamma}{t}{\nu}{\nu'}{p})

%% subst: \id_\mu[\nu/x] = \id_{\mu[\nu/x]}
%% subst: s[x/x] = s
%% subst: (s;t)[\mu/x] = s[\mu/x];t[\mu/x]
%% subst: s[\mu[\nu/x]/x] = s[\mu/x][\nu/x]
%% subst: ap q (s [\mu/x]) = (ap q s)[\mu/x] and generalization
\end{mathpar}

We write $\ap q {t/x}$ for whiskering (\dsd{ap} in book HoTT).

\item 2-cells between terms.  First, we have
  identity/composition/whiskering and associated equations (whiskering
  on the other side is given by substitution):
\begin{mathpar}
    \inferrule*{ }
          {\TermTwoT{\gamma}{\id_\mu}{\mu}{\mu}{p}}
    \qquad
    \inferrule*{{\TermTwoT{\gamma}{s_1}{\mu_1}{\mu_2}{p}} \\
                {\TermTwoT{\gamma}{s_2}{\mu_2}{\mu_3}{p}}
          }
   {\TermTwoT{\gamma}{s_1;s_2}{\mu_1}{\mu_3}{p}}

\inferrule*{{\gamma,x:p} \yields {\nu} : {q} \\
            \TermTwoT{\gamma}{s}{\mu}{\mu'}{p}\\
           } 
           {\TermTwoT{\gamma}{\ap \nu {s/x}}{\nu[\mu/x]}{\TrPlus{\ap{q}{s/x}}{\nu[\mu'/x]}}{q[\mu/x]}}

\\           
\id;s \equiv s \equiv s;\id \and
(s;s');s'' \equiv s;(s';s'') \\ 
\ap \nu {\id_{\mu}/x} \equiv \id_{\nu[\mu/x]} \and
\ap \nu {(s;t)/x} \equiv \ap \nu {s/x} ; (\ap {(\TrPlus{\ap{q}{s/x}}{y})} {\ap \nu {t/x}/y}) \\ 
\ap x {s/x} \equiv s  \and
\ap \nu {s/\_} \equiv \id_\nu \and
\ap {(\nu[\mu/x])} {s/y} \equiv \ap \nu {\ap \mu {s/y}/x} \quad
(\text{where } \gamma,y:p' \vdash \mu : p \text{ and } \gamma,x:p \vdash \nu : q)\\
t[\mu/x];\ap{\nu'}{s/x} \equiv \ap{\nu}{s/x};\ApPlus{\ap{q}{s/x}}{t[\mu'/x]} \quad
 (\text{where } \TermTwoT{\gamma,x:p}{t}{\nu}{\nu'}{q} \text{ and } \TermTwoT{\gamma}{s}{\mu}{\mu'}{p}) \\
\ap{\TrPlus{s}{\mu}}{t/x} \equiv \ApPlus{(s[\nu/x])}{\ap{\mu}{t/x}}\quad 
(\text{where } \TypeTwo{\gamma,x:p}{s}{q}{q'} \text{ and } \TermTwoT{\gamma}{t}{\nu}{\nu'}{p})
\end{mathpar}

\item We assume $1/\Sigma$ modes:

\begin{mathpar}
  \inferrule*{ } { \gamma \yields 1 \type } \and
  
  \inferrule*{ \gamma \yields p \type \\ 
               \gamma,x:p \yields q \type }
             {\gamma \yields \sigmacl{x}{p}{q} \type} \\
             
  \inferrule*{ }
             {\gamma \yields \mt : 1}
  \and 
  \mu \equiv \mt
\\
\inferrule*{
  \gamma \yields \mu : p \and
  \gamma \yields \nu : q[\mu/x]
    }
   {\gamma \yields (\mu,\nu) : \sigmacl{x}{p}{q}}
\and
\inferrule*
    {\gamma \yields \mu : \sigmacl{x}{p}{q}}
    {\gamma \yields \fst \mu : p}
\and
\inferrule*
    {\gamma \yields \mu : \sigmacl{x}{p}{q}}
    {\gamma \yields \snd \mu : q[\fst \mu / x]}
    \\
    \fst{(\mu,\nu)} \equiv \mu \and
    \snd{(\mu,\nu)} \equiv \nu \and
    p \equiv (\fst p, \snd p)
\end{mathpar}

Equations for ``transport'' in $\Sigma$:
\begin{mathpar}
\TrPlus{(\sigmacl{x}{s}{t})}{\mu} \equiv (\TrPlus{s}{\fst \mu},\TrPlus{(t[\fst \mu/x])}{\snd \mu})
\end{mathpar}

Mode type morphisms: We need congruence for $\Sigma$ to be a rule (because we don't have ap on a type variable/universes):
\begin{mathpar}
  \inferrule*
  {\TypeTwo{\gamma}{s}{p}{p'} \\
    \TypeTwo{\gamma,x':p'}{t}{q[\TrPlus{s}{x'}/x]}{q'}}
  {\TypeTwo{\gamma}{\sigmacl{x'}{s}{t}}{\sigmacl{x}{p}{q}}{\sigmacl{x'}{p'}{q'}}} \\

  \sigmacl{x'}{\id_p}{\id_q} \equiv \id_{\sigmacl{x'}{p}{q}} \and
  (\sigmacl{x'}{s}{t});(\sigmacl{x''}{s'}{t'}) \equiv \sigmacl{x''}{(s;s')}{(t[\TrPlus{s'}{x''}/x'];t')} \\

  \ap{(\sigmacl{x'}{p}{q})}{s/(y:r)} \equiv
  \sigmacl{x'}{\ap{p}{s/y}}{\ap{({q[\fst z/x,\snd z/y]})}{\extend{s}{x'}/(z:(\sigmacl{y}{r}{p}))}}
\end{mathpar}

Finally, we have the 2-cells for $1/\Sigma$-terms:
\begin{mathpar}
s \equiv \id_{()} \text{ for } \yields_1 s : () \tcell ()
\\

\inferrule*
    {\TermTwoT{\gamma}{s}{\mu}{\mu'}{p} \and
      \gamma \vdash \nu' : q[\mu'/x]
    }
      {\TermTwoT{\gamma}{\extend{s}{\nu'}}{(\mu,\TrPlus{\ap{q}{s/x}}{\nu'})}{(\mu',\nu')}{\sigmacl{x}{p}{q}}}\\
\ap {\fst(z)} {\extend{s}{\nu'}/z} \equiv s \and
\ap {\snd(z)} {\extend{s}{\nu'}/z} \equiv \id_{\TrPlus{\ap{q}{s/x}}{\nu'}}  \\
s \equiv \ap{(\fst{\mu},y)}{\ap{(\snd z)}{s/z}/y};\extend{\ap{(\fst{z})}{s/z}}{\snd{\mu'}} \quad (\text{where } \TermTwoT{\gamma}{s}{\mu}{\mu'}{\sigmacl{x}{p}{q}})
\\      
{\extend{\id_\mu}{\nu'}} \equiv \id_{(\mu,\nu')} \and
{\extend{(s;s')}{\nu''}} \equiv  \extend{s}{\TrPlus{\ap{q}{s'/x}}{\nu''}};\extend{s'}{\nu''}   \\
\extend{s}{\nu'} ; (\ap{(\mu',y)}{t/y}) \equiv
(\ap{(\mu,\TrPlus{(\ap{q}{s})}{y})}{t/y}); \extend{s}{\nu''} \qquad (\text{where }\TermTwoT{\gamma}{t}{\nu'}{\nu''}{q[\mu'/x]})
\end{mathpar}

\item
  All judgements have a substitution principle
\begin{mathpar}
  \inferrule*{\gamma,x:p,\gamma' \yields J \\
              \gamma \yields \mu : p
              }
             {\gamma,\gamma'[\mu/x] \yields J[\mu/x]} \\

J[\mu/x][\nu/y] \equiv J[\nu/y][\mu[\nu/y]/x]
\end{mathpar}

\drlnote{write out the usual rules defining this}
             
\end{enumerate}

We sometimes write \ap{\mu}{s} for \ap{\mu(x)}{s/x}, eliding the
variable name when it is clear how to view $\mu$ as a term with a
distinguished variable; e.g. $\ApPlus{s}{t}$ for
$\ap{\TrPlus{s}{x}}{t/x}$.

\subsubsection{Lemmas}

Horizontal composition:
\begin{mathpar}
  \inferrule*[Left=Derivable]
      {\TermTwoT{\gamma}{s}{\mu}{\mu'}{p} \\
    \TermTwoT{\gamma, x : p}{t}{\nu}{\nu'}{q}}
             {\TermTwoT{\gamma}{\ap{t}{s/x} :\equiv t[\mu/x];\ap{\nu'}{s/x}}{\nu[\mu/x]}{\TrPlus{\ap{q}{s/x}}{\nu'[\mu'/x]}}{q[\mu/x]}}
\\ 
\ap{\id_\nu}{s/x} \equiv \ap{\nu}{s/x} \and \ap{t}{\id_{\mu}/x} \equiv t[\mu/x]
\end{mathpar}

Pairing and projection 2-cells are definable:
\begin{mathpar}
  \inferrule*[Left=Derivable]
      {\TermTwoT{\gamma}{s}{\mu}{\mu'}{p} \\
    \TermTwoT{\gamma}{t}{\nu}{\TrPlus{\ap{q}{s}}{\nu'}}{q[\mu/x]}}
             {\TermTwoT{\gamma}{(s,t) :\equiv \ap{(\mu,y)}{t/y};\extend{s}{\nu'}}{(\mu,\nu)}{(\mu',\nu')}{\sigmacl{x}{p}{q}}}

   \inferrule*[Left=Deriv]
              { {\TermTwoT{\gamma}{s}{\mu}{\mu'}{\sigmacl{x}{p}{q}}} }
              { {\TermTwoT{\gamma}{\ap{\fst(y)}{s/y}}{\fst{\mu}}{\fst{\mu'}}{p}} }
   \and
   \inferrule*[Left=Deriv]
              { {\TermTwoT{\gamma}{s}{\mu}{\mu'}{\sigmacl{x}{p}{q}}} }
              { {\TermTwoT{\gamma}{\ap{\snd(y)}{s/y}}{\snd{\mu}}{\TrPlus{\ap{(q(\fst y/x))}{s/y}}{\snd{\mu'}}}{q[\fst{\mu}/x]}} }
\end{mathpar}


\begin{lemma}
For mode term morphisms
\begin{align*}
\TermTwoT{\gamma &}{s}{\mu}{\mu'}{p} \\
\TermTwoT{\gamma &}{t}{\nu}{\TrPlus{\ap{q}{s}}{\nu'}}{q[\mu/x]} \\
\TermTwoT{\gamma &}{s'}{\mu'}{\mu''}{p} \\
\TermTwoT{\gamma &}{t'}{\nu'}{\TrPlus{\ap{q}{s'}}{\nu''}}{q[\mu'/x]}
\end{align*}
we have
\begin{align*}
(s, t);(s', t') \equiv ((s;s'), (t;\ApPlus{\ap{q}{s}}{t'}))
\end{align*}
\end{lemma}
\begin{proof}
Follows by
\begin{align*}
(s, t);(s', t') 
&\equiv \ap{(\mu,y)}{t/y};\extend{s}{\nu'};\ap{(\mu',y)}{t'/y};\extend{s'}{\nu''} \\
&\equiv \ap{(\mu,y)}{t/y};\ap{(\mu, \TrPlus{\ap{q}{s}}{y})}{t'/y};\extend{s}{\TrPlus{\ap{q}{s'}}{\nu''}};\extend{s'}{\nu''} \\
&\equiv \ap{(\mu,y)}{t/y};\ap{(\mu, y)}{\ApPlus{\ap{q}{s}}{t'}/y};\extend{s}{\TrPlus{\ap{q}{s'}}{\nu''}};\extend{s'}{\nu''} \\
&\equiv \ap{(\mu,y)}{t;\ApPlus{\ap{q}{s}}{t'}/y};\extend{s;s'}{\nu''} \\
&\equiv ((s;s'), (t;\ApPlus{\ap{q}{s}}{t'}))
\end{align*}
\end{proof}

\subsection{Contexts}

\begin{mathpar}
  \inferrule*[Left = ctx-form]{ }
  {\yields_{\cdot} \cdot \CTX  } \and 

  \inferrule*[Left = ctx-form]{
    \yields_\gamma \Gamma \CTX \and (\text{where } \yields \gamma \ctx) \\\\
    \Gamma \yields_p A \TYPE \and (\text{where }  \gamma \yields p \type)}
  {\yields_{\gamma, x : p} \Gamma, x : A \CTX \and (\text{where } \yields \gamma,x:p \ctx)  } \\
\end{mathpar}

\subsection{Types and Terms}

\subsubsection{Structural Rules}

\begin{mathpar}
  \inferrule*[Left = var]{
    % \yields \Gamma, x : A, \Gamma' \CTX_{\gamma, x : p, \gamma'}
  }
  {\Gamma, x : A, \Gamma' \yields_x x : A \and (\text{where } \gamma,x:p,\gamma' \yields x : p)} \and

 \inferrule*[Left = rewrite]{
   \Gamma \yields_\mu M : A 
   \and \TermTwoT{\gamma}{s}{\nu}{\mu}{p}
  }
  {\Gamma \yields_\nu \rewrite{s}{M} : A} \\ \\
  
  \rewrite{\id_{\mu}}{M} \equiv M \and
  \rewrite{(s;t)}{M} \equiv \rewrite{s}{\rewrite{t}{M}} \and
  \rewrite{s}{M}[\rewrite{t}{N}/x] \equiv \rewrite{\ap{s}{t/x}}{\StI{\ap{q}{t/x}}{M[N/x]}}
\end{mathpar}

\subsubsection{Telescope Types}

\begin{mathpar}
  \inferrule*[Left=$1$-form]{~}{\Gamma \yields_{1} 1 \TYPE} \and
  \inferrule*[Left=$1$-intro]{~}{\Gamma \yields_{()} () : 1} \\
  M \equiv () \\
  \inferrule*[Left=$()$-form]{ \Gamma \yields_p A \TYPE \\
               \Gamma,x:A \yields_q B \TYPE}
             { \Gamma \yields_{\sigmacl{x}{p}{q}} \telety{x}{A}{B} \TYPE}
  \\
  \inferrule*[Left=$()$-pair]{ \Gamma \yields_\mu M : A \\
               \Gamma \yields_\nu N : B[M/x]
             }
             { \Gamma \yields_{(\mu,\nu)} (M,N) : \telety{x}{A}{B}}
  \and
  \inferrule*[Left=$()$-fst]{ \Gamma \yields_{\mu} M : \telety{x}{A}{B}}
             { \Gamma \yields_{\fst \mu} \fst{M} : A} 
  \and
  \inferrule*[Left=$()$-snd]{ \Gamma \yields_{\mu} M : \telety{x}{A}{B}}
             { \Gamma \yields_{\snd \mu} \snd{M} : B[\fst M/x]} 

    \fst{(M,N)} \equiv M \and
    \snd{(M,N)} \equiv N \and
    P \equiv (\fst P, \snd P)
\end{mathpar}


\subsubsection{Modalities}

\begin{mathpar}
  \inferrule*[Left = F-form]{
    %% \yields_\gamma \Gamma \CTX \and (\text{where } \yields \gamma \ctx)\\\\
    \Gamma \yields_p A \TYPE \and (\text{where } \gamma \yields p \type) \\\\
    \gamma, x:p \yields \mu : q 
  }
  {\Gamma \yields_q \F{x.\mu}{A} \TYPE \and (\text{where } \gamma \yields q \type) } \\
  
  \inferrule*[Left = F-intro]{
    \Gamma \yields_{\nu} M : A
    \and (\text{where } \gamma \yields {\nu} : p)
    %% \and \gamma \yields \nu : q 
    %% \and \gamma \yields \mu[\theta] : q 
    %% \and \gamma \yields (\nu \Rightarrow \mu[\theta]) : q
  }
  {\Gamma \yields_{\mu[\nu/x]} \FI{M} : \F{x.\mu}{A} \and (\text{where } \gamma \yields \mu[\nu/x] : q)} \\

  \inferrule*[Left = F-elim]{
    \Gamma, y : \F{x.\mu}{A} \yields_{r} C \TYPE \and (\text{where } \gamma, y : q \yields r \type) \\\\
    \Gamma \yields_{\nu} M : \F{x.\mu}{A} \and (\text{where } \gamma \yields \nu : q) \\\\
    \Gamma, x:A \yields_{\nu' [\mu / y]} N : C [\FI{x}/y]
    \and (\text{where } \gamma, x:p \yields \nu' [\mu / y] : r [\mu / y] )}
  {\Gamma \yields_{\nu'[\nu/y]} \FE{M}{x}{N} : C[M/y]  \and (\text{where }  \gamma \yields {\nu'[\nu/y]} : r[\nu/y])} \\
  \FE{\FI{M}}{x}{N} \equiv N[M/x] %\and
%  \text{(optionally:) }
%  \FE{M}{x}{N[\FI{x}/z]} \equiv N[M/z]
  \\ \\

  \inferrule*[Left = F-Elim]{
    \gamma,y:q \yields r \type \\\\
    \Gamma \yields_{\nu} M : \F{x.\mu}{A} \and (\text{where } \gamma \yields \nu : q) \\\\
    \Gamma, x:A \yields_{r [\mu / y]} C \TYPE
    \and (\text{where } \gamma, x:p \yields r [\mu / y] \type )}
  {\Gamma \yields_{r[\nu/y]} \FE{M}{x}{C} \TYPE \and (\text{where }  \gamma \yields {r[\nu/y]} \type)} \\
  \FE{\FI{M}}{x}{C} \equiv C[M/x] %\and
%  \text{(optionally:) }
%  \FE{M}{x}{C[\FI{x}/z]} \equiv C[M/z]
\\ \\
  \inferrule*[Left = U-form]{
    \Gamma \yields_p A \TYPE \and (\text{where } \gamma \yields p \type)\\\\
    \and \Gamma,x:A \yields_q B \TYPE \and (\text{where } \gamma,x:p \yields q \type)\\\\
    \and \gamma, x:p, c:r \yields \mu : q
  }{\Gamma \yields_r \U{c.\mu}{A}{B} \TYPE \and (\text{where } \gamma \yields r \type)} \\

  \inferrule*[Left = U-intro]{
    \Gamma,x:A \yields_{\mu[\nu/c]} M : B \and (\text{where } \gamma,x:p \yields {\mu[\nu/c]} : q)
  }
  {\Gamma \yields_{\nu} \UI {x}{M} : \U{c.\mu}{x:A}{B}
    \and (\text{where } \gamma \yields \nu : r)
  } \\
  
  \inferrule*[Left = U-elim]{
    \Gamma \yields_{\nu_1} N_1 : \U{c.\mu}{x:A}{B} \and (\text{where } \gamma \yields \nu_1 : r) \\\\
    \Gamma \yields_{\nu_2} N_2 : A \and (\text{where } \gamma \yields \nu_2 : p)
  }{
    \Gamma \yields_{\mu[\nu_2/x,\nu_1/c]} \UE{N_1}{N_2} : B[N_2/x] \and (\text{where } \gamma \yields \mu[\nu_2/x,\nu_1/c] : q)
  } \\

  \UE{(\UI{x}{M})}{N} \equiv M[N/x] \and 
  \UI{x}{\UE{N}{x}} \equiv N
\end{mathpar}

\subsubsection{Surprisingly Strict Modalities}

\begin{mathpar}
  \inferrule*[Left = s-form]{
    \Gamma \yields_p A \TYPE \and (\text{where } \gamma \yields p \type)\\\\
    \and \TypeTwo{\gamma}{s}{q}{p}
  }{\Gamma \yields_q \St{s}{A} \TYPE \and (\text{where } \gamma \yields q \type)} \\

  \inferrule*[Left = S-intro]{
    \Gamma \yields_{\mu} M : A
    \and (\text{where } \gamma \yields {\mu} : p)
  }
  {\Gamma \yields_{\TrPlus{s}{\mu}} \StI{s}{M} : \St{s}{A} \and (\text{where } \gamma \yields \TrPlus{s}{\mu} : q)} \\

  \inferrule*[Left = S-elim]{
    \Gamma, y : \St{s}{A} \yields_{r} C \TYPE \and (\text{where } \gamma, y : q \yields r \type) \and \\\\
    \Gamma \yields_{\nu} M : \St{s}{A} \and (\text{where } \gamma \yields \nu : q) \\\\
    \Gamma, x : A \yields_{\nu' [\TrPlus{s}{x} / y]} N : C [\StI{s}{x}/y]
    \and (\text{where } \gamma, x : p \yields \nu' [\TrPlus{s}{x} / y] : r[\TrPlus{s}{x} / y] )}
  {\Gamma \yields_{\nu'[\nu/y]} \StE{s}{M}{x}{N} : C[M/y]  \and (\text{where } \gamma \yields {\nu'[\nu/y]} : r[\nu/y])} \\
  \StE{s}{\StI{s}{M}}{x}{N} \equiv N[M/x] \and
  \StE{s}{M}{x}{N[\StI{s}{x}/z]} \equiv N[M/z]
  \\
  
  \inferrule*[Left = S-Elim]{
    \gamma,y:q \yields r \type \\\\
    \Gamma \yields_{\nu} M : \St{s}{A} \and (\text{where } \gamma \yields \nu : q) \\\\
    \Gamma, x:A \yields_{r [\TrPlus{s}{x} / y]} C \TYPE
    \and (\text{where } \gamma, x:p \yields r [\TrPlus{s}{x} / y] \type )}
  {\Gamma \yields_{r[\nu/y]} \StE{s}{M}{x}{C} \TYPE \and (\text{where }  \gamma \yields {r[\nu/y]} \type)} \\
  \StE{s}{\StI{s}{M}}{x}{C} \equiv C[M/x] \and
  \StE{s}{M}{x}{C[\StI{s}{x}/z]} \equiv C[M/z]
  \\ \\
\end{mathpar}

%\drlnote{Change the definition of $s$-types to $U$-types as primitive and derive $F$, so that having $\eta$ is less surprising.}

Term/type equalities:
\begin{align}
%\StI{s}{\FI{M}} &\equiv \FI{M} &\St{s}{\F{x.\mu}{A}} &\equiv \F{x.\TrPlus{s}{\mu}}{A} \\
%\FI{\StI{s}{M}} &\equiv \FI{M} &\F{x.\mu}{\St{s}{A}} &\equiv \F{x.\mu[\TrPlus{s}{x}/x]}{A} \\
%%\UStI{s}{\UI{x}{M}} &\equiv \UI{x}{M} &\St{s}{\U{c.\mu}{x:A}{B}} &\equiv \U{c.\mu[\TrCirc{s}{c}/c]}{x:A}{B} \\
%%\UI{x}{\UStI{s}{M}} &\equiv \UI{x}{M} &\U{c.\mu}{x:A}{\St{s}{B}} &\equiv \U{c.\TrCirc{s}{\mu}}{x:A}{B} \\
%\UI{x}{M} &\equiv \UI{x}{M[\StI{s}{x}/x]} &\U{c.\mu}{x:\St{s}{A}}{B} &\equiv \U{c.\mu[\TrPlus{s}{x}/x]}{x:A}{B[\StI{s}{x}/x]} \\
\label{eq:stype-pair}\StI{(s, t)}{(M, N)} &\equiv (\StI{s}{M}, \StI{t[\mu/x]}{N}) &\St{(\telety{x'}{s}{t})}{\telety{x'}{A'}{B'}} & \equiv \telety{x}{\St{s}{A'}}{\StE{s}{x}{x'}{\St{t}{B'}}} \\
\StI{s}{\StI{t}{M}} &\equiv \StI{s;t}{M} &\St{s}{\St{t}{A}} &\equiv \St{(s;t)}{A} \\
\StI{\id_p}{M} &\equiv M &\St{\id_p}{A} &\equiv A \\
\label{eq:stype-subst} \rewrite{\ap{\nu}{t/x}}{\StI{\ap{q}{t/x}}{N[M/x]}} &\equiv N[\rewrite{t}{M}/x]  &\St{(\ap{q}{t/x})}{B[M/x]} & \equiv B[\rewrite{t}{M}/x] 
%% other whiskering is a substitution rule:
%% \St{(s[\mu/x])}{B[M/x]} & \equiv (\St{s}{B})[M/x] 
\end{align}
%Where the last term equation is a special case of rewrite on substitutions. The inputs are typed $\Gamma,x:A \vdash_q B \TYPE $\ and $\Gamma \vdash_{\mu'} M : A$ and $\TermTwo{\gamma}{t}{\mu}{\mu'}$.

Due to the lack of the eta rule for $\mathsf{F}$-types, we also need to assert
\begin{mathpar}
(\FE{M}{x}{\rewrite{t[\mu/y]}{N}}) \equiv \rewrite{t[\nu/y]}{\FE{M}{x}{N}}
\end{mathpar}

\subsection{Lemmas}

\begin{lemma}
The \textsc{rewrite} rule commutes with pairing of telescope types:
\begin{align*}
%(\rewrite{s}{M},N) &\equiv \rewrite{(s, \varepsilon_\nu^{\ap{q}{s/x}})}{(M, \UnSt{\ap{q}{s/x}}{N}}) \\
%(M,\rewrite{t}{N}) &\equiv \rewrite{(\id_\mu, t)}{(M,N)} \\
\rewrite{(s, t)}{(M, N)} &\equiv (\rewrite{s}{M}, \rewrite{t}{\StI{\ap{q}{s/x}}{N}} ) 
\end{align*}
where
\begin{align*}
\TermTwoT{\gamma &}{s}{\mu}{\mu'}{p} \\
\TermTwoT{\gamma &}{t}{\nu}{\TrPlus{\ap{q}{s}}{\nu'}}{q[\mu/x]} \\
\Gamma &\yields_{\mu'} M : A \\
\Gamma &\yields_{\nu'} N : B[M/x]
\end{align*}
\end{lemma}
\begin{proof}
\begin{align*}
&\rewrite{(s, t)}{(M, N)} \\
&\equiv (\fst\rewrite{(s, t)}{(M, N)}, \snd \rewrite{(s, t)}{(M, N)} ) \\
&\equiv (\fst z [\rewrite{(s, t)}{(M, N)}/z], \snd z [\rewrite{(s, t)}{(M, N)}/z] ) \\
&\equiv (\rewrite{\ap{\fst}{(s, t)}}{\fst (M, N)}, \rewrite{\ap{\snd}{(s, t)}}{\StI{\ap{q[\fst z/x]}{(s, t)/z}}{\snd (M, N)}} ) \\
&\equiv (\rewrite{s}{M}, \rewrite{t}{\StI{\ap{q}{s/x}}{N}} )
\end{align*}
This type checks via Equation~\eqref{eq:stype-subst}; in the second component we have
\begin{align*}
&\snd \rewrite{(s, t)}{(M, N)} &&: B[\fst\rewrite{(s, t)}{(M, N)}/x] \\
\equiv~&\snd \rewrite{(s, t)}{(M, N)} &&: B[\rewrite{s}{M}/x] \\
\equiv~&\rewrite{t}{\StI{\ap{q}{s/x}}{N}} &&: \St{\ap{q}{s/x}}{B[M/x]}
\end{align*}
\end{proof}

\begin{lemma} \label{lem:rewrite-push}
The \textsc{rewrite} rule commutes with $\mathsf{F}$-, $\mathsf{U}$- and $s$- introduction and elimination.
\begin{align*}
\FI{\rewrite{s}{M}} &\equiv \rewrite{\ap{\mu}{s/x}}{\FI{M}} \\
(\FE{\rewrite{s}{M}}{x}{N}) &\equiv \rewrite{\ap{\nu'}{s/y}}{\StI{\ap{r}{s/y}}{\FE{M}{x}{N}}} \\
\UE{M}{\rewrite{s}{N}} &\equiv \rewrite{\ap{\mu[\nu_1/c]}{s/x}}{\StI{\ap{q}{s/x}}{\UE{M}{N}}} \\
\UE{(\rewrite{t}{M})}{N} &\equiv \rewrite{\ap{\mu[\nu_2/x]}{t/c}}{\UE{M}{N}} \\
\UI{x}{\rewrite{\ap{\mu}{s/c}}{M}}  &\equiv\rewrite{s}{\UI{x}{M}} \\
\StI{t}{\rewrite{s}{M}} &\equiv \rewrite{\ApPlus{t}{s}}{\StI{t}{M}} \\
(\StE{t}{\rewrite{s}{M}}{x}{N}) &\equiv \rewrite{\ap{\nu'}{s/y}}{\StI{\ap{r}{s/y}}{\StE{t}{M}{x}{N}}} \\
(\StE{t}{M}{x}{\rewrite{s[\TrPlus{t}{x}/y]}{N}}) &\equiv \rewrite{s[\nu/y]}{\StE{t}{M}{x}{N}} 
%\intertext{If we have definitional eta-expansion for \textsf{F}-types, we can also derive (rather than asserting)}
%(\FE{M}{x}{\rewrite{s[\mu/y]}{N}}) &\equiv \rewrite{s[\nu/y]}{\FE{M}{x}{N}}
\end{align*}
\end{lemma}
\begin{proof}
For \textsf{F}-types:
\begin{align*}
\FI{\rewrite{s}{M}} 
&\equiv \rewrite{\id_{\mu[\nu/x]}}{\FI{x}}[\rewrite{s}{M}/x]  \\
&\equiv \rewrite{\ap{\mu}{s/x}}{\StI{\ap{q}{s/x}}{\FI{x}[M/x]}} \\
&\equiv \rewrite{\ap{\mu}{s/x}}{\StI{\ap{q}{s/x}}{\FI{M}}} \\
&\equiv \rewrite{\ap{\mu}{s/x}}{\FI{M}} && \text{As $x$ does not appear in $q$.}\\
(\FE{\rewrite{s}{M}}{x}{N})
&\equiv \rewrite{\id_{\nu'[\nu/y]}}{\FE{y}{x}{N}}[\rewrite{s}{M}/y] \\
&\equiv \rewrite{\ap{\nu'}{s/y}}{\StI{\ap{r}{s/y}}{(\FE{y}{x}{N})[M/y]}} \\
&\equiv \rewrite{\ap{\nu'}{s/y}}{\StI{\ap{r}{s/y}}{\FE{M}{x}{N}}}
\end{align*}

For \textsf{U}-types:
\begin{align*}
\UE{M}{\rewrite{s}{N}}
&\equiv \rewrite{\id_{\mu[z/x, \nu_1/c]}}{\UE{M}{z}}[\rewrite{s}{N}/z] \\
&\equiv \rewrite{\ap{\mu[\nu_1/c]}{s/x}}{\StI{\ap{q[\nu_1/c]}{s/x}}{\UE{M}{N}}} \\
&\equiv \rewrite{\ap{\mu[\nu_1/c]}{s/x}}{\StI{\ap{q}{s/x}}{\UE{M}{N}}} \\
\UE{(\rewrite{t}{M})}{N}
&\equiv \rewrite{\id_{\mu[\nu_2/x, z/c]}}{\UE{(z)}{N}}[\rewrite{t}{M}/z] \\
&\equiv \rewrite{\ap{\mu[\nu_2/x]}{t/c}}{\StI{\ap{q[\nu_2/x]}{t/c}}{\UE{M}{N}}} \\
&\equiv \rewrite{\ap{\mu[\nu_2/x]}{t/c}}{\UE{M}{N}} && \text{(As $c$ does not appear in $q$)}\\
\rewrite{s}{\UI{x}{M}}
&\equiv \UI{y}{\UE{(\rewrite{s}{\UI{x}{M}})}{y}} \\
&\equiv \UI{y}{\rewrite{\ap{\mu[y/x]}{s/c}}{\UE{(\UI{x}{M})}{y}}} \\
&\equiv \UI{y}{\rewrite{\ap{\mu[y/x]}{s/c}}{M[y/x]}} \\
&\equiv \UI{x}{\rewrite{\ap{\mu}{s/c}}{M}}
\end{align*}

The first two equations for $s$-type follow by the same reasoning as for $\mathsf{F}$-types. For the third, we make use of the eta-expansion that is not available for $\mathsf{F}$-types.
\begin{align*}
\rewrite{s[\nu/y]}{\StE{t}{M}{x}{N}}
&\equiv \rewrite{s}{\StE{t}{y}{x}{N}} [M/y] \\
&\equiv \StE{t}{M}{z}{(\rewrite{s}{\StE{t}{y}{x}{N}}[\StI{t}{z}/y])} \\
&\equiv \StE{t}{M}{z}{(\rewrite{s[\TrPlus{t}{z}/y]}{\StE{t}{\StI{t}{z}}{x}{N}})} \\
&\equiv \StE{t}{M}{z}{\rewrite{s[\TrPlus{t}{z}/y]}{N}}
\end{align*}
\end{proof}

\begin{lemma}
The \textsc{rewrite} rule commutes with \textsf{F}- and $s$-elimination into types:
\begin{align*}
(\FE{\rewrite{s}{M}}{x}{C}) &\equiv \St{\ap{r}{s/y}}{\FE{M}{x}{C}} \\
(\StE{t}{\rewrite{s}{M}}{x}{C}) &\equiv \St{\ap{r}{s/y}}{\StE{t}{M}{x}{C}}
\end{align*}
\end{lemma}
\begin{proof}
This is analogous to elimination into terms:
\begin{align*}
\FE{\rewrite{s}{M}}{x}{C}
&\equiv (\FE{y}{x}{C})[\rewrite{s}{M}/y] \\
&\equiv \St{\ap{r}{s/y}}{(\FE{y}{x}{C})[M/y]} \\
&\equiv \St{\ap{r}{s/y}}{\FE{M}{x}{C}}
\end{align*}
and similarly for $s$-types.
\end{proof}

\begin{lemma}\label{lem:s-elim-s-elim}
\textsc{s-elim} commutes with \textsc{F-elim} and \textsc{s-elim}:
\begin{align*}
\FE{(\StE{s}{M}{x'}{N'})}{x}{N} &\equiv \StE{s}{M}{x'}{(\FE{N'}{x}{N})} \\
\StE{t}{(\StE{s}{M}{x'}{N'})}{x}{N} &\equiv \StE{s}{M}{x'}{(\StE{t}{N'}{x}{N})}
\end{align*}
\end{lemma}
\begin{proof}
Using eta for $s$-types:
\begin{align*}
&\FE{(\StE{s}{M}{x'}{N'})}{x}{N} \\
&\equiv (\FE{(\StE{s}{z}{x'}{N'})}{x}{N})[M/z] \\
&\equiv \StE{s}{M}{z'}{((\FE{(\StE{s}{z}{x'}{N'})}{x}{N})[\StI{s}{z'}/z])} \\
&\equiv \StE{s}{M}{z'}{((\FE{(\StE{s}{\StI{s}{z'}}{x'}{N'})}{x}{N}))} \\
&\equiv \StE{s}{M}{z'}{(\FE{N'[z'/x']}{x}{N})} \\
&\equiv \StE{s}{M}{x'}{(\FE{N'}{x}{N})}
\end{align*}
and similarly for the second.
\end{proof}

\begin{lemma}\label{lem:s-elim-fusion}
Fusion for \textsc{s-elim} on composites:
\begin{align*}
\StE{s;t}{M}{x}{N} \equiv \StE{s}{M}{x'}{\StE{t}{x'}{x}{N}}
\end{align*}
\end{lemma}
\begin{proof}
\begin{align*}
\StE{s;t}{M}{x}{N}
&\equiv \StE{s}{M}{x'}{\StE{s;t}{\StI{s}{x'}}{x}{N}} \\
&\equiv \StE{s}{M}{x'}{\StE{t}{x'}{x''}{\StE{s;t}{\StI{s}{\StI{t}{x''}}}{x}{N}}} \\
&\equiv \StE{s}{M}{x'}{\StE{t}{x'}{x''}{\StE{s;t}{\StI{s;t}{x''}}{x}{N}}} \\
&\equiv \StE{s}{M}{x'}{\StE{t}{x'}{x''}{N[x''/x]}} \\
&\equiv \StE{s}{M}{x'}{\StE{t}{x'}{x}{N}}
\end{align*}
\end{proof}

\begin{lemma}\label{lem:s-elim-tuple}
Fusion for \textsc{s-elim} on tuples:
\begin{align*}
\StE{(x : s, t)}{(M, M'[M/x])}{w}{N} \equiv \StE{s}{M}{x}{\StE{t[\TrPlus{s}{x}/x]}{M'[\StI{s}{x}/x]}{y}{N[(x,y)/w]}}
\end{align*}
\end{lemma}
\begin{proof}
\begin{align*}
&\StE{(x : s, t)}{(M, M'[M/x])}{w}{N} \\
&\equiv \StE{s}{M}{x}{\StE{(s, t)}{(\StI{s}{x}, M'[\StI{s}{x}/x])}{w}{N}} \\
&\equiv \StE{s}{M}{x}{(\StE{t[\TrPlus{s}{x}/x]}{M'[\StI{s}{x}/x]}{y}{\StE{(s, t)}{(\StI{s}{x}, \StI{t[\TrPlus{s}{x}/x]}{y})}{w}{N}})} \\
&\equiv \StE{s}{M}{x}{(\StE{t[\TrPlus{s}{x}/x]}{M'[\StI{s}{x}/x]}{y}{\StE{(s, t)}{\StI{(x : s, t)}{(x,y)}}{w}{N}})} \\
&\equiv \StE{s}{M}{x}{\StE{t[\TrPlus{s}{x}/x]}{M'[\StI{s}{x}/x]}{y}{N[(x,y)/w]}}
\end{align*}
\end{proof}

%\begin{lemma}
%\textsc{F-elim} fuses with \textsc{s-intro}.
%\end{lemma}
%\begin{proof}
%\begin{align*}
%&\FEs{\TrPlus{s}{\mu}}{\StI{s}{M}}{x}{N} \\
%&\equiv \StE{s}{\StI{s}{M}}{y}{(\FEs{\mu}{y}{x}{N})} \\
%&\equiv \FEs{\mu}{M}{x}{N}
%\end{align*}
%and
%\begin{align*}
%& \FEs{\mu[\TrPlus{s}{x}/x]}{M}{x}{N[\StI{s}{x}/x]} \\
%&\equiv \FEs{\mu}{M}{y}{(\StE{s}{y}{x}{N[\StI{s}{x}/x]})}  \\
%&\equiv \FEs{\mu}{M}{y}{N[y/x]} \\
%&\equiv \FEs{\mu}{M}{x}{N}
%\end{align*}
%\end{proof}

\begin{lemma}
Split for telescope types is derivable:
\begin{mathpar}
\inferrule*[Left=$()$-split]{\Gamma, w : \telety{x}{A}{B} \yields_r C \TYPE \\\\ \Gamma \yields_{\nu} M : \telety{x}{A}{B} \\\\ \Gamma, x : A, y : B \yields_{\nu'[(x,y)/w]} N : C[(x,y)/w]}{\Gamma \yields_{\nu'[\nu/w]} \TeleE{M}{x}{y}{N} : C[M/w]}
\end{mathpar}
\end{lemma}
\begin{proof}
\begin{mathpar}
\inferrule*[Left=cut]{\Gamma, x : A, y : B \yields_{\nu'[(x,y)/w]} N : C[(x,y)/w]}{\Gamma \yields_{\nu'[\nu/w]} N[\fst M/x, \snd M/y] : C[(x, y)/w][\fst M/x, \snd M/y]}
\end{mathpar}
And by eta for telescope types, $C[(x, y)/w][\fst M/x, \snd M/y] \equiv C[(\fst M, \snd M)/w] \equiv C[M/w]$.
\end{proof}

Given a term such as $\beta : \St{s}{\telety{\alpha}{A}{B}}$, we will often use $s$-elimination and then split on the pair. For clarity, rather than writing
\begin{align*}
\StE{s}{\beta}{w}{\TeleE{w}{x}{y}{(\dots)}}
\end{align*}
we will write
\begin{align*}
\StE{s}{\beta}{x, y}{(\dots)}
\end{align*}

\begin{lemma}\label{lem:s-elim-telescope-type}
\textsc{s-elim} into types commutes with telescope formation and $\mathsf{U}$-formation, in the sense:
\begin{align*}
\StE{s}{M}{x}{(y : A, B)} &\equiv (y : \StE{s}{M}{x}{A}, \StE{(s, \id)}{(M,y)}{x, y}{B}) \\
\StE{s}{M}{x}{\U{c.\mu'[\TrPlus{s}{x}/x]}{y : A}{B}} &\equiv \U{c.\mu'[\mu/x]}{y : \StE{s}{M}{x}{A}}{\StE{(s, \id)}{(M,y)}{x, y}{B}}
\end{align*}
\end{lemma}
\begin{proof}
For telescopes:
\begin{align*}
&\StE{s}{M}{x}{(y : A, B)} \\
&\equiv \StE{s}{M}{x}{(y : \StE{s}{\StI{s}{x}}{x'}{A[x'/x]}, \StE{(s, \id)}{\StI{(s, \id)}{x, y}}{x', y'}{B[x'/x, y'/y]})} \\
&\equiv \StE{s}{M}{x}{(y : \StE{s}{\StI{s}{x}}{x'}{A[x'/x]}, \StE{(s, \id)}{(\StI{s}{x},y)}{x', y'}{B[x'/x, y'/y]})} \\
&\equiv (y : \StE{s}{M}{x'}{A[x'/x]}, \StE{(s, \id)}{(M,y)}{x', y'}{B[x'/x, y'/y]}) \\
&\equiv (y : \StE{s}{M}{x}{A}, \StE{(s, \id)}{(M,y)}{x, y}{B})
\end{align*}
and for $\mathsf{U}$-types the proof is similar.
\end{proof}

\begin{lemma}\label{lem:ctxtuple}
Any context $\Gamma$ can be tupled into an iterated telescope type $\ctxtuple{\Gamma}$ of mode $\ctxtuple{\gamma}$, so that substitutions $\Gamma \yields \Theta : \Delta$ correspond bijectively with terms of $\ctxtuple{\Delta}$.
\begin{mathpar}
\inferrule*{\yields_\gamma \Gamma}
             {\cdot \yields_{\ctxtuple{\gamma}} \ctxtuple{\Gamma} \TYPE}
\and
\inferrule*{~}
             {\sigma : \ctxtuple{\Gamma} \yields_{\unpack{\gamma}{\sigma}} \unpack{\Gamma}{\sigma} : \Gamma}
\and
\inferrule*{~}
             {\Gamma \yields_{\pack{\gamma}} \pack{\Gamma} : \ctxtuple{\Gamma}}
\\
\inferrule*{~}
             {\Gamma \yields \unpack{\Gamma}{\pack{\Gamma}} \equiv \id_\Gamma}
\and
\inferrule*{~}
             {\sigma : \ctxtuple{\Gamma} \yields \pack{\Gamma}[\unpack{\Gamma}{\sigma}] \equiv \sigma}
\end{mathpar}
\end{lemma}
\begin{proof}
$\ctxtuple{\Gamma}$ is defined inductively by
\begin{align*}
\ctxtuple{\cdot} &:\equiv 1 \\
\ctxtuple{\Gamma, x : A} &:\equiv \sigmacl{\sigma}{\ctxtuple \Gamma}{A[\unpack{\Gamma}{\sigma}]}\\
\unpack{(\cdot)}{\sigma} &:\equiv \cdot \\
\unpack{\Gamma, x : A}{\sigma} &:\equiv \unpack{\Gamma}{\fst \sigma}, \snd{\sigma}/x
\end{align*}
with each definition lying over the identical one downstairs. 
For $\pack{\Gamma}$ we simultaneously verify the equation $ \unpack{\Gamma}{\pack{\Gamma}} \equiv \id_\Gamma$, as we need it to hold for $(\pack{\Gamma}, x)$ to be well typed.
\begin{align*}
\pack{(\cdot)} &:\equiv \mt : 1 \\
\pack{\Gamma, x : A} &:\equiv (\pack{\Gamma}, x) : \sigmacl{\sigma}{\ctxtuple \Gamma}{A[\unpack{\Gamma}{\sigma}]} \\
\unpack{(\cdot)}{\pack{(\cdot)}} &\equiv  \cdot \equiv \id_{(\cdot)}\\
\unpack{\Gamma, x : A}{\pack{\Gamma, x : A}} 
&\equiv \unpack{\Gamma}{\fst \pack{\Gamma, x : A}}, \snd{\pack{\Gamma, x : A}}/x \\
&\equiv \unpack{\Gamma}{\pack{\Gamma}}, x/x \\
&\equiv \id_\Gamma, x/x \\
&\equiv \id_{\Gamma, x : A}
\end{align*}
For the second equation, we check inductively that
\begin{align*}
\pack{(\cdot)}[\unpack{(\cdot)}{\sigma}] 
&\equiv \mt[\id_{(\cdot)}] \\
&\equiv \sigma \\
\pack{\Gamma, x : A}[\unpack{\Gamma, x : A}{\sigma}]
&\equiv (\pack{\Gamma}, x)[\unpack{\Gamma}{\fst \sigma}, \snd{\sigma}/x] \\
&\equiv (\pack{\Gamma}[\unpack{\Gamma}{\fst \sigma}], \snd \sigma) \\
&\equiv (\fst \sigma, \snd \sigma) \\
&\equiv \sigma
\end{align*}
\end{proof}

%\begin{lemma}
%Tupling respects substitution: for $\gamma \yields \theta : \delta$ and $\delta \yields \kappa : \lambda$, we have:
%\begin{align*}
%\ctxtuple{(\kappa[\theta])} &\equiv (\ctxtuple{\kappa})[\theta]
%\end{align*}
%\end{lemma}
%\begin{proof}
%By induction on the length of $\lambda$:
%\begin{align*}
%\ctxtuple{((\cdot)[\theta])}
%&\equiv \ctxtuple{(\cdot)} \\
%&\equiv (\ctxtuple{(\cdot)})[\theta] \\
%\ctxtuple{((\kappa, M)[\theta])}
%&\equiv \ctxtuple{(\kappa[\theta], M[\theta])} \\
%&\equiv \ctxtuple{(\kappa[\theta]), M[\theta]} \\
%&\equiv (\ctxtuple \kappa)[\theta], M[\theta] \\
%&\equiv (\ctxtuple\kappa, M)[\theta]
%\end{align*}
%\end{proof}

%\begin{definition}
%A 2-cell between mode substitutions of shape $\gamma \yields t : \theta \tcell_\delta \theta' $ is specified by a mode term 2-cell \[\gamma \yields \ctxtuple t : \ctxtuple \theta \tcell_{\ctxtuple \delta} \ctxtuple \theta'. \]
%\end{definition}
%\mvrnote{A little confusing: for contexts and substitutions $\ctxtuple{}$ is an operation, here $\ctxtuple t$ is not an operation on $t$ but rather the underlying `implementation' of $t$.}

%\begin{lemma}\label{lem:n-ary-ap-rewrite}
%N-ary ap and rewrite are admissible:
%\begin{mathpar}
%\inferrule*{\delta \vdash {q} \type \\
%            \gamma \yields t : \theta \tcell_\delta \theta'
%           } 
%           {\TypeTwo{\gamma}{\ap {q} {t}}{q[\theta]}{q[\theta']}}
%\and
%\inferrule*{\delta \yields {\nu} : {q} \\
%            \gamma\yields t : \theta \tcell \theta'
%           } 
%           {\TermTwoT{\gamma}{\ap \nu {t}}{\nu[\theta]}{\TrPlus{\ap{q}{t}}{\nu[\theta]}}{q[\theta]}} \\
% \inferrule*[Left = rewrite]{
%   \Gamma \yields_{\theta'} \Theta : \Delta \and 
%   \gamma \yields t : \theta \tcell \theta'
%  }
%  {\Gamma \yields_{\theta} \rewrite{t}{\Theta} : \Delta}
%\end{mathpar}
%\end{lemma}
%\begin{proof}
%These are
%\begin{align*}
%\ap {q} {t} &:\equiv \ap{q[\unpack{\delta}{\sigma}]}{\ctxtuple t/\sigma} \\
%\ap {\nu} {t} &:\equiv \ap{\nu[\unpack{\delta}{\sigma}]}{\ctxtuple t/\sigma} \\
%\rewrite{t}{\Theta} &:\equiv \unpack{\Delta}{\rewrite{\ctxtuple t}{\ctxtuple \Theta}}
%\end{align*}
%which are well-typed by the equation $\theta \equiv \unpack{\delta}{\ctxtuple \theta}$.
%\end{proof}
%
%In particular, we have the following rules for building and using 2-cells between substitutions.
%\begin{mathpar}
%\inferrule{\gamma \yields t : \theta \tcell_\delta \theta' \and \gamma \yields s : \mu[\theta] \tcell_{p[\theta]} \TrPlus{\ap{p}{t}}{\mu'[\theta']}}
%{ \gamma \yields (s, t) : (\theta, \mu) \tcell_{(\delta,x:p)} (\theta', \mu')} \\
%%
%\inferrule{\gamma \yields t : (\theta, \mu) \tcell_{(\delta, x : p)} (\theta', \mu')}
%{\gamma \yields \ap{\fst}{t} : \theta \tcell_\delta \theta' } \and
%%
%\inferrule{\gamma \yields t : (\theta, \mu) \tcell_{(\delta, x : p)} (\theta', \mu')}
%{\gamma \yields \ap{\snd}{t} : \mu[\theta] \tcell_{p[\theta]} \TrPlus{\ap{p}{\ap{\fst}{t}}}{\mu'[\theta']}}
%\end{mathpar}%
%
%\begin{lemma}
%N-ary ap of a 2-cell on a substitution is admissible
%\begin{mathpar}
%\inferrule{\gamma \yields t : \theta \tcell_\delta \theta' \and \delta \yields \kappa : \lambda}
%{\gamma \yields \ap{\kappa}{t} : \kappa[\theta] \tcell_\lambda \kappa[\theta']}
%\end{mathpar} 
%\end{lemma}
%\begin{proof}
%Due to the equation $\ctxtuple(\kappa[\theta]) \equiv (\ctxtuple \kappa)[\theta]$ we can just define $\ctxtuple{(\ap{\kappa}{t})} :\equiv \ap{(\ctxtuple \kappa)}{t}$.
%\end{proof}

%\begin{lemma}
%N-ary associativity and interchange hold:
%\begin{align*}
%\ap {(\nu[\theta])} {t} &\equiv \ap \nu {\ap \theta {t}} \\
%s[\theta];\ap{\nu'}{t} &\equiv \ap{\nu}{t};\ap{(\TrPlus{\ap{q}{t}}{y})}{s[\theta']/y}
%\end{align*}
%\end{lemma}
%\begin{proof}
%Unwinding definitions we find:
%\begin{align*}
%\ap {(\nu[\theta])} {t}
%&\equiv \ap{\nu[\theta][\fan{\delta}]}{\ctxtuple t/\sigma} \\
%&\equiv \ap{\nu[\fan{\delta}][\ctxtuple \theta / \sigma][\fan{\gamma}]}{\ctxtuple t/\sigma} \\
%&\equiv \ap{\nu[\fan{\delta}]}{\ap{\ctxtuple \theta[\fan{\gamma}]}{\ctxtuple t/\sigma}/\sigma} \\
%&\equiv \ap{\nu[\fan{\delta}]}{\ap{(\ctxtuple \theta)}{t}/\sigma} \\
%&\equiv \ap{\nu[\fan{\delta}]}{\ctxtuple{(\ap \theta {t})}/\sigma} \\
%&\equiv \ap \nu {\ap \theta {t}}
%\end{align*}
%and
%\begin{align*}
%s[\theta];\ap{\nu'}{t} 
%&\equiv s[\theta];\ap{\nu'[\fan{\delta}]}{\ctxtuple t/\sigma} \\
%&\equiv \ap{\nu[\fan{\delta}]}{\ctxtuple t/\sigma} ; \ApPlus{(\ap{q[\fan{\delta}]}{\ctxtuple t/\sigma})}{s[\theta']} \\
%&\equiv \ap{\nu}{t};\ap{(\TrPlus{\ap{q}{t}}{y})}{s[\theta']/y}
%\end{align*}
%\end{proof}

%\begin{lemma}
%A N-ary versions of the equations concerning rewrites hold:
%\begin{align*}
%\St{(\ap{q}{t})}{B[\Theta]} &\equiv B[\rewrite{t}{\Theta}] \\
%\rewrite{\ap{\nu}{t}}{\StI{\ap{q}{t}}{N[\Theta]}} &\equiv N[\rewrite{t}{\Theta}] \\
%\rewrite{\ap{\kappa}{t}}{\Theta;\kappa} &\equiv \rewrite{t}{\Theta};\kappa
%\end{align*}
%\end{lemma}
%\begin{proof}
%\begin{align*}
%B[\rewrite{t}{\Theta}] 
%&\equiv B[\fan{\Delta}[\rewrite{\ctxtuple t}{\ctxtuple \Theta}/\sigma]] \\
%&\equiv B[\fan{\Delta}][\rewrite{\ctxtuple t}{\ctxtuple \Theta}/\sigma] \\
%&\equiv \St{\ap{q[\fan{\delta}]}{\ctxtuple t / \sigma}}{B[\fan{\Delta}][\ctxtuple \Theta/\sigma]} \\
%&\equiv \St{\ap{q}{t}}{B[\Theta]}
%\end{align*}
%And:
%\begin{align*}
%N[\rewrite{t}{\Theta}]
%&\equiv N[\fan{\Delta}[\rewrite{\ctxtuple t}{\ctxtuple \Theta}/\sigma]] \\
%&\equiv N[\fan{\Delta}][\rewrite{\ctxtuple t}{\ctxtuple \Theta}/\sigma] \\
%&\equiv \rewrite{\ap{\nu[\fan{\delta}]}{\ctxtuple t/\sigma}}{\StI{\ap{q[\fan{\delta}]}{\ctxtuple t/\sigma}}{N[\fan{\Delta}][\ctxtuple \Theta/\sigma]}} \\
%&\equiv \rewrite{\ap{\nu}{t}}{\StI{\ap{q}{t}}{N[\Theta]}} \\
%\end{align*}
%And:
%\begin{align*}
%\rewrite{\ap{\kappa}{t}}{\Theta;\kappa}
%&\equiv \fan{\Lambda}[\rewrite{\ctxtuple{(\ap{\kappa}{t})}}{\ctxtuple {(\Theta;\kappa)}}/\sigma] \\
%&\equiv \fan{\Lambda}[\rewrite{\ap{(\ctxtuple \kappa)}{t}}{(\ctxtuple \kappa)[\Theta]}/\sigma] \\
%&\equiv \fan{\Lambda}[(\ctxtuple \kappa)[\rewrite{t}{\Theta}]/\sigma] \\
%&\equiv \fan{\Lambda}[(\ctxtuple \kappa)/\sigma][\rewrite{t}{\Theta}] \\
%&\equiv \kappa[\rewrite{t}{\Theta}] \\
%&\equiv \rewrite{t}{\Theta};\kappa
%\end{align*}
%using the previous equation.
%\end{proof}

%\begin{lemma}
%Rewriting by a tuple is a tuple of rewritings:
%\begin{align*}
%\rewrite{(t, s/x)}{\Theta, M/x} \equiv (\rewrite{t}{\Theta}, \rewrite{s}{\StI{\ap{p}{t}}{M}})
%\end{align*}
%\end{lemma}
%\begin{proof}
%Unwinding definitions:
%\begin{align*}
%\rewrite{(t, s/x)}{\Theta, M/x}
%&\equiv \fan{\Delta, x : A}[\rewrite{\ctxtuple{(t, s/x)}}{\ctxtuple{(\Theta, M/x)}}/\sigma] \\
%&\equiv (\fan\Delta[\fst{\sigma}/\sigma], \snd{\sigma})[\rewrite{(\ctxtuple t, s)}{\ctxtuple\Theta, M}/\sigma] \\
%&\equiv (\fan\Delta[\fst{\sigma}/\sigma], \snd{\sigma})[(\rewrite{\ctxtuple t}{\ctxtuple\Theta}, \rewrite{s}{\StI{\ap{p[\fan{\delta}]}{\ctxtuple t/\sigma}}{N}})/\sigma] \\
%&\equiv (\fan\Delta[\rewrite{\ctxtuple t}{\ctxtuple\Theta}/\sigma], \rewrite{s}{\StI{\ap{p[\fan{\delta}]}{\ctxtuple t/\sigma}}{N}}) \\
%&\equiv (\rewrite{t}{\Theta}, \rewrite{s}{\StI{\ap{p}{t}}{M}})
%\end{align*}
%\end{proof}
%
%As a special case, note that we have
%\begin{align*}
%\rewrite{(\id_\Theta, s/x)}{\Theta, M/x} \equiv (\Theta, \rewrite{s}{M}/x)
%\end{align*}
%
%\mvrnote{Need to say something like: Because the above $n$-ary versions have been derived, from now on we use $\ap{\mu}{s/x, t/y}$ as shorthand for the $n$-ary version}

\subsection{Examples}

\begin{itemize}
\item 
The previous two-argument \F{\mu}{x:A,y:B} (for $x :p, y:q \vdash \mu :
r$) is now \F{z.\mu[\fst z/x,\snd z/y]}{\telety{x}{A}{B}}, using the
upstairs telescope type ${\telety{x}{A}{B}}$, which has mode
$\sigmacl{x}{p}{q}$.  Iterating $\telety{x}{A}{B}$ plays the role of a
longer telescope.  
\end{itemize}

\section{Mode Theories}

\mvrnote{At some point near the start we should say:
For clarity we will often treat mode terms with a distinguished parameter as though they were functions. So if $\gamma, x : p \yields f : q$ we write $f(m)$ rather than $f[m/x]$.
}

\subsection{Adjoint Mode Terms}
\begin{definition}
A mode term $\gamma, x : p \yields f : q$ is \emph{left adjoint} to $\gamma, y : q \yields u : p$ if there are specified constants:
\begin{align*}
\gamma, x : p &\yields f : q \\
\gamma, x : p &\yields \eta_x : x \tcell_p u(f(x)) \\
\gamma, y : q &\yields \varepsilon_y : f(u(y)) \tcell_q y
\end{align*}
such that $\eta$ and $\varepsilon$ are natural:
\begin{align}
\eta_x ; \ap{u}{\ap{f}{s}} &\equiv s ; \eta_{x'} && \text{for } s : x \tcell_p x'  \\
\ap{f}{\ap{u}{t}} ; \varepsilon_{y'}  &\equiv \varepsilon_y ; t && \text{for } t : y \tcell_q y'
\end{align}
and the triangle identities hold:
\begin{align}
\eta_{u(\nu)};\ap{u}{\varepsilon_\nu} &\equiv \id_{u(\nu)} \\
\ap{f}{\eta_\mu};\varepsilon_{f(\mu)} &\equiv \id_{f(\mu)}
\end{align}
\end{definition}

\subsection{Terminal Objects}

\begin{definition}
For $p \type$, the term $\emptyset : p$ is \emph{terminal} if there are specified mode term morphisms
\begin{mathpar}
\TermTwoT{x : p}{!_x}{x}{\emptyset}{p}
\end{mathpar}
such that $t \equiv \bang_x$ for all $\TermTwoT{\gamma}{t}{\mu}{\emptyset}{p}$.
\end{definition}

%\begin{definition}
%For a dependent mode $x : p \vdash S(x) \type$, a \emph{fibred terminal object} term is specified by:
%\begin{mathpar}
%x : p \vdash \One_x : S(x) \and
%\TermTwoT{x : p, \mu : S(x)}{!_\mu}{\mu}{\One_x}{S(x)}
%\end{mathpar}
%such that
%\begin{align}
%\label{bang-unique}
%t & \equiv \bang_\mu && \text{where } \TermTwoT{\gamma}{t}{\mu}{\One_\alpha}{S(\alpha)}\\
%\label{s-plus-one-strict}
%\TrPlus{\ap{S}{s}}{\One_\alpha} &\equiv \One_\beta && \text{where } \TermTwoT{\gamma}{s}{\beta}{\alpha}{p}
%\end{align}
%\end{definition}

\subsection{Comprehension Object}

\begin{definition}[Comprehension Object]\label{def:comprehension-object}
  A \emph{comprehension object} is specified by the following
  constants:
  \begin{mathpar}
    p \type \and \alpha : p \yields \El{p}{\alpha} \type \and \alpha : p \yields \One_\alpha : \El{p}{\alpha}
    \\ 
    \TypeTwo{\cdot}{\tdot}{p}{\sigmacl{\alpha}{p}{\El{p}{\alpha}}} \and
    \TypeTwo{\cdot}{\tempty}{p}{1}
  \end{mathpar}
  such that
\begin{align*}
\alpha : p &\yields (\alpha, \One_\alpha) : \sigmacl{\alpha}{p}{\El{p}{\alpha}} \\
\alpha : p &\yields () : 1
\end{align*}
are left adjoints to $\tdot^+$ and $\tempty^+$ respectively.
\end{definition}

In particular, a comprehension object comes equipped with natural mode term morphisms
\begin{align*}
\eta^\tdot_\alpha {}&: \alpha \tcell_p \TrPlus{\tdot}{(\alpha, \One_\alpha)} \\
\varepsilon^\tdot_{(\alpha, \mu)} {}&: (\TrPlus{\tdot}{(\alpha, \mu)}, \One_{\TrPlus{\tdot}{(\alpha, \mu)}}) \tcell_{\sigmacl{\alpha}{p}{\El{p}{\alpha}}} (\alpha, \mu) \\
\eta^\tempty_\alpha {}&: \alpha \tcell_p \TrPlus{\tempty}{()} \\
\varepsilon^\tempty_{x} {}&: x \tcell_1 ()
\end{align*}
satisfying the triangle identities. (Note that $\varepsilon^\tempty_x$ is necessarily the unique morphism to $()$)

\begin{definition}
Any comprehension object supports the following derived forms (where $\gamma \yields \alpha : p$
and $\gamma \yields \beta : p$ and
$\TermTwoT{\gamma}{s}{\beta}{\alpha}{p}$ and $\gamma \yields \mu :
\alpha$ and $\gamma \yields \nu : \beta$):
  \begin{itemize}
  \item A distinguished term $\emptyset : p$ is given by
  \begin{mathpar}
  \cdot \yields \emptyset : p \and \emptyset :\equiv \TrPlus{\tempty}{()}
  \end{mathpar}
  \item Comprehension is given by $\tdot^+$:
  \begin{mathpar}
  \alpha : p, x : \El{p}{\alpha} \yields \alpha.x : p \and \alpha.x :\equiv \TrPlus{\tdot}{(\alpha, x)}
  \end{mathpar}
  \item $\pi^\alpha_\mu$ and $\var{\mu}$ are defined via the counit of the adjunction $\tdot^\circ \dashv \tdot^+$.
  \begin{mathpar}
  {\TermTwoT{\alpha:p,x:\El{p}{\alpha}}{\pi^\alpha_x}{\alpha.x}{\alpha}{p}}
  \and
  \pi^\alpha_x :\equiv \ap \fst {\varepsilon^\tdot_{(\alpha, x)}} \\
  {\TermTwoT{\alpha:p,x:\El{p}{\alpha}}{\var{x}}{\One_{\alpha.x}}{\TrPlus{\ApEl{p}{\pi^\alpha_x}}{x}}{\El{p}{\alpha.x}}} 
    \and
    \var{x} :\equiv \ap \snd {\varepsilon^\tdot_{(\alpha, x)}}
  \end{mathpar}
  \item Pairing for the comprehension object is given by ap of
  $.$ on the pairing for telescope modes:
  \begin{mathpar}
  \inferrule{\TermTwoT{\gamma}{s}{\alpha}{\beta}{p} \and
             \TermTwoT{\gamma}{m}{\mu}{\TrPlus{\ApEl{p}{s}}{\nu}}{\El{p}{\alpha}}}
            {\TermTwoT{\gamma}{s.m}{\alpha.\mu}{\beta.\nu}{p}} \and
  s \bdot m :\equiv \ApPlus{\tdot}{(s, m)}
  \end{mathpar}
  \item We write $\ApOne{s}$ as a short hand for $\ap{\One_z}{s/z}$
  \end{itemize}
\end{definition}

This definition is a type-theoretic analogue of a fibration that \emph{admits comprehension}. This is a fibration $E \to B$ with fibred terminal object $1 : B \to E$ such that the functor $1$ has a right adjoint, see \mvrnote{ref}.

In the type-theoretic definition, $p$ and $\El{p}{\alpha}$ describe a fibration $\fst : \sigmacl{\alpha}{p}{\El{p}{\alpha}} \to p$. The term $(\alpha, \One_\alpha)$ describes a functor $p \to \sigmacl{\alpha}{p}{\El{p}{\alpha}}$ that sends $\alpha$ to a distinguished object $\One_\alpha$ in the fibre over $\alpha$, and this functor has a right adjoint $\TrPlus{\tdot}{\alpha,x}$.

The primary difference is that we do not need to assume that $\One_\alpha$ is terminal in $\El{p}{\alpha}$, so $(\alpha, \One_\alpha) : p \to \sigmacl{\alpha}{p}{\El{p}{\alpha}}$ is not right adjoint to $\fst : \sigmacl{\alpha}{p}{\El{p}{\alpha}} \to p$. We also do not assume that $p$ is full, in the sense of there being a bijection between mode term morphisms $x \tcell_{\El{p}{\alpha}} y$ and mode term morphisms $\alpha.x \tcell_p \alpha.y$ that commute with projection.

Using the above derived forms, the units and counits of the adjunctions have type
\begin{align*}
\eta^\tdot_\alpha {}&: \alpha \tcell_p \alpha.\One_\alpha \\
\varepsilon^\tdot_{(\alpha, \mu)} {}&: (\alpha.\mu, \One_{\alpha.\mu}) \tcell_{\sigmacl{\alpha}{p}{\El{p}{\alpha}}} (\alpha, \mu) \\
\eta^\tempty_\alpha {}&: \alpha \tcell_p \emptyset \\
\varepsilon^\tempty_{x} {}&: x \tcell_1 () \\
\end{align*}
and the triangle equations may be written:
\begin{align}
\label{eq:chi-triangle-1} \eta^\tdot_{\alpha.x};(\pi_x^\alpha \bdot \var{x}) &\equiv \id_{\alpha.x} \\
\label{eq:chi-triangle-2} (\eta^\tdot_\alpha ; \pi^\alpha_{\One_\alpha}, \ApOne{\eta^\tdot_\alpha} ; \ApPlus{\ApEl{p}{\eta^\tdot_\alpha}}{\var{\One_\alpha}}) &\equiv \id_{(\alpha, \One_\alpha)}\\
\eta^\tempty_\emptyset &\equiv \id_\emptyset \\
\id_{()}&\equiv \id_{()}
\end{align}

Requiring a left adjoint for $\tempty^+$ is a cute way of stating the following:
\begin{lemma}
For $p$ a comprehension object, $\emptyset : p$ is terminal.
\end{lemma}
\begin{proof}
We always have a map $\eta^\tempty_\alpha : \alpha \tcell_p \emptyset$. To show it is unique, note that for any $s : \alpha \tcell_p \emptyset$,
\begin{align*}
s 
&\equiv s ; \eta^\tempty_\emptyset \\
&\equiv \eta^\tempty_\alpha ; \ApPlus{\tempty}{\ApCirc{\tempty}{s}} \\
&\equiv \eta^\tempty_\alpha ; \ApPlus{\tempty}{\id_{()}} \\
&\equiv \eta^\tempty_\alpha
\end{align*}
by naturality of $\eta^\tempty$.
\end{proof}

Some other derivable equations:
\begin{itemize}
\item Fusion for $.$
\begin{align}
\label{dot-fusion}
    (s \bdot m);(s' \bdot m') \equiv ((s;s') \bdot (m;\ApPlus{\ApEl{p}{s}} {m'}))
\end{align}
follows by fusing the ap's and the corresponding fusion rule for morphisms in telescope modes:
\begin{align*}
(s \bdot m);(s' \bdot m') &\equiv \ApPlus{\tdot}{(s, m)} ; \ApPlus{\tdot}{(s', m')} \\
&\equiv \ApPlus{\tdot}{(s, m);(s', m')} \\
&\equiv ((s;s') \bdot (m;\ApPlus{\ApEl{p}{s}} {m'}))
\end{align*}

\item $\pi^\alpha_\mu$ is natural in $\mu$:
  \begin{align}
  \label{pi-naturality}
  (s \bdot m); \pi^\alpha_\mu &\equiv \pi^\beta_\nu;s && \text{where } \TermTwoT{\gamma}{m}{\mu}{\TrPlus{\ApEl{p}{s}}{\nu}}{\El{p}{\alpha}}
  \end{align}
  by
  \begin{align*}
  (s \bdot m); \pi^\alpha_\mu 
  &\equiv \ApPlus{\tdot}{(s, m)} ; \ap \fst {\varepsilon^\tdot_{(\alpha, \mu)}} \\  
  &\equiv \ap{\fst}{\ApCirc{\tdot}{\ApPlus{\tdot}{(s, m)}}} ; \ap \fst {\varepsilon^\tdot_{(\alpha, \mu)}} \\
  &\equiv \ap{\fst}{\ApCirc{\tdot}{\ApPlus{\tdot}{(s, m)}} ; \varepsilon^\tdot_{(\alpha, \mu)}}  \\
  &\equiv \ap{\fst}{\varepsilon^\tdot_{(\beta, \nu)}; (s, m) } \\
  &\equiv \ap{\fst}{\varepsilon^\tdot_{(\beta, \nu)}} ; s\\
  &\equiv \pi^\beta_\nu ; s
  \end{align*}

\item Beta reduction for first projection:
  \begin{align}
\label{beta-pi}
\eta_\beta;(s \bdot m);\pi^\alpha_\mu &\equiv s && \text{where } \TermTwoT{\gamma}{m}{\One_\alpha}{\TrPlus{\ApEl{p}{s}}{\mu}}{\alpha}
  \end{align}
follows from naturality and the second triangle equation by:
\begin{align*}
\eta_\beta;(s \bdot m);\pi^\alpha_\mu
&\equiv \eta_\beta;\pi^\beta_{\One_\beta};s \\
&\equiv s
\end{align*}

\item Naturality of $\var{}$:
\begin{align}
\label{beta-var}
\ApOne{(s \bdot m)};\ApPlus{\ApEl{p}{(s \bdot m)}}{\var{\mu}} &\equiv \var{\nu};\ApPlus{\ApEl{p}{\pi^\beta_\nu}}{m}  && \text{where } \TermTwoT{\gamma}{m}{\nu}{\TrPlus{\ApEl{p}{s}}{\mu}}{\El{p}{\beta}}
\end{align}
is derivable by:
\begin{align*}
\ApOne{(s \bdot m)};\ApPlus{\ApEl{p}{(s \bdot m)}}{\var{\mu}} 
&\equiv \ApOne{(s \bdot m)};\ApPlus{\ApEl{p}{\ApPlus{\tdot}{(s, m)}}}{\ap \snd {\varepsilon^\tdot_{(\alpha, \mu)}}} \\
&\equiv \ap \snd {((s \bdot m), \ApOne{(s \bdot m)});\varepsilon^\tdot_{(\alpha, \mu)}} \\
&\equiv \ap \snd {\ap{\TrCirc{\tdot}{\TrPlus{\tdot}{z}}}{(s, m)/z};\varepsilon^\tdot_{(\alpha, \mu)}} \\
&\equiv \ap \snd {\varepsilon^\tdot_{(\beta, \nu)};(s, m)} \\
&\equiv \var{\nu}; \ApPlus{\ApEl{p}{\pi^\beta_{\nu}}}{m} 
\end{align*}

\item The eta principle for pairing:
\begin{align}
\label{eta-pi-var}
t &\equiv \eta_\beta;((t;\pi_\mu^\alpha) \bdot (\ApOne{t}; \ApPlus{\ApEl{p}{t}}{\var{\mu}})) && \text{where } \TermTwoT{\gamma}{t}{\beta}{\alpha.\mu}{p}
\end{align}
is derived by the first triangle equation followed by naturality of $\eta$:
\begin{align*}
t &\equiv t;\eta_{\alpha.\mu};(\pi_\mu^\alpha \bdot \var{\mu}) \\
&\equiv \eta_\beta;\ap{\TrPlus{\tdot}{\TrCirc{\tdot}{z}}}{t/z};(\pi_\mu^\alpha \bdot \var{\mu}) \\
&\equiv \eta_\beta;\ap{(z \bdot \One_z)}{t/z};(\pi_\mu^\alpha \bdot \var{\mu}) \\
&\equiv \eta_\beta;(t \bdot \ApOne{t});(\pi_\mu^\alpha \bdot \var{\mu}) \\
&\equiv \eta_\beta;((t;\pi_\mu^\alpha) \bdot (\ApOne{t}; \ApPlus{\ApEl{p}{t}}{\var{\mu}})) \\
\end{align*}

\item Naturality of $\eta^\tdot_\beta$ takes the following form:
\begin{align*}
s;\eta^\tdot_\mu \equiv \eta^\tdot_{\mu'} ; (s \bdot \ApOne{s})
\end{align*}

\end{itemize}

We think of a term of mode $p$ as a ``context'', a mode term morphism of
mode $p$ as a ``substitution'', a term of mode $\El{p}{\alpha}$ as a
``dependent type'', and a mode term morphism $\One_\alpha
\Yields_{\El{p}{\alpha}} \mu$ as a ``term'' of ``type'' $\mu$.  For the
equations: $\TrPlus{\ApEl{p}{s}}{-}$ is supposed to act like
substitution; equation \eqref{s-plus-one-strict} says that substitution
into the unit type gives the unit type in a different ``context''.

Equation \eqref{beta-var} says that ``substituting'' into a ``variable'' is second projection on the ``substitution''.  Equation \eqref{eta-pi-var} is the usual $\eta$ principle for subsitutions.

\begin{lemma}\label{sigma:total-to-fiber0} 
For any comprehension object $p$, mode term morphisms $s : \alpha \tcell_p \alpha.x$ such that $s;\pi^\alpha_x \equiv \id_\alpha$ correspond bijectively to 2-cells $\One_\alpha \tcell_{\El{p}{\alpha}} x$.
\end{lemma}
\begin{proof}
Given such an $s$, we can define
\begin{align*}
\hat{s} &: \One_\alpha \tcell_{\El{p}{\alpha}} x \\
\hat{s} &:\equiv \ApOne{s};\ApPlus{s}{\var{x}}
\end{align*}
where the identity $s;\pi^\alpha_x \equiv \id_\alpha$ is used to verify that the codomain is $\TrPlus{s}{\TrPlus{\pi^\alpha_x}{x}} \equiv \TrPlus{(s;\pi^\alpha_x)}{x} \equiv x$.

Conversely, given $m : \One_\alpha \tcell_{\El{p}{\alpha}} x$ we have:
\begin{align*}
\tilde{m} &: \alpha \tcell_p \alpha.x \\
\tilde{m} &:\equiv \eta^\tdot_\alpha ; (\id_\alpha \bdot m)
\end{align*}
And indeed $\tilde{m};\pi^\alpha_x \equiv \eta^\tdot_\alpha ; (\id_\alpha \bdot m);\pi^\alpha_x \equiv \id_\alpha$ by Equation~\eqref{beta-pi}.

It is easy to check the round trips are the identity:
\begin{align*}
&\eta^\tdot_\alpha ; (\id_\alpha \bdot \ApOne{s};\ApPlus{s}{\var{x}}) \\
&\equiv \eta^\tdot_\alpha ; (s;\pi^\alpha_x \bdot \ApOne{s};\ApPlus{s}{\var{x}}) \\
&\equiv \eta^\tdot_\alpha ; (s \bdot \ApOne{s});(\pi^\alpha_x \bdot \var{x}) \\
&\equiv s;\eta^\tdot_{\alpha.x} ; (\pi^\alpha_x \bdot \var{x}) \\
&\equiv s
\end{align*}
And:
\begin{align*}
&\ApOne{\eta^\tdot_\alpha ; (\id_\alpha \bdot m)};\ApPlus{(\eta^\tdot_\alpha ; (\id_\alpha \bdot m))}{\var{x}} \\
&\equiv \ApOne{\eta^\tdot_\alpha};\ApPlus{\eta^\tdot_\alpha}{\ApOne{(\id_\alpha \bdot m)};\ApPlus{(\id_\alpha \bdot m)}{\var{x}}} \\
&\equiv \ApOne{\eta^\tdot_\alpha};\ApPlus{\eta^\tdot_\alpha}{\var{\One_\alpha};\ApPlus{\pi^\alpha_{\One_\alpha}}{m}} \\
&\equiv \ApOne{\eta^\tdot_\alpha};\ApPlus{\eta^\tdot_\alpha}{\var{\One_\alpha}};\ApPlus{\eta^\tdot_\alpha}{\ApPlus{\pi^\alpha_{\One_\alpha}}{m}} \\
&\equiv \ApPlus{\eta^\tdot_\alpha}{\ApPlus{\pi^\alpha_{\One_\alpha}}{m}} \\
&\equiv m
\end{align*}
\end{proof}

\subsubsection{Comprehension Object with Unit}

\begin{definition}\label{def:supports-unit}
A comprehension object \emph{supports the unit type} if $\eta^\tdot_\alpha : \alpha \tcell_p \alpha.\One$ is an isomorphism, i.e.:
\begin{align}
\pi^\alpha_{\One_\alpha} ; \eta^\tdot_\alpha \equiv \id_{\alpha.\One_\alpha}
\end{align}
\end{definition}
Composition in the other direction is already the identity, by one of the triangle identities for $\tdot$.

\subsubsection{Comprehension Object with $\Sigma$}

\begin{definition}\label{def:supports-sigmas}
A comprehension object \emph{supports $\Sigma$ types} if there is a specified mode term
\begin{align*}
\alpha : p, x : \El{p}{\alpha}, y : \El{p}{\alpha.x} \vdash \Sigma_1(\alpha,x,y) : \El{p}{\alpha}
\end{align*}
and mode term morphisms
\begin{align*}
\contract{\alpha} &: \One_\alpha \tcell_{\El{p}{\alpha}} \Sigma_1(\alpha,\One_\alpha,\One_{\alpha.{\One_\alpha}}) \\
\tsplit{\alpha,x,y} &: \alpha.\Sigma_1(\alpha,x,y) \tcell_{p} \alpha.x.y
\end{align*}
such that $\tsplit{\alpha,x,y}$ is an inverse to $\pair{\alpha,x,y}$, defined by:
\begin{align*}
\fibpair{\alpha,x,y} &: \One_{\alpha.x.y} \tcell_{\El{p}{\alpha.x.y}} \TrPlus{(\pi^{\alpha.x}_y;\pi^\alpha_x)}{\Sigma_1(\alpha,x,y)} \\
\fibpair{\alpha,x,y} &:\equiv \contract{\alpha.x.y};\ap{\Sigma_1(\alpha,x,y)}{(\pi^{\alpha.x}_y;\pi^{\alpha}_x,\ApOne{\pi^{\alpha.x}_y};\ApPlus{\ApEl{p}{\pi^{\alpha.x}_y}}{\var{x}}, \ApOne{\pi^{\alpha.x.y}_{\One_{\alpha.x.y}}};\ApPlus{\ApEl{p}{\pi^{\alpha.x.y}_{\One_{\alpha.x.y}}}}{\var{y}})/(\alpha,x,y)} \\
\pair{\alpha,x,y} &: \alpha.x.y \tcell_{p} \alpha.\Sigma_1(\alpha,x,y) \\
\pair{\alpha, x, y} &:\equiv \eta^\tdot_{\alpha.x.y};((\pi^{\alpha.x}_y;\pi^\alpha_x) \bdot \fibpair{\alpha,x,y})
\end{align*}
and $\contract{\alpha}$ is natural:
\begin{align}
\contract{\alpha};\ap{\Sigma_1(\alpha,x,y)}{(s, \ApOne{s}, \ApOne{s \bdot \ApOne{s}})/(\alpha,x,y)} \equiv \ApOne{s};\ApPlus{\ApEl{p}{s}}{\contract{\beta}}
\end{align}
for any $s : \alpha \tcell_p \beta$.
\end{definition}

\mvrnote{How does this compare to left adjoint to weakening? I still don't see...}

Only $\contract{\alpha}$ is needed to define $\pair{\alpha,x,y}$, as mode terms are always functorial with respect to mode term morphisms via ap. If the requirement that $\pair{\alpha, x, y}$ be an isomorphism is dropped, we may still interpret weak $\Sigma$-types with a non-dependent eliminator.

\begin{lemma}
A comprehension object that supports $\Sigma$s satisfies the following equations:
\begin{align}
\pair{\alpha,x,y};\pi^\alpha_{\Sigma_1(\alpha, x, y)} &\equiv \pi^{\alpha.x}_y;\pi^\alpha_x \\
\tsplit{\alpha,x,y};\pi^{\alpha.x}_y;\pi^\alpha_x &\equiv \pi^\alpha_{\Sigma_1(\alpha, x, y)} %\\
%\fibpair{\alpha,\One_\alpha,\One_{\alpha.\One_\alpha}} &\equiv \ApOne{\pi^{\alpha.\One_\alpha}_{\One_{\alpha.\One_\alpha}};\pi^\alpha_{\One_\alpha}};\ApPlus{(\pi^{\alpha.\One_\alpha}_{\One_{\alpha.\One_\alpha}};\pi^\alpha_{\One_\alpha})}{\contract{\alpha}} \\
%(\id_\alpha \bdot \contract{\alpha});\tsplit{\alpha,\One_\alpha,\One_{\alpha.\One_\alpha}} &\equiv \eta^\tdot_{\alpha.\One_\alpha}
\end{align}
\end{lemma}
\begin{proof}
The first follows from the definition of $\pair{}$ and Equation~\ref{beta-pi}. The second is immediate from the first, precomposing with $\tsplit{}$. 
\end{proof}

\subsubsection{Comprehension Object with $\Pi$}

\begin{definition}\label{def:supports-pis}
A comprehension object \emph{supports $\Pi$s} if $\ApOne{\pi^\alpha_x}$ is an isomorphism.
\end{definition}

Let $\pinv{\alpha,x}$ denote the inverse of $\ApOne{\pi^\alpha_x}$.

\begin{lemma}
If $\One_\alpha : \El{p}{\alpha}$ is a fibred terminal object, then $p$ has $\Pi$s.
\end{lemma}
\begin{proof}
Take $\pinv{\alpha,x}$ to be the identity on $\One_{\alpha.x}$.
\end{proof}

\subsection{Morphism of Comprehension Objects}

Suppose we have two comprehension objects $p$ and $q$.

\begin{definition}\label{def:morphism-comprehension-object}
A \emph{morphism of comprehension objects} $f$ from $(p, \One^p, \tdot^p, \tempty^p)$ to $(q, \One^q, \tdot^q, \tempty^q)$ consists of constants
\begin{align*}
&\yields f : q \tcell p \\
\alpha : p, x : \El{p}{\alpha} &\yields f_1(x) : \El{q}{\TrPlus{f}{\alpha}} \\
\alpha : p, x : \El{p}{\alpha} &\yields \fone{\alpha} : \One^q_{\TrPlus{f}{\alpha}}  \tcell_{\El{q}{\TrPlus{f}{\alpha}}} f_1(\One^p_\alpha)
\end{align*}
such that $\fone{\alpha}$ is natural:
\begin{align}
\fone{\alpha};\ap{f_1}{s/\alpha, \ApOne{s}/x} \equiv \ApOne{\ApPlus{f}{s}};\ApPlus{\ApEl{q}{\ApPlus{f}{s}}}{\fone{\beta}}
\end{align}
\end{definition}

\begin{definition}
A morphism of comprehension objects \emph{supports right adjoint types} if $\fone{\alpha}$ is an isomorphism.
\end{definition}
Let $\foneinv{\alpha}$ denote the inverse of $\fone{\alpha}$.

\begin{definition}
A morphism of comprehension objects $f$ \emph{supports left adjoint types} if $\fdist{\alpha, x}$ defined by
\begin{align*}
\fibf{\alpha, x} &: \One_{\TrPlus{f}{\alpha.x}} \tcell_{\El{q}{\TrPlus{f}{\alpha.x}}} \ApPlus{\ApEl{q}{\ApPlus{f}{\pi^\alpha_x}}}{f_1(x)} \\
\fibf{\alpha, x} &:\equiv \fone{\alpha.x};\ap{f_1}{\pi^\alpha_x/\alpha, \var{x}/x} \\
\fdist{\alpha, x} &: \TrPlus{f}{\alpha.x} \tcell_q \TrPlus{f}{\alpha}.f_1(x) \\
\fdist{\alpha, x} &:\equiv \eta^{\tdot^q}_{\TrPlus{f}{\alpha.x}} ; (\ApPlus{f}{\pi^\alpha_x} \bdot \fibf{\alpha, x})
\end{align*}
is an isomorphism.
\end{definition}
Let $\fdistinv{\alpha,x}$ denote the inverse of $\fdist{\alpha,x}$.

\subsection{Spatial Type Theory}

Let $f$ be a morphism of comprehension objects from $p$ to itself. We give $\TrPlus{f}{-}$ the structure of a comonad as follows.

\begin{definition}
A mode term morphism $\gamma, x : p \yields \mu : p$ is a \emph{comonad} if there are morphisms
\begin{align*}
\gamma, x : p &\yields \fcounit{x} : \mu(x) \Rightarrow_p x \\
\gamma, x : p &\yields \fcomult{x} : \mu(x) \Rightarrow_p \mu(\mu(x))
\end{align*}
satisfying:
\begin{align}
\fcomult{x};\ap{\mu}{\fcomult{x}} &\equiv \fcomult{x};\fcomult{\mu(x)} \\
\fcomult{x};\ap{\mu}{\fcounit{x}} &\equiv \id_{\mu(x)} \\
\fcomult{x};\fcounit{\mu(x)} &\equiv \id_{\mu(x)}
\end{align}
A comonad is \emph{idempotent} if any of the following equivalent equations hold:
\begin{align}
\fcounit{\mu(x)} &\equiv \ap{\mu}{\fcounit{x}} \\
\ap{\mu}{\fcounit{x}} ; \fcomult{x} &\equiv \id_{\mu(\mu(x))} \\
\fcounit{\mu(x)} ; \fcomult{x} &\equiv \id_{\mu(\mu(x))} 
\end{align}
\end{definition}

\begin{definition}\label{def:supports-spatial}
A morphism of fibrations $f$ \emph{supports spatial type theory} if $\TrPlus{f}{-}$ is an idempotent comonad and $f$ supports left and right types.
\end{definition}

We would also like to talk about $\flat$-types for a non-idempotent comonad. 
\begin{definition}
A \mvrnote{copointed endofunctor I suppose} \emph{supports $\flat$-types} if there are constants
\begin{align*}
\alpha : p, x : \El{p}{\TrPlus{f}{\alpha}} &\yields \flat_1(x) : \El{p}{\TrPlus{f}{\alpha}} \\
\alpha : p &\yields \flatone{\alpha} : \One_{\TrPlus{f}{\alpha}} \tcell_{\TrPlus{f}{\alpha}} \flat_1(\One_{\TrPlus{f}{\alpha}})
\end{align*}
such that
\begin{align*}
\flatdist{\alpha, x} &: \TrPlus{f}{\TrPlus{f}{\alpha}.x} \tcell_p \TrPlus{f}{\alpha}.\flat_1(x) \\
\flatdist{\alpha, x} &:\equiv \eta^\tdot_{\TrPlus{f}{\TrPlus{f}{\alpha}.x}};((\fcounit{\TrPlus{f}{\alpha}.x};\pi^{\TrPlus{f}{\alpha}}_x) \bdot (\ApOne{\fcounit{\TrPlus{f}{\alpha}}};\ApPlus{\ApEl{p}{\fcounit{\TrPlus{f}{\alpha}}}}{\flatone{\alpha};\ap{\flat_1}{\pi^{\TrPlus{f}{\alpha}}_x/\alpha, \var{x}/x}}))
\end{align*}
is an isomorphism.
\end{definition}

\begin{lemma}
If $f$ supports spatial type theory then it supports $\flat$-types.
\end{lemma}
\begin{proof}
Define
\begin{align*}
\flat_1(x) &:\equiv \TrPlus{\ApEl{p}{\fcomult{\alpha}}}{f_1(x)} \\
\flatone{\alpha} &:\equiv \ApOne{\fcomult{\alpha}};\ApPlus{\ApEl{p}{\fcomult{\alpha}}}{\fone{\TrPlus{f}{\alpha}}}
\end{align*}

$\flatdist{\alpha, x}$ is then an isomorphism, as it is equal to:
\begin{align*}
&\flatdist{\alpha, x} \\
&\equiv \eta^\tdot_{\TrPlus{f}{\TrPlus{f}{\alpha}.x}};((\fcounit{\TrPlus{f}{\alpha}.x};\pi^{\TrPlus{f}{\alpha}}_x) \bdot (\ApOne{\fcounit{\TrPlus{f}{\alpha}}};\ApPlus{\ApEl{p}{\fcounit{\TrPlus{f}{\alpha}}}}{\flatone{\alpha};\ap{\flat_1}{\pi^{\TrPlus{f}{\alpha}}_x/\alpha, \var{x}/x}})) \\
&\equiv \eta^\tdot_{\TrPlus{f}{\TrPlus{f}{\alpha}.x}};((\fcounit{\TrPlus{f}{\alpha}.x};\pi^{\TrPlus{f}{\alpha}}_x) \bdot (\ApOne{\fcounit{\TrPlus{f}{\alpha}}};\ApPlus{\ApEl{p}{\fcounit{\TrPlus{f}{\alpha}}}}{\ApOne{\fcomult{\alpha}};\ApPlus{\ApEl{p}{\fcomult{\alpha}}}{\fone{\TrPlus{f}{\alpha}}};\ap{\TrPlus{\ApEl{p}{\fcomult{\alpha}}}{f_1(x)}}{\pi^{\TrPlus{f}{\alpha}}_x/\alpha, \var{x}/x}})) \\
&\equiv \eta^\tdot_{\TrPlus{f}{\TrPlus{f}{\alpha}.x}};((\fcounit{\TrPlus{f}{\alpha}.x};\pi^{\TrPlus{f}{\alpha}}_x) \bdot (\fone{\TrPlus{f}{\alpha}};\ap{f_1(x)}{\pi^{\TrPlus{f}{\alpha}}_x/\alpha, \var{x}/x})) \\
&\equiv \eta^\tdot_{\TrPlus{f}{\TrPlus{f}{\alpha}.x}};((\ApPlus{f}{\pi^{\TrPlus{f}{\alpha}}_x};\fcounit{\TrPlus{f}{\alpha}.x}) \bdot (\fone{\TrPlus{f}{\alpha}};\ap{f_1(x)}{\pi^{\TrPlus{f}{\alpha}}_x/\alpha, \var{x}/x})) \\
&\equiv \eta^{\tdot}_{\TrPlus{f}{\TrPlus{f}{\alpha}.x}} ; (\ApPlus{f}{\pi^{\TrPlus{f}{\alpha}}_x} \bdot \fone{\TrPlus{f}{\alpha}.x};\ap{f_1}{\pi^{\TrPlus{f}{\alpha}}_x/\alpha, \var{x}/x}) ; (\fcounit{\TrPlus{f}{\alpha}} \bdot \id_{\TrPlus{\ApEl{p}{\fcounit{\TrPlus{f}{\alpha}};\fcomult{\alpha}}}{f_1(x)}})
\end{align*}
%\begin{align*}
%\flatdist{\alpha, x} &\equiv \eta^\tdot_{\TrPlus{f}{\TrPlus{f}{\alpha}.x}};((\fcounit{\TrPlus{f}{\alpha}.x};\pi^{\TrPlus{f}{\alpha}}_x) \bdot (\flatone{\alpha};\ApPlus{\ApEl{p}{\fcounit{\TrPlus{f}{\alpha}}}}{\ap{\flat_1}{\pi^{\TrPlus{f}{\alpha}}_x/\alpha, \var{x}/x}})) \\
%&\equiv \eta^\tdot_{\TrPlus{f}{\TrPlus{f}{\alpha}.x}};((\fcounit{\TrPlus{f}{\alpha}.x};\pi^{\TrPlus{f}{\alpha}}_x) \bdot (\fone{\TrPlus{f}{\alpha}};\ApPlus{\ApEl{p}{\fcounit{\TrPlus{f}{\alpha}}}}{\ap{\TrPlus{\ApEl{p}{\fcomult{\alpha}}}{f_1(x)}}{\pi^{\TrPlus{f}{\alpha}}_x/\alpha, \var{x}/x}})) \\
%&\equiv \eta^\tdot_{\TrPlus{f}{\TrPlus{f}{\alpha}.x}};((\fcounit{\TrPlus{f}{\alpha}.x};\pi^{\TrPlus{f}{\alpha}}_x) \bdot (\fone{\TrPlus{f}{\alpha}};\ap{f_1}{\pi^{\TrPlus{f}{\alpha}}_x/\alpha, \var{x}/x})) \\
%&\equiv \eta^\tdot_{\TrPlus{f}{\TrPlus{f}{\alpha}.x}};((\ApPlus{f}{\pi^{\TrPlus{f}{\alpha}}_x};\fcounit{\TrPlus{f}{\alpha}.x}) \bdot (\fone{\TrPlus{f}{\alpha}};\ap{f_1}{\pi^{\TrPlus{f}{\alpha}}_x/\alpha, \var{x}/x})) \\
%&\equiv \eta^{\tdot}_{\TrPlus{f}{\TrPlus{f}{\alpha}.x}} ; (\ApPlus{f}{\pi^{\TrPlus{f}{\alpha}}_x} \bdot \fone{\TrPlus{f}{\alpha}.x};\ap{f_1}{\pi^{\TrPlus{f}{\alpha}}_x/\alpha, \var{x}/x}) ; (\fcounit{\TrPlus{f}{\alpha}} \bdot \id_{\TrPlus{\ApEl{p}{\fcounit{\TrPlus{f}{\alpha}};\fcomult{\alpha}}}{f_1(x)}})
%\end{align*}
which is the composite of the isomorphisms $\TrPlus{f}{\TrPlus{f}{\alpha}.x} \tcell \TrPlus{f}{\TrPlus{f}{\alpha}}.f_1(x) \tcell \TrPlus{f}{\alpha}.\TrPlus{\ApEl{p}{\fcomult{\alpha}}}{f_1(x)}$
\end{proof}

% \subsection{First Order Logic}

\subsection{Dependently Indexed Linear Types}

\mvrnote{What should this be called? V\'ak\'ar just says Linear Dependent Type Theory but that is slightly overselling it... }

\begin{definition}
A comprehension object $p$ \emph{supports a dependently indexed linear context} if there is:
\begin{align*}
\alpha : p, x : \El{p}{\alpha}, y : \El{p}{\alpha} &\yields x \otimes_\alpha y : \El{p}{\alpha}
\end{align*}
such that $\otimes_\alpha$ is symmetric and $\One_\alpha$ is a unit for $\otimes_\alpha$ \mvrnote{(strictness?)}. \mvrnote{Pending: What exactly do we need?}
\end{definition}

\begin{definition}
A comprehension object $p$ \emph{supports $!$-types} if it supports a linear context and there is \mvrnote{???}
\end{definition}

\begin{definition}
A comprehension object $p$ \emph{supports $\Sigma!$-types} if it supports a linear context and there is
\begin{align*}
\alpha : p, x : \El{p}{\alpha}, y : \El{p}{\alpha.x} \yields \Sigma!(x,y) : \El{p}{\alpha}
\end{align*}
and mode term morphisms
\begin{align*}
\alpha : p, x : \El{p}{\alpha}, y : \El{p}{\alpha.x} &\yields \linsnd{x,y} : y \tcell_{\El{p}{\alpha.x}} \TrPlus{\ApEl{p}{\pi^\alpha_x}}{\Sigma!(x,y)} \\
\alpha : p, x : \El{p}{\alpha}, z : \El{p}{\alpha} &\yields \linwk{x,z} : \Sigma!(x, \TrPlus{\ApEl{p}{\pi^\alpha_x}}{z}) \tcell_{\El{p}{\alpha}} z
\end{align*}
that exhibit $\Sigma!(x,-)$ as a left adjoint to $\TrPlus{\ApEl{p}{\pi^\alpha_x}}{-}$, and the morphism $\frob{\xi, x, y}$ defined by:
\begin{align*}
\mathtt{lax}_{v,w} &: \TrPlus{\ApEl{p}{\pi^\alpha_x}}{w} \otimes_{\alpha.x} \TrPlus{\ApEl{p}{\pi^\alpha_x}}{v} \tcell \TrPlus{\ApEl{p}{\pi^\alpha_x}}{w \otimes_\alpha v} \\
\mathtt{lax}_{v,w} &:\equiv \ap{(w \otimes_\alpha v)}{\pi^\alpha_x / \alpha, \id_{\TrPlus{\ApEl{p}{\pi^\alpha_x}}{w}}/w, \id_{\TrPlus{\ApEl{p}{\pi^\alpha_x}}{v}}/v} \\
\frob{\xi, x, y} &: \Sigma!(x, \TrPlus{\ApEl{p}{\pi^\alpha_x}}{\xi} \otimes_{\alpha.x} y) \tcell_{\El{p}{\alpha}} \xi \otimes_\alpha \Sigma!(x, y) \\
\frob{\xi, x, y} &:\equiv \ap{\Sigma!(x, \TrPlus{\ApEl{p}{\pi^\alpha_x}}{\xi} \otimes_{\alpha.x} -)}{\linsnd{x,y}} ; \ap{\Sigma!(x, -)}{\mathtt{lax}_{\xi, \Sigma!(x,y)}} ; \linwk{x, \xi \otimes \Sigma!(x,y)}
\end{align*}
is an isomorphism.
\end{definition}
\mvrnote{The latter is the Frobenius condition, we can still have a weaker $\Sigma!$-elim without it}

\section{Object Languages}

\subsection{Martin-L\"of Type Theory}

To distinguish judgements in this type theory from the ones in the framework we use $\qyields$ as the turnstile. We have the following judgements:
\begin{mathpar}
\qyields \Gamma \CTX \and \Gamma \qyields A \TYPE \and \Gamma \qyields a : A \and \Gamma \qyields \Theta : \Delta 
\end{mathpar}
with the inference rules given in Figure~\ref{fig:qit-rules}. Note that some equations require earlier equations to hold in order to typecheck.

\begin{itemize}
\item A context $\qyields \Gamma \CTX$ is represented by a framework context $\upstairs{\Gamma}$ over mode context $\downstairs{\Gamma}$, together with a mode type morphism $\cdot \yields \tshape{\Gamma} : p \tcell \ctxtuple{\downstairs{\Gamma}}$ where $\ctxtuple{\downstairs{\Gamma}}$ is the iterated telescope type of Lemma~\ref{lem:ctxtuple}. This mode type morphism describes the dependency structure of the context.
\item A type $\Gamma \qyields A \TYPE$ is represented by a framework type $\upstairs{\Gamma} \yields_{\El{p}{\modeof{\Gamma}}} \upstairs{A} \TYPE$, where
\begin{align*}
\downstairs{\Gamma} \yields \modeof{\Gamma} :\equiv \TrPlus{\tshape{\Gamma}}{\pack{\downstairs{\Gamma}}} : p
\end{align*}
Note that $\pack{\downstairs{\Gamma}}$ is simply an iterated tuple of the variables of $\downstairs{\Gamma}$ in order: $(((((), x), y), z), \dots)$.
\item A term $\Gamma \qyields a : A$ is represented by a framework term $\upstairs{\Gamma} \yields_{\One_{\modeof{\Gamma}}} \upstairs{a} : \upstairs{A}$.
\item A substitution $\Gamma \qyields \Theta : \Delta$ is represented by a term $\upstairs{\Gamma} \yields_{\modeof{\Gamma}} \upstairs{\Theta} : \St{\tshape{\Delta}}{\ctxtuple{\upstairs{\Delta}}}$
\end{itemize}

\begin{figure}
\begin{mathpar}
\inferrule*[left=ctx-empty]{~}{\cdot \CTX} \and
\inferrule*[left=ctx-ext]{\qyields \Gamma \CTX \and \Gamma \qyields A \TYPE}{\qyields \Gamma, A \CTX} \\
\inferrule*[left=type-sub]{\Delta \qyields A \TYPE \and \Gamma \qyields \Theta : \Delta}{\Gamma \qyields A[\Theta] \TYPE} \and
\inferrule*[left=term-sub]{\Delta \qyields a : A  \and \Gamma \qyields \Theta : \Delta}{\Gamma \qyields a[\Theta] : A[\Theta]} 
\\
\inferrule*[left=sub-empty]{~}{\Gamma \qyields \epsilon_\Gamma : \cdot} \and
\inferrule*[left=sub-ext]{\Gamma \qyields \Theta : \Delta \and \Gamma \qyields a : A[\Theta]}{\Gamma \qyields (\Theta, a) : \Delta, A} \\
\inferrule*[left=sub-id]{~}{\Gamma \qyields \id_\Gamma : \Gamma} \and
\inferrule*[left=sub-comp]{\Gamma \qyields \Theta : \Delta \and \Delta \qyields \kappa : \Lambda}{\Gamma \qyields \Theta ; \kappa : \Lambda} \\
\inferrule*[left=sub-proj]{~}{\Gamma, A \qyields \proj{\Gamma,A} : \Gamma} \and 
\inferrule*[left=var]{~}{\Gamma, A \qyields \qvar{\Gamma,A} : A[\proj{\Gamma,A}]} 
\end{mathpar}

\begin{align}
A[\id] &\equiv A \\
A[\Theta ; \kappa] &\equiv A[\kappa][\Theta] \\
\nonumber\\
a[\id] &\equiv a \\
a[\Theta ; \kappa] &\equiv a[\kappa][\Theta] \\
\nonumber\\
\id ; \Theta &\equiv \Theta \\
\Theta ; \id &\equiv \Theta \\
(\Theta; \kappa) ; \rho &\equiv \Theta ; (\kappa ; \rho) \\
\nonumber\\
\Theta ; (\kappa , a) &\equiv (\Theta ; \kappa) , a[\Theta] \\ 
(\Theta, a);\proj{\Gamma,A} &\equiv \Theta \\
\qvar{\Delta,A}[\Theta, a] &\equiv a \\
(\proj{\Gamma,A}, \qvar{\Gamma,A}) &\equiv \id_{\Gamma, A} \\
\Theta &\equiv \epsilon_\Gamma && \text{for } \Gamma \qyields \Theta : \cdot
\end{align}
\caption{Rules of MLTT via Explicit Substitutions}\label{fig:qit-rules}
\end{figure}

The structural rules of MLTT then have the following translations:
\begin{enumerate}
\item[\textsc{ctx-empty}] Define $\upstairs{(\cdot)}$ to be the empty framework context and $\tshape{(\cdot)} :\equiv \tempty : p \tcell 1$, noting that $\ctxtuple{(\cdot)} \equiv 1$.

\item[\textsc{ctx-ext}] Given $\upstairs{\Gamma}
  \yields_{\El{p}{\modeof{\Gamma}}} \upstairs{A} \TYPE$, define $\upstairs{\Gamma, A}$ to be the extended framework context $\upstairs{\Gamma}, x : \upstairs{A}$. We need to specify a mode type morphism $\tshape{\Gamma, A} : p \tcell \ctxtuple{\downstairs{\Gamma, A}}$. 
  By definition,
  \begin{align*}
    \ctxtuple{\downstairs{\Gamma, A}} \equiv \sum_{\sigma : \ctxtuple{\downstairs{\Gamma}}} \modeof{\Gamma}[\unpack{\Gamma}{\sigma}]
  \end{align*}
  and note that 
  \begin{align*}
  \modeof{\Gamma}[\unpack{\Gamma}{\sigma}]
  &\equiv \TrPlus{\tshape{\Gamma}}{\pack{\downstairs{\Gamma}}}[\unpack{\Gamma}{\sigma}] \\
  &\equiv \TrPlus{\tshape{\Gamma}}{\pack{\downstairs{\Gamma}}[\unpack{\Gamma}{\sigma}]} \\
  &\equiv \TrPlus{\tshape{\Gamma}}{\sigma}
  \end{align*}
  so we have a mode type morphism
  \begin{align*}
  \sigmacl{\sigma}{\tshape{\Gamma}}{\id_{\El{p}{\TrPlus{\tshape{\Gamma}}{\sigma}}}} : \sigmacl{\sigma}{p}{\El{p}{\sigma}} \tcell \sigmacl{\sigma}{\ctxtuple{\downstairs{\Gamma}}}{\El{p}{\TrPlus{\tshape{\Gamma}}{\sigma}}}
  \end{align*}
  Define $\tshape{\Gamma, A}$ to be:
  \begin{align*}
  \tshape{\Gamma, A} :\equiv \tdot ; (\sigmacl{\sigma}{\tshape{\Gamma}}{\id_{\El{p}{\TrPlus{\tshape{\Gamma}}{\sigma}}}})
  \end{align*}
  
\item[\textsc{type-sub}] Given $\upstairs{\Delta} \yields_{\El{p}{\modeof{\Delta}}} \upstairs{A} \TYPE$ and $\upstairs{\Gamma} \yields_{\modeof{\Gamma}} \upstairs{\Theta} : \St{\tshape{\Delta}}{\ctxtuple{\upstairs{\Delta}}}$, we can form:
\begin{align*}
\upstairs{\Gamma} \yields_{\El{p}{\modeof{\Gamma}}} \upstairs{A[\Theta]} :\equiv \StE{\tshape{\Delta}}{\upstairs{\Theta}}{\sigma}{\upstairs{A}[\unpack{\Delta}{\sigma}]} \TYPE
\end{align*}

\item[\textsc{term-sub}] Similarly, given $\upstairs{\Delta} \yields_{\El{p}{\One_{\modeof{\Delta}}}} \upstairs{a} : \upstairs{A} \TYPE$ and $\upstairs{\Gamma} \yields_{\modeof{\Gamma}} \upstairs{\Theta} : \St{\tshape{\Delta}}{\ctxtuple{\upstairs{\Delta}}}$, we can form:
\begin{align*}
\upstairs{\Gamma} \yields_{\One_{\modeof{\Gamma}}} \upstairs{a[\Theta]} :\equiv \StE{\tshape{\Delta}}{\upstairs{\Theta}}{\sigma}{\upstairs{a}[\unpack{\Delta}{\sigma}]} : \upstairs{A[\Theta]}
\end{align*}

\item[\textsc{sub-empty}] We are constructing a term
\begin{align*}
\upstairs{\Gamma} \yields_{\modeof{\Gamma}} \upstairs{\epsilon_\Gamma} : \St{\tshape{(\cdot)}}{\cdot}
\end{align*}
The type is equal by definition to $\St{\tempty}{1}$, so we may use the counit $\eta^\tempty_{\modeof{\Gamma}} : \modeof{\Gamma} \tcell_p \TrPlus{\tempty}{}$ to form
\begin{align*}
\upstairs{\epsilon_\Gamma} :\equiv \rewrite{\eta^\tempty_{\modeof{\Gamma}}}{\StI{\tempty}{}}
\end{align*}

\item[\textsc{sub-ext}] We are provided
\begin{align*}
\upstairs{\Gamma} &\yields_{\modeof{\Gamma}} \upstairs{\Theta} : \St{\tshape{\Delta}}{\ctxtuple{\upstairs{\Delta}}} \\
\upstairs{\Gamma} &\yields_{\One_{\modeof{\Gamma}}} \upstairs{a} : \upstairs{A[\Theta]}
\end{align*}
and we are trying to define:
\begin{align*}
\upstairs{\Gamma} &\yields_{\modeof{\Gamma}} \upstairs{\Theta, a} : \St{\tshape{\Delta, A}}{\ctxtuple{\upstairs{\Delta, A}}}
\end{align*}
Unfolding definitions:
\begin{align*}
\upstairs{A[\Theta]}
&\equiv \StE{\tshape{\Delta}}{\upstairs{\Theta}}{\sigma}{\upstairs{A}[\unpack{\Delta}{\sigma}]}
\intertext{and}
\St{\tshape{\Delta, A}}{\ctxtuple{\upstairs{\Delta, A}}}
&\equiv \St{(\tdot ; (\sigmacl{\sigma}{\tshape{\Delta}}{\id_{\El{p}{\TrPlus{\tshape{\Delta}}{\sigma}}}}))}{\ctxtuple{\upstairs{\Delta}}, \upstairs{A}[\unpack{\Delta}{\sigma}]} \\
&\equiv \St{\tdot}{\St{\sigmacl{\sigma}{\tshape{\Delta}}{\id_{\El{p}{\TrPlus{\tshape{\Delta}}{\sigma}}}}}{\ctxtuple{\upstairs{\Delta}}, \upstairs{A}[\unpack{\Delta}{\sigma}]}} \\
&\equiv \St{\tdot}{\sigma' : \St{\tshape{\Delta}}{\ctxtuple{\upstairs{\Delta}}}, \StE{\tshape{\Delta}}{\sigma'}{\sigma}{\upstairs{A}[\unpack{\Delta}{\sigma}]}}
\end{align*}
So we can form:
\begin{align*}
\upstairs{\Theta, a} :\equiv \rewrite{\eta^\tdot_{\modeof{\Gamma}}}{\StI{\tdot}{\upstairs{\Theta}, \upstairs{a}}}
\end{align*}

\item[\textsc{sub-id}] We have:
\begin{align*}
\upstairs{\Gamma} \yields_{\modeof{\Gamma}} \upstairs{\id_\Gamma} :\equiv \StI{\tshape{\Gamma}}{\pack{\upstairs{\Gamma}}} : \St{\tshape{\Gamma}}{\ctxtuple{\upstairs{\Gamma}}}
\end{align*}

\item[\textsc{sub-comp}]
Given 
\begin{align*}
\upstairs{\Gamma} &\yields_{\modeof{\Gamma}} \upstairs{\Theta} : \St{\tshape{\Delta}}{\ctxtuple{\upstairs{\Delta}}} \\
\upstairs{\Delta} &\yields_{\modeof{\Delta}} \upstairs{\kappa} : \St{\tshape{\Lambda}}{\ctxtuple{\upstairs{\Lambda}}}
\end{align*}
we can form
\begin{align*}
\upstairs{\Theta;\kappa} :\equiv \StE{\tshape{\Delta}}{\upstairs{\Theta}}{\sigma}{\upstairs{\kappa}[\unpack{\upstairs{\Delta}}{\sigma}]}
\end{align*}

\item[\textsc{sub-proj}] We are trying to construct a term
\begin{align*}
\upstairs{\Gamma}, x : \upstairs{A} \yields_{\modeof{(\Gamma, A)}} \upstairs{\proj{\Gamma, A}} : \St{\tshape{\Gamma}}{\ctxtuple{\upstairs{\Gamma}}}
\end{align*}
with mode
\begin{align*}
\modeof{(\Gamma, A)}
&\equiv \TrPlus{\tshape{(\Gamma, A)}}{\pack{\downstairs{\Gamma. x : A}}} \\
&\equiv \TrPlus{(\tdot ; (\sigmacl{\sigma}{\tshape{\Gamma}}{\id_{\El{p}{\TrPlus{\tshape{\Gamma}}{\sigma}}}}))}{\pack{\downstairs{\Gamma}}, x} \\
&\equiv \TrPlus{\tdot}{\TrPlus{\tshape{\Gamma}}{\pack{\downstairs{\Gamma}}}, x} \\
&\equiv \TrPlus{\tdot}{\modeof{\Gamma}, x} \\
&\equiv \modeof{\Gamma}.x
\end{align*}
So define:
\begin{align*}
\upstairs{\proj{\Gamma, A}} :\equiv \rewrite{\pi^{\modeof{\Gamma}}_x}{\upstairs{\id_\Gamma}}
\end{align*}

\item[\textsc{var}] We are a building a term
\begin{align*}
\upstairs{\Gamma}, x : \upstairs{A} \yields_{\One_{\modeof{(\Gamma, A)}}} \upstairs{\qvar{\Gamma, A}} : \upstairs{A[\proj{\Gamma, A}]}
\end{align*}
The type in question simplifies:
\begin{align*}
\upstairs{A[\proj{\Gamma, A}]}
&\equiv \StE{\tshape{\Gamma}}{\upstairs{\proj{\Gamma, A}}}{\sigma}{\upstairs{A}[\unpack{\Gamma}{\sigma}]} \\
&\equiv \StE{\tshape{\Gamma}}{\rewrite{\pi^{\modeof{\Gamma}}_x}{\upstairs{\id_\Gamma}}}{\sigma}{\upstairs{A}[\unpack{\Gamma}{\sigma}]} \\
&\equiv \St{\ApEl{p}{\pi^{\modeof{\Gamma}}_x}}{\StE{\tshape{\Gamma}}{\upstairs{\id_\Gamma}}{\sigma}{\upstairs{A}[\unpack{\Gamma}{\sigma}]}} \\
&\equiv \St{\ApEl{p}{\pi^{\modeof{\Gamma}}_x}}{\StE{\tshape{\Gamma}}{\StI{\tshape{\Gamma}}{\pack{\downstairs{\Gamma}}}}{\sigma}{\upstairs{A}[\unpack{\Gamma}{\sigma}]}} \\
&\equiv \St{\ApEl{p}{\pi^{\modeof{\Gamma}}_x}}{\upstairs{A}[\unpack{\Gamma}{\sigma}][\pack{\downstairs{\Gamma}}/\sigma]} \\
&\equiv \St{\ApEl{p}{\pi^{\modeof{\Gamma}}_x}}{\upstairs{A}}
\end{align*}

So define $\upstairs{\qvar{\Gamma, A}}$ by:
\begin{align*}
\upstairs{\qvar{\Gamma, A}} :\equiv \rewrite{\var{x}}{\StI{\ApEl{p}{\pi^{\modeof{\Gamma}}_x}}{x}}
\end{align*}
\end{enumerate}

Now we check that these translations satisfy the required equations.

\begin{enumerate}[style = multiline, labelwidth = 80pt]
\item[{$A[\id_\Gamma] \equiv A$}] 
\begin{align*}
\upstairs{A[\id_\Gamma]}
&\equiv \StE{\tshape{\Gamma}}{\upstairs{\id_\Gamma}}{\sigma}{\upstairs{A}[\unpack{\Gamma}{\sigma}]} \\
&\equiv \StE{\tshape{\Gamma}}{\StI{\tshape{\Gamma}}{\pack{\downstairs{\Gamma}}}}{\sigma}{\upstairs{A}[\unpack{\Gamma}{\sigma}]} \\
&\equiv \upstairs{A}[\unpack{\Gamma}{\pack{\downstairs{\Gamma}}}]\\
&\equiv \upstairs{A}
\end{align*}

\item[{$A[\Theta ; \kappa] \equiv A[\kappa][\Theta]$}] 
\begin{align*}
\upstairs{A[\Theta ; \kappa]}
&\equiv \StE{\tshape{\Lambda}}{\upstairs{\Theta ; \kappa}}{\sigma}{\upstairs{A}[\unpack{\Lambda}{\sigma}]} \\
&\equiv \StE{\tshape{\Lambda}}{(\StE{\tshape{\Delta}}{\upstairs{\Theta}}{\sigma'}{\upstairs{\kappa}[\unpack{\upstairs{\Delta}}{\sigma'}]})}{\sigma}{\upstairs{A}[\unpack{\Lambda}{\sigma}]} \\
&\equiv \StE{\tshape{\Delta}}{\upstairs{\Theta}}{\sigma'}{\StE{\tshape{\Lambda}}{\upstairs{\kappa}[\unpack{\upstairs{\Delta}}{\sigma'}]}{\sigma}{\upstairs{A}[\unpack{\Lambda}{\sigma}]}} \\
&\equiv \StE{\tshape{\Delta}}{\upstairs{\Theta}}{\sigma'}{(\StE{\tshape{\Lambda}}{\upstairs{\kappa}}{\sigma}{\upstairs{A}[\unpack{\Lambda}{\sigma}]})[\unpack{\Delta}{\sigma'}]} \\
&\equiv \StE{\tshape{\Delta}}{\upstairs{\Theta}}{\sigma'}{\upstairs{A[\kappa]}[\unpack{\Lambda}{\sigma'}]} \\
&\equiv \upstairs{A[\kappa][\Theta]}
\end{align*}

\item[{$a[\id_\Gamma] \equiv a$}] Same as for types.
\item[{$a[\Theta ; \kappa] \equiv a[\kappa][\Theta]$}] Same as for types.

\item[{$\Theta ; (\kappa , a) \equiv (\Theta ; \kappa) , a[\Theta]$}]
\begin{align*}
\upstairs{\Theta ; (\kappa , a)}
&\equiv \StE{\tshape{\Delta}}{\upstairs{\Theta}}{\sigma}{\upstairs{(\kappa , a)}[\unpack{\upstairs{\Delta}}{\sigma}]} \\
&\equiv \StE{\tshape{\Delta}}{\upstairs{\Theta}}{\sigma}{\rewrite{\eta^\tdot_{\modeof{\Lambda}}}{\StI{\tdot}{\upstairs{\kappa}, \upstairs{a}}}[\unpack{\upstairs{\Delta}}{\sigma}]} \\
&\equiv \StE{\tshape{\Delta}}{\upstairs{\Theta}}{\sigma}{\rewrite{\eta^\tdot_{\modeof{\Lambda}}}{\StI{\tdot}{\upstairs{\kappa}[\unpack{\upstairs{\Delta}}{\sigma}], \upstairs{a}[\unpack{\upstairs{\Delta}}{\sigma}]}}} \\
&\equiv \rewrite{\eta^\tdot_{\modeof{\Lambda}}}{\StI{\tdot}{(\StE{\tshape{\Delta}}{\upstairs{\Theta}}{\sigma}{\upstairs{\kappa}[\unpack{\upstairs{\Delta}}{\sigma}]}), (\StE{\tshape{\Delta}}{\upstairs{\Theta}}{\sigma}{\upstairs{a}[\unpack{\upstairs{\Delta}}{\sigma}]})}} \\
&\equiv \rewrite{\eta^\tdot_{\modeof{\Lambda}}}{\StI{\tdot}{\upstairs{(\Theta ; \kappa)}, \upstairs{a[\Theta]}}} \\
&\equiv \upstairs{(\Theta ; \kappa) , a[\Theta]}
\end{align*} 

\item[{$(\Theta, a);\proj{\Delta,A} \equiv \Theta$}]
\begin{align*}
\upstairs{(\Theta, a);\proj{\Delta,A}}
&\equiv \StE{\tshape{\Delta, x : A}}{\upstairs{\Theta, a}}{\sigma}{\upstairs{\proj{\Delta,A}}[\unpack{\upstairs{\Delta, x : A}}{\sigma}]} \\
&\equiv \StE{\tdot ; (\sigmacl{\sigma}{\tshape{\Delta}}{\id_{\El{p}{\TrPlus{\tshape{\Delta}}{\sigma}}}})}{\rewrite{\eta^\tdot_{\modeof{\Gamma}}}{\StI{\tdot}{\upstairs{\Theta}, \upstairs{a}}}}{\sigma}{\rewrite{\pi^{\modeof{\Delta}}_x}{\upstairs{\id_\Delta}}[\unpack{\upstairs{\Delta, x : A}}{\sigma}]} \\
&\equiv \StE{\tdot}{\rewrite{\eta^\tdot_{\modeof{\Gamma}}}{\StI{\tdot}{\upstairs{\Theta}, \upstairs{a}}}}{\sigma'}{\StE{(\sigmacl{\sigma}{\tshape{\Delta}}{\id_{\El{p}{\TrPlus{\tshape{\Delta}}{\sigma}}}})}{\sigma'}{\sigma}{\rewrite{\pi^{\modeof{\Delta}}_x}{\upstairs{\id_\Delta}}[\unpack{\upstairs{\Delta, x : A}}{\sigma}]}} \\
&\equiv \rewrite{\eta^\tdot_{\modeof{\Gamma}}}{\StE{\tdot}{\StI{\tdot}{\upstairs{\Theta}, \upstairs{a}}}{\sigma'}{\StE{(\sigmacl{\sigma}{\tshape{\Delta}}{\id_{\El{p}{\TrPlus{\tshape{\Delta}}{\sigma}}}})}{\sigma'}{\sigma}{\rewrite{\pi^{\modeof{\Delta}}_x}{\upstairs{\id_\Delta}}[\unpack{\upstairs{\Delta, x : A}}{\sigma}]}}} \\
&\equiv \rewrite{\eta^\tdot_{\modeof{\Gamma}}}{\StE{(\sigmacl{\sigma}{\tshape{\Delta}}{\id_{\El{p}{\TrPlus{\tshape{\Delta}}{\sigma}}}})}{(\upstairs{\Theta}, \upstairs{a})}{\sigma}{\rewrite{\pi^{\modeof{\Delta}}_x}{\upstairs{\id_\Delta}}[\unpack{\upstairs{\Delta, x : A}}{\sigma}]}} \\
&\equiv \rewrite{\eta^\tdot_{\modeof{\Gamma}}}{\StE{(\sigmacl{\sigma}{\tshape{\Delta}}{\id_{\El{p}{\TrPlus{\tshape{\Delta}}{\sigma}}}})}{(\upstairs{\Theta}, \upstairs{a})}{\sigma}{\rewrite{\pi^{\modeof{\Delta}}_x}{\upstairs{\id_\Delta}}[\unpack{\upstairs{\Delta}}{\fst \sigma}/\sigma, \snd \sigma / x]}} \\
&\equiv \rewrite{\eta^\tdot_{\modeof{\Gamma}}}{\StE{(\sigmacl{\sigma}{\tshape{\Delta}}{\id_{\El{p}{\TrPlus{\tshape{\Delta}}{\sigma}}}})}{(\upstairs{\Theta}, \upstairs{a})}{\sigma}{\rewrite{\pi^{\TrPlus{\tshape{\Delta}}{\fst \sigma}}_{\snd \sigma}}{\StI{\tshape{\Delta}}{\fst \sigma}}}} \\
&\equiv \rewrite{\eta^\tdot_{\modeof{\Gamma}}}{\rewrite{\pi^{\fst \sigma}_{\snd \sigma}}{\fst \sigma}[(\upstairs{\Theta}, \upstairs{a})/\sigma]} \\
&\equiv \rewrite{\eta^\tdot_{\modeof{\Gamma}};\pi^{\modeof{\Gamma}}_{\One_{\modeof{\Gamma}}}}{\upstairs{\Theta}} \\
&\equiv \upstairs{\Theta}
\end{align*}

\item[{$\qvar{\Delta,A}[\Theta, a] \equiv a$}]
\begin{align*}
\upstairs{\qvar{\Delta,A}[\Theta, a]}
&\equiv \StE{\tshape{\Delta, x : A}}{\upstairs{(\Theta, a)}}{\sigma}{\upstairs{\qvar{\Delta,A}}[\unpack{\Delta, x : A}{\sigma}]} \\
&\equiv \StE{\tshape{\Delta, x : A}}{\rewrite{\eta^\tdot_{\modeof{\Gamma}}}{\StI{\tdot}{\upstairs{\Theta}, \upstairs{a}}}}{\sigma}{\rewrite{\var{x}}{\StI{\ApEl{p}{\pi^{\modeof{\Delta}}_x}}{x}}[\unpack{\Delta, x : A}{\sigma}]} \\
&\equiv \rewrite{\ApOne{\eta^\tdot_{\modeof{\Gamma}}}}{\StI{\ApEl{p}{\eta^\tdot_{\modeof{\Gamma}}}}{\StE{\tshape{\Delta, x : A}}{\StI{\tdot}{\upstairs{\Theta}, \upstairs{a}}}{\sigma}{\rewrite{\var{\snd \sigma}}{\StI{\ApEl{p}{\pi^{\TrPlus{\tshape{\Delta}}{\fst \sigma}}_{\snd \sigma}}}{\snd \sigma}}}}} \\
&\equiv \rewrite{\ApOne{\eta^\tdot_{\modeof{\Gamma}}}}{\StI{\ApEl{p}{\eta^\tdot_{\modeof{\Gamma}}}}{\StE{(\sigmacl{\sigma}{\tshape{\Delta}}{\id_{\El{p}{\TrPlus{\tshape{\Delta}}{\sigma}}}})}{(\upstairs{\Theta}, \upstairs{a})}{\sigma}{\rewrite{\var{\snd \sigma}}{\StI{\ApEl{p}{\pi^{\TrPlus{\tshape{\Delta}}{\fst \sigma}}_{\snd \sigma}}}{\snd \sigma}}}}} \\
&\equiv \rewrite{\ApOne{\eta^\tdot_{\modeof{\Gamma}}}}{\StI{\ApEl{p}{\eta^\tdot_{\modeof{\Gamma}}}}{\rewrite{\var{\snd \sigma}}{\StI{\ApEl{p}{\pi^{\TrPlus{\tshape{\Delta}}{\fst \sigma}}_{\snd \sigma}}}{\snd \sigma}}[(\upstairs{\Theta}, \upstairs{a})/\sigma]}} \\
&\equiv \rewrite{\ApOne{\eta^\tdot_{\modeof{\Gamma}}}}{\StI{\ApEl{p}{\eta^\tdot_{\modeof{\Gamma}}}}{\rewrite{\var{\One_{\modeof{\Gamma}}}}{\StI{\ApEl{p}{\pi^{\modeof{\Gamma}}_{\One_{\modeof{\Gamma}}}}}{\upstairs{a}}}}} \\
&\equiv \rewrite{\ApOne{\eta^\tdot_{\modeof{\Gamma}}}}{\rewrite{\ApPlus{\ApEl{p}{\eta^\tdot_{\modeof{\Gamma}}}}{\var{\One_{\modeof{\Gamma}}}}}{\StI{\ApEl{p}{\eta^\tdot_{\modeof{\Gamma}}}}{\StI{\ApEl{p}{\pi^{\modeof{\Gamma}}_{\One_{\modeof{\Gamma}}}}}{\upstairs{a}}}}} \\
&\equiv \rewrite{\ApOne{\eta^\tdot_{\modeof{\Gamma}}};\ApPlus{\ApEl{p}{\eta^\tdot_{\modeof{\Gamma}}}}{\var{\One_{\modeof{\Gamma}}}}}{\StI{\ApEl{p}{\eta^\tdot_{\modeof{\Gamma}}};\ApEl{p}{\pi^{\modeof{\Gamma}}_{\One_{\modeof{\Gamma}}}}}{\upstairs{a}}} \\
&\equiv \upstairs{a}
\end{align*}
Where we have used that
\begin{align*}
\ApOne{\eta^\tdot_\alpha};\ApPlus{\ApEl{p}{\eta^\tdot_\alpha}}{\var{\One_\alpha}} \equiv \id_{\One_\alpha}
\end{align*}
by the triangle identity for $\tdot$, Equation~\eqref{eq:chi-triangle-2}

\item[{$(\proj{\Gamma,A}, \qvar{\Gamma,A}) \equiv \id_{\Gamma, A}$}] 
\begin{align*}
\upstairs{(\proj{\Gamma,A}, \qvar{\Gamma,A})}
&\equiv \rewrite{\eta^\tdot_{\modeof{(\Gamma, x : A)}}}{\StI{\tdot}{\upstairs{\proj{\Gamma,A}}, \upstairs{\qvar{\Gamma,A}}}} \\
&\equiv \rewrite{\eta^\tdot_{\modeof{(\Gamma, x : A)}}}{\StI{\tdot}{\rewrite{\pi^{\modeof{\Gamma}}_x}{\upstairs{\id_\Gamma}}, \rewrite{\var{x}}{\StI{\ApEl{p}{\pi^{\modeof{\Gamma}}_x}}{x}}}} \\
&\equiv \rewrite{\eta^\tdot_{\modeof{(\Gamma, x : A)}}}{\StI{\tdot}{\rewrite{(\pi^{\modeof{\Gamma}}_x, \var{x})}{(\upstairs{\id_\Gamma}, x)}}}\\
&\equiv \rewrite{\eta^\tdot_{\modeof{(\Gamma, x : A)}}}{\rewrite{\ApPlus{\tdot}{(\pi^{\modeof{\Gamma}}_x, \var{x})}}{\StI{\tdot}{\upstairs{\id_\Gamma}, x}}}\\
&\equiv \StI{\tdot}{\upstairs{\id_\Gamma}, x} \\
&\equiv \StI{\tdot}{\StI{\tshape{\Gamma}}{\pack{\downstairs{\Gamma}}}, x}\\
&\equiv \StI{\tdot ; (\sigmacl{\sigma}{\tshape{\Gamma}}{\id_{\El{p}{\TrPlus{\tshape{\Gamma}}{\sigma}}}})}{\pack{\downstairs{\Gamma}}, x} \\
&\equiv \StI{\tshape{\Gamma, A}}{\pack{\downstairs{\Gamma, A}}} \\
&\equiv \upstairs{\id_{\Gamma, A}}
\end{align*}

\item[$\Theta \equiv \epsilon_\Gamma$] 
Using eta expansion for $\St{\tempty}{}$ once in each direction:
\begin{align*}
\upstairs{\Theta}
&\equiv \StE{\tshape{(\cdot)}}{\upstairs{\Theta}}{x}{\StI{\tshape{(\cdot)}}{x}} \\
&\equiv \StE{\tempty}{\upstairs{\Theta}}{x}{\rewrite{\eta^\tempty_{\TrPlus{\tempty}{}}}{\StI{\tempty}{}}} \\
&\equiv \StE{\tempty}{\upstairs{\Theta}}{x}{\rewrite{\eta^\tempty_{\TrPlus{\tempty}{x}}}{\StI{\tempty}{}}} \\
&\equiv \rewrite{\eta^\tempty_{\modeof{\Gamma}}}{\StI{\tempty}{}} \\
&\equiv \upstairs{\epsilon_\Gamma}
\end{align*}
\end{enumerate}

We record some useful facts:
\begin{lemma}
\begin{align*}
\upstairs{B[\proj{\Gamma, A}]} &\equiv \St{\ApEl{p}{\pi^{\modeof{\Gamma}}_x}}{\upstairs{B}} \\
\upstairs{b[\proj{\Gamma, A}]} &\equiv \rewrite{\ApOne{\pi^{\modeof{\Gamma}}_x}}{\StI{\ApEl{p}{\pi^{\modeof{\Gamma}}_x}}{\upstairs{b}}} \\
\upstairs{\proj{\Gamma, A, B};\proj{\Gamma, A}} &\equiv \rewrite{\pi^{\modeof{\Gamma}.x}_y;\pi^{\modeof{\Gamma}}_x}{\upstairs{\id_{\Gamma}}} \\
\end{align*}
\end{lemma}
\begin{proof}
The first was shown in the translation of $\qvar{}$. The second is similar:
\begin{align*}
\upstairs{b[\proj{\Gamma, A}]}
&\equiv \StE{\tshape{\Gamma}}{\upstairs{\proj{\Gamma, A}}}{\sigma}{\upstairs{b}[\unpack{\Gamma}{\sigma}]} \\
&\equiv \StE{\tshape{\Gamma}}{\rewrite{\pi^{\modeof{\Gamma}}_x}{\upstairs{\id_\Gamma}}}{\sigma}{\upstairs{b}[\unpack{\Gamma}{\sigma}]} \\
&\equiv \rewrite{\ApOne{\pi^{\modeof{\Gamma}}_x}}{\StI{\ApEl{p}{\pi^{\modeof{\Gamma}}_x}}{\StE{\tshape{\Gamma}}{\upstairs{\id_\Gamma}}{\sigma}{\upstairs{b}[\unpack{\Gamma}{\sigma}]}}} \\
&\equiv \rewrite{\ApOne{\pi^{\modeof{\Gamma}}_x}}{\StI{\ApEl{p}{\pi^{\modeof{\Gamma}}_x}}{\StE{\tshape{\Gamma}}{\StI{\tshape{\Gamma}}{\pack{\downstairs{\Gamma}}}}{\sigma}{\upstairs{b}[\unpack{\Gamma}{\sigma}]}}} \\
&\equiv \rewrite{\ApOne{\pi^{\modeof{\Gamma}}_x}}{\StI{\ApEl{p}{\pi^{\modeof{\Gamma}}_x}}{\upstairs{b}[\unpack{\Gamma}{\sigma}][\pack{\downstairs{\Gamma}}/\sigma]}} \\
&\equiv \rewrite{\ApOne{\pi^{\modeof{\Gamma}}_x}}{\StI{\ApEl{p}{\pi^{\modeof{\Gamma}}_x}}{\upstairs{b}}}
\end{align*}
And the third:
\begin{align*}
\upstairs{\proj{\Gamma, A, B};\proj{\Gamma, A}} 
&\equiv \StE{\tshape{\Gamma, A}}{\upstairs{\proj{\Gamma, A, B}}}{\sigma}{\upstairs{\proj{\Gamma, A}}[\unpack{\upstairs{\Gamma, A}}{\sigma}]} \\
&\equiv \StE{\tshape{\Gamma, A}}{\rewrite{\pi^{\modeof{\Gamma}.x}_y}{\upstairs{\id_{\Gamma, A}}}}{\sigma}{\upstairs{\proj{\Gamma, A}}[\unpack{\upstairs{\Gamma, A}}{\sigma}]} \\
&\equiv \rewrite{\pi^{\modeof{\Gamma}.x}_y}{\StE{\tshape{\Gamma, A}}{\upstairs{\id_{\Gamma, A}}}{\sigma}{\upstairs{\proj{\Gamma, A}}[\unpack{\upstairs{\Gamma, A}}{\sigma}]}} \\
&\equiv \rewrite{\pi^{\modeof{\Gamma}.x}_y}{\upstairs{\id_{\Gamma, A};\proj{\Gamma, A}}} \\
&\equiv \rewrite{\pi^{\modeof{\Gamma}.x}_y}{\upstairs{\proj{\Gamma, A}}} \\
&\equiv \rewrite{\pi^{\modeof{\Gamma}.x}_y}{\rewrite{\pi^{\modeof{\Gamma}}_x}{\upstairs{\id_\Gamma}}} \\
&\equiv \rewrite{\pi^{\modeof{\Gamma}.x}_y;\pi^{\modeof{\Gamma}}_x}{\upstairs{\id_\Gamma}}
\end{align*}
For the third:

\end{proof}

\begin{lemma}
There is a derived operation
\begin{mathpar}
\inferrule*[left=derivable]{\Gamma \qyields \Theta : \Delta \and \Delta \qyields A \TYPE}{\Gamma, A[\Theta] \qyields \Theta \uparrow A : \Delta, A}
\end{mathpar}
defined by:
\begin{align*}
\Theta \uparrow A &:\equiv (\proj{\Gamma, A[\Theta]}; \Theta) , \qvar{\Gamma, A[\Theta]}
\end{align*}
such that
\begin{align*}
(\Theta \uparrow A) ; \proj{\Delta, A} &\equiv \proj{\Gamma, A[\Theta]} ; \Theta \\
\qvar{\Delta, A}[\Theta \uparrow A] &\equiv \qvar{\Gamma, A[\Theta]} \\
\upstairs{\Theta \uparrow A} &\equiv \StI{\tdot}{\upstairs{\Theta}, x} \\
\StE{\tshape{\Delta, A}}{\upstairs{\Theta \uparrow A}}{\sigma}{M[\unpack{\Delta, A}{\sigma}]} &\equiv \StE{(\tshape{\Delta}, \id)}{(\upstairs{\Theta}, x)}{\sigma}{M[\unpack{\Delta}{\sigma}, x/x]}
\end{align*}
\end{lemma}
\begin{proof}
\begin{align*}
\upstairs{\Theta \uparrow A} 
&\equiv \upstairs{(\proj{\Gamma, A[\Theta]}; \Theta) , \qvar{\Gamma, A[\Theta]}} \\
&\equiv \rewrite{\eta^\tdot_{\modeof{(\Gamma, x : A[\Theta])}}}{\StI{\tdot}{\upstairs{(\proj{\Gamma, A[\Theta]}; \Theta)}, \upstairs{\qvar{\Gamma,A[\Theta]}}}} \\
&\equiv \rewrite{\eta^\tdot_{\modeof{(\Gamma, x : A[\Theta])}}}{\StI{\tdot}{\StE{\tshape{\Gamma}}{\upstairs{\proj{\Gamma, A[\Theta]}}}{\sigma}{\upstairs{\Theta}[\unpack{\upstairs{\Gamma}}{\sigma}]}, \upstairs{\qvar{\Gamma,A[\Theta]}}}} \\
&\equiv \rewrite{\eta^\tdot_{\modeof{(\Gamma, x : A[\Theta])}}}{\StI{\tdot}{\rewrite{\pi^{\modeof{\Gamma}}_x}{\upstairs{\Theta}[\unpack{\upstairs{\Gamma}}{\sigma}][\pack{\Gamma}/\sigma]}, \upstairs{\qvar{\Gamma,A[\Theta]}}}} \\
&\equiv \rewrite{\eta^\tdot_{\modeof{(\Gamma, x : A[\Theta])}}}{\StI{\tdot}{\rewrite{\pi^{\modeof{\Gamma}}_x}{\upstairs{\Theta}}, \upstairs{\qvar{\Gamma,A[\Theta]}}}} \\
&\equiv \rewrite{\eta^\tdot_{\modeof{(\Gamma, x : A[\Theta])}}}{\StI{\tdot}{\rewrite{\pi^{\modeof{\Gamma}}_x}{\upstairs{\Theta}},  \rewrite{\var{x}}{\StI{\ApEl{p}{\pi^{\modeof{\Gamma}}_x}}{x}}}} \\
&\equiv \StI{\tdot}{\upstairs{\Theta}, x}
\end{align*}
And:
\begin{align*}
&\StE{\tshape{\Delta, A}}{\upstairs{\Theta \uparrow A}}{\sigma}{M[\unpack{\Delta, A}{\sigma}]} \\
&\equiv \StE{\tshape{\Delta, A}}{\StI{\tdot}{\upstairs{\Theta}, x}}{\sigma}{M[\unpack{\Delta, A}{\sigma}]} \\
&\equiv \StE{\tdot}{\StI{\tdot}{\upstairs{\Theta}, x}}{\sigma'}{\StE{(\tshape{\Delta}, \id)}{\sigma'}{\sigma}{M[\unpack{\Delta, A}{\sigma}]}} \\
&\equiv \StE{(\tshape{\Delta}, \id)}{(\upstairs{\Theta}, x)}{\sigma}{M[\unpack{\Delta, A}{\sigma}]} \\
&\equiv \StE{(\tshape{\Delta}, \id)}{(\upstairs{\Theta}, x)}{\sigma, x}{M[\unpack{\Delta, A}{(\sigma, x)}]} \\
&\equiv \StE{(\tshape{\Delta}, \id)}{(\upstairs{\Theta}, x)}{\sigma, x}{M[\unpack{\Delta}{\sigma}, x/x]}
\end{align*}
\end{proof}

\subsubsection{Unit type}
The rules for the unit type are given in Figure~\ref{fig:qit-unit-rules}.

\begin{figure}
\begin{mathpar}
\inferrule*[left=1-form]{~}{\Gamma \qyields `_\Gamma \TYPE} \and
\inferrule*[left=1-intro]{~}{\Gamma \qyields \star_\Gamma : 1_\Gamma} \and
\inferrule*[left=1-elim]{\Gamma \qyields c : C[\id_\Gamma, \star_\Gamma]}{\Gamma, 1_\Gamma \qyields \qunitmatch{c} : C}
\end{mathpar}
\begin{align}
1_\Delta[\Theta] &\equiv 1_\Gamma \\
\star_\Delta[\Theta] &\equiv \star_\Gamma \\ 
\qunitmatch{c}[\Theta \uparrow 1_\Gamma] &\equiv \qunitmatch{c[\Theta]} \\
\nonumber \\
\qunitmatch{c}[\id_\Gamma, \star_\Gamma] &\equiv c
\end{align}
\caption{Rules for the unit type}\label{fig:qit-unit-rules}
\end{figure}

\begin{theorem}
The rules for the unit type can be interpreted over any mode theory containing a comprehension object $p$ that has unit and such that the $\mathsf{F}$-type for $\One_\alpha$ exists.
\end{theorem}

Translating the rules:
\begin{enumerate}[style = multiline, labelwidth = 80pt]
\item[\textsc{1-form}] For any $\upstairs{\Gamma}$ we can form
\begin{align*}
\upstairs{\Gamma} \yields_{\El{p}{\modeof{\Gamma}}} \upstairs{1_\Gamma} :\equiv \F{\One_{\modeof{\Gamma}}}{1} \TYPE
\end{align*}

\item[\textsc{1-intro}] We have
\begin{align*}
\upstairs{\Gamma} \yields_{\One_{\modeof{\Gamma}}} \upstairs{\star_\Gamma} :\equiv \FIs{\One_{\modeof{\Gamma}}}{} : \upstairs{1_\Gamma}
\end{align*}

\item[\textsc{1-elim}] We are given
\begin{align*}
\upstairs{\Gamma} \yields_{\One_{\modeof{\Gamma}}} \upstairs{c} : \upstairs{C[\id_\Gamma, \star_\Gamma]} 
\end{align*}
The type of $\upstairs{c}$ can be simplified to:
\begin{align*}
\upstairs{C[\id_\Gamma, \star_\Gamma]} 
&\equiv \StE{\tshape{\Gamma, 1_\Gamma}}{\upstairs{\id_\Gamma, \star_\Gamma}}{\sigma}{\upstairs{C}[\unpack{\Gamma, 1_\Gamma}{\sigma}]} \\
&\equiv \StE{\tshape{\Gamma, 1_\Gamma}}{\rewrite{\eta^\tdot_{\modeof{\Gamma}}}{\StI{\tdot}{\upstairs{\id_\Gamma}, \upstairs{\star_\Gamma}}}}{\sigma}{\upstairs{C}[\unpack{\Gamma, 1_\Gamma}{\sigma}]} \\
&\equiv \St{\ApEl{p}{\eta^\tdot_{\modeof{\Gamma}}}}{\StE{\tshape{\Gamma, 1_\Gamma}}{\StI{\tdot}{\upstairs{\id_\Gamma}, \upstairs{\star_\Gamma}}}{\sigma}{\upstairs{C}[\unpack{\Gamma, 1_\Gamma}{\sigma}]}} \\
&\equiv \St{\ApEl{p}{\eta^\tdot_{\modeof{\Gamma}}}}{\upstairs{C}[\unpack{\Gamma, 1_\Gamma}{\sigma}][\pack{\Gamma}/\sigma, \upstairs{\star_\Gamma}/x]} \\
&\equiv \St{\ApEl{p}{\eta^\tdot_{\modeof{\Gamma}}}}{\upstairs{C}[\upstairs{\star_\Gamma}/x]} \\
&\equiv \St{\ApEl{p}{\eta^\tdot_{\modeof{\Gamma}}}}{\upstairs{C}[\FIs{\One_{\modeof{\Gamma}}}{}/x]}
\end{align*}
So as our translation use:
\begin{mathpar}
\inferrule*[Left=F-elim]{
\inferrule*[Left=rewrite]{
\inferrule*[Left=s-intro]{
\upstairs{\Gamma} \yields_{\One_{\modeof{\Gamma}}} \upstairs{c} : \St{\ApEl{p}{\eta^\tdot_{\modeof{\Gamma}}}}{\upstairs{C}[\FIs{\One_{\modeof{\Gamma}}}{}/x]}}
{\upstairs{\Gamma} \yields_{\TrPlus{\ApEl{p}{\pi^{\modeof{\Gamma}}_{\One_{\modeof{\Gamma}}}}}{\One_{\modeof{\Gamma}}}} \StI{\ApEl{p}{\pi^{\modeof{\Gamma}}_{\One_{\modeof{\Gamma}}}}}{\upstairs{c}} : \upstairs{C}[\FIs{\One_{\modeof{\Gamma}}}{}/x]}}
{\upstairs{\Gamma} \yields_{\One_{\modeof{\Gamma}.\One_{\modeof{\Gamma}}}} \rewrite{\ApOne{\pi^{\modeof{\Gamma}}_{\One_{\modeof{\Gamma}}}}}{\StI{\ApEl{p}{\pi^{\modeof{\Gamma}}_{\One_{\modeof{\Gamma}}}}}{\upstairs{c}}} : \upstairs{C}[\FIs{\One_{\modeof{\Gamma}}}{}/x]}}
{\upstairs{\Gamma}, z : \upstairs{1_\Gamma} \yields_{\One_{\modeof{\Gamma}}} \FE{z}{x}{\rewrite{\ApOne{\pi^{\modeof{\Gamma}}_{\One_{\modeof{\Gamma}}}}}{\StI{\ApEl{p}{\pi^{\modeof{\Gamma}}_{\One_{\modeof{\Gamma}}}}}{\upstairs{c}}}} : \upstairs{C}}
\end{mathpar}
\end{enumerate}

Checking the equations: \mvrnote{TODO}
\begin{enumerate}[style = multiline, labelwidth = 80pt]
\item[{$\One_\Delta[\Theta] \equiv \One_\Gamma$}:] 
\item[{$\star_\Delta[\Theta] \equiv \star_\Gamma $}:]
\item[{$\qunitmatch{c}[\Theta \uparrow \One_\Gamma] \equiv \qunitmatch{c[\Theta]}$}:]
\item[{$\qunitmatch{c}[\id_\Gamma, \star_\Gamma] \equiv c$}:] 
\end{enumerate}

\subsubsection{$\Sigma$-types}

The rules for $\Sigma$-types are given in Figure~\ref{fig:qit-sigma-rules}. 

% There are two versions of the eliminator; a weak and strong version. The choice of one or the other yields \emph{weak} $\Sigma$-types or \emph{strong} $\Sigma$-types.

\begin{figure}
\begin{mathpar}
\inferrule*[left=$\Sigma$-form]{\Gamma \qyields A \TYPE \and \Gamma, A \qyields B \TYPE}{\Gamma \qyields \Sigma_A B \TYPE} \\
\inferrule*[left=$\Sigma$-pair]{~}{\Gamma, A, B \qyields \qpair{A, B} : (\Sigma_A B)[\proj{\Gamma, A, B};\proj{\Gamma, A}]} \and
\inferrule*[left=$\Sigma$-split]{\Gamma, A, B \qyields c : C[(\proj{\Gamma, A, B};\proj{\Gamma, A}), \qpair{A,B}]}{\Gamma, \Sigma_A B \qyields \qsplit{A, B}(c) : C}
%\inferrule*[left=$\Sigma$-pair]{\Gamma \qyields a : A \and \Gamma \qyields b : B[\hat{a}]}{\Gamma \qyields (a, b) : \Sigma_A B} \and
%\inferrule*[left=$\Sigma$-$\pi_1$]{\Gamma \qyields p : \Sigma_A B}{\Gamma \qyields \pi_1(p) : A} \and
%\inferrule*[left=$\Sigma$-$\pi_2$]{\Gamma \qyields p : \Sigma_A B}{\Gamma \qyields \pi_2(p) : B[\widehat{\pi_1(p)}]} \and
\end{mathpar}
\begin{align}
(\Sigma_A B)[\Theta] &\equiv \Sigma_{A[\Theta]} B[\Theta \uparrow A] \\
\qpair{A, B}[\Theta \uparrow A \uparrow B] &\equiv \qpair{A[\Theta], B[\Theta \uparrow A]} \\
\qsplit{A,B}(c)[\Theta \uparrow \Sigma_A B] &\equiv \qsplit{A[\Theta],B[\Theta \uparrow A]}(c[\Theta \uparrow A \uparrow B])  \\
\nonumber \\
\qsplit{A,B}(c)[(\proj{\Gamma, A, B};\proj{\Gamma, A}), \qpair{A,B}] &\equiv c
%\intertext{(If framework $\mathsf{F}$-types have eta:)}
%\qsplit{A,B}(\qpair{A,B}) &\equiv \qvar{\Gamma, \Sigma_A B}
\end{align}
%\mvrnote{that eta is not general enough}
%\vspace{1cm}
%\begin{mathpar}
%\inferrule*[left=$\Sigma$-split-weak]{\Gamma, A, B \qyields c : C[\proj{\Gamma, A, B};\proj{\Gamma, A}]}{\Gamma, \Sigma_A B \qyields \mathsf{wsplit}_{A, B}(c) : C[\proj{\Gamma, \Sigma_A B}]}
%\end{mathpar}
%\begin{align}
%\mathsf{wsplit}_{A,B}(\qpair{A,B}) &\equiv \qvar{\Gamma, \Sigma_A B} \\
%\mathsf{wsplit}_{A,B}(c)[\Theta \uparrow \Sigma_A B] &\equiv \mathsf{wsplit}_{A[\Theta],B[\Theta \uparrow A]}(c[\Theta \uparrow A \uparrow B]) \\
%\end{align}
%\begin{align}
%%\pi_1(a, b) &\equiv a \\
%%\pi_2(a, b) &\equiv b \\
%%(\pi_1(p), \pi_2(p)) &\equiv p \\
%%(a, b)[\Theta] &\equiv (a[\Theta], b[\Theta])
%\end{align}
\caption{Rules for $\Sigma$-types}\label{fig:qit-sigma-rules}
\end{figure}

Suppose the comprehension object supports $\Sigma$ types in the sense of Definition~\ref{def:supports-sigmas}. In the framework, $\Sigma$-types are the $\mathsf{F}$-types for the mode term
\begin{align*}
\alpha : p, x : \El{p}{\alpha}, y : \El{p}{\alpha.x} \yields \Sigma_1(x,y) : \El{p}{\alpha}
\end{align*}

\begin{theorem}
The rules for $\Sigma$-types can be interpreted over any mode theory containing a comprehension object that supports strong $\Sigma$-types (Definition \ref{def:supports-sigmas}), \mvrnote{and such that the $\mathsf{F}$-type for $\Sigma_1(\alpha,x,y)$ exists whenever needed}
\end{theorem}

We now translate all the rules:
\begin{enumerate}[style = multiline, labelwidth = 80pt]
\item[\textsc{$\Sigma$-form}] 
We are given
\begin{align*}
\upstairs{\Gamma} &\yields_{\El{p}{\modeof{\Gamma}}} \upstairs{A} \TYPE \\
\upstairs{\Gamma}, x : \upstairs{A} &\yields_{\El{p}{\modeof{\Gamma}.x}} \upstairs{B} \TYPE
\end{align*}
which are of the right shape to form
\begin{align*}
\upstairs{\Gamma} \yields_{\El{p}{\modeof{\Gamma}}} \upstairs{\Sigma_A B} :\equiv \F{w. \Sigma_1(\alpha,\fst w, \snd w)}{x : \upstairs{A}, \upstairs{B}} \TYPE
\end{align*}

\item[\textsc{$\Sigma$-pair}] Our goal is a term:
\begin{align*}
\upstairs{\Gamma}, x : \upstairs{A}, y : \upstairs{B} \equiv_{\One_{\modeof{\Gamma}.x.y}} \upstairs{\qpair{A,B}} : \St{\ApEl{p}{\pi^{\modeof{\Gamma}.x}_y;\pi^{\modeof{\Gamma}}_x}}{\upstairs{\Sigma_A B}}
\end{align*}
and we can get this by
\begin{mathpar}
\inferrule*[Left=rewrite]{
\inferrule*[Left=s-intro]{
\inferrule*[Left=F-intro]{~}{
\alpha : \upstairs{\Gamma}, x : \upstairs{A}, y : \upstairs{B} \yields_{\Sigma(x,y)} \FIs{w. \Sigma_1(\fst w, \snd w)}{x,y} : \upstairs{\Sigma_A B}}}
{\alpha : \upstairs{\Gamma}, x : \upstairs{A}, y : \upstairs{B} \yields_{\TrPlus{(\pi^{\modeof{\Gamma}.x}_y;\pi^{\modeof{\Gamma}}_x)}{\Sigma(x,y)}} \StI{(\pi^{\modeof{\Gamma}.x}_y;\pi^{\modeof{\Gamma}}_x)}{\FIs{w. \Sigma_1(\fst w, \snd w)}{x,y}} : \upstairs{\Sigma_A B[\proj{\Gamma, A, B};\proj{\Gamma, A}]}}}
{\alpha : \upstairs{\Gamma}, x : \upstairs{A}, y : \upstairs{B} \yields_{\One_{\modeof{\Gamma}.x.y}} \rewrite{\fibpair{x,y}}{\StI{(\pi^{\modeof{\Gamma}.x}_y;\pi^{\modeof{\Gamma}}_x)}{\FIs{w. \Sigma_1(\fst w, \snd w)}{x,y}}} : \upstairs{\Sigma_A B[\proj{\Gamma, A, B};\proj{\Gamma, A}]}}
\end{mathpar}

\item[\textsc{$\Sigma$-split}] 
First, the substitution $(\proj{\Gamma, A, B};\proj{\Gamma, A}), \qpair{A,B}$ is translated to:
\begin{align*}
&\upstairs{(\proj{\Gamma, A, B};\proj{\Gamma, A}), \qpair{A,B}} \\
&\equiv \rewrite{\eta^\tdot_{\modeof{\Gamma}.x.y}}{\StI{\tdot}{\upstairs{\proj{\Gamma, A, B};\proj{\Gamma, A}}, \upstairs{\qpair{A,B}}}} \\
&\equiv \rewrite{\eta^\tdot_{\modeof{\Gamma}.x.y}}{\StI{\tdot}{\rewrite{\pi^{\modeof{\Gamma}.x}_y;\pi^{\modeof{\Gamma}}_x}{\upstairs{\id_\Gamma}}, \rewrite{\fibpair{x,y}}{\StI{(\pi^{\modeof{\Gamma}.x}_y;\pi^{\modeof{\Gamma}}_x)}{\FIs{w. \Sigma_1(\fst w, \snd w)}{x,y}}}}} \\
&\equiv \rewrite{\eta^\tdot_{\modeof{\Gamma}.x.y};(\pi^{\modeof{\Gamma}.x}_y;\pi^{\modeof{\Gamma}}_x,\fibpair{x,y})}{\StI{\tdot}{\upstairs{\id_\Gamma},\FIs{w. \Sigma_1(\fst w, \snd w)}{x,y}}} \\
&\equiv \rewrite{\pair{x,y}}{\StI{\tdot}{\upstairs{\id_\Gamma},\FIs{w. \Sigma_1(\fst w, \snd w)}{x,y}}}
\end{align*}
So the type $C[(\proj{\Gamma, A, B};\proj{\Gamma, A}), \qpair{A,B}]$ becomes:
\begin{align*}
&\upstairs{C[(\proj{\Gamma, A, B};\proj{\Gamma, A}), \qpair{A,B}]} \\
&\equiv \StE{\tshape{\Gamma, \Sigma_A B}}{\upstairs{(\proj{\Gamma, A, B};\proj{\Gamma, A}), \qpair{A,B}}}{\sigma}{\upstairs{C}[\unpack{\Gamma, \Sigma_A B}{\sigma}]} \\
&\equiv \StE{\tshape{\Gamma, \Sigma_A B}}{\rewrite{\pair{x,y}}{\StI{\tdot}{\upstairs{\id_\Gamma},\FIs{w. \Sigma_1(\fst w, \snd w)}{x,y}}}}{\sigma}{\upstairs{C}[\unpack{\Gamma, \Sigma_A B}{\sigma}]} \\
&\equiv \St{\ApEl{p}{\pair{x,y}}}{\StE{\tshape{\Gamma, \Sigma_A B}}{\StI{\tdot}{\upstairs{\id_\Gamma},\FIs{w. \Sigma_1(\fst w, \snd w)}{x,y}}}{\sigma}{\upstairs{C}[\unpack{\Gamma, \Sigma_A B}{\sigma}]}} \\
&\equiv \St{\ApEl{p}{\pair{x,y}}}{\StE{\tshape{\Gamma, \Sigma_A B}}{\StI{\tdot}{\StI{\tshape{\Gamma}}{\pack{\downstairs{\Gamma}}},\FIs{w. \Sigma_1(\fst w, \snd w)}{x,y}}}{\sigma}{\upstairs{C}[\unpack{\Gamma, \Sigma_A B}{\sigma}]}} \\
&\equiv \St{\ApEl{p}{\pair{x,y}}}{\upstairs{C}[\unpack{\Gamma, \Sigma_A B}{\sigma}][(\pack{\upstairs{\Gamma}},\FIs{w. \Sigma_1(\fst w, \snd w)}{x,y})/\sigma]} \\
&\equiv \St{\ApEl{p}{\pair{x,y}}}{\upstairs{C}[\FIs{w. \Sigma_1(\fst w, \snd w)}{x,y}/z]}
\end{align*}
So as the translation use:
\begin{mathpar}
\inferrule*[Left=F-elim]{
\inferrule*[Left=rewrite]{
\inferrule*[Left=s-intro]{\upstairs{\Gamma}, x : \upstairs{A}, y : \upstairs{B} \yields_{\One_{\modeof{\Gamma}.x.y}} \upstairs{c} : \St{\ApEl{p}{\pair{x,y}}}{\upstairs{C}[\FIs{w. \Sigma_1(\fst w, \snd w)}{x,y}/z]}}
{\upstairs{\Gamma}, x : \upstairs{A}, y : \upstairs{B} \yields_{\TrPlus{\ApEl{p}{\tsplit{x,y}}}{\One_{\modeof{\Gamma}.x.y}}} \StI{\ApEl{p}{\tsplit{x,y}}}{\upstairs{c}} : \upstairs{C}[\FIs{w. \Sigma_1(\fst w, \snd w)}{x,y}/z]}}
{\upstairs{\Gamma}, x : \upstairs{A}, y : \upstairs{B} \yields_{\One_{\modeof{\Gamma}.\Sigma_1(x,y)}} \rewrite{\ApOne{\tsplit{x,y}}}{\StI{\ApEl{p}{\tsplit{x,y}}}{\upstairs{c}}} : \upstairs{C}[\FIs{w. \Sigma_1(\fst w, \snd w)}{x,y}/z]}}
{\upstairs{\Gamma}, z : \upstairs{\Sigma_A B} \yields_{\One_{\modeof{\Gamma}.z}} \FE{z}{x,y}{\rewrite{\ApOne{\tsplit{x,y}}}{\StI{\ApEl{p}{\tsplit{x,y}}}{\upstairs{c}}}} : \upstairs{C}}
\end{mathpar}
\end{enumerate}

Now the equations:
\begin{enumerate}[style = multiline, labelwidth = 80pt]
\item[{$(\Sigma_A B)[\Theta] \equiv \Sigma_{A[\Theta]} B[\Theta \uparrow A]$}:] 
\begin{align*}
&\upstairs{(\Sigma_A B)[\Theta]} \\
&\equiv \StE{\tshape{\Delta}}{\upstairs{\Theta}}{\sigma}{\F{w. \Sigma_1(\modeof{\Delta},\fst w, \snd w)}{x : \upstairs{A}, \upstairs{B}}[\unpack{\Delta}{\sigma}]} \\
&\equiv \StE{\tshape{\Delta}}{\upstairs{\Theta}}{\sigma}{\F{w. \Sigma_1(\TrPlus{\tshape{\Delta}}{\sigma},\fst w, \snd w)}{x : \upstairs{A}[\unpack{\Delta}{\sigma}], \upstairs{B}[\unpack{\Delta}{\sigma}, x / x] }} \\
&\equiv \F{w. \Sigma_1(\modeof{\Gamma},\fst w, \snd w)}{x : \StE{\tshape{\Delta}}{\upstairs{\Theta}}{\sigma}{\upstairs{A}[\unpack{\Delta}{\sigma}]}, \StE{(\sigma : \tshape{\Delta}, \id)}{(\upstairs{\Theta},x)}{\sigma, x}{\upstairs{B}[\unpack{\Delta}{\sigma}, x/x]} } \\
&\equiv \F{w. \Sigma_1(\modeof{\Gamma},\fst w, \snd w)}{x : \StE{\tshape{\Delta}}{\upstairs{\Theta}}{\sigma}{\upstairs{A}[\unpack{\Delta}{\sigma}]}, \StE{\tshape{\Delta, A}}{\StI{\tdot}{\upstairs{\Theta},x}}{\sigma, x}{\upstairs{B}[\unpack{\Delta, A}{\sigma}]} } \\
&\equiv \F{w. \Sigma_1(\modeof{\Gamma},\fst w, \snd w)}{x : \StE{\tshape{\Delta}}{\upstairs{\Theta}}{\sigma}{\upstairs{A}[\unpack{\Delta}{\sigma}]}, \StE{\tshape{\Delta, A}}{\upstairs{\Theta \uparrow A}}{\sigma, x}{\upstairs{B}[\unpack{\Delta, A}{\sigma}]} } \\
&\equiv \F{w. \Sigma_1(\modeof{\Gamma},\fst w, \snd w)}{x : \upstairs{A[\Theta]}, \upstairs{B[\Theta \uparrow A]}} \\
&\equiv \upstairs{\Sigma_{A[\Theta]} B[\Theta \uparrow A]}
\end{align*}

\item[{$\qpair{A, B}[\Theta \uparrow A \uparrow B] \equiv \qpair{A[\Theta], B[\Theta \uparrow A]}$}:]
\begin{align*}
&\upstairs{\qpair{A, B}[\Theta \uparrow A \uparrow B]} \\
&\equiv \StE{\tshape{\Delta, A, B}}{\upstairs{\Theta \uparrow A \uparrow B}}{\sigma}{\upstairs{\qpair{A, B}}[\unpack{\Delta, A, B}{\sigma}]} \\
&\equiv \StE{(\tshape{\Delta}, \id, \id)}{(\upstairs{\Theta},x,y)}{\sigma,x,y}{\upstairs{\qpair{A, B}}[\unpack{\Delta}{\sigma}, x/x, y/y]} \\
&\equiv \StE{(\tshape{\Delta}, \id, \id)}{(\upstairs{\Theta},x,y)}{\sigma,x,y}{\rewrite{\fibpair{x,y}}{\StI{(\pi^{\modeof{\Delta}.x}_y;\pi^{\modeof{\Delta}}_x)}{\FIs{w. \Sigma_1(\fst w, \snd w)}{x,y}}}[\unpack{\Delta}{\sigma}, x/x, y/y]} \\
&\equiv \StE{(\tshape{\Delta}, \id, \id)}{(\upstairs{\Theta},x,y)}{\sigma,x,y}{\rewrite{\fibpair{x,y}}{\StI{(\pi^{\TrPlus{\tshape{\Delta}}{\sigma}.x}_y;\pi^{\TrPlus{\tshape{\Delta}}{\sigma}}_x)}{\FIs{w. \Sigma_1(\fst w, \snd w)}{x,y}}}} \\
&\equiv \rewrite{\fibpair{x,y}}{\StI{(\pi^{\modeof{\Gamma}.x}_y;\pi^{\modeof{\Gamma}}_x)}{\FIs{w. \Sigma_1(\fst w, \snd w)}{x,y}}} \\
&\equiv \upstairs{\qpair{A, B}[\Theta \uparrow A \uparrow B]}
\end{align*}

\item[{$\qsplit{A,B}(c)[\Theta \uparrow \Sigma_A B] \equiv \qsplit{A[\Theta],B[\Theta \uparrow A]}(c[\Theta \uparrow A \uparrow B])$}:]
\begin{align*}
&\upstairs{\qsplit{A,B}(c)[\Theta \uparrow \Sigma_A B]} \\
&\equiv \StE{\tshape{\Delta, \Sigma_A B}}{\upstairs{\Theta \uparrow \Sigma_A B}}{\sigma}{(\FE{z}{x,y}{\rewrite{\ApOne{\tsplit{x,y}}}{\StI{\ApEl{p}{\tsplit{x,y}}}{\upstairs{c}}}})[\unpack{\Delta, \Sigma_A B}{\sigma}]} \\
&\equiv \StE{(\tshape{\Delta}, \id)}{(\upstairs{\Theta}, z)}{\sigma, z}{(\FE{z}{x,y}{\rewrite{\ApOne{\tsplit{x,y}}}{\StI{\ApEl{p}{\tsplit{x,y}}}{\upstairs{c}}}})[\unpack{\Delta}{\sigma}, z/z]} \\
&\equiv \StE{(\tshape{\Delta}, \id)}{(\upstairs{\Theta}, z)}{\sigma, z}{\FE{z}{x,y}{\rewrite{\ApOne{\tsplit{x,y}}}{\StI{\ApEl{p}{\tsplit{x,y}}}{\upstairs{c}[\unpack{\Delta}{\sigma}, x/x, y/y]}}}} \\
&\equiv \StE{(\tshape{\Delta}, \id)}{(\upstairs{\Theta}, z)}{\sigma, z}{\FE{z}{x,y}{\rewrite{\ApOne{\tsplit{x,y}}}{\StI{\ApEl{p}{\tsplit{x,y}}}{\\&\qquad \StE{(\tshape{\Delta}, \id, \id)}{(\StI{\tshape{\Delta}}{\sigma},x, y)}{\sigma,x,y}{\upstairs{c}[\unpack{\Delta}{\sigma}, x/x, y/y]}}}}} \\
&\equiv \FE{z}{x,y}{\rewrite{\ApOne{\tsplit{x,y}}}{\StI{\ApEl{p}{\tsplit{x,y}}}{\StE{(\tshape{\Delta}, \id, \id)}{(\upstairs{\Theta},x, y)}{\sigma,x,y}{\upstairs{c}[\unpack{\Delta}{\sigma}, x/x, y/y]}}}} \\
&\equiv \FE{z}{x,y}{\rewrite{\ApOne{\tsplit{x,y}}}{\StI{\ApEl{p}{\tsplit{x,y}}}{\StE{\tshape{\Delta, A, B}}{\upstairs{\Theta \uparrow A \uparrow B}}{\sigma}{\upstairs{c}[\unpack{\Delta, A, B}{\sigma}]}}}} \\
&\equiv \FE{z}{x,y}{\rewrite{\ApOne{\tsplit{x,y}}}{\StI{\ApEl{p}{\tsplit{x,y}}}{\upstairs{c[\Theta \uparrow A \uparrow B]}}}} \\
&\equiv \upstairs{\qsplit{A[\Theta],B[\Theta \uparrow A]}(c[\Theta \uparrow A \uparrow B])}
\end{align*}

\item[{$\qsplit{A,B}(c)\allowbreak[(\proj{\Gamma, A, B};\proj{\Gamma, A}), \allowbreak\qpair{A,B}] \equiv c$}:] 
We have already calculated the translation of the substitution, when translating $\qsplit{}$.
\begin{align*}
&\upstairs{\qsplit{A,B}(c)[(\proj{\Gamma, A, B};\proj{\Gamma, A}), \qpair{A,B}]} \\
&\equiv \StE{\tshape{\Gamma, \Sigma_A B}}{\upstairs{(\proj{\Gamma, A, B};\proj{\Gamma, A}), \qpair{A,B}}}{\sigma}{\upstairs{\qsplit{A,B}(c)}[\unpack{\Gamma, \Sigma_A B}{\sigma}]} \\
&\equiv \StE{\tshape{\Gamma, \Sigma_A B}}{\rewrite{\pair{x,y}}{\StI{\tdot}{\upstairs{\id_\Gamma},\FIs{w. \Sigma_1(\fst w, \snd w)}{x,y}}}}{\sigma}{\upstairs{\qsplit{A,B}(c)}[\unpack{\Gamma, \Sigma_A B}{\sigma}]} \\
&\equiv \rewrite{\ApOne{\pair{x,y}}}{\StI{\ApEl{p}{\pair{x,y}}}{\StE{\tshape{\Gamma, \Sigma_A B}}{\StI{\tdot}{\upstairs{\id_\Gamma},\FIs{w. \Sigma_1(\fst w, \snd w)}{x,y}}}{\sigma}{\upstairs{\qsplit{A,B}(c)}[\unpack{\Gamma, \Sigma_A B}{\sigma}]}}} \\
&\equiv \rewrite{\ApOne{\pair{x,y}}}{\StI{\ApEl{p}{\pair{x,y}}}{\StE{\tshape{\Gamma, \Sigma_A B}}{\StI{\tdot}{\StI{\tshape{\Gamma}}{\pack{\Gamma}},\FIs{w. \Sigma_1(\fst w, \snd w)}{x,y}}}{\sigma}{\upstairs{\qsplit{A,B}(c)}[\unpack{\Gamma, \Sigma_A B}{\sigma}]}}} \\
&\equiv \rewrite{\ApOne{\pair{x,y}}}{\StI{\ApEl{p}{\pair{x,y}}}{\upstairs{\qsplit{A,B}(c)}[\unpack{\Gamma, \Sigma_A B}{\sigma}][(\pack{\Gamma},\FIs{w. \Sigma_1(\fst w, \snd w)}{x,y})/\sigma]}} \\
&\equiv \rewrite{\ApOne{\pair{x,y}}}{\StI{\ApEl{p}{\pair{x,y}}}{\upstairs{\qsplit{A,B}(c)}[\FIs{w. \Sigma_1(\fst w, \snd w)}{x,y}/z]}} \\
&\equiv \rewrite{\ApOne{\pair{x,y}}}{\StI{\ApEl{p}{\pair{x,y}}}{\FE{z}{x,y}{\rewrite{\ApOne{\tsplit{x,y}}}{\StI{\ApEl{p}{\tsplit{x,y}}}{\upstairs{c}}}}[\FIs{w. \Sigma_1(\fst w, \snd w)}{x,y}/z]}} \\
&\equiv \rewrite{\ApOne{\pair{x,y}}}{\StI{\ApEl{p}{\pair{x,y}}}{\FE{\FIs{w. \Sigma_1(\fst w, \snd w)}{x,y}}{x,y}{\rewrite{\ApOne{\tsplit{x,y}}}{\StI{\ApEl{p}{\tsplit{x,y}}}{\upstairs{c}}}}}} \\
&\equiv \rewrite{\ApOne{\pair{x,y}}}{\StI{\ApEl{p}{\pair{x,y}}}{\rewrite{\ApOne{\tsplit{x,y}}}{\StI{\ApEl{p}{\tsplit{x,y}}}{\upstairs{c}}}}} \\
&\equiv \rewrite{\ApOne{\pair{x,y};\tsplit{x,y}}}{\StI{\ApEl{p}{\pair{x,y};\tsplit{x,y}}}{\upstairs{c}}} \\
&\equiv \upstairs{c}
\end{align*}
\end{enumerate}

\subsubsection{$\Pi$-types}

The rules for $\Pi$ types are given in Figure~\ref{fig:qit-pi-rules}. Stability of $\qapp{}$ under substitution is derivable from stability of $\qlam{}$ using the beta and eta rules:
\begin{align*}
\qapp{f}[\Theta \uparrow A]
&\equiv \qapp{\qlam{\qapp{f}[\Theta \uparrow A]}} \\
&\equiv \qapp{\qlam{\qapp{f}}[\Theta]} \\
&\equiv \qapp{f[\Theta]}
\end{align*}

\begin{figure}
\begin{mathpar}
\inferrule*[left=$\Pi$-form]{\Gamma \qyields A \TYPE \and \Gamma, A \qyields B \TYPE}{\Gamma \qyields \Pi_A B \TYPE} \\
\inferrule*[left=$\Pi$-intro]{\Gamma, A \qyields b : B}{\Gamma \qyields \qlam{b} : \Pi_A B} \and
\inferrule*[left=$\Pi$-elim]{\Gamma \qyields f : \Pi_A B}{\Gamma, A \qyields \qapp{f} : B}
\end{mathpar}
\begin{align}
(\Pi_A B)[\Theta] &\equiv \Pi_{A[\Theta]} B[\Theta \uparrow A] \\
\qlam{b}[\Theta] &\equiv \qlam{b[\Theta \uparrow A]} \\
\nonumber \\
\qapp{\qlam{b}} &\equiv b \\
\qlam{\qapp{f}} &\equiv f
\end{align}
\caption{Rules for $\Pi$-types}\label{fig:qit-pi-rules}
\end{figure}

In the framework, $\Pi$-types are the $\mathsf{U}$-types for the mode term
\begin{align*}
\alpha : p, x : \El{p}{\alpha}, c : \El{p}{\alpha} \yields \Pi_1(\alpha,x,c) :\equiv \TrPlus{\ApEl{p}{\pi^\alpha_x}}{c} : \El{p}{\alpha.x}
\end{align*}

\begin{theorem}
The rules for $\Pi$-types can be interpreted over any mode theory with a comprehension object that supports $\Pi$ types (Definition~\ref{def:supports-pis}) and such that the $\mathsf{U}$-types for $\Pi_1$ exist.
\end{theorem}

\begin{enumerate}
\item[\textsc{$\Pi$-form}] We are given
\begin{align*}
\upstairs{\Gamma} &\yields_{\El{p}{\modeof{\Gamma}}} \upstairs{A} \TYPE \\
\upstairs{\Gamma}, x : \upstairs{A} &\yields_{\El{p}{\modeof{\Gamma}.x}} \upstairs{B} \TYPE
\end{align*}
which are of the right shape to form
\begin{align*}
\upstairs{\Gamma} \yields_{\El{p}{\modeof{\Gamma}}} \upstairs{\Pi_A B} :\equiv \U{c.\Pi_1(\modeof{\Gamma},x,c)}{x : \upstairs{A}}{\upstairs{B}} \TYPE
\end{align*}
\item[\textsc{$\Pi$-intro}] We are given a term
\begin{align*}
\upstairs{\Gamma}, x : \upstairs{A} &\yields_{\One_{\modeof{\Gamma}.x}} \upstairs{b} : \upstairs{B}
\end{align*}
So use
\begin{mathpar}
\inferrule*[Left=U-intro]{
\inferrule*[Left=rewrite]{
\upstairs{\Gamma}, x : \upstairs{A} \yields_{\One_{\modeof{\Gamma}.x}} \upstairs{b} : \upstairs{B}}
{\upstairs{\Gamma}, x : \upstairs{A} \yields_{\TrPlus{\pi^{\modeof{\Gamma}}_x}{\One_{\modeof{\Gamma}}}} \rewrite{\pinv{x}}{\upstairs{b}} : \upstairs{B}}}
{\upstairs{\Gamma} \yields_{\One_{\modeof{\Gamma}}} \UI{x}{\rewrite{\pinv{x}}{\upstairs{b}}} : \upstairs{\Pi_A B}}
\end{mathpar}
\item[\textsc{$\Pi$-elim}] We are given
\begin{align*}
\upstairs{\Gamma} &\yields_{\One_{\modeof{\Gamma}}} \upstairs{f} : \upstairs{\Pi_A B}
\end{align*}
and want to produce
\begin{align*}
\upstairs{\Gamma}, x : \upstairs{A} &\yields_{\One_{\modeof{\Gamma}.x}} \upstairs{\qapp{A,B}(f)} : \upstairs{B}
\end{align*}
This is:
\begin{mathpar}
\inferrule*[Left=rewrite]{
\inferrule*[Left=U-elim]{
\upstairs{\Gamma} \yields_{\One_{\modeof{\Gamma}}} \upstairs{f} : \upstairs{\Pi_A B}}
{\upstairs{\Gamma}, x : \upstairs{A} \yields_{\TrPlus{\ApEl{p}{\pi^{\modeof{\Gamma}}_x}}{\One_{\modeof{\Gamma}}}} \UE{\upstairs{f}}{x} : \upstairs{B}}}
{\upstairs{\Gamma}, x : \upstairs{A} \yields_{\One_{\modeof{\Gamma}.x}} \rewrite{\ApOne{\pi^{\modeof{\Gamma}}_x}}{\UE{\upstairs{f}}{x}} : \upstairs{B}}
\end{mathpar}
\end{enumerate}

And the equations:
\begin{enumerate}[style = multiline, labelwidth = 80pt]
\item[{$(\Pi_A B)[\Theta] \equiv \Pi_{A[\Theta]} B[\Theta \uparrow A]$}:] Similar to the case for $\Sigma$:
\begin{align*}
&\upstairs{(\Pi_A B)[\Theta]} \\
&\equiv \StE{\tshape{\Delta}}{\upstairs{\Theta}}{\sigma}{\U{c. \Pi_1(\modeof{\Gamma},x,c)}{x : \upstairs{A}}{\upstairs{B}}[\unpack{\Delta}{\sigma}]} \\
&\equiv \StE{\tshape{\Delta}}{\upstairs{\Theta}}{\sigma}{\U{c. \Pi_1(\TrPlus{\tshape{\Delta}}{\sigma},x,c)}{x : \upstairs{A}[\unpack{\Delta}{\sigma}]}{\upstairs{B}[\unpack{\Delta}{\sigma}], x/x}} \\
&\equiv \U{c. \Pi_1(\modeof{\Delta},x,c)}{x : \StE{\tshape{\Delta}}{\upstairs{\Theta}}{\sigma}{\upstairs{A}[\unpack{\Delta}{\sigma}]}}{\StE{(\tshape{\Delta}, \id)}{(\upstairs{\Theta}, x)}{\sigma, x}{\upstairs{B}[\unpack{\Delta}{\sigma}, x/x]}} \\
&\equiv \U{c. \Pi_1(\modeof{\Delta},x,c)}{x : \StE{\tshape{\Delta}}{\upstairs{\Theta}}{\sigma}{\upstairs{A}[\unpack{\Delta}{\sigma}]}}{\StE{\tshape{\Delta,A}}{\StI{\tdot}{\upstairs{\Theta}, x}}{\sigma}{\upstairs{B}[\unpack{\Delta, A}{\sigma}]}} \\
&\equiv \U{c. \Pi_1(\modeof{\Delta},x,c)}{x : \StE{\tshape{\Delta}}{\upstairs{\Theta}}{\sigma}{\upstairs{A}[\unpack{\Delta}{\sigma}]}}{\StE{\tshape{\Delta,A}}{\upstairs{\Theta \uparrow A}}{\sigma}{\upstairs{B}[\unpack{\Delta, A}{\sigma}]}} \\
&\equiv \U{c. \Pi_1(\modeof{\Delta},x,c)}{x : \upstairs{A[\Theta]}}{\upstairs{B[\Theta \uparrow A]}} \\
&\equiv \upstairs{\Pi_{A[\Theta]} B[\Theta \uparrow A]}
\end{align*}

\item[{$\qlam{b}[\Theta] \equiv \qlam{b[\Theta \uparrow A]}$}:]
\begin{align*}
&\upstairs{\qlam{b}[\Theta]} \\
&\equiv \StE{\tshape{\Delta}}{\upstairs{\Theta}}{\sigma}{\upstairs{\qlam{b}}[\unpack{\Delta}{\sigma}]} \\
&\equiv \StE{\tshape{\Delta}}{\upstairs{\Theta}}{\sigma}{\UI{x}{\rewrite{\pinv{x}}{\upstairs{b}}}[\unpack{\Delta}{\sigma}]} \\
&\equiv \StE{\tshape{\Delta}}{\upstairs{\Theta}}{\sigma}{\UI{x}{\rewrite{\pinv{x}}{\upstairs{b}[\unpack{\Delta}{\sigma}, x/x]}}} \\
&\equiv \UI{x}{\rewrite{\pinv{x}}{\StE{(\tshape{\Delta}, \id)}{(\upstairs{\Theta}, x)}{\sigma, x}{\upstairs{b}[\unpack{\Delta}{\sigma}, x/x]}}} \\
&\equiv \UI{x}{\rewrite{\pinv{x}}{\StE{\tshape{\Delta, A}}{\upstairs{\Theta \uparrow A}}{\sigma}{\upstairs{b}[\unpack{\Delta,A}{\sigma}]}}} \\
&\equiv \UI{x}{\rewrite{\pinv{x}}{\upstairs{b[\Theta \uparrow A]}}} \\
&\equiv \upstairs{\qlam{b[\Theta \uparrow A]}}
\end{align*}

\item[{$\qapp{\qlam{b}} \equiv b$}:] 
\begin{align*}
&\upstairs{\qapp{\qlam{b}}} \\
&\equiv \rewrite{\ApOne{\pi^{\modeof{\Gamma}}_x}}{\UE{\upstairs{\qlam{b}}}{x}} \\
&\equiv \rewrite{\ApOne{\pi^{\modeof{\Gamma}}_x}}{\UE{\UI{x}{\rewrite{\pinv{x}}{\upstairs{b}}}}{x}} \\
&\equiv \rewrite{\ApOne{\pi^{\modeof{\Gamma}}_x}}{\UE{\UI{x}{\rewrite{\pinv{x}}{\upstairs{b}}}}{x}} \\
&\equiv \rewrite{\ApOne{\pi^{\modeof{\Gamma}}_x}}{\rewrite{\pinv{x}}{\upstairs{b}}} \\
&\equiv \upstairs{b}
\end{align*}

\item[{$\qlam{\qapp{f}} \equiv f$}]
\begin{align*}
&\upstairs{\qlam{\qapp{f}}} \\
&\equiv \UI{x}{\rewrite{\pinv{x}}{\upstairs{\qapp{f}}}} \\
&\equiv \UI{x}{\rewrite{\pinv{x}}{\rewrite{\ApOne{\pi^{\modeof{\Gamma}}_x}}{\UE{\upstairs{f}}{x}}}} \\
&\equiv \UI{x}{\UE{\upstairs{f}}{x}} \\
&\equiv \upstairs{f}
\end{align*}
\end{enumerate}

\subsection{Adjoint Type Theory}

Dependent right adjoint types have previously been studied in \mvrnote{cite}. An operation $\lock$ on contexts and a type former $\Rtype{}$ are added, with $\lock$ adjoint to $\Rtype{}$ in the sense that terms of $\Gamma, \lock \qyields A \TYPE$ correspond bijectively with terms of $\Gamma \qyields \Rtype{A} \TYPE$.

We give the rules of dependent right adjoint types in Figure~\ref{fig:qit-adjoint-rules}. These are presented in an algebraic style with explicit weakenings and substitutions; rules written in this form are more convenient to translate into the framework. The turnstile is written as $\qyields$ to distinguish judgements in the object language from judgements in the framework.

The rules of right adjoint types are mildly generalised to allow an adjoint between two different comprehension objects $p$ and $q$. Each judgement is marked with the comprehension object to which it belongs. Object-level contexts are translated to framework contexts as before, with $\qyields_p \Gamma \CTX$ given an associated mode type morphism $\tshape{\Gamma} : p \tcell \ctxtuple{\downstairs{\Gamma}}$ and $\qyields_q \Delta \CTX$ given a mode type morphism $\tshape{\Delta} : q \tcell \ctxtuple{\downstairs{\Delta}}$.

Note that as our adjoint is no longer an endofunctor, judgements with subscript $p$ necessarily have no locks in the context and those with subscript $q$ have exactly one. To recover the type theory of \mvrnote{ref}, erase the subscripts; this also allows a context to contain multiple locks. 

The framework suggests rules for dependent \emph{left} adjoints $\Ltype{}$, also given in Figure~\ref{fig:qit-adjoint-rules}. 

Similar to $\Pi$-types, stability of $\RE{}$ under substitution follows from stability of $\RI{}$ using eta expansion:
\begin{align*}
\RE{b}[\Theta, \lock] 
&\equiv \RE{\RI{\RE{b}[\Theta, \lock]}} \\
&\equiv \RE{\RI{\RE{b}}[\Theta]} \\
&\equiv \RE{b[\Theta]}
\end{align*}

\begin{figure}
\begin{mathpar}
\inferrule*[left=ctx-$\lock$]{\qyields_p \Gamma \CTX}{\qyields_q \Gamma, \lock \CTX} \and
\inferrule*[left=sub-$\lock$]{\Gamma \qyields_p \Theta : \Delta}{\Gamma, \lock \qyields_q \Theta, \lock : \Delta, \lock} \\
\inferrule*[left=R-form]{\Gamma, \lock \qyields_q A \TYPE}{\Gamma \qyields_p \Rtype{A} \TYPE} \\
\inferrule*[left=R-intro]{\Gamma, \lock \qyields_q a : A}{\Gamma \qyields_p \RI{a} : \Rtype{A}} \and
\inferrule*[left=R-elim]{\Gamma \qyields_p b : \Rtype{B}}{\Gamma, \lock \qyields_q \RE{b} : B} \\
\inferrule*[left=L-form]{\Gamma \qyields_p A \TYPE}{\Gamma, \lock \qyields_q \Ltype{A} \TYPE} \\
\inferrule*[left=L-intro]{~}{\Gamma, A, \lock \qyields_q \LI{A} : \Ltype{A}[\proj{\Gamma, A}, \lock]} \and
\inferrule*[left=L-elim]{\Gamma, A, \lock \qyields_q c : C[\proj{\Gamma, A}, \lock, \LI{A}]}{\Gamma, \lock, \Ltype{A} \qyields_q \LE{c} : C} \\
\end{mathpar}
\begin{align}
\id_\Gamma , \lock &\equiv \id_{\Gamma, \lock} \\
(\Theta, \lock) ; (\kappa, \lock) &\equiv (\Theta ; \kappa), \lock \\
\nonumber \\
\Rtype{A}[\Theta] &\equiv \Rtype{(A[\Theta, \lock])} \\
\RI{a}[\Theta] &\equiv \RI{a[\Theta, \lock]} \\
\RE{\RI{a}} &\equiv a \\
\RI{\RE{b}} &\equiv b \\
\nonumber \\
\Ltype{A}[\Theta, \lock] &\equiv \Ltype{(A[\Theta])} \\
\LI{A}[\Theta \uparrow A, \lock] &\equiv \LI{A[\Theta]} \\
\LE{c}[\Theta, \lock \uparrow \Ltype{A}] &\equiv \LE{c[\Theta \uparrow A, \lock]} \\
\LE{c}[\proj{\Gamma, A}, \lock, \LI{A}] &\equiv c
\end{align}
\caption{Rules for Dependent Adjoints}\label{fig:qit-adjoint-rules}
\end{figure}

Suppose our mode theory has two comprehension objects $p$ and $q$, with a morphism $f$ between them (Definition~\ref{def:morphism-comprehension-object}). 

First the structural rules.
\begin{enumerate}
\item[\textsc{ctx-$\lock$}] Given $\yields \upstairs{\Gamma} \CTX$, we define $\upstairs{\Gamma, \lock} :\equiv \upstairs{\Gamma}$, but with
\begin{align*}
\tshape{\Gamma, \lock} :\equiv f ; \tshape{\Gamma}
\end{align*}
%
\item[\textsc{sub-$\lock$}] We are given
\begin{align*}
\upstairs{\Gamma} \yields_{\modeof{\Gamma}} \upstairs{\Theta} : \St{\tshape{\Delta}}{\ctxtuple{\upstairs{\Delta}}}
\end{align*}
and need to produce:
\begin{align*}
\upstairs{\Gamma, \lock} \yields_{\modeof{(\Gamma, \lock)}} \upstairs{\Theta, \lock} : \St{\tshape{\Delta, \lock}}{\ctxtuple{\upstairs{\Delta, \lock}}}
\end{align*}
But because $\lock$ does not change the translation of the context and
\begin{align*}
\modeof{(\Gamma, \lock)} \equiv \TrPlus{\tshape{\Gamma, \lock}}{\pack{\downstairs{\Gamma, \lock}}} \equiv \TrPlus{(f;\tshape{\Gamma})}{\pack{\downstairs{\Gamma}}} \equiv \TrPlus{f}{\modeof{\Gamma}}
\end{align*}
we can define:
\begin{align*}
\upstairs{\Theta, \lock} :\equiv \StI{f}{\upstairs{\Theta}}
\end{align*}
\end{enumerate}

The equations are then straightforward:
\begin{enumerate}[style = multiline, labelwidth = 80pt]
\item[{$\id_\Gamma , \lock \equiv \id_{\Gamma, \lock}$}:]
\begin{align*}
\upstairs{\id_\Gamma , \lock}
&\equiv \StI{f}{\upstairs{\id_\Gamma}} \\
&\equiv \StI{f}{\StI{\tshape{\Gamma}}{\pack{\Gamma}}} \\
&\equiv \StI{f;\tshape{\Gamma}}{\pack{\Gamma}} \\
&\equiv \StI{\tshape{\Gamma, \lock}}{\pack{\Gamma, \lock}} \\
&\equiv \upstairs{\id_{\Gamma, \lock}}
\end{align*}

\item[{$(\Theta, \lock) ; (\kappa, \lock) \equiv (\Theta ; \kappa), \lock$}:] 
\begin{align*}
&\upstairs{(\Theta, \lock) ; (\kappa, \lock)} \\
&\equiv \StE{\tshape{\Delta, \lock}}{\upstairs{\Theta, \lock}}{\sigma}{\upstairs{\kappa, \lock}[\unpack{\upstairs{\Delta, \lock}}{\sigma}]} \\
&\equiv \StE{f;\tshape{\Delta}}{\StI{f}{\upstairs{\Theta}}}{\sigma}{\StI{f}{\upstairs{\kappa}}[\unpack{\upstairs{\Delta}}{\sigma}]} \\
&\equiv \StE{\tshape{\Delta}}{\upstairs{\Theta}}{\sigma}{\StI{f}{\upstairs{\kappa}}[\unpack{\upstairs{\Delta}}{\sigma}]} \\
&\equiv \StE{\tshape{\Delta}}{\upstairs{\Theta}}{\sigma}{\StI{f}{\upstairs{\kappa}[\unpack{\upstairs{\Delta}}{\sigma}]}} \\
&\equiv \StI{f}{\StE{\tshape{\Delta}}{\upstairs{\Theta}}{\sigma}{\upstairs{\kappa}[\unpack{\upstairs{\Delta}}{\sigma}]}} \\
&\equiv \StI{f}{\upstairs{\Theta ; \kappa}} \\
&\equiv \upstairs{(\Theta ; \kappa), \lock}
\end{align*}
\end{enumerate}

\subsubsection{$\mathsf{R}$-types}

\begin{enumerate}
\item[\textsc{R-form}] We are given 
\begin{align*}
\upstairs{\Gamma, \lock} \yields_{\El{q}{\TrPlus{f}{\modeof{\Gamma}}}} \upstairs{A} \TYPE
\end{align*}
and can form
\begin{align*}
\upstairs{\Gamma} \yields_{\El{p}{\modeof{\Gamma}}} \upstairs{\Rtype{A}} :\equiv \U{f_1}{1}{\upstairs{A}} \TYPE
\end{align*}

\item[\textsc{R-intro}] We are given 
\begin{align*}
\upstairs{\Gamma,\lock} \yields_{\One_{\TrPlus{f}{\modeof{\Gamma}}}} \upstairs{a} : \upstairs{A}
\end{align*}
and can do:
\begin{mathpar}
\inferrule*[Left=U-intro]{
\inferrule*[Left=rewrite]{
\upstairs{\Gamma,\lock} \yields_{\One_{\TrPlus{f}{\modeof{\Gamma}}}} \upstairs{a} : \upstairs{A}}
{\upstairs{\Gamma,\lock} \yields_{f_1(\One_{\modeof{\Gamma}})} \rewrite{\foneinv{\modeof{\Gamma}}}{\upstairs{a}} : \upstairs{A}}}
{\upstairs{\Gamma} \yields_{\One_{\modeof{\Gamma}}} \UI{()}{\rewrite{\foneinv{\modeof{\Gamma}}}{\upstairs{a}}} : \upstairs{\Rtype{A}}}
\end{mathpar}

\item[\textsc{R-elim}] Given
\begin{align*}
\upstairs{\Gamma} \yields_{\One_{\modeof{\Gamma}}} \upstairs{b} : \U{f_1}{1}{\upstairs{B}}
\end{align*}
we can do:
\begin{mathpar}
\inferrule*[Left=rewrite]{
\inferrule*[Left=U-elim]{
\upstairs{\Gamma} \yields_{\One_{\modeof{\Gamma}}} \upstairs{b} : \U{f_1}{1}{\upstairs{B}}}
{\upstairs{\Gamma} \yields_{f_1(\One_{\modeof{\Gamma}})} \UE{\upstairs{b}}{} : \upstairs{B}}}
{\upstairs{\Gamma} \yields_{\One_{\TrPlus{f}{\modeof{\Gamma}}}} \rewrite{\fone{\modeof{\Gamma}}}{\UE{\upstairs{b}}{}} : \upstairs{B}}
\end{mathpar}
\end{enumerate}
The equations are verified in more or less the same way as for $\Pi$-types.
\begin{enumerate}
\item[{$\Rtype{A}[\Theta] \equiv \Rtype{(A[\Theta, \lock])}$}:] 
\begin{align*}
&\upstairs{\Rtype{A}[\Theta]} \\
&\equiv \StE{\tshape{\Delta}}{\upstairs{\Theta}}{\sigma}{\U{f_1}{1}{\upstairs{A}}[\unpack{\Delta}{\sigma}]} \\
&\equiv \StE{\tshape{\Delta}}{\upstairs{\Theta}}{\sigma}{\U{f_1}{1}{\upstairs{A}[\unpack{\Delta}{\sigma}]}} \\
&\equiv \U{f_1}{1}{\StE{\tshape{\Delta}}{\upstairs{\Theta}}{\sigma}{\upstairs{A}[\unpack{\Delta}{\sigma}]}} \\
&\equiv \U{f_1}{1}{\StE{\tshape{\Delta, \lock}}{\StI{f}{\upstairs{\Theta}}}{\sigma}{\upstairs{A}[\unpack{\Delta}{\sigma}]}} \\
&\equiv \U{f_1}{1}{\StE{\tshape{\Delta, \lock}}{\upstairs{\Theta, \lock}}{\sigma}{\upstairs{A}[\unpack{\Delta, \lock}{\sigma}]}} \\
&\equiv \upstairs{\Rtype{(A[\Theta, \lock])}}
\end{align*}
\mvrnote{The subscript $f_1$ is actually $f_1(\modeof{\Delta})$ which becomes $f_1(\TrPlus{\tshape{\Delta}}{\sigma})$ when the substitution is moved in, like $\Pi$-types}

\item[{$\RI{a}[\Theta] \equiv \RI{a[\Theta, \lock]}$}:] 
\begin{align*}
&\upstairs{\RI{a}[\Theta]} \\
&\equiv \StE{\tshape{\Delta}}{\upstairs{\Theta}}{\sigma}{\UI{()}{\rewrite{\foneinv{\modeof{\Delta}}}{\upstairs{a}}}[\unpack{\Delta}{\sigma}]} \\
&\equiv \StE{\tshape{\Delta}}{\upstairs{\Theta}}{\sigma}{\UI{()}{\rewrite{\foneinv{\TrPlus{\tshape{\Delta}}{\sigma}}}{\upstairs{a}[\unpack{\Delta}{\sigma}]}}} \\
&\equiv \rewrite{\foneinv{\modeof{\Gamma}}}{\UI{()}{\StE{\tshape{\Delta}}{\upstairs{\Theta}}{\sigma}{\upstairs{a}[\unpack{\Delta, \lock}{\sigma}]}}} \\
&\equiv \rewrite{\foneinv{\modeof{\Gamma}}}{\UI{()}{\StE{\tshape{\Delta, \lock}}{\upstairs{\Theta, \lock}}{\sigma}{\upstairs{a}[\unpack{\Delta, \lock}{\sigma}]}}} \\
&\equiv \upstairs{\RI{a[\Theta, \lock]}}
\end{align*}

\item[{$\RE{\RI{a}} \equiv a$}:] As for $\Pi$-types.
\begin{align*}
&\upstairs{\RE{\RI{a}}} \\
&\equiv \rewrite{\fone{\modeof{\Gamma}}}{\UE{\upstairs{\RI{a}}}{}} \\
&\equiv \rewrite{\fone{\modeof{\Gamma}}}{\UE{(\UI{()}{\rewrite{\foneinv{\modeof{\Gamma}}}{\upstairs{a}}})}{}} \\
&\equiv \rewrite{\fone{\modeof{\Gamma}}}{\rewrite{\foneinv{\modeof{\Gamma}}}{\upstairs{a}}} \\
&\equiv \upstairs{a}
\end{align*}

\item[{$\RI{\RE{b}} \equiv b$}:] As for $\Pi$-types.
\begin{align*}
&\upstairs{\RI{\RE{b}}} \\
&\equiv \UI{()}{\rewrite{\foneinv{\modeof{\Gamma}}}{\upstairs{\RE{b}}}} \\
&\equiv \UI{()}{\rewrite{\foneinv{\modeof{\Gamma}}}{\rewrite{\fone{\modeof{\Gamma}}}{\UE{\upstairs{b}}{}}}} \\
&\equiv \UI{()}{\UE{\upstairs{b}}{}} \\
&\equiv \upstairs{b}
\end{align*}
\end{enumerate}

\subsubsection{$\mathsf{L}$-types}

And now the left adjoint:

\begin{enumerate}
\item[\textsc{L-form}] We are given
\begin{align*}
\upstairs{\Gamma} \yields_{\El{p}{\modeof{\Gamma}}} \upstairs{A} \TYPE
\end{align*}
So we can form
\begin{mathpar}
\upstairs{\Gamma, \lock} \yields_{\El{p}{\TrPlus{f}{\modeof{\Gamma}}}} \upstairs{\Ltype{A}} :\equiv \F{f_1}{\upstairs{A}} \TYPE
\end{mathpar}

\item[\textsc{L-intro}] Our goal is a term $\upstairs{\Gamma}, x : \upstairs{A} \yields_{\One_{\modeofq{(\Gamma, A, \lock)}}} \upstairs{\LI{A}} : \upstairs{\Ltype{A}[\proj{\Gamma, A}, \lock]}$. The substitution can be simplified:
\begin{align*}
&\upstairs{\proj{\Gamma, A}, \lock} \\
&\equiv \StI{f}{\upstairs{\proj{\Gamma, A}}} \\
&\equiv \StI{f}{\rewrite{\pi^{\modeof{\Gamma}}_x}{\upstairs{\id_\Gamma}}} \\ 
&\equiv \rewrite{\TrPlus{f}{\pi^{\modeof{\Gamma}}_x}}{\StI{f}{\upstairs{\id_\Gamma}}}
\end{align*}
So the type $\upstairs{\Ltype{A}[\proj{\Gamma, A}, \lock]}$ is translated to:
\begin{align*}
&\upstairs{\Ltype{A}[\proj{\Gamma, A}, \lock]} \\
&\equiv \StE{\tshape{\Delta, \lock}}{\upstairs{\proj{\Gamma, A}, \lock}}{\sigma}{\upstairs{\Ltype{A}}[\unpack{\Delta, \lock}{\sigma}]} \\
&\equiv \StE{\tshape{\Delta, \lock}}{\rewrite{\TrPlus{f}{\pi^{\modeof{\Gamma}}_x}}{\StI{f}{\upstairs{\id_\Gamma}}}}{\sigma}{\upstairs{\Ltype{A}}[\unpack{\Delta, \lock}{\sigma}]} \\
&\equiv \St{\El{q}{\TrPlus{f}{\pi^{\modeof{\Gamma}}_x}}}{\StE{\tshape{\Delta, \lock}}{\StI{f}{\upstairs{\id_\Gamma}}}{\sigma}{\upstairs{\Ltype{A}}[\unpack{\Delta, \lock}{\sigma}]}} \\
&\equiv \St{\El{q}{\TrPlus{f}{\pi^{\modeof{\Gamma}}_x}}}{\StE{\tshape{\Delta}}{\upstairs{\id_\Gamma}}{\sigma}{\upstairs{\Ltype{A}}[\unpack{\Delta, \lock}{\sigma}]}} \\
&\equiv \St{\El{q}{\TrPlus{f}{\pi^{\modeof{\Gamma}}_x}}}{\upstairs{\Ltype{A}}}
\end{align*}
So define $\LI{A}$ to be:
\begin{mathpar}
\inferrule*[Left=rewrite]{
\inferrule*[Left=s-intro]{
\inferrule*[Left=F-intro]{
\upstairs{\Gamma}, x : \upstairs{A} \yields_x x : \upstairs{A}}
{\upstairs{\Gamma}, x : \upstairs{A} \yields_{f_1(x)} \FI{x} : \upstairs{\Ltype{A}}}}
{\upstairs{\Gamma}, x : \upstairs{A} \yields_{\TrPlus{\ApEl{q}{\TrPlus{f}{\pi^{\modeof{\Gamma}}_x}}}{f_1(x)}} \StI{\ApEl{q}{\TrPlus{f}{\pi^{\modeof{\Gamma}}_x}}}{\FI{x}} : \St{\ApEl{q}{\TrPlus{f}{\pi^{\modeof{\Gamma}}_x}}}{\upstairs{\Ltype{A}}}}}
{\upstairs{\Gamma}, x : \upstairs{A} \yields_{\One_{\TrPlus{f}{\modeof{\Gamma}.x}}} \rewrite{\fibf{x}}{\StI{\ApEl{q}{\TrPlus{f}{\pi^{\modeof{\Gamma}}_x}}}{\FI{x}}} : \St{\ApEl{q}{\TrPlus{f}{\pi^{\modeof{\Gamma}}_x}}}{\upstairs{\Ltype{A}}}}
\end{mathpar}

\item[\textsc{L-elim}] We are given a term $\upstairs{\Gamma}, x : \upstairs{A} \yields_{\One_{\TrPlus{f}{\modeof{\Gamma}.x}}} \upstairs{c} : \upstairs{C[\proj{\Gamma, A}, \lock, \LI{A}]}$. 

The substitution $\proj{\Gamma, A}, \lock, \LI{A}$ is translated to 
\begin{align*}
&\upstairs{\proj{\Gamma, A}, \lock, \LI{A}} \\
\equiv{} &\rewrite{\eta^\tdot_{\modeof{\Gamma}}}{\StI{\chi}{\upstairs{\proj{\Gamma, A}, \lock}, \upstairs{\LI{A}}}} \\
\equiv{} &\rewrite{\eta^\tdot_{\modeof{\Gamma}}}{\StI{\chi}{\rewrite{\TrPlus{f}{\pi^{\modeof{\Gamma}}_x}}{\StI{f}{\upstairs{\id_\Gamma}}}, \rewrite{\fibf{x}}{\StI{\ApEl{q}{\TrPlus{f}{\pi^{\modeof{\Gamma}}_x}}}{\FI{x}}}}} \\
\equiv{} &\rewrite{\eta^\tdot_{\modeof{\Gamma}};(\TrPlus{f}{\pi^{\modeof{\Gamma}}_x} \bdot \fibf{x})}{\StI{\chi}{\StI{f}{\upstairs{\id_\Gamma}}, \FI{x}}} \\
\equiv{} &\rewrite{\fdist{x}}{\StI{\chi}{\StI{f}{\upstairs{\id_\Gamma}}, \FI{x}}} 
\end{align*}
So $\upstairs{C[\proj{\Gamma, A}, \lock, \LI{A}]}$ becomes:
\begin{align*}
&\upstairs{C[\proj{\Gamma, A}, \lock, \LI{A}]} \\
&\equiv \StE{\tshape{\Gamma, \lock, \Ltype{A}}}{\upstairs{\proj{\Gamma, A}, \lock, \LI{A}}}{\sigma}{\upstairs{C}[\unpack{\Gamma, \lock, \Ltype{A}}{\sigma}]} \\
&\equiv \StE{\tshape{\Gamma, \lock, \Ltype{A}}}{\rewrite{\fdist{x}}{\StI{\chi}{\StI{f}{\upstairs{\id_\Gamma}}, \FI{x}}}}{\sigma}{\upstairs{C}[\unpack{\Gamma, \lock, \Ltype{A}}{\sigma}]} \\
&\equiv \St{\ApEl{q}{\fdist{x}}}{\StE{\tshape{\Gamma, \lock, \Ltype{A}}}{\StI{\chi}{\StI{f}{\upstairs{\id_\Gamma}}, \FI{x}}}{\sigma}{\upstairs{C}[\unpack{\Gamma, \lock, \Ltype{A}}{\sigma}]}} \\
&\equiv \St{\ApEl{q}{\fdist{x}}}{\StE{\tshape{\Gamma, \lock, \Ltype{A}}}{\StI{\chi}{\StI{f}{\StI{\tshape{\Gamma}}{\pack{\Gamma}}, \FI{x}}}}{\sigma}{\upstairs{C}[\unpack{\Gamma, \lock, \Ltype{A}}{\sigma}]}} \\
&\equiv \St{\ApEl{q}{\fdist{x}}}{\upstairs{C}[\unpack{\Gamma, \lock, \Ltype{A}}{\sigma}][\pack{\Gamma}/\sigma, \FI{x}/z]} \\
&\equiv \St{\ApEl{q}{\fdist{x}}}{\upstairs{C}[\FI{x}/z]}
\end{align*}
Define the translation of $\LE{c}$ to be:
\begin{mathpar}
\inferrule*[Left=F-elim]{
\inferrule*[Left=rewrite]{
\inferrule*[Left=s-intro]{
\upstairs{\Gamma}, x : \upstairs{A} \yields_{\One_{\TrPlus{f}{\modeof{\Gamma}.x}}} \upstairs{c} : \St{\ApEl{q}{\fdist{x}}}{\upstairs{C}[\FI{x}/z]}}
{\upstairs{\Gamma}, x : \upstairs{A} \yields_{\TrPlus{\ApEl{q}{\fdistinv{x}}}{\One_{\TrPlus{f}{\modeof{\Gamma}.x}}}} \StI{\ApEl{q}{\fdistinv{x}}}{\upstairs{c}} : \upstairs{C}[\FI{x}/z]}}
{\upstairs{\Gamma}, x : \upstairs{A} \yields_{\One_{\TrPlus{f}{\modeof{\Gamma}}.f_1(x)}} \rewrite{\ApOne{\fdistinv{x}}}{\StI{\ApEl{q}{\fdistinv{x}}}{\upstairs{c}}} : \upstairs{C}[\FI{x}/z]}}
{\upstairs{\Gamma}, z : \upstairs{\Ltype{A}} \yields_{\One_{\TrPlus{f}{\modeof{\Gamma}}.z}} \FE{z}{x}{\rewrite{\ApOne{\fdistinv{x}}}{\StI{\ApEl{q}{\fdistinv{x}}}{\upstairs{c}}}} : \upstairs{C}}
\end{mathpar}
And this is of the right mode, as $\TrPlus{f}{\modeof{\Gamma}}.z \equiv \modeofq{(\Gamma, \lock, \Ltype{A})}$.
\end{enumerate}

And the equations: 

\begin{enumerate}[style = multiline, labelwidth = 80pt]
\item[{$\Ltype{A}[\Theta, \lock] \equiv \Ltype{(A[\Theta])}$}:]
\begin{align*}
&\upstairs{\Ltype{A}[\Theta, \lock]} \\
&\equiv \StE{\tshape{\Delta, \lock}}{\upstairs{\Theta, \lock}}{\sigma}{\upstairs{\Ltype{A}}[\unpack{\Delta, \lock}{\sigma}]}
&\equiv \StE{f;\tshape{\Delta}}{\StI{f}{\upstairs{\Theta}}}{\sigma}{ \F{f_1}{\upstairs{A}}[\unpack{\Delta, \lock}{\sigma}]}\\
&\equiv \StE{\tshape{\Delta}}{\upstairs{\Theta}}{\sigma}{ \F{f_1}{\upstairs{A}}[\unpack{\Delta, \lock}{\sigma}]}\\
&\equiv \StE{\tshape{\Delta}}{\upstairs{\Theta}}{\sigma}{
\F{f_1}{\upstairs{A}[\unpack{\Delta}{\sigma}]}}\\
&\equiv \F{f_1}{\StE{\tshape{\Delta}}{\upstairs{\Theta}}{\sigma}{\upstairs{A}[\unpack{\Delta}{\sigma}]}}\\
&\equiv \F{f_1}{\upstairs{A[\Theta]}} \\
&\equiv \upstairs{\Ltype{(A[\Theta])}}
\end{align*}

\item[{$\LI{A}[\Theta \uparrow A, \lock] \equiv \LI{A[\Theta]}$}:]
\begin{align*}
&\upstairs{\LI{A}[\Theta \uparrow A, \lock]} \\
&\equiv \StE{\tshape{\Delta, A, \lock}}{\upstairs{\Theta \uparrow A, \lock}}{\sigma}{\upstairs{\LI{A}}[\unpack{\Delta, A, \lock}{\sigma}]} \\
&\equiv \StE{f;\tshape{\Delta, A}}{\StI{f}{\upstairs{\Theta \uparrow A}}}{\sigma}{\upstairs{\LI{A}}[\unpack{\Delta, A, \lock}{\sigma}]} \\
&\equiv \StE{(\tshape{\Delta}, \id)}{(\upstairs{\Theta}, x)}{\sigma, x}{\upstairs{\LI{A}}[\unpack{\Delta}{\sigma}, x/x]} \\
&\equiv \StE{(\tshape{\Delta}, \id)}{(\upstairs{\Theta}, x)}{\sigma, x}{\rewrite{\fibf{x}}{\StI{\ApEl{q}{\TrPlus{f}{\pi^{\modeof{\Gamma}}_x}}}{\FI{x}}}[\unpack{\Delta}{\sigma}, x/x]} \\
&\equiv \StE{(\tshape{\Delta}, \id)}{(\upstairs{\Theta}, x)}{\sigma, x}{\rewrite{\fibf{x}}{\StI{\ApEl{q}{\TrPlus{f}{\pi^{\TrPlus{\tshape{\Gamma}}{\sigma}}_x}}}{\FI{x}}}} \\
&\equiv \rewrite{\fibf{x}}{\StI{\ApEl{q}{\TrPlus{f}{\pi^{\modeof{\Gamma}}_x}}}{\FI{x}}} \\
&\equiv \upstairs{\LI{A[\Theta]}}
\end{align*}

\item[{$\LE{c}[\Theta, \lock \uparrow \Ltype{A}] \equiv \LE{c[\Theta \uparrow A, \lock]}$}:]
\begin{align*}
&\upstairs{\LE{c}[\Theta, \lock \uparrow \Ltype{A}]} \\
&\equiv \StE{\tshape{\Delta, \lock, \Ltype{A}}}{\upstairs{\Theta, \lock \uparrow \Ltype{A}}}{\sigma}{\upstairs{\LE{c}}[\unpack{\Delta, \lock, \Ltype{A}}{\sigma}]} \\
&\equiv \StE{(\tshape{\Delta, \lock}, \id)}{(\upstairs{\Theta, \lock}, z)}{\sigma, z}{\upstairs{\LE{c}}[\unpack{\Delta, \lock}{\sigma}, z/z]} \\
&\equiv \StE{((f;\tshape{\Delta}), \id)}{(\StI{f}{\upstairs{\Theta}}, z)}{\sigma, z}{\upstairs{\LE{c}}[\unpack{\Delta, \lock}{\sigma}, z/z]} \\
&\equiv \StE{(\tshape{\Delta}, \id)}{(\upstairs{\Theta}, z)}{\sigma, z}{\upstairs{\LE{c}}[\unpack{\Delta}{\sigma}, z/z]} \\
&\equiv \StE{(\tshape{\Delta}, \id)}{(\upstairs{\Theta}, z)}{\sigma, z}{\FE{z}{x}{\rewrite{\ApOne{\fdistinv{x}}}{\StI{\ApEl{q}{\fdistinv{x}}}{\upstairs{c}}}}[\unpack{\Delta}{\sigma}, z/z]} \\
&\equiv \StE{(\tshape{\Delta}, \id)}{(\upstairs{\Theta}, z)}{\sigma, z}{\FE{z}{x}{\rewrite{\ApOne{\fdistinv{x}}}{\StI{\ApEl{q}{\fdistinv{x}}}{\upstairs{c}[\unpack{\Delta}{\sigma}, x/x]}}}} \\
&\equiv \StE{(\tshape{\Delta}, \id)}{(\upstairs{\Theta}, z)}{\sigma, z}{\FE{z}{x}{\rewrite{\ApOne{\fdistinv{x}}}{\StI{\ApEl{q}{\fdistinv{x}}}{\\
&\qquad \StE{(\tshape{\Delta}, x)}{(\StI{\tshape{\Delta}}{\sigma}, x)}{\sigma, x}{\upstairs{c}[\unpack{\Delta}{\sigma}, x/x]}}}}} \\
&\equiv \FE{z}{x}{\rewrite{\ApOne{\fdistinv{x}}}{\StI{\ApEl{q}{\fdistinv{x}}}{\StE{(\tshape{\Delta}, x)}{(\upstairs{\Theta}, x)}{\sigma, x}{\upstairs{c}[\unpack{\Delta}{\sigma}, x/x]}}}} \\
&\equiv \FE{z}{x}{\rewrite{\ApOne{\fdistinv{x}}}{\StI{\ApEl{q}{\fdistinv{x}}}{\StE{\tshape{\Delta,A}}{\upstairs{\Theta \uparrow A}}{\sigma}{\upstairs{c}[\unpack{\Delta, A}{\sigma}]}}}} \\
&\equiv \FE{z}{x}{\rewrite{\ApOne{\fdistinv{x}}}{\StI{\ApEl{q}{\fdistinv{x}}}{\StE{\tshape{\Delta,A,\lock}}{\upstairs{\Theta \uparrow A, \lock}}{\sigma}{\upstairs{c}[\unpack{\Delta, A}{\sigma}]}}}} \\
&\equiv \FE{z}{x}{\rewrite{\ApOne{\fdistinv{x}}}{\StI{\ApEl{q}{\fdistinv{x}}}{\upstairs{c[\Theta \uparrow A, \lock]}}}} \\
&\equiv \upstairs{\LE{c[\Theta \uparrow A, \lock]}}
\end{align*}

\item[{$\LE{c}[\proj{\Gamma, A}, \lock, \LI{A}] \equiv c$}:]
\begin{align*}
&\upstairs{\LE{c}[\proj{\Gamma, A}, \lock, \LI{A}]} \\
&\equiv \StE{\tshape{\Gamma, \lock, \Ltype{A}}}{\upstairs{\proj{\Gamma, A}, \lock, \LI{A}}}{\sigma}{\upstairs{\LE{c}}[\unpack{\Gamma, \lock, \Ltype{A}}{\sigma}]} \\
&\equiv \StE{\tshape{\Gamma, \lock, \Ltype{A}}}{\rewrite{\fdist{x}}{\StI{\chi}{\StI{f}{\upstairs{\id_\Gamma}}, \FI{x}}} }{\sigma}{\upstairs{\LE{c}}[\unpack{\Gamma, \lock, \Ltype{A}}{\sigma}]} \\
&\equiv \rewrite{\ApOne{\fdist{x}}}{\StI{\ApEl{p}{\fdist{x}}}{\StE{\tshape{\Gamma, \lock, \Ltype{A}}}{\StI{\chi}{\StI{f}{\upstairs{\id_\Gamma}}, \FI{x}} }{\sigma}{\upstairs{\LE{c}}[\unpack{\Gamma, \lock, \Ltype{A}}{\sigma}]}}} \\
&\equiv \rewrite{\ApOne{\fdist{x}}}{\StI{\ApEl{p}{\fdist{x}}}{\StE{\tshape{\Gamma, \lock, \Ltype{A}}}{\StI{\chi}{\StI{f}{\StI{\tshape{\Gamma}}{\pack{\Gamma}}}, \FI{x}} }{\sigma}{\upstairs{\LE{c}}[\unpack{\Gamma, \lock, \Ltype{A}}{\sigma}]}}} \\
&\equiv \rewrite{\ApOne{\fdist{x}}}{\StI{\ApEl{p}{\fdist{x}}}{\upstairs{\LE{c}}[\unpack{\Gamma, \lock, \Ltype{A}}{\sigma}][(\pack{\Gamma}, \FI{x})/\sigma]}} \\
&\equiv \rewrite{\ApOne{\fdist{x}}}{\StI{\ApEl{p}{\fdist{x}}}{\upstairs{\LE{c}}[\FI{x}/z]}} \\
&\equiv \rewrite{\ApOne{\fdist{x}}}{\StI{\ApEl{p}{\fdist{x}}}{\FE{z}{x}{\rewrite{\ApOne{\fdistinv{x}}}{\StI{\ApEl{q}{\fdistinv{x}}}{\upstairs{c}}}}[\FI{x}/z]}} \\
&\equiv \rewrite{\ApOne{\fdist{x}}}{\StI{\ApEl{p}{\fdist{x}}}{\FE{\FI{x}}{x}{\rewrite{\ApOne{\fdistinv{x}}}{\StI{\ApEl{q}{\fdistinv{x}}}{\upstairs{c}}}}}} \\
&\equiv \rewrite{\ApOne{\fdist{x}}}{\StI{\ApEl{p}{\fdist{x}}}{\rewrite{\ApOne{\fdistinv{x}}}{\StI{\ApEl{q}{\fdistinv{x}}}{\upstairs{c}}}}} \\
&\equiv \upstairs{c}
\end{align*}
\end{enumerate}

\subsection{Spatial Type Theory}

\mvrnote{TODO}

\subsection{Dependently Indexed Linear Types}

The structural rules of \mvrnote{DILTT} are given by the the rules for ordinary MLTT and the additional ones in Figure~\ref{fig:qit-linear-structural-rules}. Judgements are translated as follows:

\begin{itemize}
\item A context $\qyields \Gamma \CTX$ is represented as in MLTT, by a framework context $\upstairs{\Gamma}$  together with a mode type morphism $\cdot \yields \tshape{\Gamma} : p \tcell \ctxtuple{\downstairs{\Gamma}}$.
\item A context $\qyields \Gamma \mid \Xi \CTX$ is represented as a framework context $\upstairs{\Gamma}, \upstairs{\Xi}$. DILTT maintains the invariant that all types in the linear zone are well-formed in the intuitionistic context, so the types in $\upstairs{\Xi}$ are all in mode $\El{p}{\modeof{\Gamma}}$. We also require an associated term $\tfibshape{\Xi} : \El{p}{\modeof{\Gamma}}$ that describes the shape of the linear zone.
\item A type $\Gamma \qyields A \TYPE$ is represented as in MLTT, by a framework type $\upstairs{\Gamma} \yields_{\El{p}{\modeof{\Gamma}}} \upstairs{A} \TYPE$`.
\item A term $\Gamma \mid \Xi \qyields a : A$ is represented by a framework term $\upstairs{\Gamma}, \upstairs{\Xi} \yields_{\tfibshape{\Xi}} \upstairs{a} : \upstairs{A}$.
\item A substitution $\Gamma \qyields \Theta : \Delta$ is represented by a term $\upstairs{\Gamma} \yields_{\modeof{\Gamma}} \upstairs{\Theta} : \St{\tshape{\Delta}}{\ctxtuple{\upstairs{\Delta}}}$.
\item A linear substitution $\Gamma \mid \Xi \qyields \Theta : \Xi'$ is represented by a framework substitution $\upstairs{\Gamma}, \upstairs{\Xi} \yields_{\downstairs{\Theta}} \upstairs{\Theta} : \upstairs{\Xi'}$. \mvrnote{This is our only option unless we want to require $\otimes$-types. This is also a little different to other cases, we have to externally argue that every variable in $\upstairs{\Xi}$ is used once to make sure term-lin-sub results in a term with the correct mode.}
\end{itemize}

\begin{figure}
\begin{mathpar}
\inferrule*[left=ctx-lin-empty]{\qyields \Gamma \CTX}{\qyields \Gamma \mid \cdot \CTX} \and
\inferrule*[left=ctx-lin-ext]{\qyields \Gamma \mid \Xi \CTX \and \Gamma \qyields A \TYPE}{\qyields \Gamma \mid \Xi, A \CTX} \\
\inferrule*[left=sub-lin-empty]{~}{\Gamma \mid \cdot \qyields \varepsilon_{\Gamma} : \cdot} \and
\inferrule*[left=sub-lin-ext]{\Gamma \mid \Xi \yields \Theta : \Xi' \and \Gamma \mid \Xi'' \qyields a : A}{\Gamma \mid \Xi, \Xi'' \qyields (\Theta, a) : \Xi', A} \\
\inferrule*[left=sub-lin-id]{~}{\Gamma \mid \Xi \qyields \id_{\Xi} : \Xi} \and
\inferrule*[left=sub-lin-comp]{\Gamma \mid \Xi \yields \Theta : \Xi' \and \Gamma \mid \Xi' \qyields \kappa : \Xi''}{\Gamma \mid \Xi \qyields (\Theta ; \kappa) : \Xi''} \\
\inferrule*[left=term-sub]{\Delta \mid \Xi \qyields a : A  \and \Gamma \qyields \Theta : \Delta}{\Gamma \mid \Xi[\Theta] \qyields a[\Theta] : A[\Theta]} \and
\inferrule*[left=term-lin-sub]{\Gamma \mid \Xi \qyields \Theta :  \Xi' \and \Gamma \mid \Xi' \qyields b : B}{\Gamma \mid \Xi \qyields b\{\Theta\} : B} \\
\inferrule*[left=var]{~}{\Gamma, A \mid \cdot \qyields \qvar{\Gamma,A} : A[\proj{\Gamma,A}]}  \and
\inferrule*[left=lin-var]{~}{\Gamma \mid A \qyields \qlinvar{A} : A} 
\end{mathpar}
\begin{align}
b\{\id_\Xi\} &\equiv b \\
\id_\Xi;\Theta &\equiv \Theta \\
\Theta;\id_{\Xi'} &\equiv \Theta \\
(\Theta;\kappa);\rho &\equiv \Theta ; (\kappa; \rho) \\
\nonumber \\
\id_{\Xi, A} &\equiv (\id_\Xi, \qlinvar{A}) \\
\qlinvar{A}[\Theta] &\equiv \qlinvar{A[\Theta]}
\end{align}
\caption{Structural Rules for Dependently Indexed Linear Types}\label{fig:qit-linear-structural-rules}
\end{figure}

\begin{enumerate}
\item[\textsc{ctx-lin-empty}] $\Gamma \mid \cdot$ is translated to $\upstairs{\Gamma}$, with $\tfibshape{(\cdot)} :\equiv \One_{\modeof{\Gamma}}$.
\item[\textsc{ctx-lin-ext}] $\Gamma \mid \Xi, A$ is translated to $\upstairs{\Gamma}, \upstairs{\Xi}, x : \upstairs{A}$, with $\tfibshape{\Xi, A} :\equiv \tfibshape{\Xi, A} \otimes x$
\item[\textsc{sub-lin-empty}]  The linear substitution $\Gamma \mid \cdot \qyields \varepsilon_{\Gamma} : \cdot$ is translated to the empty framework substitution.
\item[\textsc{sub-lin-ext}] Given translations of $\Gamma \mid \Xi \yields \Theta : \Xi'$ and $\Gamma \mid \Xi'' \qyields a : A$, define $\upstairs{(\Theta, a)}$ to be the extended framework substitution 
\begin{align*}
\upstairs{\Gamma}, \upstairs{\Xi}, \upstairs{\Xi''} \yields_{\downstairs{\Theta}, \tfibshape{\Xi''}} (\upstairs{\Theta}, \upstairs{a}) : \upstairs{\Xi'}, \upstairs{A}
\end{align*}
\item[\textsc{sub-lin-id}] We use the identity framework substitution $\upstairs{\Gamma}, \upstairs{\Xi} \yields_{\id_{\downstairs{\Xi}}} \upstairs{\id_\Xi} :\equiv \id_{\upstairs{\Xi}} : \upstairs{\Xi}$
\item[\textsc{sub-lin-comp}] This is composition of framework substitutions. \mvrnote{And contraction on the $\Gamma$ part}
\item[\textsc{term-sub}] This is translated similarly to substitution in MLTT: \mvrnote{This is a little messy actually... We need to pack the linear variables and include them when we do the let.} In the one variable case, 
\begin{align*}
\upstairs{\Gamma}, x : \upstairs{\Xi[\Theta]} \yields_{\tfibshape{\Xi[\Theta]}} \upstairs{a[\Theta]} :\equiv \StE{(\tshape{\Delta}, \id)}{(\upstairs{\Theta}, x)}{\sigma, x}{\upstairs{a}[\unpack{\Delta}{\sigma}, x/x]} : \upstairs{A[\Theta]}
\end{align*}
\item[\textsc{term-lin-sub}] 
\item[\textsc{var}] This is translated the same as $\qvar{A}$ in MLTT. This is a term in mode $\One_{\modeof{\Gamma, A}}$, i.e., with empty linear context.
\item[\textsc{lin-var}] This is exactly the framework variable rule: $\upstairs{\Gamma}, x : \upstairs{A} \yields_x x : \upstairs{A}$.
\end{enumerate}

Note that terms with empty linear zone always have mode $\One_{\modeof{\Gamma}}$. The translations of the rules for ordinary $\Sigma$-, $\Pi$-, etc.\ types therefore remain valid, so long as they only operate on contexts with empty linear zone. For example:
\begin{mathpar}
\inferrule*[left=$\Sigma$-split]{\Gamma, A, B \mid \cdot \qyields c : C[(\proj{\Gamma, A, B};\proj{\Gamma, A}), \qpair{A,B}]}{\Gamma, \Sigma_A B \mid \cdot \qyields \qsplit{A, B}(c) : C}
\end{mathpar}

\subsubsection{Types in the Fiber}

The linear zone of DILTT can be made to support the ordinary type formers of linear logic. In Figure~\ref{fig:qit-linear-fiber-rules}, the rules for $I$- and $\otimes$-types are given in algebraic style.

\begin{figure}
\begin{mathpar}
\inferrule*[Left=$I$-form]{~}{\Gamma \qyields I_\Gamma \TYPE} \\
\inferrule*[Left=$I$-intro]{~}{\Gamma \mid \cdot \qyields \star_\Gamma : I_\Gamma } \and
\inferrule*[Left=$I$-elim]{\Gamma \mid \Xi \qyields c : C}{\Gamma \mid \Xi, I \qyields \qunitmatch{c} : C} \\
%%%
\inferrule*[Left=$\otimes$-form]{\Gamma \qyields A \TYPE \and \Gamma \qyields B \TYPE}{\Gamma \qyields A \otimes B \TYPE} \\
\inferrule*[Left=$\otimes$-intro]{~}{\Gamma \mid A, B \qyields \otimespair{A,B} : A \otimes B} \and
\inferrule*[Left=$\otimes$-elim]{\Gamma \mid \Xi, A, B \qyields c : C}{\Gamma \mid \Xi, A \otimes B \qyields \otimessplit{c} : C} \\
%%%
\end{mathpar}
\mvrnote{TODO: the other linear type formers? $\multimap, \top, \&, \oplus, 0$}
\caption{Rules for \mvrnote{Types in the Fiber}}\label{fig:qit-linear-fiber-rules}
\end{figure}

For $I$- and $\otimes$-types, the rules are immediate from the framework rules for $\mathsf{F}$-types; no unpacking or rewriting is necessary. \mvrnote{We are using strict associativity/unit here...} Because \textsc{$I$-elim} is non-dependent, it is not necessary for the comprehension object to support the unit type (in the sense of Definition~\ref{def:supports-unit}) in order to interpret the $I$ type.
\begin{align*}
\upstairs{I_\Gamma} :&\equiv \F{\One_{\modeof{\Gamma}}}{1} \\
\upstairs{\star_\Gamma} :&\equiv \FIs{\One_{\modeof{\Gamma}}}{} \\
\upstairs{\qunitmatch{c}} :&\equiv \FEs{\One_{\modeof{\Gamma}}}{z}{}{\upstairs{c}} \\
\upstairs{A \otimes B} :&\equiv \F{w. (\fst w) \otimes_{\modeof{\Gamma}} (\snd w)}{\upstairs{A}, \upstairs{B}} \\
\upstairs{\otimespair{A,B}} :&\equiv \FIs{w. (\fst w) \otimes_{\modeof{\Gamma}} (\snd w)}{\upstairs{a}, \upstairs{b}} \\
\upstairs{\otimessplit{c}} :&\equiv \FEs{w. (\fst w) \otimes_{\modeof{\Gamma}} (\snd w)}{z}{x, y}{\upstairs{c}}
\end{align*}

\subsubsection{Dependent Types}

The rules for $\Sigma!$ and $\Pi!$ in algebraic style are given in Figure~\ref{fig:qit-linear-dependent-rules}. Rules such as $\Sigma!$-elim and $\Pi!$-intro are presented with only a single linear type $\Xi$ in the linear context. In the presence of $I$- and $\otimes$-types, we may pack a more general linear context into a single iterated $\otimes$-type, apply the rule, then unpack the $\otimes$-type.

\begin{figure}
\begin{mathpar}
%\inferrule*[Left=$!$-form]{\Gamma \qyields A \TYPE}{\Gamma \qyields \bang A \TYPE} \\
%\inferrule*[Left=$!$-intro]{~}{\Gamma, A \mid \cdot \qyields \qbang{A} : \bang A[\proj{\Gamma, A}]} \and
%\inferrule*[Left=$!$-elim]{\Gamma, A \mid \Xi[\proj{\Gamma, A}] \qyields c : C[\proj{\Gamma, A}]}{\Gamma \mid \Xi, \bang A \qyields \letbang{c} : C} \\
%%%
\inferrule*[Left=$\Sigma!$-form]{\Gamma \qyields A \TYPE \and \Gamma, A \qyields B \TYPE}{\Gamma \qyields \Sigma!_A B \TYPE } \\
\inferrule*[Left=$\Sigma!$-intro]{~}{\Gamma, A \mid B \qyields \linpair{A,B} : \Sigma!_A B [\proj{\Gamma, A}]} \and
\inferrule*[Left=$\Sigma!$-elim]{\Gamma, A \mid \Xi[\proj{\Gamma, A}], B \qyields c : C[\proj{\Gamma, A}]}{\Gamma \mid \Xi, \Sigma!_A B \qyields \linsplit{c} : C} \\
%%%
\inferrule*[Left=$\Pi!$-form]{\Gamma \qyields A \TYPE \and \Gamma, A \qyields B \TYPE}{\Gamma \qyields \Pi!_A B \TYPE } \\
\inferrule*[left=$\Pi!$-intro]{\Gamma, A \mid \Xi[\proj{\Gamma, A}]\qyields b : B}{\Gamma \mid \Xi \qyields \linlam(b) : \Pi!_A B} \and
\inferrule*[left=$\Pi!$-elim]{~}{\Gamma, A \mid (\Pi_A B)[\proj{\Gamma, A}] \qyields \linapp{A, B} : B}
\end{mathpar}
\mvrnote{TODO: equations}
\caption{Rules for Dependently Indexed Linear Types}\label{fig:qit-linear-dependent-rules}
\end{figure}

%\begin{enumerate}
%\item[\textsc{$!$-form}]
%\item[\textsc{$!$-intro}] We want:
%\begin{align*}
%\upstairs{\Gamma}, x : \upstairs{A} \yields_{\One_{\modeof{\Gamma}.x}} \upstairs{\qbang{A}} : \St{\ApEl{p}{\pi^{\modeof{\Gamma}}_x}}{\upstairs{\bang A}}
%\end{align*}
%
%\item[\textsc{$!$-elim}] The translation of the provided term has type:
%\begin{align*}
%\upstairs{\Gamma}, x : \upstairs{A}, \xi : \St{\ApEl{p}{\pi^{\modeof{\Gamma}}_x}}{\upstairs{\Xi}} \yields_\xi \upstairs{c} : \St{\ApEl{p}{\pi^{\modeof{\Gamma}}_x}}{\upstairs{C}}
%\end{align*}
%\begin{mathpar}
%\inferrule*[Left=F-elim]{
%\inferrule*[Left=rewrite]{
%\inferrule*[Left=s-elim]{
%\inferrule*[Left=cut]{
%\upstairs{\Gamma}, x : \upstairs{A}, \xi : \St{\ApEl{p}{\pi^{\modeof{\Gamma}}_x}}{\upstairs{\Xi}} \yields_\xi \upstairs{c} : \St{\ApEl{p}{\pi^{\modeof{\Gamma}}_x}}{\upstairs{C}}}
%{\upstairs{\Gamma}, x : \upstairs{A}, \xi : \upstairs{\Xi} \yields_{\TrPlus{\ApEl{p}{\pi^{\modeof{\Gamma}}_x}}{\xi}} \upstairs{c}[\StI{\ApEl{p}{\pi^{\modeof{\Gamma}}_x}}{\xi}/\xi] : \St{\ApEl{p}{\pi^{\modeof{\Gamma}}_x}}{\upstairs{C}}
%}\and
%\upstairs{\Gamma}, x : \upstairs{A}, z : \upstairs{C} \yields_{\Sigma!(x, \TrPlus{\ApEl{p}{\pi^{\modeof{\Gamma}}_x}}{z})} \rewrite{\linwk{x,z}}{z} : \upstairs{C}}
%{\upstairs{\Gamma}, x : \upstairs{A}, \xi : \upstairs{\Xi}, y : \upstairs{B} \yields_{\Sigma!(x, \TrPlus{\ApEl{p}{\pi^{\modeof{\Gamma}}_x}}{\xi} \otimes y)} \StE{\ApEl{p}{\pi^{\modeof{\Gamma}}_x}}{\upstairs{c}[\StI{\ApEl{p}{\pi^{\modeof{\Gamma}}_x}}{\xi}/\xi]}{z}{\rewrite{\linwk{x,z}}{z}} : \upstairs{C} }}
%{\upstairs{\Gamma}, x : \upstairs{A}, \xi : \upstairs{\Xi}, y : \upstairs{B} \yields_{\xi \otimes \Sigma!(x, y)} \rewrite{\frob{\xi,x,y}}{\StE{\ApEl{p}{\pi^{\modeof{\Gamma}}_x}}{\upstairs{c}[\StI{\ApEl{p}{\pi^{\modeof{\Gamma}}_x}}{\xi}/\xi]}{z}{\rewrite{\linwk{x,z}}{z}}} : \upstairs{C}}}
%{\upstairs{\Gamma}, \xi : \upstairs{\Xi}, z : \upstairs{\Sigma!_A B} \yields_{\xi \otimes z} \FE{z}{x,y}{\rewrite{\frob{\xi,x,y}}{\StE{\ApEl{p}{\pi^{\modeof{\Gamma}}_x}}{\upstairs{c}[\StI{\ApEl{p}{\pi^{\modeof{\Gamma}}_x}}{\xi}/\xi]}{z}{\rewrite{\linwk{x,z}}{z}}}} : \upstairs{C}}
%\end{mathpar}
%\end{enumerate}

\begin{enumerate}
\item[\textsc{$\Sigma!$-form}] The same as for $\Sigma$-types:
\begin{align*}
\upstairs{\Sigma!_A B} :\equiv \F{w. \Sigma!(\fst w, \snd w)}{\upstairs{A}, \upstairs{B}}
\end{align*}

\item[\textsc{$\Sigma!$-intro}] We want a term
\begin{align*}
\upstairs{\Gamma}, x : \upstairs{A}, y : \upstairs{B} \yields_y \upstairs{\linpair{A,B}} : \TrPlus{\ApEl{p}{\pi^{\modeof{\Gamma}}_x}}{\upstairs{\Sigma!_A B}}
\end{align*}
And we can produce such a term by:
\begin{mathpar}
\inferrule*[Left=rewrite]{
\inferrule*[Left=s-intro]{
\inferrule*[Left=F-intro]{~}
{\upstairs{\Gamma}, x : \upstairs{A}, y : \upstairs{B} \yields_{\Sigma!(x,y)} \FI{x,y} : \upstairs{\Sigma!_A B}}}
{\upstairs{\Gamma}, x : \upstairs{A}, y : \upstairs{B} \yields_{\TrPlus{\ApEl{p}{\pi^{\modeof{\Gamma}}_x}}{\Sigma!(x,y)}} \StI{\ApEl{p}{\pi^{\modeof{\Gamma}}_x}}{\FI{x,y}} : \TrPlus{\ApEl{p}{\pi^{\modeof{\Gamma}}_x}}{\upstairs{\Sigma!_A B}}}}
{\upstairs{\Gamma}, x : \upstairs{A}, y : \upstairs{B} \yields_y \rewrite{\linsnd{x,y}}{\StI{\ApEl{p}{\pi^{\modeof{\Gamma}}_x}}{\FI{x,y}}} : \TrPlus{\ApEl{p}{\pi^{\modeof{\Gamma}}_x}}{\upstairs{\Sigma!_A B}}}
\end{mathpar}

\item[\textsc{$\Sigma!$-elim}]
The translation of the provided term has type:
\begin{align*}
\upstairs{\Gamma}, x : \upstairs{A}, \xi : \St{\ApEl{p}{\pi^{\modeof{\Gamma}}_x}}{\upstairs{\Xi}}, y : \upstairs{B} \yields_{\xi \otimes y} \upstairs{c} : \St{\ApEl{p}{\pi^{\modeof{\Gamma}}_x}}{\upstairs{C}}
\end{align*}
So use:
\begin{mathpar}
\inferrule*[Left=F-elim]{
\inferrule*[Left=rewrite]{
\inferrule*[Left=s-elim]{
\inferrule*[Left=cut]{
\upstairs{\Gamma}, x : \upstairs{A}, \xi : \St{\ApEl{p}{\pi^{\modeof{\Gamma}}_x}}{\upstairs{\Xi}}, y : \upstairs{B} \yields_{\xi \otimes y} \upstairs{c} : \St{\ApEl{p}{\pi^{\modeof{\Gamma}}_x}}{\upstairs{C}}}
{\upstairs{\Gamma}, x : \upstairs{A}, \xi : \upstairs{\Xi}, y : \upstairs{B} \yields_{\TrPlus{\ApEl{p}{\pi^{\modeof{\Gamma}}_x}}{\xi} \otimes y} \upstairs{c}[\StI{\ApEl{p}{\pi^{\modeof{\Gamma}}_x}}{\xi}/\xi] : \St{\ApEl{p}{\pi^{\modeof{\Gamma}}_x}}{\upstairs{C}}
}\and
\upstairs{\Gamma}, x : \upstairs{A}, z : \upstairs{C} \yields_{\Sigma!(x, \TrPlus{\ApEl{p}{\pi^{\modeof{\Gamma}}_x}}{z})} \rewrite{\linwk{x,z}}{z} : \upstairs{C}}
{\upstairs{\Gamma}, x : \upstairs{A}, \xi : \upstairs{\Xi}, y : \upstairs{B} \yields_{\Sigma!(x, \TrPlus{\ApEl{p}{\pi^{\modeof{\Gamma}}_x}}{\xi} \otimes y)} \StE{\ApEl{p}{\pi^{\modeof{\Gamma}}_x}}{\upstairs{c}[\StI{\ApEl{p}{\pi^{\modeof{\Gamma}}_x}}{\xi}/\xi]}{z}{\rewrite{\linwk{x,z}}{z}} : \upstairs{C} }}
{\upstairs{\Gamma}, x : \upstairs{A}, \xi : \upstairs{\Xi}, y : \upstairs{B} \yields_{\xi \otimes \Sigma!(x, y)} \rewrite{\frob{\xi,x,y}}{\StE{\ApEl{p}{\pi^{\modeof{\Gamma}}_x}}{\upstairs{c}[\StI{\ApEl{p}{\pi^{\modeof{\Gamma}}_x}}{\xi}/\xi]}{z}{\rewrite{\linwk{x,z}}{z}}} : \upstairs{C}}}
{\upstairs{\Gamma}, \xi : \upstairs{\Xi}, z : \upstairs{\Sigma!_A B} \yields_{\xi \otimes z} \FE{z}{x,y}{\rewrite{\frob{\xi,x,y}}{\StE{\ApEl{p}{\pi^{\modeof{\Gamma}}_x}}{\upstairs{c}[\StI{\ApEl{p}{\pi^{\modeof{\Gamma}}_x}}{\xi}/\xi]}{z}{\rewrite{\linwk{x,z}}{z}}}} : \upstairs{C}}
\end{mathpar}
\mvrnote{This is simpler with a non-frobenius eliminator, worth writing that out too?}
\end{enumerate}
%
%\begin{enumerate}
%\item[\textsc{$\Pi!$-form}] Identical to $\Pi$-formation.
%
%\item[\textsc{$\Pi!$-intro}]
%\begin{mathpar}
%\inferrule*[Left=U-intro]{
%\inferrule*[Left=cut]{\beta : \upstairs{\Gamma, A}, \xi : \upstairs{\Xi[\proj{\Gamma, A}]} \yields_\xi \upstairs{b} : \upstairs{B}}
%{\alpha : \upstairs{\Gamma}, x : \upstairs{A}, \xi : \upstairs{\Xi} \yields_{\TrPlus{\ApEl{p}{\pi^\alpha_x}}{\xi}} \upstairs{b}[\StI{\chi}{\alpha,x}/\beta, \StI{\ApEl{p}{\pi^\alpha_x}}{\xi}/\xi] : \upstairs{B}[\StI{\chi}{\alpha,x}/\beta]}}
%{\alpha : \upstairs{\Gamma}, \xi : \upstairs{\Xi} \yields_\xi \UI{x}{\upstairs{b}[\StI{\chi}{\alpha,x}, \StI{\ApEl{p}{\pi^\alpha_x}}{\xi}/\xi]} : \upstairs{\Pi!_A B}}
%\end{mathpar}
%
%\item[\textsc{$\Pi!$-elim}]
%\begin{mathpar}
%\inferrule*[Left=s-elim]{
%\inferrule*[Left=s-elim]{
%\inferrule*[Left=U-elim]{~}
%{\alpha : \upstairs{\Gamma}, x : \upstairs{A}, f : \upstairs{\Pi!_A B} \yields_{\TrPlus{\ApEl{p}{\pi^\alpha_x}}{f}} \UE{f}{x} : \upstairs{B}[\StI{\chi}{\alpha, x} / \beta]}}
%{\alpha : \upstairs{\Gamma}, x : \upstairs{A}, g : \St{\ApEl{p}{\pi^\alpha_x}}{\upstairs{\Pi!_A B}} \yields_g \StE{\ApEl{p}{\pi^\alpha_x}}{g}{f}{\UE{f}{x}} : \upstairs{B}[\StI{\chi}{\alpha, x} / \beta]}}
%{\beta : \upstairs{\Gamma, A}, g : \upstairs{\Pi!_A B[\proj{\Gamma, A}]} \yields_g \StE{\chi}{\beta}{\alpha,x}{\StE{\ApEl{p}{\pi^\alpha_x}}{g}{f}{\UE{f}{x}}} : \upstairs{B}}
%\end{mathpar}
%\end{enumerate}

\section{Semantics}
\label{sec:semantics}

\subsection{2-categories with families}
\label{sec:2cwfs}

The ``canonical'' semantics should interpret each judgment as follows:

Mode theory judgements:
\begin{enumerate}
\item $\mm{\gamma \ctx}$ is a category.
\item $\mm{\gamma \yields p \type}$ is a functor $\mm{\gamma}\op \to \Cat$.
\item $\mm{\TypeTwo{\gamma}{s}{p}{q}}$ is a natural transformation $\mm{\gamma \yields q} \Rightarrow \mm{\gamma \yields p}$ (note reversal of direction; this is because mode morphisms act contravariantly on mode terms and on upstairs subscripts).
\item $\mm{\gamma \yields \mu : p}$ is a section of the projection from the Grothendieck construction $\int\mm{\gamma\yields p} \to \mm{\gamma}$.
  In particular, it assigns to every object $x\in \mm{\gamma}$ an object $\mm{\gamma \yields \mu : p}(x)\in \mm{\gamma\yields p}(x)$.
\item $\mm{\TermTwoT{\gamma}{s}{\mu}{\nu}{p}}$ is a natural transformation of such sections over the identity, i.e.\ whose composite with the projection is the identity natural transformation of the identity functor.
\end{enumerate}

Top judgements: 
\begin{itemize}
\item $\mm{\yields_\gamma \Gamma \CTX}$ is an object of $\mm{\gamma}$
\item $\mm{\Gamma \yields_p A \TYPE}$ is an object of $\mm{\gamma \yields p}(\mm{\Gamma})$.
\item $\mm{\Gamma \yields_\mu M : A}$ is a morphism from $\mm{\gamma \yields \mu : p}(\mm{\Gamma})$ to $\mm{\Gamma \yields_p A}$ in $\mm{\gamma \yields p}(\mm{\Gamma})$.
\end{itemize}

\drlnote{Are these right?}
\msnote{Yes!}
Total substitutions in the mode theory, 2-cells between them, and
upstairs substitutions above them are not part of the primitive syntax
of the framework, but can be defined by tupling terms.
These can be interpreted as: 
\begin{itemize}
\item $\mm{\gamma \yields \theta : \delta}$ is a functor from $\mm{\gamma}$ to $\mm{\delta}$
\item $\mm{\gamma \yields \theta_1 \tcell_\delta \theta_2}$ is a natural
  transformation from $\mm{\gamma \yields \theta_1 : \delta}$ to
  $\mm{\gamma \yields \theta_2 : \delta}$
\item $\mm{\Gamma_{\gamma} \yields_\theta \Theta : \Delta_\delta}$ is a
  morphism from $\mm{\theta}(\mm{\Gamma})$ to $\mm{\Delta}$ in
  $\mm{\delta}$.
  Natural transformations act contravariantly because $\theta$ is in the
  domain of the morphism.  
  \end{itemize}
So the semantics of contexts/substitutions is a lot like 
one-variable adjoint logic with only $F$ types~\cite{ls15adjoint}.  

The general categorical semantics is an abstraction of these structures --- categories, contravariant $\Cat$-valued functors, Grothendieck constructions, sections, natural transformations, objects, and morphisms.
In this subsection we will concern ourselves only with the ``downstairs'' mode theory, which means abstracting the behavior of Grothendieck constructions in $\Cat$; in \S\ref{sec:fib-2cwf} we will reintroduce the ``upstairs'' type theory by additionally abstracting the behavior of ``objects and morphisms''.

We work with two universes in the metatheory, leading to two categories of categories denoted $\Cat$ and $\CAT$, such that $\Cat$ is an object of $\CAT$.
If necessary, we refer to objects of $\Cat$ as \emph{small}, objects of $\CAT$ (including $\Cat$) as \emph{large}, and other categories (such as $\CAT$ itself) as \emph{very large}.
However, for the most part size issues can be ignored.



\begin{enumerate}
\item Categories form a 2-category.
  In general we will stipulate an arbitrary (strict) 2-category $\M$.
\item Given a category $C$, the collection of functors and (strict) natural transformations $C\op \to \Cat$ forms a (large) category $[C\op, \Cat]$, and if we reverse the directions of the natural transformations we get $[C\op, \Cat]\op$.
  Moreover, precomposition with functors $C\to C'$ and natural transformations between them makes $[(-)\op, \Cat]\op$ into a strict 2-functor $\Cat\op \to \CAT$; syntactically these are $q[\mu/x]$ and $\ap{q}{\mu/x}$ respectively.

  Here $\CAT$ denotes the very large 2-category of large categories.
  Note that this 2-functor is covariant on 2-cells: the two $(-)\op$s cancel each other out at that level.
  (In fact, this 2-functor can be identified with the representable $[-,\Cat\op]$; this will be useful below.)

  Thus, in general we will stipulate a strict 2-functor $\Mty:\M\op \to \CAT$.
  \addtocounter{enumi}{1}
\item Given a category $C$ and a functor $T:C\op\to Cat$, the sections of the projection $\int T \to C$, and natural transformations over the identity, form a category.
  Moreover, such sections also vary functorially as $C$ does.
  Thus, in general we will stipulate another strict 2-functor $\Mtm : \M\op\to\CAT$ with a strictly 2-natural projection map $\Mtm\to\Mty$.

  The contravariant action of mode morphisms $\TypeTwo{\gamma}{s}{p}{q}$ on mode terms tells us that the morphisms of $\Mty(C)$ must act on the objects of $\Mtm$ contravariantly.
  Moreover, this action is strictly functorial, and respected by substitution.
  Thus, we stipulate that $\Mtm \to \Mty$ is a \emph{split fibration} internal to the 2-category $[\M\op,\CAT]$, which means that each functor $\Mtm(C) \to \Mty(C)$ is a split fibration and that all the naturality squares
  \begin{center}
    \begin{tikzcd}
      \Mtm(C) \ar[d] \ar[r] & \Mtm(C')\ar[d] \\
      \Mty(C) \ar[r] & \Mty(C')
    \end{tikzcd}
  \end{center}
  are strict morphisms of split fibrations (preserve the splittings on the nose).

  To represent the Grothendieck construction $\int T$ itself, we stipulate that this projection is additionally a \emph{representable morphism} in that for any $C\in \M$, if we form the pullback in $[\M\op,\CAT]$
  \begin{equation}
    \begin{tikzcd}
      \int T \ar[d] \ar[r] & \Mtm \ar[d] \\
      y(C) \ar[r,"\name{T}"] & \Mty
    \end{tikzcd}\label{eq:rep-pb}
  \end{equation}
  then the pullback object is also of the form $y(\int T)$ for some object $\int T\in \M$.
  Here $y(C)$ denotes the representable functor $y(C)(-) = \M(-,C) : \M\op\to\CAT$.
  The Yoneda lemma implies that (strict) 2-natural transformations $\name{T} : y(C) \to \Mty$ are in bijection with objects $T\in \Mty(C)$; thus every $T\in \Mty(C)$ induces an object $\int T$ with a projection $\int T \to C$ in $\M$, such that elements of $\Mtm(C)$ over $T$ are in bijection with sections of this projection.
  To make the notion algebraic, we require the object $\int T$ to be a specified function of $C$ and $T$; we call this being \textbf{algebraically representable}.

  Since~\eqref{eq:rep-pb} is a pullback in a 2-category, it also has a universal property for 2-cells.
  Thus, morphisms in $\Mtm$ (over the identity in $\Mty$) are in bijection with 2-cells between sections (over the identity).

  The 2-categorical Yoneda lemma also says that morphisms $\mu : S\to T$ in $\Mty(C)$ correspond bijectively to modifications $\name{\mu}:\name{S} \to \name{T}$.
  Since $\Mtm\to\Mty$ is a fibration, such a $\name{\mu}$ induces a map in the other direction $\int \mu : \int T\to \int S$ (see Hermida, Buckley, Johnstone), such that postcomposing with $\int \mu$ corresponds to the split (contravariant) fibrational action of $\mu$ on elements of $\Mtm$.
\end{enumerate}

Thus, the entire mode theory except for 1- and $\Sigma$-modes is encapsulated semantically by:

\begin{definition}
  A \textbf{2-category with families} is a 2-category $\M$ together with two 2-functors $\Mty,\Mtm : \M\op\to\CAT$ and a 2-natural transformation $\Ups:\Mtm\to \Mty$ that is both (1) an internal split fibration and (2) algebraically representable.
\end{definition}

Pleasingly, this is a straightforward categorification of the standard notion of category with families.
The only really new ingredient is the requirement that $\Mtm\to \Mty$ be an internal fibration, which has no analogue for 1-categories.

The ruminations above suggest that there should be a 2-category with families where $\M=\Cat$ and $\Mty(C) = [C\op,\Cat]\op$.
In fact we can construct this precisely as an instance of a much more general operation, similar to the ``global universe'' coherence method for ordinary type theory introduced by Voevodsky.
(There should also be a ``local universe'' method analogous to that of Lumsdaine--Warren, but we do not need that here.)

\begin{definition}
  A \textbf{2-category with a universe} is a large 2-category $\M$ together with a fully faithful 2-functor $y:\M \hookrightarrow \Mhat$, where $\Mhat$ is a very large and locally large 2-category, and a morphism $\Upshat : \Mtmhat \to \Mtyhat$ in $\Mhat$ that is both (1) an internal split fibration and (2) algebraically $\M$-representable, in the sense that for any map $y(C)\to \Mtyhat$, where $C\in \M$, has a specified pullback that also lies in the image of $y$.
\end{definition}

Usually we will treat $y$ as an implicit coercion, identifying objects of $\M$ with their images in $\Mhat$.

Note that every 2-category with families is also a 2-category with a universe, taking $\Mhat = [\M\op,\CAT]$ and $\Mtyhat=\Mty$, $\Mtmhat = \Mtm$, $\Upshat = \Ups$.
Indeed a 2-category with families is precisely a 2-category with a universe such that $\M \hookrightarrow \Mhat$ is the Yoneda embedding.
Conversely we have:

\begin{theorem}\label{thm:2cwf-univ}
  Any 2-category with a universe has an underlying 2-category with families, defined by
  \begin{align*}
    \Mty(C) &= \Mhat(C,\Mtyhat)\\
    \Mtm(C) &= \Mhat(C,\Mtmhat)\\
    \Ups &= \Mhat(C,\Upshat)
  \end{align*}
\end{theorem}
\begin{proof}
  The definitions of $\Mty$ and $\Mtm$ are certainly 2-functors, since they are representable, and the morphism $\Upshat$ induces a 2-natural transformation $\Ups:\Mtm\to\Mty$.
  More abstractly, we are applying the \emph{restricted Yoneda embedding} $\Mhat \to [\M\op,\CAT]$ (the term is somewhat of a misnomer since, unlike the ordinary Yoneda embedding, it may not be fully faithful).
  Since split fibrations are defined representably, they are preserved by the restricted Yoneda embedding; thus $\Ups$ is also a split fibration.
  Finally, since the restricted Yoneda embedding also preserves pullbacks, it preserves any pullback of $\Upshat$ to an object of $\M$; hence the vertex of any such pullback is again a specified representable.
  Thus, $\Ups$ is algebraically representable.
\end{proof}

\begin{example}\label{eg:cat-2cwf}
  Let $\M=\Cat$ and $\Mhat=\CAT$, with $\Mtyhat = \Cat\op$, and $\Mtmhat$ the Grothendieck construction of the contravariant functor $(\Cat\op)\op \cong \Cat \hookrightarrow \CAT$.
  Explicitly, the objects of $\Mtmhat$ are categories $A$ equipped with an object $a\in A$, and its morphisms $(A,a) \to (B,b)$ are functors $f:B\to A$ paired with a morphism $\phi : a\to f(b)$.
  Thus, $\Cat$ becomes a 2-category with a universe, and hence a 2-category with families, as in the above motivating discussion.
  We have $\Mty(C) \cong [C,\Cat\op] \cong [C\op,\Cat]\op$, and $\Mtm(C)$ consists of functors $D:C\op\to\Cat$ together with a section $s:C\to \int D$ of their Grothendieck construction.
\end{example}

On the other hand, we should also have the syntactic model:

\begin{example}\label{eg:syn-2cwf}
  Let $\M$ be the 2-category whose:
  \begin{itemize}
  \item objects are mode contexts $\gamma \ctx$,
  \item morphisms are total mode substitutions $\gamma \yields \theta : \delta$, and
  \item 2-cells are total mode 2-cells $\gamma \yields_\delta \theta_1 \tcell \theta_2$.
  \end{itemize}
  We define $\Mty:\M\op\to\CAT$ such that
  \begin{itemize}
  \item the objects of $\Mty(\gamma)$ are modes $\gamma \yields p\type$, and
  \item the morphisms of $\Mty(\gamma)$ are mode morphisms $\TypeTwo{\gamma}{s}{p}{q}$,
  \item with functorial action by substitution.
  \end{itemize}
  Finally, we define $\Mtm:\M\op\to\CAT$ such that
  \begin{itemize}
  \item the objects of $\Mtm(\gamma)$ are mode terms $\gamma \yields \mu:p$, and
  \item the morphisms of $\Mtm(\gamma)$ are mode 2-cells $\TermTwoT{\gamma}{s}{\mu}{\nu}{p}$,
  \item with functorial action by substitution.
  \end{itemize}
  The projection $\Ups:\Mtm\to\Mty$ is clear.
  Its representability is given, as usual, by context extension $(\gamma, x:p) \ctx$; the universal property of this follows from the definition of total substitutions and 2-cells as tuples.
  Finally, its split fibration structure is given by the contravariant action of mode morphisms on mode terms ${\gamma \yields \TrPlus{s}{\mu} : p}$.
  Note that, as mentioned above, so far we are only seeing the ``downstairs'' mode theory.
\end{example}


\subsection{$\Sigma$-modes}
\label{sec:2-sigmas}

\mvrnote{For consistency these should probably be called telescope modes in this section...}

Emboldened by this success, let's just try to write down a notion of $\Sigma$-types for 2-categories with families by categorifying the usual one for 1-categories.
As formulated by Awodey, $\Sigma$-types for a 1-category with families involve the dependent product of the presheaf $\mathrm{Ty}$ along the projection $\mathrm{Tm}\to \mathrm{Ty}$, to encode the notion of one type dependent on another one.
Unlike a presheaf 1-category, the presheaf 2-category $[\M\op,\CAT]$ is not locally cartesian closed, so not all dependent products exist; but fortunately, dependent products along fibrations do exist.

\begin{lemma}\label{thm:fib-exp}
  For a 2-category $\M$, any split fibration $\Ups : S\to T$ in $[\M\op,\CAT]$ is exponentiable.
\end{lemma}
\begin{proof}
  Suppose given $R\to S$; we want to define $\Pi_S[R]$ over $T$.
  Given an object $x\in T(C)$, corresponding to a morphism $\name{x}:y(C)\to T$, the objects of $\Pi_S[R](C)$ over $x$ must be bijective to the lifts of $\name{x}$ to $\Pi_S [R]$.
  The desired universal property of $\Pi_S [R]$ means that such lifts are bijective to maps $x^* S \to R$ over $S$.
  So we take the latter as the \emph{definition} the objects of $\Pi_S [R]$, i.e.\
  \[ \ob (\Pi_S [R](C)) = \sum_{x:\ob (T(C))} \ob (\M_{/S}(x^*S, R)). \]
  Next, given a morphism $\alpha :x\to y$ in $T(C)$ and objects $u,v$ of $\Pi_S[R]$ over $x,y$ respectively, morphisms $u\to v$ over $\alpha$ must correspond bijectively to lifts of the corresponding 2-cell $\name{\alpha} :\name{x} \to \name{y} :y(C) \to T$ to a 2-cell between $\name{u},\name{v} : y(C) \to \Pi_S[R]$.
  However, the desired universal property of $\Pi_S[R]$ as a dependent product doesn't obviously determine such lifts, because the 2-cell $\name{\alpha}$ doesn't live in the slice category $\M/T$.
  But we can use the fibration structure of $\Ups$, which induces a map $\alpha^*S : y^*S \to x^*S$ over $y(C)$ and a 2-cell
  \begin{center}
    \begin{tikzcd}
      x^*S \ar[drr] \\
      & \Downarrow\overline{\alpha} & S\\
      y^*S \ar[urr] \ar[uu,"\alpha^*S"]
    \end{tikzcd}
  \end{center}
  over $\alpha$ with a universal property (see Hermida, Buckley, Johnstone).
  Thus, if $u$ and $v$ are determined by maps $\hat{u}:x^*S \to R$ and $\hat{v}:y^*S\to R$ over $S$, we can define a morphism $u\to v$ over $\alpha$ to be determined by a lift $\hat{\alpha}$ of $\overline{\alpha}$ to $R$.

  The maps $\alpha^*S $ and 2-cells $\overline{\alpha}$ are functorial in $\alpha$ in all the ways one would hope (strictly so, since $\Ups$ is split --- although this is not really necessary), so it is straightforward to make $\Pi_S[R](C)$ thusly defined into a category and $\Pi_S[R]$ into a functor $\M\op\to\CAT$ over $T$.

  Note in particular that when $\alpha$ is an identity, so is $\overline{\alpha}$, so the fiber of $\Pi_S[R]$ over an object $x$ is the category $\M_{/S}(x^*S, R)$.
  This means that $\Pi_S[R]$ has the desired universal property with respect to maps out of representables, i.e.\ we have a natural isomorphism of hom-categories
  \[\M_{/T}(x:y(C)\to T,\Pi_S[R]) \cong \M_{/S}(x^*S, R).\]
  It is straightforward to extend this, using the Yoneda lemma, to the full universal property.
\end{proof}

Proceeding by analogy with the 1-categorical case, consider the dependent product $\Pi_\Ups[\Mty]$, where $\Mty$ denotes abusively the pullback $\Mtm \times \Mty \to \Mtm$ in $\M/\Mtm$.
The universal structure of $\Pi_\Ups[\Mty]$ consists of an ``evaluation'' map $\ev:\Pi_\Ups[\Mty] \times_\Mty \Mtm \to \Mty$.
Note that the projection $\Pi_\Ups[\Mty] \times_\Mty \Mtm \to \Pi_\Ups[\Mty]$ is a split fibration, since it is a pullback of $\Ups:\Mtm \to \Mty$.
And we can also pull $\Ups$ back along the evaluation map to get a further split fibration:
\begin{center}
  \begin{tikzcd}
    \ev^* \Mtm \ar[r] \ar[d,->>] \ar[dr,phantom,near start,"\lrcorner"] & \Mtm \ar[d,->>]\\
    \Pi_\Ups[\Mty] \times_\Mty \Mtm \ar[r] \ar[d,->>] \ar[ddr,phantom,near start,"\lrcorner"] \ar[dr] & \Mty\\
    \Pi_\Ups[\Mty] \ar[dr] & \Mtm \ar[d,->>] \\
    & \Mty
  \end{tikzcd}
\end{center}
Note that the composite of two split fibrations is again a split fibration.

\begin{definition}
  A 2-category with a universe (such as a 2-category with families) has \textbf{$\Sigma$-types} if $\Upshat$ is exponentiable, and it is equipped with maps $\Pi_\Upshat[\Mtyhat] \to \Mtyhat$ and $\ev^* \Mtmhat \to \Mtmhat$ such that the square
  \begin{center}
    \begin{tikzcd}
      \ev^* \Mtmhat \ar[r] \ar[d,->>] & \Mtmhat \ar[dd,->>]\\
      \Pi_\Upshat[\Mtyhat] \times_\Mtyhat \Mtmhat \ar[d,->>] \\
      \Pi_\Upshat[\Mtyhat] \ar[r] & \Mtyhat
    \end{tikzcd}
  \end{center}
  (1) commutes, (2) is a pullback, and (3) is a strict morphism of split fibrations.
\end{definition}

\begin{example}\label{eg:syn-sig}
  Continuing Example \ref{eg:syn-2cwf}, we argue that the syntactic model has $\Sigma$-types in this sense if and only if it has $\Sigma$-modes in the syntactic sense.
By the construction in Lemma \ref{thm:fib-exp}, an object of $\Pi_\Ups[\Mty](\gamma)$ lives over a type $p\in \Mty(\gamma)$, and consists of the additional data of a map $p^*\Mtm \to \Mty$.
But since $\Ups$ is representable, $p^* \Mtm$ is also representable, by the extended context $\gamma.p$; thus this additional data is a type $\gamma,x:p \yields q \type$.
The morphism $\Pi_\Ups[\Mty] \to \Mty$ thus assigns to any such pair a mode type $\gamma \yields \sigmacl{x}{p}{q} \type$.

Similarly, an object of $\ev^* \Mtm$ over $(p,q)$ consists of a pair of a term $\gamma \yields \mu : p$ and $\gamma \yields \nu : q[\mu/x]$, and so the morphism $\ev^* \Mtm$ assigns to this the pair $(p,q):\sigmacl{x}{p}{q}$.
The fact that the square is a pullback (on objects) means that we have $\fst$ and $\snd$ collectively forming an inverse isomorphism, i.e.\ the $\beta$- and $\eta$-rules hold for $\Sigma$-modes.

The action of $\Pi_\Ups[\Mty] \to \Mty$ on morphisms gives the congruence rules for $\Sigma$ on mode type morphisms, along with its functoriality laws (since this map over each $\gamma$ is a functor) and its naturality law for substitution (since this map is a natural transformation as $\gamma$ varies).
Similarly, the action of $\ev^* \Mtm \to \Mtm$ on morphisms gives the 2-cell operation on $\Sigma$-modes (although the latter is currently written to only apply when one of the morphisms being paired is cartesian), and the fact that the square is a pullback on morphisms gives the $\beta$- and $\eta$-rules for these.

Finally, the fact that this square is a strict morphism of split fibrations gives the equation for ``transport'' in $\Sigma$.
\end{example}

To construct our canonical model, we again appeal to a universal case.

\begin{theorem}\label{thm:sig-univ}
  If a 2-category with a universe has $\Sigma$-types, then so does its underlying 2-category with families (as in Theorem \ref{thm:2cwf-univ}).
\end{theorem}
\begin{proof}
  All the structure involved in the definition of $\Sigma$-types (dependent products, pullbacks, morphisms of split fibrations) is preserved by any restricted Yoneda embedding.
\end{proof}

\begin{example}\label{eg:cat-sig}
  Continuing Example \ref{eg:cat-2cwf}, we want to show that $\Mtyhat=\Cat\op$ has $\Sigma$-types in the sense of Theorem \ref{thm:sig-univ}.
  The map $\Upshat:\Mtmhat\to \Mtyhat$ is exponentiable because it is a split fibration (all fibrations are exponentiable in $\CAT$).
  An object of $\Pi_\Upshat[\Mtyhat]$ consists of a category $A\in \Cat\op$ together with a functor from the fiber of $\Upshat$ over $A$ to $\Mtyhat$, which is to say simply a functor $A\to \Cat\op$, or equivalently $B:A\op\to\Cat$.
  Similarly, a morphism $(A,B)\to (A',B')$ in $\Pi_\Upshat[\Mtyhat]$ consists of a functor $f:A'\to A$ together with a natural transformation $g:B' \to B \circ f$.

  The fiber of $\Pi_\Upshat[\Mtyhat] \times_\Mtyhat \Mtmhat$ over $(A,B)\in \Pi_\Upshat[\Mtyhat]$ is just the category $A$, and its contravariant split fibrational action on $(f,g)$ is just the action of the functor $f$.
  Thus, a morphism therein from $(A,B,x)$ to $(A',B',x')$ consists of $f:A'\to A$ and $g:B' \to B \circ f$ together with $\xi : x \to f(x')$.

  The functor $\ev$ sends $(A,B,x)$ to the category $B(x)$.
  Thus, the fiber of $\ev^*\Mtmhat$ over $(A,B,x)$ is just the category $B(x)$, with contravariant split fibrational action on $(f,g,\xi)$ given by the components of $g$.
  So a morphism from $(A,B,x,y)$ to $(A',B',x',y')$ consists of $f,g,\xi$ and $\zeta : y \to B(\xi)(g_{x'}(y'))$.

  It follows that the fiber of the composite $\ev^*\Mtmhat \to \Pi_\Upshat[\Mtyhat] \times_\Mtyhat \Mtmhat \to \Pi_\Upshat[\Mtyhat]$ over $(A,B)$ is (isomorphic to) the category of pairs $(x,y)$ with $x\in A$ and $y\in B(x)$, and with morphisms $(x,y)\to (x',y')$ being pairs $(\xi,\zeta)$ where $\xi:x\to x'$ and $\beta: y \to B(\xi)(y')$.
  The split fibrational action of a morphism $(f:A'\to A,g:B' \to B \circ f)$ sends $(x',y')$ to $(f(x'),g(y'))$, with a similar action on morphisms.
  Thus, we can define the desired functor $\Pi_\Upshat[\Mtyhat] \to \Mtyhat = \Cat\op$ by sending $(A,B)$ to precisely this category of pairs $(x,y)$, which is none other than the ordinary Grothendieck construction of $B:A\op\to \Cat$.
  Defining this functor on morphisms by the above split fibrational action, we find tautologically that the desired square is both a pullback and a morphism of split fibrations.
  (More abstractly, we are simply using the fact that $\Mtmhat\to \Mtyhat$ is a classifier in $\CAT$ of split fibrations with small fibers.)

  Thus, the canonical 2-category with families structure on $\Cat$ has $\Sigma$-types.
\end{example}


\subsection{Fibered 2-categories with families}
\label{sec:fib-2cwf}

As in our previous work, with the 2-categorical analogue of the downstairs mode theory in place, the upstairs type theory corresponds to a ``local fibration'' over it.
To define this, we first need to talk about morphisms between 2-categories with families, or more generally with universes.

\begin{definition}\label{defn:fibmor}
  Let $\vp : \mathcal{A}\to \mathcal{B}$ be any 2-functor, and $p:S\to T$ and $q:U\to V$ be split fibrations in $\mathcal{A}$ and $\mathcal{B}$ respectively.
  A \textbf{strict $\vp$-morphism from $p$ to $q$} consists of a commutative square in $\mathcal{B}$:
  \begin{equation}\label{eq:fibmor}
    \begin{tikzcd}
      \vp S \ar[r] \ar[d] & U\ar[d]\\
      \vp T \ar[r] & V
    \end{tikzcd}
  \end{equation}
  such that for any $X\in \mathcal{A}$, the induced commutative square of categories
  \begin{equation}\label{eq:fibmor2}
    \begin{tikzcd}
      \mathcal{A}(X,S) \ar[r] \ar[d] & \mathcal{B}(\vp X, \vp S) \ar[r] \ar[d] & \mathcal{B}(\vp X, U) \ar[d] \\
      \mathcal{A}(X,T) \ar[r] & \mathcal{B}(\vp X, \vp T) \ar[r] & \mathcal{B}(\vp X, V)
    \end{tikzcd}
  \end{equation}
  is a strict morphism of split fibrations.
\end{definition}

If $\vp$ has a right adjoint $\vpst$, then~\eqref{eq:fibmor} is equivalent to a square in $\mathcal{A}$:
\begin{equation}\label{eq:fibmor-mate}
  \begin{tikzcd}
    S \ar[r] \ar[d] & \vpst U\ar[d]\\
    T \ar[r] & \vpst V
  \end{tikzcd}
\end{equation}
and the condition in Definition \ref{defn:fibmor} is equivalent to asking that~\eqref{eq:fibmor-mate} be a strict morphism of internal split fibrations in $\mathcal{A}$.
This makes sense since $\vpst$, as a right adjoint, preserves limits and hence split fibrations, while an analogous phrasing for~\eqref{eq:fibmor} would not since $\vp$ is not assumed to preserve fibrations.
However, if $\vp$ \emph{does} preserve split fibrations, in the algebraic sense that it preserves the chosen cartesian lifts, and if~\eqref{eq:fibmor} is a strict morphism of split fibrations internal to $\M$, then it is also a strict $\vp$-morphism (but the converse need not hold).

% Suppose $\C$ and $\M$ are 2-categories with universes and $\vp:\C\to\M$ is a 2-functor.
% Since $\C$ is large and $\Mhat$ has large colimits, the composite $\C \xrightarrow{\vp} \M \hookrightarrow \Mhat$ has a pointwise left Kan extension along the inclusion $\C\hookrightarrow \Chat$; we denote it by $\vp_! : \Chat \to \Mhat$.
% Since $\C\hookrightarrow \Chat$ is fully faithful, $\vp_!$ is an honest extension, i.e.\ it restricts to $\vp$ on $\C$.

% If $\C$ is a 2-category with families, $\vp_!$ has a right adjoint $\vpst : \Mhat \to \Chat = [\C\op,\CAT]$ by precomposition: $\vpst(X)(C) = \Mhat(\vp(C),X)$.
% In general, any right adjoint of $\vp_!$ comes with a mate natural transformation
% \begin{equation}
%   \begin{tikzcd}
%     \C \ar[r] \ar[d,"\vp"'] \ar[dr,phantom,"{\Downarrow^{\vptil}}"] & \Chat\\
%     \M \ar[r] & \Mhat. \ar[u,"\vpst"']
%   \end{tikzcd}\label{eq:vpmate}
% \end{equation}
% If $\C$ and $\M$ are 2-categories with families, then $\vptil$ is just the action of $\vp$ on homs, $\C(-,C) \to \M(\vp(-),\vp(C))$.

\begin{definition}\label{defn:mor-2cwu}
  Let $\C$ and $\M$ be two 2-categories with universes.
  A \textbf{morphism of 2-categories with universes} consists of:
  \begin{enumerate}
  \item A commutative square of 2-functors
    \begin{equation}\label{eq:mor-2cwu}
      \begin{tikzcd}
        \C \ar[r] \ar[d,"\vp"'] & \Chat \ar[d,"\vpsh"] \\
        \M \ar[r] & \Mhat
      \end{tikzcd}
    \end{equation}
  \item A $\vpsh$-morphism from $\Upshat:\Ctmhat \to \Ctyhat$ to $\Upshat : \Mtmhat \to \Mtyhat$ as in Definition \ref{defn:fibmor}.
  % \item Morphisms $\vpty$ and $\vptm$ forming a commutative square in $\Chat$:
  %   \begin{center}
  %     \begin{tikzcd}
  %       \Ctmhat \ar[r,"\vptm"] \ar[d,->>,"\Upshat"'] & \vpst \Mtmhat\ar[d,->>,"\vpst(\Upshat)"] \\
  %       \Ctyhat \ar[r,"\vpty"] & \vpst \Mtyhat
  %     \end{tikzcd}
  %   \end{center}
  %   Note that the right-hand vertical morphism is still a split fibration, since split fibration structure is defined using 2-categorical limits, and $\vpst$ is a right adjoint and hence preserves all such structure.
  % \item This commutative square is a strict morphism of split fibrations.
  \item This morphism furthermore ``preserves the algebraic representations'' in the following sense: given $C\in \C$ and $T:C \to \Ctyhat$, with specified pullback $\int T$, the composite square
    \begin{equation}
      \begin{tikzcd}
        \vp(\int T) \ar[r] \ar[d] & \vpsh \Ctmhat \ar[r] \ar[d] & \Mtmhat \ar[d]\\
        \vp(C) \ar[r,"\vpsh(T)"'] & \vpsh \Ctyhat \ar[r] & \Mtyhat
      \end{tikzcd}\label{eq:mor-presrep}
    \end{equation}
    is also a specified pullback.
% , the composite $\vpty T : C \to \vpst \Mtyhat$ has an adjunct $\overline{\vpty T}:\vp(C) = \vp_!(C) \to \Mtyhat$, and thus two algebraically specified pullback squares
%     \begin{center}
%       \begin{tikzcd}
%         \int T \ar[r] \ar[d] \ar[dr,phantom,near start,"\lrcorner"] & \Ctmhat \ar[d] &
%         \int \overline{ \vpty T} \ar[r] \ar[d] \ar[dr,phantom,near start,"\lrcorner"] & \Mtmhat \ar[d] \\
%         C \ar[r,"T"'] & \Ctyhat &
%         \vp(C) \ar[r,"{\overline{\vpty T}}"'] & \Mtyhat
%       \end{tikzcd}
%     \end{center}
%     Since $\vpst$ preserves pullbacks, the left-hand composite square below factors uniquely as on the right:
%     \begin{center}
%       \begin{tikzcd}
%         \int T \ar[r] \ar[d] &
%         \Ctmhat \ar[r,"\vptm"] \ar[d] & \vpst \Mtmhat\ar[d] \ar[dr,phantom,"="] &
%         \int T \ar[r,dashed]\ar[d] \ar[dr,phantom,"(*)"] &
%         \vpst \int \overline{\vpty T} \ar[r] \ar[d] \ar[dr,phantom,near start,"\lrcorner"] & \vpst\Mtmhat \ar[d] \\
%         C \ar[r] & \Ctyhat \ar[r,"\vpty"] & \vpst \Mtyhat &
%         C \ar[r,"\vptil"'] & \vpst \vp (C) \ar[r] & \vpst \Mtyhat
%       \end{tikzcd}
%     \end{center}
%     where $\vptil$ denotes a component of~\eqref{eq:vpmate}.
%     The condition we require is then that the square $(*)$ above coincides with the naturality square of $\vptil$:
%     \begin{center}
%       \begin{tikzcd}
%         \int T \ar[r,"\vptil"] \ar[d] & \vpst\vp(\int T) \ar[d] \\
%         C \ar[r,"\vptil"] & \vpst\vp(C)
%       \end{tikzcd}
%     \end{center}
%     (hence, in particular, that $\int \overline{\vpty T} = \vp(\int T)$).
  \end{enumerate}
\end{definition}

If $\M$ is a 2-category with a universe for which $\Mhat$ has large colimits, $\C$ is a 2-category with families, and $\vp:\C\to\M$ is any 2-functor, then the composite $\C \xrightarrow{\vp} \M \hookrightarrow \Mhat$ has a left Kan extension along $\C\hookrightarrow \Chat$.
Since the latter inclusion is fully faithful, $\vpsh$ is an honest extension, i.e.\ the universal square~\eqref{eq:mor-2cwu} commutes (up to isomorphism, at least, and $\vpsh$ can be chosen to make it commute strictly).
Moreover, in this case $\vpsh$ always has a right adjoint $\vpst : \Mhat \to \Chat = [\C\op,\CAT]$ defined by precomposition: $\vpst(X)(C) = \Mhat(\vp(C),X)$.

\begin{definition}
  If $\C$ and $\M$ are 2-categories with families, a \textbf{morphism of 2-categories with families} is a morphism of 2-categories with universes such that $\vpsh$ is this left Kan extension.
\end{definition}

\begin{remark}
  If $\M$ is a 2-category with a universe for which $\Mhat$ has large colimits, and $\M'$ denotes the 2-category with families constructed from it as in Theorem \ref{thm:2cwf-univ}, then the identity 2-functor $\M'\to\M$ is a morphism of 2-categories with universes with $\vpsh$ a left Kan extension as above.
  Morally, this should be some kind of ``coreflection'', but we will not try to make this precise.
\end{remark}

\begin{definition}\label{thm:2cwf-ldf}
  A morphism of 2-categories with universes $\vp : \C\to\M$ is a \textbf{local discrete fibration} if
  \begin{enumerate}
  \item The 2-functor $\vp:\C\to\M$ is a local discrete fibration, i.e.\ the induced functors on hom-categories $\C(X,Y) \to \M(\vp(X),\vp(Y))$ are discrete fibrations.\footnote{If they were non-discrete fibrations, there would be an additional compatibility condition on composition, but in the discrete case this is automatic.}\label{item:ldf1}
  \item The horizontal functors in the relevant instance of~\eqref{eq:fibmor2}:
    \begin{gather*}
      \Chat(X,\Ctmhat) \to \Mhat(\vp X, \vpsh \Ctmhat) \to \Mhat(\vp X, \Mtmhat)\\
      \Chat(X,\Ctyhat) \to \Mhat(\vp X, \vpsh \Ctyhat) \to \Mhat(\vp X, \Mtyhat)
    \end{gather*}
    are discrete fibrations.\label{item:ldf2}
  \end{enumerate}
  In this case we refer to $\C$ as a \textbf{fibered 2-category with a universe} (or \textbf{with families}, if $\C$ and $\M$ are such and $\vp$ is a morphism of such) over $\M$.
\end{definition}

If $\vpsh$ has a right adjoint $\vpst$ (such as if $\vp$ is a morphism of 2-categories with families), then condition \ref{item:ldf2} is equivalent to asking that the adjunct morphisms $\Ctmhat \to \vpst \Mtmhat$ and $\Ctyhat \to \vpst \Mtyhat$ are internal discrete fibrations in $\Chat$.

\begin{example}\label{eg:syn-fib-2cwf}
  Returning to Example \ref{eg:syn-2cwf}, we construct a fibered 2-category with families over the syntactic one $\M$, now using the ``upstairs'' type theory.
  \begin{itemize}
  \item An object of $\C$ is a context $\yields_\gamma \Gamma \CTX$.
    Of course, its image in $\M$ is $\yields \gamma \ctx$.
  \item A morphism in $\C$ is a total substitution $\Gamma_{\gamma} \yields_\theta \Theta : \Delta_\delta$, lying over $\gamma \yields \theta : \delta$.
  \item The fibrational action of 2-cells in $\M$ on morphisms in $\C$ is given by N-ary rewrite (Lemma \ref{lem:n-ary-ap-rewrite}).
  \item An object of $\Cty(\Gamma)$ is a type $\Gamma_\gamma \yields_p A \TYPE$, lying over $\gamma \yields p\type$ in $\Mty(\gamma) = \Mty(\vp(\Gamma))$.
    The fibrational action of morphisms in $\Mty$ is given by the $\St{s}{A}$ types.
  \item An object of $\Ctm(\Gamma)$ over $A\in \Cty(\Gamma)$ is a term $\Gamma_\gamma \yields_\mu M:A_p$, lying over $\gamma \yields \mu:p$ in $\Mtm$.
    To act on such a term by a morphism in $\Mtm(\gamma)$, we first factor the latter morphism as a mode 2-cell in a fiber $\TermTwoT{\gamma}{s}{\mu}{\TrPlus{t}{\nu}}{p}$ followed by the cartesian arrow corresponding to the action of a mode morphism $\TypeTwo{\gamma}{t}{p}{q}$ on $\nu$.
    The cartesian morphism then acts on $M$ by S-intro, $\StI{t}{M} : \St{t}{A}$, and then we rewrite with the fiber 2-cell $\rewrite{s}{\StI{t}{M}}$.
  \end{itemize}
\end{example}

To construct the canonical model, we appeal to a more general ``co-universe'' construction, which we decompose into three pieces.

\begin{theorem}\label{thm:corefl-ldfib}
  If $\vp:\C\to \M$ is a local discrete fibration of 2-categories with universes, then its coreflection into 2-categories with families is also a local discrete fibration.
\end{theorem}
\begin{proof}
  We first have to check that it is actually a morphism of 2-categories with families.
  The morphism $\vpsh \Ctyhat \to \Mtyhat$ induces by composition
  \begin{equation}
    \Chat(X,\Ctyhat) \to \Mhat(\vp X, \vpsh \Ctyhat) \to \Mhat(\vp X,\Mtyhat)\label{eq:corefl-map}
  \end{equation}
  and hence a map of the induced presheaves $\Cty \to \vpst \Mty$, and similarly for the $\Ctm \to \vpst \Mtm$.
  Moreover,~\eqref{eq:fibmor2} refers only to the functors represented by $p$ and $q$, so we can restrict it to $X\in\C$ to obtain the analogous condition for these underlying presheaves.
  And~\eqref{eq:mor-presrep} is defined in terms of~\eqref{eq:corefl-map}, so it carries over to the analogous condition as well.

  Finally, condition~\ref{item:ldf1} of Definition \ref{thm:2cwf-ldf} is evident since the coreflection doesn't change the underlying categories, while condition \ref{item:ldf2} follows since it is also defined in terms of~\eqref{eq:corefl-map}.
\end{proof}

\begin{theorem}\label{thm:ldf-lift}
  Suppose $\M$ is a 2-category with a universe, and we have a pullback square
  \begin{equation}\label{eq:ldf-mor}
    \begin{tikzcd}
      \C \ar[r] \ar[d,"\vp"'] & \Chat \ar[d,"\vpsh"] \\
      \M \ar[r] & \Mhat
    \end{tikzcd}
  \end{equation}
  such that $\vpsh$ is a local discrete fibration.
  Moreover, suppose we are given a lifting of $\Upshat : \Mtmhat \to \Mtyhat$ to a morphism $\Ctmhat \to \Ctyhat$ of $\Chat$, and that $\vpsh$ creates pullbacks of $\Ctmhat \to \Ctyhat$.
  Then $\C\hookrightarrow\Chat$ is a 2-category with a universe and~\eqref{eq:ldf-mor} is a local discrete fibration of 2-categories with universes.
\end{theorem}
\begin{proof}
  Since fully faithful 2-functors and local discrete fibrations are closed under pullback, $\C \to \Chat$ is fully faithful and $\vp$ is a local discrete fibration (of 2-categories only, for now).
  To show that $\Ctmhat \to \Ctyhat$ is a split fibration, suppose given the following data in $\Chat$, mapping to $\Mhat$ as shown on the right:
  \begin{equation*}
    \begin{tikzcd}[row sep=large]
      C \ar[d,equals] \ar[r,bend right,near end,"g'"'] & \Ctmhat \ar[d] \ar[dr,phantom,"\mapsto"] &
      \vpsh C \ar[d,equals] \ar[r,bend right,near end,"\vpsh g'"'] & \Mtmhat \ar[d] \\
      C \ar[r,bend left,"f"] \ar[r,bend right,"g"'] \ar[r,phantom,"\Downarrow^\alpha"] & \Ctyhat &
      \vpsh C \ar[r,bend left,"\vpsh f"] \ar[r,bend right,"\vpsh g"'] \ar[r,phantom,"\Downarrow^{\vpsh \alpha}"] & \Mtyhat
    \end{tikzcd}
  \end{equation*}
  We then have a specified cartesian lift $\gamma : h \to \vpsh g'$ in $\Mhat$.
  Since $\vpsh$ is a local discrete fibration, $\gamma$ has a unique lift $\beta : f' \to g'$ in $\Chat$.
  To show $\beta$ is cartesian, suppose given $\theta : k\to g'$ in $\Chat$ with $\Upshat\circ \theta = \alpha *\mu$ for some $\mu$\footnote{We are writing $\circ$ for composition of 1- and 2-morphsims along objects, including whiskering, and $*$ for composition of 2-morphisms along 1-morphisms}.
  Then $\Upshat \circ \vpsh\theta = \vpsh\alpha *\vpsh\mu$, so $\vpsh\theta$ factors uniquely as $\gamma* \nu$ since $\gamma$ is cartesian.
  Now since $\vpsh$ is a local discrete fibration, $\nu$ has a unique lift $\eta : k' \to f'$; but then $\beta*\eta : k' \to g'$ is a lift of $\vpsh\theta$, hence equal by uniqueness to $\theta$.
  Uniqueness of these factorizations follows similarly, and splitness and functoriality is immediate.
  Furthermore, by construction the identities $\vpsh\Ctyhat = \Mtyhat$ and $\vpsh\Ctmhat =\Mtmhat$ are a strict $\vpsh$-morphism of split fibrations, and since $\vpsh$ (not just $\vp$) is a local discrete fibration, condition~\ref{item:ldf2} of Definition \ref{thm:2cwf-ldf} is automatic.

  It remains to show that $\Ctmhat \to \Ctyhat$ is algebraically $\C$-representable and that $\vp$ preserves these representations.
  Since the right-hand square in~\eqref{eq:mor-presrep} is an identity, it suffices to show that given $T:C\to \Ctyhat$ with $C\in \C$, the specified pullback $\int (\vpsh T)$ lifts to a pullback in $\Chat$ with vertex in $\C$.
  But by assumption $\vpsh$ creates pullbacks of $\Ctmhat\to\Ctyhat$, and the vertex of such a pullback must lie in $\C$ since~\eqref{eq:ldf-mor} is a pullback of 2-categories.
\end{proof}

\begin{theorem}\label{thm:laxslice-2cwf}
  Let $\M$ be a 2-category with a universe, and let $\one$ be an arbitrary object of $\M$ equipped with a commutative triangle
  \begin{equation}
    \begin{tikzcd}
      & \Mtmhat \ar[d,"\Upshat"] \\
      \one \ar[ur,"t"] \ar[r,"\top"'] & \Mtyhat.
    \end{tikzcd}\label{eq:laxslice-tri}
  \end{equation}
  Let $\C = \one\sslash \M$ be the lax slice 2-category of $\M$ under $\one$: its objects are morphisms $c:\one\to C$ in $\M$, its morphisms are 2-cells inhabiting triangles
  \begin{equation}\label{eq:laxslice-mor}
    \begin{tikzcd}
      & \one \ar[dl,"a"'] \ar[dr,"b"] \ar[d,phantom,"\overset{\phi}{\Rightarrow}"] \\
      A \ar[rr,"f"'] & {}& B
    \end{tikzcd}
  \end{equation}
  and its 2-cells are 2-cells $\alpha :f\to g$ in $\M$ such that $\alpha a \circ \phi = \psi$.
  Similarly, let $\Chat = \one\sslash\Mhat$, with the evident inclusion $\C\hookrightarrow\Chat$.
  Let $\Ctyhat = (\top:\one \to \Mtyhat)$ and $\Ctmhat = (t:\one\to\Mtmhat)$, with~\eqref{eq:laxslice-tri} defining a map $\Ctmhat \to \Ctyhat$ in $\Chat$.
  Then $\C$ is a 2-category with a universe, and the forgetful 2-functor $\vp:\C\to\M$ is a local discrete fibration.
\end{theorem}
\begin{proof}
  First note that the forgetful functor $\vp:\C=(\one\sslash\M)\to\M$ is a local discrete fibration of 2-categories: given a morphism $(f,\phi):(A,a) \to (B,b)$ as in~\eqref{eq:laxslice-mor} and a 2-cell $\alpha:\psi\to\phi$ in $\M$, the unique lift is $\alpha$ with domain $(f,\phi * (\alpha \circ a))$.
  Similarly, the forgetful functor $\vpsh:\Chat=(\one\sslash\Mhat)\to\Mhat$ is a local discrete fibration, and $\C$ is the pullback of $\Chat$ and $\M$ over $\Mhat$ (the objects of $\C$ are just those of $\Chat$ whose underlying objects in $\Mhat$ lie in $\M$).
  Thus, to apply Theorem \ref{thm:ldf-lift} it remains only to check that $\vpsh$ creates pullbacks of~\eqref{eq:laxslice-tri} regarded as a morphism in $\one\sslash\Mhat$.

  For brevity, write $q:X\to Y$ for $\Upshat :\Mtmhat \to \Mtyhat$; the only thing we will use about $q$ is that it is a fibration.
  Note that we have assumed~\eqref{eq:laxslice-tri} to be commutative rather than simply inhabited by a 2-cell, so regarded as a lifting of $q$ it is special in this way as well among morphisms of $\one\sslash\Mhat$.
  Now suppose we have a morphism $(f,\phi) : (A,a) \to (Y,\top)$ in $\one\sslash\Mhat$, and a pullback in $\Mhat$:
  \begin{center}
    \begin{tikzcd}
      P \ar[r,"j"] \ar[d,"k"'] & X \ar[d,"q"]\\
      A \ar[r,"f"'] & Y.
    \end{tikzcd}
  \end{center}
  Let $\overline{\phi} : i \to t$ be a cartesian 2-cell lifting $\phi : f \circ a \to \top$.
  Then $q \circ i = f\circ a$, so the universal property of $P$ induces a map $p:\one\to P$ with $j\circ p = i$ and $k\circ p = a$.
  The latter identity makes $k$ into a morphism $(P,p) \to (A,a)$ in $\one\sslash\Mhat$, while $j\circ p = i \xrightarrow{\overline{\phi}} t$ makes $j$ into a morphism $(P,p) \to (X,t)$.
  We claim the resulting square
  \begin{center}
    \begin{tikzcd}
      (P,p) \ar[r,"{(j,\overline{\phi})}"] \ar[d,"{(k,1)}"'] & (X,t) \ar[d,"{(q,1)}"]\\
      (A,a) \ar[r,"{(f,\phi)}"'] & (Y,\top)
    \end{tikzcd}
  \end{center}
  (which commutes because $q \circ \overline{\phi} = \phi$) is a pullback in $\one\sslash\Mhat$.
  To show this, suppose given $(Z,z:\one\to Z)$ and morphisms $(g,\gamma):(Z,z) \to (X,t)$ and $(h,\theta):(Z,z) \to (A,a)$ forming a commutative square, which means $q\circ g = f\circ h$ and $q\circ \gamma = \phi * (f\circ \theta)$.
  The former equality induces, by the universal property of $P$, a unique map $\ell : Z\to P$ such that $j\circ \ell = g$ and $k\circ \ell = h$.

  To make $\ell$ into a morphism in $\one\sslash\Mhat$, we need a 2-cell $\ell \circ z \to p$.
  By the 2-cell universal property of $P$, it suffices to give 2-cells $g\circ z \to j\circ p = i$ and $h\circ z \to k \circ p = a$ that agree in $Y$.
  For the latter, we can simply take $\theta$.
  For the first, note that $\overline{\phi}:i\to t$ is cartesian over $\phi$, so there is a unique 2-cell $\eta : g \circ z \to i$ such that  $\overline{\phi} * \eta = \gamma$ and $q \circ \eta = f\circ \theta$.

  The latter equation is the requisite agreement in $Y$, so we have a unique 2-cell $\lambda:\ell \circ z \to p$ such that $k\circ \lambda = \theta$ and $j\circ \lambda = \eta$.
  The latter equation is equivalent, by uniqueness of $\eta$, to $q\circ j \circ \lambda = f\circ \theta$ (which follows from $k\circ \lambda = \theta$ and $q\circ j = f\circ k$) and $\overline{\phi} * (j\circ \lambda) = \gamma$.
  Thus, these conditions are precisely the statements that $(j,\overline{\phi})\circ (\ell,\lambda) = (g,\gamma)$  and $(k,1)\circ (\ell,\lambda) = (h,\theta)$.
  This shows the 1-cell universal property of $(P,p)$ in $\one\sslash\Mhat$.
  The 2-cell universal property follows immediately from the 2-cell universal property of $P$.
\end{proof}

Combining Theorems \ref{thm:corefl-ldfib} and \ref{thm:laxslice-2cwf}, we can construct local discrete fibrations of 2-categories with families.

\begin{example}
  Consider the 2-category with a universe from Example \ref{eg:cat-2cwf}, with $\M=\Cat$, $\Mhat = \CAT$, and $\Mtyhat = \Cat\op$.
  Let $\one$ be the terminal category in $\M=\Cat$, with $\top:\one \to \Cat\op$ picking out the terminal category and $t$ the unique object of the latter.
  Thus we obtain a 2-category with a universe $\C$ where:
  \begin{itemize}
  \item The objects of $\C$ (resp.\ $\Chat$) are large (resp.\ very large) categories $C$ equipped with a chosen object $c\in C$.
  \item The morphisms $(C,c) \to (D,d)$ are functors $f:C\to D$ together with a morphism $\phi :f(c) \to d$.
  \item Natural transformations act contravariantly on such pairs by precomposition $f(c) \xrightarrow{\alpha_c} g(c) \to d$.
  \item $\Ctyhat$ is the category $\Mtyhat = \Cat\op$ equipped with the terminal category $\top$.
  \item Similarly, $\Ctmhat$ is the category $\Mtmhat$ of pointed categories $(A,a)$ with morphisms $(A,a) \to (B,b)$ being functors $f:B\to A$ paired with a morphism $\phi : a\to f(b)$, equipped with the terminal pointed category $(\top,t)$.
  \end{itemize}
  If we now apply the coreflection into 2-categories with families, we get a local discrete fibration over the ``canonical'' 2-category with families from Example \ref{eg:cat-2cwf}, where:
  \begin{itemize}
  \item The objects of $\Cty(C,c)$ are morphisms $(C,c) \to \Ctyhat = (\Cat\op,\top)$ in $Chat = \one\sslash\CAT$, which is to say functors $D:C\op\to \Cat$ equipped with a morphism $D(c) \to \top$ in $\Cat\op$, i.e.\ a functor $\top \to D(c)$, which is to say an object $d\in D(c)$.
  \item The objects of $\Ctm(C,c)$ are such pairs $(D,d)$ with a section $s:C\to \int D$ of the Grothendieck construction, along with an object $d\in D(c)$ and a morphism $s(c) \to d$ in $D(c)$.
  \end{itemize}
  Thus, this reproduces our informal expectation from \S\ref{sec:2cwfs}.
  Note that while $\Cty$ and $\Ctm$ in this example simply ``pick out the objects and morphisms'' of the categories $C\in\M$, in general we can regard the fibered 2-category with families $\C$ as an ``abstraction'' of the notions of ``objects and morphisms'' applicable to any 2-category with families $\M$.
\end{example}

\end{document}

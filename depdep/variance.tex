\documentclass[10pt]{article}
  \usepackage{xcolor}
  \definecolor{darkgreen}{rgb}{0,0.45,0}
  \usepackage[pagebackref,colorlinks,citecolor=darkgreen,linkcolor=darkgreen]{hyperref}
  \usepackage{pdflscape}

\usepackage[sort]{natbib}

\usepackage{fullpage}
\usepackage{amssymb,amsthm,bbm}
\usepackage[centertags]{amsmath}
\usepackage[mathscr]{euscript}
\usepackage{fontawesome}
\usepackage{tikz-cd}
\usepackage{mathpartir}
\usepackage{enumitem}
\usepackage[status=draft,inline,nomargin]{fixme}
\FXRegisterAuthor{ms}{anms}{\color{blue}MS}
\FXRegisterAuthor{mvr}{anmvr}{\color{olive}MVR}
\FXRegisterAuthor{drl}{andrl}{\color{purple}DRL}
\usepackage{stmaryrd}
\usepackage{mathtools}
\usepackage{bbold}

\newtheorem{theorem}{Theorem}
\newtheorem{proposition}{Proposition}
\newtheorem{lemma}{Lemma}
\newtheorem{corollary}{Corollary}
\newtheorem{problem}{Problem}
\newenvironment{constr}{\begin{proof}[Construction]}{\end{proof}}

\theoremstyle{definition}
\newtheorem{definition}{Definition}
\newtheorem{remark}{Remark}
\newtheorem{example}{Example}

\let\oldemptyset\emptyset%
\let\emptyset\varnothing

% Fix wedge/smash
\let\oldwedge\wedge%
\renewcommand{\wedge}{\vee}
\newcommand{\topsmash}{\oldwedge}

% Fix equiv and defeq
\let\oldequiv\equiv%
\renewcommand{\equiv}{\simeq}
\newcommand{\defeq}{\oldequiv}
\newcommand{\ndefeq}{\not\defeq}

\newcommand\dsd[1]{\ensuremath{\mathsf{#1}}}

\newcommand{\yields}{\vdash}
\newcommand{\Yields}{\tcell}
\newcommand{\tcell}{\Rightarrow}
\newcommand{\cbar}{\, | \,}
\newcommand{\judge}{\mathcal{J}}

\newcommand{\Id}[3]{\mathsf{Id}_{{#1}}(#2,#3)}
\newcommand{\ctx}{\,\,\mathsf{ctx}}
\newcommand{\type}{\,\,\mathsf{type}}
\newcommand{\tele}{\,\,\mathsf{tele}}
\newcommand*{\proj}{\mathsf{pr}}

\newcommand\act[2]{\ensuremath{#1 \{ #2 \} \ }}
\newcommand\ap[2]{\ensuremath{#1 \langle #2 \rangle }}
\newcommand\ApPlus[2]{\ensuremath{{#1}^+ \langle #2 \rangle }}
\newcommand\ApCirc[2]{\ensuremath{{#1}^\circ \langle #2 \rangle }}

\makeatletter
\def\smsym{\sum}
\newcommand{\@thesum}[1]{\smsym_{(#1)}}
\newcommand{\sm}[1]{\@ifnextchar\bgroup{\@sm{#1}\sm}{\@sm{#1}}}
\newcommand{\@sm}[1]{\mathchoice{{\textstyle\@thesum{#1}}}{\@thesum{#1}}{\@thesum{#1}}{\@thesum{#1}}}
\def\prdsym{\prod}
\newcommand{\@theprd}[1]{\prdsym_{(#1)}}
\newcommand{\prd}[1]{\@ifnextchar\bgroup{\@prd{#1}\prd}{\@prd{#1}}}
\newcommand{\@prd}[1]{\mathchoice{{\textstyle\@theprd{#1}}}{\@theprd{#1}}{\@theprd{#1}}{\@theprd{#1}}}
\makeatother

\newcommand{\Arr}{\mathrm{Arr}}
\newcommand{\Two}{\mathbb{2}}
\newcommand{\comp}{\mathtt{comp}}
\newcommand{\lift}{\mathtt{lift}}
\newcommand{\uni}{\mathtt{uni}}

\newcommand{\op}{\mathsf{op}}
\newcommand{\opc}[1]{\overline{#1}}
\newcommand{\unop}{\mathsf{unop}}

\newcommand{\fop}{\mathsf{fop}}

\newcommand{\Tw}{\mathrm{Tw}}

\title{Another Directed Theory}
\author{Mitchell Riley}
\date{}

\begin{document}
\maketitle

Variances: $-, +, \pm, 0$.

Judgements:
\begin{itemize}
\item $\gamma \ctx$ meaning a category $\gamma$
\item $\gamma \yields A \type^v$ meaning a category $A$ and a functor $A : \gamma.A \to \gamma$, so that $A$ is a fibration / opfibration / bifibration / nothing according to $v$.
\item $\gamma \yields a : A$ is a section $a : \gamma \to \gamma.A$ of $A$, and the variance of $A$ doesn't come into it.
\end{itemize}

So the only thing he variance controls is which direction you can act with 2-cells. Any variable in the context is usable, not just those marked with $+$. And so there is also no variance restriction on substitutions.

\subsection{Contexts}
\begin{mathpar}
\inferrule*{\gamma \yields A \type^v}{\gamma, x :^v A \ctx} \and 
\inferrule*{~}{\gamma, x :^v A, \gamma' \yields x : A} \\
\inferrule*[fraction={-{\,-\,}-}]{\gamma \yields a : A \and \gamma, x :^v A, \gamma' \yields b : B}{\gamma, \gamma'[a/x] \yields b[a/x] : B[a/x]}
\end{mathpar}

\subsection{Sums}

Let
\begin{align*}
v \cap v &\defeq v \\
\pm \cap w &\defeq w \\
w \cap \pm &\defeq w \\
v \cap w &\defeq 0 \quad \text{otherwise}
\end{align*}

Then the type is as usual, just tracking the variance in the type former
\begin{mathpar}
\inferrule*{\gamma \yields A \type^v \and \gamma, x :^v A \yields B \type^w}{\gamma \yields \sm{x :^v A} B^w \type^{v \cap w}} \\
\inferrule*{\gamma \yields a : A \and \gamma \yields b : B[a/x]}{\gamma \yields (a, b) : \sm{x :^v A} B^w} \and
\inferrule*{\gamma \yields p : \sm{x :^v A} B^w}{\gamma \yields \proj_1(p) : A} \and
\inferrule*{\gamma \yields p : \sm{x :^v A} B^w}{\gamma \yields \proj_2(p) : B[a/x]}
\end{mathpar}

\subsection{Products}
The variance of the $\Pi$ type is given by $\pm \rhd w :\defeq w$ and $v \rhd w :\defeq 0$ otherwise. Non-dependent functions could be more permissive, also allow: $- \blacktriangleright + \defeq +$ and $+ \blacktriangleright - \defeq -$. (Or is there some trick that lets this work for dependent $\Pi$?)
\begin{mathpar}
\inferrule*{\gamma \yields A \type^v \and \gamma, x :^v A \yields B \type^w}{\gamma \yields \prd{x :^v A} B^w \type^{v \rhd w}} \\
\inferrule*{\gamma, x :^v A \yields b : B}{\gamma \yields \lambda x. b : \prd{x :^v A} B^w} \and
\inferrule*{\gamma \yields f : \prd{x :^v A} B^w \and \gamma \yields a : A}{\gamma \yields f(a) : B[a/x]}
\end{mathpar}

\subsection{Extensional Equality}
\begin{mathpar}
\inferrule*{\gamma \yields A \type^v \\\\ \gamma \yields a : A \and \gamma \yields a' : A}{\gamma \yields \Id{A}{a}{a'} \type^0} \\
\inferrule*{\gamma \yields a : A}{\gamma \yields \mathsf{refl}_{a} : \Id{A}{a}{a}} \and
\inferrule*{\gamma \yields e : \Id{A}{a}{a'}}{\gamma \yields a \defeq a' : A} \\
\end{mathpar}

\subsection{Arrows}

Let's add a type $\Two$ (not sure if we want this to really be a type)

\begin{mathpar}
\inferrule*{\gamma \ctx}{\gamma, i :^\pm \Two \ctx} \\
\inferrule*{~}{\gamma \yields 0 : \Two} \and \inferrule*{~}{\gamma \yields 1 : \Two}
\end{mathpar}

And a type of arrows with fixed end-points:
\begin{mathpar}
\inferrule*{\gamma, i :^\pm \Two \yields B \type^w \\\\ \gamma \yields b_0 : B[0/i] \and \gamma \yields b_1 : B[1/i]}{\gamma \yields \Arr_B(b_0,b_1) \type^0} \\
\inferrule*{\gamma, i :^\pm \Two \yields b : B \\\\ b_0 \defeq b[0/i] \and b_1 \defeq b[1/i]}{\gamma \yields \lambda i. b : \Arr_B(b_0,b_1)} \and
\inferrule*{\gamma \yields f : \Arr_B(b_0,b_1) \and \gamma \yields j : \Two}{\gamma \yields f \mathbin{@} j : B[j/i]}
\end{mathpar}

I don't think the variance of $\Arr_B(b_0,b_1)$ can be $w$, because we fixed the endpoints. If we use the free arrow category, we can have any variance:
\begin{mathpar}
\begin{tikzcd}
\Arr \ar[r] \ar[d,two heads] & C^2 \ar[d, "C^2",two heads] \\
\Gamma \ar[r, "\eta" swap] & (\Gamma \times 2)^2
\end{tikzcd}
\end{mathpar}
But this doesn't seem very useful.

We could also allow a more general variance on $\Arr_B(b_0,b_1)$ if we are allowed to assume that $B$ `has pullbacks', because then both the domain/codomain functors are fibrations.

\subsection{Topes}

Add a judgement $\gamma \yields i \leq j$:
\begin{mathpar}
\inferrule*{\gamma \yields x : \Two \and \gamma \yields y : \Two}{\gamma, (x \leq y) \ctx}
\end{mathpar}
where $x$ and $y$ are variables or 0 or 1. And (phrased as axioms)
\begin{align*}
(x \leq y) &\yields (x \leq y) \\
&\yields (x \leq x) \\
(x \leq y), (y \leq z) &\yields (x \leq z) \\
(x \leq y), (y \leq x) &\yields (x \defeq y) \\
&\yields (0 \leq x) \\
&\yields (x \leq 1) \\
&\yields (x \leq y) \vee (y \leq x)
\end{align*}
(the same as in RS)

\subsection{Composition}
All types are Segal:

\begin{mathpar}
\inferrule*{
\gamma, i :^\pm \Two, j :^\pm \Two, i \leq j \yields A \type^v \\\\
\gamma \yields r : \Two \and \gamma \yields r' : \Two \and \gamma \yields r \leq r' \\\\
\gamma, j^\pm : \Two \yields f : A[0/i] \and \gamma, i^\pm : \Two \yields g : A[1/j] \\\\
f[1/j] \defeq g[0/i]
}{\gamma \yields \comp^{r,r'}_{i.j.A}(f,g) : A[r/i,r'/j] \\\\
\comp^{0,r'}_{i.j.A}(f,g) \defeq f[r'/j] \and \comp^{r,1}_{i.j.A}(f,g) \defeq g[r/i]}
\end{mathpar}

We also want this to definitionally be the unique filler, how can we arrange that? We could add extension types, and have an eta principle for this kind of extension? 

Another way to phrase what we want is that
\begin{align*}
\gamma, i :^\pm \Two, j :^\pm \Two, i \leq j \yields a : A
\end{align*}
and
\begin{align*}
\gamma, i :^\pm \Two, j :^\pm \Two, (i = 0) \vee (j = 1) \yields a : A
\end{align*}
should be identical.

Something else we could do is have n-ary composition, and definitionally fuse the composition.

\subsection{Lifts}

\begin{mathpar}
\inferrule*{
\gamma, i :^\pm \Two \yields P \type^+ \\\\
\gamma \yields r : \Two \and \gamma \yields r' : \Two \and \gamma \yields r \leq r' \\\\
\gamma \yields p_0 : P[r/i]
}{\gamma \yields \lift^{r \to r'}_{i.P}(p_0) : P[r'/i] \and \lift^{r \to r}_{i.P}(p_0) \defeq p_0} \and 
\inferrule*{
\gamma, i :^\pm \Two \yields N \type^- \\\\
\gamma \yields r : \Two \and \gamma \yields r' : \Two \and \gamma \yields r \leq r' \\\\
\gamma \yields n_1 : N[r'/i]
}{\gamma \yields \lift^{r \leftarrow r'}_{i.N}(n_1) : N[r/i] \and \lift^{r' \leftarrow r'}_{i.N}(n_1) \defeq n_1}
\end{mathpar}
Define
\begin{align*}
\lift^{r \to r'}_{i.\sm{x:^+ A}B^+}(a,b) &:\defeq (\lift^{r \to r'}_{i.A}(a), \lift^{r\to r'}_{j.B[j/i][\lift^{r \to r'}_{i.A}(a)/x]}(b)) \\
\lift^{r\to r'}_{i.\prd{x:^\pm A} B^+}(f) &:\defeq \lambda a. \lift^{r \to r'}_{j.B[j/i][\lift^{r \leftarrow r'}_{i.A}(a)/x]}(f(\lift^{r \leftarrow r'}_{i.A}(a)))
%\lift^{r\to r'}_{i.\Arr_C(-,c_1)}(f) &:\defeq \Lambda j.\lift^{r\to r'}_{i.C}(f \mathbin{@} j)
\end{align*}
Do we want to make this compute on $\comp$?
\begin{align*}
\lift^{r \to r'}_{i.A[s/k,s'/l]}(\comp^{s,s'}_{k.l.A[r/i]}(f,g)) :\defeq \comp^{s,s'}_{k.l.A[r'/i]}(\lift^{r \to r'}_{i.A}(f),\lift^{r \to r'}_{i.A}(g))
\end{align*}
seems to typecheck fine.

We are yet to make the lifts universal. We can try factoring maps through the lift as another filling problem:
\begin{mathpar}
\inferrule*{
\gamma, i :^\pm \Two \yields P \type^+ \\\\
\gamma, i :^\pm \Two \yields p : P \\\\
\gamma \yields r : \Two \and \gamma \yields r' : \Two \and \gamma \yields r \leq r'
}{
\gamma \yields \uni^{r\to r'}_{i.P}(p) : P[r'/i] \\\\
\uni^{0 \to 0}_{i.P}(p) \defeq p[0/i] \and 
\uni^{0 \to 1}_{i.P}(p) \defeq \lift^{0 \to 1}_{i.P}(p[0/i]) \and 
\uni^{1 \to 1}_{i.P}(p) \defeq p[1/i]
}
\end{mathpar}
Then we get arrows
\begin{align*}
\Lambda j. \uni^{0 \to j}_{i.P}(p) &: \Arr_{i.P}(p[0/i], \lift^{0\to 1}_{i.P}(p[0/i])) \\
\Lambda k. \uni^{k \to 1}_{i.P}(p) &: \Arr_{i.P[1/i]}(\lift^{0\to 1}_{i.P}(p[0/i]), p[1/i])
\end{align*}

But it looks like we can just define
\begin{align*}
\uni^{r\to r'}_{i.P}(p) :\defeq \lift^{r \to r'}_{i.P}(p[r/i])
\end{align*}
So the only thing missing is the uniqueness of this factorisation.

Is there a way we could make a lift of a lift fuse into a single multi-lift?

\subsection{Op?}
Now we have to follow other systems and mark when variables are $\op$ped. Let us do this by $\opc{x}$. On variances, just flip the sign. On contexts, flip both the variance and the direction:
\begin{align*}
\opc{(\gamma, x :^v A)} &:\defeq \opc{\gamma}, \opc{x} :^{v^\op} A \\
\opc{(\gamma, \opc{x} :^v A)} &:\defeq \opc{\gamma}, x :^{v^\op} A
\end{align*}

Context:
\begin{mathpar}
\inferrule*{\opc{\gamma} \yields A \type^{v^\op}}{\gamma, \opc{x} :^v A \ctx}
\end{mathpar}

Type former:
\begin{mathpar}
\inferrule*{\opc{\gamma} \yields A \type^{v^\op}}{\gamma \yields A^\op \type^v} \\
\inferrule*{\opc{\gamma} \yields a : A}{\gamma \yields a^\op : A^\op} \and \inferrule*{\opc{\gamma} \yields a : A^\op}{\gamma \yields a^\unop : A}
\end{mathpar}

This takes a functor $A \to \Gamma^\op$ to a functor $A^\op \to \Gamma$, no matter what the functor is.

But there is a different operation we could do: if $B \to \Gamma$ is a fibration, then we can instead calculate the  fiberwise opposite.  Is this the more important operation?

\begin{mathpar}
\inferrule*{\gamma \yields B \type^+}{\gamma \yields B^\fop \type^+} \\
\end{mathpar}
But I don't know how we would introduce a term of such a thing. It might involve some magic to do with $\Two$, like switching $0$s with $1$s or something.

I think we can use this to describe a different version of arrows where transport should work the way `it is supposed to', backwards on $b_0$ and forwards on $b_1$.

\begin{mathpar}
\inferrule*{\gamma \yields B \type^+ \\\\ \gamma \yields b_0 : B^\fop \and \gamma \yields b_1 : B}{\gamma \yields \Tw_B(b_0, b_1) \type^+} \\
\end{mathpar}

\iffalse
\section{Old Junk}

\subsection{2-cells}
I want to make these as admissible as possible:
\begin{mathpar}
\inferrule*[fraction={-{\,-\,}-}]{\gamma \yields s : a \Yields a' : A \\\\ \gamma, x :^v A \yields N \type^- \and \gamma \yields n : N[a'/x]}{\gamma \yields \act{n}{s/x^-} : N[a/x]} \and
\inferrule*[fraction={-{\,-\,}-}]{\gamma \yields s : a \Yields a' : A \\\\ \gamma, x :^v A \yields P \type^+ \and \gamma \yields p : P[a/x]}{\gamma \yields \act{p}{s/x^+} : P[a'/x]} \\

\inferrule*[fraction={-{\,-\,}-}]{\gamma \yields s : a \Yields a' : A \\\\ \gamma, x :^v A \yields N \type^- \and \gamma, x :^v A \yields n : N}{\gamma \yields \ap{n}{s/x^-} : n[a/x] \Yields \act{n[a'/x]}{s/x^-} : N[a/x]} \and
\inferrule*[fraction={-{\,-\,}-}]{\gamma \yields s : a \Yields a' : A \\\\ \gamma, x :^v A \yields P \type^+ \and \gamma \yields p : P[a/x]}{\gamma \yields \ap{p}{s/x^+} : \act{p[a/x]}{s/x^+} \Yields p[a'/x] : P[a'/x]} 
\end{mathpar}

Or more generally: (just the negative ones)
\begin{mathpar}
\inferrule*[fraction={-{\,-\,}-}]{\gamma \yields s : a \Yields a' : A \\\\ \gamma, x :^v A, \gamma'^{<} \yields N \type^- \and \gamma,\gamma'^{<}[a'/x] \yields n : N[a'/x]}{\gamma,\gamma'^{<}[a'/x] \yields \act{n}{s/x^-} : N[a/x]} \\
\inferrule*[fraction={-{\,-\,}-}]{\gamma \yields s : a \Yields a' : A \\\\ \gamma, x :^v A \yields N, \gamma'^{<} \type^- \and \gamma, x :^v A,\gamma'^{<} \yields n : N}{\gamma,\gamma'^{<}[a'/x] \yields \ap{n}{s/x^-} : n[a/x] \Yields \act{n[a'/x]}{s/x^-} : N[a/x]} 
\end{mathpar}
where $\gamma'^{<}$ denotes a context where the variances are all either $-$ or $\pm$ (and so $N$ is a fibration over $A$)



Internalising the 2-cell judgement:

\begin{mathpar}
\inferrule*{\gamma \yields A \type^v \and \gamma \yields a : A \and \gamma \yields a' : A}{\gamma \yields \Arr(a,a') \type^0} \\
\inferrule*{\gamma \yields s : a \Yields a' : A}{\gamma \yields s : \Arr(a,a')} \and 
\inferrule*{\gamma \yields s : \Arr(a,a')}{\gamma \yields s : a \Yields a' : A} \\
\end{mathpar}

\subsection{Mixed Variance}
But not having a fibrant $\Arr$ is annoying. This is where things probably fall apart.

We want to be able to form 
\begin{mathpar}
\inferrule*{\gamma \yields N \type^- \and \gamma \yields P \type^+}{\gamma \yields N \times P \type^{??}}
\end{mathpar}
and use a composition operator 
\begin{mathpar}
\inferrule*{
\gamma, i :^\pm \Two \yields N \times P \type^{??} \\\\
\gamma \yields t : N[1/i] \times P[0/i] \and
\gamma \yields r : \Two
}{\gamma \yields \comp^{r}_{N \times P}(t) : N[r/i] \times P[r/i] \and \comp^{?}_{N \times P}(t) \defeq ?}
\end{mathpar}

\subsection{Lifting}

\begin{mathpar}
\inferrule*{
\gamma, x :^v A \yields B \type^+ \\\\
\gamma, i :^\pm \Two \yields a : A \\\\
\gamma \yields b_0 : B[a[0/i]/x] \and
\gamma \yields r : \Two
}{\gamma \yields \lift^{0 \to r}_{x.B}(i.a,b_0) : B[a[r/i]/x] \and \lift^{0 \to 0}_{x.B}(i.a,b_0) \defeq b_0}
\end{mathpar}

\begin{align*}
\lift^{0 \to r}_{x.\sm{y:B} B'}(i.a,(b_0,b_0')) :\defeq (\lift^{0 \to r}_{x.B}(i.a,b_0), ?
\end{align*}


\subsection{Based Arrows}

Could also try a diagram
\begin{mathpar}
\begin{tikzcd}
\Arr(-,c_1) \ar[r] \ar[d,two heads] & C^2 \ar[d, "1",two heads] \\
\Gamma \ar[r, "c_1" swap] & C
\end{tikzcd}
\end{mathpar}

with rules

\begin{mathpar}
\inferrule*{\gamma \yields C \type^+ \and \gamma \yields c_1 : C}{\gamma \yields \Arr_C(-,c_1) \type^+} \\
\inferrule*{\Gamma, i^\pm : \Two \yields c : C \and c[1/i] \defeq c_1}{\gamma \yields \Lambda i. c : \Arr_C(-,c_1)} \and
\inferrule*{\gamma \yields f : \Arr_C(-,c_1) \and \gamma \yields j : \Two}{\gamma \yields f \mathbin{@} j : C \and f \mathbin{@} 1 \defeq c_1}
\end{mathpar}
\fi

\end{document}
